% Copyright (c) 2004-2012 H. Turgut Uyar <uyar@itu.edu.tr>
%
% Bu notlar "Creative Commons Attribution-NonCommercial-ShareAlike License" ile
% lisanslanmıştır. Yazarının açıkça belirtilmesi koşuluyla ve ticari olmayan
% amaçlarla kullanılabilir ve dağıtılabilir. Bu notlardan yola çıkılarak
% oluşturulacak çalışmaların da aynı lisansa bağlı olmaları gerekir.
%
% Lisans ile ilgili ayrıntılı bilgi almak için şu sayfaya başvurabilirsiniz:
% http://creativecommons.org/licenses/by-nc-sa/3.0/

\documentclass[dvipsnames]{beamer}

\mode<presentation>
{
  \usetheme{Rochester}
  \usecolortheme[named=Mahogany]{structure}
  \setbeamercovered{transparent}
}

\usepackage{ae}
\usepackage[T1]{fontenc}
\usepackage[turkish]{babel}
\usepackage[utf8]{inputenc}
\setbeamertemplate{navigation symbols}{}

\title{Bilişim Etiği}
\subtitle{İnternet}

\author{H. Turgut Uyar}
\date{2004-2012}

\AtBeginSubsection[]
{
  \begin{frame}<beamer>
    \frametitle{Konular}
    \tableofcontents[currentsection,currentsubsection]
  \end{frame}
}

%\beamerdefaultoverlayspecification{<+->}

\theoremstyle{definition}
\newtheorem{tanim}[theorem]{Tanım}

\theoremstyle{example}
\newtheorem{ornek}[theorem]{Örnek}

\theoremstyle{plain}

\pgfdeclareimage[width=2cm]{license}{../../license}

\pgfdeclareimage[width=5.6cm]{tomcruise}{tomcruise}
\pgfdeclareimage[width=5.8cm]{milka}{milka}
\pgfdeclareimage[width=5.8cm]{gmail}{gmail}
\pgfdeclareimage[width=5.8cm]{goggle}{goggle}
\pgfdeclareimage[width=5.6cm]{icann}{icann}
\pgfdeclareimage[width=5.6cm]{sucks}{sucks}
\pgfdeclareimage[height=6.8cm]{china}{china}
\pgfdeclareimage[width=5.6cm]{neutrality}{neutrality}
\pgfdeclareimage[height=6.8cm]{turkey}{turkey}

\begin{document}

\begin{frame}
  \titlepage
\end{frame}

\begin{frame}
  \frametitle{Lisans}

  \pgfuseimage{license}\hfill
  \copyright 2004-2012 H. Turgut Uyar

  \vfill
  \begin{tiny}
    You are free:
    \begin{itemize}
      \item to Share — to copy, distribute and transmit the work
      \item to Remix — to adapt the work
    \end{itemize}

    Under the following conditions:
    \begin{itemize}
      \item Attribution — You must attribute the work in the manner specified by
        the author or licensor (but not in any way that suggests that they
        endorse you or your use of the work).

      \item Noncommercial — You may not use this work for commercial purposes.

      \item Share Alike — If you alter, transform, or build upon this work, you
        may distribute the resulting work only under the same or similar license
        to this one.
    \end{itemize}
  \end{tiny}

  \vfill
  Legal code (the full license):\\
  \url{http://creativecommons.org/licenses/by-nc-sa/3.0/}
\end{frame}

\begin{frame}
  \frametitle{Konular}
  \tableofcontents
\end{frame}

\section{İnternet}

\subsection{Alan Adları}

\begin{frame}
  \frametitle{Alan Adları}

  \begin{itemize}
    \item alan adlarının dağıtımı adil mi?

    \pause
    \bigskip
    \item 1998'e kadar NSF dağıtıyor
    \begin{itemize}
      \item "ilk gelen alır" politikası
      \item \emph{sanal çöreklenme} (cybersquatting)
    \end{itemize}
    
    \item 1998'den sonra ICANN'e geçiyor
    \begin{itemize}
      \item kar amacı gütmeyen bir kuruluş
      \item \emph{UDRP}: Uniform Domain-Name Dispute-Resolution Policy
      \item tescilli markalar geçerli
    \end{itemize}
  \end{itemize}
\end{frame}

\begin{frame}
  \frametitle{Anlaşmazlıkların Çözümü}

  \begin{itemize}
    \item WIPO Hakemlik ve Arabuluculuk Merkezi

    \medskip
    \item bir alan adının aktarılması için
    \begin{itemize}
      \item isim tescilli markayla aynı ya da çok benzer olmalı
      \item eski sahibinin bu isim üzerinde hiçbir hakkı olmamalı
      \item kötü niyet olmalı
    \end{itemize}
  \end{itemize}
\end{frame}

\begin{frame}
  \frametitle{Örnek: TomCruise.com}

  \begin{columns}
    \column{.48\textwidth}
    \begin{center}
      \pgfuseimage{tomcruise}
    \end{center}

    \column{.52\textwidth}
    \begin{itemize}
      \item WIPO, TomCruise.com\\
        alanını, kaydeden kişiden alıp\\
        sinema oyuncusu\\
        Tom Cruise'a veriyor (2006)
    \end{itemize}
  \end{columns}

  \medskip
  \tiny{\url{http://www.theregister.co.uk/2006/07/23/tom_cruise_dotcom_win/}}\\
  \tiny{\url{http://www.theregister.co.uk/2004/12/17/ronaldinho_scores_own_domain_name/}}\\
  \tiny{\url{http://www.theregister.co.uk/2006/10/13/rooney_wins_dotcom/}}\\
  \tiny{\url{http://www.theregister.co.uk/2012/03/19/pope_benedict_cybersquatter/}}\\
\end{frame}

\begin{frame}
  \frametitle{Örnek: milka.fr}

  \begin{columns}
    \column{.48\textwidth}
    \begin{center}
      \pgfuseimage{milka}
    \end{center}

    \column{.52\textwidth}
    \begin{itemize}
      \item Fransız modacı\\
        Milka Budimir almış
      \item Kraft istiyor
      \item mahkeme, Kraft'ı\\
        haklı buluyor (2005)
    \end{itemize}
  \end{columns}

  \medskip
  \tiny{\url{http://news.bbc.co.uk/2/hi/europe/4348585.stm}}\\
\end{frame}

\begin{frame}
  \frametitle{Örnek: Gmail}

  \begin{columns}
    \column{.48\textwidth}
    \begin{center}
      \pgfuseimage{gmail}
    \end{center}

    \column{.52\textwidth}
    \begin{itemize}
      \item Google, İngiltere'de (2005)\\
        ve Almanya'da (2007)\\
        Gmail ismini kullanamıyor
    \end{itemize}
  \end{columns}

  \medskip
  \tiny{\url{http://news.bbc.co.uk/2/hi/business/4354954.stm}}\\
  \tiny{\url{https://mashable.com/2007/10/02/google-german-domain/}}\\
\end{frame}

\begin{frame}
  \frametitle{İsim Benzerlikleri}

  \begin{itemize}
    \item benzeyen isimler de anlaşmazlık konusu olabiliyor
    \begin{itemize}
      \item mikerowesoft.com: Mike Rowe - Microsoft
      \item mocosoft.com: Mocosoft - Microsoft
    \end{itemize}

    \medskip
    \item \emph{yazım hatasına çöreklenme} (typosquatting)
    \begin{itemize}
      \item arama motorları ve tarayıcılar aynı şekilde para kazanmıyor mu?
    \end{itemize}
  \end{itemize}

  \medskip
  \tiny{\url{http://www.theregister.co.uk/2004/01/19/microsoft_lawyers_threaten_mike_rowe/}}\\
  \tiny{\url{http://www.theregister.co.uk/2004/12/15/mocosoft_beats_microsoft/}}\\
  \tiny{\url{http://www.theregister.co.uk/2008/10/23/google_and_typosquatting/}}\\
\end{frame}

\begin{frame}
  \frametitle{Örnek: Goggle}

  \begin{columns}
    \column{.48\textwidth}
    \begin{center}
      \pgfuseimage{goggle}
    \end{center}

    \column{.52\textwidth}
    \begin{itemize}
      \item Google, çeşitli alan adlarını\\
        kazanıyor (2005):\\
        googkle.com, ghoogle.com,\\
        gfoogle.com, gooigle.com
      \item goggle.com alan adını\\
        kazanamıyor (2011)
    \end{itemize}
  \end{columns}

  \medskip
  \tiny{\url{http://www.theregister.co.uk/2005/07/11/google_ruling/}}\\
  \tiny{\url{http://www.theregister.co.uk/2011/10/12/google_v_goggle/}}\\
\end{frame}

\begin{frame}
  \frametitle{Örnek: ICANN çıkar çatışmaları (2012)}

  \begin{columns}
    \column{.48\textwidth}
    \begin{center}
      \pgfuseimage{icann}
    \end{center}

    \column{.52\textwidth}
    \begin{itemize}
      \item ICANN başkanı,\\
        yönetim kurulundakilerin\\
        çıkar çatışmalarına\\
        dikkat çekiyor (2012)
    \end{itemize}
  \end{columns}

  \medskip
  \tiny{\url{http://www.theregister.co.uk/2012/03/19/icann_president_calls_out_his_own_board_over_conflicts_of_interest/}}\\
\end{frame}

\subsection{İfade Özgürlüğü}

\begin{frame}
  \frametitle{İfade Özgürlüğü}

  \begin{itemize}
    \item neler ifade özgürlüğüne girmez?

    \pause
    \bigskip
    \item çocuk pornografisi

    \pause
    \item nefret ya da şiddet propagandası

    \pause
    \item suça ya da zararlı davranışa özendirme
    \begin{itemize}
      \item nasıl bomba yapılır?
      \item nasıl acısız intihar edilir?
    \end{itemize}

    \pause
    \item hakaret
  \end{itemize}
\end{frame}

\begin{frame}
  \frametitle{Protesto Siteleri}

  \begin{itemize}
    \item insanlar memnun olmadığı kurumları protesto etmek için\\
      site açıyorlar

    \medskip
    \item iki açıdan değerlendiriliyor:
    \begin{itemize}
      \item sitenin içeriği (yalan, hakaret, ifade özgürlüğü)
      \item alan adı (tescilli marka)
    \end{itemize}
  \end{itemize}
\end{frame}

\begin{frame}
  \frametitle{Örnek: Air France, Wal-Mart}

  \begin{columns}
    \column{.48\textwidth}
    \begin{center}
      \pgfuseimage{sucks}
    \end{center}

    \column{.52\textwidth}
    \begin{itemize}
      \item WIPO, airfrancesucks.com,\\
        wal-martcanadasucks.com\\
        gibi alan adlarını\\
        ilgili şirketlere veriyor
      \item ABD Temyiz Mahkemesi,\\
        ticari amaç gütmeyen\\
        sitelerin ifade özgürlüğüne
        gireceğini söylüyor (2005)
    \end{itemize}
  \end{columns}

  \medskip
  \tiny{\url{http://www.theregister.co.uk/2005/04/05/bosley_case_appeal/}}\\
  \tiny{\url{http://www.theregister.co.uk/2006/12/29/wipo_rules_against_ryanair/}}\\
  \tiny{\url{http://www.theregister.co.uk/2009/07/28/wipo_free_speech/}}\\
\end{frame}

\subsection{Demokrasi}

\begin{frame}
  \frametitle{Demokrasi}

  \begin{itemize}
    \item İnternet demokratik bir ortam mı?
    \item İnternet demokrasiye katkı yapıyor mu?
    \item İnternet'in demokrasiye katkı yapması gerekli mi?
  \end{itemize}
\end{frame}

\begin{frame}
  \frametitle{Bilgiye Erişim}

  \begin{itemize}
    \item \emph{evet}: düşük maliyetle bilgiye erişim sağlıyor

    \pause
    \medskip
    \item \emph{ama}: bilginin önemli bir kısmı yanlış
  \end{itemize}
\end{frame}

\begin{frame}
  \frametitle{Aşırılıkları Azaltma}

  \begin{itemize}
    \item \emph{evet}: insanlar coğrafyadan bağımsız olarak biraraya
      gelebiliyor
    \begin{itemize}
      \item farklı insanları ve kültürleri tanımak sivrilikleri törpüler
    \end{itemize}

    \pause
    \medskip
    \item \emph{ama}: kişisel tercihler ters yönde etki yapıyor
    \begin{itemize}
      \item bakacağımız siteleri, haber alacağımız kaynakları\\
        süzgeçten geçiriyoruz
      \item kendimiz gibi düşünen insanlarla biraraya geliyoruz
      \item tek bilgi kaynağı İnternet olursa hiç yeni düşünce\\
        ya da görüş duymuyoruz
      \item aşırılıklara kaymak kolaylaşıyor
    \end{itemize}
  \end{itemize}
\end{frame}

\begin{frame}
  \frametitle{Politik Sürece Katılım}

  \begin{itemize}
    \item \emph{evet}: bireyler ve azınlık grupları seslerini duyurabiliyor

    \pause
    \medskip
    \item \emph{ama}: kaotik bir ortam
    \begin{itemize}
      \item sesini duyurabilmek için insanların ilgisini çekmek gerek
      \item web sitesi hazırlama, arama motorlarına kaydettirme
      \item kaynak gerektiriyor: güçlüye daha çok güç mü veriyor?
    \end{itemize}
  \end{itemize}
\end{frame}

\begin{frame}
  \frametitle{Arama Motorlarının Rolü}

  \begin{itemize}
    \item pek çok insan için giriş noktası arama motorları
    \item arama motorlarının sonuçları çok önemli:
    \begin{itemize}
      \item hangi sayfalar gösterilecek?
      \item ne sırayla gösterilecek?
    \end{itemize}
  \end{itemize}
\end{frame}

\begin{frame}
  \frametitle{Örnek: Arama motorları - Çin}

  \begin{columns}
    \column{.48\textwidth}
    \begin{center}
      \pgfuseimage{china}
    \end{center}

    \column{.52\textwidth}
    \begin{itemize}
      \item Çin'den yapılan aramalarda\\
        arama sonuçları sansürleniyor
      \item Google sansürlü servisini\\
        iptal ediyor (2010)
      \item içerik sağlayıcı lisansını\\
        kaybediyor
    \end{itemize}
  \end{columns}

  \medskip
  \tiny{\url{http://www.guardian.co.uk/technology/2010/mar/25/china-microsoft-free-speech-google}}\\
\end{frame}

\begin{frame}
  \frametitle{Ağ Tarafsızlığı}

  \begin{itemize}
    \item \alert{ağ tarafsızlığı}:\\
      İnternet'in her kullanıcıya, her uygulama sağlayıcıya\\
      ve her taşıyıcıya açık, erişilebilir ve ayrımsız olması

    \medskip
    \item aksi örnekler:
    \begin{itemize}
      \item taşıyıcıların bazı uygulamaları engellemesi\\
        ya da harcadıkları bant genişliklerini kısıtlaması (örneğin VoIP)
      \item taşıyıcıların bazı site ya da hizmetlere erişimi engellemesi\\
        ya da önceliklendirmesi
    \end{itemize}
  \end{itemize}
\end{frame}

\begin{frame}
  \frametitle{Ağ Tarafsızlığı İlkeleri}

  \begin{itemize}
    \item kullanıcıların her türlü yasal İnternet içeriğine ulaşabilmesi
    \item kullanıcıların istedikleri uygulama ve hizmetleri kullanabilmesi
    \item ağa zarar vermedikleri sürece,\\
      kullanıcıların ağa istedikleri aygıtla bağlanabilmesi
    \item kullanıcıların taşıyıcılar, içerik sağlayıcılar ve\\
      uygulama sağlayıcılar arasındaki rekabetten yararlanabilmesi

    \pause
    \medskip
    \item hizmet sağlayıcıların yasal İnternet içerik uygulamaları\\
      ve hizmetleri arasında ayrım yapmaması
    \item hizmet sağlayıcıların uyguladıkları ağ yönetimi konusunda\\
      bilgi vermeleri
  \end{itemize}

  \medskip
  \tiny{\url{http://www.computerworlduk.com/in-depth/it-business/3028/net-neutrality-a-simple-guide/}}\\
\end{frame}

\begin{frame}
  \frametitle{Örnek: Ağ tarafsızlığı yasaları}

  \begin{columns}
    \column{.48\textwidth}
    \begin{center}
      \pgfuseimage{neutrality}
    \end{center}

    \column{.52\textwidth}
    \begin{itemize}
      \item Şili'de (2010) ve\\
        Hollanda'da (2011)\\
        ağ tarafsızlığını koruma\\
        yasaları çıkıyor
    \end{itemize}
  \end{columns}

  \medskip
  \tiny{\url{http://www.theregister.co.uk/2011/06/23/netherlands_net_neutrality/}}\\
\end{frame}

\subsection*{Kaynaklar}

\begin{frame}
  \frametitle{Kaynaklar}

  \begin{block}{Okunacak: Tavani}
    \begin{itemize}
      \item Chapter 9: \alert{Regulating Commerce and Speech in Cyberspace}
      \item Chapter 11:\\
        Social Issues II: Community and Identity in Cyberspace
      \begin{itemize}
          \item 11.1. \alert{Online Communities}
          \item 11.2. \alert{Democracy and the Internet}
      \end{itemize}
    \end{itemize}
  \end{block}
\end{frame}

\section{Toplumsal Etkiler}

\subsection{Sayısal Uçurum}

\begin{frame}
  \frametitle{Sayısal Uçurum}

  \begin{tanim}
    \alert{sayısal uçurum}:
      bilişim teknolojilerinden yararlanmada eşitsizlik
  \end{tanim}

  \pause
  \begin{itemize}
    \item ülkeler arasındaki farklar
    \begin{itemize}
      \item BT kullanıcılarının çoğunluğu Kuzey Amerika ve Avrupa'da
    \end{itemize}

    \pause
    \medskip
    \item toplum içindeki farklar
    \begin{itemize}
      \item gelir düzeyi
      \item cinsiyet
      \item fiziksel engeller
    \end{itemize}
  \end{itemize}
\end{frame}

\begin{frame}
  \frametitle{Kadınların Katılımı}

  \begin{itemize}
    \item kadınlar bilişim teknolojileri alanında etkin değil

    \bigskip
    \item ilgi duyan az
    \begin{itemize}
      \item kültürel roller: "kadınlar matematik sevmez"
    \end{itemize}

    \pause
    \item yüksek eğitim alan az

    \pause
    \item meslekte çalışan az
    \begin{itemize}
      \item doğumdan sonra geri dönmek zor
    \end{itemize}

    \pause
    \item karar mekanizmalarında yer alan az
  \end{itemize}
\end{frame}

\begin{frame}
  \frametitle{Örnek: Türkiye'de kadınların katılımı}

  \begin{columns}
    \column{.47\textwidth}
    \begin{center}
      \pgfuseimage{turkey}
    \end{center}

    \column{.53\textwidth}
    \begin{itemize}
      \item Türkiye'de bilgisayar alanında\\
        yüksek eğitim gören\\
        kadınların oranı,\\
        diğer ülkelere göre yüksek
      \item lisede matematiğin zorunlu olmasına bağlanıyor
    \end{itemize}
  \end{columns}

  \medskip
  \tiny{\url{http://www.theregister.co.uk/2005/08/15/women_it_maths_mandatory/}}\\
\end{frame}

\begin{frame}
  \frametitle{Fiziksel Engellilerin Katılımı}

  \begin{itemize}
    \item fiziksel engellilerin erişimi için düzenlemeler, yönergeler
    \item World Wide Web Consortium (W3C):\\
      Web Accessibility Initiative (WAI)
  \end{itemize}
\end{frame}

\begin{frame}
  \frametitle{Ahlaki Sorun}

  \begin{itemize}
    \item sayısal uçurum ahlaki bir sorun mu?

    \medskip
    \item bilgiye erişim
    \item ekonomik sisteme katılım
    \item politik sürece katılım
  \end{itemize}
\end{frame}

\begin{frame}
  \frametitle{İnternet Erişimi}

  \begin{itemize}
    \item İnternet erişimi pozitif bir hak olmalı mı?
    \begin{itemize}
      \item eğitimde fırsat eşitliği
    \end{itemize}

    \pause
    \medskip
    \item yalnızca erişimi sağlamak yeterli mi?
    \begin{itemize}
      \item kullanma becerilerinin kazandırılması
    \end{itemize}
  \end{itemize}
\end{frame}

\subsection{İş Yaşamı}

\begin{frame}
  \frametitle{İşsizlik}

  \begin{itemize}
    \item bilişim teknolojileri işsizliğe yol açıyor mu?

    \bigskip
    \item \emph{evet}: otomasyon nedeniyle pek çok insan işinden oldu
    \item \emph{ama}: pek çok yeni iş alanı yarattı
    \item işler nitelik değiştirdi: daha fazla eğitim gerektiriyor
  \end{itemize}
\end{frame}

\begin{frame}
  \frametitle{Bilişim Sektöründe Çalışma Koşulları}

  \begin{itemize}
    \item iş güvencesi:
    \begin{itemize}
      \item işten çıkarma yaygın
      \item proje bazında geçici işçilik yaygın
      \item taşeronluk ve küreselleşme: işler başka ülkelere gidiyor
    \end{itemize}

    \pause
    \medskip
    \item çalışma saatleri
    \begin{itemize}
      \item mesai saatleri dışında çalıştırma yaygın
      \item fazla mesai, tazminat zorlukları
    \end{itemize}
  \end{itemize}
\end{frame}

\begin{frame}
  \frametitle{İşyerinde Gözetleme}

  \begin{itemize}
    \item çalışanlar mesai saatlerinde İnternet'e iş dışı amaçlarla\\
      bağlanıyor

    \item işverenler çeşitli önlemler alıyor:
    \begin{itemize}
      \item web kullanımının gözetlenmesi
      \item iletişim kanallarının (e-posta, mesajlaşma) gözetlenmesi
    \end{itemize}
    \item bu önlemler mahremiyet sorunları yaratıyor

    \medskip
    \item bazı ilkeler:
    \begin{itemize}
      \item çalışanlara gözetleme yapıldığının açıkça bildirilmesi
      \item çalışanların kendileriyle ilgili toplanan bilgileri görebilmesi\\
        ve itiraz edebilmesi
      \item toplanan bilgilere göre değerlendirme yapmadan önce\\
        bilgilerin doğrulanması
    \end{itemize}
  \end{itemize}
\end{frame}

\begin{frame}
  \frametitle{Evden Çalışma}

  \begin{itemize}
    \item evden çalışma (\emph{telecommuting})
    \begin{itemize}
      \item şirketler için cazip: ofis kiraları yüksek
    \end{itemize}

    \medskip
    \item avantajları:
    \begin{itemize}
      \item kadınların katılımı: özellikle doğumdan sonra
      \item engellilerin katılımı
    \end{itemize}

    \pause
    \medskip
    \item dezavantajları:
    \begin{itemize}
      \item işten çıkartmalar ve yükseltmelerde dezavantaj
      \item çalışma saatleri belirsizleşiyor
    \end{itemize}
  \end{itemize}
\end{frame}

\subsection*{Kaynaklar}

\begin{frame}
  \frametitle{Kaynaklar}

  \begin{block}{Okunacak: Tavani}
    \begin{itemize}
      \item Chapter 10:\\
        \alert{Social Issues I: Equity and Access, Employment and Work}
    \end{itemize}
  \end{block}
\end{frame}

\end{document}
