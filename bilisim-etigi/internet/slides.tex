% Copyright (c) 2004-2015 H. Turgut Uyar <uyar@itu.edu.tr>
%
% This work is licensed under a "Creative Commons
% Attribution-NonCommercial-ShareAlike 4.0 International License".
% For more information, please visit:
% https://creativecommons.org/licenses/by-nc-sa/4.0/

\documentclass[dvipsnames]{beamer}

\usepackage{ae}
\usepackage[T1]{fontenc}
\usepackage[turkish]{babel}
\usepackage[utf8]{inputenc}
\setbeamertemplate{navigation symbols}{}
\setbeamersize{text margin left=2em, text margin right=2em}

\mode<presentation>
{
  \usetheme{Rochester}
  \usecolortheme[named=Mahogany]{structure}
  \setbeamercovered{transparent}
}

\title{Bilişim Etiği}
\subtitle{İnternet}

\author{H. Turgut Uyar}
\date{2004-2015}

\AtBeginSubsection[]
{
  \begin{frame}<beamer>
    \frametitle{Konular}
    \tableofcontents[currentsection,currentsubsection]
  \end{frame}
}

%\beamerdefaultoverlayspecification{<+->}

\theoremstyle{plain}

\pgfdeclareimage[width=2cm]{license}{../license}

\pgfdeclareimage[width=5.6cm]{tomcruise}{tomcruise}
\pgfdeclareimage[width=5.8cm]{milka}{milka}
\pgfdeclareimage[width=5.8cm]{gmail}{gmail}
\pgfdeclareimage[width=5.8cm]{goggle}{goggle}
\pgfdeclareimage[width=5.6cm]{icann}{icann}
\pgfdeclareimage[width=5.6cm]{sucks}{sucks}
\pgfdeclareimage[height=6.8cm]{china}{china}
\pgfdeclareimage[width=5.6cm]{neutrality}{neutrality}
\pgfdeclareimage[height=6.8cm]{turkey}{turkey}

\begin{document}

\begin{frame}
  \titlepage
\end{frame}

\begin{frame}
  \frametitle{License}

  \pgfuseimage{license}\hfill
  \copyright~2004-2015 H. Turgut Uyar

  \vfill
  \begin{footnotesize}
    You are free to:
    \begin{itemize}
      \itemsep0em
      \item Share -- copy and redistribute the material in any medium or format
      \item Adapt -- remix, transform, and build upon the material
    \end{itemize}

    Under the following terms:
    \begin{itemize}
      \itemsep0em
      \item Attribution -- You must give appropriate credit, provide a link to
        the license, and indicate if changes were made.

      \item NonCommercial -- You may not use the material for commercial
        purposes.

      \item ShareAlike -- If you remix, transform, or build upon the material,
        you must distribute your contributions under the same license as the
        original.
    \end{itemize}
  \end{footnotesize}

  \begin{small}
    For more information:\\
    \url{https://creativecommons.org/licenses/by-nc-sa/4.0/}

    \smallskip
    Read the full license:\\
    \url{https://creativecommons.org/licenses/by-nc-sa/4.0/legalcode}
  \end{small}
\end{frame}

\begin{frame}
  \frametitle{Konular}
  \tableofcontents
\end{frame}

\section{İnternet}

\subsection{Alan Adları}

\begin{frame}
  \frametitle{Alan Adları}

  \begin{itemize}
    \item alan adlarının dağıtımı adil mi?

    \bigskip
    \item 1998'e kadar: NSF
    \item ``ilk gelen alır``
    \item başkasının alan adını alma: \emph{sanal çöreklenme}

    \pause
    \medskip
    \item 1998'den sonra: ICANN
    \item kar amacı gütmeyen bir kuruluş
    \item Uniform Domain-Name Dispute-Resolution Policy
    \item tescilli markalar geçerli
  \end{itemize}
\end{frame}

\begin{frame}
  \frametitle{Anlaşmazlıkların Çözümü}

  \begin{itemize}
    \item WIPO Hakemlik ve Arabuluculuk Merkezi

    \bigskip
    \item bir alan adının aktarılması için:
    \smallskip
    \item isim tescilli markayla aynı ya da çok benzer olmalı
    \item eski sahibinin bu isim üzerinde hiçbir hakkı olmamalı
    \item kötü niyet olmalı
  \end{itemize}
\end{frame}

\begin{frame}
  \frametitle{Örnek: TomCruise.com}

  \begin{columns}
    \column{.48\textwidth}
    \begin{center}
      \pgfuseimage{tomcruise}
    \end{center}

    \column{.52\textwidth}
    \begin{itemize}
      \item WIPO, TomCruise.com\\
        alanını, kaydeden kişiden alıp\\
        sinema oyuncusu\\
        Tom Cruise'a veriyor (2006)
    \end{itemize}
  \end{columns}

  \medskip
  \tiny{\url{http://www.theregister.co.uk/2006/07/23/tom_cruise_dotcom_win/}}\\
  \tiny{\url{http://www.theregister.co.uk/2004/12/17/ronaldinho_scores_own_domain_name/}}\\
  \tiny{\url{http://www.theregister.co.uk/2006/10/13/rooney_wins_dotcom/}}\\
  \tiny{\url{http://www.theregister.co.uk/2012/03/19/pope_benedict_cybersquatter/}}\\
\end{frame}

\begin{frame}
  \frametitle{Örnek: milka.fr}

  \begin{columns}
    \column{.48\textwidth}
    \begin{center}
      \pgfuseimage{milka}
    \end{center}

    \column{.52\textwidth}
    \begin{itemize}
      \item Fransız modacı Milka Budimir\\
        milka.fr alanını kaydettirmiş
      \item Kraft istiyor
      \item mahkeme ismi Kraft'a\\
        veriyor (2005)
    \end{itemize}
  \end{columns}

  \medskip
  \tiny{\url{http://news.bbc.co.uk/2/hi/europe/4348585.stm}}\\
\end{frame}

\begin{frame}
  \frametitle{Örnek: Gmail}

  \begin{columns}
    \column{.48\textwidth}
    \begin{center}
      \pgfuseimage{gmail}
    \end{center}

    \column{.52\textwidth}
    \begin{itemize}
      \item Google, İngiltere'de (2005)\\
        ve Almanya'da (2007)\\
        Gmail ismini kullanamıyor
    \end{itemize}
  \end{columns}

  \medskip
  \tiny{\url{http://news.bbc.co.uk/2/hi/business/4354954.stm}}\\
  \tiny{\url{https://mashable.com/2007/10/02/google-german-domain/}}\\
\end{frame}

\begin{frame}
  \frametitle{İsim Benzerlikleri}

  \begin{itemize}
    \item benzeyen isimler de anlaşmazlık konusu olabiliyor
    \item Microsoft: mikerowesoft.com, mocosoft.com

    \pause
    \bigskip
    \item \emph{yazım hatasına çöreklenme}
    \item arama motorları ve tarayıcılar aynı şekilde para kazanmıyor mu?
  \end{itemize}

  \medskip
  \tiny{\url{http://www.theregister.co.uk/2004/01/19/microsoft_lawyers_threaten_mike_rowe/}}\\
  \tiny{\url{http://www.theregister.co.uk/2004/12/15/mocosoft_beats_microsoft/}}\\
  \tiny{\url{http://www.theregister.co.uk/2008/10/23/google_and_typosquatting/}}\\
\end{frame}

\begin{frame}
  \frametitle{Örnek: Google}

  \begin{columns}
    \column{.48\textwidth}
    \begin{center}
      \pgfuseimage{goggle}
    \end{center}

    \column{.52\textwidth}
    \begin{itemize}
      \item Google, çeşitli alan adlarını\\
        kazanıyor (2005):\\
        googkle.com, ghoogle.com,\\
        gfoogle.com, gooigle.com
      \item goggle.com alan adını\\
        kazanamıyor (2011)
    \end{itemize}
  \end{columns}

  \medskip
  \tiny{\url{http://www.theregister.co.uk/2005/07/11/google_ruling/}}\\
  \tiny{\url{http://www.theregister.co.uk/2011/10/12/google_v_goggle/}}\\
\end{frame}

\begin{frame}
  \frametitle{Örnek: ICANN çıkar çatışmaları}

  \begin{columns}
    \column{.48\textwidth}
    \begin{center}
      \pgfuseimage{icann}
    \end{center}

    \column{.52\textwidth}
    \begin{itemize}
      \item ICANN başkanı,\\
        yönetim kurulundakilerin\\
        çıkar çatışmalarına\\
        dikkat çekiyor (2012)
    \end{itemize}
  \end{columns}

  \medskip
  \tiny{\url{http://www.theregister.co.uk/2012/03/19/icann_president_calls_out_his_own_board_over_conflicts_of_interest/}}\\
\end{frame}

\subsection{İfade Özgürlüğü}

\begin{frame}
  \frametitle{İfade Özgürlüğü}

  \begin{itemize}
    \item neler ifade özgürlüğüne girmez?

    \bigskip
    \item çocuk pornografisi
    \item nefret ya da şiddet propagandası
    \item suça ya da zararlı davranışa özendirme
    \begin{itemize}
      \item nasıl bomba yapılır, nasıl acısız intihar edilir?
    \end{itemize}
    \item hakaret
  \end{itemize}
\end{frame}

\begin{frame}
  \frametitle{Protesto Siteleri}

  \begin{itemize}
    \item insanlar memnun olmadığı kurumları protesto etmek için\\
      site açıyorlar

    \bigskip
    \item anlaşmazlıklar iki açıdan değerlendiriliyor:
    \smallskip
    \item sitenin içeriği (yalan, hakaret, ifade özgürlüğü)
    \item alan adı (tescilli marka)
  \end{itemize}
\end{frame}

\begin{frame}
  \frametitle{Örnek: Air France, Wal-Mart}

  \begin{columns}
    \column{.48\textwidth}
    \begin{center}
      \pgfuseimage{sucks}
    \end{center}

    \column{.52\textwidth}
    \begin{itemize}
      \item WIPO, airfrancesucks.com,\\
        wal-martcanadasucks.com\\
        gibi alan adlarını\\
        ilgili şirketlere veriyor
      \item ABD Temyiz Mahkemesi,\\
        ticari amaç gütmeyen\\
        protesto sitelerinin\\
        ifade özgürlüğüne girdiğine\\
        karar veriyor (2005)
    \end{itemize}
  \end{columns}

  \medskip
  \tiny{\url{http://www.theregister.co.uk/2005/04/05/bosley_case_appeal/}}\\
  \tiny{\url{http://www.theregister.co.uk/2006/12/29/wipo_rules_against_ryanair/}}\\
  \tiny{\url{http://www.theregister.co.uk/2009/07/28/wipo_free_speech/}}\\
\end{frame}

\subsection{Demokrasi}

\begin{frame}
  \frametitle{Demokrasi}

  \begin{itemize}
    \item İnternet demokratik bir platform mu?
    \item İnternet demokrasiye katkı yapıyor mu?
    \item İnternet'in demokrasiye katkı yapması beklenmeli mi?
  \end{itemize}
\end{frame}

\begin{frame}
  \frametitle{Demokrasiye Katkı}

  \begin{itemize}
    \item \emph{evet}: düşük maliyetle bilgiye erişim sağlıyor

    \medskip
    \item \emph{ama}: bilginin önemli bir kısmı yanlış
  \end{itemize}
\end{frame}

\begin{frame}
  \frametitle{Demokrasiye Katkı}

  \begin{itemize}
    \item \emph{evet}: insanlar coğrafyadan bağımsız olarak biraraya
      gelebiliyor
    \item farklı insanları ve kültürleri tanımak sivrilikleri törpüler

    \medskip
    \item \emph{ama}: kişisel tercihler ters yönde etki yapıyor
    \item kendimiz gibi düşünen insanlarla biraraya geliyoruz
    \item aşırılıklara kaymak kolaylaşıyor
  \end{itemize}
\end{frame}

\begin{frame}
  \frametitle{Demokrasiye Katkı}

  \begin{itemize}
    \item \emph{evet}: bireyler ve azınlık grupları seslerini duyurabiliyor

    \medskip
    \item \emph{ama}: kaotik bir platform
    \item insanların ilgisini çekmek gerek
    \item arama motorlarına kaydettirme
    \item kaynak gerektiriyor: güçlüye daha çok güç mü veriyor?
  \end{itemize}
\end{frame}

\begin{frame}
  \frametitle{Arama Motorları}

  \begin{itemize}
    \item çoğu insan için İnternet'e giriş noktası arama motorları

    \medskip
    \item arama motorlarının sonuçları çok önemli
    \item hangi sayfalar gösterilecek?
    \item ne sırayla gösterilecek?
  \end{itemize}
\end{frame}

\begin{frame}
  \frametitle{Örnek: Google - Çin}

  \begin{columns}
    \column{.48\textwidth}
    \begin{center}
      \pgfuseimage{china}
    \end{center}

    \column{.52\textwidth}
    \begin{itemize}
      \item Çin'den yapılan aramalarda\\
        arama sonuçları sansürleniyor
      \item Google sansürlü servisini\\
        iptal ediyor (2010)
      \item içerik sağlayıcı lisansını\\
        kaybediyor
    \end{itemize}
  \end{columns}

  \medskip
  \tiny{\url{http://www.guardian.co.uk/technology/2010/mar/25/china-microsoft-free-speech-google}}\\
\end{frame}

\begin{frame}
  \frametitle{Ağ Tarafsızlığı}

  \begin{itemize}
    \item \alert{ağ tarafsızlığı}:\\
      İnternet'in her kullanıcıya, her uygulama sağlayıcıya\\
      ve her taşıyıcıya açık, erişilebilir ve ayrımsız olması

    \medskip
    \item aksi örnekler:
    \item uygulama engelleme (örneğin VoIP)
    \item uygulama bant genişliği kısıtlama (örneğin VoIP)
    \item bazı site ya da hizmetlere erişimi engelleme
    \item bazı site ya da hizmetlere erişimi önceliklendirme
  \end{itemize}
\end{frame}

\begin{frame}
  \frametitle{Ağ Tarafsızlığı İlkeleri}

  \begin{itemize}
    \item her türlü yasal içeriğe ulaşabilme
    \item istediği uygulama ve hizmetleri kullanabilme
    \item ağa zarar vermediği sürece, istediği aygıtla bağlanabilmesi
    \item içerik ve uygulama sağlayıcılar arasındaki rekabetten\\
      yararlanabilme
  \end{itemize}

  \medskip
  \tiny{\url{http://www.computerworlduk.com/in-depth/it-business/3028/net-neutrality-a-simple-guide/}}\\
\end{frame}

\begin{frame}
  \frametitle{Örnek: Ağ tarafsızlığı yasaları}

  \begin{columns}
    \column{.48\textwidth}
    \begin{center}
      \pgfuseimage{neutrality}
    \end{center}

    \column{.52\textwidth}
    \begin{itemize}
      \item Şili'de (2010) ve\\
        Hollanda'da (2011)\\
        ağ tarafsızlığını koruma\\
        yasaları çıkıyor
    \end{itemize}
  \end{columns}

  \medskip
  \tiny{\url{http://www.theregister.co.uk/2011/06/23/netherlands_net_neutrality/}}\\
\end{frame}

\subsection*{Kaynaklar}

\begin{frame}
  \frametitle{Kaynaklar}

  \begin{block}{Okunacak: Tavani}
    \begin{itemize}
      \item Chapter 9: \alert{Regulating Commerce and Speech in Cyberspace}
      \item Chapter 11:\\
        Social Issues II: Community and Identity in Cyberspace
      \begin{itemize}
          \item 11.1. \alert{Online Communities}
          \item 11.2. \alert{Democracy and the Internet}
      \end{itemize}
    \end{itemize}
  \end{block}
\end{frame}

\section{Toplumsal Etkiler}

\subsection{Sayısal Uçurum}

\begin{frame}
  \frametitle{Sayısal Uçurum}

  \begin{itemize}
    \item \alert{sayısal uçurum}:
      bilişim teknolojilerinden yararlanmada eşitsizlik

    \medskip
    \item ülkeler arasındaki farklar
    \item BT kullanıcılarının çoğunluğu Kuzey Amerika ve Avrupa'da

    \medskip
    \item toplum içindeki farklar
    \item gelir düzeyi, cinsiyet, fiziksel engeller
  \end{itemize}
\end{frame}

\begin{frame}
  \frametitle{Etik Sorunu}

  \begin{itemize}
    \item sayısal uçurum bir etik sorunu mu?

    \medskip
    \item bilgiye erişim
    \item ekonomik sisteme katılım
    \item politik sürece katılım
  \end{itemize}
\end{frame}

\begin{frame}
  \frametitle{Kadınların Katılımı}

  \begin{itemize}
    \item kadınlar bilişim teknolojileri alanında etkin değil

    \bigskip
    \item ilgi duyan az
    \item kültürel roller: ''kadınlar matematik sevmez``

    \medskip
    \item yüksek eğitim alan daha az

    \medskip
    \item meslekte çalışan daha da az
    \item doğumdan sonra geri dönmek zor

    \medskip
    \item yönetici pozisyonlarında çalışan çok az
  \end{itemize}
\end{frame}

\begin{frame}
  \frametitle{Örnek: Türkiye}

  \begin{columns}
    \column{.47\textwidth}
    \begin{center}
      \pgfuseimage{turkey}
    \end{center}

    \column{.53\textwidth}
    \begin{itemize}
      \item Türkiye'de bilişim alanında\\
        kadınların oranı diğer ülkelere\\
        göre yüksek
      \item lisede matematiğin zorunlu\\
        olmasına bağlanıyor
    \end{itemize}
  \end{columns}

  \medskip
  \tiny{\url{http://www.theregister.co.uk/2005/08/15/women_it_maths_mandatory/}}\\
\end{frame}

\begin{frame}
  \frametitle{Fiziksel Engellilerin Katılımı}

  \begin{itemize}
    \item fiziksel engelli kullanıcıların kullanımı için yönergeler
    \item World Wide Web Consortium (W3C):\\
      Web Accessibility Initiative (WAI)
  \end{itemize}
\end{frame}

\begin{frame}
  \frametitle{İnternet Erişimi}

  \begin{itemize}
    \item İnternet erişimi pozitif bir hak olmalı mı?
    \item eğitimde fırsat eşitliği

    \medskip
    \item yalnızca erişimi sağlamak yeterli mi?
    \item kullanma becerilerinin kazandırılması
  \end{itemize}
\end{frame}

\subsection{İş Yaşamı}

\begin{frame}
  \frametitle{İşsizlik}

  \begin{itemize}
    \item bilişim teknolojileri işsizliğe yol açıyor mu?

    \bigskip
    \item \emph{evet}: otomasyon nedeniyle pek çok insan işinden oldu
    \item \emph{ama}: pek çok yeni iş alanı yarattı
    \item işler nitelik değiştirdi: daha fazla eğitim gerektiriyor
  \end{itemize}
\end{frame}

\begin{frame}
  \frametitle{Bilişim Sektöründe Çalışma Koşulları}

  \begin{itemize}
    \item iş güvencesi
    \item işten çıkarmalar yaygın
    \item proje bazında geçici işçilik yaygın
    \item taşeronluk ve küreselleşme: işler başka ülkelere gidiyor

    \pause
    \medskip
    \item çalışma saatleri
    \item mesai saatleri dışında çalıştırma yaygın
    \item fazla mesai, tazminat zorlukları
  \end{itemize}
\end{frame}

\begin{frame}
  \frametitle{İşyerinde Gözetleme}

  \begin{itemize}
    \item çalışanlar mesai saatlerinde İnternet'e iş dışı amaçlarla\\
      bağlanıyor

    \medskip
    \item işverenler çeşitli önlemler alıyor
    \smallskip
    \item web kullanımının gözetlenmesi
    \item iletişim kanallarının (e-posta, mesajlaşma) gözetlenmesi
    \item mahremiyet sorunları yaratıyor
  \end{itemize}
\end{frame}

\begin{frame}
  \frametitle{İşyerinde Gözetleme}

  \begin{itemize}
    \item çalışanlara gözetleme yapıldığının açıkça bildirilmesi
    \item çalışanların kendileriyle ilgili toplanan bilgileri görebilmesi\\
      ve itiraz edebilmesi
    \item toplanan veriye göre değerlendirme yapmadan önce\\
      verinin doğrulanması
  \end{itemize}
\end{frame}

\begin{frame}
  \frametitle{Evden Çalışma}

  \begin{itemize}
    \item evden çalışma (\emph{telecommuting})
    \item şirketler için cazip: ofis kiraları yüksek

    \medskip
    \item kadınların katılımı, özellikle doğumdan sonra
    \item engellilerin katılımı

    \pause
    \medskip
    \item işten çıkartmalar ve yükseltmelerde dezavantaj
    \item çalışma saatleri belirsizleşiyor
  \end{itemize}
\end{frame}

\subsection*{Kaynaklar}

\begin{frame}
  \frametitle{Kaynaklar}

  \begin{block}{Okunacak: Tavani}
    \begin{itemize}
      \item Chapter 10:\\
        \alert{Social Issues I: Equity and Access, Employment and Work}
    \end{itemize}
  \end{block}
\end{frame}

\end{document}
