% Copyright (c) 2004-2015 H. Turgut Uyar <uyar@itu.edu.tr>
%
% This work is licensed under a "Creative Commons
% Attribution-NonCommercial-ShareAlike 4.0 International License".
% For more information, please visit:
% https://creativecommons.org/licenses/by-nc-sa/4.0/

\documentclass[dvipsnames]{beamer}

\usepackage{ae}
\usepackage[T1]{fontenc}
\usepackage[utf8]{inputenc}
\usepackage[turkish]{babel}
\setbeamertemplate{navigation symbols}{}
\setbeamersize{text margin left=2em, text margin right=2em}

\mode<presentation>
{
  \usetheme{Rochester}
  \usecolortheme[named=Mahogany]{structure}
  \setbeamercovered{transparent}
}

\title{Bilişim Etiği}
\subtitle{Etik Kuramları}

\author{H. Turgut Uyar}
\date{2004-2015}

\AtBeginSubsection[]
{
  \begin{frame}<beamer>
    \frametitle{Konular}
    \tableofcontents[currentsection,currentsubsection]
  \end{frame}
}

%\beamerdefaultoverlayspecification{<+->}

\theoremstyle{plain}

\pgfdeclareimage[height=1cm]{license}{../license}

\pgfdeclareimage[height=4.5cm]{bentham}{bentham}
\pgfdeclareimage[height=4.5cm]{kant}{kant}
\pgfdeclareimage[height=4.5cm]{hobbes}{hobbes}
\pgfdeclareimage[height=4.5cm]{platon}{platon}

\begin{document}

\begin{frame}
  \titlepage
\end{frame}

\begin{frame}
  \frametitle{Lisans}

  \pgfuseimage{license}\hfill
  \copyright~2004-2015 H. Turgut Uyar

  \vfill
  \begin{footnotesize}
    You are free to:
    \begin{itemize}
      \itemsep0em
      \item Share -- copy and redistribute the material in any medium or format
      \item Adapt -- remix, transform, and build upon the material
    \end{itemize}

    Under the following terms:
    \begin{itemize}
      \itemsep0em
      \item Attribution -- You must give appropriate credit, provide a link to
        the license, and indicate if changes were made.

      \item NonCommercial -- You may not use the material for commercial
        purposes.

      \item ShareAlike -- If you remix, transform, or build upon the material,
        you must distribute your contributions under the same license as the
        original.
    \end{itemize}
  \end{footnotesize}

  \begin{small}
    For more information:\\
    \url{https://creativecommons.org/licenses/by-nc-sa/4.0/}

    \smallskip
    Read the full license:\\
    \url{https://creativecommons.org/licenses/by-nc-sa/4.0/legalcode}
  \end{small}
\end{frame}

\begin{frame}
  \frametitle{Konular}
  \tableofcontents
\end{frame}

\section{Giriş}

\subsection{Tanımlar}

\begin{frame}
  \frametitle{Deskriptif Önermeler}

  \begin{itemize}
    \item \alert{deskriptif} önerme: ne olduğu
    \item sosyoloji, psikoloji, antropoloji, politik bilimlerin konusu
  \end{itemize}

  \pause
  \begin{exampleblock}{örnek}
    \begin{quote}
      bilgisayar kullanıcılarının \%85'i yazılımların\\
      lisans anlaşmalarına uymuyor
    \end{quote}
  \end{exampleblock}
\end{frame}

\begin{frame}
  \frametitle{Normatif Önermeler}

  \begin{itemize}
    \item \alert{normatif} önerme: ne olması gerektiği
    \item felsefenin konusu
  \end{itemize}

  \pause
  \begin{exampleblock}{örnek}
    \begin{quote}
      bilgisayar kullanıcıları yazılımların lisans anlaşmalarına uymalı
    \end{quote}
  \end{exampleblock}
\end{frame}

\begin{frame}
  \frametitle{Ahlak Sistemi}

  \begin{itemize}
    \item davranış kuralları: bireysel, toplumsal
    \item değerlendirme ilkeleri: toplumsal yarar, \ldots

    \pause
    \bigskip
    \item \emph{açık}: herkes biliyor
    \item \emph{resmi değil}: zorlayıcı yetkili yok
    \item \emph{akılcı}: akıl yürüterek anlaşılabilir
    \item \emph{tarafsız}: bütün katılımcılar eşit
  \end{itemize}
\end{frame}

\begin{frame}
  \frametitle{Kuralların Oluşturulması}

  \begin{itemize}
    \item \alert{temel değerler} gözönüne alınarak
    \item \emph{özünde iyi}: mutluluk, özerklik, mahremiyet, \ldots
    \item \emph{araç olarak iyi}: para, \ldots

    \pause
    \bigskip
    \item ahlak ilkelerinin dayandırıldığı zeminden yola çıkılarak
    \item din
    \item hukuk
    \item felsefe
  \end{itemize}
\end{frame}

\subsection{Metodoloji}

\begin{frame}
  \frametitle{Felsefe Metodolojisi}

  \begin{itemize}
    \item \alert{diyalektik}

    \medskip
    \item bir tez ortaya koy, ilkeyle açıkla
    \item ilkeyi çeşitli durumlarda sına
    \item tezini ve/veya ilkeyi uyarla
  \end{itemize}
\end{frame}

\begin{frame}
  \frametitle{Diyalektik Örneği}

  \begin{itemize}
    \item ``ötanazi yanlıştır\\
      çünkü insan yaşamı kasıtlı olarak sona erdirilmemelidir''

    \pause
    \medskip
    \item bilinçli ve çok acı çekiyor
    \item bilinçsiz ve beyin hasarı var
    \item genç - yaşlı

    \pause
    \smallskip
    \item ``yaşam kalitesi''

    \pause
    \medskip
    \item başka sorunlara uygulamada tutarlılık:\\
      savaş, idam, kürtaj, \ldots
    \item ``kendini savunma, başka yaşam kurtarma, \ldots''
  \end{itemize}
\end{frame}

\subsection{Tartışma Sorunları}

\begin{frame}
  \frametitle{Tartışma Sorunları}

  \begin{itemize}
    \item felsefeciler ahlaki sorunların çözümünde anlaşamıyor
    \item başkaları nasıl anlaşsın?

    \pause
    \bigskip
    \item başka alanlarda da uzmanlar anlaşamayabiliyor
    \item ışık: dalga mı, parçacık mı?

    \pause
    \medskip
    \item birçok konuda anlaşma var

    \pause
    \medskip
    \item ilkelerde anlaşmazlık - durumlarda anlaşmazlık
  \end{itemize}
\end{frame}

\begin{frame}
  \frametitle{Relativizm}

  \begin{itemize}
    \item kültürel relativizm
    \item ``değişik toplumlar ahlaken neyin doğru neyin yanlış olduğu\\
      konusunda farklı inanışlara sahiptir''

    \pause
    \medskip
    \item deskriptif önerme, normatif olursa:
    \item ``bir toplumdaki bireyler için neyin doğru neyin yanlış olduğuna\\
      yalnızca o toplum karar verebilir''
    \item ahlaki relativizm

    \pause
    \medskip
    \item bazı evrensel ahlak kuralları var
  \end{itemize}
\end{frame}

\subsection{Davranış İlkeleri}

\begin{frame}
  \frametitle{İlkeler}

  \begin{itemize}
    \item etik kuramları neden gerekiyor?

    \medskip
    \item seçme durumunda kaldığımızda karar vermemize yardımcı
    \item ahlaki konuların çözümlenmesinde yardımcı
  \end{itemize}
\end{frame}

\begin{frame}
  \frametitle{Altın Kural}

  \begin{itemize}
    \item ``başkaları sana nasıl davransın istiyorsan\\
      sen de onlara öyle davran''
  \end{itemize}

  \pause
  \medskip
  \begin{exampleblock}{karşı örnek}
    \begin{itemize}
      \item Ben bir yazılım geliştiriciyim.
      \item Başkalarının benim yazılımlarımı benim iznim olmadan\\
        çoğaltıp dağıtmaları benim için sorun değil.
      \item O halde ben de başkalarının yazılımlarını\\
        onların izni olmadan çoğaltıp dağıtabilirim.
    \end{itemize}
  \end{exampleblock}
\end{frame}

\section{Etik Kuramlar}

\subsection{Faydacılık}

\begin{frame}
  \frametitle{Faydacılık}

  \begin{columns}
    \column{.5\textwidth}
    \begin{block}{faydacılık}
      bir davranış, etkilenecek olanların\\
      en büyük kısmını en mutlu edecek\\
      sonuçlara yol açacaksa\\
      ahlaken doğrudur
    \end{block}

    \begin{itemize}
      \item sonuca dayalı
    \end{itemize}

    \column{.5\textwidth}
    \begin{center}
      \pgfuseimage{bentham}\\
      Jeremy Bentham (1748-1832)
    \end{center}
  \end{columns}
\end{frame}

\begin{frame}
  \frametitle{Faydacılık Sorun Örneği}

  \begin{itemize}
    \item Bir mağazaya giriyorum ve bir gömlek beğeniyorum.
    \item Çalsam mı?
    \item Hesap yap ve karar ver.

    \pause
    \bigskip
    \item Çıkıyorum, başka bir mağazaya giriyorum ve bir kravat beğeniyorum.
    \item Çalsam mı?
    \item \ldots
  \end{itemize}
\end{frame}

\begin{frame}
  \frametitle{Faydacılık}

  \begin{columns}[t]
    \column{.5\textwidth}
    \begin{block}{eylem faydacılığı}
      daha çok sayıda kişinin\\
      daha mutlu olmasını\\
      sağlayacak şekilde davran
    \end{block}

    \pause
    \column{.5\textwidth}
    \begin{block}{kural faydacılığı}
      herkes böyle davransaydı\\
      daha çok sayıda insanın\\
      daha mutlu olmasını\\
      sağlayacak şekilde davran
    \end{block}
  \end{columns}
\end{frame}

\begin{frame}
  \frametitle{Faydacılık Sorun Örnekleri}

  \begin{exampleblock}{eylem faydacılığı}
    \begin{itemize}
      \item "bir kişiyi öldürüp organlarıyla on kişiyi kurtaralım"
      \item "toplumun \%1'ini diğer \%99'un kölesi olarak çalıştıralım"
    \end{itemize}
  \end{exampleblock}

  \pause
  \begin{exampleblock}{kural faydacılığı}
    \begin{itemize}
      \item "toplumun \%1'ini köle olarak çalıştırmak toplum huzurunu bozar"
    \end{itemize}
  \end{exampleblock}
\end{frame}

\begin{frame}
  \frametitle{Faydacılığın Eleştirisi}

  \begin{itemize}
    \item ahlak mutluluğa ya da zevke bağlanıyor

    \pause
    \medskip
    \item davranışın sonucu önceden bilinmiyor: \emph{ahlak şansı}

    \pause
    \medskip
    \item karar vermeye yardımcı değil
    \item fayda hesabı nasıl yapılacak?

    \pause
    \medskip
    \item iyi sonuçların adil dağılımı?
  \end{itemize}
\end{frame}

\subsection{Deontoloji}

\begin{frame}
  \frametitle{Deontoloji}

  \begin{columns}
    \column{.5\textwidth}
    \begin{itemize}
      \item mutluluk için içgüdü yeterlidir
      \item oysa insanı hayvandan ayıran\\
        akıl yürütme yeteneğidir
      \item bu yetenek insana\\
        ahlaki bir ödev yükler
    \end{itemize}

    \begin{itemize}
      \item ödeve dayalı
    \end{itemize}

    \column{.5\textwidth}
    \begin{center}
      \pgfuseimage{kant}\\
      Immanuel Kant (1724-1804)
    \end{center}
  \end{columns}
\end{frame}

\begin{frame}
  \frametitle{Kesin Buyruk}

  \begin{block}{kesin buyruk}
    \begin{itemize}
      \item insanları asla yalnızca bir amaca ulaşmak için\\
        araç olarak kullanma
      \item öyle bir kurala göre davran ki,\\
        bu kural bütün insanlar için bağlayıcı olsun
    \end{itemize}
  \end{block}

  \pause
  \begin{itemize}
    \item ödevler çelişirse ne olacak?
  \end{itemize}
\end{frame}

\begin{frame}
  \frametitle{Kesin Buyruk Örneği}

  \begin{itemize}
    \item kölelik yanlıştır çünkü

    \medskip
    \item bir grup insan bir amaca hizmet için araç olarak kullanılmış olur
    \item tarafsız ve evrensel bir kural olarak uygulanmasını kimse istemez
  \end{itemize}
\end{frame}

\subsection{Toplumsal Sözleşme}

\begin{frame}
  \frametitle{Toplumsal Sözleşme Kuramı}

  \begin{columns}
    \column{.5\textwidth}
    \begin{center}
      \pgfuseimage{hobbes}\\
      Thomas Hobbes (1588-1679)
    \end{center}

    \pause
    \column{.5\textwidth}
    \begin{itemize}
      \item ahlak öncesi durum:\\
        herkes kendi çıkarına göre\\
        davranıyor
      \item özgürlük algısı var\\
        ama sürekli tehdit de var
      \item özgürlüğümüzün bir kısmını\\
        egemen bir yetkiye devrederiz
    \end{itemize}

    \begin{itemize}
      \item sözleşmeye dayalı
    \end{itemize}
  \end{columns}
\end{frame}

\begin{frame}
  \frametitle{Toplumsal Sözleşme Kuramının Eleştirisi}

  \begin{itemize}
    \item sözleşme yoksa ahlaki sorun da yok
    \item kimsenin kimseye yardım etmesi gerekmiyor

    \pause
    \medskip
    \item yasaya aykırı olan ahlaka aykırı olmayabilir
  \end{itemize}

  \begin{exampleblock}{örnek: ırk ayrımı yasaları}
    \begin{itemize}
      \item ABD, Güney Afrika (apartheid)
      \item bu yasalara itaatsizlik yanlış mı?
    \end{itemize}
  \end{exampleblock}
\end{frame}

\begin{frame}
  \frametitle{Haklar}

  \begin{columns}
    \column{.5\textwidth}
    \begin{itemize}
      \item \alert{negatif} haklar: engellenmeme
    \end{itemize}
    \begin{exampleblock}{örnekler}
      \begin{itemize}
        \item oy verme
        \item yüksek eğitim
      \end{itemize}
    \end{exampleblock}

    \pause
    \column{.5\textwidth}
    \begin{itemize}
      \item \alert{pozitif} haklar: desteklenme
    \end{itemize}
    \begin{exampleblock}{örnekler}
      \begin{itemize}
        \item temel eğitim
        \item sağlık?
      \end{itemize}
    \end{exampleblock}
  \end{columns}
\end{frame}

\subsection{Erdem Ahlakı}

\begin{frame}
  \frametitle{Erdem Ahlakı}

  \begin{columns}
    \column{.4\textwidth}
    \begin{center}
      \pgfuseimage{platon}\\
      Platon (İ.Ö. 4. yy)
    \end{center}

    \pause
    \column{.6\textwidth}
    \begin{itemize}
      \item iyi kişilik özellikleri edinme

      \item ``şu durumda ne yapmalıyım'' değil\\
        ``nasıl bir insan olmalıyım''
    \end{itemize}

    \begin{itemize}
      \item kişiliğe dayalı
    \end{itemize}
  \end{columns}
\end{frame}

\section*{Kaynaklar}

\begin{frame}
  \frametitle{Kaynaklar}

  \begin{block}{Okunacak: Tavani}
    \begin{itemize}
      \item Chapter 2: \alert{Ethical Concepts and Ethical Theories}
    \end{itemize}
  \end{block}
\end{frame}

\end{document}
