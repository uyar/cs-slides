% Copyright (c) 2004-2012 H. Turgut Uyar <uyar@itu.edu.tr>
%
% Bu notlar "Creative Commons Attribution-NonCommercial-ShareAlike License" ile
% lisanslanmıştır. Yazarının açıkça belirtilmesi koşuluyla ve ticari olmayan
% amaçlarla kullanılabilir ve dağıtılabilir. Bu notlardan yola çıkılarak
% oluşturulacak çalışmaların da aynı lisansa bağlı olmaları gerekir.
%
% Lisans ile ilgili ayrıntılı bilgi almak için şu sayfaya başvurabilirsiniz:
% http://creativecommons.org/licenses/by-nc-sa/3.0/

\documentclass[dvipsnames]{beamer}

\mode<presentation>
{
  \usetheme{Rochester}
  \usecolortheme[named=Mahogany]{structure}
  \setbeamercovered{transparent}
}

\usepackage{ae}
\usepackage[T1]{fontenc}
\usepackage[utf8]{inputenc}
\usepackage[turkish]{babel}
\setbeamertemplate{navigation symbols}{}

\title{Bilişim Etiği}
\subtitle{Kuramlar}

\author{H. Turgut Uyar}
\date{2004-2012}

\AtBeginSubsection[]
{
  \begin{frame}<beamer>
    \frametitle{Konular}
    \tableofcontents[currentsection,currentsubsection]
  \end{frame}
}

%\beamerdefaultoverlayspecification{<+->}

\theoremstyle{definition}
\newtheorem{tanim}[theorem]{Tanım}

\theoremstyle{example}
\newtheorem{ornek}[theorem]{Örnek}

\theoremstyle{plain}

\pgfdeclareimage[width=2cm]{license}{../../license}

\pgfdeclareimage[height=4.5cm]{bentham}{bentham}
\pgfdeclareimage[height=4.5cm]{kant}{kant}
\pgfdeclareimage[height=4.5cm]{hobbes}{hobbes}
\pgfdeclareimage[height=4.5cm]{platon}{platon}

\begin{document}

\begin{frame}
  \titlepage
\end{frame}

\begin{frame}
  \frametitle{Lisans}

  \pgfuseimage{license}\hfill
  \copyright 2004-2012 H. Turgut Uyar

  \vfill
  \begin{tiny}
    You are free:
    \begin{itemize}
      \item to Share -- to copy, distribute and transmit the work
      \item to Remix -- to adapt the work
    \end{itemize}

    Under the following conditions:
    \begin{itemize}
      \item Attribution -- You must attribute the work in the manner specified by
        the author or licensor (but not in any way that suggests that they
        endorse you or your use of the work).

      \item Noncommercial -- You may not use this work for commercial purposes.

      \item Share Alike -- If you alter, transform, or build upon this work, you
        may distribute the resulting work only under the same or similar license
        to this one.
    \end{itemize}
  \end{tiny}

  \vfill
  Legal code (the full license):\\
  \url{http://creativecommons.org/licenses/by-nc-sa/3.0/}
\end{frame}

\begin{frame}
  \frametitle{Konular}
  \tableofcontents
\end{frame}

\section{Giriş}

\subsection{Tanımlar}

\begin{frame}
  \frametitle{Deskriptif Önermeler}

  \begin{tanim}
    \alert{deskriptif önerme}: durumun ne olduğunu söyler

    \begin{itemize}
      \item sosyoloji, psikoloji, antropoloji, politik bilimlerin konusu
    \end{itemize}
  \end{tanim}

  \pause
  \begin{ornek}
    \begin{quote}
      bilgisayar kullanıcılarının \%85'i yazılımları\\
      lisansına aykırı kullanıyor
    \end{quote}
  \end{ornek}
\end{frame}

\begin{frame}
  \frametitle{Normatif Önermeler}

  \begin{tanim}
    \alert{normatif önerme}: durumun ne olması gerektiğini söyler

    \begin{itemize}
      \item felsefenin konusu
    \end{itemize}
  \end{tanim}

  \pause
  \begin{ornek}
    \begin{quote}
      bilgisayar kullanıcıları yazılımları\\
      lisansına aykırı kullanmamalı
    \end{quote}
  \end{ornek}
\end{frame}

\begin{frame}
  \frametitle{Ahlak Sistemi}

  \begin{tanim}
    \alert{ahlak sistemi}:

    \begin{itemize}
      \item davranış kuralları: bireysel, toplumsal
      \item değerlendirme ilkeleri: toplumsal yarar, ...
    \end{itemize}
  \end{tanim}

  \pause
  \begin{itemize}
    \item{özellikleri
      \begin{itemize}
        \item \emph{açık}: herkes biliyor
        \item \emph{resmi değil}: zorlayıcı yetkili yok
        \item \emph{akılcı}: akıl yürüterek anlaşılabilir
        \item \emph{tarafsız}: bütün katılımcılar eşit
      \end{itemize}}
  \end{itemize}
\end{frame}

\begin{frame}
  \frametitle{Kuralların Oluşturulması}

  \begin{itemize}
    \item \alert{temel değerler} gözönüne alınarak:

    \begin{itemize}
      \item \emph{özünde iyi}: mutluluk, özerklik, mahremiyet, ...
      \item \emph{araç olarak iyi}: para, ...
    \end{itemize}

    \pause
    \item ahlak ilkelerinin dayandırıldığı \alert{zemin}den yola çıkılarak:

    \begin{itemize}
      \item din
      \item hukuk
      \item felsefe
    \end{itemize}
  \end{itemize}
\end{frame}

\subsection{Metodoloji}

\begin{frame}
  \frametitle{Felsefe Metodolojisi}

  \begin{tanim}
    \alert{diyalektik}: \emph{çelişmelerle ilerleyen}

    \begin{itemize}
      \item tez
      \item antitez
      \item sentez
    \end{itemize}
  \end{tanim}
\end{frame}

\begin{frame}
  \frametitle{Diyalektik Örneği}

  \begin{ornek}
    \alert{"ötanazi yanlıştır çünkü insan yaşamı kasıtlı sona erdirilmemelidir"}

    \pause
    \begin{itemize}
      \item bilinçli ve çok acı çekiyor
      \item bilinçsiz ve beyin hasarı var
      \item ölümcül hasta
      \item genç - yaşlı

      \pause
      \item \emph{yaşam kalitesi}
    \end{itemize}

    \pause
    aynı ilkenin uygulanabileceği durumlar: savaş, idam, kürtaj

    \begin{itemize}
      \item \emph{kendini savunma, başka yaşam kurtarma, ...}
    \end{itemize}
  \end{ornek}
\end{frame}

\subsection{Yanılgılar}

\begin{frame}
  \frametitle{Tartışma Yanılgıları}

  \begin{quote}
    felsefeciler ahlaki sorunların çözümünde anlaşamıyor
  \end{quote}

  \pause
  \begin{itemize}
    \item başka alanlarda da uzmanların anlaşamadığı konular var

    \pause
    \item bazı konularda anlaşma var

    \pause
    \item ilkelerde anlaşmazlık - durumlarda anlaşmazlık
  \end{itemize}
\end{frame}

\begin{frame}
  \frametitle{Kültürel Relativizm}

  \begin{quote}
    ahlaki normlar toplumdan topluma,\\
    hatta aynı toplumda zaman içinde değişir
  \end{quote}

  \pause
  \begin{itemize}
    \item deskriptif bir önerme, normatif olması için:
  \end{itemize}

  \begin{quote}
    neyin doğru neyin yanlış olduğuna\\
    o toplumda yaşayanların karar vermesi gerekir
  \end{quote}
\end{frame}

\begin{frame}
  \frametitle{Relativizmin Eleştirisi}

  \begin{ornek}
    \begin{itemize}
      \item A toplumu barışçıl bir toplum,\\
        ayrıca kültürel relativizme inanıyor

      \item B toplumu pek barışçıl değil, farklı kültürlerden hoşlanmıyor,\\
        A toplumunun yaşadığı yeri işgal ediyor

      \item A toplumu B toplumunun davranışının yanlış olduğunu
        söyleyemez
    \end{itemize}
  \end{ornek}
\end{frame}

\section{Etik Kuramlar}

\subsection{Faydacılık}

\begin{frame}
  \frametitle{Faydacılık}

  \begin{columns}
    \column{.45\textwidth}
    \begin{tanim}
      \alert{faydacılık}:\\
      bir davranış, etkilenenlerin çoğunluğunu mutlu edecekse ahlaken doğrudur
    \end{tanim}

    \begin{itemize}
      \item sonuca dayanıyor
    \end{itemize}

    \column{.55\textwidth}
    \begin{center}
      \pgfuseimage{bentham}
    \end{center}
    \begin{itemize}
      \item Jeremy Bentham (18-19.yy)\\
        Epikuros (İ.Ö. 3-4.yy)
    \end{itemize}
  \end{columns}
\end{frame}

\begin{frame}
  \frametitle{Faydacılık}

  \begin{columns}[t]
    \column{.5\textwidth}
    \begin{block}{eylem faydacılığı}
      daha çok sayıda kişinin\\
      daha mutlu olmasını\\
      sağlayacak şekilde davran
    \end{block}

    \pause
    \column{.5\textwidth}
    \begin{block}{kural faydacılığı}
      herkes böyle davransaydı\\
      daha çok sayıda insanın\\
      daha mutlu olmasını\\
      sağlayacak şekilde davran
    \end{block}
  \end{columns}
\end{frame}

\begin{frame}
  \frametitle{Faydacılık Sorun Örnekleri}

  \begin{ornek}[eylem faydacılığı]
    \begin{itemize}
      \item "bir kişiyi öldürüp organlarıyla 10 kişiyi kurtaralım"
      \item "toplumun \%1'ini köle olarak çalıştıralım"
    \end{itemize}
  \end{ornek}

  \pause
  \begin{ornek}[kural faydacılığı]
    \begin{itemize}
      \item "toplumun \%1'ini köle olarak çalıştırmak toplumda
        gerginlik yaratır"
    \end{itemize}
  \end{ornek}
\end{frame}

\begin{frame}
  \frametitle{Faydacılığın Eleştirisi}

  \begin{itemize}
    \item ahlak mutluluğa ya da zevke bağlanıyor

    \pause
    \item eylemi doğru ya da yanlış yapan şey eylemin dışında

    \begin{itemize}
      \item sonuç önceden bilinmiyor: \emph{ahlak şansı}
    \end{itemize}

    \pause
    \item karar vermeye yardımcı değil

    \begin{itemize}
      \item üretilecek mutlulukları kestirmek çok zor, çoğu zaman olanaksız
    \end{itemize}

    \pause
    \item iyi sonuçların adil dağılımıyla ilgilenmiyor
  \end{itemize}
\end{frame}

\subsection{Deontoloji}

\begin{frame}
  \frametitle{Deontoloji}

  \begin{columns}
    \column{.45\textwidth}
    \begin{itemize}
      \item mutluluk için içgüdü yeterlidir
      \item oysa insanı hayvandan ayıran akıl yürütme yeteneğidir
      \item bu yetenek insana ahlaki bir ödev yükler
    \end{itemize}

    \begin{itemize}
      \item ödeve dayanıyor
    \end{itemize}

    \column{.55\textwidth}
    \begin{center}
      \pgfuseimage{kant}
    \end{center}
    \begin{itemize}
      \item Immanuel Kant (18.yy)
    \end{itemize}
  \end{columns}
\end{frame}

\begin{frame}
  \frametitle{Kesin Buyruk}

  \begin{tanim}
    \alert{kesin buyruk}:

    \begin{itemize}
      \item kimseyi yalnızca bir amaca hizmet edecek araç olarak kullanma

      \item öyle bir kurala göre davran ki, bu kuralın evrensel bir kural haline
        gelmesini isteyebilesin
    \end{itemize}
  \end{tanim}
\end{frame}

\begin{frame}
  \frametitle{Kesin Buyruk Örnekleri}

  \begin{ornek}
    \alert{"kölelik yanlıştır çünkü"}

    \begin{itemize}
      \item "bir grup insan bir amaca hizmet için araç olarak kullanılmış olur"
      \item "tarafsız ve evrensel bir kural olarak uygulanmasını kimse istemez"
    \end{itemize}
  \end{ornek}
\end{frame}

\begin{frame}
  \frametitle{Deontolojinin Eleştirisi}

  \begin{itemize}
    \item çelişen ödevler varsa ne yapılacak?
  \end{itemize}

  \pause
  \begin{ornek}
    \begin{itemize}
      \item arkadaşınıza buluşmak için söz verdiniz
      \item buluşmaya giderken büyükannenizin hastaneye kaldırıldığını
        öğrendiniz
      \item arkadaşınıza haber veremiyorsunuz
    \end{itemize}
  \end{ornek}
\end{frame}

\subsection{Toplumsal Sözleşme}

\begin{frame}
  \frametitle{Toplumsal Sözleşme Kuramı}

  \begin{columns}
    \column{.55\textwidth}
    \begin{center}
      \pgfuseimage{hobbes}
    \end{center}
    \begin{itemize}
      \item Thomas Hobbes (17.yy)
    \end{itemize}

    \pause
    \column{.45\textwidth}
    \begin{itemize}
      \item \emph{ahlak öncesi devlet}: herkes kendi çıkarlarına göre davranıyor
      \item bireyler özgür görünüyor ancak sürekli başkalarından gelecek tehditle
        mücadele gerek
      \item özgürlüğümüzün bir kısmını egemen bir yetkiye devrederiz
    \end{itemize}

    \pause
    \begin{itemize}
      \item sözleşmeye dayanıyor
    \end{itemize}
  \end{columns}
\end{frame}

\begin{frame}
  \frametitle{Toplumsal Sözleşme Kuramının Eleştirisi}

  \begin{itemize}
    \item sözleşme yoksa ahlaki sorun da yok

    \begin{itemize}
      \item kimsenin kimseye yardım etmesi gerekmiyor
    \end{itemize}

    \pause
    \item yasaya aykırı olan ahlaka aykırı olmayabilir

    \begin{ornek}
      ırk ayrımına karşı sivil itaatsizlik
    \end{ornek}
  \end{itemize}
\end{frame}

\begin{frame}
  \frametitle{Haklar}

  \begin{itemize}
    \item \alert{negatif haklar}: engellenmemek
  \end{itemize}

  \begin{ornek}
    \begin{itemize}
      \item oy verme
      \item yüksek eğitim
    \end{itemize}
  \end{ornek}

  \pause
  \begin{itemize}
    \item \alert{pozitif haklar}: desteklenmek
  \end{itemize}

  \begin{ornek}
    \begin{itemize}
      \item temel eğitim
      \item sağlık?
    \end{itemize}
  \end{ornek}
\end{frame}

\subsection{Erdem Ahlakı}

\begin{frame}
  \frametitle{Erdem Ahlakı}

  \begin{columns}
    \column{.55\textwidth}
    \begin{center}
      \pgfuseimage{platon}
    \end{center}
    \begin{itemize}
      \item Platon (İ.Ö. 4.yy)\\
        Aristoteles
    \end{itemize}

    \pause
    \column{.45\textwidth}
    \begin{itemize}
      \item iyi kişilik özelliklerinin geliştirilmesi

      \item "şu durumda ne yapmalıyım?" değil\\
        "nasıl bir insan olmalıyım?"

      \item nezaket, doğru sözlülük, dürüstlük, güvenilirlik, yardımseverlik,
        cömertlik, adillik
    \end{itemize}

    \pause
    \begin{itemize}
      \item kişiliğe dayanıyor
    \end{itemize}
  \end{columns}
\end{frame}

\section*{Kaynaklar}

\begin{frame}
  \frametitle{Kaynaklar}

  \begin{block}{Okunacak: Tavani}
    \begin{itemize}
      \item Chapter 2: \alert{Ethical Concepts and Ethical Theories}
    \end{itemize}
  \end{block}
\end{frame}

\end{document}
