% Copyright (c) 2004-2015 H. Turgut Uyar <uyar@itu.edu.tr>
%
% This work is licensed under a "Creative Commons
% Attribution-NonCommercial-ShareAlike 4.0 International License".
% For more information, please visit:
% https://creativecommons.org/licenses/by-nc-sa/4.0/

\documentclass[dvipsnames]{beamer}

\usepackage{ae}
\usepackage[T1]{fontenc}
\usepackage[utf8]{inputenc}
\usepackage[turkish]{babel}
\setbeamertemplate{navigation symbols}{}
\setbeamersize{text margin left=2em, text margin right=2em}

\mode<presentation>
{
  \usetheme{Rochester}
  \usecolortheme[named=Mahogany]{structure}
  \setbeamercovered{transparent}
}

\title{Bilişim Etiği}
\subtitle{Profesyonel Etik}

\author{H. Turgut Uyar}
\date{2004-2015}

\AtBeginSubsection[]
{
  \begin{frame}<beamer>
    \frametitle{Konular}
    \tableofcontents[currentsection,currentsubsection]
  \end{frame}
}

%\beamerdefaultoverlayspecification{<+->}

\theoremstyle{plain}

\pgfdeclareimage[width=2cm]{license}{../license}

\pgfdeclareimage[width=7.4cm]{ibm-nazi}{ibm-nazi}
\pgfdeclareimage[height=5.3cm]{iran}{iran}
\pgfdeclareimage[height=4.5cm]{ariane}{ariane}
\pgfdeclareimage[width=7cm]{aegis}{aegis}
\pgfdeclareimage[width=6.5cm]{robot-sweden}{robot-sweden}
\pgfdeclareimage[width=7.5cm]{therac}{therac}
\pgfdeclareimage[height=4.5cm]{latimes}{latimes}
\pgfdeclareimage[height=4.5cm]{gta}{gta}
\pgfdeclareimage[width=6.5cm]{challenger}{challenger}
\pgfdeclareimage[width=6.5cm]{boisjoly}{boisjoly}

\begin{document}

\begin{frame}
  \titlepage
\end{frame}

\begin{frame}
  \frametitle{Lisans}

  \pgfuseimage{license}\hfill
  \copyright~2004-2015 H. Turgut Uyar

  \vfill
  \begin{footnotesize}
    You are free to:
    \begin{itemize}
      \itemsep0em
      \item Share -- copy and redistribute the material in any medium or format
      \item Adapt -- remix, transform, and build upon the material
    \end{itemize}

    Under the following terms:
    \begin{itemize}
      \itemsep0em
      \item Attribution -- You must give appropriate credit, provide a link to
        the license, and indicate if changes were made.

      \item NonCommercial -- You may not use the material for commercial
        purposes.

      \item ShareAlike -- If you remix, transform, or build upon the material,
        you must distribute your contributions under the same license as the
        original.
    \end{itemize}
  \end{footnotesize}

  \begin{small}
    For more information:\\
    \url{https://creativecommons.org/licenses/by-nc-sa/4.0/}

    \smallskip
    Read the full license:\\
    \url{https://creativecommons.org/licenses/by-nc-sa/4.0/legalcode}
  \end{small}
\end{frame}

\begin{frame}
  \frametitle{Konular}
  \tableofcontents
\end{frame}

\section{Meslek Ahlakı}

\subsection{Giriş}

\begin{frame}
  \frametitle{Meslek Ahlakı}

  \begin{itemize}
    \item profesyonel roller ahlaki değerlendirmelerde farklılık gerektirir mi?
    \item meslekler farklı ahlaki yükümlülükler getirir mi?

    \pause
    \bigskip
    \item farklılaşmış meslekler: özel yetkiler, özel sorumluluklar
    \item doktorluk, avukatlık, lisanslı mühendislik
    \item yasal koruma - toplumun güveni
  \end{itemize}
\end{frame}

\begin{frame}
  \frametitle{Yazılım Meslekleri}

  \begin{itemize}
    \item profesyonel etik tartışmalarında kimleri gözönüne alacağız?
    \item bilişim teknolojileri alanında çalışan herkes?
    \item yalnızca yazılım mühendisleri?

    \pause
    \medskip
    \item yazılım mühendisleri, kalite analistleri, teknik belgelemeciler,\\
      proje yöneticileri
    \item yazılım sistemlerinin analizinde, tasarımında, geliştirilmesinde,\\
      sertifikasyonunda, bakımında, sınanmasında çalışan herkes
  \end{itemize}
\end{frame}

\begin{frame}
  \frametitle{Farklılaşmış Meslek Karakteristikleri}

  \begin{itemize}
    \item toplumsal işlev görür

    \pause
    \medskip
    \item özel bilgi ve eğitim gerektirir
    \item pratisyen - araştırmacı farkı vardır

    \pause
    \medskip
    \item emir almaktan çok inisiyatif kullanır

    \pause
    \medskip
    \item mesleki örgütlenmeleri (meslek odaları) vardır
    \item çalışmak için lisans gerekir

    \pause
    \medskip
    \item etik kodları ve mesleki davranış kuralları vardır
  \end{itemize}
\end{frame}

\subsection{Etik Kodları}

\begin{frame}
  \frametitle{Etik Kodları}

  \begin{itemize}
    \item etik kodlarının amacı:

    \medskip
    \item esin verme
    \item yol gösterme
    \item eğitme
    \item disiplin altına alma

    \pause
    \bigskip
    \item bu mesleğin üyelerinden ne beklenebilir?
  \end{itemize}
\end{frame}

% \subsection{ACM/IEEE-CS}

\begin{frame}
  \frametitle{ACM/IEEE-CS Yazılım Mühendisliği Etik Kodu}

  \begin{itemize}
    \item \alert{kamu} çıkarlarına uygun davranmak
    \item \alert{müşteri ve işveren}inin çıkarlarına uygun davranmak
    \item \alert{ürün}ünün en yüksek mesleki standartlarda olmasını sağlamak
    \item mesleki \alert{değerlendirme}lerinde dürüst ve bağımsız olmak
    \item \alert{yönetici} olarak ahlaklı yaklaşımı benimsemek ve teşvik etmek
    \item \alert{meslek} saygınlığını artırmak
    \item \alert{meslektaş}larına adil davranmak ve desteklemek
    \item yaşam boyu öğrenme ilkesine sadık kalmak (\alert{kendi})
  \end{itemize}
\end{frame}

\begin{frame}
  \frametitle{Davranış Kuralı Örnekleri}

  \begin{itemize}
    \item yaptığının sorumluluğunu almak
    \item işverenin, müşterinin ve kullanıcıların çıkarlarını\\
      kamu yararıyla uzlaştırmak
    \item fiziksel engeller, kaynakların dağılımı gibi etkenleri\\
      gözönünde bulundurmak
    \item mesleğiyle ilgili kamu eğitimine katkıda bulunmak
    \item yazılım ve ilişkili belgelerin yeterince sınanmasını,\\
      hatalarının ayıklanmasını ve gözden geçirilmesini sağlamak
    \item etkilenecek olanların mahremiyetine saygı gösterecek şekilde\\
      yazılım geliştirmek
  \end{itemize}
\end{frame}

\begin{frame}
  \frametitle{Davranış Kuralı Örnekleri}

  \begin{itemize}
    \item çıkar çatışmalarını ilgili bütün taraflara açıkça belirtmek
    \item hiçbir yazılım mühendisinden bu kodla çelişecek\\
      bir istekte bulunmamak
    \item bu koda uygun davranmaya çalışan\\
      diğer yazılım mühendislerini desteklemek
    \item bu kodla çelişen örgüt ve işyerleriyle ilişki kurmamak
    \item meslektaşlarının kendilerini geliştirmelerine yardımcı olmak
    \item başkalarının yaptıkları işleri sahiplenmemek
  \end{itemize}
\end{frame}

\begin{frame}
  \frametitle{Kodun Yorumlanması}

  \begin{itemize}
    \item ``Bu kod doğru davranışı bulmak için verilmiş\\
      bir algoritma \alert{değildir}.''

    \pause
    \bigskip
    \item kimler etkilenir?
    \item diğer insanlara hak ettikleri saygı gösterilmiş olur mu?
    \item toplum yeterli bilgiye sahip olsa onaylar mı?
    \item en güçsüzler nasıl etkilenir?
    \item ideal bir yazılım mühendisine yakışır mı?
  \end{itemize}
\end{frame}

\section{Sorumluluk}

\subsection{Giriş}

\begin{frame}
  \frametitle{Sorumluluk}

  \begin{itemize}
    \item ahlaki sorumluluk
    \item yasal sorumluluk
    \smallskip
    \item hesap verebilirlik

    \pause
    \bigskip
    \item ahlaki sorumluluk için iki şart aranır:
    \smallskip
    \item yol açma
    \item kasıt
  \end{itemize}
\end{frame}

\begin{frame}
  \frametitle{Sorumluluk}

  \begin{itemize}
    \item bazı mühendisler ve araştırmacılar\\
      ahlaken yanlış buldukları projelerde çalışmayı reddeder

    \medskip
    \item silah sistemleri
    \item gözetleme sistemleri
  \end{itemize}
\end{frame}

\begin{frame}
  \frametitle{Örnek: IBM - Nazi Hükümeti}

  \begin{columns}
    \column{.6\textwidth}
    \pgfuseimage{ibm-nazi}

    \column{.4\textwidth}
    \begin{itemize}
      \item IBM Nazi hükümetine\\
        teknoloji satıyor
      \item gereksinimlerine göre\\
        özel uygulamalar\\
        geliştiriyor (1940'lar)
    \end{itemize}
  \end{columns}

  \medskip
  \tiny{\url{http://news.cnet.com/Selling-technology-to-the-Nazis/2010-1071_3-876539.html}}\\
\end{frame}

\begin{frame}
  \frametitle{Örnek: Siemens ve Nokia - İran}

  \begin{center}
    \pgfuseimage{iran}
  \end{center}

  \begin{itemize}
    \item İran'ın gözetleme sistemleri Siemens ve Nokia'dan (2009)
  \end{itemize}

  \medskip
  \tiny{\url{http://online.wsj.com/article/SB124562668777335653.html}}\\
\end{frame}

\begin{frame}
  \frametitle{Bilişim Sistemleri}

  \begin{itemize}
    \item bilişim sistemlerinin getirdiği ahlaki sorumluluklar var mı?

    \bigskip
    \item güvenlik açısından kritik uygulamalar
    \smallskip
    \item tıp
    \item hava trafiği, toplu taşımacılık
    \item güç santralleri
    \item silah sistemleri
  \end{itemize}
\end{frame}

\begin{frame}
  \frametitle{Örnek: Ariane 5}

  \begin{center}
    \pgfuseimage{ariane}
  \end{center}

  \begin{itemize}
    \item Fransız uydu fırlatma sisteminden atılan roket\\
      kendini imha ediyor: 500 milyon \$ zarar (1996)
    \item 64 bit kayan noktalı sayıyı 16 bit tamsayıya çevirirken taşma
  \end{itemize}

  \medskip
  \tiny{\url{http://www.ima.umn.edu/~arnold/disasters/ariane5rep.html}}\\
\end{frame}

\begin{frame}
  \frametitle{Örnek: Aegis radar sistemi}

  \begin{columns}
    \column{.6\textwidth}
    \pgfuseimage{aegis}

    \column{.4\textwidth}
    \begin{itemize}
      \item ABD uçak gemisi\\
        İran yolcu uçağını\\
        düşürüyor:\\
        290 kişi ölüyor (1988)
    \end{itemize}
  \end{columns}

  \medskip
  \tiny{\url{http://news.bbc.co.uk/onthisday/hi/dates/stories/july/3/newsid_4678000/4678707.stm}}\\
\end{frame}

\begin{frame}
  \frametitle{Örnek: İsveç fabrika robotu}

  \begin{columns}
    \column{.55\textwidth}
    \pgfuseimage{robot-sweden}

    \column{.45\textwidth}
    \begin{itemize}
      \item İsveç'te bir robot\\
        bir fabrika işçisini\\
        neredeyse öldürüyor\\
        (2009)
    \end{itemize}
  \end{columns}

  \medskip
  \tiny{\url{http://www.thelocal.se/19120/20090428/}}\\
\end{frame}

\begin{frame}
  \frametitle{Örnek: Therac-25 radyoterapi cihazı}

  \begin{columns}
    \column{.66\textwidth}
    \pgfuseimage{therac}

    \column{.34\textwidth}
    \begin{itemize}
      \item radyoterapi cihazı\\
        aşırı dozda\\
        radyasyon veriyor:\\
        3 hasta ölüyor,\\
        2 hasta ağır yaralanıyor (1985-87)

      \item çok sayıda hata:\\
        arayüz tasarımı,\\
        yarış koşulları
      \item donanım\\
        korumaları\\
        kaldırılmış
    \end{itemize}
  \end{columns}

  \medskip
  \tiny{\url{http://www.mendeley.com/research/investigation-therac25-accidents/}}\\
\end{frame}

\begin{frame}
  \frametitle{Örnek: CT tarama cihazı}

  \begin{center}
    \pgfuseimage{latimes}
  \end{center}

  \begin{itemize}
    \item Los Angeles'da bir hastanede CT tarama cihazı\\
      normalin 8 katı dozda radyasyon veriyor (2009)
  \end{itemize}

  \medskip
  \tiny{\url{http://articles.latimes.com/2009/oct/13/local/me-cedars13}}\\
\end{frame}

\begin{frame}
  \frametitle{Örnek: Grand Theft Auto}

  \begin{center}
    \pgfuseimage{gta}
  \end{center}

  \begin{itemize}
    \item ABD'de bir genç GTA oyununu taklit ediyor: 3 polis ölüyor (2003)
  \end{itemize}

  \medskip
  \tiny{\url{http://www.gamespot.com/news/grand-theft-auto-sparks-another-lawsuit-6118699}}\\
\end{frame}

\begin{frame}
  \frametitle{Örnek: Bilgisayar oyunları}

  \begin{itemize}
    \item GTA: "bütün Haitilileri öldür"
    \item pornografik eklentiler

    \bigskip
    \item bazı ülkeler şiddet içeren bilgisayar oyunlarını yasaklıyor:
    \item Tayland'da GTA (2008), Avustralya'da Manhunt (2004)
  \end{itemize}

  \medskip
  \tiny{\url{http://www.gamespot.com/news/haitian-americans-protest-vice-city-6084645}}\\
  \tiny{\url{http://www.gamespot.com/news/prostitutes-call-for-ban-on-gta-6144286}}\\
  \tiny{\url{http://www.theregister.co.uk/2009/09/02/take_two_settles_investor_class_action_lawsuit/}}\\
  \tiny{\url{http://www.reghardware.com/2008/08/04/gta_ban_thai/}}\\
  \tiny{\url{http://www.theregister.co.uk/2004/09/30/oz_manhunt_ban/}}\\
\end{frame}

\subsection{Sadakat}

\begin{frame}
  \frametitle{Sadakat}

  \begin{itemize}
    \item çalışanlar işverenlerine sadık olmalı

    \medskip
    \item işverenlerin çalışanlarına sadakati?
    \item sadakat ilişkisinin karşılıklı olması beklenebilir mi?

    \pause
    \bigskip
    \item etik kodundan:
    \smallskip
    \item işverenin çıkarlarına uygun davranmalı
    \item kamu çıkarıyla uzlaştırmalı
  \end{itemize}
\end{frame}

\begin{frame}
  \frametitle{Örnek: Challenger uzay mekiği}

  \begin{columns}
    \column{.55\textwidth}
    \pgfuseimage{challenger}

    \column{.45\textwidth}
    \begin{itemize}
      \item riski bilinmesine rağmen\\
        mekik fırlatılıyor:\\
        7 astronot ölüyor (1986)
    \end{itemize}
  \end{columns}

  \medskip
  \tiny{\url{http://news.bbc.co.uk/onthisday/hi/dates/stories/january/28/newsid_2506000/2506161.stm}}\\
\end{frame}

\begin{frame}
  \frametitle{Örnek: Challenger uzay mekiği}

  \begin{columns}
    \column{.55\textwidth}
    \pgfuseimage{boisjoly}

    \column{.45\textwidth}
    \begin{itemize}
      \item Roger Boisjoly\\
        fırlatmayı durdurmaya\\
        çalışıyor
      \item olay sonrasındaki\\
        soruşturmalarda\\
        baskıya rağmen\\
        gerçekleri söylüyor
    \end{itemize}
  \end{columns}

  \medskip
  \tiny{\url{http://www.latimes.com/news/obituaries/la-me-roger-boisjoly-20120207,0,2248999.story}}\\
\end{frame}

\begin{frame}
  \frametitle{Örnek: Bay Area Rapid Transit}

  \begin{itemize}
    \item bilgisayar denetimli toplu taşıma sistemi
    \item süresini ve bütçesini çok aşmış
    \item mühendisler sistemin risklerini basına duyuruyor:\\
      işten atılıyorlar (1970'ler)

    \medskip
    \item bu tip davranışı özendirici yasal korumalar geliyor
  \end{itemize}
\end{frame}

\begin{frame}
  \frametitle{Kamuya Duyurma}

  \begin{itemize}
    \item ne zaman izin var?
    \smallskip
    \item ciddi hasar yaratacaksa
    \item kurum içinde bütün yollar denendiyse

    \pause
    \bigskip
    \item ne zaman ahlaki yükümlülük?
    \smallskip
    \item kanıt varsa
    \item ortaya çıkması zararı engelleyecekse
  \end{itemize}
\end{frame}

\begin{frame}
  \frametitle{Kolektif Sorumluluk}

  \begin{itemize}
    \item yazılım mühendisleri çoğu zaman projenin bütününü görmüyor
    \item ahlaki sorumluluk her zaman bireysel mi?

    \pause
    \medskip
    \item bireyler düzeyinde değil meslek düzeyinde düşünce ve davranış
  \end{itemize}
\end{frame}

\section*{Kaynaklar}

\begin{frame}
  \frametitle{Kaynaklar}

  \begin{block}{Okunacak: Tavani}
    \begin{itemize}
      \item Chapter 4: \alert{Professional Ethics}
    \end{itemize}
  \end{block}
\end{frame}

\end{document}
