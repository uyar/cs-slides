% Copyright (c) 2004-2012 H. Turgut Uyar <uyar@itu.edu.tr>
%
% Bu notlar "Creative Commons Attribution-NonCommercial-ShareAlike License" ile
% lisanslanmıştır. Yazarının açıkça belirtilmesi koşuluyla ve ticari olmayan
% amaçlarla kullanılabilir ve dağıtılabilir. Bu notlardan yola çıkılarak
% oluşturulacak çalışmaların da aynı lisansa bağlı olmaları gerekir.
%
% Lisans ile ilgili ayrıntılı bilgi almak için şu sayfaya başvurabilirsiniz:
% http://creativecommons.org/licenses/by-nc-sa/3.0/

\documentclass[dvipsnames]{beamer}

\mode<presentation>
{
  \usetheme{Rochester}
  \usecolortheme[named=Mahogany]{structure}
  \setbeamercovered{transparent}
}

\usepackage{ae}
\usepackage[T1]{fontenc}
\usepackage[turkish]{babel}
\usepackage[utf8]{inputenc}
\setbeamertemplate{navigation symbols}{}

\title{Bilişim Etiği}
\subtitle{Profesyonel Etik}

\author{H. Turgut Uyar}
\date{2004-2012}

\AtBeginSubsection[]
{
  \begin{frame}<beamer>
    \frametitle{Konular}
    \tableofcontents[currentsection,currentsubsection]
  \end{frame}
}

%\beamerdefaultoverlayspecification{<+->}

\theoremstyle{definition}
\newtheorem{tanim}[theorem]{Tanım}

\theoremstyle{example}
\newtheorem{ornek}[theorem]{Örnek}

\theoremstyle{plain}

\pgfdeclareimage[width=2cm]{license}{../../license}

\pgfdeclareimage[width=7.4cm]{ibm-nazi}{ibm-nazi}
\pgfdeclareimage[height=5.3cm]{iran}{iran}
\pgfdeclareimage[height=4.5cm]{ariane}{ariane}
\pgfdeclareimage[width=7cm]{aegis}{aegis}
\pgfdeclareimage[width=6.5cm]{robot-sweden}{robot-sweden}
\pgfdeclareimage[width=7.5cm]{therac}{therac}
\pgfdeclareimage[height=4.5cm]{latimes}{latimes}
\pgfdeclareimage[height=4.5cm]{gta}{gta}
\pgfdeclareimage[width=6.5cm]{challenger}{challenger}
\pgfdeclareimage[width=6.5cm]{boisjoly}{boisjoly}

\begin{document}

\begin{frame}
  \titlepage
\end{frame}

\begin{frame}
  \frametitle{Lisans}

  \pgfuseimage{license}\hfill
  \copyright 2004-2012 H. Turgut Uyar

  \vfill
  \begin{tiny}
    You are free:
    \begin{itemize}
      \item to Share — to copy, distribute and transmit the work
      \item to Remix — to adapt the work
    \end{itemize}

    Under the following conditions:
    \begin{itemize}
      \item Attribution — You must attribute the work in the manner specified by
        the author or licensor (but not in any way that suggests that they
        endorse you or your use of the work).

      \item Noncommercial — You may not use this work for commercial purposes.

      \item Share Alike — If you alter, transform, or build upon this work, you
        may distribute the resulting work only under the same or similar license
        to this one.
    \end{itemize}
  \end{tiny}

  \vfill
  Legal code (the full license):\\
  \url{http://creativecommons.org/licenses/by-nc-sa/3.0/}
\end{frame}

\begin{frame}
  \frametitle{Konular}
  \tableofcontents
\end{frame}

\section{Mesleki Davranış}

\subsection{Meslek Ahlakı}

\begin{frame}
  \frametitle{Bilgisayarcı Kimdir?}

  \begin{itemize}
    \item hangi işleri yapanlara bilgisayarcı denecek?
    \begin{itemize}
      \item bilişim teknolojileri alanında çalışan herkes?
      \item yalnızca yazılım mühendisleri?
    \end{itemize}

    \pause
    \medskip
    \item profesyonel etik tartışmalarında gözönüne alacaklarımız:\\
      yazılım mühendisleri, kalite analistleri, teknik belgelemeciler,\\
      proje yöneticileri
    \begin{itemize}
      \item bir yazılımın analizinde, belirtimlerinde, tasarımında,\\
        geliştirilmesinde, sertifikasyonunda, bakımında ve sınanmasında\\
        çalışan herkes
    \end{itemize}
  \end{itemize}
\end{frame}

\begin{frame}
  \frametitle{Meslek Ahlakı}

  \begin{itemize}
    \item sıradan insanlar için geçerli olan etik kuralları\\
      meslek sahipleri için farklılık gösterir mi?

    \item meslek sahiplerinin sıradan insanlara göre\\
      daha fazla etik sorumlulukları olabilir mi?
  \end{itemize}
\end{frame}

\begin{frame}
  \frametitle{Farklılaşmış Meslekler}

  \begin{itemize}
    \item bazı meslekler ahlaki değerlendirme açısından\\
      diğerlerinden farklı özellikler gösterir
  \end{itemize}

  \begin{ornek}
    doktorluk, avukatlık
  \end{ornek}

  \pause
  \begin{columns}
    \column{.5\textwidth}
    \begin{itemize}
      \item özel yetkiler
      \item yasal koruma
    \end{itemize}

    \column{.5\textwidth}
    \begin{itemize}
      \item toplumun güvenini kazanma
    \end{itemize}
  \end{columns}

  \pause
  \bigskip
  \begin{itemize}
    \item yazılım mühendisliği farklılaşmış bir meslek mi?
  \end{itemize}
\end{frame}

\begin{frame}
  \frametitle{Farklılaşmış Meslek Karakteristikleri}

  \begin{itemize}
    \item toplumsal işlev görürler

    \pause
    \item özel bilgi ve eğitim gerektirirler
    \begin{itemize}
      \item pratisyen - araştırmacı farkı vardır
    \end{itemize}

    \pause
    \item emir almaktan çok inisiyatif kullanırlar

    \pause
    \item mesleki örgütlenmeleri (meslek odaları) vardır
    \begin{itemize}
      \item çalışmak için lisans gerekir
    \end{itemize}

    \pause
    \item etik kodları ve mesleki davranış kuralları vardır
  \end{itemize}
\end{frame}

\subsection{Etik Kodları}

\begin{frame}
  \frametitle{Etik Kodları}

  \begin{itemize}
    \item etik kodlarının amacı:

    \begin{itemize}
      \item esin verme
      \item yol gösterme
      \item eğitme
      \item disiplin altına alma
    \end{itemize}

    \pause
    \item bu mesleğin üyelerinden ne beklenebilir?
  \end{itemize}
\end{frame}

% \subsection{ACM/IEEE-CS}

\begin{frame}
  \frametitle{ACM/IEEE-CS Yazılım Mühendisliği Etik Kodları}

  \begin{itemize}
    \item kamu çıkarlarına uygun davranmak
    \item müşterisinin ve işvereninin çıkarlarını gözetmek
    \item ürününün en yüksek mesleki standartlarda olmasını sağlamak
    \item mesleki değerlendirmelerinde dürüst ve bağımsız olmak
    \item yönetici olarak ahlaklı davranmak\\
      ve bu tür davranışları teşvik etmek
    \item mesleğinin dürüstlüğünü ve saygınlığını ilerletmek
    \item meslektaşlarına adil davranmak ve desteklemek
    \item yaşam boyu öğrenme ilkesine\\
      ve ahlaki davranışlara sadık kalmak
  \end{itemize}
\end{frame}

\begin{frame}
  \frametitle{Davranış Kuralı Örnekleri}

  \begin{itemize}
    \item yaptığının sorumluluğunu almak
    \item işveren, müşteri ve kullanıcı çıkarlarını\\
      kamu yararıyla uzlaştırmak
    \item fiziksel engeller, kaynakların dağılımı gibi etkenleri\\
      gözönünde bulundurmak
    \item mesleğiyle ilgili kamu eğitimine katkıda bulunmak
    \item yazılım ve ilişkili belgelerin yeterince sınanmasını,\\
      hatalarının ayıklanmasını ve gözden geçirilmesini sağlamak
    \item etkilenecek olanların mahremiyetine saygı gösterecek şekilde\\
      yazılım geliştirmek
  \end{itemize}
\end{frame}

\begin{frame}
  \frametitle{Davranış Kuralı Örnekleri}

  \begin{itemize}
    \item çıkar çatışmalarını ilgili bütün taraflara açıkça belirtmek
    \item hiçbir yazılım mühendisinden bu kodla çelişecek\\
      bir istekte bulunmamak
    \item bu koda uygun davranmaya çalışan\\
      diğer yazılım mühendislerini desteklemek
    \item bu kodla çelişen örgüt ve işyerleriyle ilişki kurmamak
    \item meslektaşlarının kendilerini geliştirmelerine yardımcı olmak
    \item başkalarının yaptıkları işleri sahiplenmemek
  \end{itemize}
\end{frame}

\begin{frame}
  \frametitle{Kodun Yorumlanması}

  \begin{itemize}
    \item bu kod doğru davranışı bulmak için verilmiş\\
      bir algoritma \alert{DEĞİLDİR}

    \pause
    \item karar verirken şunları gözönüne alın:

    \begin{itemize}
      \item kimler etkilenir?
      \item herkese gereken saygı gösterilmiş olur mu?
      \item toplum yeterli bilgiye sahip olsa onaylar mı?
      \item en güçsüzler nasıl etkilenir?
      \item ideal bir yazılım mühendisine yakışır mı?
    \end{itemize}
  \end{itemize}
\end{frame}

\section{Sorumluluk}

\subsection{Giriş}

\begin{frame}
  \frametitle{Sorumluluk}

  \begin{itemize}
    \item ahlaki sorumluluk
    \item yasal sorumluluk

    \medskip
    \item hesap verebilirlik
  \end{itemize}

  \pause
  \begin{tanim}
    ahlaki sorumluluk için iki şart aranır:

    \begin{enumerate}
      \item yol açma
      \item kasıt
    \end{enumerate}
  \end{tanim}
\end{frame}

\begin{frame}
  \frametitle{Sorumluluk}

  \begin{itemize}
    \item bazı mühendisler ve akademisyenler\\
      ahlaken yanlış buldukları projelerde çalışmayı reddeder
    \begin{itemize}
      \item silah sistemleri
      \item gözetleme sistemleri
    \end{itemize}
  \end{itemize}
\end{frame}

\begin{frame}
  \frametitle{Örnek Olay: IBM - Nazi Hükümeti (1940'lar)}

  \begin{columns}
    \column{.6\textwidth}
    \pgfuseimage{ibm-nazi}

    \column{.4\textwidth}
    \begin{itemize}
      \item IBM Nazi hükümetine\\
        teknoloji satıyor
      \item gereksinimlerine göre\\
        özel uygulamalar\\
        geliştiriyor
    \end{itemize}
  \end{columns}

  \medskip
  \tiny{\url{http://news.cnet.com/Selling-technology-to-the-Nazis/2010-1071_3-876539.html}}
\end{frame}

\begin{frame}
  \frametitle{Örnek Olay: Siemens - Nokia - İran (2009)}

  \begin{center}
    \pgfuseimage{iran}
  \end{center}

  \begin{itemize}
    \item İran'ın gözetleme teknolojileri Siemens ve Nokia'dan
  \end{itemize}

  \medskip
  \tiny{\url{http://online.wsj.com/article/SB124562668777335653.html}}
\end{frame}

\subsection{Bilişimcilerin Sorumluluğu}

\begin{frame}
  \frametitle{Mesleki Hatalar}

  \begin{itemize}
    \item güvenlik açısından kritik uygulamalar
    \begin{itemize}
      \item tıp
      \item hava trafiği
      \item toplu taşımacılık
      \item güç santralleri
      \item silah sistemleri
    \end{itemize}
  \end{itemize}
\end{frame}

\begin{frame}
  \frametitle{Örnek Olay: Ariane 5 (1996)}

  \begin{center}
    \pgfuseimage{ariane}
  \end{center}

  \begin{itemize}
    \item Fransız uydu fırlatma sisteminden atılan roket\\
      kendini imha ediyor: 500 milyon \$ zarar

    \item 64 bit kayan noktalı sayıyı 16 bit tamsayıya çevirirken taşma
  \end{itemize}

  \medskip
  \tiny{\url{http://www.ima.umn.edu/~arnold/disasters/ariane5rep.html}}
\end{frame}

\begin{frame}
  \frametitle{Örnek Olay: Aegis radar sistemi (1988)}

  \begin{columns}
    \column{.65\textwidth}
    \pgfuseimage{aegis}

    \column{.35\textwidth}
    \begin{itemize}
      \item ABD uçak gemisi\\
        İran yolcu uçağını\\
        düşürüyor:\\
        290 kişi ölüyor
    \end{itemize}
  \end{columns}

  \medskip
  \tiny{\url{http://news.bbc.co.uk/onthisday/hi/dates/stories/july/3/newsid_4678000/4678707.stm}}
\end{frame}

\begin{frame}
  \frametitle{Örnek Olay: İsveç fabrika robotu (2009)}

  \begin{columns}
    \column{.6\textwidth}
    \pgfuseimage{robot-sweden}

    \column{.4\textwidth}
    \begin{itemize}
      \item İsveç'te bir robot\\
        bir fabrika işçisini\\
        ağır yaralıyor
    \end{itemize}
  \end{columns}

  \medskip
  \tiny{\url{http://www.thelocal.se/19120/20090428/}}
\end{frame}

\begin{frame}
  \frametitle{Örnek Olay: Therac-25 radyoterapi cihazı (1985-87)}

  \begin{columns}
    \column{.66\textwidth}
    \pgfuseimage{therac}

    \column{.34\textwidth}
    \begin{itemize}
      \item radyoterapi cihazı\\
        aşırı dozda\\
        radyasyon veriyor:\\
        3 hasta ölüyor,\\
        2 hasta ağır yaralanıyor

      \item çok sayıda hata:\\
        arayüz tasarımı,\\
        yarış koşulları
      \item donanım\\
        korumaları yok
    \end{itemize}
  \end{columns}

  \medskip
  \tiny{\url{http://www.mendeley.com/research/investigation-therac25-accidents/}}
\end{frame}

\begin{frame}
  \frametitle{Örnek Olay: CT tarama cihazı (2009)}

  \begin{center}
    \pgfuseimage{latimes}
  \end{center}

  \begin{itemize}
    \item Los Angeles'da bir hastanede CT tarama cihazı\\
      normalin 8 katı dozda radyasyon veriyor
  \end{itemize}

  \medskip
  \tiny{\url{http://articles.latimes.com/2009/oct/13/local/me-cedars13}}
\end{frame}

\begin{frame}
  \frametitle{Örnek Olay: Grand Theft Auto (2003)}

  \begin{center}
    \pgfuseimage{gta}
  \end{center}

  \begin{itemize}
    \item ABD'de bir genç GTA oyununu taklit ediyor: 3 polis ölüyor
  \end{itemize}

  \medskip
  \tiny{\url{http://www.gamespot.com/news/grand-theft-auto-sparks-another-lawsuit-6118699}}
\end{frame}

\begin{frame}
  \frametitle{Örnek Olay: Bilgisayar oyunları}

  \begin{itemize}
    \item GTA: "bütün Haitilileri öldür"
    \item pornografik eklentiler

    \bigskip
    \item bazı ülkeler şiddet içeren bilgisayar oyunlarını yasaklıyor:
    \begin{itemize}
      \item Tayland'da GTA (2008), Avustralya'da Manhunt (2004)
    \end{itemize}
  \end{itemize}

  \medskip
  \tiny{\url{http://www.gamespot.com/news/haitian-americans-protest-vice-city-6084645}}\\
  \tiny{\url{http://www.gamespot.com/news/prostitutes-call-for-ban-on-gta-6144286}}\\
  \tiny{\url{http://www.theregister.co.uk/2009/09/02/take_two_settles_investor_class_action_lawsuit/}}\\
  \tiny{\url{http://www.reghardware.com/2008/08/04/gta_ban_thai/}}\\
  \tiny{\url{http://www.theregister.co.uk/2004/09/30/oz_manhunt_ban/}}\\
\end{frame}

\subsection{Sadakat}

\begin{frame}
  \frametitle{Sadakat}

  \begin{itemize}
    \item çalışanlar işverene sadık olmalı

    \medskip
    \item işverenin çalışanlara sadakati?
    \begin{itemize}
      \item sadakat ilişkisinin karşılıklı olması beklenebilir mi?
    \end{itemize}

    \pause
    \medskip
    \item etik kodundan:
    \begin{itemize}
      \item işverenin çıkarlarına aykırı davranmamalı
      \item daha önemli ahlaki ilke çiğnenmiyorsa
    \end{itemize}
  \end{itemize}
\end{frame}

\begin{frame}
  \frametitle{Örnek Olay: Challenger uzay mekiği (1986)}

  \begin{columns}
    \column{.57\textwidth}
    \pgfuseimage{challenger}

    \column{.43\textwidth}
    \begin{itemize}
      \item riski bilinmesine rağmen\\
        mekik fırlatılıyor:\\
        7 astronot ölüyor
    \end{itemize}
  \end{columns}

  \medskip
  \tiny{\url{http://news.bbc.co.uk/onthisday/hi/dates/stories/january/28/newsid_2506000/2506161.stm}}
\end{frame}

\begin{frame}
  \frametitle{Örnek Olay: Challenger uzay mekiği (1986)}

  \begin{columns}
    \column{.57\textwidth}
    \pgfuseimage{boisjoly}

    \column{.43\textwidth}
    \begin{itemize}
      \item Roger Boisjoly fırlatmayı\\
        durdurmaya\\
        çalışıyor
      \item olay sonrasındaki\\
        soruşturmalarda\\
        baskıya rağmen\\
        gerçekleri söylüyor
    \end{itemize}
  \end{columns}

  \medskip
  \tiny{\url{http://www.latimes.com/news/obituaries/la-me-roger-boisjoly-20120207,0,2248999.story}}
\end{frame}

\begin{frame}
  \frametitle{Örnek Olay: Bay Area Rapid Transit (1970'ler)}

  \begin{itemize}
    \item bilgisayar denetimli toplu taşıma sistemi
    \item mühendisler sistemin risklerini basına duyuruyor:\\
      işten atılıyorlar

    \medskip
    \item bu tip davranışı özendirici yasal korumalar geliyor
  \end{itemize}
\end{frame}

\begin{frame}
  \frametitle{Kamuya Duyurma}

  \begin{itemize}
    \item ne zaman izin var?
    \begin{itemize}
      \item ürün ciddi hasar yaratacaksa
      \item durum üstlere bildirildiyse
      \item kurum içinde bütün yollar denendiyse
    \end{itemize}

    \pause
    \medskip
    \item ne zaman ahlaki yükümlülük?
    \begin{itemize}
      \item kanıt varsa
      \item ortaya çıkması zararı engelleyecekse
    \end{itemize}
  \end{itemize}
\end{frame}

\begin{frame}
  \frametitle{Kolektif Sorumluluk}

  \begin{itemize}
    \item yazılım mühendisleri çoğu zaman projenin bütününü görmüyor

    \pause
    \item ahlaki sorumluluk her zaman bireysel midir?

    \pause
    \item mühendisler etik sorunların çözümü için bireyler
      düzeyinde değil meslek düzeyinde düşünmeli ve
      davranmalı
  \end{itemize}
\end{frame}

\section*{Kaynaklar}

\begin{frame}
  \frametitle{Kaynaklar}

  \begin{block}{Okunacak: Tavani}
    \begin{itemize}
      \item Chapter 4: \alert{Professional Ethics}
    \end{itemize}
  \end{block}
\end{frame}

\end{document}
