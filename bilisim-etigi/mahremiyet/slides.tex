% Copyright (c) 2004-2012 H. Turgut Uyar <uyar@itu.edu.tr>
%
% Bu notlar "Creative Commons Attribution-NonCommercial-ShareAlike License" ile
% lisanslanmıştır. Yazarının açıkça belirtilmesi koşuluyla ve ticari olmayan
% amaçlarla kullanılabilir ve dağıtılabilir. Bu notlardan yola çıkılarak
% oluşturulacak çalışmaların da aynı lisansa bağlı olmaları gerekir.
%
% Lisans ile ilgili ayrıntılı bilgi almak için şu sayfaya başvurabilirsiniz:
% http://creativecommons.org/licenses/by-nc-sa/3.0/

\documentclass[dvipsnames]{beamer}

\mode<presentation>
{
  \usetheme{Rochester}
  \usecolortheme[named=Mahogany]{structure}
  \setbeamercovered{transparent}
}

\usepackage{ae}
\usepackage[T1]{fontenc}
\usepackage[turkish]{babel}
\usepackage[utf8]{inputenc}
\setbeamertemplate{navigation symbols}{}

\title{Bilişim Etiği}
\subtitle{Mahremiyet}

\author{H. Turgut Uyar}
\date{2004-2012}

\AtBeginSubsection[]
{
  \begin{frame}<beamer>
    \frametitle{Konular}
    \tableofcontents[currentsection,currentsubsection]
  \end{frame}
}

%\beamerdefaultoverlayspecification{<+->}

\theoremstyle{definition}
\newtheorem{tanim}[theorem]{Tanım}

\theoremstyle{example}
\newtheorem{ornek}[theorem]{Örnek}

\theoremstyle{plain}

\pgfdeclareimage[width=2cm]{license}{../../license}

\pgfdeclareimage[width=5.8cm]{zuckerberg}{zuckerberg}
\pgfdeclareimage[width=6.5cm]{pennsylvania}{pennsylvania}
\pgfdeclareimage[width=5.8cm]{iphone}{iphone}
\pgfdeclareimage[width=7.5cm]{carrieriq}{carrieriq}
\pgfdeclareimage[width=7.5cm]{rfid-immigrants}{rfid-immigrants}
\pgfdeclareimage[width=7.2cm]{rfid-passport}{rfid-passport}
\pgfdeclareimage[width=7cm]{schneier}{schneier}
\pgfdeclareimage[width=7cm]{schmidt}{schmidt}
\pgfdeclareimage[height=7cm]{street-view}{street-view}
\pgfdeclareimage[width=6.5cm]{face-recognition}{face-recognition}
\pgfdeclareimage[width=7.2cm]{burglars}{burglars}
\pgfdeclareimage[width=7.2cm]{oz}{oz}
\pgfdeclareimage[width=7.5cm]{aol}{aol}
\pgfdeclareimage[width=6.5cm]{italy}{italy}
\pgfdeclareimage[height=6.5cm]{brown}{brown}
\pgfdeclareimage[width=6.5cm]{york}{york}
\pgfdeclareimage[width=6.5cm]{btk}{btk}
\pgfdeclareimage[height=6.5cm]{bork}{bork}
\pgfdeclareimage[width=6.4cm]{schaeffer}{schaeffer}
\pgfdeclareimage[width=6.5cm]{toysmart}{toysmart}
\pgfdeclareimage[width=6.5cm]{myspace}{myspace}
\pgfdeclareimage[width=6.5cm]{politico}{politico}
\pgfdeclareimage[width=6.5cm]{google-doj}{google-doj}
\pgfdeclareimage[width=6.2cm]{abacus}{abacus}
\pgfdeclareimage[height=5cm]{dna}{dna}
\pgfdeclareimage[width=6.2cm]{reddit}{reddit}
\pgfdeclareimage[width=6cm]{flights}{flights}
\pgfdeclareimage[width=6cm]{facebook-evolution}{facebook-evolution}
\pgfdeclareimage[width=6.5cm]{survey}{survey}
\pgfdeclareimage[height=6.5cm]{chrome-history}{chrome-history}

\begin{document}

\begin{frame}
  \titlepage
\end{frame}

\begin{frame}
  \frametitle{Lisans}

  \pgfuseimage{license}\hfill
  \copyright 2004-2012 H. Turgut Uyar

  \vfill
  \begin{tiny}
    You are free:
    \begin{itemize}
      \item to Share — to copy, distribute and transmit the work
      \item to Remix — to adapt the work
    \end{itemize}

    Under the following conditions:
    \begin{itemize}
      \item Attribution — You must attribute the work in the manner specified by
        the author or licensor (but not in any way that suggests that they
        endorse you or your use of the work).

      \item Noncommercial — You may not use this work for commercial purposes.

      \item Share Alike — If you alter, transform, or build upon this work, you
        may distribute the resulting work only under the same or similar license
        to this one.
    \end{itemize}
  \end{tiny}

  \vfill
  Legal code (the full license):\\
  \url{http://creativecommons.org/licenses/by-nc-sa/3.0/}
\end{frame}

\begin{frame}
  \frametitle{Konular}
  \tableofcontents
\end{frame}

\section{Mahremiyet}

\subsection{Giriş}

\begin{frame}
  \frametitle{Mahremiyet}

  \begin{itemize}
    \item erişim mahremiyeti
    \begin{itemize}
      \item rahat bırakılma
    \end{itemize}

    \pause
    \item karar verme mahremiyeti
    \begin{itemize}
      \item kişisel kararlara karışılmaması
    \end{itemize}

    \pause
    \item bilgi mahremiyeti
    \begin{itemize}
      \item kişisel verileri üzerinde denetim sahibi olma
    \end{itemize}
  \end{itemize}
\end{frame}

\begin{frame}
  \frametitle{Mahremiyet Değeri}

  \begin{columns}
    \column{.46\textwidth}
    \pgfuseimage{zuckerberg}

    \column{.54\textwidth}
    \begin{itemize}
      \item Eric Schmidt (Google - 2009):
      \begin{quote}
        Başkalarının bilmesini istemeyeceğiniz bir şey yapıyorsanız,\\
        belki de zaten yapmamanız gerekiyordur.
      \end{quote}

      \pause
      \item Mark Zuckerberg\\
        (Facebook - 2010):
      \begin{quote}
        Mahremiyet artık bir toplumsal norm değil.
      \end{quote}
    \end{itemize}
  \end{columns}

  \medskip
  \tiny{\url{http://www.telegraph.co.uk/technology/facebook/6966628/Facebooks-Mark-Zuckerberg-says-privacy-is-no-longer-a-social-norm.html}}\\
  \tiny{\url{http://www.theregister.co.uk/2007/12/03/zuckerberg_files/}}\\
\end{frame}

\begin{frame}
  \frametitle{Mahremiyet Değeri}

  \begin{itemize}
    \item mahremiyet evrensel bir değer değil

    \pause
    \medskip
    \item mahremiyet özünde iyi bir değer mi?
    \begin{itemize}
      \item mahremiyet olmadan güvenlik, özerklik mümkün değil
      \item farklı türden ilişkiler kurabilme
    \end{itemize}
  \end{itemize}
\end{frame}

\subsection{Gözetleme}

\begin{frame}
  \frametitle{Gözetleme}

  \begin{itemize}
    \item kişilerin izlenmesini sağlayan teknolojiler arttı
  \end{itemize}

  \pause
  \begin{ornek}
    \begin{itemize}
      \item kredi kartları
      \item cep telefonları
      \item video kameralar
      \item yüz tanıma, plaka tanıma
      \item \ldots
    \end{itemize}
  \end{ornek}
\end{frame}

\begin{frame}
  \frametitle{Örnek: A.B.D. öğrenci bilgisayar kameraları}

  \begin{columns}
    \column{.52\textwidth}
    \pgfuseimage{pennsylvania}

    \column{.48\textwidth}
    \begin{itemize}
      \item bir okul,\\
        öğrencilerine dağıttığı\\
        bilgisayarlardaki kameraları\\
        uzaktan etkinleştirerek\\
        öğrencileri gözetliyor (2010)
    \end{itemize}
  \end{columns}

  \medskip
  \tiny{\url{https://www.computerworld.com/s/article/9190722/Update_School_district_settles_webcam_spying_suit_for_610_000_}}\\
\end{frame}

\begin{frame}
  \frametitle{Örnek: Cep telefonlarında konum takibi}

  \begin{columns}
    \column{.5\textwidth}
    \pgfuseimage{iphone}

    \column{.5\textwidth}
    \begin{itemize}
      \item Apple aygıtları\\
        bulundukları konumu\\
        bir siteye gizlice bildiriyor (2011)
      \item aynı durum Google Android\\
        ve Microsoft Windows\\
        telefonlarda da ortaya çıkıyor (2011)
    \end{itemize}
  \end{columns}

  \medskip
  \tiny{\url{http://edition.cnn.com/2011/TECH/mobile/04/20/iphone.tracking/index.html?hpt=T1}}\\
  \tiny{\url{http://www.theregister.co.uk/2011/04/22/apple_iphone_location_tracking_analysis/}}\\
  \tiny{\url{http://news.cnet.com/8301-31921_3-20057329-281.html}}\\
\end{frame}

\begin{frame}
  \frametitle{Örnek: Cep telefonlarında tuş takibi}

  \begin{columns}
    \column{.65\textwidth}
    \pgfuseimage{carrieriq}

    \column{.45\textwidth}
    \begin{itemize}
      \item Carrier IQ yazılımı,\\
        kullanıcının bastığı tuşları\\
        kaydediyor (2011)
    \end{itemize}
  \end{columns}

  \medskip
  \tiny{\url{http://www.theregister.co.uk/2011/11/30/smartphone_spying_app/}}\\
\end{frame}

\begin{frame}
  \frametitle{Gözetleme Teknolojileri: RFID}

  \begin{columns}
    \column{.3\textwidth}
    \begin{itemize}
      \item envanter takibi
      \item hayvancılık

      \pause
      \medskip
      \item evcil hayvanlar,\\
        çocuklar
      \item göçmenler,\\
        mahkumlar
    \end{itemize}

    \column{.7\textwidth}
    \pgfuseimage{rfid-immigrants}
  \end{columns}

  \medskip
  \tiny{\url{http://www.livescience.com/10498-proposal-implant-tracking-chips-immigrants.html}}\\
\end{frame}

\begin{frame}
  \frametitle{Örnek: RFID çipli kimlikler}

  \begin{columns}
    \column{.6\textwidth}
    \pgfuseimage{rfid-passport}

    \column{.4\textwidth}
    \begin{itemize}
      \item RFID çipli kimlikler\\
        herkes tarafından\\
        kolayca okunabiliyor
      \item insanlar izlenebilir
      \item elde edilen veriler\\
        kimlik hırsızlığında\\
        kullanılabilir
    \end{itemize}
  \end{columns}

  \medskip
  \tiny{\url{http://www.foxnews.com/story/0,2933,531720,00.html}}\\
\end{frame}

\begin{frame}
  \frametitle{Gözetleme Teknolojileri: İnternet}

  \begin{columns}
    \column{.4\textwidth}
    \begin{itemize}
      \item çerezler
      \item kalıcı çerezler (Flash)
      \item JavaScript kodları
      \item böcekler
      \item IP adresleri
      \item HTTP başlık bilgileri
    \end{itemize}

    \pause
    \column{.6\textwidth}
    \begin{ornek}[Google Analytics]
      \begin{itemize}
        \item pek çok site, ziyaretçi sayılarını\\
          Google Analytics ile ölçüyor
        \item Google, bu sitelerin ziyaretçilerinin\\
          IP adreslerini izliyor
        \item ziyaretçinin Google hesabı varsa\\
          kim olduğunu belirleyebiliyor
      \end{itemize}
    \end{ornek}
  \end{columns}
\end{frame}

\subsection{Veri Koruma}

\begin{frame}
  \frametitle{Veri Koruma}

  \begin{itemize}
    \item devlet ve özel kurumlar kişiler hakkında\\
      büyük miktarda veri topluyor
    \begin{itemize}
      \item kredi kartı şirketleri, bankalar, alışveriş marketleri,\\
        İnternet siteleri, \ldots
      \item reklam ya da hizmet kalitesini artırma amaçlı
    \end{itemize}

    \pause
    \medskip
    \item bireyler kurumlar karşısında güçsüzleşiyor:
    \begin{itemize}
      \item kurum birey hakkında çok şey biliyor
      \item birey kurum hakkında neredeyse hiçbir şey bilmiyor
    \end{itemize}
  \end{itemize}
\end{frame}

\begin{frame}
  \frametitle{Bruce Schneier (2010)}

  \begin{columns}
    \column{.55\textwidth}
    \pgfuseimage{schneier}

    \column{.45\textwidth}
    \begin{quote}
      Facebook'un müşterisi olduğunuzu sanmak yanılgısına düşmeyin, değilsiniz
      -- siz ürünsünüz. Müşterileri, reklam verenler.
    \end{quote}
  \end{columns}

  \medskip
  \tiny{\url{http://www.information-age.com/channels/security-and-continuity/news/1290603/facebook-is-deliberately-killing-privacy-says-schneier.thtml}}\\
\end{frame}

\begin{frame}
  \frametitle{Sorunlar Yeni Mi?}

  \begin{itemize}
    \item toplanan verinin miktarı
    \item başka yerlere aktarılmasındaki hız
    \item verinin kalıcılığı
    \item bilginin kalitesi: ayrıntılı profil çıkartılabiliyor
  \end{itemize}
\end{frame}

\begin{frame}
  \frametitle{Eric Schmidt (2010)}

  \begin{columns}
    \column{.55\textwidth}
    \pgfuseimage{schmidt}

    \column{.45\textwidth}
    \begin{quote}
      Gençler reşit olacakları yaşa geldiklerinde isim değiştirme hakkına sahip olmalı.
    \end{quote}

    \pause
    \begin{quote}
      Her şey bulunabilir, bilinebilir ve herkes tarafından her zaman
      kaydedilebilir olduğunda ne olacağını toplumun anladığına inanmıyorum.
    \end{quote}
  \end{columns}

  \medskip
  \tiny{\url{https://www.readwriteweb.com/archives/google_ceo_suggests_you_change_your_name_to_escape.php}}\\
\end{frame}

\begin{frame}
  \frametitle{Hangi Veriler Değerli?}

  \begin{itemize}
    \item isim, doğum tarihi, adres, telefon, \ldots
    \item yapılan her alışveriş
    \item telefon görüşmeleri, mesajlaşmalar, e-postalar, \ldots
    \item politik ve dini görüşler
    \item cinsel tercihler
    \item sağlık verileri
    \item gelir verileri
    \item ziyaret edilen web siteleri
    \item arama motorlarında yapılan aramalar
    \item \ldots
  \end{itemize}
\end{frame}

\begin{frame}
  \frametitle{Hangi Veriler Duyarlı?}

  \begin{itemize}
    \item veriler duyarlılık düzeylerine ayrılabilir:\\
      gizli, yakın çevre, güven, toplumsal, açık
    \begin{itemize}
      \item kişiden kişiye değişir
      \item bir bilgi kendi başına duyarlı olmayabilir,\\
        ama başka bilgilerle birleştirildiğinde kritik hale gelebilir
    \end{itemize}

    \bigskip
    \item bütün kişisel veriler duyarlı
  \end{itemize}
\end{frame}

\begin{frame}
  \frametitle{Örnek: Sosyal ağ bilgileriyle hırsızlık}

  \begin{columns}
    \column{.58\textwidth}
    \pgfuseimage{burglars}

    \column{.42\textwidth}
    \begin{itemize}
      \item hırsızlar, soyacakları evi\\
        Facebook gibi\\
        sosyal ağlardaki\\
        durum güncellemelerine\\
        bakarak seçiyor (2010)
    \end{itemize}
  \end{columns}

  \medskip
  \tiny{\url{http://www.theregister.co.uk/2010/09/13/social_network_burglary_gang/}}\\
\end{frame}

\begin{frame}
  \frametitle{Sorunlar}

  \begin{itemize}
    \item verilerin izinsiz ya da yasalara aykırı şekilde toplanması
    \item verilerin toplama amacına uygun olmayan şekilde kullanılması
    \item verilerin güvenliğinin sağlanmaması
    \item verilerin kişinin denetimi dışında başkalarına aktarılması
    \item verilerin kişinin denetimi dışında başka verilerle birleştirilmesi
    \item kişilerin kendileriyle ilgili ne veri tutulduğunu bilememesi,\\
      yanlışları düzeltememesi
  \end{itemize}
\end{frame}

\section{Sorunlar}

\subsection{Veri Toplama}

\begin{frame}
  \frametitle{Veri Toplama}

  \begin{itemize}
    \item hangi kurumlar hangi verileri toplayabilsin?
    \begin{itemize}
      \item amaca uygunluk
    \end{itemize}

    \pause
    \medskip
    \item veriler hangi koşullarda toplanabilsin?
    \begin{itemize}
      \item yasalara uygunluk
      \item ilgili kişinin bilgilendirilmesi, gerekiyorsa izninin alınması
    \end{itemize}
  \end{itemize}
\end{frame}

\begin{frame}
  \frametitle{Örnek: Google Street View}

  \begin{columns}
    \column{.45\textwidth}
    \pgfuseimage{street-view}

    \column{.55\textwidth}
    \begin{itemize}
      \item Google Street View araçları,\\
        WiFi erişim noktaları ile\\
        bunları kullanan aygıtların\\
        ayırıcı verilerini topluyor (2010)
    \end{itemize}
  \end{columns}

  \medskip
  \tiny{\url{http://news.cnet.com/8301-31921_3-20082777-281/street-view-cars-grabbed-locations-of-phones-pcs/}}\\
\end{frame}

\begin{frame}
  \frametitle{Örnek: Facebook yüz tanıma}

  \begin{columns}
    \column{.55\textwidth}
    \pgfuseimage{face-recognition}

    \column{.45\textwidth}
    \begin{itemize}
      \item Facebook, resimlerdeki\\
        yüzleri tanıyarak\\
        otomatik etiketliyor (2011)
      \item kullanıcılardan izin almıyor
      \item Almanya, bu özelliği\\
        kaldırmasını ve\\
        topladığı verileri\\
        silmesini söylüyor
    \end{itemize}
  \end{columns}

  \medskip
  \tiny{\url{https://www.pcworld.com/article/229742/why_facebooks_facial_recognition_is_creepy.html}}\\
  \tiny{\url{http://www.dw.de/dw/article/0,,15290120,00.html}}\\
\end{frame}

\subsection{Kullanım Şekli}

\begin{frame}
  \frametitle{Verilerin Kullanımı}

  \begin{itemize}
    \item verilerin kullanımının toplanma amacına uygunluğu
    \begin{itemize}
      \item ilgili kişinin aleyhine kullanılması: kimlik hırsızlığı, şantaj
    \end{itemize}

    \pause
    \medskip
    \item kimlerin hangi koşullarda erişeceğine ilişkin kurallar ve denetim
    \begin{itemize}
      \item teknik personelin erişimi
    \end{itemize}
  \end{itemize}
\end{frame}

\begin{frame}
  \frametitle{Örnek: Avustralya yurttaşlık kayıtları}

  \begin{columns}
    \column{.58\textwidth}
    \pgfuseimage{oz}

    \column{.42\textwidth}
    \begin{itemize}
      \item devlet görevlileri\\
        yurttaşlık kayıtlarını\\
        kimlik hırsızlığında\\
        kullanıyor (2006)
    \end{itemize}
  \end{columns}

  \medskip
  \tiny{\url{http://www.theregister.co.uk/2006/08/28/oz_id_database_misused/}}\\
\end{frame}

\subsection{Veri Güvenliği}

\begin{frame}
  \frametitle{Verilerin Güvenliği}

  \begin{itemize}
    \item güvenlik açıkları, hatalar ya da düşüncesizlik\\
      nedeniyle veriler açığa çıkabiliyor
    \item veri toplayanlar bu verilerin güvenliğini sağlamalı
  \end{itemize}
\end{frame}

\begin{frame}
  \frametitle{Örnek: York Üniversitesi öğrenci kayıtları}

  \begin{columns}
    \column{.55\textwidth}
    \pgfuseimage{york}

    \column{.45\textwidth}
    \begin{itemize}
      \item York Üniversitesi\\
        öğrencilerinin\\
        kişisel verileri\\
        çalınıyor (2011)
    \end{itemize}
  \end{columns}

  \medskip
  \tiny{\url{http://www.bbc.co.uk/news/uk-england-york-north-yorkshire-12756951}}\\
\end{frame}

\begin{frame}
  \frametitle{Örnek: Türkiye BTK kayıtları}

  \begin{columns}
    \column{.55\textwidth}
    \pgfuseimage{btk}

    \column{.45\textwidth}
    \begin{itemize}
      \item bir hacker grubu,\\
        BTK bilgisayarlarından\\
        kişisel verileri\\
        çalarak yayımlıyor (2012)
      \item diğer bir hacker grubu,\\
        polis bilgisayarlarından\\
        yurttaşların\\
        ihbar duyurularını\\
        çalarak yayımlıyor (2012)
    \end{itemize}
  \end{columns}

  \medskip
  \tiny{\url{http://www.radikal.com.tr/Radikal.aspx?aType=RadikalDetayV3&ArticleID=1078717&CategoryID=77&Rdkref=6}}\\
  \tiny{\url{http://www.radikal.com.tr/Radikal.aspx?aType=RadikalDetayV3&ArticleID=1080108&CategoryID=77&Rdkref=6}}\\
\end{frame}

\begin{frame}
  \frametitle{Örnek: İngiltere çocuk yardımı kayıtları}

  \begin{columns}
    \column{.4\textwidth}
    \pgfuseimage{brown}

    \column{.6\textwidth}
    \begin{itemize}
      \item Gelirler Dairesi, 25 milyon kişinin\\
        çocuk yardımı kayıtlarını içeren\\
        CD'yi postada kaybediyor (2007)
    \end{itemize}
  \end{columns}

  \medskip
  \tiny{\url{http://www.guardian.co.uk/politics/2007/nov/21/economy.uk}}\\
  \tiny{\url{http://news.bbc.co.uk/2/hi/entertainment/7174760.stm}}\\
\end{frame}

\begin{frame}
  \frametitle{Örnek: AOL arama kayıtları}

  \begin{columns}
    \column{.6\textwidth}
    \pgfuseimage{aol}

    \column{.4\textwidth}
    \begin{itemize}
      \item AOL, yüzbinlerce\\
        kullanıcısının\\
        arama kayıtlarını\\
        yayımlıyor (2006)
    \end{itemize}
  \end{columns}

  \medskip
  \tiny{\url{http://news.cnet.com/AOLs-disturbing-glimpse-into-users-lives/2100-1030_3-6103098.html}}\\
\end{frame}

\begin{frame}
  \frametitle{Örnek: İtalya vergi kayıtları}

  \begin{columns}
    \column{.55\textwidth}
    \pgfuseimage{italy}

    \column{.45\textwidth}
    \begin{itemize}
      \item Vergi Dairesi,\\
        bütün İtalyanların\\
        isim, adres, doğum tarihi,\\
        gelir ve vergi bilgilerini\\
        yayımlıyor (2008)
    \end{itemize}
  \end{columns}

  \medskip
  \tiny{\url{http://www.theregister.co.uk/2008/05/01/italy_publishes_tax_details/}}\\
\end{frame}

\subsection{Veri Aktarma}

\begin{frame}
  \frametitle{Verilerin Aktarılması}

  \begin{itemize}
    \item veriler hangi koşullar altında başkalarına aktarılabilsin?
    \begin{itemize}
      \item reklam verenler, iş ortakları, devlet güvenlik kurumları
      \item ilgili kişilerden izin alınması
    \end{itemize}
  \end{itemize}
\end{frame}

\begin{frame}
  \frametitle{Örnek: Robert Bork}

  \begin{columns}
    \column{.4\textwidth}
    \pgfuseimage{bork}

    \column{.6\textwidth}
    \begin{itemize}
      \item bir yargıç yüksek mahkemeye\\
        aday gösteriliyor
      \item bir gazeteci, yargıcın abone olduğu\\
        video şirketinden aldığı filmlerin\\
        listesini yayımlıyor (1988)

      \medskip
      \item \emph{Video Privacy Protection Act}
    \end{itemize}
  \end{columns}

  \medskip
  \tiny{\url{http://www.theatlanticwire.com/technology/2011/07/why-robert-bork-indirectly-kept-netflix-facebook/40408/}}\\
\end{frame}

\begin{frame}
  \frametitle{Örnek: Rebecca Schaeffer cinayeti}

  \begin{columns}
    \column{.53\textwidth}
    \pgfuseimage{schaeffer}

    \column{.47\textwidth}
    \begin{itemize}
      \item bir hayranı\\
        evinin kapısında öldürüyor
      \item Motorlu Taşıtlar Dairesi\\
        ehliyet bilgilerini\\
        isteyene satıyor (1989)

      \medskip
      \item \emph{Driver's Privacy Protection Act (1994)}
    \end{itemize}
  \end{columns}

  \medskip
  \tiny{\url{http://investigation.discovery.com/investigation/hollywood-crimes/schaeffer/rebecca-schaeffer.html}}\\
\end{frame}

\begin{frame}
  \frametitle{Örnek: Toysmart müşteri bilgileri}

  \begin{columns}
    \column{.53\textwidth}
    \pgfuseimage{toysmart}

    \column{.47\textwidth}
    \begin{itemize}
      \item Toysmart firmasının\\
        mahremiyet anlaşmasında\\
        müşteri bilgilerinin\\
        üçüncü şahıslara\\
        verilmeyeceği yazıyor
      \item firma iflas edince\\bilgileri satışa çıkarıyor (2001)
      \item dava açılıyor,\\
        veriler yok ediliyor
    \end{itemize}
  \end{columns}

  \medskip
  \tiny{\url{http://www.wired.com/politics/law/news/2001/01/41102}}\\
\end{frame}

\begin{frame}
  \frametitle{Örnek: MySpace kullanıcı verileri}

  \begin{columns}
    \column{.55\textwidth}
    \pgfuseimage{myspace}

    \column{.45\textwidth}
    \begin{itemize}
      \item MySpace\\
        kullanıcı verilerini\\
        satışa çıkarıyor (2010)
      \item isimler, posta kodları,\\
        fotoğraflar, şarkı listeleri,\\
        blog yazıları, \ldots
    \end{itemize}
  \end{columns}

  \medskip
  \tiny{\url{https://www.readwriteweb.com/archives/myspace_bulk_data.php}}\\
\end{frame}

\begin{frame}
  \frametitle{Örnek: ABD başkan adaylığı anketleri}

  \begin{columns}
    \column{.55\textwidth}
    \pgfuseimage{politico}

    \column{.45\textwidth}
    \begin{itemize}
      \item Facebook,\\
        Cumhuriyetçi Parti\\
        başkan adaylığı için\\
        hangi adayın\\
        daha popüler olduğunu\\
        ölçmek üzere\\
        kullanıcı verilerini\\
        Politico'ya veriyor (2012)
    \end{itemize}
  \end{columns}

  \medskip
  \tiny{\url{https://www.facebook.com/notes/us-politics-on-facebook/politico-facebook-team-up-to-measure-gop-candidate-buzz/10150461091205882}}\\
\end{frame}

\begin{frame}
  \frametitle{Örnek: Google - ABD Adalet Bakanlığı}

  \begin{columns}
    \column{.55\textwidth}
    \pgfuseimage{google-doj}

    \column{.45\textwidth}
    \begin{itemize}
      \item Adalet Bakanlığı, Google'dan kişilerin\\
        ne arama yaptıkları\\
        verilerini istiyor,\\
        Google vermiyor (2006)

      \pause
      \item Amazon, müşterilerinin\\
        alışveriş bilgileri istenince\\
        dava açıyor, kazanıyor (2010)
    \end{itemize}
  \end{columns}

  \medskip
  \tiny{\url{http://news.bbc.co.uk/2/hi/technology/4630694.stm}}\\
  \tiny{\url{http://www.theregister.co.uk/2010/10/27/amazon_sales/}}\\
\end{frame}

\subsection{Veri Birleştirme}

\begin{frame}
  \frametitle{Örnek: DoubleClick - Abacus}

  \begin{columns}
    \column{.5\textwidth}
    \pgfuseimage{abacus}

    \column{.5\textwidth}
    \begin{itemize}
      \item reklam firması DoubleClick,\\
        tüketici alışkanlıkları firması Abacus
        ile birleşmek istiyor\\
      \item mahremiyet baskısı\\
        nedeniyle vazgeçiyor (2000)

      \pause
      \medskip
      \item Google, DoubleClick'i\\
        satın alıyor (2008)
    \end{itemize}
  \end{columns}

  \medskip
  \tiny{\url{http://news.bbc.co.uk/2/hi/technology/4630694.stm}}\\
  \tiny{\url{http://www.businessweek.com/magazine/content/11_12/b4220038620504.htm}}\\
\end{frame}

\section{Önlemler}

\subsection{Yasal}

\begin{frame}
  \frametitle{Yasal Düzenlemeler}

  \begin{itemize}
    \item Avrupa Birliği'nde kapsamlı düzenlemeler
    \begin{itemize}
      \item ilk veri koruma yasası Almanya'nın Hessen eyaletinde (1970)
    \end{itemize}

    \pause
    \item ABD'de alana özel düzenlemeler:\\
      kredi, ehliyet, video, sağlık, ...
    \begin{itemize}
      \item yeni bir mahremiyet yasa taslağı (2012)
    \end{itemize}

    \pause
    \item Türkiye'de Avrupa Birliği ile ilişkiler ve uluslararası sözleşmeler\\
      çerçevesinde kalıyor
  \end{itemize}
\end{frame}

\begin{frame}
  \frametitle{Avrupa Birliği}

  \begin{block}{AB Temel Haklar Bildirgesi}
    \begin{enumerate}
      \item kişisel bilgilerin korunmasını isteme hakkı
      \item amaca uygun, yasal ve adil kullanım;\\
        kendisiyle ilgili bilgilere erişme ve düzeltme hakkı
      \item kurallara uyulmasını denetleyen bağımsız makam
    \end{enumerate}
  \end{block}

  \begin{itemize}
    \item üye ülkeler bu ilkelere uygun yasal düzenlemeler yapmalı
  \end{itemize}
\end{frame}

\begin{frame}
  \frametitle{Avrupa Yasaları}

  \begin{block}{İngiltere - Data Protection Act (1984, 1998)}
    \begin{enumerate}
      \item adil ve hukuki işleme
      \item açıkça belirtilmiş ve yasal amaçlar
      \item yeterlilik, ilgililik, gerekenden fazla olmama
      \item doğruluk ve güncellik
      \item amaç yerine geldikten sonra silinme
      \item diğer kişilerin hakları
      \item yetkisiz erişim ve kayba karşı yeterli önlem
      \item yeterli koruma sağlamayan ülkelere aktarmama
    \end{enumerate}
  \end{block}
\end{frame}

\begin{frame}
  \frametitle{Örnek: İngiltere DNA veri bankası}

  \begin{center}
    \pgfuseimage{dna}
  \end{center}

  \begin{itemize}
    \item Avrupa İnsan Hakları Mahkemesi kararı (2008):
    \begin{quote}
      Hüküm giymemiş kişilerin DNA bilgileri saklanamaz.
    \end{quote}
  \end{itemize}

  \medskip
  \tiny{\url{http://news.bbc.co.uk/2/hi/technology/4630694.stm}}\\
\end{frame}

\begin{frame}
  \frametitle{Örnek: Reddit - Facebook bilgi edinme istekleri}

  \begin{columns}
    \column{.55\textwidth}
    \pgfuseimage{reddit}

    \column{.45\textwidth}
    \begin{itemize}
      \item Reddit kullanıcıları\\
        bilgi edinme istekleriyle\\
        Facebook'a yoğun\\
        iş yükü çıkarıyorlar (2011)
      \item 24~yaşında bir öğrenciye\\
        1200~sayfalık döküm geliyor
    \end{itemize}
  \end{columns}

  \medskip
  \tiny{\url{https://www.zdnet.com/blog/facebook/reddit-users-overwhelm-facebook-with-data-requests/4165}}\\
  \tiny{\url{https://threatpost.com/en_us/blogs/twenty-something-asks-facebook-his-file-and-gets-it-all-1200-pages-121311}}\\
\end{frame}

\begin{frame}
  \frametitle{Örnek: ABD - Avrupa yolcu verileri}

  \begin{columns}
    \column{.48\textwidth}
    \pgfuseimage{flights}

    \column{.52\textwidth}
    \begin{itemize}
      \item havayolu şirketlerinden ABD'ye uçuşlarda yolcularla ilgili 34~parça
        bilgi isteniyor
      \item Avrupa Adalet Divanı\\
        anlaşmayı iptal ediyor (2006)
      \item daha sonraki anlaşmaya\\
        Avrupa Veri Koruma\\
        Denetçisi karşı çıkıyor (2011)

      \pause
      \item ABD istihbarat birimlerinin Avrupa banka kayıtlarına erişimi
        anlaşmasını Avrupa Parlamentosu engelliyor (2010)
    \end{itemize}
  \end{columns}

  \medskip
  \tiny{\url{http://news.bbc.co.uk/2/hi/europe/5028918.stm}}\\
  \tiny{\url{http://www.theregister.co.uk/2011/03/29/europe_passenger_name_wrong/}}\\
  \tiny{\url{http://news.bbc.co.uk/2/hi/europe/8510471.stm}}\\
\end{frame}

\begin{frame}
  \frametitle{ABD}

  \begin{block}{Code of Fair Information Practices (1974)}
    \begin{enumerate}
      \item varlığı gizli bir veri kaydı tutma sistemi olamaz
      \item kişiler, kendileriyle ilgili hangi bilgilerin tutulduğunu\\
        ve bu bilgilerin nasıl kullanıldığını öğrenebilmelidir
      \item kişiler, bir amaçla verdikleri bilgilerin, izinleri alınmadan\\
        başka amaçlarla kullanımını önleyebilmelidir
      \item kişiler, kendileriyle ilgili bilgileri düzeltebilmelidir
      \item kişisel veri toplayan kurumlar, bu verilerin ilgili amaçlar için\\
        güvenilirliğini sağlamalı ve kötüye kullanımını önlemelidir
    \end{enumerate}
  \end{block}
\end{frame}

\begin{frame}
  \frametitle{ABD}

  \begin{block}{Code of Fair Information Practices (1974)}
    \begin{itemize}
      \item yalnızca kamu kurumlarını bağlar
      \item kayıtların ayırdedici bir veriye göre aranabilmesi gerekir
      \item yasayı uygulayıcı bir yetkili tanımlı değil
      \item "sıradan kullanımlar için" veri alışverişine izin var
    \end{itemize}
  \end{block}
\end{frame}

\begin{frame}
  \frametitle{ABD}

  \begin{block}{Consumer Privacy Bill of Rights (2012)}
    \begin{itemize}
      \item tüketiciler, firmaların kendileriyle ilgili hangi verileri
        topladıklarını ve bunları nasıl kullandıkları denetleyebilmeli
      \item tüketiciler, mahremiyet ve güvenlik yöntemleriyle ilgili kolayca
        ve anlaşılabilir şekilde bilgi alabilmeli
      \item toplama, kullanma ve yaymanın amaca uygunluğu
      \item verinin güvenli ve sorumlu şekilde işlenmesi
    \end{itemize}
  \end{block}
\end{frame}

\begin{frame}
  \frametitle{Türkiye}

  \begin{itemize}
    \item Anayasa: özel ve aile yaşamına saygı bekleme hakkı
    \item Anayasa: iletişimin gizliliği
    \item Kişisel Verilerin Otomatik İşlenmesine İlişkin\\
      Uluslararası Sözleşme
    \item Bilgi Edinme Hakkı Yasası
  \end{itemize}
\end{frame}

\begin{frame}
  \frametitle{Kişisel Verilerin Korunması Yasa Taslağı}

  \begin{block}{Kişisel Verilerin Korunması Yasa Taslağı}
    \begin{itemize}
      \item \emph{Kişisel Verileri Koruma Yüksek Kurulu}
    \end{itemize}
  \end{block}
\end{frame}

\begin{frame}
  \frametitle{Bilgi Edinme Hakkı Yasası}

  \begin{block}{Bilgi Edinme Hakkı Yasası (2003)}
    \begin{itemize}
      \item bilgi edinme hakkı
      \item bilgi verme yükümlülüğü
      \item itirazlar için \emph{Bilgi Edinme Değerlendirme Kurulu}
      \item ihmal, kusur ya da kasıt durumunda ceza kovuşturması

      \pause
      \item sınırlamalar:\\
        devlet sırrı, ülkenin ekonomik çıkarları, istihbarat,\\
        idari ve adli soruşturmalar, özel hayatın gizliliği,\\
        haberleşmenin gizliliği, ticari sır
    \end{itemize}
  \end{block}
\end{frame}

\subsection{Kurumsal}

\begin{frame}
  \frametitle{Kurum Politikaları}

  \begin{itemize}
    \item kurumlar kullanıcılarının mahremiyetine değişik düzeylerde\\
      özen gösteriyor: mahremiyet politikaları
    \begin{itemize}
      \item hangi verileri hangi amaçla topladığını açıkça belirtmek
      \item gerekmeyen verileri toplamamak
      \item amaca aykırı kullanmamak
      \item izin almadan başkalarına aktarmamak
    \end{itemize}

    \pause
    \medskip
    \item "istemiyorsa çıksın" (opt-out) değil, "istiyorsa girsin" (opt-in)

    \pause
    \medskip
    \item HTTP protokolu için yeni başlık alanı: Do Not Track
  \end{itemize}
\end{frame}

\begin{frame}
  \frametitle{Örnek: Facebook}

  \begin{columns}
    \column{.48\textwidth}
    \pgfuseimage{facebook-evolution}

    \column{.52\textwidth}
    \begin{itemize}
      \item hesabınızı kapatsanız da\\
        bilgileriniz kalıyor
      \item varsayılan ayarlarda\\
        açık olan veriler\\
        giderek artıyor
      \item mahremiyet ayarlarının\\
        karmaşıklığı eleştiriliyor\\
      \item işyerleri, okullar v.b.\\
        hesabınıza erişim istiyor
    \end{itemize}
  \end{columns}

  \medskip
  \tiny{\url{http://mattmckeon.com/facebook-privacy/}}\\
  \tiny{\url{https://www.nytimes.com/interactive/2010/05/12/business/facebook-privacy.html}}\\
  \tiny{\url{http://redtape.msnbc.msn.com/_news/2012/03/06/10585353-govt-agencies-colleges-demand-applicants-facebook-passwords}}\\
\end{frame}

\begin{frame}
  \frametitle{Örnek: Google}

  \begin{itemize}
    \item gerçek isim politikası
    \item hizmetleri birleştiren mahremiyet politikası,\\
      hem ahlaki hem de yasal açıdan eleştiriliyor (2012)
  \end{itemize}

  \medskip
  \tiny{\url{http://www.theregister.co.uk/2012/03/06/why_google_privacy_policy_is_so_difficult_to_follow/}}\\
\end{frame}

\begin{frame}
  \frametitle{Örnek: Arama motorları}

  \begin{itemize}
    \item arama motorları arama verilerini bir süre sonra\\
      anonimize ediyor
    \begin{itemize}
      \item anonimizasyonun işe yaramadığı iddia ediliyor
    \end{itemize}

    \medskip
    \item bazı arama motorları IP adreslerini kaydetmiyor:\\
      Startpage, DuckDuckGo
  \end{itemize}

  \medskip
  \tiny{\url{http://arstechnica.com/tech-policy/news/2009/09/your-secrets-live-online-in-databases-of-ruin.ars}}\\
\end{frame}

\subsection{Kişisel}

\begin{frame}
  \frametitle{Kişisel Önlemler}

  \begin{itemize}
    \item kişisel verileri konusunda duyarlı olmak
    \begin{itemize}
      \item gerekmeyen yerlere vermemek
      \item yanlış bilgiler vermek
      \item veri "gürültüsü" yaratmak
    \end{itemize}

    \pause
    \medskip
    \item teknolojiden yararlanmak
    \begin{itemize}
      \item şifreleme yazılımları
      \item mahremiyet artırıcı yazılımlar
    \end{itemize}
  \end{itemize}
\end{frame}

\begin{frame}
  \frametitle{Örnek: Anket soruları}

  \begin{columns}
    \column{.52\textwidth}
    \pgfuseimage{survey}

    \column{.48\textwidth}
    \begin{itemize}
      \item katılanların \%90'dan fazlası\\
        duyarlı verilerini veriyor (2005)
      \begin{itemize}
        \item evcil hayvanının adı
        \item annesinin evlenmeden önceki soyadı
      \end{itemize}

      \item Facebook'da arkadaşlık isteklerini düşünmeden\\
        kabul ediyor (2009)
    \end{itemize}
  \end{columns}

  \medskip
  \tiny{\url{http://news.bbc.co.uk/2/hi/technology/4378253.stm}}\\
  \tiny{\url{http://nakedsecurity.sophos.com/2009/12/06/facebook-id-probe-2009/}}\\
\end{frame}

\begin{frame}
  \frametitle{Örnek: Google Chrome tarayıcı izleme}

  \begin{columns}
    \column{.48\textwidth}
    \pgfuseimage{chrome-history}

    \column{.52\textwidth}
    \begin{itemize}
      \item Google, bakılan her sayfanın izleneceği bir programa katılanlara yılda 25\$ veriyor (2012)

      \item ücret farkı çok az bile olsa,\\
        insanlar mahremiyeti koruyan\\
        servisi değil, ucuz olan servisi\\
        seçiyor (2012)
    \end{itemize}
  \end{columns}

  \medskip
  \tiny{\url{http://www.forbes.com/sites/kashmirhill/2012/02/09/your-online-privacy-is-worth-less-than-a-six-pack-of-marshmallow-fluff/}}\\
  \tiny{\url{http://www.theregister.co.uk/2012/03/21/privacy_economics/}}\\
\end{frame}

\begin{frame}
  \frametitle{Veri Gürültüsü}

  \begin{itemize}
    \item çok miktarda yanlış ya da ilgisiz veri üreterek\\
      gerçek verilerin hangisi olduğunun belirlenmesi zorlaştırılabilir
  \end{itemize}

  \pause
  \begin{ornek}
    \begin{itemize}
      \item alışveriş kartlarını başkalarıyla karşılıklı değişmek
      \item Firefox eklentisi TrackMeNot, belli başlı arama motorlarında
        rasgele aramalar yapıyor
    \end{itemize}
  \end{ornek}
\end{frame}

\begin{frame}
  \frametitle{Mahremiyet Artırıcı Yazılımlar}

  \begin{itemize}
    \item vekil sunucular (proxy)
    \begin{itemize}
      \item Privoxy: reklamları ve kötü niyetli kod kalıplarını engelliyor
    \end{itemize}

    \pause
    \medskip
    \item Firefox tarayıcı eklentileri
    \begin{itemize}
      \item NoScript: JavaScript beyaz listesi
      \item Cookie Monster: çerez beyaz listesi
      \item Ghostery: böcek kara listesi
    \end{itemize}
  \end{itemize}
\end{frame}

\section*{Kaynaklar}

\begin{frame}
  \frametitle{Kaynaklar}

  \begin{block}{Okunacak: Tavani}
    \begin{itemize}
      \item Chapter 5: \alert{Privacy and Cyberspace}
    \end{itemize}
  \end{block}
\end{frame}

\end{document}
