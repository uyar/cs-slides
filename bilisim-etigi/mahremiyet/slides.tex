% Copyright (c) 2004-2012 H. Turgut Uyar <uyar@itu.edu.tr>
%
% Bu notlar "Creative Commons Attribution-NonCommercial-ShareAlike License" ile
% lisanslanmıştır. Yazarının açıkça belirtilmesi koşuluyla ve ticari olmayan
% amaçlarla kullanılabilir ve dağıtılabilir. Bu notlardan yola çıkılarak
% oluşturulacak çalışmaların da aynı lisansa bağlı olmaları gerekir.
%
% Lisans ile ilgili ayrıntılı bilgi almak için şu sayfaya başvurabilirsiniz:
% http://creativecommons.org/licenses/by-nc-sa/3.0/

\documentclass[dvipsnames]{beamer}

\mode<presentation>
{
  \usetheme{Rochester}
  \usecolortheme[named=Mahogany]{structure}
  \setbeamercovered{transparent}
}

\usepackage{ae}
\usepackage[T1]{fontenc}
\usepackage[turkish]{babel}
\usepackage[utf8]{inputenc}
\setbeamertemplate{navigation symbols}{}

\title{Bilişim Etiği}
\subtitle{Mahremiyet}

\author{H. Turgut Uyar}
\date{2004-2012}

\AtBeginSubsection[]
{
  \begin{frame}<beamer>
    \frametitle{Konular}
    \tableofcontents[currentsection,currentsubsection]
  \end{frame}
}

%\beamerdefaultoverlayspecification{<+->}

\theoremstyle{definition}
\newtheorem{tanim}[theorem]{Tanım}

\theoremstyle{example}
\newtheorem{ornek}[theorem]{Örnek}

\theoremstyle{plain}

\pgfdeclareimage[width=2cm]{license}{../../license}

\pgfdeclareimage[width=5.8cm]{zuckerberg}{zuckerberg}
\pgfdeclareimage[width=7cm]{pennsylvania}{pennsylvania}
\pgfdeclareimage[width=5.8cm]{iphone}{iphone}
\pgfdeclareimage[width=7.5cm]{carrieriq}{carrieriq}
\pgfdeclareimage[width=7.5cm]{rfid-immigrants}{rfid-immigrants}
\pgfdeclareimage[width=7.2cm]{rfid-passport}{rfid-passport}
\pgfdeclareimage[width=7cm]{schneier}{schneier}
\pgfdeclareimage[width=7cm]{schmidt}{schmidt}
\pgfdeclareimage[width=7.2cm]{hirsiz}{hirsiz}
\pgfdeclareimage[width=7.2cm]{avustralya}{avustralya}
\pgfdeclareimage[width=7.5cm]{aol}{aol}
\pgfdeclareimage[width=7.2cm]{italya}{italya}
\pgfdeclareimage[height=6.5cm]{brown}{brown}
\pgfdeclareimage[width=6.5cm]{btk}{btk}

\begin{document}

\begin{frame}
  \titlepage
\end{frame}

\begin{frame}
  \frametitle{Lisans}

  \pgfuseimage{license}\hfill
  \copyright 2004-2012 H. Turgut Uyar

  \vfill
  \begin{tiny}
    You are free:
    \begin{itemize}
      \item to Share — to copy, distribute and transmit the work
      \item to Remix — to adapt the work
    \end{itemize}

    Under the following conditions:
    \begin{itemize}
      \item Attribution — You must attribute the work in the manner specified by
        the author or licensor (but not in any way that suggests that they
        endorse you or your use of the work).

      \item Noncommercial — You may not use this work for commercial purposes.

      \item Share Alike — If you alter, transform, or build upon this work, you
        may distribute the resulting work only under the same or similar license
        to this one.
    \end{itemize}
  \end{tiny}

  \vfill
  Legal code (the full license):\\
  \url{http://creativecommons.org/licenses/by-nc-sa/3.0/}
\end{frame}

\begin{frame}
  \frametitle{Konular}
  \tableofcontents
\end{frame}

\section{Mahremiyet}

\subsection{Giriş}

\begin{frame}
  \frametitle{Mahremiyet}

  \begin{itemize}
    \item erişim mahremiyeti
    \begin{itemize}
      \item rahat bırakılma
    \end{itemize}

    \pause
    \item karar verme mahremiyeti
    \begin{itemize}
      \item kişisel kararlara karışılmaması
    \end{itemize}

    \pause
    \item bilgi mahremiyeti
    \begin{itemize}
      \item kişisel verileri üzerinde denetim sahibi olma
    \end{itemize}
  \end{itemize}
\end{frame}

\begin{frame}
  \frametitle{Mahremiyet Değeri}

  \begin{columns}
    \column{.46\textwidth}
    \pgfuseimage{zuckerberg}

    \column{.54\textwidth}
    \begin{itemize}
      \item Eric Schmidt (Google - 2009):
      \begin{quote}
        Başkalarının bilmesini istemeyeceğiniz bir şey yapıyorsanız,\\
        belki de zaten yapmamanız gerekiyordur.
      \end{quote}

      \pause
      \item Mark Zuckerberg\\
        (Facebook - 2010):
      \begin{quote}
        Mahremiyet artık bir toplumsal norm değil.
      \end{quote}
    \end{itemize}
  \end{columns}

  \medskip
  \tiny{\url{http://www.telegraph.co.uk/technology/facebook/6966628/Facebooks-Mark-Zuckerberg-says-privacy-is-no-longer-a-social-norm.html}}\\
  \tiny{\url{http://www.theregister.co.uk/2007/12/03/zuckerberg_files/}}\\
\end{frame}

\begin{frame}
  \frametitle{Mahremiyet Değeri}

  \begin{itemize}
    \item mahremiyet evrensel bir değer değil

    \pause
    \medskip
    \item mahremiyet özünde iyi bir değer mi?
    \begin{itemize}
      \item güvenlik, özerklik mahremiyet olmadan mümkün değil
      \item farklı türden ilişkiler
    \end{itemize}
  \end{itemize}
\end{frame}

\subsection{Gözetleme}

\begin{frame}
  \frametitle{Gözetleme}

  \begin{itemize}
    \item kişilerin izlenmesini sağlayan teknolojiler arttı
  \end{itemize}

  \pause
  \begin{ornek}
    \begin{itemize}
      \item kredi kartları
      \item cep telefonları
      \item video kameralar
      \item yüz tanıma, plaka tanıma
      \item \ldots
    \end{itemize}
  \end{ornek}
\end{frame}

\begin{frame}
  \frametitle{Örnek Olay: A.B.D. öğrenci bilgisayarları (2010)}

  \begin{columns}
    \column{.57\textwidth}
    \pgfuseimage{pennsylvania}

    \column{.43\textwidth}
    \begin{itemize}
      \item bir okul,\\
        öğrencilere dağıttığı\\
        bilgisayarlardaki
        web kameralarını\\
        uzaktan etkinleştirerek\\
        öğrencileri gözetliyor
    \end{itemize}
  \end{columns}

  \medskip
  \tiny{\url{https://www.computerworld.com/s/article/9190722/Update_School_district_settles_webcam_spying_suit_for_610_000_}}
\end{frame}

\begin{frame}
  \frametitle{Örnek Olay: Cep telefonlarında konum takibi (2011)}

  \begin{columns}
    \column{.5\textwidth}
    \pgfuseimage{iphone}

    \column{.5\textwidth}
    \begin{itemize}
      \item Apple aygıtları\\
        bulundukları konumu,\\
        sahiplerinin haberi olmadan\\
        bir siteye bildiriyor
      \item aynı durum Google Android\\
        ve Microsoft Windows\\
        telefonlarda da ortaya çıkıyor
    \end{itemize}
  \end{columns}

  \medskip
  \tiny{\url{http://edition.cnn.com/2011/TECH/mobile/04/20/iphone.tracking/index.html?hpt=T1}}\\
  \tiny{\url{http://www.theregister.co.uk/2011/04/22/apple_iphone_location_tracking_analysis/}}\\
  \tiny{\url{http://news.cnet.com/8301-31921_3-20057329-281.html}}\\
\end{frame}

\begin{frame}
  \frametitle{Örnek Olay: Cep telefonlarında tuş takibi (2011)}

  \begin{columns}
    \column{.65\textwidth}
    \pgfuseimage{carrieriq}

    \column{.45\textwidth}
    \begin{itemize}
      \item Carrier IQ yazılımı,\\
        kullanıcının bastığı tuşları\\
        kaydediyor
    \end{itemize}
  \end{columns}

  \medskip
  \tiny{\url{http://www.theregister.co.uk/2011/11/30/smartphone_spying_app/}}
\end{frame}

\begin{frame}
  \frametitle{Gözetleme Teknolojileri: RFID}

  \begin{columns}
    \column{.3\textwidth}
    \begin{itemize}
      \item envanter takibi
      \item hayvancılık

      \pause
      \medskip
      \item evcil hayvanlar,\\
        çocuklar
      \item göçmenler,\\
        mahkumlar
    \end{itemize}

    \column{.7\textwidth}
    \pgfuseimage{rfid-immigrants}
  \end{columns}

  \medskip
  \tiny{\url{http://www.livescience.com/10498-proposal-implant-tracking-chips-immigrants.html}}
\end{frame}

\begin{frame}
  \frametitle{Örnek Olay: RFID çipli kimlikler (2009)}

  \begin{columns}
    \column{.6\textwidth}
    \pgfuseimage{rfid-passport}

    \column{.4\textwidth}
    \begin{itemize}
      \item RFID çipli kimlikler\\
        herkes tarafından\\
        kolayca okunabiliyor
      \item insanlar izlenebilir
      \item elde edilen veriler\\
        kimlik hırsızlığında\\
        kullanılabilir
    \end{itemize}
  \end{columns}

  \medskip
  \tiny{\url{http://www.theregister.co.uk/2010/09/13/social_network_burglary_gang/}}
\end{frame}

\begin{frame}
  \frametitle{Gözetleme Teknolojileri: İnternet}

  \begin{columns}
    \column{.4\textwidth}
    \begin{itemize}
      \item çerezler
      \item kalıcı çerezler (Flash)
      \item JavaScript kodları
      \item böcekler
      \item IP adresleri
      \item HTTP başlık bilgileri
    \end{itemize}

    \pause
    \column{.6\textwidth}
    \begin{ornek}[Google Analytics]
      \begin{itemize}
        \item pek çok site, ziyaretçi sayılarını\\
          Google Analytics ile ölçüyor
        \item Google, bu sitelerin ziyaretçilerinin\\
          IP adreslerini izliyor
        \item ziyaretçinin Google hesabı varsa\\
          kim olduğunu belirleyebiliyor
      \end{itemize}
    \end{ornek}
  \end{columns}
\end{frame}

\subsection{Veri Koruma}

\begin{frame}
  \frametitle{Veri Koruma}

  \begin{itemize}
    \item devlet ve özel kurumlar kişiler hakkında\\
      büyük miktarda veri topluyor
    \begin{itemize}
      \item kredi kartı şirketleri, bankalar, alışveriş marketleri,\\
        İnternet siteleri, \ldots
      \item reklam ya da hizmet kalitesini artırma amaçlı
    \end{itemize}

    \pause
    \medskip
    \item bireyler kurumlar karşısında güçsüzleşiyor:
    \begin{itemize}
      \item kurum birey hakkında çok şey biliyor
      \item birey kurum hakkında neredeyse hiçbir şey bilmiyor
    \end{itemize}
  \end{itemize}
\end{frame}

\begin{frame}
  \frametitle{Bruce Schneier (2010)}

  \begin{columns}
    \column{.55\textwidth}
    \pgfuseimage{schneier}

    \column{.45\textwidth}
  \begin{quote}
    Facebook'un müşterisi olduğunuzu sanmak yanılgısına düşmeyin, değilsiniz
    -- siz ürünsünüz. Müşterileri, reklam verenler.
  \end{quote}
  \end{columns}

  \medskip
  \tiny{\url{http://www.information-age.com/channels/security-and-continuity/news/1290603/facebook-is-deliberately-killing-privacy-says-schneier.thtml}}
\end{frame}

\begin{frame}
  \frametitle{Sorunlar Yeni Mi?}

  \begin{itemize}
    \item toplanan verinin miktarı
    \item başka yerlere aktarılmasındaki hız
    \item verinin kalıcılığı
    \item bilginin kalitesi: ayrıntılı profil çıkartılabiliyor
  \end{itemize}
\end{frame}

\begin{frame}
  \frametitle{Verinin Kalıcılığı: Eric Schmidt (2010)}

  \begin{columns}
    \column{.55\textwidth}
    \pgfuseimage{schmidt}

    \column{.45\textwidth}
    \begin{quote}
      Gençler reşit olacakları yaşa geldiklerinde isim değiştirme hakkına sahip olmalı.
    \end{quote}

    \pause
    \begin{quote}
      Her şey bulunabilir, bilinebilir ve herkes tarafından her zaman
      kaydedilebilir olduğunda ne olacağını toplumun anladığına inanmıyorum.
    \end{quote}
  \end{columns}

  \medskip
  \tiny{\url{https://www.readwriteweb.com/archives/google_ceo_suggests_you_change_your_name_to_escape.php}}
\end{frame}

\begin{frame}
  \frametitle{Hangi Veriler Değerli?}

  \begin{itemize}
    \item isim, doğum tarihi, adres, telefon, \ldots
    \item yapılan her alışveriş
    \item telefon görüşmeleri, mesajlaşmalar, e-postalar, \ldots
    \item politik ve dini görüşler
    \item cinsel tercihler
    \item sağlık verileri
    \item gelir verileri
    \item ziyaret edilen web siteleri
    \item arama motorlarında yapılan aramalar
    \item \ldots
  \end{itemize}
\end{frame}

\begin{frame}
  \frametitle{Hangi Veriler Duyarlı?}

  \begin{itemize}
    \item veriler duyarlılık düzeylerine ayrılabilir:\\
      gizli, yakın çevre, güven, toplumsal, açık
    \begin{itemize}
      \item kişiden kişiye değişir
      \item bir bilgi kendi başına duyarlı olmayabilir,\\
        başka bilgilerle birleştiğinde kritik hale gelebilir
    \end{itemize}

    \pause
    \bigskip
    \item bütün kişisel veriler duyarlı
  \end{itemize}
\end{frame}

\begin{frame}
  \frametitle{Tehlikeler}

  \begin{itemize}
    \item verinin ilgili kişilerin aleyhine kullanılması
    \begin{itemize}
      \item kimlik hırsızlığı, şantaj, \ldots
    \end{itemize}

    \item izinli toplanmış da olsa verinin amacı dışında kullanılması
    \begin{itemize}
      \item kimlerin eriştiğinin iyi denetlenmemesi
    \end{itemize}
  \end{itemize}
\end{frame}

\begin{frame}
  \frametitle{Örnek Olay: Sosyal ağ bilgileriyle hırsızlık (2010)}

  \begin{columns}
    \column{.58\textwidth}
    \pgfuseimage{hirsiz}

    \column{.42\textwidth}
    \begin{itemize}
      \item hırsızlar, soyacakları evi\\
        Facebook gibi\\
        sosyal ağlardaki\\
        durum güncellemelerine\\
        bakarak seçiyor
    \end{itemize}
  \end{columns}

  \medskip
  \tiny{\url{http://www.theregister.co.uk/2010/09/13/social_network_burglary_gang/}}
\end{frame}

\begin{frame}
  \frametitle{Örnek Olay: Avustralya yurttaşlık kayıtları (2006)}

  \begin{columns}
    \column{.58\textwidth}
    \pgfuseimage{avustralya}

    \column{.42\textwidth}
    \begin{itemize}
      \item devlet görevlileri\\
        yurttaşlık kayıtlarını\\
        kişisel amaçlarla\\
        kullanıyorlar
      \item kimlik hırsızlığı
    \end{itemize}
  \end{columns}

  \medskip
  \tiny{\url{http://www.theregister.co.uk/2006/08/28/oz_id_database_misused/}}
\end{frame}

\begin{frame}
  \frametitle{Tehlikeler}

  \begin{itemize}
    \item hatalar ya da güvenlik açıkları nedeniyle\\
      verilerin yanlış kişilerin eline geçmesi
  \end{itemize}
\end{frame}

\begin{frame}
  \frametitle{Örnek Olay: AOL arama kayıtları (2006)}

  \begin{columns}
    \column{.6\textwidth}
    \pgfuseimage{aol}

    \column{.4\textwidth}
    \begin{itemize}
      \item AOL, yüzbinlerce\\
        kullanıcısının\\
        arama kayıtlarını\\
        yayımlıyor
    \end{itemize}
  \end{columns}

  \medskip
  \tiny{\url{http://news.cnet.com/AOLs-disturbing-glimpse-into-users-lives/2100-1030_3-6103098.html}}
\end{frame}

\begin{frame}
  \frametitle{Örnek Olay: İtalya vergi kayıtları (2008)}

  \begin{columns}
    \column{.58\textwidth}
    \pgfuseimage{italya}

    \column{.42\textwidth}
    \begin{itemize}
      \item Vergi Dairesi,\\
        bütün İtalyanların\\
        isim, adres,\\
        doğum tarihi,\\
        gelir ve vergi\\
        bilgilerini yayımlıyor
    \end{itemize}
  \end{columns}

  \medskip
  \tiny{\url{http://www.theregister.co.uk/2008/05/01/italy_publishes_tax_details/}}
\end{frame}

\begin{frame}
  \frametitle{Örnek Olay: İngiltere çocuk yardımı kayıtları (2007)}

  \begin{columns}
    \column{.4\textwidth}
    \pgfuseimage{brown}

    \column{.6\textwidth}
    \begin{itemize}
      \item Gelirler Dairesi, 25 milyon kişinin\\
        çocuk yardımı kayıtlarını içeren\\
        CD'yi postada kaybediyor
    \end{itemize}
  \end{columns}

  \medskip
  \tiny{\url{http://www.guardian.co.uk/politics/2007/nov/21/economy.uk}}
\end{frame}

\begin{frame}
  \frametitle{Örnek Olay: Türkiye BTK kayıtları (2012)}

  \begin{columns}
    \column{.55\textwidth}
    \pgfuseimage{btk}

    \column{.45\textwidth}
    \begin{itemize}
      \item bir hacker grubu,\\
        BTK bilgisayarlarından\\
        kişisel verileri\\
        alarak yayımlıyor
      \item diğer bir hacker grubu,\\
        polis bilgisayarlarından\\
        vatandaşların ihbar duyurularını\\
        alarak yayımlıyor
    \end{itemize}
  \end{columns}

  \medskip
  \tiny{\url{http://www.radikal.com.tr/Radikal.aspx?aType=RadikalDetayV3&ArticleID=1078717&CategoryID=77&Rdkref=6}}\\
  \tiny{\url{http://www.radikal.com.tr/Radikal.aspx?aType=RadikalDetayV3&ArticleID=1080108&CategoryID=77&Rdkref=6}}\\
\end{frame}

\begin{frame}
  \frametitle{Tehlikeler}

  \begin{itemize}
    \item veriyi toplayan başkalarına da verebiliyor:
    \begin{itemize}
      \item iş ortakları
      \item reklam verenler
      \item devlet
    \end{itemize}
  \end{itemize}
\end{frame}

\begin{frame}
  \frametitle{Örnek Olay}

  \begin{ornek}[Toysmart (2001)]
    \begin{itemize}
      \item Toysmart mahremiyet anlaşmasında müşteri bilgilerinin asla
        üçüncü şahıslara aktarılmayacağı sözü var
      \item iflas edince müşteri bilgilerini satışa çıkarıyor
      \item dava açılıyor $\rightarrow$ veri tabanı yok ediliyor
    \end{itemize}
  \end{ornek}
\end{frame}

\begin{frame}
  \frametitle{Örnek Olay}

  \begin{ornek}[Rebecca Schaeffer cinayeti (1989)]
    \begin{itemize}
      \item hayranı evinin kapısında öldürüyor
      \item ehliyet bilgileri Motorlu Taşıtlar Dairesi'nden satın alınabiliyor

      \pause
      \medskip
      \item \emph{Driver's Privacy Protection Act (1994)}
    \end{itemize}
  \end{ornek}

  \pause
  \begin{ornek}[Bork Yasası (1988)]
    \begin{itemize}
      \item muhafazakar yargıç Robert Bork yüksek mahkemeye aday gösteriliyor
      \item abone olduğu video şirketinden aldığı filmler sorulabiliyor

      \pause
      \medskip
      \item \emph{Video Privacy Protection Act}
    \end{itemize}
  \end{ornek}
\end{frame}

\begin{frame}
  \frametitle{Örnek Olay}

  \begin{ornek}[Google - A.B.D. Adalet Bakanlığı (2006)]
    \begin{itemize}
      \item Adalet Bakanlığı, Google'dan kişilerin yaptıkları aramalarla ilgili
        bilgi istiyor
    \end{itemize}
  \end{ornek}
\end{frame}

\begin{frame}
  \frametitle{Verilerin Birleştirilmesi}

  \begin{itemize}
    \item toplanması amaca uygun olabilir ancak birleştirilmesi zarar verebilir
  \end{itemize}

  \begin{ornek}
    \begin{itemize}
      \item kredi kartı: iş, aylık gelir, kaç yıl hangi işte çalıştığı,
        evsahipliği, kaç yıl nerede oturduğu v.b.

      \item sağlık sigortası: sağlık durumu, geçirdiği hastalıklar, ameliyatlar
        v.b.

      \item İnternet alışveriş sitesi: hangi ürünleri aldığı, hangi ürünleri
        incelediği v.b.
    \end{itemize}
  \end{ornek}
\end{frame}

\begin{frame}
  \frametitle{Örnek Olay}

  \begin{ornek}[DoubleClick - Abacus (2000)]
    \begin{itemize}
      \item reklam firması DoubleClick tüketici alışkanlıkları firması Abacus
        ile birleşmek istiyor\\
        $\rightarrow$ mahremiyet konusundaki baskı nedeniyle vazgeçiyor
    \end{itemize}
  \end{ornek}

  \pause
  \begin{ornek}[Google - DoubleClick (2008)]
    \begin{itemize}
      \item Google DoubleClick'i satın alıyor
    \end{itemize}
  \end{ornek}
\end{frame}

\subsection{Veri Madenciliği}

\begin{frame}
  \frametitle{Veri Madenciliği}

  \begin{tanim}
    \alert{veri madenciliği}

    \begin{itemize}
      \item veride örtülü kalıpların aranması
      \item bariz olmayan yeni kategoriler
    \end{itemize}
  \end{tanim}

  \pause
  \begin{itemize}
    \item karar destek sistemleri:\\
      \emph{alışveriş kartları, kredi puanlama}

    \pause
    \item toplanan bilginin kendisi ticari bir değer taşıyor
  \end{itemize}
\end{frame}

\begin{frame}
  \frametitle{Örnek Olay}

  \begin{ornek}[Total Information Awareness]
    \begin{itemize}
      \item finans, sağlık, iletişim, yolculuk, ...
      \item kuşkulu desenler: terörist?
    \end{itemize}

    \pause
    \begin{itemize}
      \item hatalı bulgular
      \item teknik personelin erişimi
      \item veritabanının kendisi hedef
      \item yöntem kayması: suçsuzluğu kanıtlama
    \end{itemize}
  \end{ornek}
\end{frame}

\section{Önlemler}

\subsection{Giriş}

\begin{frame}
  \frametitle{Önlemler}

  \begin{itemize}
    \item kişisel
    \item mesleki
    \item kurumsal
    \item yasal
  \end{itemize}
\end{frame}

\begin{frame}
  \frametitle{Kişisel Önlemler}

  \begin{itemize}
    \item kişisel verilerini vermemek / yanlış vermek
    \begin{itemize}
      \item alışveriş marketi kartlarını başkasıyla değişmek
    \end{itemize}

    \item teknolojiden yararlanmak
    \begin{itemize}
      \item şifreleme yazılımları:\\
        PGP, TrueCrypt, ...
      \item mahremiyet artırıcı yazılımlar:\\
        Tor, Privoxy, Ghostery, TrackMeNot, ...
    \end{itemize}
  \end{itemize}
\end{frame}

\begin{frame}
  \frametitle{Örnek Olay}

  \begin{ornek}[anketler]
    \begin{itemize}
      \item sorulanların \%90'dan fazlası hassas bilgileri veriyor
      \begin{itemize}
        \item evcil hayvanının adı
        \item annesinin evlenmeden önceki soyadı
      \end{itemize}
    \end{itemize}
  \end{ornek}
\end{frame}

\begin{frame}
  \frametitle{Kurumsal Önlemler}

  \begin{itemize}
    \item hizmet veren kurumlar müşterilerinin mahremiyetine özen gösterebilir
    \begin{itemize}
      \item Google arama bilgilerini 18 ay sonra anonimize ediyor;
        Microsoft Bing 6 ay sonra
      \item anonimizasyonun işe yaramadığı iddia ediliyor
      \item Startpage arama motoru IP adreslerini kaydetmiyor
      \item Facebook: "Hotel California" politikası
    \end{itemize}

    \pause
    \item "çıkmak" değil "girmek" isteğe bağlı olmalı (opt-in / opt-out)
  \end{itemize}
\end{frame}

\begin{frame}
  \frametitle{Örnek Olay}

  \begin{ornek}[Google RSS hizmeti (2007)]
    \begin{itemize}
      \item etkinliklerini seçilen tanıdıklarla paylaşma
      \item tanıdıklar listesindeki herkese açılıyor
    \end{itemize}
  \end{ornek}
\end{frame}

\begin{frame}
  \frametitle{Yasal Düzenlemeler}

  \begin{itemize}
    \item ABD'de soruna özel düzenlemeler:\\
      kredi, ehliyet, video, sağlık, ...

    \pause
    \item Avrupa'da sorunun kaynağına yönelik düzenlemeler

    \pause
    \item Türkiye'de yeni
  \end{itemize}
\end{frame}

\subsection{ABD}

\begin{frame}
  \frametitle{ABD Yasaları}

  \begin{block}{Code of Fair Information Practices (1974)}
    \begin{enumerate}
      \item varlığı gizli bir veri kaydı tutma sistemi olamaz

      \pause
      \item kişiler kendileriyle ilgili hangi bilgilerin tutulduğunu ve bu
        bilgilerin nasıl kullanıldığını öğrenebilmelidir

      \pause
      \item kişiler bir amaçla verdikleri bilgilerin izinleri alınmadan başka
        amaçlarla kullanımını önleyebilmelidir

      \pause
      \item kişiler kendileriyle ilgili bilgileri düzeltebilmelidir

      \pause
      \item kişisel veri toplayan kurumlar bu verilerin ilgili amaçlar için
        güvenilirliğini sağlamalı ve kötü kullanımını önlemelidir
    \end{enumerate}
  \end{block}
\end{frame}

\begin{frame}
  \frametitle{ABD Yasaları}

  \begin{itemize}
    \item Code of Fair Information Practices

    \begin{itemize}
      \item yalnızca kamu kurumlarını bağlıyor

      \pause
      \item kayıtların ayırdedici bir veriye göre aranabilmesi gerekir

      \pause
      \item yasayı uygulayıcı bir yetkili tanımlı değil

      \pause
      \item "sıradan kullanımlar için" veri alışverişine izin var
    \end{itemize}
  \end{itemize}
\end{frame}

\subsection{Avrupa}

\begin{frame}
  \frametitle{Avrupa Birliği}

  \begin{block}{AB Temel Haklar Bildirgesi - 8. Madde}
    \begin{enumerate}
      \item kişisel bilgilerin korunmasını isteme hakkı

      \pause
      \item amaca uygun, yasal ve adil kullanım;\\
        kendisiyle ilgili bilgilere erişme ve düzeltme hakkı

      \pause
      \item kurallara uyulmasını denetleyen bağımsız makam
    \end{enumerate}
  \end{block}
\end{frame}

\begin{frame}
  \frametitle{Avrupa Yasaları}

  \begin{block}{Data Protection Act (1984, 1998)}

    \begin{enumerate}
      \item adil ve hukuki işleme
      \item açıkça belirtilmiş ve yasal amaçlar
      \item yeterlilik, ilgililik, gerekenden fazla olmama
      \item doğruluk ve güncellik
      \item amaç yerine geldikten sonra silinme
      \item diğer kişilerin hakları
      \item yetkisiz erişim ve kayba karşı yeterli önlem
      \item yeterli koruma sağlamayan ülkelere aktarmama
    \end{enumerate}
  \end{block}
\end{frame}

\begin{frame}
  \frametitle{Örnek Olay}

  \begin{ornek}[İngiltere DNA veri bankası (2008)]
    \begin{itemize}
      \item AİHM: hüküm giymemiş kişilerin DNA bilgilerinin saklanması
        yasaya aykırı
    \end{itemize}
  \end{ornek}
\end{frame}

\begin{frame}
  \frametitle{Örnek Olay}

  \begin{ornek}[Almanya İSS kayıtları (2006)]
    \begin{itemize}
      \item Almanya Yüksek Temyiz Mahkemesi servis sağlayıcının
        isteyen kullanıcıların IP kayıtlarını silmek zorunda
        olduğuna karar veriyor
    \end{itemize}
  \end{ornek}
\end{frame}

\begin{frame}
  \frametitle{Örnek Olay}

  \begin{ornek}[A.B.D. - Avrupa veri aktarımı]
    \begin{itemize}
      \item havayolu şirketleri A.B.D.'ye uçuşlarda yolcularla ilgili 34~parça
        bilgiyi yetkililere teslim etmek zorunda
      \item A.B.D. istihbarat birimlerinin Avrupa banka kayıtlarına erişimi
        anlaşmasını Avrupa Parlamentosu bloke ediyor (2010)
    \end{itemize}
  \end{ornek}
\end{frame}

\subsection{Türkiye}

\begin{frame}
  \frametitle{Türkiye}

  \begin{itemize}
    \item Anayasa: özel ve aile yaşamına saygı bekleme hakkı (20)
    \item Anayasa: iletişim gizliliği (22)
    \item kişisel verilerin otomatik işlenmesine ilişkin uluslararası sözleşme
    \item bilgi edinme hakkı yasası
  \end{itemize}
\end{frame}

\begin{frame}
  \frametitle{Kişisel Verilerin Korunması Yasa Taslağı}

  \begin{block}{Kişisel Verilerin Korunması Yasa Taslağı}
    \begin{itemize}
      \item \emph{Kişisel Verileri Koruma Yüksek Kurulu}
    \end{itemize}
  \end{block}
\end{frame}

\begin{frame}
  \frametitle{Bilgi Edinme Hakkı Yasası}

  \begin{block}{Bilgi Edinme Hakkı Yasası (2003)}
    \begin{itemize}
      \item bilgi edinme hakkı (4)

      \pause
      \item bilgi verme yükümlülüğü (5)

      \pause
      \item itirazlar için \emph{Bilgi Edinme Değerlendirme Kurulu} (13)

      \pause
      \item sınırlamalar: devlet sırrı (16), ülkenin ekonomik çıkarları (17),
        istihbarat (18), idari ve adli soruşturmalar (19,20), özel hayatın
        gizliliği (21), haberleşmenin gizliliği (22), ticari sır (23)

      \pause
      \item ihmal, kusur ya da kasıt durumunda ceza kovuşturması (29)
    \end{itemize}
  \end{block}
\end{frame}

\section*{Kaynaklar}

\begin{frame}
  \frametitle{Kaynaklar}

  \begin{block}{Okunacak: Tavani}
    \begin{itemize}
      \item Chapter 5: \alert{Privacy and Cyberspace}
    \end{itemize}
  \end{block}
\end{frame}

\end{document}
