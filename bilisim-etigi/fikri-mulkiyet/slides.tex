% Copyright (c) 2004-2015 H. Turgut Uyar <uyar@itu.edu.tr>
%
% This work is licensed under a "Creative Commons
% Attribution-NonCommercial-ShareAlike 4.0 International License".
% For more information, please visit:
% https://creativecommons.org/licenses/by-nc-sa/4.0/

\documentclass[dvipsnames]{beamer}

\usepackage{ae}
\usepackage[T1]{fontenc}
\usepackage[turkish]{babel}
\usepackage[utf8]{inputenc}
\usepackage{eurosym}
\setbeamertemplate{navigation symbols}{}
\setbeamersize{text margin left=2em, text margin right=2em}

\mode<presentation>
{
  \usetheme{Rochester}
  \usecolortheme[named=Mahogany]{structure}
  \setbeamercovered{transparent}
}

\title{Bilişim Etiği}
\subtitle{Fikri Mülkiyet}

\author{H. Turgut Uyar}
\date{2004-2015}

\AtBeginSubsection[]
{
  \begin{frame}<beamer>
    \frametitle{Konular}
    \tableofcontents[currentsection,currentsubsection]
  \end{frame}
}

%\beamerdefaultoverlayspecification{<+->}

\theoremstyle{plain}

\pgfdeclareimage[width=2cm]{license}{../../license}

\pgfdeclareimage[height=4.5cm]{locke}{locke}
\pgfdeclareimage[height=4.5cm]{hegel}{hegel}

\pgfdeclareimage[width=6cm]{lawyer}{lawyer}
\pgfdeclareimage[width=5.7cm]{timezone}{timezone}
\pgfdeclareimage[width=5.7cm]{wow}{wow}
\pgfdeclareimage[width=5.7cm]{kwik-fit}{kwik-fit}
\pgfdeclareimage[height=6cm]{lauren}{lauren}
\pgfdeclareimage[height=6cm]{book-search}{book-search}
\pgfdeclareimage[height=6cm]{turnitin}{turnitin}
\pgfdeclareimage[height=6cm]{limewire}{limewire}
\pgfdeclareimage[height=6cm]{pirate-bay}{pirate-bay}
\pgfdeclareimage[height=6cm]{dvd-jon}{dvd-jon}
\pgfdeclareimage[width=10cm]{sklyarov}{sklyarov}
\pgfdeclareimage[height=6cm]{sony}{sony}
\pgfdeclareimage[height=6cm]{mpaa}{mpaa}
\pgfdeclareimage[height=6cm]{sarkozy}{sarkozy}
\pgfdeclareimage[width=7.5cm]{liability}{liability}
\pgfdeclareimage[height=6.3cm]{autodesk}{autodesk}
\pgfdeclareimage[height=6cm]{personaldatamining}{personaldatamining}
\pgfdeclareimage[height=6cm]{google-estimation}{google-estimation}
\pgfdeclareimage[height=5.5cm]{eolas}{eolas}
\pgfdeclareimage[height=5.5cm]{apple-samsung}{apple-samsung}

\begin{document}

\begin{frame}
  \titlepage
\end{frame}

\begin{frame}
  \frametitle{License}

  \pgfuseimage{license}\hfill
  \copyright~2004-2015 H. Turgut Uyar

  \vfill
  \begin{footnotesize}
    You are free to:
    \begin{itemize}
      \itemsep0em
      \item Share -- copy and redistribute the material in any medium or format
      \item Adapt -- remix, transform, and build upon the material
    \end{itemize}

    Under the following terms:
    \begin{itemize}
      \itemsep0em
      \item Attribution -- You must give appropriate credit, provide a link to
        the license, and indicate if changes were made.

      \item NonCommercial -- You may not use the material for commercial
        purposes.

      \item ShareAlike -- If you remix, transform, or build upon the material,
        you must distribute your contributions under the same license as the
        original.
    \end{itemize}
  \end{footnotesize}

  \begin{small}
    For more information:\\
    \url{https://creativecommons.org/licenses/by-nc-sa/4.0/}

    \smallskip
    Read the full license:\\
    \url{https://creativecommons.org/licenses/by-nc-sa/4.0/legalcode}
  \end{small}
\end{frame}

\begin{frame}
  \frametitle{Konular}
  \tableofcontents
\end{frame}

\section{Fikri Mülkiyet}

\subsection{Giriş}

\begin{frame}
  \frametitle{Mülkiyet}

  \begin{itemize}
    \item nesneler üzerinden değil ilişkiler üzerinden tanım

    \medskip
    X kişisi Y nesnesinin sahibiyse, başka kişilerin\\
    Y nesnesiyle ilişkilerini denetleyebilir.

    \medskip
    \item elle tutulur nesneler sözkonusu olunca daha kolay anlaşılıyor
  \end{itemize}
\end{frame}

\begin{frame}
  \frametitle{Fikri Eser}

  \begin{itemize}
    \item yaratıcı eserler: sanat eserleri
    \item edebiyat, müzik, sinema, resim
    \item bilgisayar programı

    \medskip
    \item işlevsel eserler: buluşlar
  \end{itemize}

  \pause
  \bigskip
  \begin{itemize}
    \item dışlayıcı değil: alınması sahibinin kullanmasını kısıtlamıyor

    \medskip
    \item (sayısal formatlar) kısıtlı kaynak değil: kolaylıkla çoğaltılabiliyor
  \end{itemize}
\end{frame}

\begin{frame}
  \frametitle{Fikri Mülkiyet}

  \begin{itemize}
    \item sahibine ne haklar verilecek?
    \smallskip
    \item fiziksel mülke sahip olmakla aynı şey değil
    \item rekabeti ve ilerlemeyi köstekler

    \pause
    \bigskip
    \item fikirlere mülkiyet hakkı verilmesi yeğlenmiyor
    \smallskip
    \item üreticinin fikrini ortaya koymasını teşvik etmek
    \item sahibine kazanç sağladıktan sonra kamu malı haline gelmesi
  \end{itemize}
\end{frame}

\begin{frame}
  \frametitle{Dışavurum}

  \begin{itemize}
    \item fikrin belli bir \alert{ifade}sine (dışavurumuna) mülkiyet

    \medskip
    \item yaratıcı fikirler somut bir ortamda ``sabitlenmeli''
    \item kitap, müzik CD'si, \ldots
    \item \alert{telif hakkı}

    \pause
    \medskip
    \item işlevsel fikirler somut biçimde ortaya konmalı
    \item makina
    \item \alert{patent}
  \end{itemize}
\end{frame}

\begin{frame}
  \frametitle{Ticari Sır}

  \begin{itemize}
    \item formül, proses, tasarım, müşteri listesi, \ldots
    \item rekabette avantaj
    \item gizli kalması için önlemler alınmalı: gizlilik anlaşmaları

    \pause
    \medskip
    \item süresi dolmaz, yayımlanması gerekmez
    \item açığa çıkarsa sırlığı kalmaz

    \medskip
    \item \alert{tersine mühendislik} yapılabilir
  \end{itemize}
\end{frame}

\subsection{Kuramlar}

\begin{frame}
  \frametitle{Emek Kuramı}

  \begin{columns}
    \column{.45\textwidth}
    \begin{center}
      \pgfuseimage{locke}

      John Locke (17.yy)
    \end{center}

    \column{.55\textwidth}
    \begin{block}{emek kuramı}
        herkes emeğini kattığı şey üzerinde\\
        doğal bir mülkiyet hakkı edinir
    \end{block}
    \begin{itemize}
      \item ``doğal hak''
      \item gerekenden fazlası alınmamalı
    \end{itemize}
  \end{columns}
\end{frame}

\begin{frame}
  \frametitle{Faydacı Kuram}

  \begin{block}{faydacı kuram}
    fikri eserlere mülkiyet hakkı tanımak topluma yarar sağlar
  \end{block}

  \begin{itemize}
    \item insanlar fikri eserlerinden kazanç elde edecek olurlarsa\\
      bunları topluma açarlar
  \end{itemize}
\end{frame}

\begin{frame}
  \frametitle{Kişilik Kuramı}

  \begin{columns}
    \column{.45\textwidth}
    \begin{center}
      \pgfuseimage{hegel}

      Hegel (19.yy)
    \end{center}

    \column{.55\textwidth}
    \begin{block}{kişilik kuramı}
      bir fikri bir eser yaratıcısının\\
      kişiliğinin bir uzantısıdır
    \end{block}

    \begin{itemize}
      \item yaratıcısı nasıl kullanılacağını\\
        denetleyebilmelidir
    \end{itemize}
  \end{columns}
\end{frame}

\begin{frame}
  \frametitle{Yazılım Mülkiyeti}

  \begin{itemize}
    \item yazılımların lisanslarına aykırı olarak çoğaltılması\\
      neden yanlış?

    \pause
    \medskip
    \item emek kuramı açısından

    \pause
    \item faydacı kuram açısından

    \pause
    \item kişilik kuramı açısından

    \pause
    \item toplumsal sözleşme açısından
  \end{itemize}
\end{frame}

\section{Telif Hakları}

\subsection{Giriş}

\begin{frame}
  \frametitle{Telif Hakkı}

  \begin{itemize}
    \item telif hakkı bir fikrin \alert{ifadesine} verilir
    \item yazılım: algoritma fikirdir, program ifadedir

    \pause
    \medskip
    \item özgün olmalı
    \item işlevsel olmamalı
    \item somut bir ortamda sabitlenmeli

    \pause
    \medskip
    \item farklı kişiler birbirlerinden bağımsız olarak aynı ifadeyi\\
      ortaya koymuş olabilirler
  \end{itemize}
\end{frame}

\begin{frame}
  \frametitle{Telif Hakkının Kapsamı}

  \begin{itemize}
    \item çoğaltma
    \item dağıtma
    \item yeni eserler türetme (örneğin çeviri, romanın filme çekilmesi)
    \item temsil (örneğin tiyatro oyunları)
    \item teşhir (örneğin resimler)
  \end{itemize}
\end{frame}

\begin{frame}
  \frametitle{Örnek: E-posta iletisi telifi}

  \begin{columns}
    \column{.5\textwidth}
    \begin{center}
      \pgfuseimage{lawyer}
    \end{center}

    \column{.5\textwidth}
    \begin{itemize}
      \item bir e-posta iletisi,\\
        yazarının izni alınmadan\/
        bir listeye gönderiliyor
      \item yazarı, telif hakkının\\
        çiğnendiği gerekçesiyle\\
        dava açıyor
      \item mahkeme, iletinin\\
        yaratıcı eser olmadığına\\
        karar veriyor (2011)
    \end{itemize}
  \end{columns}

  \medskip
  \tiny{\url{http://www.theregister.co.uk/2011/04/12/email_not_creative_enough_for_copyright_protection/}}\\
\end{frame}

\begin{frame}
  \frametitle{Örnek: Zaman dilimi veri tabanı}

  \begin{columns}
    \column{.45\textwidth}
    \begin{center}
      \pgfuseimage{timezone}
    \end{center}

    \column{.55\textwidth}
    \begin{itemize}
      \item bazı bilgisayarlarda kullanılan\\
        zaman dilimi veri tabanının\\
        bir kısmında Astrolabe firmasının\\
        atlasından yararlanılmış
      \item Astrolabe, veri tabanının\\
        kullanımını engellemek için\\
        dava açıyor
      \item daha sonra şikayetini\\
        geri alıyor (2012)
      \item tarihsel olgular üzerinde\\
        telif hakkı olmaz
    \end{itemize}
  \end{columns}

  \medskip
  \tiny{\url{https://www.eff.org/press/releases/eff-wins-protection-time-zone-database}}\\
\end{frame}

\begin{frame}
  \frametitle{Örnek: Blizzard - MDY}

  \begin{columns}
    \column{.45\textwidth}
    \begin{center}
      \pgfuseimage{wow}
    \end{center}

    \column{.55\textwidth}
    \begin{itemize}
      \item MDY, Blizzard'ın\\
        World of Warcraft oyunu için ``bot'' satıyor
      \item Blizzard dava açıyor:\\
        ``program belleğe kopyalanıyor''
      \item mahkeme Blizzard'ı haklı buluyor\\
        (2008)
    \end{itemize}
  \end{columns}

  \medskip
  \tiny{\url{http://news.bbc.co.uk/2/hi/technology/7314353.stm}}\\
  \tiny{\url{http://virtuallyblind.com/category/lawsuits/mdy-v-blizzard/}}\\
\end{frame}

\begin{frame}
  \frametitle{Örnek: PRS - Kwik-Fit}

  \begin{columns}
    \column{.45\textwidth}
    \begin{center}
      \pgfuseimage{kwik-fit}
    \end{center}

    \column{.55\textwidth}
    \begin{itemize}
      \item Performing Rights Society,\\
        müzik endüstrisinin fikri mülkiyet\\
        haklarını koruyan bir örgüt
      \item bir araba tamir firmasına,\\
        çalışanları radyo dinledikleri için\\
        dava açıyor: ``yayın'' (2007)
      \item bir market çalışanına\\
        işyerinde şarkı söylediği için\\
        dava açıyor: ``temsil`` (2009)
    \end{itemize}
  \end{columns}

  \medskip
  \tiny{\url{http://news.bbc.co.uk/2/hi/uk_news/scotland/edinburgh_and_east/7029892.stm}}\\
  \tiny{\url{http://news.bbc.co.uk/2/hi/uk_news/scotland/tayside_and_central/8317952.stm}}\\
\end{frame}

\begin{frame}
  \frametitle{Yazılım Telif Hakkı}

  \begin{itemize}
    \item telif hakkı bir yazılımı işlevinin taklit edilmesine karşı korumaz
  \end{itemize}

  \begin{exampleblock}{örnek: Lotus 123 - Borland (1995)}
    \begin{itemize}
      \item çizelgeleme programının görünüşü ve işlevinin benzerliği
      \item programın görünümü telif korumasına girer mi?
      \item mahkeme: ''evet``, temyiz: ''hayır``
    \end{itemize}
  \end{exampleblock}

  \pause
  \begin{exampleblock}{örnek: Apple - Microsoft/HP}
    \begin{itemize}
      \item masaüstü arayüzü, ikonlar, \ldots
      \item mahkeme: ''video düğmeleri ya da araba göstergelerine benzer``
    \end{itemize}
  \end{exampleblock}
\end{frame}

\subsection{Yasalar}

\begin{frame}
  \frametitle{Uluslararası Anlaşmalar}

  \begin{itemize}
    \item Bern Anlaşması (1887)
    \item TRIPS: Trade-Related Aspects of Intellectual Property Rights (1995)

    \medskip
    \item WIPO Copyright Treaty (2002)
    \item World Intellectual Property Organization
  \end{itemize}
\end{frame}

\begin{frame}
  \frametitle{Telif Yasaları}

  \begin{itemize}
    \item Türkiye: Fikir ve Sanat Eserleri Kanunu (1995)
    \item ABD: Digital Millenium Copyright Act (1998)

    \medskip
    \item koruma süresi: eser sahibinin ölümünden sonra 70~yıl
    \item ücretle yaptırılan işlerde 95~yıl

    \pause
    \medskip
    \item telif hakkı kendiliğinden oluşur, bir yere tescili gerekmez
  \end{itemize}
\end{frame}
%
% TODO: orphaned works

\begin{frame}
  \frametitle{İlkeler}

  \begin{itemize}
    \item \alert{makul kullanım} (fair use):\\
      bazı durumlarda kullanım için izin gerekmez
    \item kullanım amacı: eleştiri, haber, eğitim, araştırma
    \item eserin doğası: kurgu - kurgu değil
    \item kullanımın boyutu: parçası, tamamı
    \item eserin satışına etkisi

    \pause
    \bigskip
    \item \alert{ilk satış} (first sale):\\
      ilk satışla birlikte telif sahibinin kopya üzerinde hakkı kalmaz
  \end{itemize}
\end{frame}

\begin{frame}
  \frametitle{Örnek: Ralph Lauren reklam kampanyası}

  \begin{columns}
    \column{.45\textwidth}
    \begin{center}
      \pgfuseimage{lauren}
    \end{center}

    \column{.55\textwidth}
    \begin{itemize}
      \item Ralph Lauren giyim firmasının\\
        bir reklam fotoğrafı eleştiriliyor
      \item firma, fotoğrafın kullanılmasını\\
        engellemeye çalışıyor (2009)
    \end{itemize}
  \end{columns}

  \medskip
  \tiny{\url{http://boingboing.net/2009/10/06/the-criticism-that-r.html}}\\
\end{frame}

\begin{frame}
  \frametitle{Örnek: Google Book Search}

  \begin{columns}
    \column{.45\textwidth}
    \begin{center}
      \pgfuseimage{book-search}
    \end{center}

    \column{.55\textwidth}
    \begin{itemize}
      \item Google basılı kitapları tarayarak\\
        arama sonuçlarında gösteriyor
      \item Yazarlar Sendikası dava açıyor,\\
        Google makul kullanıma\\
        girdiğini iddia ediyor (2005)
      \item 125~milyon~\$ anlaşma (2008)
      \item Fransa'da 300~bin~\euro~ceza (2009)
    \end{itemize}
  \end{columns}

  \medskip
  \tiny{\url{http://www.theregister.co.uk/2008/10/28/google_settles_book_suit/}}\\
  \tiny{\url{http://news.bbc.co.uk/2/hi/technology/8420876.stm}}\\
\end{frame}

\begin{frame}
  \frametitle{Örnek: Turnitin}

  \begin{columns}
    \column{.5\textwidth}
    \begin{center}
      \pgfuseimage{turnitin}
    \end{center}

    \column{.5\textwidth}
    \begin{itemize}
      \item kopyacılığı önleme hizmeti
      \item üniversiteler parayla abone oluyor
      \item ödevleri birbirleriyle,\\
        eskiden gönderilen ödevlerle\\
        ve İnternet kaynaklarıyla\\
        karşılaştırıyor
      \item öğrenciler Turnitin'i\\
        telif hakkı çiğnenmesi\\
        gerekçesiyle dava ediyor
      \item mahkeme, makul kullanıma\\
        girdiğine karar veriyor (2009)
    \end{itemize}
  \end{columns}

  \medskip
  \tiny{\url{http://www.wired.com/threatlevel/2009/04/fair-use-bolste/}}\\
\end{frame}

% \begin{frame}
%   \frametitle{Kişisel Kullanım}
%
%   \begin{itemize}
%     \item Fikir ve Sanat Eserleri Kanunu'na göre:
%     \begin{itemize}
%       \item kar amacı güdülmeksizin kişisel kullanım
%       \item hukuki yollardan edinme
%       \item yedekleme kopyası
%     \end{itemize}
%   \end{itemize}
% \end{frame}

\begin{frame}
  \frametitle{Yasal Bağışıklık}

  \begin{itemize}
    \item bazı kurumlar kullanıcıların yaptıkları yüzünden\\
      telif yasasını çiğnemekten dava edilemez

    \medskip
    \item servis sağlayıcılar
    \item arama motorları
    \item Internet Archive
  \end{itemize}
\end{frame}

\subsection{DRM}

\begin{frame}
  \frametitle{Dosya Paylaşımı}

  \begin{itemize}
    \item yaygın dosya paylaşımı, telif haklarının çiğnenmesi sorununu\\
      çok artırdı
    \item merkezi ağlar: Napster, Kazaa, \ldots
    \item dağıtık ağlar: BitTorrent (arama motoru gerekiyor)
    \item dosya barındırma servisleri: Rapidshare, Megaupload, \ldots
  \end{itemize}
\end{frame}

\begin{frame}
  \frametitle{Dosya Paylaşımı Tartışmaları}

  \begin{itemize}
    \item dosya paylaşımının yasal kullanımları da var
    \item mahkemeler servislerin yaygın telif hakkı ihlallerini\\
      önleme yükümlülüğü olduğuna karar veriyor
    \item istenen tazminat miktarları gerçekçi mi?
    \item elde edilen tazminat gelirleri nasıl paylaşılacak?
  \end{itemize}
\end{frame}

\begin{frame}
  \frametitle{Örnek: Betamax}

  \begin{itemize}
    \item Universal film şirketi, Sony'ye Betamax video kaydedici\\
      nedeniyle dava açıyor (1970):\\
      ''bu aygıt telif haklarının çiğnenmesinde kullanılabilir``
    \item mahkeme: ''yasal kullanım şekilleri de var''
  \end{itemize}
\end{frame}

\begin{frame}
  \frametitle{Örnek: The Pirate Bay}

  \begin{columns}
    \column{.45\textwidth}
    \begin{center}
      \pgfuseimage{pirate-bay}
    \end{center}

    \column{.55\textwidth}
    \begin{itemize}
      \item TPB: BitTorrent arama sitesi
      \item kurucularına 4-10~ay hapis\\
        ve 6.5~milyon~\$ tazminat cezası\\
        (2009-2010)

      \pause
      \medskip
      \item yaptıkları Google'dan farklı mı?
    \end{itemize}
  \end{columns}

  \medskip
  \tiny{\url{http://www.bbc.co.uk/news/technology-11847200}}\\
  \tiny{\url{https://torrentfreak.com/google-defends-hotfile-and-megaupload-in-court-120319/}}\\
\end{frame}

\begin{frame}
  \frametitle{Örnek: LimeWire}

  \begin{columns}
    \column{.45\textwidth}
    \begin{center}
      \pgfuseimage{limewire}
    \end{center}

    \column{.55\textwidth}
    \begin{itemize}
      \item LimeWire paylaşım ağı\\
        davasında plak şirketleri\\
        75~trilyon~\$ tazminat istiyor
      \item yargıç miktarı ``saçma'' buluyor\\
        (2011)
    \end{itemize}
  \end{columns}

  \medskip
  \tiny{\url{http://www.theregister.co.uk/2011/03/24/judge_slaps_music_biz/}}\\
\end{frame}

\begin{frame}
  \frametitle{DRM}

  \begin{itemize}
    \item Digital Rights Management
    \item çoğaltma ve dağıtma kurallarını teknoloji yardımıyla uygulama

    \medskip
    \item DVD'lerde bölge koruması
    \item CD'lerde kopyalama koruması
    \item sayısal dosyalarda başka ortama aktarmayı engelleme

    \pause
    \medskip
    \item yeni telif hakkı yasalarında ``atlatmayı önleme''\\
      (anticircumvention) maddeleriyle destekleniyor
  \end{itemize}
\end{frame}

\begin{frame}
  \frametitle{DRM Tartışmaları}

  \begin{itemize}
    \item tüketici hakları açısından
    \item tersine mühendislik açısından
    \item teknik açıdan, DRM gerçekten işe yarıyor mu?
  \end{itemize}
\end{frame}

\begin{frame}
  \frametitle{Örnek: Adobe - Sklyarov}

  \begin{center}
    \pgfuseimage{sklyarov}
  \end{center}

  \medskip
  \tiny{\url{http://www.infotoday.com/it/nov01/ardito.htm}}\\
\end{frame}

\begin{frame}
  \frametitle{Örnek: Adobe - Sklyarov}

  \begin{itemize}
    \item Sklyarov, PDF formatındaki kitaplardan şifre korumasını\\
      kaldıran bir yazılım geliştiriyor
    \item çalıştığı şirket Elcomsoft bu yazılımı satıyor
    \item Sklyarov ABD'ye gelişinde tutuklanıyor (2001)
    \item Adobe ve ABD Adalet Bakanlığı, Sklyarov ve Elcomsoft'a\\
      dava açıyor
    \item kamuoyu tepkisi nedeniyle Sklyarov davası geri çekiliyor
    \item Elcomsoft davası beraatle sonuçlanıyor (2002)

    \pause
    \medskip
    \item makul kullanım? (yedekleme kopyası)
    \item ilk satış?
  \end{itemize}
\end{frame}

\begin{frame}
  \frametitle{Örnek: Jon Johansen}

  \begin{columns}
    \column{.45\textwidth}
    \begin{center}
      \pgfuseimage{dvd-jon}
    \end{center}

    \column{.55\textwidth}
    \begin{itemize}
      \item Jon Johansen, Linux'da\\
        DVD seyredebilmek için\\
        koruma şifresini kıran\\
        bir yazılım geliştiriyor
      \item Paramount, Universal, MGM\\
        dava açıyor
      \item Johansen beraat ediyor (2003)

      \pause
      \medskip
      \item Johansen - Apple: iTunes'dan\\
        müzik satın alabilme (2005)
    \end{itemize}
  \end{columns}

  \medskip
  \tiny{\url{http://news.cnet.com/Norway-piracy-case-brings-activists-hope/2100-1025_3-979769.html}}\\
  \tiny{\url{http://news.cnet.com/DVD-Jon-reopens-iTunes-back-door/2100-1027_3-5630703.html}}\\
\end{frame}

\begin{frame}
  \frametitle{Örnek: Sony müzik CD'leri}

  \begin{columns}
    \column{.45\textwidth}
    \begin{center}
      \pgfuseimage{sony}
    \end{center}

    \column{.55\textwidth}
    \begin{itemize}
      \item Sony müzik CD'leri\\
        kullanıcıya haber vermeden\\
        bir kopya koruma yazılımı\\
        kuruyor
      \item Sony özür diliyor,\\
        CD'leri geri topluyor (2005)

      \pause
      \item kopya koruma yazılımı,\\
        açık kaynaklı projelerden\\
        çalıntı çıkıyor
    \end{itemize}
  \end{columns}

  \medskip
  \tiny{\url{http://news.bbc.co.uk/2/hi/technology/4456970.stm}}\\
  \tiny{\url{http://www.theregister.co.uk/2005/11/18/sony_copyright_infringement/}}\\
\end{frame}

\begin{frame}
  \frametitle{Örnek: MPAA}

  \begin{columns}
    \column{.5\textwidth}
    \begin{center}
      \pgfuseimage{mpaa}
    \end{center}

    \column{.5\textwidth}
    \begin{itemize}
      \item Film Yapımcıları Birliği,\\
        "This Film Is Not Yet Rated"\\
        filmini izinsiz çoğaltarak\\
        çalışanlarına dağıtıyor (2004)
    \end{itemize}
  \end{columns}

  \medskip
  \tiny{\url{https://www.eff.org/deeplinks/2006/01/mpaa-copying-movies-ok-our-families-not-yours}}\\
\end{frame}

\begin{frame}
  \frametitle{Örnek: Nicolas Sarkozy}

  \begin{columns}
    \column{.46\textwidth}
    \begin{center}
      \pgfuseimage{sarkozy}
    \end{center}

    \column{.54\textwidth}
    \begin{itemize}
      \item Sarkozy'nin partisi\\
        MGMT grubunun bir şarkısını\\
        seçim kampanyasında\\
        izin almadan kullanıyor (2009)
      \item Başkanlık Ofisi,\\
        Sarkozy ile ilgili bir belgeseli\\
        kopya DVD'lere basıyor (2009)
    \end{itemize}
  \end{columns}

  \medskip
  \tiny{\url{http://www.huffingtonpost.com/2009/10/08/nicolas-sarkozy-french-pr_n_313723.html}}\\
\end{frame}

\subsection{Lisans Anlaşmaları}

\begin{frame}
  \frametitle{Ürün - Hizmet}

  \begin{itemize}
    \item yazılım ürün mü, hizmet mi?
    \item kitle satışı $\rightarrow$ ürün
    \item kişisel satış $\rightarrow$ hizmet

    \pause
    \bigskip
    \item kitle satışıysa: ürünün güvenliğini sağlamalı $\rightarrow$ sıkı sorumluluk
    \item risk maliyetleri satışlara yayılabilir

    \medskip
    \item kişisel satışsa: ticari dolaşıma sokmuyor $\rightarrow$ ihmal
    \item riski müşterilere yayamaz
  \end{itemize}
\end{frame}

\begin{frame}
  \frametitle{Özel Mülkiyet Modeli}

  \begin{itemize}
    \item kaynak kodu: ticari sır
    \item derlenmiş kodun çoğaltılması ve dağıtılması: telif hakkı
    \item kodun kullanım şekli: lisans anlaşması

    \pause
    \medskip
    \item tüketicinin aldığı şey program değil, programı kullanma \alert{izni}
  \end{itemize}
\end{frame}

\begin{frame}
  \frametitle{Son Kullanıcı Anlaşmaları}

  \begin{itemize}
    \item tüketici lisansı kabul etmiyorsa? (önyüklü yazılımlar)
    \item etkinleştirme, donanım değişikliği
    \item tersine mühendislik: yok ya da yasanın izin verdiği kadar
    \item garanti yokluğu
    \item dava edilemezlik
  \end{itemize}
\end{frame}

\begin{frame}
  \frametitle{Özgür/Açık Yazılım Modeli}

  \begin{itemize}
    \item kendine göre kişiselleştirebilme olanağı
    \item daha hızlı güncellenme
    \item üreticinin yaşayabileceği sıkıntılardan etkilenmeme
  \end{itemize}
\end{frame}

\begin{frame}
  \frametitle{GNU General Public License - GPL}

  \begin{itemize}
    \item kullanım: kısıtlama yok
    \item dağıtım: kısıtlama yok (satılması dahil)
    \item değiştirme: kısıtlama yok
    \item değiştirilenin dağıtılması: kaynak kod verilmeli
    \item garanti yok
    \item dava edilemez
  \end{itemize}
\end{frame}

\begin{frame}
  \frametitle{Diğer Özgür/Açık Lisanslar}

  \begin{itemize}
    \item BSD: değiştirilen kodda istenen lisans uygulanabilir
    \item Lesser GPL, Apache, Mozilla, \ldots
    \item dual licensing: MySQL, Qt

    \pause
    \medskip
    \item belgeleme için: GNU Free Documentation License
    \item yaratıcı eserler için: Creative Commons
  \end{itemize}
\end{frame}

\begin{frame}
  \frametitle{Open Source Initiative (OSI)}

  \begin{itemize}
    \item dağıtım özgürlüğü
    \item kaynak kodun açıklığı
    \item değişikliklere izin
    \item özgün kaynak kodunun bütünlüğü
    \item kişi ve gruplara karşı ayrımcılık yapılmaması
    \item iş alanlarına karşı ayrımcılık yapılmaması
    \item lisansın dağıtımı
    \item lisansın ürüne özel olmaması
    \item lisansın başka yazılımları kısıtlamaması
    \item lisansın teknolojiden bağımsız olması
  \end{itemize}
\end{frame}

\begin{frame}
  \frametitle{Örnek: İlk satış}

  \begin{columns}
    \column{.4\textwidth}
    \begin{center}
      \pgfuseimage{autodesk}
    \end{center}

    \column{.6\textwidth}
    \begin{itemize}
      \item ABD'de bir mahkeme,\\
        yazılımın lisanslanmadığına,\\
        satın alındığına karar veriyor (2009)
    \end{itemize}
  \end{columns}

  \medskip
  \tiny{\url{http://www.out-law.com/page-10421}}\\
\end{frame}

\begin{frame}
  \frametitle{Örnek: Dava edilemezlik}

  \begin{center}
    \pgfuseimage{liability}
  \end{center}

  \begin{itemize}
    \item İngiltere'de Yüksek Mahkeme, yazılımın kötü performans\\
      nedeniyle dava edilememesini kabul etmiyor (2010)
  \end{itemize}

  \medskip
  \tiny{\url{http://www.channelregister.co.uk/2010/05/12/red_sky_liability_ruling/}}\\
\end{frame}

\section{Patentler}

\subsection{Giriş}

\begin{frame}
  \frametitle{Patentler}

  \begin{itemize}
    \item patent koruması \alert{buluşlar} içindir

    \medskip
    \item \emph{yeni}: tekniğin bilinen durumunun aşılması
    \item \emph{yararlı}: sanayiye uygulanabilirlik
    \item \emph{bariz değil}

    \pause
    \medskip
    \item başkası bağımsız olarak bulsa da kullanamaz
    \item çiğnenmesinden müşterilere de dava açılabiliyor
  \end{itemize}
\end{frame}

\begin{frame}
  \frametitle{Patent Korumasının Kapsamı}

  \begin{itemize}
    \item üretme
    \item kullanma
    \item satma
    \item başkalarına bu izinleri verme (lisanslama)
    \item \alert{yasal tekel}

    \pause
    \medskip
    \item koruma süresi: 20~yıl civarı
    \item koruma bölgesi
  \end{itemize}
\end{frame}

\begin{frame}
  \frametitle{Patent Zorlukları}

  \begin{itemize}
    \item patent ofisine tescil ettirilmeli
    \item alınması çok masraflı

    \pause
    \medskip
    \item patent başvurularının değerlendirilmesi zor
    \item önceki başvurulara bakılıyor

    \pause
    \medskip
    \item çalışılan alanda ne patentler olduğunu bulmak zor

    \pause
    \medskip
    \item patent ofisinin patenti vermiş olması,\\
      mahkemenin mutlaka kabul edeceği anlamına gelmiyor
  \end{itemize}
\end{frame}

\subsection{Yazılım Patentleri}

\begin{frame}
  \frametitle{Yazılım Patentleri}

  \begin{itemize}
    \item ABD'de 1981'e kadar yazılım patenti başvuruları reddediliyor
    \item bir davada kabul edilmesinden sonra\\
      yazılımlara patent verilmeye başlanıyor

    \pause
    \medskip
    \item yazılımlara patent verilip verilmeyeceği çoğu ülkede tartışılıyor
    \item ``yazılım matematiktir''
  \end{itemize}
\end{frame}

\begin{frame}
  \frametitle{Örnek: Benson}

  \begin{itemize}
    \item Benson, BCD~sayıları ikili sayılara çeviren algoritma için\\
      patent başvurusunda bulunuyor
    \item patent ofisi reddediyor, Benson dava açıyor
    \item mahkeme, bu algoritmanın patentlenemeyeceğine\\
      karar veriyor (1972)
  \end{itemize}
\end{frame}

\begin{frame}
  \frametitle{Patent Sorunları}

  \begin{itemize}
    \item ölçütleri sağlamadığı halde verilen patentler

    \medskip
    \item patentin rakipleri engellemeye amacıyla kullanılması
    \item karşılıklı olarak birbirlerini engellemeleri teknolojiyi tıkayabilir:\\
      \emph{patent havuzları}
    \item kritik patentler için ``adil, makul ve ayrımsız'' lisanslama\\
      (Fair, Reasonable and Non-Discriminatory - FRAND)

    \medskip
    \item yalnızca dava açma amacıyla patent alınması (patent trolling)

    \medskip
    \item patentin yaygınlaşana kadar gizli tutulması ya da uygulanmayıp\\
      sonradan uygulanması
  \end{itemize}
\end{frame}

\begin{frame}
  \frametitle{Patent Sorunları}

  \begin{itemize}
    \item patentler karşılıklı silah haline gelmiş durumda
    \item pek çok firma savunma amaçlı patent alıyor

    \medskip
    \item sistem yaratıcılığı destekleme amacından uzaklaştı
    \item tam tersine, yeni başlayan ve küçük şirketlerin aleyhine işliyor
  \end{itemize}
\end{frame}

\begin{frame}
  \frametitle{Örnek: Amazon - Tek tıkla alışveriş}

  \begin{itemize}
    \item Amazon tek tıkla alışveriş için patent alıyor (1999)
    \item Barnes and Noble'a dava açıyor
    \item patent ofisi tekrar incelemeye alıyor, bazı yönlerini reddediyor\\
      (2007)
    \item değiştirilmiş patent başvurusunu kabul ediyor (2010)

    \pause
    \medskip
    \item Bilski davası: Temyiz Mahkemesi iş yöntemi patentlerini\\
      zorlaştıran bir karar veriyor (2008)
  \end{itemize}

  \medskip
  \tiny{\url{http://papers.ssrn.com/sol3/papers.cfm?abstract_id=1725009}}\\
\end{frame}

\begin{frame}
  \frametitle{Örnek: Microsoft - Kişisel veri madenciliği}

  \begin{columns}
    \column{.35\textwidth}
    \begin{center}
      \pgfuseimage{personaldatamining}
    \end{center}

    \column{.65\textwidth}
    \begin{itemize}
      \item Microsoft: kişisel veri madenciliği (2010)
      \item Microsoft AOL'den 800~patent\\
        satın alıyor (2012)
    \end{itemize}
  \end{columns}

  \medskip
  \tiny{\url{http://techflash.com/seattle/2010/02/gates_ozzie_other_microsoft_execs_patent_personal_data_mining.html}}\\
  \tiny{\url{http://www.theregister.co.uk/2012/04/09/aol_microsoft_patent_deal/}}\\
\end{frame}

\begin{frame}
  \frametitle{Örnek: Google - Gönderi süresi tahmini}

  \begin{columns}
    \column{.35\textwidth}
    \begin{center}
      \pgfuseimage{google-estimation}
    \end{center}

    \column{.65\textwidth}
    \begin{itemize}
      \item Google: gönderi ne sürede gelecek?\\
        (2011)
    \end{itemize}
  \end{columns}

  \medskip
  \tiny{\url{http://www.theregister.co.uk/2011/08/12/google_customer_notification_patent/}}\\
\end{frame}

\begin{frame}
  \frametitle{Örnek: Apple - Samsung - Google}

  \begin{columns}
    \column{.45\textwidth}
    \begin{center}
      \pgfuseimage{apple-samsung}
    \end{center}

    \column{.55\textwidth}
    \begin{itemize}
      \item Apple, bir Samsung tabletin\\
        Almanya'da satışını engelliyor\\
        (2011)
      \item Motorola patenti, Almanya'da\\
        Apple kullanıcılarının\\
        ``push email'' kullanmasını engelliyor (2012)
      \item HTC, Google'ın verdiği\\
        patentlerle Apple'a dava açıyor\\
        (2011)
    \end{itemize}
  \end{columns}

  \medskip
  \tiny{\url{https://www.pcworld.com/article/245493/apple_to_samsung_dont_make_thin_or_rectangular_tablets_or_smartphones.html}}\\
  \tiny{\url{http://www.theregister.co.uk/2012/02/24/apple_patent_motorola/}}\\
  \tiny{\url{http://www.bloomberg.com/news/2011-09-07/htc-sues-apple-alleging-infringement-of-four-u-s-patents.html}}\\
\end{frame}

% TODO: Apple-Samsung davasını ekle

\begin{frame}
  \frametitle{Örnek: Microsoft - Eolas}

  \begin{columns}
    \column{.52\textwidth}
    \begin{center}
      \pgfuseimage{eolas}
    \end{center}

    \column{.48\textwidth}
    \begin{itemize}
      \item Eolas patenti:\\
        tarayıcı içinden\\
        uygulama çalıştırma
      \item Microsoft'a dava açıyor (1999)
      \item mahkeme dışı anlaşma (2007)

      \pause
      \medskip
      \item Eolas daha sonra\\
        Apple ve Google'a\\
        dava açıyor (2010)
      \item davayı kaybediyor (2012)
    \end{itemize}
  \end{columns}

  \medskip
  \tiny{\url{http://www.theregister.co.uk/2007/08/31/microsoft_eolas_settlement/}}\\
  \tiny{\url{http://www.wired.com/threatlevel/2012/02/interactive-web-patent/}}\\
\end{frame}

\begin{frame}
  \frametitle{Örnek: Compuserve - GIF resim formatı}

  \begin{itemize}
    \item Compuserve patenti: GIF'de kullanılan sıkıştırma algoritması
    \item patent korumasını yıllarca uygulamıyor
    \item GIF kullanımı yaygınlaşınca Compuserve bazı web sitelerine\\
      dava açacağını duyuruyor (1994)

    \pause
    \medskip
    \item alternatif olarak geliştirilen PNG resim formatı GIF'in yerini alıyor
  \end{itemize}
\end{frame}

\section*{Kaynaklar}

\begin{frame}
  \frametitle{Kaynaklar}

  \begin{block}{Okunacak: Tavani}
    \begin{itemize}
      \item Chapter 8: \alert{Intellectual Property Disputes in Cyberspace}
    \end{itemize}
  \end{block}
\end{frame}

\end{document}
