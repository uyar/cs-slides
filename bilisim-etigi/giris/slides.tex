% Copyright (c) 2004-2015 H. Turgut Uyar <uyar@itu.edu.tr>
%
% This work is licensed under a "Creative Commons
% Attribution-NonCommercial-ShareAlike 4.0 International License".
% For more information, please visit:
% https://creativecommons.org/licenses/by-nc-sa/4.0/

\documentclass[dvipsnames]{beamer}

\usepackage{ae}
\usepackage[T1]{fontenc}
\usepackage[utf8]{inputenc}
\usepackage[turkish]{babel}
\setbeamertemplate{navigation symbols}{}
\setbeamersize{text margin left=2em, text margin right=2em}

\mode<presentation>
{
  \usetheme{Rochester}
  \usecolortheme[named=Mahogany]{structure}
  \setbeamercovered{transparent}
}

\title{Bilişim Etiği}
\subtitle{Giriş}

\author{H. Turgut Uyar}
\date{2004-2015}

\AtBeginSubsection[]
{
  \begin{frame}<beamer>
    \frametitle{Konular}
    \tableofcontents[currentsection,currentsubsection]
  \end{frame}
}

%\beamerdefaultoverlayspecification{<+->}

\theoremstyle{plain}

\pgfdeclareimage[height=1cm]{license}{../license}

\pgfdeclareimage[width=8cm]{usatoday}{usatoday}
\pgfdeclareimage[width=7cm]{bbc}{bbc}
\pgfdeclareimage[width=8cm]{msnbc}{msnbc}
\pgfdeclareimage[width=7cm]{register}{register}

\begin{document}

\begin{frame}
  \titlepage
\end{frame}

\begin{frame}
  \frametitle{Lisans}

  \pgfuseimage{license}\hfill
  \copyright~2004-2015 H. Turgut Uyar

  \vfill
  \begin{footnotesize}
    You are free to:
    \begin{itemize}
      \itemsep0em
      \item Share -- copy and redistribute the material in any medium or format
      \item Adapt -- remix, transform, and build upon the material
    \end{itemize}

    Under the following terms:
    \begin{itemize}
      \itemsep0em
      \item Attribution -- You must give appropriate credit, provide a link to
        the license, and indicate if changes were made.

      \item NonCommercial -- You may not use the material for commercial
        purposes.

      \item ShareAlike -- If you remix, transform, or build upon the material,
        you must distribute your contributions under the same license as the
        original.
    \end{itemize}
  \end{footnotesize}

  \begin{small}
    For more information:\\
    \url{https://creativecommons.org/licenses/by-nc-sa/4.0/}

    \smallskip
    Read the full license:\\
    \url{https://creativecommons.org/licenses/by-nc-sa/4.0/legalcode}
  \end{small}
\end{frame}

\begin{frame}
  \frametitle{Konular}
  \tableofcontents
\end{frame}

\section{Teknolojinin Etkileri}

\begin{frame}
  \frametitle{Teknolojinin Etkileri}

  \begin{itemize}
    \item her yeni teknolojinin getirdikleri ve götürdükleri oluyor
    \item değişik boyutlarda
  \end{itemize}

  \begin{exampleblock}{örnekler}
    \begin{itemize}
      \item nükleer enerji
      \item akıllı karayolları
      \item aerosol kutuları
    \end{itemize}
  \end{exampleblock}
\end{frame}

\begin{frame}
  \frametitle{Politika Boşlukları}

  \begin{itemize}
    \item her ilerlemede politika boşlukları oluşuyor

    \bigskip
    \item yeni ahlaki değerlendirme gerek
    \item sakıncalar azaltılmalı
    \item gerekirse o teknolojiden vazgeçilebilir
  \end{itemize}
\end{frame}

\begin{frame}
  \frametitle{Politika Boşlukları}

  \begin{itemize}
    \item oluşan boşluklar nasıl doldurulacak?

    \bigskip
    \item yasalar
    \item kurum politikaları
    \item kişisel ilkeler
  \end{itemize}
\end{frame}

\section{Bilişim Teknolojisi}

\begin{frame}
  \frametitle{Bilişim Teknolojisinin Etkileri}

  \begin{columns}
    \column{.5\textwidth}
    \begin{block}{getirdikleri}
      \begin{itemize}
        \item iletişim
        \item tıp uygulamaları
        \item uzay araştırmaları
        \item \ldots
      \end{itemize}
    \end{block}

    \pause
    \column{.5\textwidth}
    \begin{block}{götürdükleri}
      \begin{itemize}
        \item mahremiyet
        \item suç işlemenin kolaylaşması
        \item yabancılaşma
        \item \ldots
      \end{itemize}
    \end{block}
  \end{columns}
\end{frame}

\begin{frame}
  \frametitle{Sorunlar}

  \begin{itemize}
    \item sorunlar yeni mi?
  \end{itemize}

  \begin{block}{yeni değil}
    \begin{itemize}
      \item gerçek dünyada karşılıkları var: dolandırıcılık
      \item gerçek dünyada benzerleri var:
        yazılım kopyalama $\rightarrow$ kitap kopyalama
    \end{itemize}
  \end{block}

  \pause
  \begin{block}{yeni}
    \begin{itemize}
      \item bilişim teknolojisi olmadan yapılamaz
      \item gerçek dünyadaki riskler ve kısıtlamalar yok
    \end{itemize}
  \end{block}
\end{frame}

\begin{frame}
  \frametitle{Sorunlar}

  \begin{itemize}
    \item doğası yeni olmasa bile:

    \medskip
    \item ölçeğin büyüklüğü
    \item bilginin gücü
    \item bilginin kalıcılığı
    \item teknolojinin güvenilmezliği
  \end{itemize}
\end{frame}

\begin{frame}
  \frametitle{Bilginin Gücü: ABD terör gözetim listesi (2007)}

  \begin{columns}
    \column{.7\textwidth}
    \pgfuseimage{usatoday}

    \column{.3\textwidth}
    \begin{itemize}
      \item 15,000 kişi\\
        isim benzerliği\\
        nedeniyle listede
      \item havayollarını\\
        kullanamıyorlar
    \end{itemize}
  \end{columns}

  \medskip
  \tiny{\url{http://www.usatoday.com/news/washington/2007-11-06-watchlist_N.htm}}
\end{frame}

\begin{frame}
  \frametitle{Bilginin Kalıcılığı: İngiltere polis kayıtları (2008)}

  \begin{columns}
    \column{.56\textwidth}
    \pgfuseimage{bbc}

    \column{.44\textwidth}
    \begin{itemize}
      \item Bilgi Komiseri polisten\\
        eski suç kayıtlarını\\
        silmesini istiyor

      \medskip
      \item 16 yaşında bir çocuk\\
        99p değerinde\\
        et çalmış (1984)
      \item 13 yaşında bir çocuk\\
        küçük bir kavga\\
        nedeniyle uyarı almış
    \end{itemize}
  \end{columns}

  \medskip
  \tiny{\url{http://news.bbc.co.uk/2/hi/uk_news/england/7072241.stm}}
\end{frame}

\begin{frame}
  \frametitle{Güvenilmezlik: ABD devlet kayıtları (2008)}

  \begin{columns}
    \column{.65\textwidth}
    \pgfuseimage{msnbc}

    \column{.33\textwidth}
    \begin{itemize}
      \item onbinlerce kişi\\
        yaşadıkları halde\\
        kayıtlarda\\
        ölü görünüyor
      \item vergi iadelerini\\
        alamıyorlar
    \end{itemize}
  \end{columns}

  \medskip
  \tiny{\url{http://www.msnbc.msn.com/id/23378093/}}
\end{frame}

\begin{frame}
  \frametitle{Güvenilmezlik: Fransız Veri Koruma Ajansı raporu (2009)}

  \begin{columns}
    \column{.57\textwidth}
    \pgfuseimage{register}

    \column{.43\textwidth}
    \begin{itemize}
      \item 1 milyon Fransız\\
        sabıka kayıtlarındaki\\
        hatalar nedeniyle\\
        işten atılmış\\
        ya da iş bulamamış
    \end{itemize}
  \end{columns}

  \medskip
  \tiny{\url{http://www.theregister.co.uk/2009/01/30/stic/}}
\end{frame}

\begin{frame}
  \frametitle{Veri Madenciliği}

  \begin{itemize}
    \item \alert{veri madenciliği}
    \item veride örtülü kalıpların aranması
    \item bariz olmayan yeni kategoriler ortaya çıkarma

    \medskip
    \item karar destek sistemlerinde kullanılıyor:\\
      alışveriş kartları, kredi puanlama
  \end{itemize}
\end{frame}

\begin{frame}
  \frametitle{Örnek: Total Information Awareness}

  \begin{itemize}
    \item terörist saldırıların önüne geçme amaçlı

    \medskip
    \item pek çok değişik alandan veri toplansın
    \item finans, sağlık, iletişim, yolculuk, \ldots
    \item benzer davranış kalıplarından terörist olabilecekler belirlensin

    \bigskip
    \item hatalı sonuçlar
    \item yöntem kayması: suçu değil suçsuzluğu kanıtlamak gerekmesi
  \end{itemize}
\end{frame}

\section{Yöntem}

\begin{frame}
  \frametitle{Gelenekçi Yöntem}

  \begin{itemize}
    \item var olan ilkeleri yeni sorunlara uyarlayalım

    \bigskip
    \item yazılım $\rightarrow$ kitap, film, \ldots
    \item iletişim $\rightarrow$ yüzyüze, telefon, basın, \ldots
  \end{itemize}
\end{frame}

\begin{frame}
  \frametitle{Gelenekçi Yöntem}

  \begin{itemize}
    \item kavramları eşleştirmek zor olabilir
  \end{itemize}

  \begin{exampleblock}{yazılım neye benzer?}
    \begin{itemize}
      \item kitap, film $\rightarrow$ telif
      \item ilaç $\rightarrow$ patent
      \item Coca-Cola'nın formülü $\rightarrow$ ticari sır
    \end{itemize}
  \end{exampleblock}
\end{frame}

\begin{frame}
  \frametitle{Gelenekçi Yöntemin Eleştirisi}

  \begin{itemize}
    \item teknolojinin getirdiği olanakları kullanmıyor
    \item geçmişin hatalarını kopyalıyor

    \bigskip
    \item katı olunmaması kaydıyla iyi bir başlangıç noktası
  \end{itemize}
\end{frame}

\section*{Kaynaklar}

\begin{frame}
  \frametitle{Kaynaklar}

  \begin{block}{Okunacak: Tavani}
    \begin{itemize}
      \item Chapter 1: \alert{Introduction to Cyberethics}
    \end{itemize}
  \end{block}
\end{frame}

\end{document}
