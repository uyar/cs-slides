% Copyright (c) 2004-2012 H. Turgut Uyar <uyar@itu.edu.tr>
%
% Bu notlar "Creative Commons Attribution-NonCommercial-ShareAlike License" ile
% lisanslanmıştır. Yazarının açıkça belirtilmesi koşuluyla ve ticari olmayan
% amaçlarla kullanılabilir ve dağıtılabilir. Bu notlardan yola çıkılarak
% oluşturulacak çalışmaların da aynı lisansa bağlı olmaları gerekir.
%
% Lisans ile ilgili ayrıntılı bilgi almak için şu sayfaya başvurabilirsiniz:
% http://creativecommons.org/licenses/by-nc-sa/3.0/

\documentclass[dvipsnames]{beamer}

\mode<presentation>
{
  \usetheme{Rochester}
  \usecolortheme[named=Mahogany]{structure}
  \setbeamercovered{transparent}
}

\usepackage{ae}
\usepackage[T1]{fontenc}
\usepackage[utf8]{inputenc}
\usepackage[turkish]{babel}
\setbeamertemplate{navigation symbols}{}

\title{Bilişim Etiği}
\subtitle{Giriş}

\author{H. Turgut Uyar}
\date{2004-2012}

\AtBeginSubsection[]
{
  \begin{frame}<beamer>
    \frametitle{Konular}
    \tableofcontents[currentsection,currentsubsection]
  \end{frame}
}

%\beamerdefaultoverlayspecification{<+->}

\theoremstyle{definition}
\newtheorem{tanim}[theorem]{Tanım}

\theoremstyle{example}
\newtheorem{ornek}[theorem]{Örnek}

\theoremstyle{plain}

\pgfdeclareimage[width=2cm]{license}{../../license}
\pgfdeclareimage[width=8cm]{usatoday}{usatoday}
\pgfdeclareimage[width=7cm]{bbc}{bbc}
\pgfdeclareimage[width=8cm]{msnbc}{msnbc}
\pgfdeclareimage[width=7cm]{register}{register}

\begin{document}

\begin{frame}
  \titlepage
\end{frame}

\begin{frame}
  \frametitle{Lisans}

  \pgfuseimage{license}\hfill
  \copyright 2004-2012 H. Turgut Uyar

  \vfill
  \begin{tiny}
    You are free:
    \begin{itemize}
      \item to Share — to copy, distribute and transmit the work
      \item to Remix — to adapt the work
    \end{itemize}

    Under the following conditions:
    \begin{itemize}
      \item Attribution — You must attribute the work in the manner specified by
        the author or licensor (but not in any way that suggests that they
        endorse you or your use of the work).

      \item Noncommercial — You may not use this work for commercial purposes.

      \item Share Alike — If you alter, transform, or build upon this work, you
        may distribute the resulting work only under the same or similar license
        to this one.
    \end{itemize}
  \end{tiny}

  \vfill
  Legal code (the full license):\\
  \url{http://creativecommons.org/licenses/by-nc-sa/3.0/}
\end{frame}

\begin{frame}
  \frametitle{Konular}
  \tableofcontents
\end{frame}

\section{Teknolojinin Etkileri}

\begin{frame}
  \frametitle{Teknolojinin Etkileri}

  \begin{itemize}
    \item her yeni teknolojinin getirdikleri ve götürdükleri oluyor
  \end{itemize}

  \begin{ornek}
    \begin{itemize}
      \item nükleer enerji
      \item akıllı karayolları
      \item aerosol kutuları
    \end{itemize}
  \end{ornek}
\end{frame}

\begin{frame}
  \frametitle{Politika Boşlukları}

  \begin{itemize}
    \item her ilerlemede politika boşlukları oluşuyor
    \begin{itemize}
      \item her aşamada yeni ahlaki değerlendirme gerek
      \item sakıncalar azaltılmalı
      \item gerekirse o teknolojiden vazgeçilebilir
    \end{itemize}

    \pause
    \medskip
    \item oluşan boşluklar doldurulmalı
    \begin{itemize}
      \item yasalar
      \item kurum politikaları
      \item kişisel davranışlar
    \end{itemize}
  \end{itemize}
\end{frame}

\section{Bilişim Teknolojisi}

\begin{frame}
  \frametitle{Bilişim Teknolojisinin Etkileri}

  \begin{columns}
    \column{.5\textwidth}
    \begin{block}{getirdikleri}
      \begin{itemize}
        \item iletişim
        \item tıp
        \item uzay yolculuğu
        \item ...
      \end{itemize}
    \end{block}

    \pause
    \column{.5\textwidth}
    \begin{block}{götürdükleri}
      \begin{itemize}
        \item mahremiyet
        \item suç işlemenin kolaylaşması
        \item yabancılaşma
        \item ...
      \end{itemize}
    \end{block}
  \end{columns}
\end{frame}

\begin{frame}
  \frametitle{Sorunların Yeniliği}

  \begin{itemize}
    \item "sorunlar yeni değil" diyenler:
    \begin{itemize}
      \item suçların gerçek dünyada karşılığı var:\\
        dolandırıcılık
      \item sorunların gerçek dünyada benzerleri var:\\
        yazılım mülkiyeti $\rightarrow$ fikri mülkiyet\\
        mahremiyet $\rightarrow$ basın
    \end{itemize}

    \pause
    \medskip
    \item "sorunlar yeni" diyenler:
    \begin{itemize}
      \item bilişim teknolojisi olmadan yapılamaz
      \item gerçek dünyadaki zorluklar, kısıtlamalar yok
    \end{itemize}
  \end{itemize}
\end{frame}

\begin{frame}
  \frametitle{Sorunların Yeniliği}

  \begin{itemize}
    \item ölçeğin büyüklüğü
    \item bilginin gücü
    \item bilginin kalıcılığı
    \item teknolojinin güvenilmezliği
  \end{itemize}
\end{frame}

\begin{frame}
  \frametitle{Bilginin Gücü: ABD terör gözetim listesi (2007)}

  \begin{columns}
    \column{.7\textwidth}
    \pgfuseimage{usatoday}

    \column{.3\textwidth}
    \begin{itemize}
      \item 15000 kişi isim benzerliği nedeniyle listede
      \item havayollarını kullanamıyorlar
    \end{itemize}
  \end{columns}

  \medskip
  \tiny{\url{http://www.usatoday.com/news/washington/2007-11-06-watchlist_N.htm}}
\end{frame}

\begin{frame}
  \frametitle{Bilginin Kalıcılığı: İngiltere polis kayıtları (2008)}

  \begin{columns}
    \column{.6\textwidth}
    \pgfuseimage{bbc}

    \column{.4\textwidth}
    \begin{itemize}
      \item Bilgi Komiseri polisten eski suç kayıtlarını silmesini istiyor

      \medskip
      \item 1984'te 16 yaşında bir çocuk 99p değerinde et çalmış
      \item 13 yaşında bir çocuk küçük bir kavga nedeniyle uyarı almış
    \end{itemize}
  \end{columns}

  \medskip
  \tiny{\url{http://news.bbc.co.uk/2/hi/uk_news/england/7072241.stm}}
\end{frame}

\begin{frame}
  \frametitle{Güvenilmezlik: ABD devlet kayıtları (2008)}

  \begin{columns}
    \column{.7\textwidth}
    \pgfuseimage{msnbc}

    \column{.3\textwidth}
    \begin{itemize}
      \item onbinlerce kişi yaşadığı halde kayıtlarda ölü görünüyor
    \end{itemize}
  \end{columns}

  \medskip
  \tiny{\url{http://www.msnbc.msn.com/id/23378093/}}
\end{frame}

\begin{frame}
  \frametitle{Güvenilmezlik: Fransız Veri Koruma Ajansı raporu (2009)}

  \begin{columns}
    \column{.6\textwidth}
    \pgfuseimage{register}

    \column{.4\textwidth}
    \begin{itemize}
      \item 1 milyon Fransız sabıka kayıtlarındaki hatalar nedeniyle işten
        atılmış ya da iş bulamamış
    \end{itemize}
  \end{columns}

  \medskip
  \tiny{\url{http://www.theregister.co.uk/2009/01/30/stic/}}
\end{frame}

\section{Yöntem}

\begin{frame}
  \frametitle{Gelenekçi Yöntem}

  \begin{itemize}
    \item var olan ilkelerin yeni sorunlara uyarlanması
    \begin{itemize}
      \item yazılım: kitap, film, ...
      \item iletişim: yüzyüze, telefon, posta, basın, ...
    \end{itemize}

    \pause
    \bigskip
    \item kavramları yerleştirmek gerek
  \end{itemize}
\end{frame}

\begin{frame}
  \frametitle{Gelenekçi Yöntem}

  \begin{ornek}[Yazılım neye benzer?]
    \begin{itemize}
      \item kitap, film $\rightarrow$ telif
      \item ilaç $\rightarrow$ patent
      \item Coca-Cola'nın formülü $\rightarrow$ ticari sır
    \end{itemize}
  \end{ornek}
\end{frame}

\begin{frame}
  \frametitle{Gelenekçi Yöntemin Eleştirisi}

  \begin{itemize}
    \item teknolojinin getirdiği olanakları kullanmıyor
    \item geçmişin hatalarını kopyalıyor

    \pause
    \bigskip
    \item katı olunmaması kaydıyla iyi bir başlangıç noktası
  \end{itemize}
\end{frame}

\section*{Kaynaklar}

\begin{frame}
  \frametitle{Kaynaklar}

  \begin{block}{Okunacak: Tavani}
    \begin{itemize}
      \item Chapter 1: \alert{Introduction to Cyberethics}
    \end{itemize}
  \end{block}
\end{frame}

\end{document}
