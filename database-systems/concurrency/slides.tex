% Copyright (c) 2002-2014
%       H. Turgut Uyar <uyar@itu.edu.tr>
%       Şule Gündüz Öğüdücü <sgunduz@itu.edu.tr>
%
% This work is licensed under a "Creative Commons
% Attribution-NonCommercial-ShareAlike 4.0 International License".
% For more information, please visit:
% https://creativecommons.org/licenses/by-nc-sa/4.0/

\documentclass[dvipsnames]{beamer}

\usepackage{ae}
\usepackage[scaled=0.88]{beramono}
\usepackage[T1]{fontenc}
\usepackage[utf8]{inputenc}
\setbeamertemplate{navigation symbols}{}
\setbeamersize{text margin left=2em, text margin right=2em}
\usepackage[labelformat=empty, aboveskip=1pt, belowskip=1pt]{caption}

\usepackage{listings}
\lstdefinelanguage{FullSQL}[]{SQL}{
  morekeywords={BINARY, BOOLEAN, CYCLE, FINAL, INCREMENT, IS, LARGE, MAXVALUE,
                MINVALUE, NO_ACTION, OBJECT, REFERENCES, RENAME, SEQUENCE,
                START, TO, TYPE, VACUUM}
}
\lstdefinelanguage{ExtendedSQL}[]{FullSQL}{
  morekeywords={ACCESS, BEGIN, COMMITTED, ERROR, EXCLUSIVE, FOR, GOTO, LOCK,
                MODE, ON, REPEATABLE, ROW, SERIALIZABLE, SHARE, UNCOMMITTED,
                WORK}
}
\lstset{basicstyle=\ttfamily, keywordstyle=\color{ForestGreen},
        showstringspaces=false}
\lstset{language=ExtendedSQL}

\mode<presentation>
{
  \usetheme{Warsaw}
  \usecolortheme[named=ForestGreen]{structure}
  \setbeamercovered{transparent}
}

\title{Database Systems}
\subtitle{Concurrency}

\author{H. Turgut Uyar \and Şule Öğüdücü}
\date{2002-2014}

\AtBeginSubsection[]{
  \begin{frame}<beamer>
    \frametitle{Topics}
    \tableofcontents[currentsection,currentsubsection]
  \end{frame}
}

\theoremstyle{plain}

\pgfdeclareimage[width=2cm]{license}{../license}

\pgfdeclareimage[height=4.5cm]{recovery}{recovery}
\pgfdeclareimage{wait}{wait}
\pgfdeclareimage[height=4.5cm]{precedence}{precedence}

\begin{document}

\begin{frame}
  \titlepage
\end{frame}

\begin{frame}
  \frametitle{License}

  \pgfuseimage{license}\hfill
  \copyright~2002-2014 T. Uyar, Ş. Öğüdücü

  \vfill
  \begin{footnotesize}
    You are free to:
    \begin{itemize}
      \itemsep0em
      \item Share -- copy and redistribute the material in any medium or format
      \item Adapt -- remix, transform, and build upon the material
    \end{itemize}

    Under the following terms:
    \begin{itemize}
      \itemsep0em
      \item Attribution -- You must give appropriate credit, provide a link to
        the license, and indicate if changes were made.

      \item NonCommercial -- You may not use the material for commercial
        purposes.

      \item ShareAlike -- If you remix, transform, or build upon the material,
        you must distribute your contributions under the same license as the
        original.
    \end{itemize}
  \end{footnotesize}

  \begin{small}
    For more information:\\
    \url{https://creativecommons.org/licenses/by-nc-sa/4.0/}

    \smallskip
    Read the full license:\\
    \url{https://creativecommons.org/licenses/by-nc-sa/4.0/legalcode}
  \end{small}
\end{frame}

\begin{frame}
  \frametitle{Topics}
  \tableofcontents
\end{frame}

\section{Transactions}

\subsection{Introduction}

\begin{frame}
  \frametitle{Transactions}

  \begin{itemize}
    \item a group of operations may have to be carried out together
    \begin{itemize}
      \item finishing some operations while failing on others\\
	might cause inconsistency
    \end{itemize}
    \item \alert{transaction}: a logical unit of work

    \pause
    \medskip
    \item no guarantee that multiple operations will all be finished
    \item at least return to the state before the changes
  \end{itemize}
\end{frame}

\begin{frame}[fragile]
  \frametitle{Transaction Example}

  \begin{itemize}
    \item transfer money from one bank account to another

    \medskip
    \begin{lstlisting}
UPDATE ACCOUNTS SET BALANCE = BALANCE - 100
  WHERE ACCOUNTID = 123

UPDATE ACCOUNTS SET BALANCE = BALANCE + 100
  WHERE ACCOUNTID = 456
    \end{lstlisting}
  \end{itemize}
\end{frame}

\begin{frame}
  \frametitle{Transaction Properties}

  \begin{itemize}
    \item \alert{A}tomicity: all or nothing

    \pause
    \medskip
    \item \alert{C}onsistency: move from one consistent state to another

    \pause
    \medskip
    \item \alert{I}solation: whether operations of an unfinished transaction\\
      affect other transactions or not

    \pause
    \medskip
    \item \alert{D}urability: when a transaction is finished,\\
      its changes are permanent even if there is a system failure
  \end{itemize}
\end{frame}

\begin{frame}[fragile]
  \frametitle{SQL Transactions}

  \begin{itemize}
    \item starting a transaction
    \begin{lstlisting}
BEGIN [ WORK | TRANSACTION ]
    \end{lstlisting}

    \item finishing a transaction
    \begin{lstlisting}
COMMIT [ WORK | TRANSACTION ]
    \end{lstlisting}

    \item cancelling a transaction
    \begin{lstlisting}
ROLLBACK [ WORK | TRANSACTION ]
    \end{lstlisting}
  \end{itemize}
\end{frame}

\begin{frame}[fragile]
  \frametitle{Transaction Example}

  \begin{lstlisting}
  BEGIN TRANSACTION
  ON ERROR GOTO UNDO
  UPDATE ACCOUNTS SET BALANCE = BALANCE - 100
    WHERE (ACCOUNTID = 123)
  UPDATE ACCOUNTS SET BALANCE = BALANCE + 100
    WHERE (ACCOUNTID = 456)
  COMMIT
  ...

UNDO:
  ROLLBACK
  \end{lstlisting}
\end{frame}

\subsection{Recovery}

\begin{frame}
  \frametitle{Recovery}

  \begin{itemize}
    \item consider a system failure during a transaction
    \item buffer cache has not been flushed to the disk

    \pause
    \medskip
    \item how to guarantee durability?
    \item derive the data from other sources in the system
  \end{itemize}
\end{frame}

\begin{frame}
  \frametitle{Transaction Log}

  \begin{itemize}
    \item \alert{log}: values of every affected tuple\\
      before and after every operation

    \medskip
    \item \alert{write-ahead log rule}:\\
      log must be flushed before the transaction is committed

    \pause
    \medskip
    \item accessing records in the log is sequential by nature
  \end{itemize}
\end{frame}

\begin{frame}
  \frametitle{Checkpoints}

  \begin{itemize}
    \item create \alert{checkpoints} at certain intervals:

    \medskip
    \item flush buffer cache to the physical medium
    \item note the checkpoint
    \item note the continuing transactions
  \end{itemize}
\end{frame}

\begin{frame}
  \frametitle{Recovery Lists}

  \begin{itemize}
    \item after the failure: which transactions will be undone,\\
      which transactions will be made permanent?
    \item create two lists: \emph{undo} (U), \emph{redo} (R)

    \pause
    \medskip
    \item $t_C$: last checkpoint in the log
    \begin{itemize}
      \item add the transactions which are active at $t_C$ to U
    \end{itemize}

    \item scan records from $t_C$ to end of log
    \begin{itemize}
      \item add any starting transaction to U
      \item move any finishing transaction to R
    \end{itemize}
  \end{itemize}
\end{frame}

\begin{frame}
  \frametitle{Recovery Example}

  \begin{columns}[t]
    \column{.4\textwidth}
    \begin{center}
      \pgfuseimage{recovery}
    \end{center}

    \pause
    \column{.5\textwidth}
    \begin{itemize}
      \item $t_C$:\\
        $U=[T_2,T_3]$
        $R=[]$

      \pause
      \item $T_4$ started:\\
        $U=[T_2,T_3,T_4]$
        $R=[]$

      \pause
      \item $T_2$ finished:\\
        $U=[T_3,T_4]$
        $R=[T_2]$

      \pause
      \item $T_5$ started:\\
        $U=[T_3,T_4,T_5]$
        $R=[T_2]$

      \pause
      \item $T_4$ finished:\\
        $U=[T_3,T_5]$
        $R=[T_2,T_4]$
    \end{itemize}
  \end{columns}
\end{frame}

\begin{frame}
  \frametitle{Recovery Process}

  \begin{itemize}
    \item scan records backwards from end of log
    \item undo the changes made by the transactions in $U$

    \medskip
    \item scan records forwards
    \item redo the changes made by the transactions in $R$
  \end{itemize}
\end{frame}

\subsection{Two-Phase Commit}

\begin{frame}
  \frametitle{Two-Phase Commit}

  \begin{itemize}
    \item different source managers
    \item different undo / redo mechanisms

    \medskip
    \item modifications on data on different source managers
    \item either commit in all or rollback in all

    \medskip
    \item coordinator
  \end{itemize}
\end{frame}

\begin{frame}
  \frametitle{Protocol}

  \begin{itemize}
    \item coordinator $\rightarrow$ participants:\\
      flush all data regarding the transaction

    \pause
    \item coordinator $\rightarrow$ participants:\\
      start transaction and report back the result

    \pause
    \medskip
    \item all participants succeeded: success
    \item otherwise: failure

    \medskip
    \item if success, coordinator $\rightarrow$ participants: commit
    \item if failure, coordinator $\rightarrow$ participants: rollback
  \end{itemize}
\end{frame}

\subsection*{References}

\begin{frame}
  \frametitle{References}

  \begin{block}{Required Reading: Date}
    \begin{itemize}
      \item Chapter 15: \alert{Recovery}
    \end{itemize}
  \end{block}
\end{frame}

\section{Concurrency}

\subsection{Introduction}

\begin{frame}
  \frametitle{Concurrency}

  \begin{itemize}
    \item problems that might arise due to simultaneuous transactions:

    \bigskip
    \item lost update
    \item uncommitted dependency
    \item inconsistent analysis
  \end{itemize}
\end{frame}

\begin{frame}[fragile]
  \frametitle{Lost Update}

  \begin{table}
    \begin{tabular}{ll}
Transaction A & Transaction B\\\hline
...           & ...          \\\pause
RETRIEVE p    & ...          \\\pause
...           & ...          \\
...           & RETRIEVE p   \\\pause
...           & ...          \\
UPDATE p      & ...          \\\pause
...           & ...          \\
...           & UPDATE p     \\
...           & ...
    \end{tabular}
  \end{table}
\end{frame}

\begin{frame}[fragile]
  \frametitle{Uncommitted Dependency}

  \begin{table}
    \begin{tabular}{ll}
Transaction A & Transaction B\\\hline
...           & ...          \\\pause
...           & UPDATE p     \\\pause
...           & ...          \\
RETRIEVE p    & ...          \\\pause
...           & ...          \\
...           & ROLLBACK     \\
...           &
    \end{tabular}
  \end{table}
\end{frame}

\begin{frame}[fragile]
  \frametitle{Inconsistent Analysis}

  \begin{itemize}
    \item compute sum of accounts: acc1=40, acc2=50, acc3=30
    \smallskip
    \begin{table}
      \begin{tabular}{ll}
Transaction A         & Transaction B                    \\\hline
...                   & ...                              \\\pause
RETRIEVE acc1 ($40$)  & ...                              \\\pause
RETRIEVE acc2 ($90$)  & ...                              \\\pause
...                   & ...                              \\
...                   & UPDATE acc3 ($30 \rightarrow 20$)\\
...                   & UPDATE acc1 ($40 \rightarrow 50$)\\
...                   & COMMIT                           \\\pause
...                   & ...                              \\
RETRIEVE acc3 ($110$) &                                  \\
...                   &
      \end{tabular}
    \end{table}
  \end{itemize}
\end{frame}

\begin{frame}
  \frametitle{Conflicts}

  \begin{itemize}
    \item A reads, B reads: no problem
    \item A reads, B writes: non-repeatable read (inconsistent analysis)
    \item A writes, B reads: dirty read (uncommitted dependency)
    \item A writes, B writes: dirty write (lost update)
  \end{itemize}
\end{frame}

\subsection{Locking}

\begin{frame}
  \frametitle{Locking}

  \begin{itemize}
    \item let transactions lock the tuples they work on

    \medskip
    \item \alert{shared} lock (S)
    \item \alert{exclusive} lock (X)
  \end{itemize}
\end{frame}

\begin{frame}
  \frametitle{Lock Requests}

  \begin{table}
    \caption{lock type compatibility matrix}
    \begin{tabular}{|c||c|c|c|}\hline
  & S & X\\\hline\hline
- & Y & Y\\\hline
S & Y & N\\\hline
X & N & N\\\hline
    \end{tabular}
  \end{table}

  \medskip
  \begin{itemize}
    \item existing shared lock: only shared lock requests are allowed
    \item existing exclusive lock: all lock requests are denied
  \end{itemize}
\end{frame}

\begin{frame}
  \frametitle{Locking Protocol}

  \begin{itemize}
    \item transaction requests lock depending on operation
    \item promote a shared lock to an exclusive lock

    \medskip
    \item if request denied, it starts waiting
    \item it continues when the transaction that holds the lock releases it

    \medskip
    \item \alert{starvation}
  \end{itemize}
\end{frame}

\begin{frame}[fragile]
  \frametitle{Lost Update}

  \begin{table}
    \begin{tabular}{ll}
Transaction A   & Transaction B  \\\hline
...             & ...            \\\pause
RETRIEVE p (S+) & ...            \\\pause
...             & ...            \\
...             & RETRIEVE p (S+)\\\pause
...             & ...            \\
UPDATE p (X-)   & ...            \\
wait            & ...            \\\pause
wait            & UPDATE p (X-)  \\
wait            & wait
    \end{tabular}
  \end{table}
\end{frame}

\begin{frame}[fragile]
  \frametitle{Uncommitted Dependency}

  \begin{table}
    \begin{tabular}{ll}
Transaction A   & Transaction B\\\hline
...             & ...          \\\pause
...             & UPDATE p (X+)\\\pause
...             & ...          \\
RETRIEVE p (S-) & ...          \\
wait            & ...          \\\pause
wait            & ROLLBACK     \\
RETRIEVE p (S+) &              \\
...             &
    \end{tabular}
  \end{table}
\end{frame}

\begin{frame}[fragile]
  \frametitle{Inconsistent Analysis}

  \begin{table}
    \begin{tabular}{ll}
Transaction A        & Transaction B   \\\hline
...                  & ...             \\\pause
RETRIEVE acc1 (S+)   & ...             \\\pause
RETRIEVE acc2 (S+)   & ...             \\\pause
...                  & ...             \\
...                  & UPDATE acc3 (X+)\\\pause
...                  & UPDATE acc1 (X-)\\
...                  & wait            \\\pause
RETRIEVE acc3 (S-)   & wait            \\
wait                 & wait
    \end{tabular}
  \end{table}
\end{frame}

\begin{frame}
  \frametitle{Deadlock}

  \begin{itemize}
    \item \alert{deadlock}: transactions waiting for each other to release the locks
    \item almost always between two transactions

    \medskip
    \item detect and solve
    \item prevent
  \end{itemize}
\end{frame}

\begin{frame}
  \frametitle{Solving Deadlocks}

  \begin{columns}[t]
    \column{.5\textwidth}
    \begin{example}
      \begin{center}
        \pgfuseimage{wait}
      \end{center}
    \end{example}

    \column{.5\textwidth}
    \begin{itemize}
      \item wait graph
      \item choose a \alert{victim} and kill it
    \end{itemize}
  \end{columns}
\end{frame}

\begin{frame}
  \frametitle{Preventing Deadlocks}

  \begin{itemize}
    \item every transaction has a starting timestamp
    \item if A's request conflicts with B's lock:

    \pause
    \item \alert{wait-die}: A waits if older than B, otherwise it dies\\
      A is rolled back and restarted

    \item \alert{wound-wait}: A waits if younger than B, otherwise it wounds B\\
      B is rolled back and restarted

    \pause
    \item timestamp of restarted transaction is not changed
  \end{itemize}
\end{frame}

\begin{frame}[fragile]
  \frametitle{Lock Statements}

  \begin{itemize}
    \item shared lock
    \begin{lstlisting}
SELECT query FOR SHARE
    \end{lstlisting}

    \item exclusive lock
    \begin{lstlisting}
SELECT query FOR UPDATE
    \end{lstlisting}
  \end{itemize}
\end{frame}

\subsection{Isolation Levels}

\begin{frame}
  \frametitle{Isolation Levels}

  \begin{itemize}
    \item if isolation is decreased, concurrency can be increased
    \item various isolation levels

    \medskip
    \item serializable
    \item repeatable read
    \item read committed
    \item read uncommitted
  \end{itemize}
\end{frame}

\begin{frame}
  \frametitle{Serializability}

  \begin{itemize}
    \item \emph{serial execution}:
      a transaction starts only after another is finished

    \pause
    \item \alert{serializable}: result of concurrent execution is always\\
      the same as one of the serial executions
  \end{itemize}

  \pause
  \begin{example}
    \begin{itemize}
      \item $x=10$
      \item transaction A: $x=x+1$
      \item transaction B: $x=2*x$

      \pause
      \medskip
      \item first A, then B: $x=22$
      \item first B, then A: $x=21$
    \end{itemize}
  \end{example}
\end{frame}

\begin{frame}
  \frametitle{Two-Phase Locking Protocol}

  \begin{itemize}
    \item \alert{two-phase locking}:\\
      after any lock is released, no more new lock requests
    \item expansion phase: gather locks
    \item contraction phase: release locks

    \medskip
    \item \alert{two-phase strict locking}:\\
      all locks are released at the end of the transaction

    \pause
    \medskip
    \item \emph{If all transactions obey the two-phase locking protocol,\\
      all concurrent executions are serializable.}
  \end{itemize}
\end{frame}

\begin{frame}[fragile]
  \frametitle{Read Committed}

  \begin{itemize}
    \item read committed: only X locks are held until end of transaction
  \end{itemize}

  \begin{table}
    \begin{tabular}{ll}
Transaction A   & Transaction B\\\hline
...             & ...          \\\pause
RETRIEVE p (S+) & ...          \\\pause
...             & ...          \\
release lock    & ...          \\\pause
...             & ...          \\
...             & UPDATE p (X+)\\
...             & COMMIT       \\\pause
RETRIEVE p (S+) &
    \end{tabular}
  \end{table}
\end{frame}

\begin{frame}
  \frametitle{Phantoms}

  \begin{itemize}
    \item \alert{phantom}: when the query is executed again, new tuples appear
  \end{itemize}

  \begin{example}
    \begin{itemize}
      \item A computes the average of a customer's account balances:\\
        \smallskip
        $\frac{100+100+100}{3}=100$

      \pause
      \item B creates new account ($200$) for the same customer
      \item A computes again:\\
        \smallskip
        $\frac{100+100+100+200}{4}=125$
    \end{itemize}
  \end{example}
\end{frame}

\begin{frame}[fragile]
  \frametitle{Isolation Levels}

  \begin{itemize}
    \item setting an isolation level
    \begin{lstlisting}
SET TRANSACTION ISOLATION LEVEL
  [ SERIALIZABLE | REPEATABLE READ |
    READ COMMITTED | READ UNCOMMITTED ]
    \end{lstlisting}
  \end{itemize}
\end{frame}

\begin{frame}[fragile]
  \frametitle{Isolation Level Problems}

  \begin{table}
    \begin{tabular}{|l||c|c|c|}\hline
                 & dirty & non-repeatable & phantom\\
                 & read  & read           &        \\\hline\hline
READ UNCOMMITTED & Y     & Y              & Y      \\\hline
READ COMMITTED   & N     & Y              & Y      \\\hline
REPEATABLE READ  & N     & N              & Y      \\\hline
SERIALIZABLE     & N     & N              & N      \\\hline
    \end{tabular}
  \end{table}
\end{frame}

\subsection{Intent Locks}

\begin{frame}
  \frametitle{Locking Granularity}

  \begin{itemize}
    \item lock relations instead of tuples
    \item even the entire database

    \medskip
    \item granularity: unit of locking
    \item if granularity is increased, concurrency is decreased

    \pause
    \medskip
    \item hard to find locks on tuples\\
      $\rightarrow$ first, get \alert{intent locks} on relation variables
  \end{itemize}
\end{frame}

\begin{frame}
  \frametitle{Intent Locks}

  \begin{itemize}
    \item Intent Shared (IS):\\
      transaction intends to read some tuples

    \item Intent Exclusive (IX):\\
      IS + transaction intends to write some tuples

    \item Shared (S):\\
      concurrent readers are allowed but no concurrent writers

    \item Shared + Intent Exclusive (SIX)

    \item Exclusive (X):\\
      no concurrency allowed on this relation
  \end{itemize}
\end{frame}

\begin{frame}
  \frametitle{Lock Requests}

  \begin{table}
    \caption{lock compatibility matrix}
    \begin{tabular}{|c||c|c|c|c|c|}\hline
    &  IS &  IX &   S & SIX &   X\\\hline\hline
  - &   Y &   Y &   Y &   Y &   Y\\\hline
 IS &   Y &   Y &   Y &   Y &   N\\\hline
 IX &   Y &   Y &   N &   N &   N\\\hline
  S &   Y &   N &   Y &   N &   N\\\hline
SIX &   Y &   N &   N &   N &   N\\\hline
  X &   N &   N &   N &   N &   N\\\hline
    \end{tabular}
  \end{table}
\end{frame}

\begin{frame}
  \frametitle{Lock Precedence}

  \begin{columns}[t]
    \column{.4\textwidth}
    \begin{center}
      \pgfuseimage{precedence}
    \end{center}

    \column{.6\textwidth}
    \begin{itemize}
      \item for an S on a tuple,\\
	at least an IS on the relation
      \item for an X on a tuple,\\
	at least an IX on the relation
    \end{itemize}
  \end{columns}
\end{frame}

\begin{frame}[fragile]
  \frametitle{Locking Statements}

  \begin{itemize}\item lock a table
    \begin{lstlisting}
LOCK [ TABLE ] table_name
     [ IN lock_mode MODE ]
    \end{lstlisting}

    \item lock modes:
    \begin{itemize}
      \item \lstinline!ACCESS SHARE!
      \item \lstinline!ROW SHARE!
      \item \lstinline!ROW EXCLUSIVE!
      \item \lstinline!SHARE UPDATE EXCLUSIVE!
      \item \lstinline!SHARE!
      \item \lstinline!SHARE ROW EXCLUSIVE!
      \item \lstinline!EXCLUSIVE!
      \item \lstinline!ACCESS EXCLUSIVE!
    \end{itemize}
  \end{itemize}
\end{frame}

\subsection*{References}

\begin{frame}
  \frametitle{References}

  \begin{block}{Required Reading: Date}
    \begin{itemize}
      \item Chapter 16: \alert{Concurrency}
    \end{itemize}
  \end{block}
\end{frame}

\end{document}
