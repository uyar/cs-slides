% Copyright (c) 2002-2010
%       H. Turgut Uyar <uyar@itu.edu.tr>
%       Şule Gündüz Öğüdücü <sgunduz@itu.edu.tr>
%
% These notes are licensed using the
% "Creative Commons Attribution-NonCommercial-ShareAlike License".
% You are free to copy, distribute and transmit the work, and to adapt the work
% as long as you attribute the authors, do not use it for commercial purposes,
% and any derivative work is under the same or a similar license.
%
% Read the full legal code at:
% http://creativecommons.org/licenses/by-nc-sa/3.0/

\documentclass[dvipsnames]{beamer}

\usepackage{ae}
\usepackage[T1]{fontenc}
\usepackage[utf8]{inputenc}
\setbeamertemplate{navigation symbols}{}

\usepackage{listings}
\lstdefinelanguage{ExtendedSQL}[]{SQL}{
  morekeywords={ACCESS,BEGIN,COMMITTED,ERROR,EXCLUSIVE,FOR,GOTO,LOCK,MODE,ON,
                REPEATABLE,ROW,SERIALIZABLE,SHARE,UNCOMMITTED,WORK}
}
\lstset{language=ExtendedSQL}

\mode<presentation>
{
  \usetheme{Warsaw}
  \usecolortheme[named=ForestGreen]{structure}
  \setbeamercovered{transparent}
}

\title{Database Systems}
\subtitle{Concurrency}

\author{H. Turgut Uyar \and Şule Öğüdücü}
\date{2002-2010}

\AtBeginSubsection[]{
  \begin{frame}<beamer>
    \frametitle{Topics}
    \tableofcontents[currentsection,currentsubsection]
  \end{frame}
}

\theoremstyle{plain}

\pgfdeclareimage[width=2cm]{license}{../../license}

\pgfdeclareimage[height=4.5cm]{recovery}{recovery}
\pgfdeclareimage{wait}{wait}
\pgfdeclareimage[height=4.5cm]{precedence}{precedence}

\begin{document}

\begin{frame}
  \titlepage
\end{frame}

\begin{frame}
  \frametitle{License}

  \pgfuseimage{license}\hfill
  \copyright 2002-2010 T. Uyar, Ş. Öğüdücü

  \vfill
  \begin{tiny}
    You are free:
    \begin{itemize}
      \item to Share — to copy, distribute and transmit the work
      \item to Remix — to adapt the work
    \end{itemize}

    Under the following conditions:
    \begin{itemize}
      \item Attribution — You must attribute the work in the manner specified by
        the author or licensor (but not in any way that suggests that they
        endorse you or your use of the work).

      \item Noncommercial — You may not use this work for commercial purposes.

      \item Share Alike — If you alter, transform, or build upon this work, you
        may distribute the resulting work only under the same or similar license
        to this one.
    \end{itemize}
  \end{tiny}

  \vfill
  Legal code (the full license):\\
  \url{http://creativecommons.org/licenses/by-nc-sa/3.0/}
\end{frame}

\begin{frame}
  \frametitle{Topics}
  \tableofcontents
\end{frame}

\section{Transactions}

\subsection{Introduction}

\begin{frame}
  \frametitle{Transactions}

  \begin{itemize}
    \item a group of operations to be carried out together
    \begin{itemize}
      \item doing one operation while omitting the other might cause
        inconsistency
    \end{itemize}
  \end{itemize}

  \pause
  \begin{definition}
    \alert{transaction}:\\
      a logical unit of work
  \end{definition}
\end{frame}

\begin{frame}[fragile]
  \frametitle{Transaction Example}

  \begin{example}[transferring money from one bank account to another]
    \begin{lstlisting}
UPDATE ACCOUNTS SET BALANCE = BALANCE - 100
  WHERE ACCOUNTID = 123

UPDATE ACCOUNTS SET BALANCE = BALANCE + 100
  WHERE ACCOUNTID = 456
    \end{lstlisting}
  \end{example}
\end{frame}

\begin{frame}
  \frametitle{Transaction Management}

  \begin{itemize}
    \item no guarantee that a group of operations will be carried out together
    \begin{itemize}
      \item we should at least be able to return to the state before the
        changes
    \end{itemize}
  \end{itemize}
\end{frame}

\begin{frame}
  \frametitle{Transaction Properties}

  \begin{itemize}
    \item atomicity
    \begin{itemize}
      \item all or nothing
    \end{itemize}

    \pause
    \item consistency
    \begin{itemize}
      \item from one consistent state to another
    \end{itemize}

    \pause
    \item isolation
    \begin{itemize}
      \item operations of an unfinished transaction do not affect other
        transactions
    \end{itemize}

    \pause
    \item durability
    \begin{itemize}
      \item when a transaction is finished, its changes are permament even if
        there is a system failure
    \end{itemize}
  \end{itemize}
\end{frame}

\begin{frame}[fragile]
  \frametitle{Transaction Start/End}

  \begin{block}{starting a transaction}
    \begin{lstlisting}
BEGIN [ WORK | TRANSACTION ]
    \end{lstlisting}
  \end{block}

  \pause
  \begin{block}{successfully ending a transaction}
    \begin{lstlisting}
COMMIT [ WORK | TRANSACTION ]
    \end{lstlisting}
  \end{block}

  \pause
  \begin{block}{unsuccessfully ending a transaction}
    \begin{lstlisting}
ROLLBACK [ WORK | TRANSACTION ]
    \end{lstlisting}
  \end{block}
\end{frame}

\begin{frame}[fragile]
  \frametitle{Transaction Example}

  \begin{example}
    \begin{lstlisting}
  BEGIN TRANSACTION
  ON ERROR GOTO UNDO
  UPDATE ACCOUNTS SET BALANCE = BALANCE - 100
    WHERE ACCOUNTID = 123
  UPDATE ACCOUNTS SET BALANCE = BALANCE + 100
    WHERE ACCOUNTID = 456
  COMMIT
  ...
UNDO: ROLLBACK
    \end{lstlisting}
  \end{example}
\end{frame}

\subsection{Recovery}

\begin{frame}
  \frametitle{Recovery}

  \begin{itemize}
    \item system failure during a transaction
    \begin{itemize}
      \item buffer cache has not been flushed to the disk
    \end{itemize}

    \pause
    \item how to guarantee durability?
  \end{itemize}
\end{frame}

\begin{frame}
  \frametitle{Transaction Log}

  \begin{itemize}
    \item data can be derived from some other source in the system
    \begin{itemize}
      \item internal level
    \end{itemize}

    \pause
    \medskip
    \item the values of every tuple before and after the operation is noted
      in the \alert{log}
      \begin{itemize}
        \item \emph{write-ahead rule}:\\
          the log must be flushed to the physical medium before the transaction
          is finished
      \end{itemize}
  \end{itemize}
\end{frame}

\begin{frame}
  \frametitle{Checkpoints}

  \begin{itemize}
    \item create \alert{checkpoints} in the log at certain intervals
    \begin{itemize}
      \item flush buffer cache to the physical medium
      \item note the checkpoint:\\
        continuing transactions
    \end{itemize}
  \end{itemize}
\end{frame}

\begin{frame}
  \frametitle{Recovery Lists}

  \begin{itemize}
    \item after the failure, which transactions will be undone, which
      transactions will be made permanent?
    \begin{itemize}
      \item two lists: \emph{undo} (U), \emph{redo} (R)
    \end{itemize}

    \pause
    \item $t_C$: last checkpoint in the log
    \begin{itemize}
      \item add the transactions which are active at $t_C$ to the undo list
    \end{itemize}

    \pause
    \item scan records from $t_C$ to end of log
    \begin{itemize}
      \item add any starting transaction to the undo list
      \item add any finishing transaction to the redo list
    \end{itemize}
  \end{itemize}
\end{frame}

\begin{frame}
  \frametitle{Recovery Example}

  \begin{example}
    \begin{columns}[t]
      \column{.4\textwidth}
      \begin{center}\pgfuseimage{recovery}\end{center}

      \pause
      \column{.5\textwidth}
      \begin{itemize}
        \item $t_C$:\\
          $U=\{T_2$,$T_3\}$
          $R=\emptyset$

        \pause
        \item $T_4$ started:\\
          $U=\{T_2,T_3,T_4\}$
          $R=\emptyset$

        \pause
        \item $T_2$ finished:\\
          $U=\{T_3,T_4\}$
          $R=\{T_2\}$

        \pause
        \item $T_5$ started:\\
          $U=\{T_3,T_4,T_5\}$
          $R=\{T_2\}$

        \pause
        \item $T_4$ finished:\\
          $U=\{T_3,T_5\}$
          $R=\{T_2,T_4\}$
      \end{itemize}
    \end{columns}
  \end{example}
\end{frame}

\begin{frame}
  \frametitle{Recovery Process}

  \begin{itemize}
    \item scan records from end of log backwards
    \begin{itemize}
      \item undo the operations of the transactions in the undo list
    \end{itemize}

    \pause
    \item scan records forwards
    \begin{itemize}
      \item redo the operations of the transactions in the redo list
    \end{itemize}
  \end{itemize}
\end{frame}

\subsection{Two-Phase Commit}

\begin{frame}
  \frametitle{Two-Phase Commit}

  \begin{itemize}
    \item different source managers
    \begin{itemize}
      \item different undo / redo mechanisms
    \end{itemize}

    \pause
    \item modifications on data that reside on different source managers
    \begin{itemize}
      \item either commit in all sources\\
        or rollback in all sources
    \end{itemize}

    \pause
    \item \alert{coordinator}
  \end{itemize}
\end{frame}

\begin{frame}
  \frametitle{Protocol}

  \begin{itemize}
    \item coordinator tells all participants to flush the data regarding
      the transaction to the physical medium

    \pause
    \item coordinator tells all participants to start the transaction and
      report back the result
    \begin{itemize}
      \item if all participants report success, coordinator decides to
        commit
      \item if one or more participants report failure, coordinator decides
        to rollback
    \end{itemize}

    \pause
    \item coordinator informs the participants about the decision
  \end{itemize}
\end{frame}

\subsection*{References}

\begin{frame}
  \frametitle{References}

  \begin{block}{Required text: Date}
    \begin{itemize}
      \item Chapter 15: \alert{Recovery}
    \end{itemize}
  \end{block}
\end{frame}

\section{Concurrency}

\subsection{Introduction}

\begin{frame}
  \frametitle{Concurrency}

  \begin{itemize}
    \item lost update
    \item uncommitted dependency
    \item inconsistent analysis
  \end{itemize}
\end{frame}

\begin{frame}[fragile]
  \frametitle{Lost Update}

  \begin{example}
    \begin{table}
      \begin{tabular}{ll}
Transaction A & Transaction B\\\hline
...           & ...          \\\pause
RETRIEVE p    & ...          \\\pause
...           & ...          \\
...           & RETRIEVE p   \\\pause
...           & ...          \\
UPDATE p      & ...          \\\pause
...           & ...          \\
...           & UPDATE p     \\
...           & ...
      \end{tabular}
    \end{table}
  \end{example}
\end{frame}

\begin{frame}[fragile]
  \frametitle{Uncommitted Dependency}

  \begin{example}
    \begin{table}
      \begin{tabular}{ll}
Transaction A & Transaction B\\\hline
...           & ...          \\\pause
...           & UPDATE p     \\\pause
...           & ...          \\
RETRIEVE p    & ...          \\\pause
...           & ...          \\
...           & ROLLBACK     \\
...           &
      \end{tabular}
    \end{table}
  \end{example}
\end{frame}

\begin{frame}[fragile]
  \frametitle{Inconsistent Analysis}

  \begin{example}[sum of accounts: acc1=40, acc2=50, acc3=30]
    \begin{table}
      \begin{tabular}{ll}
Transaction A         & Transaction B                    \\\hline
...                   & ...                              \\\pause
RETRIEVE acc1 ($40$)  & ...                              \\\pause
RETRIEVE acc2 ($90$)  & ...                              \\\pause
...                   & ...                              \\
...                   & UPDATE acc3 ($30 \rightarrow 20$)\\
...                   & UPDATE acc1 ($40 \rightarrow 50$)\\
...                   & COMMIT                           \\\pause
...                   & ...                              \\
RETRIEVE acc3 ($110$) &                                  \\
...                   &
      \end{tabular}
    \end{table}
  \end{example}
\end{frame}

\begin{frame}
  \frametitle{Conflicts}

  \begin{itemize}
    \item A reads, B reads
    \begin{itemize}
      \item no problem
    \end{itemize}

    \pause
    \item A reads, B writes
    \begin{itemize}
      \item non-repeatable read (inconsistent analysis)
    \end{itemize}

    \pause
    \item A writes, B reads
    \begin{itemize}
      \item dirty read (uncommitted dependency)
    \end{itemize}

    \pause
    \item A writes, B writes
    \begin{itemize}
      \item dirty write (lost update)
    \end{itemize}
  \end{itemize}
\end{frame}

\subsection{Locking}

\begin{frame}
  \frametitle{Locking}

  \begin{itemize}
    \item transactions lock the tuples they work on
    \begin{itemize}
      \item shared lock (S)
      \item exclusive lock (X)
    \end{itemize}

    \item they release the locks when they are done
  \end{itemize}
\end{frame}

\begin{frame}
  \frametitle{Lock Requests}

  \begin{block}{lock type compatibility matrix}
    \begin{table}
      \begin{tabular}{|c||c|c|c|}\hline
  & - & S & X\\\hline\hline
S & Y & Y & N\\\hline
X & N & N & N\\\hline
      \end{tabular}
    \end{table}
  \end{block}

  \begin{itemize}
    \item if shared lock
    \begin{itemize}
      \item shared lock requests are granted
      \item exclusive lock requests are denied
    \end{itemize}

    \item if exclusive lock, all lock requests are denied
  \end{itemize}
\end{frame}

\begin{frame}
  \frametitle{Locking Protocol}

  \begin{itemize}
    \item the transaction requests a lock depending on the operation it wants
      to perform
    \begin{itemize}
      \item promote a shared lock to an exclusive lock
    \end{itemize}

    \pause
    \item if the request cannot be granted, it starts waiting
    \begin{itemize}
      \item it continues when the transaction that holds the lock releases it
      \item \alert{starvation}
    \end{itemize}
  \end{itemize}
\end{frame}

\begin{frame}
  \frametitle{Two-Phase Locking}

  \begin{itemize}
    \item \alert{two-phase locking}:\\
      after any lock is released there will be no more new lock requests
    \begin{itemize}
      \item expansion phase: gather locks
      \item contraction phase: release locks
    \end{itemize}

    \pause
    \item \alert{two-phase strict locking}:\\
      all locks are released at the end of the transaction
  \end{itemize}
\end{frame}

\begin{frame}[fragile]
  \frametitle{Lost Update}

  \begin{example}
    \begin{table}
      \begin{tabular}{ll}
Transaction A   & Transaction B  \\\hline
...             & ...            \\\pause
RETRIEVE p (S+) & ...            \\\pause
...             & ...            \\
...             & RETRIEVE p (S+)\\\pause
...             & ...            \\
UPDATE p (X-)   & ...            \\
wait            & ...            \\\pause
wait            & UPDATE p (X-)  \\
wait            & wait
      \end{tabular}
    \end{table}
  \end{example}
\end{frame}

\begin{frame}[fragile]
  \frametitle{Uncommitted Dependency}

  \begin{example}
    \begin{table}
      \begin{tabular}{ll}
Transaction A   & Transaction B\\\hline
...             & ...          \\\pause
...             & UPDATE p (X+)\\\pause
...             & ...          \\
RETRIEVE p (S-) & ...          \\
wait            & ...          \\\pause
wait            & ROLLBACK     \\
RETRIEVE p (S+) &              \\
...             &
      \end{tabular}
    \end{table}
  \end{example}
\end{frame}

\begin{frame}[fragile]
  \frametitle{Inconsistent Analysis}

  \begin{example}[sum of accounts: acc1=40, acc2=50, acc3=30]
    \begin{table}
      \begin{tabular}{ll}
Transaction A        & Transaction B   \\\hline
...                  & ...             \\\pause
RETRIEVE acc1 (S+)   & ...             \\\pause
RETRIEVE acc2 (S+)   & ...             \\\pause
...                  & ...             \\
...                  & UPDATE acc3 (X+)\\\pause
...                  & UPDATE acc1 (X-)\\
...                  & wait            \\\pause
RETRIEVE acc3 (S-)   & wait            \\
wait                 & wait
      \end{tabular}
    \end{table}
  \end{example}
\end{frame}

\begin{frame}
  \frametitle{Deadlock}

  \begin{definition}
    \alert{deadlock}:\\
      transactions are waiting for each other to release the locks
  \end{definition}

  \pause
  \begin{itemize}
    \item almost always between two transactions
    \item countermeasures:
    \begin{itemize}
      \item detecting and solving
      \item preventing
    \end{itemize}
  \end{itemize}
\end{frame}

\begin{frame}
  \frametitle{Solving Deadlocks}

  \begin{columns}[t]
    \column{.5\textwidth}
    \begin{example}
      \begin{center}
        \pgfuseimage{wait}
      \end{center}
    \end{example}

    \column{.5\textwidth}
    \begin{itemize}
      \item wait graph

      \pause
      \item choose a \alert{victim} and kill it
    \end{itemize}
  \end{columns}
\end{frame}

\begin{frame}
  \frametitle{Preventing Deadlocks}

  \begin{itemize}
    \item every transaction has a starting timestamp

    \pause
    \item if the lock request of transaction A conflicts with a lock held by
      transaction B:
    \begin{itemize}
      \item \alert{wait-die}: A waits if it is older than B, otherwise it dies\\
        A is rolled back and restarted

      \item \alert{wound-wait}: A waits if it is younger than B, otherwise it
        wounds B\\
        B is rolled back and restarted
    \end{itemize}

    \pause
    \item the timestamp of a restarted transaction is not changed
  \end{itemize}
\end{frame}

\begin{frame}[fragile]
  \frametitle{Lock Statements}

  \begin{block}{shared Lock}
    \begin{lstlisting}
SELECT query FOR SHARE
    \end{lstlisting}
  \end{block}

  \pause
  \begin{block}{exclusive Lock}
    \begin{lstlisting}
SELECT query FOR UPDATE
    \end{lstlisting}
  \end{block}
\end{frame}

\subsection{Isolation Levels}

\begin{frame}
  \frametitle{Isolation Levels}

  \begin{itemize}
    \item if isolation is decreased, concurrency can be increased:
    \begin{itemize}
      \item serializable
      \item repeatable read
      \item read committed
      \item read uncommitted
    \end{itemize}
  \end{itemize}
\end{frame}

\begin{frame}
  \frametitle{Serializable}

  \begin{itemize}
    \item \emph{serial execution}: one transaction starts after the other is
      finished
  \end{itemize}

  \pause
  \begin{example}
    \begin{itemize}
      \item $x=10$
      \item transaction A: $x=x+1$
      \item transaction B: $x=2*x$

      \pause
      \medskip
      \item first A, then B: $x=22$
      \item first B, then A: $x=21$
    \end{itemize}
  \end{example}
\end{frame}

\begin{frame}
  \frametitle{Serializability}

  \begin{itemize}
    \item \alert{serializable}:\\
      the result of concurrent execution is always the same as one of the
      serial executions

    \pause
    \item \emph{if all transactions obey the two-phase locking protocol,\\
      all concurrent executions are serializable}
  \end{itemize}
\end{frame}

\begin{frame}[fragile]
  \frametitle{Read Committed}

  \begin{itemize}
    \item only exclusive locks are held until end of transaction
  \end{itemize}

  \begin{example}
    \begin{table}
      \begin{tabular}{ll}
Transaction A   & Transaction B\\\hline
...             & ...          \\\pause
RETRIEVE p (S+) & ...          \\\pause
...             & ...          \\
release lock    & ...          \\\pause
...             & ...          \\
...             & UPDATE p (X+)\\
...             & COMMIT       \\\pause
RETRIEVE p (S+) &
      \end{tabular}
    \end{table}
  \end{example}
\end{frame}

\begin{frame}
  \frametitle{Phantoms}

  \begin{definition}
    \alert{phantom}:\\
      when query is executed again, new tuples appear in the result
  \end{definition}

  \pause
  \begin{example}
    \begin{itemize}
      \item transaction A computes the average of a customer's account
        balances:\\
        $\frac{100+100+100}{3}=100$

      \pause
      \item transaction B creates a new account with balance $200$ for the same
        customer
      \item transaction A computes again:\\
        $\frac{100+100+100+200}{4}=125$
    \end{itemize}
  \end{example}
\end{frame}

\begin{frame}[fragile]
  \frametitle{Setting Isolation Levels}

  \begin{block}{statement}
    \begin{lstlisting}
SET TRANSACTION ISOLATION LEVEL
  [ SERIALIZABLE | REPEATABLE READ |
    READ COMMITTED | READ UNCOMMITTED ]
    \end{lstlisting}
  \end{block}
\end{frame}

\begin{frame}[fragile]
  \frametitle{Isolation Level Problems}

  \begin{table}
    \begin{tabular}{|l||c|c|c|}\hline
Isolation level  & Dirty & Non-repeatable & Phantom\\
                 & read  & read           &        \\\hline\hline
READ UNCOMMITTED & Y     & Y              & Y      \\\hline
READ COMMITTED   & N     & Y              & Y      \\\hline
REPEATABLE READ  & N     & N              & Y      \\\hline
SERIALIZABLE     & N     & N              & N      \\\hline
    \end{tabular}
  \end{table}
\end{frame}

\subsection{Intent Locks}

\begin{frame}
  \frametitle{Locking Granularity}

  \begin{itemize}
    \item locking relations instead of tuples
    \begin{itemize}
      \item even the entire database
    \end{itemize}

    \item if granularity is increased, concurrency is decreased

    \pause
    \item hard to find locks on tuples\\
      $\rightarrow$ first, get \alert{intent locks} on relation variables
  \end{itemize}
\end{frame}

\begin{frame}
  \frametitle{Intent Locks}

  \begin{itemize}
    \item Intent Shared (IS):\\
      the transaction intends to read some tuples

    \pause
    \item Intent Exclusive (IX):\\
      IS + the transaction intends to write some tuples

    \pause
    \item Shared (S):\\
      concurrent readers are allowed but no concurrent writers

    \pause
    \item Shared + Intent Exclusive (SIX):\\
      S + IX

    \pause
    \item Exclusive (X):\\
      no concurrency allowed on this relation
  \end{itemize}
\end{frame}

\begin{frame}
  \frametitle{Lock Requests}

  \begin{block}{lock compatibility matrix}
    \begin{table}
      \begin{tabular}{|c||c|c|c|c|c|c|}\hline
    & - & IS & S & IX & SIX & X\\\hline\hline
 IS & Y & Y  & Y & Y  &  Y  & N\\\hline
  S & Y & Y  & Y & N  &  N  & N\\\hline
 IX & Y & Y  & N & Y  &  N  & N\\\hline
SIX & Y & Y  & N & N  &  N  & N\\\hline
  X & Y & N  & N & N  &  N  & N\\\hline
      \end{tabular}
    \end{table}
  \end{block}
\end{frame}

\begin{frame}
  \frametitle{Lock Precedence}

  \begin{columns}[t]
    \column{.5\textwidth}
    \begin{center}
      \pgfuseimage{precedence}
    \end{center}

    \pause
    \column{.5\textwidth}
    \begin{itemize}
      \item for a shared lock on a tuple, at least an IS lock on the relation
      \item for an exclusive lock on a tuple, at least an IX lock on the
        relation
    \end{itemize}
  \end{columns}
\end{frame}

\begin{frame}[fragile]
  \frametitle{Locking Statements}

  \begin{block}{statement}
    \begin{lstlisting}
LOCK [ TABLE ] table_name
     [ IN lock_mode MODE ]
    \end{lstlisting}

    \pause
    \begin{itemize}
      \item lock modes:
      \begin{itemize}
        \item \lstinline!ACCESS SHARE!
        \item \lstinline!ROW SHARE!
        \item \lstinline!ROW EXCLUSIVE!
        \item \lstinline!SHARE UPDATE EXCLUSIVE!
        \item \lstinline!SHARE!
        \item \lstinline!SHARE ROW EXCLUSIVE!
        \item \lstinline!EXCLUSIVE!
        \item \lstinline!ACCESS EXCLUSIVE!
      \end{itemize}
    \end{itemize}

  \end{block}
\end{frame}

\subsection*{References}

\begin{frame}
  \frametitle{References}

  \begin{block}{Required text: Date}
    \begin{itemize}
      \item Chapter 16: \alert{Concurrency}
    \end{itemize}
  \end{block}
\end{frame}

\end{document}
