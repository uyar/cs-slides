% Copyright (c) 2002-2014
%       H. Turgut Uyar <uyar@itu.edu.tr>
%       Şule Gündüz Öğüdücü <sgunduz@itu.edu.tr>
%
% This work is licensed under a "Creative Commons
% Attribution-NonCommercial-ShareAlike 4.0 International License".
% For more information, please visit:
% https://creativecommons.org/licenses/by-nc-sa/4.0/

\documentclass[dvipsnames]{beamer}

\usepackage{ae}
\usepackage[scaled=0.88]{beramono}
\usepackage[T1]{fontenc}
\usepackage[utf8]{inputenc}
\setbeamersize{text margin left=2em, text margin right=2em}
\usepackage[labelformat=empty, aboveskip=1pt, belowskip=1pt]{caption}

\usepackage{listings}
\lstdefinelanguage{FullSQL}[]{SQL}{
  morekeywords={BINARY, BOOLEAN, CYCLE, FINAL, INCREMENT, IS, LARGE, MAXVALUE,
                MINVALUE, NO_ACTION, OBJECT, REFERENCES, RENAME, SEQUENCE,
                START, TO, TYPE, VACUUM}
}
\lstdefinelanguage{ExtendedSQL}[]{FullSQL}{
  morekeywords={AFTER, BEFORE, DO, EACH, FOR, FUNCTION, INSTEAD, LANGUAGE,
                OPTION, PROCEDURE, RETURNS, ROW, RULE, SNAPSHOT, STATEMENT,
                WITH}
}
\lstset{basicstyle=\ttfamily, keywordstyle=\color{ForestGreen},
        showstringspaces=false}

\mode<presentation>
{
  \usetheme{Warsaw}
  \usecolortheme[named=ForestGreen]{structure}
  \setbeamercovered{transparent}
}

\title{Database Systems}
\subtitle{Database Design}

\author{H. Turgut Uyar \and Şule Öğüdücü}
\date{2002-2014}

\AtBeginSubsection[]{
  \begin{frame}<beamer>
    \frametitle{Topics}
    \tableofcontents[currentsection,currentsubsection]
  \end{frame}
}

\pgfdeclareimage[width=2cm]{license}{../license}

\pgfdeclareimage{norm1}{norm1}
\pgfdeclareimage{norm2}{norm2}
\pgfdeclareimage{norm3}{norm3}
\pgfdeclareimage{imdb1}{imdb1}
\pgfdeclareimage{imdb2}{imdb2}

\begin{document}

\begin{frame}
  \titlepage
\end{frame}

\begin{frame}
  \frametitle{License}

  \pgfuseimage{license}\hfill
  \copyright~2002-2014 T. Uyar, Ş. Öğüdücü

  \vfill
  \begin{footnotesize}
    You are free to:
    \begin{itemize}
      \itemsep0em
      \item Share -- copy and redistribute the material in any medium or format
      \item Adapt -- remix, transform, and build upon the material
    \end{itemize}

    Under the following terms:
    \begin{itemize}
      \itemsep0em
      \item Attribution -- You must give appropriate credit, provide a link to
        the license, and indicate if changes were made.

      \item NonCommercial -- You may not use the material for commercial
        purposes.

      \item ShareAlike -- If you remix, transform, or build upon the material,
        you must distribute your contributions under the same license as the
        original.
    \end{itemize}
  \end{footnotesize}

  \begin{small}
    For more information:\\
    \url{https://creativecommons.org/licenses/by-nc-sa/4.0/}

    \smallskip
    Read the full license:\\
    \url{https://creativecommons.org/licenses/by-nc-sa/4.0/legalcode}
  \end{small}
\end{frame}

\begin{frame}
  \frametitle{Topics}
  \tableofcontents
\end{frame}

\section{Normalization}

\subsection{Introduction}

\begin{frame}
  \frametitle{Functional Dependency}

  \begin{itemize}
    \item let $Z$ be the set of all attributes of the relation $R$
    \item let $A,B \subseteq Z$
    \item \alert{$A$ functionally determines $B$}: $A \rightarrow B$\\
      for every $A$ value there can only be one $B$ value

    \pause
    \medskip
    \item every functional dependency is an integrity constraint
  \end{itemize}
\end{frame}

\begin{frame}[label=example_db_1]
  \frametitle{Example Relation}

  \begin{tiny}
  \begin{table}
    \caption{R}
    \begin{tabular}{|r|l|c|c|r|l|r|}\hline
\underline{MOVIEID} & TITLE    & COU & LANG & \underline{ACTORID} & NAME & ORD\\[2pt]\hline\hline
      6 & Usual Suspects       & UK  &  EN  &     308 & Gabriel Byrne    &   2\\\hline
    228 & Ed Wood              & US  &  EN  &      26 & Johnny Depp      &   1\\\hline
     70 & Being John Malkovich & US  &  EN  &     282 & Cameron Diaz     &   2\\\hline
   1512 & Suspiria             & IT  &  IT  &     745 & Udo Kier         &   9\\\hline
     70 & Being John Malkovich & US  &  EN  &     503 & John Malkovich   &  14\\\hline
    \end{tabular}
  \end{table}
  \end{tiny}

  \pause
  \begin{itemize}
    \item assumption: the language of the movie\\
      is the language of the country where it was made
  \end{itemize}
\end{frame}

\begin{frame}
  \frametitle{Functional Dependency Examples}

  \begin{itemize}
    \item MOVIEID $\rightarrow$ TITLE
    \item MOVIEID $\rightarrow$ \{TITLE, COUNTRY, LANGUAGE\}
    \item ACTORID $\rightarrow$ NAME
    \item \{MOVIEID, ACTORID\} $\rightarrow$ ORD

    \pause
    \medskip
    \item trivial: MOVIEID $\rightarrow$ MOVIEID
    \item redundant: \{MOVIEID, ACTORID\} $\rightarrow$ COUNTRY
  \end{itemize}
\end{frame}

\begin{frame}
  \frametitle{Irreducible Set}

  \begin{itemize}
    \item let $S$ be the set of all FDs of the relation
    \item let $T \subseteq S$

    \medskip
    \item $T$ is an irreducible set of FDs if:

    \smallskip
    \item $T$ contains as few elements as possible, and
    \item every FD in $S$ can be derived from the FDs in $T$

    \pause
    \medskip
    \item let there be only one attribute on the right hand side of FDs
  \end{itemize}
\end{frame}

\begin{frame}
  \frametitle{Irreducible Set Example}

  \begin{itemize}
    \item MOVIEID $\rightarrow$ TITLE
    \item MOVIEID $\rightarrow$ COUNTRY
    \item COUNTRY $\rightarrow$ LANGUAGE
    \item ACTORID $\rightarrow$ NAME
    \item \{MOVIEID, ACTORID\} $\rightarrow$ ORD
  \end{itemize}
\end{frame}

\begin{frame}
  \frametitle{Dependency Example}

  \begin{center}
    \pgfuseimage{norm1}
  \end{center}
\end{frame}

\subsection{Normal Forms}

\begin{frame}
  \frametitle{Normal Forms}

  \begin{itemize}
    \item \alert{normal form}: set of conditions that relations must satisfy

    \medskip
    \item 1NF, 2NF, 3NF, BCNF, 4NF, 5NF
    \item every form adds stricter conditions to the previous form
    \item every relation in 2NF is also in 1NF,\\
      every relation in 3NF is also in 2NF, ...
    \item NF of database = NF of weakest relation

    \pause
    \medskip
    \item \alert{1NF}: attribute values have to be atomic
  \end{itemize}
\end{frame}

\begin{frame}
  \frametitle{Normalization}

  \begin{itemize}
    \item \alert{normalization}:\\
      transforming relations from one form to the next, stricter form

    \item transformation must be lossless
  \end{itemize}

  \pause
  \begin{theorem}[Heath]
    \begin{itemize}
      \item let $Z$ be the set of all attributes of the relation $R$
      \item let $A,B,C \subseteq Z$

      \medskip
      \item $A \rightarrow B \Rightarrow R = (A,B) ~join~ (A,C)$
    \end{itemize}
  \end{theorem}
\end{frame}

\begin{frame}
  \frametitle{Lossless Transition Example}

  \begin{columns}[c]
    \column{.65\textwidth}
    \vspace{-12pt}
    \begin{footnotesize}
    \begin{table}
      \caption{R1}
      \vspace{-6pt}
      \begin{tabular}{|r|l|c|c|}\hline
MOVIEID & TITLE                & COU & LANG\\\hline\hline
      6 & Usual Suspects       & UK  &  EN \\\hline
    228 & Ed Wood              & US  &  EN \\\hline
     70 & Being John Malkovich & US  &  EN \\\hline
   1512 & Suspiria             & IT  &  IT \\\hline
      \end{tabular}
    \end{table}
    \end{footnotesize}

    \vspace{-12pt}
    \begin{footnotesize}
    \begin{table}
      \caption{R2}
      \begin{tabular}{|r|r|l|r|}\hline
MOVIEID & ACTORID & NAME           & ORD\\\hline\hline
      6 &     308 & Gabriel Byrne  &   2\\\hline
    228 &      26 & Johnny Depp    &   1\\\hline
     70 &     282 & Cameron Diaz   &   2\\\hline
   1512 &     745 & Udo Kier       &   9\\\hline
     70 &     503 & John Malkovich &  14\\\hline
      \end{tabular}
    \end{table}
    \end{footnotesize}

    \column{.35\textwidth}
    \begin{itemize}
      \item $R = R1 ~join~ R2$
    \end{itemize}
  \end{columns}
\end{frame}

\begin{frame}
  \frametitle{Lossy Transition Example}

  \begin{columns}[c]
    \column{.62\textwidth}
    \vspace{-12pt}
    \begin{footnotesize}
    \begin{table}
      \caption{R1}
      \vspace{-6pt}
      \begin{tabular}{|r|l|c|c|}\hline
MOVIEID & TITLE                & COU & LANG\\\hline\hline
      6 & Usual Suspects       & UK  &  EN \\\hline
    228 & Ed Wood              & US  &  EN \\\hline
     70 & Being John Malkovich & US  &  EN \\\hline
   1512 & Suspiria             & IT  &  IT \\\hline
      \end{tabular}
    \end{table}
    \end{footnotesize}

    \vspace{-12pt}
    \begin{footnotesize}
    \begin{table}
      \caption{R2}
      \begin{tabular}{|c|r|l|r|}\hline
COU & ACTORID & NAME           & ORD\\\hline\hline
UK  &     308 & Gabriel Byrne  &   2\\\hline
US  &      26 & Johnny Depp    &   1\\\hline
US  &     282 & Cameron Diaz   &   2\\\hline
IT  &     745 & Udo Kier       &   9\\\hline
US  &     503 & John Malkovich &  14\\\hline
      \end{tabular}
    \end{table}
    \end{footnotesize}

    \column{.38\textwidth}
    \begin{itemize}
      \item $R \neq R1 ~join~ R2$

      \pause
      \item \footnotesize{\{MOVIEID, ACTORID\} \\ $\rightarrow$ ORD}
    \end{itemize}
  \end{columns}
\end{frame}

\begin{frame}
  \frametitle{Anomalies}

  \begin{itemize}
    \item \emph{insert}: data is known but can not be inserted

    \medskip
    \item \emph{delete}: deleting some data causes some other data to be lost

    \medskip
    \item \emph{update}: updating data requires modifications in multiple tuples
  \end{itemize}
\end{frame}

\begin{frame}
  \frametitle{Anomaly Examples}

  \hyperlink{example_db_1}{\beamergotobutton{example database}}

  \begin{itemize}
    \item the country of the movie ``Gattaca'' is known to be US,\\
      but this cannot be inserted if there is no actor in the movie

    \pause
    \medskip
    \item deleting that Gabriel Byrne acts in the movie ``Usual Suspects''\\
      also deletes that the movie was made in the UK

    \pause
    \medskip
    \item changing the country of the movie ``Being John Malkovich''\\
      requires modifications in two tuples
  \end{itemize}
\end{frame}

\begin{frame}
  \frametitle{2nd Normal Form}

  \begin{itemize}
    \item \alert{2NF}: every non-key attribute depends on the primary key

    \pause
    \medskip
    \item in a relation $R$ that conforms to 1NF, if:
    \item $R(A,B,C,D)$, primary key: $(A,B)$, and\\
      $A \rightarrow D$

    \medskip
    \item to transform to 2NF, divide into:
    \item $R1(A,D)$, primary key: $A$, and\\
      $R2(A,B,C)$, primary key: $(A,B)$, where\\
      $A$ is a foreign key referencing $R1$
  \end{itemize}
\end{frame}

\begin{frame}
  \frametitle{1NF-2NF Transition Example}

  \begin{itemize}
    \item among non-key attributes, only ORD depends on primary key
    \item $A$: \{MOVIEID\}\\
      $B$: \{ACTORID\}\\
      $C$: \{NAME, ORD\}\\
      $D$: \{TITLE, COUNTRY, LANGUAGE\}

    \pause
    \medskip
    \item R1(MOVIEID, TITLE, COUNTRY, LANGUAGE)\\
      primary key: MOVIEID

    \item R2(MOVIEID, ACTORID, NAME, ORD)\\
      primary key: \{MOVIEID, ACTORID\}\\
      MOVIEID is a foreign key referencing R1
  \end{itemize}
\end{frame}

\begin{frame}
  \frametitle{1NF-2NF Transition Example}

  \begin{itemize}
    \item R2 still not 2NF: ACTORID $\rightarrow$ NAME
    \item $A$: \{ACTORID\}\\
      $B$: \{MOVIEID\}\\
      $C$: \{ORD\}\\
      $D$: \{NAME\}

    \pause
    \medskip
    \item R3(ACTORID, NAME)\\
      primary key: ACTORID

    \item R4(MOVIEID, ACTORID, ORD)\\
      primary key: \{MOVIEID, ACTORID\}\\
      ACTORID is a foreign key referencing R3
  \end{itemize}
\end{frame}

\begin{frame}[label=example_db_2]
  \frametitle{2NF Relation Examples}

  \vspace{-12pt}
  \begin{center}
    \begin{footnotesize}
    \begin{table}
      \caption{R1}
      \begin{tabular}{|r|l|c|c|}\hline
\underline{MOVIEID} & TITLE    & COU & LANG\\[2pt]\hline\hline
      6 & Usual Suspects       & UK  &  EN \\\hline
    228 & Ed Wood              & US  &  EN \\\hline
     70 & Being John Malkovich & US  &  EN \\\hline
   1512 & Suspiria             & IT  &  IT \\\hline
      \end{tabular}
    \end{table}
    \end{footnotesize}
  \end{center}

  \vspace{-12pt}
  \begin{columns}[t]
    \column{.5\textwidth}
    \begin{footnotesize}
    \begin{table}
      \caption{R3}
      \begin{tabular}{|r|l|}\hline
\underline{ACTORID} & NAME\\[2pt]\hline\hline
     308 & Gabriel Byrne \\\hline
      26 & Johnny Depp   \\\hline
     282 & Cameron Diaz  \\\hline
     745 & Udo Kier      \\\hline
     503 & John Malkovich\\\hline
      \end{tabular}
    \end{table}
    \end{footnotesize}

    \column{.5\textwidth}
    \begin{footnotesize}
    \begin{table}
      \caption{R4}
      \begin{tabular}{|r|r|r|}\hline
\underline{MOVIEID} & \underline{ACTORID} & ORD\\[2pt]\hline\hline
   6 & 308 &  2\\\hline
 228 &  26 &  1\\\hline
  70 & 282 &  2\\\hline
1512 & 745 &  9\\\hline
  70 & 503 & 14\\\hline
      \end{tabular}
    \end{table}
    \end{footnotesize}
  \end{columns}
\end{frame}

\begin{frame}
  \frametitle{Dependency Example}

  \begin{center}
    \pgfuseimage{norm2}
  \end{center}
\end{frame}

\begin{frame}
  \frametitle{2NF Corrected Anomalies}

  \hyperlink{example_db_2}{\beamergotobutton{example database}}

  \begin{itemize}
    \item if the country of the movie ``Gattaca'' is known to be US,\\
      this can be inserted to R1

    \pause
    \medskip
    \item if Gabriel Byrne is deleted from the movie ``Usual Suspects'',\\
      the country of the movie is still kept in R1

    \pause
    \medskip
    \item changing the country of the movie ``Being John Malkovich''\\
      requires updating only one tuple in R1
  \end{itemize}
\end{frame}

\begin{frame}
  \frametitle{2NF Remaining Anomalies}

  \hyperlink{example_db_2}{\beamergotobutton{example database}}

  \begin{itemize}
    \item it is known that movies made in Brazil are in Portuguese\\
      but this can not be inserted if there is no movie made in Brazil

    \pause
    \medskip
    \item deleting the movie ``Suspiria'' also deletes that\\
      the language of movies made in Italy is Italian

    \pause
    \medskip
    \item changing the language of the movies made in the US\\
      requires two tuples to be updated
  \end{itemize}
\end{frame}

\subsection{3rd Normal Form}

\begin{frame}
  \frametitle{3rd Normal Form}

  \begin{itemize}
    \item \alert{3NF}: non-key attributes do not depend on any attributes\\
      other than the primary key

    \pause
    \medskip
    \item in a relation $R$ that conforms to 2NF, if:
    \item $R(A,B,C,D)$, primary key: $A$, and\\
      $C \rightarrow D$

    \medskip
    \item to transform to 3NF, divide into:
    \item $R1(C,D)$, primary key: $C$, and\\
      $R2(A,B,C)$, primary key: $A$, where\\
      $C$ is a foreign key referencing $R1$
  \end{itemize}
\end{frame}

\begin{frame}
  \frametitle{2NF-3NF Transition Example}

  \begin{itemize}
    \item R1: COUNTRY $\rightarrow$ LANGUAGE
    \item $A$: \{MOVIEID\}\\
      $B$: \{TITLE\}\\
      $C$: \{COUNTRY\}\\
      $D$: \{LANGUAGE\}

    \pause
    \medskip
    \item R5(COUNTRY, LANGUAGE)\\
      primary key: COUNTRY

    \item R6(MOVIEID, TITLE, COUNTRY)\\
      primary key: MOVIEID\\
      COUNTRY is a foreign key referencing R5
  \end{itemize}
\end{frame}

\begin{frame}[label=example_db_3]
  \frametitle{3NF Relation Examples}

  \vspace{-24pt}
  \begin{columns}[t]
    \column{.6\textwidth}
    \begin{footnotesize}
    \begin{table}
      \caption{R6}
      \begin{tabular}{|r|l|c|}\hline
\underline{MOVIEID} & TITLE    & COU\\[2pt]\hline\hline
      6 & Usual Suspects       & UK \\\hline
    228 & Ed Wood              & US \\\hline
     70 & Being John Malkovich & US \\\hline
   1512 & Suspiria             & IT \\\hline
      \end{tabular}
    \end{table}
    \end{footnotesize}

    \column{.4\textwidth}
    \begin{footnotesize}
    \begin{table}
      \caption{R5}
      \begin{tabular}{|c|c|}\hline
\underline{COU} & LANG\\[2pt]\hline\hline
UK & EN\\\hline
US & EN\\\hline
IT & IT\\\hline
      \end{tabular}
    \end{table}
    \end{footnotesize}
  \end{columns}

  \vspace{-24pt}
  \begin{columns}[t]
    \column{.5\textwidth}
    \begin{footnotesize}
    \begin{table}
      \caption{R3}
      \begin{tabular}{|r|l|}\hline
\underline{ACTORID} & NAME\\[2pt]\hline\hline
      308 & Gabriel Byrne \\\hline
       26 & Johnny Depp   \\\hline
      282 & Cameron Diaz  \\\hline
      745 & Udo Kier      \\\hline
      503 & John Malkovich\\\hline
      \end{tabular}
    \end{table}
    \end{footnotesize}

    \column{.5\textwidth}
    \begin{footnotesize}
    \begin{table}
      \caption{R4}
      \begin{tabular}{|r|r|r|}\hline
\underline{MOVIEID} & \underline{ACTORID} & ORD\\[2pt]\hline\hline
   6 & 308 &  2\\\hline
 228 &  26 &  1\\\hline
  70 & 282 &  2\\\hline
1512 & 745 &  9\\\hline
  70 & 503 & 14\\\hline
      \end{tabular}
    \end{table}
    \end{footnotesize}
  \end{columns}
\end{frame}

\begin{frame}
  \frametitle{Dependency Example}

  \begin{center}
    \pgfuseimage{norm3}
  \end{center}
 \end{frame}

\begin{frame}
  \frametitle{3NF Corrected Anomalies}

  \hyperlink{example_db_3}{\beamergotobutton{example database}}

  \begin{itemize}
    \item if movies made in Brazil are in known to be in Portuguese,\\
      this can be inserted into R5

    \pause
    \medskip
    \item if the movie "Suspiria" is deleted, R5 still keeps\\
      that movies made in Italy are in Italian

    \pause
    \medskip
    \item changing the language of the movies made in the US\\
      requires modifying only one tuple in R5
  \end{itemize}
\end{frame}

\begin{frame}
  \frametitle{Boyce-Codd Normal Form}

  \begin{itemize}
    \item \alert{BCNF}: all functional dependencies must be on candidate keys
    \item consider dependencies between key attributes
  \end{itemize}
\end{frame}

\begin{frame}
  \frametitle{BCNF Example}

  \hyperlink{example_db_1}{\beamergotobutton{example database}}

  \begin{itemize}
    \item let movie titles be unique
    \item candidate keys:\\
      \{MOVIEID, ACTORID\}\\
      \{TITLE, ACTORID\}

    \pause
    \medskip
    \item non-conforming functional dependencies:\\
      MOVIEID $\rightarrow$ TITLE\\
      TITLE $\rightarrow$ MOVIEID
  \end{itemize}
\end{frame}

\subsection{Views}

\lstset{language=ExtendedSQL}

\begin{frame}[fragile]
  \frametitle{Views}

  \begin{itemize}
    \item presenting a derived table like a base table: \alert{view}
    \item isolating users and application programs\\
      from changes in database structure

    \pause
    \bigskip
    \item creating a view:
    \begin{lstlisting}
CREATE VIEW view_name AS
  SELECT ...
    \end{lstlisting}

    \medskip
    \item \lstinline!SELECT! will be executed every time the view is used
  \end{itemize}
\end{frame}

\begin{frame}[fragile]
  \frametitle{View Example}

  \begin{itemize}
    \item identifiers, titles and years of new movies

    \medskip
    \begin{lstlisting}
CREATE VIEW NEW_MOVIE AS
  SELECT ID, TITLE, YR FROM MOVIE
    WHERE (YR > 1995)

SELECT * FROM NEW_MOVIE
    \end{lstlisting}
  \end{itemize}
\end{frame}

\begin{frame}[fragile]
  \frametitle{Updating Views}

  \begin{itemize}
    \item any change will have to performed on the base tables
    \item rules need to be defined

    \pause
    \bigskip
    \item creating a rule:
    \begin{lstlisting}
CREATE RULE rule_name AS
  ON event TO view_name
  [ WHERE condition ]
  DO [ INSTEAD ] sql_statement
    \end{lstlisting}
  \end{itemize}
\end{frame}

\begin{frame}[fragile]
  \frametitle{View Rule Example}

  \begin{itemize}
    \item modify the title of a new movie
    \begin{lstlisting}
UPDATE NEW_MOVIE SET TITLE = ...
  WHERE (ID = ...)
    \end{lstlisting}

    \pause
    \medskip
    \item rule for updating the base table
    \begin{lstlisting}
CREATE RULE UPDATE_TITLE AS
  ON UPDATE TO NEW_MOVIE
  DO INSTEAD
    UPDATE MOVIE SET TITLE = new.TITLE
      WHERE (ID = old.ID)
    \end{lstlisting}
  \end{itemize}
\end{frame}

\subsection*{References}

\begin{frame}
  \frametitle{References}

  \begin{block}{Required Reading: Date}
    \begin{itemize}
      \item Chapter 11: \alert{Functional Dependencies}
      \item Chapter 12: \alert{Further Normalization I: 1NF, 2NF, 3NF, BCNF}
      \item Chapter 10: \alert{Views}
    \end{itemize}
  \end{block}
\end{frame}

\section{Entity/Relationship Model}

\subsection{Introduction}

\begin{frame}
  \frametitle{Entity/Relationship Model}

  \begin{itemize}
    \item modeling approach (Chen 1976)

    \medskip
    \item entities
    \item properties
    \item relationships
  \end{itemize}
\end{frame}

\begin{frame}
  \frametitle{Entities}

  \begin{itemize}
    \item \alert{entity}: set of ``things'' with the same attributes
    \item elements of the set are \emph{instances} of the entity

    \medskip
    \item \emph{strong}: can exist by itself
    \item \emph{weak}: existence depends on other entities
  \end{itemize}
\end{frame}

\begin{frame}
  \frametitle{Entity Examples}

  \begin{itemize}
    \item entity: movie, person
    \item person instance: Johnny Depp

    \pause
    \medskip
    \item strong entity: person
    \item weak entity: movie
  \end{itemize}
\end{frame}

\begin{frame}
  \frametitle{Properties}

  \begin{itemize}
    \item \alert{property}: data describing entity

    \medskip
    \item simple / composite
    \item key
    \item single / multiple valued
    \item empty
    \item base / derived
  \end{itemize}
\end{frame}

\begin{frame}
  \frametitle{Property Examples}

  \begin{itemize}
    \item property: title, country, language

    \pause
    \medskip
    \item simple: first name, last name
    \item composite: full name

    \pause
    \medskip
    \item base: date of birth
    \item derived: age
  \end{itemize}
\end{frame}

\begin{frame}
  \frametitle{Relationships}

  \begin{itemize}
    \item \alert{relationship}: connection between entities

    \medskip
    \item \emph{participant}: entities in the relationship
    \item \emph{degree}: number of participants
    \item \emph{total}: all instances of the entity
      participate in the relationship\\
      (otherwise \emph{partial})
  \end{itemize}
\end{frame}

\begin{frame}
  \frametitle{Relationship Types}

  \begin{itemize}
    \item \alert{one-to-one}
    \item e.g. capital city relationship between countries and cities

    \pause
    \medskip
    \item \alert{one-to-many}
    \item e.g. management relationship between employees and projects

    \pause
    \medskip
    \item \alert{many-to-many}
    \item e.g. enrollment relationship between students and courses
  \end{itemize}
\end{frame}

\subsection{E/R Diagrams}

\begin{frame}
  \frametitle{Entity/Relationship Diagrams}

  \begin{itemize}
    \item entity: box
    \item weak: double lines

    \pause
    \medskip
    \item property: circle
    \item composite: sub-circles

    \pause
    \medskip
    \item relationship: diamond
    \item between weak and strong: double lines
    \item total: connection double lines
    \item 1 or n depending on the type of the relationship
 \end{itemize}
\end{frame}

\begin{frame}
  \frametitle{Entity/Relationship Diagram Example}

  \begin{center}
    \pgfuseimage{imdb1}
  \end{center}
\end{frame}

\begin{frame}
  \frametitle{Entity/Relationship Diagrams}

  \begin{itemize}
    \item properties excluded if diagram too large

    \bigskip
    \item alternative representation for relationship: line
    \item name of relationship above the line
    \item fork at the end of line if ``many''
    \item circle at the end of line: empty if partial, filled if total

    \medskip
    \item no many-to-many relationships
    \item add an entity and use two one-to-many relationships
 \end{itemize}
\end{frame}

\begin{frame}
  \frametitle{Entity/Relationship Diagram Example}

  \begin{center}
    \pgfuseimage{imdb2}
  \end{center}
\end{frame}

\begin{frame}
  \frametitle{Applying to Design}

  \begin{itemize}
    \item every entity is a relation
    \item every property is an attribute

    \pause
    \medskip
    \item every one-to-many relationship is a foreign key\\
      from the ``many'' side to the ``one'' side

    \pause
    \medskip
    \item every many-to-many relationship is a relation\\
      with foreign keys to participating entities
 \end{itemize}
\end{frame}

\subsection*{References}

\begin{frame}
  \frametitle{References}

  \begin{block}{Required Reading: Date}
    \begin{itemize}
      \item Chapter 14: \alert{Semantic Modeling}
    \end{itemize}
  \end{block}
\end{frame}

\end{document}
