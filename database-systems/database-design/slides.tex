% Copyright (c) 2002-2011
%       H. Turgut Uyar <uyar@itu.edu.tr>
%       Şule Gündüz Öğüdücü <sgunduz@itu.edu.tr>
%
% These notes are licensed using the
% "Creative Commons Attribution-NonCommercial-ShareAlike License".
% You are free to copy, distribute and transmit the work, and to adapt the work
% as long as you attribute the authors, do not use it for commercial purposes,
% and any derivative work is under the same or a similar license.
%
% Read the full legal code at:
% http://creativecommons.org/licenses/by-nc-sa/3.0/

\documentclass[dvipsnames]{beamer}

\usepackage{ae}
\usepackage[T1]{fontenc}
\usepackage[utf8]{inputenc}
\setbeamertemplate{navigation symbols}{}

\mode<presentation>
{
  \usetheme{Warsaw}
  \usecolortheme[named=ForestGreen]{structure}
  \setbeamercovered{transparent}
}

\title{Database Systems}
\subtitle{Database Design}

\author{H. Turgut Uyar \and Şule Öğüdücü}
\date{2002-2011}

\AtBeginSubsection[]{
  \begin{frame}<beamer>
    \frametitle{Topics}
    \tableofcontents[currentsection,currentsubsection]
  \end{frame}
}

\pgfdeclareimage[width=2cm]{license}{../../license}

\pgfdeclareimage{norm1}{norm1}
\pgfdeclareimage{norm2}{norm2}
\pgfdeclareimage{norm3}{norm3}
\pgfdeclareimage{imdb1}{imdb1}
\pgfdeclareimage{imdb2}{imdb2}

\begin{document}

\begin{frame}
  \titlepage
\end{frame}

\begin{frame}
  \frametitle{License}

  \pgfuseimage{license}\hfill
  \copyright 2002-2011 T. Uyar, Ş. Öğüdücü

  \vfill
  \begin{tiny}
    You are free:
    \begin{itemize}
      \item to Share — to copy, distribute and transmit the work
      \item to Remix — to adapt the work
    \end{itemize}

    Under the following conditions:
    \begin{itemize}
      \item Attribution — You must attribute the work in the manner specified by
        the author or licensor (but not in any way that suggests that they
        endorse you or your use of the work).

      \item Noncommercial — You may not use this work for commercial purposes.

      \item Share Alike — If you alter, transform, or build upon this work, you
        may distribute the resulting work only under the same or similar license
        to this one.
    \end{itemize}
  \end{tiny}

  \vfill
  Legal code (the full license):\\
  \url{http://creativecommons.org/licenses/by-nc-sa/3.0/}
\end{frame}

\begin{frame}
  \frametitle{Topics}
  \tableofcontents
\end{frame}

\section{Normalization}

\subsection{Introduction}

\begin{frame}
  \frametitle{Functional Dependency}

  \begin{definition}
    \begin{itemize}
      \item $Z$: the set of all attributes of relation $R$
      \item $A,B \subseteq Z$

      \pause
      \medskip
      \item \alert{$A$ functionally determines $B$}: $A \rightarrow B$\\
        for every $A$ value there is only one $B$ value
    \end{itemize}
  \end{definition}

  \pause
  \begin{itemize}
    \item every functional dependency is an integrity constraint
  \end{itemize}
\end{frame}

\begin{frame}
  \frametitle{Functional Dependency Examples}

  \begin{example}
    \begin{tiny}
    \begin{table}
      \caption{R}
      \begin{tabular}{|r|l|l|l|r|l|r|}\hline
\underline{MOVIEID} & TITLE    & COU & LANG & \underline{ACTORID} & NAME & ORD\\\hline\hline
      6 & Usual Suspects       & UK  &   EN &     308 & Gabriel Byrne    &   2\\\hline
    228 & Ed Wood              & US  &   EN &      26 & Johnny Depp      &   1\\\hline
     70 & Being John Malkovich & US  &   EN &     282 & Cameron Diaz     &   2\\\hline
   1512 & Suspiria             & IT  &   IT &     745 & Udo Kier         &   9\\\hline
     70 & Being John Malkovich & US  &   EN &     503 & John Malkovich   &  14\\\hline
      \end{tabular}
    \end{table}
    \end{tiny}

    \pause
    \begin{itemize}
      \item assumption: the language of the movie\\
	is the language of the country where it was made
    \end{itemize}
  \end{example}
\end{frame}

\begin{frame}
  \frametitle{Functional Dependency Examples}

  \begin{example}
    \begin{itemize}
      \item MOVIEID $\rightarrow$ COUNTRY

      \pause
      \item ACTORID $\rightarrow$ NAME

      \pause
      \item MOVIEID $\rightarrow$ \{TITLE, COUNTRY, LANGUAGE\}

      \pause
      \item \{MOVIEID, ACTORID\} $\rightarrow$ COUNTRY

      \pause
      \item \{MOVIEID, ACTORID\} $\rightarrow$ MOVIEID

      \pause
      \item \{MOVIEID, ACTORID\} $\rightarrow$ ORD

      \pause
      \item \{MOVIEID, ACTORID\} $\rightarrow$ \{COUNTRY, ORD\}

      \pause
      \item COUNTRY $\rightarrow$ LANGUAGE
    \end{itemize}
  \end{example}
\end{frame}

\begin{frame}
  \frametitle{Irreducible Set}

  \begin{itemize}
    \item $S$: the set of all FDs of the relation

    \item $T \subseteq S$, such that
    \begin{itemize}
      \item $T$ contains as few elements as possible
      \item every FD in $S$ can be derived by the FDs in $T$
    \end{itemize}

    \pause
    \medskip
    \item let there be only one attribute on the right hand side of FDs
  \end{itemize}
\end{frame}

\begin{frame}
  \frametitle{Irreducible Set Example}

  \begin{example}
    \begin{itemize}
      \item MOVIEID $\rightarrow$ TITLE
      \item MOVIEID $\rightarrow$ COUNTRY
      \item COUNTRY $\rightarrow$ LANGUAGE
      \item ACTORID $\rightarrow$ NAME
      \item \{MOVIEID, ACTORID\} $\rightarrow$ ORD
    \end{itemize}
  \end{example}
\end{frame}

\begin{frame}
  \frametitle{Dependence Diagram}

  \begin{example}
    \begin{center}
      \pgfuseimage{norm1}
    \end{center}
  \end{example}
\end{frame}

\subsection{Normal Forms}

\begin{frame}
  \frametitle{Normal Forms}

  \begin{itemize}
    \item 1NF, 2NF, 3NF, BCNF, 4NF, 5NF

    \item every form narrows down the scope of the previous form
    \begin{itemize}
      \item every relation in 2NF is also in 1NF
      \item every relation in 3NF is also in 2NF, ...
    \end{itemize}

    \pause
    \medskip
    \item 1NF: attribute values are atomic
  \end{itemize}
\end{frame}

\begin{frame}
  \frametitle{Normalization}

  \begin{definition}
    \alert{normalization}:\\
      transition from one form to the next, narrower form

    \begin{itemize}
      \item transition between normal forms must be lossless
    \end{itemize}
  \end{definition}

  \pause
  \begin{theorem}[Heath]
    \begin{itemize}
      \item $Z$: $R$ the set of all attributes of the relation
      \item $A,B,C \subseteq Z$

      \pause
      \item if $A \rightarrow B$, then $R$ can be obtained\\
	by joining the relations $\{A,B\}$ and $\{A,C\}$
    \end{itemize}
  \end{theorem}
\end{frame}

\begin{frame}
  \frametitle{Lossless Transition Example}

  \begin{example}
    \begin{columns}[c]
      \column{.55\textwidth}
      \begin{tiny}
      \begin{table}
        \caption{R1}
        \begin{tabular}{|r|l|l|l|}\hline
MOVIEID & TITLE                & COU & LANG\\\hline\hline
      6 & Usual Suspects       & UK  & EN  \\\hline
    228 & Ed Wood              & US  & EN  \\\hline
     70 & Being John Malkovich & US  & EN  \\\hline
   1512 & Suspiria             & IT  & IT  \\\hline
        \end{tabular}
      \end{table}
      \end{tiny}

      \vspace{-0.7cm}
      \begin{tiny}
      \begin{table}
        \caption{R2}
        \begin{tabular}{|r|r|l|r|}\hline
MOVIEID & ACTORID & NAME           & ORD\\\hline\hline
      6 &     308 & Gabriel Byrne  &   2\\\hline
    228 &      26 & Johnny Depp    &   1\\\hline
     70 &     282 & Cameron Diaz   &   2\\\hline
   1512 &     745 & Udo Kier       &   9\\\hline
     70 &     503 & John Malkovich &  14\\\hline
        \end{tabular}
      \end{table}
      \end{tiny}

      \pause
      \column{.42\textwidth}
      \begin{itemize}
        \item $R = natjoin~(R1)~(R2)$
      \end{itemize}
    \end{columns}
  \end{example}
\end{frame}

\begin{frame}
  \frametitle{Lossy Transition Example}

  \begin{example}
    \begin{columns}[c]
      \column{.55\textwidth}
      \begin{tiny}
      \begin{table}
        \caption{R1}
        \begin{tabular}{|r|l|l|l|}\hline
MOVIEID & TITLE                & COU & LANG\\\hline\hline
      6 & Usual Suspects       & UK  & EN  \\\hline
    228 & Ed Wood              & US  & EN  \\\hline
     70 & Being John Malkovich & US  & EN  \\\hline
   1512 & Suspiria             & IT  & IT  \\\hline
        \end{tabular}
      \end{table}
      \end{tiny}

      \vspace{-0.7cm}
      \begin{tiny}
      \begin{table}
        \caption{R2}
        \begin{tabular}{|l|r|l|r|}\hline
COU & ACTORID & NAME           & ORD\\\hline\hline
UK  &     308 & Gabriel Byrne  &   2\\\hline
US  &      26 & Johnny Depp    &   1\\\hline
US  &     282 & Cameron Diaz   &   2\\\hline
IT  &     745 & Udo Kier       &   9\\\hline
US  &     503 & John Malkovich &  14\\\hline
        \end{tabular}
      \end{table}
      \end{tiny}

      \pause
      \column{.42\textwidth}
      \begin{itemize}
        \item $R \neq natjoin~(R1)~(R2)$

        \pause
        \item \tiny{\{MOVIEID, ACTORID\} $\rightarrow$ ORD}
      \end{itemize}
    \end{columns}
  \end{example}
\end{frame}

\begin{frame}
  \frametitle{Anomalies}

  \begin{itemize}
    \item \emph{insert}
    \begin{itemize}
      \item data is known but can not be inserted due to constraints
    \end{itemize}

    \pause
    \item \emph{delete}
    \begin{itemize}
        \item deleting some data causes some other data to be lost
    \end{itemize}

    \pause
    \item \emph{update}
    \begin{itemize}
      \item updating some data requires modifications in multiple tuples
    \end{itemize}
  \end{itemize}
\end{frame}

\begin{frame}
  \frametitle{Anomaly Examples}

  \begin{example}
    \begin{itemize}
      \item it is known that the country of the movie Gattaca is US\\
	but it cannot be inserted if there is no actor in the movie

      \pause
      \item deleting that Gabriel Byrne is in the movie Usual Suspects\\
        also deletes that the movie was made in the UK

      \pause
      \item changing the country of the movie Being John Malkovich\\
        requires modifications in two tuples
    \end{itemize}
  \end{example}
\end{frame}

\begin{frame}
  \frametitle{2nd Normal Form}

  \begin{definition}
    \alert{2NF}: every non-key attribute depends on the primary key
  \end{definition}

  \begin{block}{transition from 1NF to 2NF}
    \begin{itemize}
      \item in an $R$ relation that conforms to 1NF:
      \begin{itemize}
        \item $R(A,B,C,D)$, primary key: $\{A,B\}$
        \item $A \rightarrow D$
      \end{itemize}

      \pause
      \item for it to be 2NF:
      \begin{itemize}
        \item $R1(A,D)$, primary key: $A$
        \item $R2(A,B,C)$, primary key: $\{A,B\}$\\
          $A$ is a foreign key referencing $R1$
      \end{itemize}
    \end{itemize}
  \end{block}
\end{frame}

\begin{frame}
  \frametitle{1NF-2NF Transition Example}

  \begin{example}
    \begin{itemize}
      \item among the non-key attributes,\\
	only ORD depends on the primary key

      \pause
      \begin{itemize}
        \item $A$: MOVIEID
        \item $B$: ACTORID
        \item $C$: \{NAME,ORD\}
        \item $D$: \{TITLE,COUNTRY,LANGUAGE\}
      \end{itemize}
    \end{itemize}
  \end{example}
\end{frame}

\begin{frame}
  \frametitle{1NF-2NF Transition Example}

  \begin{example}
    \begin{itemize}
      \item R1(MOVIEID,TITLE,COUNTRY,LANGUAGE)\\
        primary key: MOVIEID

      \pause
      \item R2(MOVIEID,ACTORID,NAME,ORD)\\
        primary key: \{MOVIEID,ACTORID\}\\
        MOVIEID is a foreign key referencing R1
    \end{itemize}
 \end{example}
\end{frame}

\begin{frame}
  \frametitle{1NF-2NF Transition Example}

  \begin{example}
    \begin{itemize}
      \item R2 still not 2NF: ACTORID $\rightarrow$ NAME

      \pause
      \begin{itemize}
        \item $A$: ACTORID
        \item $B$: MOVIEID
        \item $C$: ORD
        \item $D$: NAME
      \end{itemize}
    \end{itemize}

    \pause
    \begin{itemize}
      \item R3(ACTORID,NAME)\\
        primary key: ACTORID

      \pause
      \item R4(MOVIEID,ACTORID,ORD)\\
        primary key: \{MOVIEID,ACTORID\}\\
        ACTORID is a foreign key referencing R3
    \end{itemize}
  \end{example}
\end{frame}

\begin{frame}
  \frametitle{2NF Relation Examples}

  \begin{example}
    \begin{center}
      \begin{tiny}
      \begin{table}
        \caption{R1}
        \begin{tabular}{|r|l|l|l|}\hline
\underline{MOVIEID} & TITLE    & COU & LANG\\\hline\hline
      6 & Usual Suspects       & UK  & EN  \\\hline
    228 & Ed Wood              & US  & EN  \\\hline
     70 & Being John Malkovich & US  & EN  \\\hline
   1512 & Suspiria             & IT  & IT  \\\hline
        \end{tabular}
      \end{table}
      \end{tiny}
    \end{center}

    \vspace{-0.7cm}
    \begin{columns}[t]
      \column{.5\textwidth}
      \begin{tiny}
      \begin{table}
        \caption{R3}
        \begin{tabular}{|r|l|}\hline
\underline{ACTORID} & NAME\\\hline\hline
      308 & Gabriel Byrne \\\hline
       26 & Johnny Depp   \\\hline
      282 & Cameron Diaz  \\\hline
      745 & Udo Kier      \\\hline
      503 & John Malkovich\\\hline
        \end{tabular}
      \end{table}
      \end{tiny}

      \column{.5\textwidth}
      \begin{tiny}
      \begin{table}
        \caption{R4}
        \begin{tabular}{|r|r|r|}\hline
\underline{MOVIEID} & \underline{ACTORID} & ORD\\\hline\hline
   6 & 308 &  2\\\hline
 228 &  26 &  1\\\hline
  70 & 282 &  2\\\hline
1512 & 745 &  9\\\hline
  70 & 503 & 14\\\hline
        \end{tabular}
      \end{table}
      \end{tiny}
    \end{columns}
  \end{example}
\end{frame}

\begin{frame}
  \frametitle{Dependency Diagram Example}

  \begin{example}
    \begin{center}
      \pgfuseimage{norm2}
    \end{center}
  \end{example}
\end{frame}

\begin{frame}
  \frametitle{2NF Corrected Anomalies}

  \begin{example}
    \begin{itemize}
      \item if the country of the movie Gattaca is US,\\
	it can be inserted to $R1$

      \pause
      \item if Gabriel Byrne is deleted from the movie Usual Suspects,\\
	the country of the movie is retained in $R1$

      \pause
      \item changing the country of the movie Being John Malkovich\\
	requires updating only one tuple in $R1$
    \end{itemize}
  \end{example}
\end{frame}

\begin{frame}
  \frametitle{2NF Remaining Anomalies}

  \begin{example}
    \begin{itemize}
      \item it is known that movies made in Brasil are in Portuguese\\
	but it can not be inserted if there is no movie made in Brasil

      \pause
      \item deleting the movie Suspiria\\
	also deletes that movies made in Italy are in Italian

      \pause
      \item changing the language of the movies made in the US\\
	requires two tuples to be updated
    \end{itemize}
  \end{example}
\end{frame}

\subsection{3rd Normal Form}

\begin{frame}
  \frametitle{3rd Normal Form}

  \begin{definition}
    \alert{3NF}: non-key attributes do not depend\\
      on any attribute or attribute group other than the primary key
  \end{definition}

  \pause
  \begin{block}{transition from 2NF to 3NF}
    \begin{itemize}
      \item in an $R$ relation that conforms to 2NF:
      \begin{itemize}
        \item $R(A,B,C,D)$, primary key: $A$
        \item $C \rightarrow D$
      \end{itemize}

      \pause
      \item for it to be 3NF:
      \begin{itemize}
        \item $R1(C,D)$, primary key: $C$
        \item $R2(A,B,C)$, primary key: $A$\\
          $C$ is a foreign key referencing $R1$
      \end{itemize}
    \end{itemize}
 \end{block}
\end{frame}

\begin{frame}
  \frametitle{2NF-3NF Transition Example}

  \begin{example}
    \begin{itemize}
      \item R1: COUNTRY $\rightarrow$ LANGUAGE

      \pause
      \begin{itemize}
        \item $A$: MOVIEID
        \item $B$: TITLE
        \item $C$: COUNTRY
        \item $D$: LANGUAGE
      \end{itemize}
    \end{itemize}

    \pause
    \begin{itemize}
      \item R5(COUNTRY, LANGUAGE)\\
        primary key: COUNTRY

      \pause
      \item R6(MOVIEID, TITLE, COUNTRY)\\
        primary key: MOVIEID\\
        COUNTRY is a foreign key referencing R5
    \end{itemize}
  \end{example}
\end{frame}

\begin{frame}
  \frametitle{3NF Relation Examples}

  \begin{example}
    \begin{columns}[t]
      \column{.5\textwidth}
      \begin{tiny}
      \begin{table}
        \caption{R6}
        \begin{tabular}{|r|l|l|}\hline
\underline{MOVIEID} & TITLE    & COU\\\hline\hline
      6 & Usual Suspects       & UK \\\hline
    228 & Ed Wood              & US \\\hline
     70 & Being John Malkovich & US \\\hline
   1512 & Suspiria             & IT \\\hline
        \end{tabular}
      \end{table}
      \end{tiny}

      \column{.5\textwidth}
      \begin{tiny}
      \begin{table}
        \caption{R5}
        \begin{tabular}{|l|l|}\hline
\underline{COU} & LANG\\\hline\hline
UK & EN\\\hline
US & EN\\\hline
IT & IT\\\hline
        \end{tabular}
      \end{table}
      \end{tiny}
    \end{columns}

    \vspace{-0.7cm}
    \begin{columns}[t]
      \column{.5\textwidth}
      \begin{tiny}
      \begin{table}
        \caption{R3}
        \begin{tabular}{|r|l|}\hline
\underline{ACTORID} & NAME\\\hline\hline
      308 & Gabriel Byrne \\\hline
       26 & Johnny Depp   \\\hline
      282 & Cameron Diaz  \\\hline
      745 & Udo Kier      \\\hline
      503 & John Malkovich\\\hline
        \end{tabular}
      \end{table}
      \end{tiny}

      \column{.5\textwidth}
      \begin{tiny}
      \begin{table}
        \caption{R4}
        \begin{tabular}{|r|r|r|}\hline
\underline{MOVIEID} & \underline{ACTORID} & ORD\\\hline\hline
   6 & 308 &  2\\\hline
 228 &  26 &  1\\\hline
  70 & 282 &  2\\\hline
1512 & 745 &  9\\\hline
  70 & 503 & 14\\\hline
        \end{tabular}
      \end{table}
      \end{tiny}
    \end{columns}
  \end{example}
\end{frame}

\begin{frame}
  \frametitle{Dependency Diagram Example}

  \begin{example}
    \begin{center}
      \pgfuseimage{norm3}
    \end{center}
  \end{example}
 \end{frame}

\begin{frame}
  \frametitle{3NF Corrected Anomalies}

  \begin{example}
    \begin{itemize}
      \item if movies made in Brasil are in Portuguese,\\
	it can be inserted into $R5$.

      \pause
      \item if the movie Suspiria is deleted, $R5$ retains the data\\
        that movies made in Italy are in Italian

      \pause
      \item changing the language of the movies made in the US\\
	requires modifying only one tuple in $R5$
    \end{itemize}
  \end{example}
\end{frame}

\begin{frame}
  \frametitle{Boyce-Codd Normal Form}

  \begin{definition}
    \alert{BCNF}: all functional dependencies must be on candidate keys
  \end{definition}

  \pause
  \begin{itemize}
    \item dependencies between attributes that are part of the keys\\
      have to be considered
  \end{itemize}
\end{frame}

\begin{frame}
  \frametitle{BCNF Example}

  \begin{example}[let movie titles be unique]
    \begin{itemize}
      \item candidate keys:
      \begin{itemize}
        \item \{MOVIEID,ACTORID\}
        \item \{TITLE,ACTORID\}
      \end{itemize}

      \pause
      \item non-conforming functional dependencies:
      \begin{itemize}
        \item MOVIEID $\rightarrow$ TITLE
        \item TITLE $\rightarrow$ MOVIEID
      \end{itemize}
    \end{itemize}
  \end{example}
\end{frame}

\subsection*{References}

\begin{frame}
  \frametitle{References}

  \begin{block}{Required text: Date}
    \begin{itemize}
      \item Chapter 11: \alert{Functional Dependencies}
      \item Chapter 12: \alert{Further Normalization I: 1NF, 2NF, 3NF, BCNF}
    \end{itemize}
  \end{block}
\end{frame}

\section{Entity/Relationship Model}

\subsection{Introduction}

\begin{frame}
  \frametitle{Entity/Relationship Model}

  \begin{itemize}
    \item modeling approach
    \begin{itemize}
      \item Chen 1976
    \end{itemize}

    \pause
    \item components
    \begin{itemize}
      \item entities
      \item properties
      \item relationships
    \end{itemize}
  \end{itemize}
\end{frame}

\begin{frame}
  \frametitle{Entities}

  \begin{definition}
    \alert{entity}:\\
      set of "things" with the same attributes

    \pause
    \begin{itemize}
      \item elements of the set are \emph{instances} of the entity
    \end{itemize}

    \pause
    \begin{itemize}
      \item \emph{strong}: can exist by itself
      \item \emph{weak}: existence depends on another entity
    \end{itemize}
  \end{definition}
\end{frame}

\begin{frame}
  \frametitle{Entity Examples}

  \begin{example}
    \begin{itemize}
      \item entity: movie, director, actor

      \pause
      \item instance: Johnny Depp

      \pause
      \medskip
      \item strong entity: director
      \item weak entity: movie
    \end{itemize}
  \end{example}
\end{frame}

\begin{frame}
  \frametitle{Properties}

  \begin{definition}
    \alert{property}:\\
      data describing entities or relationships

    \pause
    \begin{itemize}
      \item simple / composite
      \item key
      \item single / multiple valued
      \item empty
      \item base / derived
    \end{itemize}
  \end{definition}
\end{frame}

\begin{frame}
  \frametitle{Property Examples}

  \begin{example}
    \begin{itemize}
      \item property: title, country, language

      \pause
      \medskip
      \item simple: first name, last name
      \item composite: full name

      \pause
      \medskip
      \item base: date of birth
      \item derived: age
    \end{itemize}
  \end{example}
\end{frame}

\begin{frame}
  \frametitle{Relationships}

  \begin{definition}
    \alert{relationship}:\\
      connections between entities

    \pause
    \begin{itemize}
      \item \emph{participant}: entities in the relationship
      \item \emph{degree}: number of participants
      \item \emph{total / partial}: all instances of the entity\\
        do / don't participate in the relationship
    \end{itemize}
  \end{definition}
\end{frame}

\begin{frame}
  \frametitle{Relationship Types}

  \begin{itemize}
    \item \emph{one-to-one}
    \item \emph{one-to-many}
    \item \emph{many-to-many}
  \end{itemize}
\end{frame}

\begin{frame}
  \frametitle{Relationship Examples}

  \begin{example}[one-to-one]
    \begin{itemize}
      \item the capital relationship between countries and cities
    \end{itemize}
  \end{example}

  \pause
  \begin{example}[one-to-many]
    \begin{itemize}
      \item the management relationship between employees and projects
    \end{itemize}
  \end{example}

  \pause
  \begin{example}[many-to-many]
    \begin{itemize}
      \item the enrollment relationship between students and courses
    \end{itemize}
  \end{example}
\end{frame}

\subsection{E/R Diagrams}

\begin{frame}
  \frametitle{Entity/Relationship Diagrams}

  \begin{itemize}
    \item entity: rectangle
    \begin{itemize}
      \item weak: double lines
    \end{itemize}

    \pause
    \item property: ellipsis
    \begin{itemize}
      \item derived: dashed lines
      \item multi-valued: double lines
      \item composite: sub-ellipses
    \end{itemize}

    \pause
    \item relationship: diamond
    \begin{itemize}
      \item between weak and strong: double lines
      \item total: connection double lines
      \item 1 or n depending on the type of the relationship
    \end{itemize}
 \end{itemize}
\end{frame}

\begin{frame}
  \frametitle{Entity/Relationship Diagram Example}

  \begin{example}
    \begin{center}
      \pgfuseimage{imdb1}
    \end{center}
  \end{example}
\end{frame}

\begin{frame}
  \frametitle{Entity/Relationship Diagram Example}

  \begin{example}
    \begin{center}
      \pgfuseimage{imdb2}
    \end{center}
  \end{example}
\end{frame}

\begin{frame}
  \frametitle{Applying to Design}

  \begin{itemize}
    \item every entity a relation

    \pause
    \item every property an attribute

    \pause
    \item every many-to-many relationship a relation
    \begin{itemize}
      \item foreign keys to participating entities
    \end{itemize}

    \pause
    \item for every one-to-many relationship a foreign from the
      "many" side to the "one" side
 \end{itemize}
\end{frame}

\subsection*{References}

\begin{frame}
  \frametitle{References}

  \begin{block}{Required text: Date}
    \begin{itemize}
      \item Chapter 14: \alert{Semantic Modeling}
    \end{itemize}
  \end{block}
\end{frame}

\end{document}
