% Copyright (c) 2002-2010
%       H. Turgut Uyar <uyar@itu.edu.tr>
%       Şule Gündüz Öğüdücü <sgunduz@itu.edu.tr>
%
% These notes are licensed using the
% "Creative Commons Attribution-NonCommercial-ShareAlike License".
% You are free to copy, distribute and transmit the work, and to adapt the work
% as long as you attribute the authors, do not use it for commercial purposes,
% and any derivative work is under the same or a similar license.
%
% Read the full legal code at:
% http://creativecommons.org/licenses/by-nc-sa/3.0/

\documentclass[dvipsnames]{beamer}

\usepackage{ae}
\usepackage[T1]{fontenc}
\usepackage[utf8]{inputenc}
\setbeamertemplate{navigation symbols}{}

\usepackage{listings}
\lstdefinelanguage{FullSQL}[]{SQL}{
  morekeywords={BINARY,BOOLEAN,CYCLE,FINAL,INCREMENT,IS,LARGE,MAXVALUE,MINVALUE,
                OBJECT,REFERENCES,RENAME,SEQUENCE,START,TO,TYPE,VACUUM}
}
\lstset{language=FullSQL}

\mode<presentation>
{
  \usetheme{Warsaw}
  \usecolortheme[named=ForestGreen]{structure}
  \setbeamercovered{transparent}
}

\title{Database Systems}
\subtitle{Relational Model}

\author{H. Turgut Uyar \and Şule Öğüdücü}
\date{2002-2010}

\AtBeginSubsection[]{
  \begin{frame}<beamer>
    \frametitle{Topics}
    \tableofcontents[currentsection,currentsubsection]
  \end{frame}
}

\theoremstyle{plain}

\pgfdeclareimage[width=2cm]{license}{../../license}

\begin{document}

\begin{frame}
  \titlepage
\end{frame}

\begin{frame}
  \frametitle{License}

  \pgfuseimage{license}\hfill
  \copyright 2002-2010 T. Uyar, Ş. Öğüdücü

  \vfill
  \begin{tiny}
    You are free:
    \begin{itemize}
      \item to Share — to copy, distribute and transmit the work
      \item to Remix — to adapt the work
    \end{itemize}

    Under the following conditions:
    \begin{itemize}
      \item Attribution — You must attribute the work in the manner specified by
        the author or licensor (but not in any way that suggests that they
        endorse you or your use of the work).

      \item Noncommercial — You may not use this work for commercial purposes.

      \item Share Alike — If you alter, transform, or build upon this work, you
        may distribute the resulting work only under the same or similar license
        to this one.
    \end{itemize}
  \end{tiny}

  \vfill
  Legal code (the full license):\\
  \url{http://creativecommons.org/licenses/by-nc-sa/3.0/}
\end{frame}

\begin{frame}
  \frametitle{Topics}
  \tableofcontents
\end{frame}

\section{Relational Model}

\subsection{Introduction}

\begin{frame}
  \frametitle{Data Models}

  \begin{itemize}
    \item previous models:
    \begin{itemize}
      \item inverted list
      \item hierarchical
      \item network
    \end{itemize}

    \pause
    \item relational model:
    \begin{itemize}
      \item Dr. E. F. Codd, 1970
    \end{itemize}

    \pause
    \item later models:
    \begin{itemize}
      \item object
      \item object / relational
    \end{itemize}
  \end{itemize}
\end{frame}

\begin{frame}
  \frametitle{Relational Model}

  \begin{itemize}
    \item data is modelled as \alert{relations}:\\
      $\alpha \subseteq A \times B \times C \times ...$

    \pause
    \medskip
    \item each element of the relation is a \alert{tuple}
    \item each data of the element is an \alert{attributes}

    \pause
    \medskip
    \item relations are represented using tables
    \begin{itemize}
      \item the user should \emph{perceive} all data as tables
      \item relation $\rightarrow$ table, tuple $\rightarrow$ row,
        attribute $\rightarrow$ column
    \end{itemize}
  \end{itemize}
\end{frame}

\begin{frame}
  \frametitle{Relation Example}

  \begin{example}[movie relation]
    \begin{tiny}
    \begin{table}
      \caption{MOVIE}
      \begin{tabular}{|l|r|l|c|r|r|}\hline
TITLE                &   YR & DIRECTOR      & COUNTRY & SCORE & VOTES\\\hline\hline
Usual Suspects       & 1995 & Bryan Singer  & UK      &   8.7 &  3502\\\hline
Suspiria             & 1977 & Dario Argento & IT      &   7.1 &  1004\\\hline
Being John Malkovich & 1999 & Spike Jonze   & US      &   8.3 & 13809\\\hline
...                  &  ... & ...           & ...     &   ... &   ...\\\hline
      \end{tabular}
    \end{table}
    \end{tiny}

    \pause
    \begin{itemize}
      \item \texttt{(Usual Suspects,1995,Bryan Singer,UK,8.7,3502)}\\
        is a tuple of the MOVIE relation
      \item \texttt{COUNTRY} is an attribute of the MOVIE relation
    \end{itemize}
  \end{example}
\end{frame}

\begin{frame}
  \frametitle{Relation Structure}

  \begin{columns}[t]
    \column{.5\textwidth}
    \begin{block}{relation header}
      \begin{itemize}
        \item the set of attributes making up the relation
        \item specified when the relation is created
        \item affected by the data definition language statements
      \end{itemize}
    \end{block}

    \pause
    \column{.5\textwidth}
    \begin{block}{relation body}
      \begin{itemize}
        \item the set of tuples in the relation
        \item affected by the data manipulation language statements
      \end{itemize}
    \end{block}
  \end{columns}
\end{frame}

\begin{frame}
  \frametitle{Relation Predicate}

  \begin{definition}
    \alert{relation predicate}:\\
      the sentence expressing the meaning of the relation

    \begin{itemize}
      \item each tuple takes the value \emph{True} or \emph{False} according to
        this predicate
    \end{itemize}
  \end{definition}
\end{frame}

\begin{frame}
  \frametitle{Relation Predicate Example}

  \begin{example}[movie relation predicate]
    \begin{itemize}
      \item the movie titled as \texttt{TITLE} was filmed in the year \texttt{YR},
        by the director \texttt{DIRECTOR}, in the country \texttt{COUNTRY}; the
        average of \texttt{VOTES} votes is \texttt{SCORE}.

      \pause
      \medskip
      \item the tuple \texttt{(Suspiria,1977,Dario Argento,IT,1004,7.1)}\\
        is True
      \item the tuple \texttt{(Suspiria,1877,Dario Argento,IT,1004,7.1)}\\
        is False
    \end{itemize}
  \end{example}
\end{frame}

\begin{frame}
  \frametitle{Tuple Order}

  \begin{itemize}
    \item the order of tuples is insignificant
  \end{itemize}

  \pause
  \begin{example}
    \begin{itemize}
      \item the following two relations are equivalent:
    \end{itemize}

    \begin{columns}
      \column{.5\textwidth}
      \begin{tiny}
      \begin{table}
        \begin{tabular}{|l|l|}\hline
TITLE                & ... \\\hline\hline
Usual Suspects       & ... \\\hline
Suspiria             & ... \\\hline
Being John Malkovich & ... \\\hline
        \end{tabular}
      \end{table}
      \end{tiny}

      \column{.5\textwidth}
      \begin{tiny}
      \begin{table}
        \begin{tabular}{|l|l|}\hline
TITLE                & ... \\\hline\hline
Suspiria             & ... \\\hline
Being John Malkovich & ... \\\hline
Usual Suspects       & ... \\\hline
        \end{tabular}
      \end{table}
      \end{tiny}
    \end{columns}
  \end{example}
\end{frame}

\begin{frame}
  \frametitle{Attribute Order}

  \begin{itemize}
    \item the order of attributes is insignificant
  \end{itemize}

  \pause
  \begin{example}
    \begin{itemize}
      \item the following two relations are equivalent:
    \end{itemize}

    \begin{columns}
      \column{.5\textwidth}
      \begin{tiny}
      \begin{table}
        \begin{tabular}{|l|r|l|}\hline
TITLE                &   YR & ... \\\hline\hline
Usual Suspects       & 1995 & ... \\\hline
Suspiria             & 1977 & ... \\\hline
Being John Malkovich & 1999 & ... \\\hline
        \end{tabular}
      \end{table}
      \end{tiny}

      \column{.5\textwidth}
      \begin{tiny}
      \begin{table}
        \begin{tabular}{|r|l|l|}\hline
  YR & TITLE                & ... \\\hline\hline
1995 & Usual Suspects       & ... \\\hline
1977 & Suspiria             & ... \\\hline
1999 & Being John Malkovich & ... \\\hline
        \end{tabular}
      \end{table}
      \end{tiny}
    \end{columns}
  \end{example}
\end{frame}

\begin{frame}
  \frametitle{Identical Tuples}

  \begin{itemize}
    \item there can not be identical tuples in a relation
    \begin{itemize}
      \item each tuple has to be uniquely identifiable
    \end{itemize}
  \end{itemize}

  \pause
  \begin{example}
    \begin{tiny}
    \begin{table}
      \begin{tabular}{|l|r|l|c|r|r|}\hline
TITLE                &   YR & DIRECTOR      & COUNTRY & SCORE & VOTES \\\hline\hline
Usual Suspects       & 1995 & Bryan Singer  & UK      &   8.7 &  3502 \\\hline
Suspiria             & 1977 & Dario Argento & IT      &   7.1 &  1004 \\\hline
Being John Malkovich & 1999 & Spike Jonze   & US      &   8.3 & 13809 \\\hline
...                  &  ... & ...           & ...     &   ... &   ... \\\hline
Suspiria             & 1977 & Dario Argento & IT      &   7.1 &  1004 \\\hline
...                  &  ... & ...           & ...     &   ... &   ... \\\hline
      \end{tabular}
    \end{table}
    \end{tiny}
  \end{example}

  \begin{picture}(20,40)(0,-63)
    \color[rgb]{1,0.2,0.1}
    \put(0,20){\vector(2,-1){25}}
    \put(0,20){\vector(2,1){25}}
  \end{picture}
\end{frame}

\begin{frame}
  \frametitle{Domain}

  \begin{itemize}
    \item all values for the same attribute have to be selected from the same
      domain

    \begin{itemize}
      \item comparison only makes sense between values from the same domain
    \end{itemize}

    \pause
    \item in practice, mostly data types are used
  \end{itemize}
\end{frame}

\begin{frame}
  \frametitle{Domain Example}

  \begin{example}
    \begin{itemize}
      \item \texttt{TITLE} from the titles domain, \texttt{YR} from the years
        domain, \texttt{COUNTRY} from the countries domain, ...

      \pause
      \item if data types are used:\\
        \texttt{TITLE} string, \texttt{YR} integer, \texttt{COUNTRY} string, ...

      \begin{itemize}
        \item assigning the value \texttt{Woody Allen} to the attribute \texttt{COUNTRY}
          is correct in terms of data types, but wrong according to predicate

        \item \texttt{YR} and \texttt{VOTES} are integers but it does not make sense
          to compare them
      \end{itemize}
    \end{itemize}
  \end{example}
\end{frame}

\begin{frame}
  \frametitle{Null Value}

  \begin{columns}[t]
    \column{.5\textwidth}
    \begin{itemize}
      \item the value of the attribute is not known for this tuple
    \end{itemize}

    \begin{example}
      \begin{itemize}
        \item the director of the movie \texttt{Blade} is not known
      \end{itemize}
    \end{example}

    \pause
    \column{.5\textwidth}
    \begin{itemize}
      \item this tuple does not have a value for the attribute
    \end{itemize}

    \begin{example}
      \begin{itemize}
        \item no votes for the movie \texttt{Star Wars} yet, therefore
          \texttt{SCORE} is empty
      \end{itemize}
    \end{example}
  \end{columns}
\end{frame}

\begin{frame}
  \frametitle{Default Value}

  \begin{itemize}
    \item default values can be used instead of null values for unknown attribute
      values

    \begin{itemize}
      \item the default value may not be one of the valid values of the attribute
    \end{itemize}
  \end{itemize}

  \pause
  \begin{example}
    \begin{itemize}
      \item the default value for the \texttt{SCORE} attribute can be chosen
        as \texttt{-1}
    \end{itemize}
  \end{example}
\end{frame}

\subsection{Data Definition Language}

\begin{frame}
  \frametitle{Data Definition Language}

  \begin{itemize}
    \item creating and deleting relations

    \pause
    \item changing the relation header
    \begin{itemize}
      \item changing the relation name
      \item adding attributes
      \item changing the name of an attribute
    \end{itemize}
  \end{itemize}
\end{frame}

\begin{frame}
  \frametitle{SQL Data Types}

  \begin{itemize}
    \item number
    \item string
    \item boolean
    \item date/time
    \item large objects
  \end{itemize}
\end{frame}

\begin{frame}
  \frametitle{Basic Types}

  \begin{itemize}
    \item \lstinline!INTEGER!
    \begin{itemize}
      \item \lstinline!SMALLINT!
    \end{itemize}

    \pause
    \medskip
    \item \lstinline!NUMERIC (precision, scale)!
    \begin{itemize}
      \item \texttt{precision}: total number of digits
      \item \texttt{scale}: number of digits after the decimal point
      \item same as: \lstinline!DECIMAL (precision, scale)!
    \end{itemize}

    \pause
    \medskip
    \item \lstinline!FLOAT (p)!
    \begin{itemize}
      \item \texttt{p}: lowest acceptable precision
    \end{itemize}

    \pause
    \medskip
    \item \lstinline!BOOLEAN!
  \end{itemize}
\end{frame}

\begin{frame}
  \frametitle{String Types}

  \begin{itemize}
    \item \lstinline!CHARACTER [ VARYING ] (n)!
    \begin{itemize}
      \item in the type \lstinline!CHARACTER (n)!, the string will be padded
        with spaces if it is shorter than \texttt{n} characters
    \end{itemize}

    \pause
    \item abbreviations:
    \begin{itemize}
      \item \lstinline!CHAR (n)! instead of \lstinline!CHARACTER (n)!
      \item \lstinline!VARCHAR (n)! instead of \lstinline!CHARACTER VARYING (n)!
    \end{itemize}
  \end{itemize}
\end{frame}

\begin{frame}
  \frametitle{Date/Time Types}

  \begin{itemize}
    \item \lstinline!DATE!
    \begin{itemize}
      \item value example: \texttt{2005-09-26}
    \end{itemize}

    \pause
    \medskip
    \item \lstinline!TIME!
    \begin{itemize}
      \item value example: \texttt{11:59:22.078717}
    \end{itemize}

    \pause
    \medskip
    \item \lstinline!TIMESTAMP!
    \begin{itemize}
      \item value example: \texttt{2005-09-26 11:59:22.078717}
    \end{itemize}

    \pause
    \medskip
    \item \lstinline!INTERVAL!
    \begin{itemize}
      \item value example: \texttt{3 days}
    \end{itemize}
  \end{itemize}
\end{frame}

\begin{frame}
  \frametitle{Large Object Types}

  \begin{itemize}
    \item arbitrary length
    \item can not be used in queries

    \pause
    \bigskip
    \item binary: \lstinline!BINARY LARGE OBJECT (n)!
    \begin{itemize}
      \item \lstinline!BLOB!
      \item picture, audio, etc.
    \end{itemize}

    \pause
    \item text: \lstinline!CHARACTER LARGE OBJECT (n)!
    \begin{itemize}
      \item \lstinline!CLOB!
    \end{itemize}
  \end{itemize}
\end{frame}

\begin{frame}[fragile]
  \frametitle{Creating Tables}

  \begin{block}{Statement}
    \begin{lstlisting}
CREATE TABLE table_name (
  { column_name data_type
              [ DEFAULT default_value ] }
  [, ... ]
)
    \end{lstlisting}
  \end{block}

  \pause
  \begin{block}{Deleting Tables}
    \begin{lstlisting}
DROP TABLE table_name [, ... ]
    \end{lstlisting}
  \end{block}
\end{frame}

\begin{frame}[fragile]
  \frametitle{Table Deletion Example}

  \begin{example}
    \begin{lstlisting}
CREATE TABLE MOVIE (
  TITLE VARCHAR(80),
  YR NUMERIC(4),
  DIRECTOR VARCHAR(40),
  COUNTRY CHAR(2),
  SCORE FLOAT,
  VOTES INTEGER DEFAULT 0
)
    \end{lstlisting}
  \end{example}
\end{frame}

\begin{frame}[fragile]
  \frametitle{Renaming Tables}

  \begin{block}{Statement}
    \begin{lstlisting}
ALTER TABLE table_name
  RENAME TO new_name
    \end{lstlisting}
  \end{block}

  \pause
  \begin{example}
    \begin{lstlisting}
ALTER TABLE MOVIE
  RENAME TO FILM
    \end{lstlisting}
  \end{example}
\end{frame}

\begin{frame}[fragile]
  \frametitle{Adding Columns}

  \begin{block}{Statement}
    \begin{lstlisting}
ALTER TABLE table_name
  ADD [ COLUMN ] column_name data_type
                 [ DEFAULT default_value ]
    \end{lstlisting}
  \end{block}

  \pause
  \begin{example}
    \begin{lstlisting}
ALTER TABLE MOVIE
  ADD COLUMN LANGUAGE CHAR(2)
    \end{lstlisting}
  \end{example}
\end{frame}

\begin{frame}[fragile]
  \frametitle{Deleting Columns}

  \begin{block}{Statement}
    \begin{lstlisting}
ALTER TABLE table_name
  DROP [ COLUMN ] column_name
    \end{lstlisting}
  \end{block}

  \pause
  \begin{example}
    \begin{lstlisting}
ALTER TABLE MOVIE
  DROP COLUMN LANGUAGE
    \end{lstlisting}
  \end{example}
\end{frame}

\begin{frame}[fragile]
  \frametitle{Renaming Columns}

  \begin{block}{Statement}
    \begin{lstlisting}
ALTER TABLE table_name
  RENAME [ COLUMN ] column_name TO new_name
    \end{lstlisting}
  \end{block}

  \pause
  \begin{example}
    \begin{lstlisting}
ALTER TABLE MOVIE
  RENAME COLUMN TITLE TO NAME
    \end{lstlisting}
  \end{example}
\end{frame}

\begin{frame}[fragile]
  \frametitle{Changing Column Default Value}

  \begin{block}{Statement}
    \begin{lstlisting}
ALTER TABLE table_name
  ALTER [ COLUMN ] column_name
  SET DEFAULT default_value
    \end{lstlisting}
  \end{block}

  \pause
  \begin{example}
    \begin{lstlisting}
ALTER TABLE MOVIE
  ALTER COLUMN SCORE
  SET DEFAULT -1
    \end{lstlisting}
  \end{example}
\end{frame}

\begin{frame}[fragile]
  \frametitle{Removing Column Default Value}

  \begin{block}{Statement}
    \begin{lstlisting}
ALTER TABLE table_name
  ALTER [ COLUMN ] column_name
  DROP DEFAULT
    \end{lstlisting}
  \end{block}

  \pause
  \begin{example}
    \begin{lstlisting}
ALTER TABLE MOVIE
  ALTER COLUMN SCORE
  DROP DEFAULT
    \end{lstlisting}
  \end{example}
\end{frame}

\subsection{Data Manipulation Language}

\begin{frame}
  \frametitle{Data Manipulation Language}

  \begin{itemize}
    \item adding tuples
    \item deleting tuples
    \item updating tuples

    \pause
    \bigskip
    \item operations are carried out on tuple sets
    \begin{itemize}
      \item all tuples satisfying a condition
    \end{itemize}
  \end{itemize}
\end{frame}

\begin{frame}[fragile]
  \frametitle{Adding Rows}

  \begin{block}{Statement}
    \begin{lstlisting}
INSERT INTO table_name
  [ ( column_name [, ...] ) ]
  VALUES ( column_value [, ...] )
    \end{lstlisting}
  \end{block}

  \pause
  \begin{itemize}
    \item the order of values has to match the order of columns
    \item unspecified attributes will take default value
    \item if the column names are not specified, the column values have to given
      in the order used when creating the table
  \end{itemize}
\end{frame}

\begin{frame}[fragile]
  \frametitle{Row Adding Example}

  \begin{example}
    \begin{lstlisting}
INSERT INTO MOVIE VALUES (
  'Usual Suspects',
  1995,
  'Bryan Singer',
  'UK',
  8.7,
  35027
)
    \end{lstlisting}
  \end{example}
\end{frame}

\begin{frame}[fragile]
  \frametitle{Row Adding Example}

  \begin{example}
    \begin{lstlisting}
INSERT INTO MOVIE (YR, TITLE) VALUES (
  1995,
  'Usual Suspects'
)
    \end{lstlisting}
  \end{example}
\end{frame}

\begin{frame}[fragile]
  \frametitle{Deleting Rows}

  \begin{block}{Statement}
    \begin{lstlisting}
DELETE FROM table_name
  [ WHERE condition ]
    \end{lstlisting}
  \end{block}

  \pause
  \begin{itemize}
    \item if no condition is specified, all rows will be deleted
  \end{itemize}
\end{frame}

\begin{frame}[fragile]
  \frametitle{Row Deleting Example}

  \begin{example}[delete all movies with scores less than 3]
    \begin{lstlisting}
DELETE FROM MOVIE
  WHERE (SCORE < 3)
    \end{lstlisting}
  \end{example}
\end{frame}

\begin{frame}[fragile]
  \frametitle{Row Deleting Example}

  \begin{example}[delete all movies with scores less than 3
                  and at least 5 votes]
    \begin{lstlisting}
DELETE FROM MOVIE
  WHERE ((SCORE < 3) AND (VOTES >= 5))
    \end{lstlisting}
  \end{example}
\end{frame}

\begin{frame}[fragile]
  \frametitle{Updating Rows}

  \begin{block}{Statement}
    \begin{lstlisting}
UPDATE table_name
  SET { column_name = column_value } [, ...]
  [ WHERE condition ]
    \end{lstlisting}
  \end{block}

  \pause
  \begin{itemize}
    \item if no condition is specified, all rows will be updated
  \end{itemize}
\end{frame}

\begin{frame}[fragile]
  \frametitle{Row Updating Example}

  \begin{example}[reset the scores and votes of all movies filmed before 1997
                  to zero]
    \begin{lstlisting}
UPDATE MOVIE
  SET SCORE = 0, VOTES = 0
  WHERE (YR < 1997)
    \end{lstlisting}
  \end{example}
\end{frame}

\begin{frame}[fragile]
  \frametitle{Row Updating Example}

  \begin{example}[vote 9 for "Suspiria"]
    \begin{lstlisting}
UPDATE MOVIE
  SET SCORE = (SCORE * VOTES + 9) / (VOTES + 1),
      VOTES = VOTES+1
  WHERE (TITLE = 'Suspiria')
    \end{lstlisting}
  \end{example}
\end{frame}

\section{Integrity}

\subsection{Introduction}

\begin{frame}
  \frametitle{Integrity}

  \begin{itemize}
    \item \emph{problem}: preventing tuples that contradict the relation
      predicate

    \pause
    \bigskip
    \item rules are expressed using the data definition language
    \begin{itemize}
      \item value constraints
      \item keys
    \end{itemize}
  \end{itemize}
\end{frame}

\begin{frame}
  \frametitle{Value Constraint}

  \begin{definition}
    \alert{value constraint}:\\
      values must satisfy a boolean expression
  \end{definition}

  \pause
  \begin{example}
    \begin{itemize}
      \item \texttt{YR} value can not be smaller than \texttt{1880}
      \item \texttt{SCORE} values must be between \texttt{0} and \texttt{10}
      \item \texttt{TITLE} values can not be empty
    \end{itemize}
  \end{example}
\end{frame}

\begin{frame}[fragile]
  \frametitle{Specifying Value Constraints}

  \begin{block}{Statement}
    \begin{lstlisting}
CREATE TABLE table_name (
  ...
  [ { CHECK ( expression ) } [, ...] ]
  ...
)
    \end{lstlisting}
  \end{block}
\end{frame}

\begin{frame}[fragile]
  \frametitle{Value Constraint Example}

  \begin{example}[YR values can not be smaller than 1880]
    \begin{lstlisting}
CREATE TABLE MOVIE (
  TITLE VARCHAR(80),
  YR NUMERIC(4),
  ...,
  VOTES INTEGER DEFAULT 0,
  CHECK (YR >= 1880)
)
    \end{lstlisting}
  \end{example}
\end{frame}

\begin{frame}[fragile]
  \frametitle{Null Value Constraint}

  \begin{block}{Expressing NULL condition}
    \begin{lstlisting}
CREATE TABLE table_name (
  ...
  CHECK ( column_name IS { NULL | NOT NULL } )
  ...
)
    \end{lstlisting}
  \end{block}

  \pause
  \begin{itemize}
    \item can be specified at column definition:\\
      \lstinline!column_name data_type [ { NULL | NOT NULL } ]!
  \end{itemize}
\end{frame}

\begin{frame}[fragile]
  \frametitle{Null Value Constraint Example}

  \begin{example}[TITLE attributes can not be empty]
    \begin{lstlisting}
CREATE TABLE MOVIE (
  TITLE VARCHAR(80),
  YR NUMERIC(4),
  ..,
  VOTES INTEGER DEFAULT 0,
  CHECK (TITLE IS NOT NULL)
)
    \end{lstlisting}
  \end{example}
\end{frame}

\begin{frame}[fragile]
  \frametitle{Null Value Constraint Example}

  \begin{example}[TITLE attributes can not be empty]
    \begin{lstlisting}
CREATE TABLE MOVIE (
  TITLE VARCHAR(80) NOT NULL,
  YR NUMERIC(4),
  ...,
  VOTES INTEGER DEFAULT 0
)
    \end{lstlisting}
  \end{example}
\end{frame}

\begin{frame}[fragile]
  \frametitle{Creating Domains}

  \begin{block}{Statement}
    \begin{lstlisting}
CREATE DOMAIN domain_name [ AS ] base_type
  [ DEFAULT default_value ]
  [ constraint_definition [, ...] ]
    \end{lstlisting}
  \end{block}

  \pause
  \begin{itemize}
    \item the created domain can be used as a column type
  \end{itemize}

  \pause
  \begin{block}{Deleting Domains}
    \begin{lstlisting}
DROP DOMAIN domain_name [, ...]
    \end{lstlisting}
  \end{block}
\end{frame}

\begin{frame}[fragile]
  \frametitle{Domain Example}

  \begin{example}[domain for year values]
    \begin{lstlisting}
CREATE DOMAIN YEARS AS NUMERIC(4)
  DEFAULT 2005
  CHECK (VALUE >= 1880)
    \end{lstlisting}

    \pause
    \begin{lstlisting}
CREATE TABLE MOVIE (
  TITLE VARCHAR(80),
  YR YEARS,
  ...
)
    \end{lstlisting}
  \end{example}
\end{frame}

\begin{frame}[fragile]
  \frametitle{Managing Constraints}

  \begin{block}{Statement}
    \begin{lstlisting}
ALTER TABLE table_name
  ADD [ CONSTRAINT constraint_name ]
    constraint_definition
    \end{lstlisting}
  \end{block}

  \pause
  \begin{itemize}
    \item what about existing tuples?
  \end{itemize}

  \pause
  \begin{block}{Removing Constraints}
    \begin{lstlisting}
ALTER TABLE table_name
  DROP [ CONSTRAINT ] constraint_name
    \end{lstlisting}
  \end{block}
\end{frame}

\begin{frame}[fragile]
  \frametitle{Constraint Management Example}

  \begin{example}[SCORE values can not be greater than 10]
    \begin{lstlisting}
ALTER TABLE MOVIE
  ADD CHECK (SCORE <= 10)
    \end{lstlisting}
  \end{example}
\end{frame}

\begin{frame}[fragile]
  \frametitle{Constraint Management Examples}

  \begin{example}[SCORE values can not be smaller than 0]
    \begin{lstlisting}
ALTER TABLE MOVIE
  ADD CONSTRAINT SCORE_POSITIVE
      CHECK (SCORE >= 0)
    \end{lstlisting}
  \end{example}

  \pause
  \begin{example}[SCORE values can be smaller than 0]
    \begin{lstlisting}
ALTER TABLE MOVIE
  DROP CONSTRAINT SCORE_POSITIVE
    \end{lstlisting}
  \end{example}
\end{frame}

\subsection{Keys}

\begin{frame}
  \frametitle{Candidate Key}

  \begin{itemize}
    \item let $B$ be the attributes of the relation and let $A \subseteq B$

    \item in order for $A$ to be a candidate key, the following conditions
      must hold:

    \pause
    \begin{itemize}
      \item \alert{uniqueness}: no two tuples have the same values for all
        the attributes in $A$

      \pause
      \item \alert{irreducibility}: no subset of $A$ satisfies the uniqueness
        property
    \end{itemize}

    \pause
    \item every relation has at least one candidate key

    \pause
    \item every candidate key is a constraint
  \end{itemize}
\end{frame}

\begin{frame}
  \frametitle{Candidate Key Example}

  \begin{example}[candidate keys for movie relation]
    \begin{itemize}
      \item \texttt{\{TITLE\}}

      \pause
      \item \texttt{\{TITLE,YR\}}

      \pause
      \item \texttt{\{TITLE,DIRECTOR\}}

      \pause
      \item \texttt{\{TITLE,YR,DIRECTOR\}}
    \end{itemize}
  \end{example}
\end{frame}

\begin{frame}
  \frametitle{Surrogate Keys}

  \begin{itemize}
    \item if a \emph{natural key} can not be selected a \emph{surrogate key}
      can be defined

    \pause
    \medskip
    \item identity attributes
    \begin{itemize}
      \item can be generated by the system
    \end{itemize}
  \end{itemize}
\end{frame}

\begin{frame}
  \frametitle{Surrogate Key Example}

  \begin{example}
    \begin{tiny}
    \begin{table}
      \caption{MOVIE}
      \begin{tabular}{|r|l|r|l|c|r|r|}\hline
  ID & TITLE                &   YR & DIRECTOR      & COUNTRY & SCORE & VOTES\\\hline\hline
 ... & ...                  &  ... & ...           & ...     & ...   &   ...\\\hline
   6 & Usual Suspects       & 1995 & Bryan Singer  & ...     & ...   &   ...\\\hline
1512 & Suspiria             & 1977 & Dario Argento & ...     & ...   &   ...\\\hline
  70 & Being John Malkovich & 1999 & Spike Jonze   & ...     & ...   &   ...\\\hline
 ... & ...                  &  ... & ...           & ...     & ...   &   ...\\\hline
      \end{tabular}
    \end{table}
    \end{tiny}

    \pause
    \begin{itemize}
      \item \texttt{\{ID\}} is a candidate key
      \item \texttt{\{ID,TITLE\}} is not a candidate key
    \end{itemize}
  \end{example}
\end{frame}

\begin{frame}[fragile]
  \frametitle{Defining Candidate Keys}

  \begin{block}{Statement}
    \begin{lstlisting}
CREATE TABLE table_name (
  ...
  [ { UNIQUE ( column_name [, ...] ) }
    [, ...] ]
  ...
)
    \end{lstlisting}
  \end{block}

  \begin{itemize}
    \item if a candidate key consists of only one column, it can be specified
      when the column is defined:\\
      \lstinline!column_name data_type UNIQUE!
  \end{itemize}
\end{frame}

\begin{frame}[fragile]
  \frametitle{Candidate Key Definition Example}

  \begin{example}[\texttt{{ID}} and \texttt{\{TITLE,DIRECTOR\}} are unique]
    \begin{lstlisting}
CREATE TABLE MOVIE (
  ID INTEGER,
  TITLE VARCHAR(80),
  ...
  VOTES INTEGER DEFAULT 0,
  CHECK (YR >= 1880),
  UNIQUE (ID),
  UNIQUE (TITLE, DIRECTOR)
)
    \end{lstlisting}
  \end{example}
\end{frame}

\begin{frame}[fragile]
  \frametitle{Candidate Key Definition Example}

  \begin{example}[\texttt{ID} is unique and can not be empty]
    \begin{lstlisting}
CREATE TABLE MOVIE (
  ID INTEGER UNIQUE NOT NULL,
  TITLE VARCHAR(80),
  ...
  VOTES INTEGER DEFAULT 0,
  CHECK (YR >= 1880),
  UNIQUE (TITLE , DIRECTOR)
)
   \end{lstlisting}
  \end{example}
\end{frame}

\begin{frame}[fragile]
  \frametitle{Automatic Identity Values}

  \begin{itemize}
    \item product-specific definitions

    \begin{itemize}
      \item PostgreSQL: \texttt{SERIAL} data type
      \item MySQL: \texttt{AUTO\_INCREMENT} property
    \end{itemize}

    \pause
    \item not specified when adding rows
  \end{itemize}
\end{frame}

\begin{frame}[fragile]
  \frametitle{Automatic Identity Value Example}

  \begin{example}[PostgreSQL]
    \begin{lstlisting}
CREATE TABLE MOVIE (
  ID SERIAL UNIQUE NOT NULL,
  TITLE VARCHAR(80),
  ...
)
    \end{lstlisting}
  \end{example}
\end{frame}

\begin{frame}[fragile]
  \frametitle{Automatic Identity Value Example}

  \begin{example}[MySQL]
    \begin{lstlisting}
CREATE TABLE MOVIE (
  ID INTEGER UNIQUE NOT NULL AUTO_INCREMENT,
  TITLE VARCHAR(80),
  ...
)
    \end{lstlisting}
  \end{example}
\end{frame}

\begin{frame}
  \frametitle{Primary Key}

  \begin{itemize}
    \item if a relation has more than one candidate key:
    \begin{itemize}
      \item one of them is selected as the \alert{primary key}
      \item the others are \alert{alternate keys}
    \end{itemize}

    \pause
    \item every relation must have a primary key

    \pause
    \item any attribute that is part of the primary key can not be empty
      in any tuple

    \pause
    \item the names of the attributes in the primary key are underlined
  \end{itemize}
\end{frame}

\begin{frame}
  \frametitle{Primary Key Example}

  \begin{example}
    \begin{tiny}
    \begin{table}
      \caption{MOVIE}
      \begin{tabular}{|r|l|r|l|c|r|r|}\hline
\underline{ID} & TITLE      &   YR & DIRECTOR      & COUNTRY & SCORE & VOTES\\\hline\hline
 ... & ...                  &  ... & ...           & ...     &   ... &   ...\\\hline
   6 & Usual Suspects       & 1995 & Bryan Singer  & ...     &   ... &   ...\\\hline
1512 & Suspiria             & 1977 & Dario Argento & ...     &   ... &   ...\\\hline
  70 & Being John Malkovich & 1999 & Spike Jonze   & ...     &   ... &   ...\\\hline
 ... & ...                  &  ... & ...           & ...     &   ... &   ...\\\hline
      \end{tabular}
    \end{table}
    \end{tiny}
  \end{example}
\end{frame}

\begin{frame}[fragile]
  \frametitle{Defining Primary Keys}

  \begin{block}{Statement}
    \begin{lstlisting}
CREATE TABLE table_name (
  ...
  [ PRIMARY KEY ( column_name [, ...] ) ]
  ...
)
    \end{lstlisting}
  \end{block}

  \begin{itemize}
    \item if the primary key consists of only one column, it can be specified
      when the column is defined:\\
      \lstinline!column_name data_type PRIMARY KEY!
  \end{itemize}
\end{frame}

\begin{frame}[fragile]
  \frametitle{Primary Key Definition Example}

  \begin{example}[{ID} is the primary key]
    \begin{lstlisting}
CREATE TABLE MOVIE (
  ID INTEGER,
  TITLE VARCHAR(80),
  ...
  VOTES INTEGER DEFAULT 0,
  CHECK (YR >= 1880),
  UNIQUE (TITLE, DIRECTOR),
  PRIMARY KEY (ID)
)
    \end{lstlisting}
  \end{example}
\end{frame}

\begin{frame}[fragile]
  \frametitle{Primary Key Definition Example}

  \begin{example}[ID is the primary key]
    \begin{lstlisting}
CREATE TABLE MOVIE (
  ID INTEGER PRIMARY KEY,
  TITLE VARCHAR(80),
  ...
  VOTES INTEGER DEFAULT 0,
  CHECK (YR >= 1880),
  UNIQUE (TITLE, DIRECTOR)
)
    \end{lstlisting}
  \end{example}
\end{frame}

\subsection{Referential Integrity}

\begin{frame}
  \frametitle{Scalar Values}

  \begin{itemize}
    \item attribute values must be scalar
    \begin{itemize}
      \item no arrays, lists, records etc. are allowed
    \end{itemize}

    \pause
    \item values have to be duplicated in order to meet this requirement
    \begin{itemize}
      \item duplication causes problems
    \end{itemize}
  \end{itemize}
\end{frame}

\begin{frame}
  \frametitle{Scalar Value Example}

  \begin{example}[how to store actor data?]
    \begin{tiny}
    \begin{table}
      \begin{tabular}{|r|l|c|l|}\hline
 ID & TITLE                & ... & ACTORS                     \\\hline\hline
  6 & Usual Suspects       & ... & Gabriel Byrne              \\\hline
... & ...                  & ... & ...                        \\\hline
 70 & Being John Malkovich & ... & Cameron Diaz,John Malkovich\\\hline
... & ...                  & ... & ...                        \\\hline
      \end{tabular}
    \end{table}
    \end{tiny}

    \pause
    \begin{picture}(90,5)(-163,-29)
      \color[rgb]{1,0.2,0.1}
      \thicklines
      \only<2->{
        \put(0,0){\line(1,0){90}}
      }
    \end{picture}

    \pause
    \begin{tiny}
    \begin{table}
      \begin{tabular}{|r|l|c|l|}\hline
 ID & TITLE                & ... & ACTOR         \\\hline\hline
  6 & Usual Suspects       & ... & Gabriel Byrne \\\hline
... & ...                  & ... & ...           \\\hline
 70 & Being John Malkovich & ... & Cameron Diaz  \\\hline
 70 & Being John Malkovich & ... & John Malkovich\\\hline
... & ...                  & ... & ...           \\\hline
      \end{tabular}
    \end{table}
    \end{tiny}
  \end{example}
\end{frame}

\begin{frame}
  \frametitle{Scalar Value Example}

  \begin{example}[movies and actors]
  \begin{tiny}
  \begin{table}
    \caption{MOVIE}
    \begin{tabular}{|r|l|c|}\hline
\underline{ID} & TITLE      & ...\\\hline\hline
   6 & Usual Suspects       & ...\\\hline
1512 & Suspiria             & ...\\\hline
  70 & Being John Malkovich & ...\\\hline
 ... & ...                  & ...\\\hline
    \end{tabular}
  \end{table}
  \end{tiny}

  \begin{columns}
    \column{.5\textwidth}
    \begin{tiny}
      \begin{table}
        \caption{ACTOR}
        \begin{tabular}{|r|l|}\hline
\underline{ID} & NAME\\\hline\hline
308 & Gabriel Byrne  \\\hline
282 & Cameron Diaz   \\\hline
503 & John Malkovich \\\hline
... & ...            \\\hline
        \end{tabular}
      \end{table}
    \end{tiny}

    \column{.5\textwidth}
    \begin{tiny}
    \begin{table}
      \caption{CASTING}
      \begin{tabular}{|r|r|r|}\hline
\underline{MOVIEID} & \underline{ACTORID} & ORD\\\hline\hline
  6 &  308 &   2\\\hline
 70 &  282 &   2\\\hline
 70 &  503 &  14\\\hline
... &  ... & ...\\\hline
      \end{tabular}
    \end{table}
    \end{tiny}
  \end{columns}
  \end{example}
\end{frame}

\begin{frame}
  \frametitle{Scalar Value Example}

  \begin{example}[how to store director data?]
    \begin{itemize}
      \item separate relation or together with actors?
    \end{itemize}

    \pause
    \begin{columns}[t]
      \column{.6\textwidth}
      \begin{tiny}
      \begin{table}
        \caption{MOVIE}
        \begin{tabular}{|r|l|c|r|}\hline
\underline{ID} & TITLE      & ... & DIRECTORID\\\hline\hline
   6 & Usual Suspects       & ... &        639\\\hline
1512 & Suspiria             & ... &       2259\\\hline
  70 & Being John Malkovich & ... &       1485\\\hline
 ... & ...                  & ... &        ...\\\hline
        \end{tabular}
      \end{table}
      \end{tiny}

      \column{.4\textwidth}
      \begin{tiny}
      \begin{table}
        \caption{PERSON}
        \begin{tabular}{|r|l|}\hline
\underline{ID} & NAME\\\hline\hline
 308 & Gabriel Byrne \\\hline
1485 & Spike Jonze   \\\hline
 639 & Bryan Singer  \\\hline
 282 & Cameron Diaz  \\\hline
2259 & Dario Argento \\\hline
 503 & John Malkovich\\\hline
 ... & ...           \\\hline
        \end{tabular}
      \end{table}
      \end{tiny}
    \end{columns}
  \end{example}
\end{frame}

\begin{frame}[fragile]
  \frametitle{Creating the Example Tables}

  \begin{example}[creating the MOVIE table]
    \begin{lstlisting}
CREATE TABLE MOVIE (
  ID INTEGER PRIMARY KEY,
  TITLE VARCHAR(80) NOT NULL,
  YR NUMERIC(4),
  COUNTRY CHAR(2),
  SCORE FLOAT,
  VOTES INTEGER DEFAULT 0,
  DIRECTORID INTEGER
)
    \end{lstlisting}
  \end{example}
\end{frame}

\begin{frame}[fragile]
  \frametitle{Creating the Example Tables}

  \begin{example}[creating the PERSON table]
    \begin{lstlisting}
CREATE TABLE PERSON (
  ID INTEGER PRIMARY KEY,
  NAME VARCHAR(40) UNIQUE NOT NULL
)
    \end{lstlisting}
  \end{example}
\end{frame}

\begin{frame}[fragile]
  \frametitle{Creating the Example Tables}

  \begin{example}[creating the CASTING table]
    \begin{lstlisting}
CREATE TABLE CASTING (
  MOVIEID INTEGER,
  ACTORID INTEGER,
  ORD INTEGER,
  PRIMARY KEY (MOVIEID, ACTORID)
)
    \end{lstlisting}
  \end{example}
\end{frame}

\begin{frame}
  \frametitle{Foreign Keys}

  \begin{definition}
    \alert{foreign key}:\\
      the attribute of a relation is the candidate key of another relation
  \end{definition}
\end{frame}

\begin{frame}
  \frametitle{Foreign Key Example}

  \begin{example}[DIRECTORID attribute of MOVIE relation]
    \begin{columns}[t]
      \column{.6\textwidth}
      \begin{tiny}
      \begin{table}
        \caption{MOVIE}
        \begin{tabular}{|r|l|c|r|}\hline
\underline{ID} & TITLE      & ... & DIRECTORID\\\hline\hline
   6 & Usual Suspects       & ... &        639\\\hline
1512 & Suspiria             & ... &       2259\\\hline
  70 & Being John Malkovich & ... &       1485\\\hline
 ... & ...                  & ... &        ...\\\hline
        \end{tabular}
      \end{table}
      \end{tiny}

      \column{.4\textwidth}
      \begin{tiny}
      \begin{table}
        \caption{PERSON}
        \begin{tabular}{|r|l|}\hline
\underline{ID} & NAME\\\hline\hline
 308 & Gabriel Byrne \\\hline
1485 & Spike Jonze   \\\hline
 639 & Bryan Singer  \\\hline
 282 & Cameron Diaz  \\\hline
2259 & Dario Argento \\\hline
 503 & John Malkovich\\\hline
 ... & ...           \\\hline
        \end{tabular}
      \end{table}
      \end{tiny}
    \end{columns}
  \end{example}

  \begin{picture}(100,50)(-140,-90)
    \color[rgb]{0.1,0.6,0.1}
    \thicklines
    \only<2->{
      \put(12,22){\oval(20,8)}         % movie.directorid=2259
      \put(88,0){\oval(20,8)}          % person.id=2259
      \put(23,21){\vector(3,-1){55}}   % movie.directorid=2259->person.id=2259
    }
    \only<3->{
      \put(0,39){\oval(45,9)}          % movie.directorid
      \put(90,39){\oval(20,9)}         % person.id
      \put(24,39){\vector(1,0){56}}    % movie.directorid->person.id
    }
  \end{picture}
\end{frame}

\begin{frame}
  \frametitle{Foreign Key Example}

  \begin{example}[foreign keys in movie database]
    \begin{columns}
      \column{.6\textwidth}
      \begin{tiny}
      \begin{table}
        \caption{MOVIE}
        \begin{tabular}{|r|l|c|r|}\hline
\underline{ID} & TITLE & ... & DIRECTORID\\\hline\hline
   6 & Usual Suspects  & ... &        639\\\hline
1512 & Suspiria        & ... &       2259\\\hline
 ... & ...             & ... &        ...\\\hline
        \end{tabular}
      \end{table}
      \end{tiny}

      \column{.4\textwidth}
      \begin{tiny}
      \begin{table}
        \caption{PERSON}
        \begin{tabular}{|r|l|}\hline
\underline{ID} & NAME\\\hline\hline
 308 & Gabriel Byrne \\\hline
1485 & Spike Jonze   \\\hline
 ... & ...           \\\hline
        \end{tabular}
      \end{table}
      \end{tiny}
    \end{columns}

    \begin{tiny}
    \begin{table}
      \caption{CASTING}
      \begin{tabular}{|r|r|r|}\hline
\underline{MOVIEID} & \underline{ACTORID} & ORD\\\hline\hline
  6 & 308 &   2\\\hline
 70 & 282 &   2\\\hline
... & ... & ...\\\hline
      \end{tabular}
    \end{table}
    \end{tiny}
  \end{example}

  \begin{picture}(210,80)(-25,-126)
    \color[rgb]{0.1,0.6,0.1}
    \thicklines
    \only<2->{
      \put(105,82){\oval(45,9)}        % movie.directorid
      \put(209,82){\oval(20,9)}        % person.id
      \put(128,82){\vector(1,0){71}}   % movie.directorid->person.id
    }
    \only<3->{
      \put(96,2){\oval(35,9)}          % casting.movieid
      \put(0,82){\oval(20,9)}          % movie.id
      \put(96,7){\vector(-4,3){92}}    % casting.movieid->movie.id
    }
    \only<4->{
      \put(136,2){\oval(35,9)}         % casting.personid
      \put(150,7){\vector(3,4){53}}    % casting.personid->person.id
    }
  \end{picture}
\end{frame}

\begin{frame}
  \frametitle{Referential Integrity}

  \begin{definition}
    \alert{referential integrity}:\\
      all values of a foreign key attribute must be present in the
      corresponding attribute of the referenced relation
  \end{definition}

  \pause
  \begin{itemize}
    \item what if a data manipulation request conflicts with referential
      integrity?
  \end{itemize}
\end{frame}

\begin{frame}
  \frametitle{Referential Integrity Example}

  \begin{example}
    \begin{columns}
      \column{.5\textwidth}
      \begin{tiny}
      \begin{table}
        \caption{MOVIE}
        \begin{tabular}{|r|l|c|r|}\hline
\underline{ID} & TITLE & ... & DIRECTORID\\\hline\hline
 ... & ...             & ... &        ...\\\hline
1512 & Suspiria        & ... &       2259\\\hline
 ... & ...             & ... &        ...\\\hline
        \end{tabular}
      \end{table}
      \end{tiny}

      \column{.5\textwidth}
      \begin{tiny}
      \begin{table}
        \caption{PERSON}
        \begin{tabular}{|r|l|}\hline
\underline{ID} & NAME\\\hline\hline
 ... & ...           \\\hline
2259 & Dario Argento \\\hline
 ... & ...           \\\hline
        \end{tabular}
      \end{table}
      \end{tiny}
    \end{columns}

    \pause
    \begin{itemize}
      \item delete the tuple \texttt{(2259,Dario Argento)}
      \item update the tuple \texttt{(2259,Dario Argento)} as\\
        \texttt{(2871,Dario Argento)}
    \end{itemize}
  \end{example}
\end{frame}

\begin{frame}
  \frametitle{Integrity Constraints}

  \begin{itemize}
    \item do not allow the operation
      (\lstinline!RESTRICT! | \lstinline!NO\_ACTION!)

    \pause
    \item reflect the change to affected tuples (\lstinline!CASCADE!)

    \pause
    \item assign null value (\lstinline!SET NULL!)
    \begin{itemize}
      \item if null values are allowed
    \end{itemize}

    \pause
    \item assign default value (\lstinline!SET DEFAULT!)
  \end{itemize}
\end{frame}

\begin{frame}[fragile]
  \frametitle{Referential Integrity Examples}

  \begin{example}[restrict when deleting]
    \begin{itemize}
      \item do not allow deleting if the \texttt{ID} attribute of the director
        to be deleted from the \texttt{PERSON} relation is present among the
        current values in the \texttt{DIRECTORID} attribute of the
        \texttt{MOVIE} relation
    \end{itemize}
  \end{example}

  \pause
  \begin{example}[cascade when updating]
    \begin{itemize}
      \item when the \texttt{ID} attribute value of a director has been
        updated in the \texttt{PERSON} relation, also update the
        \texttt{DIRECTORID} attribute of corresponding tuples in the
        \texttt{MOVIE} relation
    \end{itemize}
  \end{example}
\end{frame}

\begin{frame}[fragile]
  \frametitle{Referential Integrity Example}

  \begin{example}[cascade when deleting]
    \begin{itemize}
      \item delete a director from the \texttt{PERSON} relation

      \pause
      \item delete all tuples from the \texttt{MOVIE} relation
        where the \texttt{ID} attribute of the deleted director
        has the same value as the \texttt{DIRECTORID} attribute

      \pause
      \item for each deleted \texttt{MOVIE} tuple, delete all tuples
        from the \texttt{CASTING} relation where the \texttt{ID}
        attribute of the deleted movie has the same value as the
        \texttt{MOVIEID} attribute
    \end{itemize}
  \end{example}
\end{frame}

\begin{frame}[fragile]
  \frametitle{Defining Foreign Keys}

  \begin{block}{Statement}
    \begin{lstlisting}
CREATE TABLE table_name (
  ...
  [ { FOREIGN KEY ( column_name [, ...] )
      REFERENCES table_name
          [ ( column_name [, ...] ) ]
      [ ON DELETE option ]
      [ ON UPDATE option ] } [, ...] ]
  ...
)
    \end{lstlisting}
  \end{block}
\end{frame}

\begin{frame}[fragile]
  \frametitle{Foreign Key Definition Example}

  \begin{example}[creating the MOVIE table]
    \begin{lstlisting}
CREATE TABLE MOVIE (
  ID INTEGER PRIMARY KEY,
  TITLE VARCHAR(80) NOT NULL,
  ...
  DIRECTORID INTEGER,
  FOREIGN KEY DIRECTORID REFERENCES PERSON (ID)
    ON DELETE RESTRICT
    ON UPDATE CASCADE
)
    \end{lstlisting}
  \end{example}
\end{frame}

\begin{frame}[fragile]
  \frametitle{Defining Foreign Keys}

  \begin{itemize}
    \item if the foreign key is the primary key in the referenced table
      it does not have to be specified in the \lstinline!REFERENCES! part

    \pause
    \item if the foreign key consists of only one attribute, it can be
      specified in the column definition:\\
      \lstinline!column_name data_type REFERENCES table_name [ ( column_name ) ]!
  \end{itemize}
\end{frame}

\begin{frame}[fragile]
  \frametitle{Creating the Example Tables}

  \begin{example}[creating the MOVIE table]
    \begin{lstlisting}
CREATE TABLE MOVIE (
  ID INTEGER PRIMARY KEY,
  TITLE VARCHAR(80) NOT NULL,
  YR NUMERIC(4),
  COUNTRY CHAR(2),
  SCORE FLOAT,
  VOTES INTEGER DEFAULT 0,
  DIRECTORID INTEGER REFERENCES PERSON
)
    \end{lstlisting}
  \end{example}
\end{frame}

\begin{frame}[fragile]
  \frametitle{Creating the Example Tables}

  \begin{example}[creating the PERSON table]
    \begin{lstlisting}
CREATE TABLE PERSON (
  ID INTEGER PRIMARY KEY,
  NAME VARCHAR(40) UNIQUE NOT NULL
)
    \end{lstlisting}
  \end{example}
\end{frame}

\begin{frame}[fragile]
  \frametitle{Creating the Example Tables}

  \begin{example}[creating the CASTING table]
    \begin{lstlisting}
CREATE TABLE CASTING (
  MOVIEID INTEGER REFERENCES MOVIE,
  ACTORID INTEGER REFERENCES PERSON,
  ORD INTEGER,
  PRIMARY KEY (MOVIEID, ACTORID)
)
    \end{lstlisting}
  \end{example}
\end{frame}

\section*{References}

\begin{frame}
  \frametitle{References}

  \begin{block}{Required text: Date}
    \begin{itemize}
      \item Chapter 3: An Introduction to Relational Databases
      \begin{itemize}
        \item 3.2. \alert{An Informal Look at the Relational Model}
        \item 3.3. \alert{Relations and Relvars}
      \end{itemize}

      \item Chapter 6: \alert{Relations}

      \item Chapter 9: Integrity
      \begin{itemize}
        \item 9.10. \alert{Keys}
        \item 9.12. \alert{SQL Facilities}
      \end{itemize}
    \end{itemize}
  \end{block}
\end{frame}

\end{document}
