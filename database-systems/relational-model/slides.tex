% Copyright (c) 2002-2012
%       H. Turgut Uyar <uyar@itu.edu.tr>
%       Şule Gündüz Öğüdücü <sgunduz@itu.edu.tr>
%
% These notes are licensed using the
% "Creative Commons Attribution-NonCommercial-ShareAlike License".
% You are free to copy, distribute and transmit the work, and to adapt the work
% as long as you attribute the authors, do not use it for commercial purposes,
% and any derivative work is under the same or a similar license.
%
% Read the full legal code at:
% http://creativecommons.org/licenses/by-nc-sa/3.0/

\documentclass[dvipsnames]{beamer}

\usepackage{ae}
\usepackage[T1]{fontenc}
\usepackage[utf8]{inputenc}
\setbeamertemplate{navigation symbols}{}

\usepackage{listings}
\lstdefinelanguage{TutorialD}[]{}{
  morekeywords={BASE, BOOL, CHAR, DELETE, DROP, INSERT, INTEGER, KEY, POSSREP,
                RATIONAL, RELATION, TUPLE, TYPE, UPDATE, VAR, WHERE}
}
\lstdefinelanguage{FullSQL}[]{SQL}{
  morekeywords={BINARY, BOOLEAN, CYCLE, FINAL, INCREMENT, IS, LARGE, MAXVALUE,
                MINVALUE, NO_ACTION, OBJECT, REFERENCES, RENAME, SEQUENCE,
                START, TO, TYPE, VACUUM},
  deletekeywords={YEAR}
}

\mode<presentation>
{
  \usetheme{Warsaw}
  \usecolortheme[named=ForestGreen]{structure}
  \setbeamercovered{transparent}
}

\title{Database Systems}
\subtitle{Relational Model}

\author{H. Turgut Uyar \and Şule Öğüdücü}
\date{2002-2012}

\AtBeginSubsection[]{
  \begin{frame}<beamer>
    \frametitle{Topics}
    \tableofcontents[currentsection,currentsubsection]
  \end{frame}
}

\theoremstyle{plain}

\pgfdeclareimage[width=2cm]{license}{../../license}

\begin{document}

\begin{frame}
  \titlepage
\end{frame}

\begin{frame}
  \frametitle{License}

  \pgfuseimage{license}\hfill
  \copyright 2002-2012 T. Uyar, Ş. Öğüdücü

  \vfill
  \begin{tiny}
    You are free:
    \begin{itemize}
      \item to Share -- to copy, distribute and transmit the work
      \item to Remix -- to adapt the work
    \end{itemize}

    Under the following conditions:
    \begin{itemize}
      \item Attribution -- You must attribute the work in the manner specified by
        the author or licensor (but not in any way that suggests that they
        endorse you or your use of the work).

      \item Noncommercial -- You may not use this work for commercial purposes.

      \item Share Alike -- If you alter, transform, or build upon this work, you
        may distribute the resulting work only under the same or similar license
        to this one.
    \end{itemize}
  \end{tiny}

  \vfill
  Legal code (the full license):\\
  \url{http://creativecommons.org/licenses/by-nc-sa/3.0/}
\end{frame}

\begin{frame}
  \frametitle{Topics}
  \tableofcontents
\end{frame}

\lstset{language=TutorialD}

\section{Relational Model}

\subsection{Introduction}

\begin{frame}
  \frametitle{Data Models}

  \begin{itemize}
    \item previous models:
    \begin{itemize}
      \item inverted list
      \item hierarchical
      \item network
    \end{itemize}

    \pause
    \item relational model:
    \begin{itemize}
      \item Dr. E. F. Codd, 1970
    \end{itemize}

    \pause
    \item later models:
    \begin{itemize}
      \item object
      \item object / relational
    \end{itemize}
  \end{itemize}
\end{frame}

\begin{frame}
  \frametitle{Relational Model}

  \begin{itemize}
    \item data is modelled as \alert{relations}:\\
      $\alpha \subseteq A \times B \times C \times ...$

    \pause
    \medskip
    \item each element of a relation is a \alert{tuple}
    \item each data of an element is an \alert{attribute}

    \pause
    \medskip
    \item relations are represented using tables
    \begin{itemize}
      \item the user should \emph{perceive} all data as tables
      \item relation $\rightarrow$ table, tuple $\rightarrow$ row,
        attribute $\rightarrow$ column
    \end{itemize}
  \end{itemize}
\end{frame}

\begin{frame}
  \frametitle{Relation Example}

  \begin{example}[movie relation]
    \begin{tiny}
    \begin{table}
      \begin{tabular}{|l|r|l|c|r|r|}\hline
TITLE                & YEAR & DIRECTOR      & SCORE & VOTES\\\hline\hline
Usual Suspects       & 1995 & Bryan Singer  &   8.7 &  3502\\\hline
Suspiria             & 1977 & Dario Argento &   7.1 &  1004\\\hline
Being John Malkovich & 1999 & Spike Jonze   &   8.3 & 13809\\\hline
...                  &  ... & ...           &   ... &   ...\\\hline
      \end{tabular}
    \end{table}
    \end{tiny}

    \pause
    \begin{itemize}
      \item \texttt{(Usual Suspects,1995,Bryan Singer,8.7,3502)}\\
        is a tuple of the movie relation
      \item \texttt{YEAR} is an attribute of the movie relation
    \end{itemize}
  \end{example}
\end{frame}

\begin{frame}
  \frametitle{Relation Structure}

  \begin{block}{relation header}
    \begin{itemize}
      \item the set of attributes making up the relation
      \item specified when the relation is created
      \item affected by data definition language statements
    \end{itemize}
  \end{block}

  \pause
  \begin{block}{relation body}
    \begin{itemize}
      \item the set of tuples in the relation
      \item affected by data manipulation language statements
    \end{itemize}
  \end{block}
\end{frame}

\begin{frame}
  \frametitle{Relation Predicate}

  \begin{definition}
    \alert{relation predicate}:\\
      the sentence expressing the "meaning" of the relation
  \end{definition}

  \begin{itemize}
    \item each tuple is either \emph{True} or \emph{False} according to the
      predicate
  \end{itemize}
\end{frame}

\begin{frame}
  \frametitle{Relation Predicate Example}

  \begin{example}[movie relation predicate]
    \begin{itemize}
      \item the movie with the title \texttt{TITLE} was filmed in the year
	\texttt{YEAR},\\
        by the director \texttt{DIRECTOR}; the average of \texttt{VOTES} votes\\
        is \texttt{SCORE}.

      \pause
      \medskip
      \item the tuple \texttt{(Suspiria,1977,Dario Argento,1004,7.1)}\\
        is True
      \item the tuple \texttt{(Suspiria,1978,Dario Argento,1004,7.1)}\\
        is False
    \end{itemize}
  \end{example}
\end{frame}

\begin{frame}
  \frametitle{Tuple Order}

  \begin{itemize}
    \item the order of tuples is insignificant
  \end{itemize}

  \pause
  \begin{example}
    \begin{itemize}
      \item the following two relations are equivalent:
    \end{itemize}

    \begin{columns}
      \column{.5\textwidth}
      \begin{tiny}
      \begin{table}
        \begin{tabular}{|l|l|}\hline
TITLE                & ... \\\hline\hline
Usual Suspects       & ... \\\hline
Suspiria             & ... \\\hline
Being John Malkovich & ... \\\hline
        \end{tabular}
      \end{table}
      \end{tiny}

      \column{.5\textwidth}
      \begin{tiny}
      \begin{table}
        \begin{tabular}{|l|l|}\hline
TITLE                & ... \\\hline\hline
Suspiria             & ... \\\hline
Being John Malkovich & ... \\\hline
Usual Suspects       & ... \\\hline
        \end{tabular}
      \end{table}
      \end{tiny}
    \end{columns}
  \end{example}
\end{frame}

\begin{frame}
  \frametitle{Attribute Order}

  \begin{itemize}
    \item the order of attributes is insignificant
  \end{itemize}

  \pause
  \begin{example}
    \begin{itemize}
      \item the following two relations are equivalent:
    \end{itemize}

    \begin{columns}
      \column{.5\textwidth}
      \begin{tiny}
      \begin{table}
        \begin{tabular}{|l|r|l|}\hline
TITLE                & YEAR & ... \\\hline\hline
Usual Suspects       & 1995 & ... \\\hline
Suspiria             & 1977 & ... \\\hline
Being John Malkovich & 1999 & ... \\\hline
        \end{tabular}
      \end{table}
      \end{tiny}

      \column{.5\textwidth}
      \begin{tiny}
      \begin{table}
        \begin{tabular}{|r|l|l|}\hline
YEAR & TITLE                & ... \\\hline\hline
1995 & Usual Suspects       & ... \\\hline
1977 & Suspiria             & ... \\\hline
1999 & Being John Malkovich & ... \\\hline
        \end{tabular}
      \end{table}
      \end{tiny}
    \end{columns}
  \end{example}
\end{frame}

\begin{frame}
  \frametitle{Attribute Values}

  \begin{itemize}
    \item attribute values must be scalar
    \begin{itemize}
      \item arrays, lists, records etc. are not allowed
    \end{itemize}
  \end{itemize}

  \pause
  \begin{example}[multiple directors]
    \begin{tiny}
    \begin{table}
      \begin{tabular}{|l|c|l|c|}\hline
TITLE  & ... & DIRECTORS                      & ...\\\hline\hline
...    & ... & ...                            & ...\\\hline
Matrix & ... & Andy Wachowski, Lana Wachowski & ...\\\hline
...    & ... & ...                            & ...\\\hline
      \end{tabular}
    \end{table}
    \end{tiny}

    \pause
    \begin{picture}(90,5)(-123,-29)
      \color[rgb]{1,0.2,0.1}
      \thicklines
      \only<3->{
        \put(0,0){\line(1,0){90}}
      }
    \end{picture}
  \end{example}
\end{frame}

\begin{frame}
  \frametitle{Null Value}

  \begin{columns}[t]
    \column{.5\textwidth}
    \begin{itemize}
      \item the value of the attribute\\
	is not known for this tuple
    \end{itemize}

    \begin{example}
      \begin{itemize}
        \item the director of the movie\\
	  is not known
      \end{itemize}
    \end{example}

    \pause
    \column{.5\textwidth}
    \begin{itemize}
      \item this tuple does not have\\
        a value for this attribute
    \end{itemize}

    \begin{example}
      \begin{itemize}
        \item nobody voted for the movie,\\
	  therefore there is no \texttt{SCORE}
      \end{itemize}
    \end{example}
  \end{columns}
\end{frame}

\begin{frame}
  \frametitle{Default Value}

  \begin{itemize}
    \item default values can be used instead of null values
    \begin{itemize}
      \item it may not be one of the valid values for the attribute
    \end{itemize}
  \end{itemize}

  \pause
  \begin{example}
    \begin{itemize}
      \item if \texttt{SCORE} values are between 1.0 and 10.0,\\
	the default value can be chosen as \texttt{0.0}
    \end{itemize}
  \end{example}
\end{frame}

\begin{frame}
  \frametitle{Duplicate Tuples}

  \begin{itemize}
    \item there can not be duplicate tuples in a relation
    \begin{itemize}
      \item each tuple must be uniquely identifiable
    \end{itemize}
  \end{itemize}

  \pause
  \begin{example}
    \begin{tiny}
    \begin{table}
      \begin{tabular}{|l|r|l|c|r|r|}\hline
TITLE                & YEAR & DIRECTOR      & SCORE & VOTES\\\hline\hline
Usual Suspects       & 1995 & Bryan Singer  &   8.7 &  3502\\\hline
Suspiria             & 1977 & Dario Argento &   7.1 &  1004\\\hline
Being John Malkovich & 1999 & Spike Jonze   &   8.3 & 13809\\\hline
...                  &  ... & ...           &   ... &   ...\\\hline
Suspiria             & 1977 & Dario Argento &   7.1 &  1004\\\hline
...                  &  ... & ...           &   ... &   ...\\\hline
      \end{tabular}
    \end{table}
    \end{tiny}
  \end{example}

  \begin{picture}(20,40)(0,-63)
    \color[rgb]{1,0.2,0.1}
    \put(20,20){\vector(2,-1){25}}
    \put(20,20){\vector(2,1){25}}
  \end{picture}
\end{frame}

\begin{frame}
  \frametitle{Keys}

  \begin{itemize}
    \item let $B$ be the attributes of the relation and let $A \subseteq B$

    \item in order for $A$ to be a candidate key, the following must hold:

    \pause
    \begin{itemize}
      \item \alert{uniqueness}: no two tuples have the same values\\
        for all attributes in $A$

      \pause
      \item \alert{irreducibility}: no subset of $A$ satisfies the uniqueness
        property
    \end{itemize}

    \pause
    \item every relation has at least one candidate key
  \end{itemize}
\end{frame}

\begin{frame}
  \frametitle{Candidate Key Example}

  \begin{example}[candidate keys for movie relation]
    \begin{itemize}
      \item \texttt{\{TITLE\}}

      \pause
      \item \texttt{\{TITLE,YEAR\}}

      \pause
      \item \texttt{\{TITLE,DIRECTOR\}}

      \pause
      \item \texttt{\{TITLE,YEAR,DIRECTOR\}}
    \end{itemize}
  \end{example}
\end{frame}

\begin{frame}
  \frametitle{Surrogate Keys}

  \begin{itemize}
    \item if a \alert{natural key} can not be selected\\
      a \alert{surrogate key} can be defined

    \pause
    \medskip
    \item identity attributes
    \begin{itemize}
      \item their values do not matter
      \item they can be generated by the system
    \end{itemize}
  \end{itemize}
\end{frame}

\begin{frame}
  \frametitle{Surrogate Key Example}

  \begin{example}
    \begin{tiny}
    \begin{table}
      \begin{tabular}{|r|l|r|l|c|r|r|}\hline
MOVIE\# & TITLE                & YEAR & DIRECTOR      & SCORE & VOTES\\\hline\hline
    ... & ...                  &  ... & ...           & ...   &   ...\\\hline
      6 & Usual Suspects       & 1995 & Bryan Singer  & ...   &   ...\\\hline
   1512 & Suspiria             & 1977 & Dario Argento & ...   &   ...\\\hline
     70 & Being John Malkovich & 1999 & Spike Jonze   & ...   &   ...\\\hline
    ... & ...                  &  ... & ...           & ...   &   ...\\\hline
      \end{tabular}
    \end{table}
    \end{tiny}

    \pause
    \begin{itemize}
      \item \texttt{\{MOVIE\#\}} is a candidate key
      \item \texttt{\{MOVIE\#, TITLE\}} is not a candidate key
    \end{itemize}
  \end{example}
\end{frame}

\begin{frame}
  \frametitle{Primary Key}

  \begin{itemize}
    \item if a relation has more than one candidate key:
    \begin{itemize}
      \item one of them is selected as the \alert{primary key}
      \item the others are \alert{alternate keys}
    \end{itemize}

    \pause
    \item every relation must have a primary key

    \pause
    \item any attribute that is part of the primary key\\
      can not be empty in any tuple
  \end{itemize}
\end{frame}

\begin{frame}
  \frametitle{Primary Key Example}

  \begin{itemize}
    \item the names of the attributes in the primary key are underlined
  \end{itemize}

  \begin{example}
    \begin{tiny}
    \begin{table}
      \begin{tabular}{|r|l|r|l|c|r|r|}\hline
\underline{MOVIE\#} & TITLE & YEAR & DIRECTOR      & SCORE & VOTES\\\hline\hline
 ... & ...                  &  ... & ...           &   ... &   ...\\\hline
   6 & Usual Suspects       & 1995 & Bryan Singer  &   ... &   ...\\\hline
1512 & Suspiria             & 1977 & Dario Argento &   ... &   ...\\\hline
  70 & Being John Malkovich & 1999 & Spike Jonze   &   ... &   ...\\\hline
 ... & ...                  &  ... & ...           &   ... &   ...\\\hline
      \end{tabular}
    \end{table}
    \end{tiny}
  \end{example}
\end{frame}

\subsection{Data Definition}

\begin{frame}
  \frametitle{Data Definition Language}

  \begin{itemize}
    \item creating and deleting relations

    \pause
    \item changing the relation header
    \begin{itemize}
      \item adding and deleting attributes
      \item adding and deleting constraints
    \end{itemize}

    \pause
    \bigskip
    \item these notes are based on the "Tutorial D" language,\\
      "Rel" implementation
  \end{itemize}
\end{frame}

\begin{frame}
  \frametitle{Data Types}

  \begin{itemize}
    \item all values for the same attribute have to be selected\\
      from the same domain
    \begin{itemize}
      \item comparison only makes sense between values\\
        chosen from the same domain
    \end{itemize}

    \pause
    \medskip
    \item in practice, data types are used instead
  \end{itemize}
\end{frame}

\begin{frame}
  \frametitle{Domain Example}

  \begin{example}
    \begin{itemize}
      \item \texttt{TITLE} from the titles domain, \texttt{YEAR} from the years
        domain,\\
       \texttt{DIRECTOR} from the directors domain, ...

      \pause
      \item if data types are used:\\
        \texttt{TITLE} string, \texttt{YEAR} integer, \texttt{DIRECTOR} string,
          ...

      \begin{itemize}
        \item assigning the value \texttt{"Usual Suspects"} to the attribute
          \texttt{DIRECTOR} is correct in terms of data types\\
          but incorrect according to predicate

        \item \texttt{YEAR} and \texttt{VOTES} are integers\\
          but it does not make sense to compare them
      \end{itemize}
    \end{itemize}
  \end{example}
\end{frame}

\begin{frame}
  \frametitle{Data Types}

  \begin{itemize}
    \item \lstinline!INTEGER!
    \item \lstinline!RATIONAL!
    \item \lstinline!BOOL!
    \item \lstinline!CHAR!
  \end{itemize}
\end{frame}

\begin{frame}[fragile]
  \frametitle{Type Definition}

  \begin{itemize}
    \item new types can be derived from existing types
    \item values must be appropriately generated for derived type
  \end{itemize}

  \pause
  \begin{block}{Statement}
    \begin{lstlisting}
TYPE type_name
  POSSREP { field_name field_type [, ...] }
    \end{lstlisting}
  \end{block}

  \pause
  \begin{itemize}
    \item to generate a value: \lstinline!type_name(base_value [, ...])!

    \pause
    \item to delete a type: \lstinline!DROP TYPE type_name!
  \end{itemize}
\end{frame}

\begin{frame}[fragile]
  \frametitle{Type Definition Examples}

  \begin{example}
    \begin{lstlisting}
TYPE MOVIE# POSSREP { VALUE INTEGER }

TYPE YEAR POSSREP { VALUE INTEGER }

TYPE SCORE POSSREP { VALUE RATIONAL }
    \end{lstlisting}

    \pause
    \bigskip
    \begin{itemize}
      \item to generate a \texttt{SCORE} value: \lstinline!SCORE(8.7)!

      \pause
      \item to delete the \texttt{YEAR} type: \lstinline!DROP TYPE YEAR!
    \end{itemize}
  \end{example}
\end{frame}

\begin{frame}[fragile]
  \frametitle{Relation Definition}

  \begin{block}{Statement}
    \begin{lstlisting}
RELATION
  { attribute_name attribute_type [, ...] }
  KEY { key_attribute [, ...] }
    \end{lstlisting}
  \end{block}
\end{frame}

\begin{frame}[fragile]
  \frametitle{Relation Definition Example}

  \begin{example}
    \begin{lstlisting}
RELATION
  { MOVIE# MOVIE#,
    TITLE CHAR,
    YEAR YEAR,
    DIRECTOR CHAR,
    SCORE SCORE,
    VOTES INTEGER }
  KEY { MOVIE# }
    \end{lstlisting}
  \end{example}
\end{frame}

\begin{frame}[fragile]
  \frametitle{Base Relation Variable Creation}

  \begin{block}{Statement}
    \begin{lstlisting}
VAR variable_name BASE RELATION
  { ... }
  KEY { ... }
    \end{lstlisting}
  \end{block}

  \pause
  \begin{itemize}
    \item to delete a variable: \lstinline!DROP VAR variable_name!
  \end{itemize}
\end{frame}

\begin{frame}[fragile]
  \frametitle{Base Relation Variable Creation Example}

  \begin{example}
    \begin{lstlisting}
VAR MOVIE BASE RELATION
  { MOVIE# MOVIE#,
    TITLE CHAR,
    YEAR YEAR,
    DIRECTOR CHAR,
    SCORE SCORE,
    VOTES INTEGER }
  KEY { MOVIE# }
    \end{lstlisting}

    \pause
    \bigskip
    \begin{itemize}
      \item to delete the \texttt{MOVIE} variable: \lstinline!DROP VAR MOVIE!
    \end{itemize}
  \end{example}
\end{frame}

\begin{frame}[fragile]
  \frametitle{Tuple and Relation Generation}

  \begin{block}{Tuple Generation}
    \begin{lstlisting}
TUPLE {
  attribute_name attribute_value
  [, ...]
}
    \end{lstlisting}
  \end{block}

  \pause
  \begin{block}{Relation Generation}
    \begin{lstlisting}
RELATION {
  TUPLE { ... }
  [, ...]
}
    \end{lstlisting}
  \end{block}
\end{frame}

\begin{frame}[fragile]
  \frametitle{Relation Variable Assignment}

  \begin{block}{Statement}
    \begin{lstlisting}
variable_name := RELATION { ... }
    \end{lstlisting}
  \end{block}
\end{frame}

\begin{frame}[fragile]
  \frametitle{Relation Variable Assignment Example}

  \begin{example}
    \begin{lstlisting}
MOVIE := RELATION {
  TUPLE { MOVIE# MOVIE#(6),
    TITLE "Usual Suspects",
    YEAR YEAR(1995), DIRECTOR "Bryan Singer",
    SCORE SCORE(8.7), VOTES 35027 },
  TUPLE { MOVIE# MOVIE#(70),
    TITLE "Being John Malkovich",
    YEAR YEAR(1999), DIRECTOR "Spike Jonze",
    SCORE SCORE(8.3), VOTES 13809 }
}
    \end{lstlisting}
  \end{example}
\end{frame}

\subsection{Data Manipulation}

\begin{frame}
  \frametitle{Data Manipulation Language}

  \begin{itemize}
    \item inserting tuples
    \item deleting tuples
    \item updating tuples

    \pause
    \bigskip
    \item operations are carried out on sets of tuples
    \begin{itemize}
      \item all tuples satisfying a condition
    \end{itemize}
  \end{itemize}
\end{frame}

\begin{frame}[fragile]
  \frametitle{Tuple Insertion}

  \begin{block}{Statement}
    \begin{lstlisting}
INSERT relation_variable_name RELATION {
  TUPLE { ... }
  [, ...]
}
    \end{lstlisting}
  \end{block}
\end{frame}

\begin{frame}[fragile]
  \frametitle{Tuple Insertion Example}

  \begin{example}
    \begin{lstlisting}
INSERT MOVIE RELATION {
  TUPLE { MOVIE# MOVIE#(1512),
    TITLE "Suspiria",
    YEAR YEAR(1977), DIRECTOR "Dario Argento",
    SCORE SCORE(7.1), VOTES 1004 }
}
    \end{lstlisting}
  \end{example}
\end{frame}

\begin{frame}[fragile]
  \frametitle{Tuple Deletion}

  \begin{block}{Statement}
    \begin{lstlisting}
DELETE relation_variable_name
  [ WHERE condition ]
    \end{lstlisting}
  \end{block}

  \pause
  \begin{itemize}
    \item if no condition is specified, all tuples will be deleted
  \end{itemize}
\end{frame}

\begin{frame}[fragile]
  \frametitle{Tuple Deletion Example}

  \begin{example}
    \begin{lstlisting}
DELETE MOVIE
  WHERE (SCORE < SCORE(3.0))
    \end{lstlisting}
  \end{example}
\end{frame}

\begin{frame}[fragile]
  \frametitle{Tuple Update}

  \begin{block}{Statement}
    \begin{lstlisting}
UPDATE relation_variable_name
  [ WHERE condition ]
  { attribute_name := expression [, ...] }
    \end{lstlisting}
  \end{block}

  \pause
  \begin{itemize}
    \item if no condition is specified, all tuples will be updated
  \end{itemize}
\end{frame}

\begin{frame}[fragile]
  \frametitle{Tuple Update Example}

  \begin{example}
    \begin{lstlisting}
UPDATE MOVIE
  WHERE (YEAR > YEAR(1990))
  { VOTES := VOTES + 1, SCORE := 5.0 }
    \end{lstlisting}
  \end{example}
\end{frame}

\subsection{Integrity}

\begin{frame}
  \frametitle{Value Constraint}

  \begin{definition}
    \alert{value constraint}:\\
      value must satisfy a Boolean expression
  \end{definition}

  \pause
  \begin{example}
    \begin{itemize}
      \item \texttt{SCORE} values must be between \texttt{1.0} and \texttt{10.0}
    \end{itemize}
  \end{example}
\end{frame}

\begin{frame}
  \frametitle{Scalar Values}

  \begin{itemize}
    \item in order to meet the scalar value requirement,\\
      tuples may have to be partly repeated
  \end{itemize}

  \begin{example}[how to store actor data?]
    \begin{tiny}
    \begin{table}
      \begin{tabular}{|r|l|c|l|}\hline
\underline{MOVIE\#} & TITLE    & ... & ACTORS                     \\\hline\hline
      6 & Usual Suspects       & ... & Gabriel Byrne              \\\hline
    ... & ...                  & ... & ...                        \\\hline
     70 & Being John Malkovich & ... & Cameron Diaz,John Malkovich\\\hline
    ... & ...                  & ... & ...                        \\\hline
      \end{tabular}
    \end{table}
    \end{tiny}

    \pause
    \begin{picture}(90,5)(-173,-29)
      \color[rgb]{1,0.2,0.1}
      \thicklines
      \only<2->{
        \put(0,0){\line(1,0){90}}
      }
    \end{picture}

    \pause
    \begin{tiny}
    \begin{table}
      \begin{tabular}{|r|l|c|l|}\hline
MOVIE\# & TITLE                & ... & ACTOR         \\\hline\hline
      6 & Usual Suspects       & ... & Gabriel Byrne \\\hline
    ... & ...                  & ... & ...           \\\hline
     70 & Being John Malkovich & ... & Cameron Diaz  \\\hline
     70 & Being John Malkovich & ... & John Malkovich\\\hline
    ... & ...                  & ... & ...           \\\hline
      \end{tabular}
    \end{table}
    \end{tiny}
  \end{example}
\end{frame}

\begin{frame}
  \frametitle{Scalar Value Example}

  \begin{example}[movies and actors]
    \begin{tiny}
    \begin{table}
      \caption{MOVIE}
      \begin{tabular}{|r|l|c|}\hline
\underline{MOVIE\#} & TITLE                & ...\\\hline\hline
                  6 & Usual Suspects       & ...\\\hline
               1512 & Suspiria             & ...\\\hline
                 70 & Being John Malkovich & ...\\\hline
                ... & ...                  & ...\\\hline
      \end{tabular}
    \end{table}
    \end{tiny}

    \begin{columns}
      \column{.5\textwidth}
      \begin{tiny}
      \begin{table}
        \caption{ACTOR}
        \begin{tabular}{|r|l|}\hline
\underline{ACTOR\#} & NAME          \\\hline\hline
                308 & Gabriel Byrne \\\hline
                282 & Cameron Diaz  \\\hline
                503 & John Malkovich\\\hline
                ... & ...           \\\hline
        \end{tabular}
      \end{table}
      \end{tiny}

      \column{.5\textwidth}
      \begin{tiny}
      \begin{table}
        \caption{CASTING}
        \begin{tabular}{|r|r|r|}\hline
\underline{MOVIE\#} & \underline{ACTOR\#} & ORD\\\hline\hline
                  6 &                 308 &   2\\\hline
                 70 &                 282 &   2\\\hline
                 70 &                 503 &  14\\\hline
                ... &                 ... & ...\\\hline
        \end{tabular}
      \end{table}
      \end{tiny}
    \end{columns}
  \end{example}
\end{frame}

\begin{frame}
  \frametitle{Scalar Value Example}

  \begin{example}[how to store director data?]
    \begin{tiny}
    \begin{table}
      \caption{MOVIE}
      \begin{tabular}{|r|l|c|r|}\hline
\underline{MOVIE\#} & TITLE                & ... & DIRECTOR\#\\\hline\hline
                  6 & Usual Suspects       & ... &        639\\\hline
               1512 & Suspiria             & ... &       2259\\\hline
                 70 & Being John Malkovich & ... &       1485\\\hline
                ... & ...                  & ... &        ...\\\hline
      \end{tabular}
    \end{table}
    \end{tiny}

    \begin{columns}
      \column{.5\textwidth}
      \begin{tiny}
      \begin{table}
        \caption{PERSON}
        \begin{tabular}{|r|l|}\hline
\underline{PERSON\#} & NAME\\\hline\hline
                 308 & Gabriel Byrne \\\hline
                1485 & Spike Jonze   \\\hline
                 639 & Bryan Singer  \\\hline
                 282 & Cameron Diaz  \\\hline
                2259 & Dario Argento \\\hline
                 503 & John Malkovich\\\hline
                 ... & ...           \\\hline
        \end{tabular}
      \end{table}
      \end{tiny}

      \column{.5\textwidth}
      \begin{tiny}
      \begin{table}
        \caption{CASTING}
        \begin{tabular}{|r|r|r|}\hline
\underline{MOVIE\#} & \underline{ACTOR\#} & ORD\\\hline\hline
                  6 &                 308 &   2\\\hline
                 70 &                 282 &   2\\\hline
                 70 &                 503 &  14\\\hline
                ... &                 ... & ...\\\hline
        \end{tabular}
      \end{table}
      \end{tiny}
    \end{columns}
  \end{example}
\end{frame}

\begin{frame}
  \frametitle{Foreign Keys}

  \begin{definition}
    \alert{foreign key}:\\
      an attribute of a relation is the candidate key of another relation
  \end{definition}
\end{frame}

\begin{frame}
  \frametitle{Foreign Key Example}

  \begin{example}[DIRECTORID attribute of MOVIE relation]
    \begin{columns}[t]
      \column{.6\textwidth}
      \begin{tiny}
      \begin{table}
        \caption{MOVIE}
        \begin{tabular}{|r|l|c|r|}\hline
\underline{ID} & TITLE      & ... & DIRECTORID\\\hline\hline
   6 & Usual Suspects       & ... &        639\\\hline
1512 & Suspiria             & ... &       2259\\\hline
  70 & Being John Malkovich & ... &       1485\\\hline
 ... & ...                  & ... &        ...\\\hline
        \end{tabular}
      \end{table}
      \end{tiny}

      \column{.4\textwidth}
      \begin{tiny}
      \begin{table}
        \caption{PERSON}
        \begin{tabular}{|r|l|}\hline
\underline{ID} & NAME\\\hline\hline
 308 & Gabriel Byrne \\\hline
1485 & Spike Jonze   \\\hline
 639 & Bryan Singer  \\\hline
 282 & Cameron Diaz  \\\hline
2259 & Dario Argento \\\hline
 503 & John Malkovich\\\hline
 ... & ...           \\\hline
        \end{tabular}
      \end{table}
      \end{tiny}
    \end{columns}
  \end{example}

  \begin{picture}(100,50)(-140,-90)
    \color[rgb]{0.1,0.6,0.1}
    \thicklines
    \only<2->{
      \put(12,22){\oval(20,8)}         % movie.directorid=2259
      \put(88,0){\oval(20,8)}          % person.id=2259
      \put(23,21){\vector(3,-1){55}}   % movie.directorid=2259->person.id=2259
    }
    \only<3->{
      \put(0,39){\oval(45,9)}          % movie.directorid
      \put(90,39){\oval(20,9)}         % person.id
      \put(24,39){\vector(1,0){56}}    % movie.directorid->person.id
    }
  \end{picture}
\end{frame}

\begin{frame}
  \frametitle{Foreign Key Example}

  \begin{example}[foreign keys in movie database]
    \begin{columns}
      \column{.6\textwidth}
      \begin{tiny}
      \begin{table}
        \caption{MOVIE}
        \begin{tabular}{|r|l|c|r|}\hline
\underline{ID} & TITLE & ... & DIRECTORID\\\hline\hline
   6 & Usual Suspects  & ... &        639\\\hline
1512 & Suspiria        & ... &       2259\\\hline
 ... & ...             & ... &        ...\\\hline
        \end{tabular}
      \end{table}
      \end{tiny}

      \column{.4\textwidth}
      \begin{tiny}
      \begin{table}
        \caption{PERSON}
        \begin{tabular}{|r|l|}\hline
\underline{ID} & NAME\\\hline\hline
 308 & Gabriel Byrne \\\hline
1485 & Spike Jonze   \\\hline
 ... & ...           \\\hline
        \end{tabular}
      \end{table}
      \end{tiny}
    \end{columns}

    \begin{tiny}
    \begin{table}
      \caption{CASTING}
      \begin{tabular}{|r|r|r|}\hline
\underline{MOVIEID} & \underline{ACTORID} & ORD\\\hline\hline
  6 & 308 &   2\\\hline
 70 & 282 &   2\\\hline
... & ... & ...\\\hline
      \end{tabular}
    \end{table}
    \end{tiny}
  \end{example}

  \begin{picture}(210,80)(-25,-126)
    \color[rgb]{0.1,0.6,0.1}
    \thicklines
    \only<2->{
      \put(105,82){\oval(45,9)}        % movie.directorid
      \put(209,82){\oval(20,9)}        % person.id
      \put(128,82){\vector(1,0){71}}   % movie.directorid->person.id
    }
    \only<3->{
      \put(96,2){\oval(35,9)}          % casting.movieid
      \put(0,82){\oval(20,9)}          % movie.id
      \put(96,7){\vector(-4,3){92}}    % casting.movieid->movie.id
    }
    \only<4->{
      \put(136,2){\oval(35,9)}         % casting.personid
      \put(150,7){\vector(3,4){53}}    % casting.personid->person.id
    }
  \end{picture}
\end{frame}

\begin{frame}
  \frametitle{Referential Integrity}

  \begin{definition}
    \alert{referential integrity}:\\
      all values of a foreign key attribute must be present\\
      in the corresponding attribute of the referenced relation
  \end{definition}

  \pause
  \begin{itemize}
    \item what if a request conflicts with referential integrity?
  \end{itemize}
\end{frame}

\begin{frame}
  \frametitle{Referential Integrity Example}

  \begin{example}
    \begin{columns}
      \column{.5\textwidth}
      \begin{tiny}
      \begin{table}
        \caption{MOVIE}
        \begin{tabular}{|r|l|c|r|}\hline
\underline{ID} & TITLE & ... & DIRECTORID\\\hline\hline
 ... & ...             & ... &        ...\\\hline
1512 & Suspiria        & ... &       2259\\\hline
 ... & ...             & ... &        ...\\\hline
        \end{tabular}
      \end{table}
      \end{tiny}

      \column{.5\textwidth}
      \begin{tiny}
      \begin{table}
        \caption{PERSON}
        \begin{tabular}{|r|l|}\hline
\underline{ID} & NAME\\\hline\hline
 ... & ...           \\\hline
2259 & Dario Argento \\\hline
 ... & ...           \\\hline
        \end{tabular}
      \end{table}
      \end{tiny}
    \end{columns}

    \pause
    \begin{itemize}
      \item delete the tuple \texttt{(2259,Dario Argento)}
      \item update the tuple \texttt{(2259,Dario Argento)} as\\
        \texttt{(2871,Dario Argento)}
    \end{itemize}
  \end{example}
\end{frame}

\lstset{language=FullSQL}

\section{SQL}

\subsection{Data Definition}

\begin{frame}
  \frametitle{Data Types}

  \begin{itemize}
    \item \lstinline!INTEGER!
    \begin{itemize}
      \item \lstinline!SMALLINT!
    \end{itemize}

    \pause
    \medskip
    \item \lstinline!NUMERIC (precision, scale)!
    \begin{itemize}
      \item \texttt{precision}: total number of digits
      \item \texttt{scale}: number of digits after the decimal point
      \item same as: \lstinline!DECIMAL (precision, scale)!
    \end{itemize}

    \pause
    \medskip
    \item \lstinline!FLOAT (p)!
    \begin{itemize}
      \item \texttt{p}: lowest acceptable precision
    \end{itemize}

    \pause
    \medskip
    \item \lstinline!BOOLEAN!
  \end{itemize}
\end{frame}

\begin{frame}
  \frametitle{Data Types}

  \begin{itemize}
    \item \lstinline!CHARACTER [ VARYING ] (n)!
    \begin{itemize}
      \item in \lstinline!CHARACTER (n)!, if the string is shorter than
        \texttt{n} characters\\
        it will be padded with spaces
    \end{itemize}

    \pause
    \item abbreviations:
    \begin{itemize}
      \item \lstinline!CHAR (n)! instead of \lstinline!CHARACTER (n)!
      \item \lstinline!VARCHAR (n)! instead of \lstinline!CHARACTER VARYING (n)!
    \end{itemize}
  \end{itemize}
\end{frame}

\begin{frame}
  \frametitle{Data Types}

  \begin{itemize}
    \item \lstinline!DATE!
    \begin{itemize}
      \item value example: \texttt{2005-09-26}
    \end{itemize}

    \pause
    \medskip
    \item \lstinline!TIME!
    \begin{itemize}
      \item value example: \texttt{11:59:22.078717}
    \end{itemize}

    \pause
    \medskip
    \item \lstinline!TIMESTAMP!
    \begin{itemize}
      \item value example: \texttt{2005-09-26 11:59:22.078717}
    \end{itemize}

    \pause
    \medskip
    \item \lstinline!INTERVAL!
    \begin{itemize}
      \item value example: \texttt{3 days}
    \end{itemize}
  \end{itemize}
\end{frame}

\begin{frame}
  \frametitle{Data Types}

  \begin{itemize}
    \item arbitrary length
    \item can not be used in queries

    \pause
    \bigskip
    \item text: \lstinline!CHARACTER LARGE OBJECT (n)!
    \begin{itemize}
      \item \lstinline!CLOB!
    \end{itemize}

    \pause
    \item binary: \lstinline!BINARY LARGE OBJECT (n)!
    \begin{itemize}
      \item \lstinline!BLOB!
      \item picture, audio, etc.
    \end{itemize}
  \end{itemize}
\end{frame}

\begin{frame}[fragile]
  \frametitle{Table Creation}

  \begin{block}{Statement}
    \begin{lstlisting}
CREATE TABLE table_name (
  { column_name data_type
              [ NULL | NOT NULL ]
              [ DEFAULT default_value ] }
  [, ... ]
)
    \end{lstlisting}
  \end{block}

  \pause
  \begin{itemize}
    \item \lstinline!NULL!: the attribute is allowed to be NULL (default)
    \item \lstinline!NOT NULL!: the attribute is not allowed to be NULL
  \end{itemize}
\end{frame}

\begin{frame}[fragile]
  \frametitle{Table Deletion}

  \begin{block}{Statement}
    \begin{lstlisting}
DROP TABLE table_name [, ... ]
    \end{lstlisting}
  \end{block}
\end{frame}

\begin{frame}[fragile]
  \frametitle{Table Deletion Example}

  \begin{example}
    \begin{lstlisting}
CREATE TABLE MOVIE (
  TITLE VARCHAR(80) NOT NULL,
  YEAR NUMERIC(4),
  DIRECTOR VARCHAR(40),
  COUNTRY CHAR(2),
  SCORE FLOAT,
  VOTES INTEGER DEFAULT 0
)
    \end{lstlisting}
  \end{example}
\end{frame}

\begin{frame}[fragile]
  \frametitle{Renaming Tables}

  \begin{block}{Statement}
    \begin{lstlisting}
ALTER TABLE table_name
  RENAME TO new_name
    \end{lstlisting}
  \end{block}

  \pause
  \begin{example}
    \begin{lstlisting}
ALTER TABLE MOVIE
  RENAME TO FILM
    \end{lstlisting}
  \end{example}
\end{frame}

\begin{frame}[fragile]
  \frametitle{Adding Columns}

  \begin{block}{Statement}
    \begin{lstlisting}
ALTER TABLE table_name
  ADD [ COLUMN ] column_name data_type
                 [ NULL | NOT NULL ]
                 [ DEFAULT default_value ]
    \end{lstlisting}
  \end{block}

  \pause
  \begin{example}
    \begin{lstlisting}
ALTER TABLE MOVIE
  ADD COLUMN LANGUAGE CHAR(2)
    \end{lstlisting}
  \end{example}
\end{frame}

\begin{frame}[fragile]
  \frametitle{Deleting Columns}

  \begin{block}{Statement}
    \begin{lstlisting}
ALTER TABLE table_name
  DROP [ COLUMN ] column_name
    \end{lstlisting}
  \end{block}

  \pause
  \begin{example}
    \begin{lstlisting}
ALTER TABLE MOVIE
  DROP COLUMN LANGUAGE
    \end{lstlisting}
  \end{example}
\end{frame}

\begin{frame}[fragile]
  \frametitle{Renaming Columns}

  \begin{block}{Statement}
    \begin{lstlisting}
ALTER TABLE table_name
  RENAME [ COLUMN ] column_name TO new_name
    \end{lstlisting}
  \end{block}

  \pause
  \begin{example}
    \begin{lstlisting}
ALTER TABLE MOVIE
  RENAME COLUMN TITLE TO NAME
    \end{lstlisting}
  \end{example}
\end{frame}

\begin{frame}[fragile]
  \frametitle{Changing Column Default Value}

  \begin{block}{Statement}
    \begin{lstlisting}
ALTER TABLE table_name
  ALTER [ COLUMN ] column_name
  SET DEFAULT default_value
    \end{lstlisting}
  \end{block}

  \pause
  \begin{example}
    \begin{lstlisting}
ALTER TABLE MOVIE
  ALTER COLUMN SCORE
  SET DEFAULT 0
    \end{lstlisting}
  \end{example}
\end{frame}

\begin{frame}[fragile]
  \frametitle{Removing Column Default Value}

  \begin{block}{Statement}
    \begin{lstlisting}
ALTER TABLE table_name
  ALTER [ COLUMN ] column_name
  DROP DEFAULT
    \end{lstlisting}
  \end{block}

  \pause
  \begin{example}
    \begin{lstlisting}
ALTER TABLE MOVIE
  ALTER COLUMN SCORE
  DROP DEFAULT
    \end{lstlisting}
  \end{example}
\end{frame}

\begin{frame}[fragile]
  \frametitle{Candidate Key Definition}

  \begin{block}{Statement}
    \begin{lstlisting}
CREATE TABLE table_name (
  ...
  [ { UNIQUE ( column_name [, ...] ) }
    [, ...] ]
  ...
)
    \end{lstlisting}
  \end{block}

  \begin{itemize}
    \item if a candidate key consists of only one column,\\
      it can be specified at column definition:\\
      \lstinline!column_name data_type UNIQUE!
  \end{itemize}
\end{frame}

\begin{frame}[fragile]
  \frametitle{Candidate Key Definition Example}

  \begin{example}[\texttt{{ID}} and \texttt{\{TITLE, DIRECTOR\}} are unique]
    \begin{lstlisting}
CREATE TABLE MOVIE (
  ID INTEGER,
  TITLE VARCHAR(80) NOT NULL,
  ...
  VOTES INTEGER DEFAULT 0,
  CHECK (YEAR >= 1888),
  UNIQUE (ID),
  UNIQUE (TITLE, DIRECTOR)
)
    \end{lstlisting}
  \end{example}
\end{frame}

\begin{frame}[fragile]
  \frametitle{Candidate Key Definition Example}

  \begin{example}[\texttt{ID} is unique and can not be empty]
    \begin{lstlisting}
CREATE TABLE MOVIE (
  ID INTEGER UNIQUE NOT NULL,
  TITLE VARCHAR(80) NOT NULL,
  ...
  VOTES INTEGER DEFAULT 0,
  CHECK (YEAR >= 1888),
  UNIQUE (TITLE, DIRECTOR)
)
   \end{lstlisting}
  \end{example}
\end{frame}

\begin{frame}[fragile]
  \frametitle{Automatic Identity Values}

  \begin{itemize}
    \item product-specific definitions

    \begin{itemize}
      \item PostgreSQL: \texttt{SERIAL} data type
      \item MySQL: \texttt{AUTO\_INCREMENT} property
    \end{itemize}

    \pause
    \item not specified when adding rows
  \end{itemize}
\end{frame}

\begin{frame}[fragile]
  \frametitle{Automatic Identity Value Example}

  \begin{example}[PostgreSQL]
    \begin{lstlisting}
CREATE TABLE MOVIE (
  ID SERIAL UNIQUE NOT NULL,
  TITLE VARCHAR(80) NOT NULL,
  ...
)
    \end{lstlisting}
  \end{example}
\end{frame}

\begin{frame}[fragile]
  \frametitle{Automatic Identity Value Example}

  \begin{example}[MySQL]
    \begin{lstlisting}
CREATE TABLE MOVIE (
  ID INTEGER UNIQUE NOT NULL AUTO_INCREMENT,
  TITLE VARCHAR(80) NOT NULL,
  ...
)
    \end{lstlisting}
  \end{example}
\end{frame}

\begin{frame}[fragile]
  \frametitle{Primary Key Definition}

  \begin{block}{Statement}
    \begin{lstlisting}
CREATE TABLE table_name (
  ...
  [ PRIMARY KEY ( column_name [, ...] ) ]
  ...
)
    \end{lstlisting}
  \end{block}

  \begin{itemize}
    \item if the primary key consists of only one column,\\
      it can be specified at column definition:\\
      \lstinline!column_name data_type PRIMARY KEY!
  \end{itemize}
\end{frame}

\begin{frame}[fragile]
  \frametitle{Primary Key Definition Example}

  \begin{example}[{ID} is the primary key]
    \begin{lstlisting}
CREATE TABLE MOVIE (
  ID INTEGER,
  TITLE VARCHAR(80) NOT NULL,
  ...
  VOTES INTEGER DEFAULT 0,
  CHECK (YEAR >= 1888),
  UNIQUE (TITLE, DIRECTOR),
  PRIMARY KEY (ID)
)
    \end{lstlisting}
  \end{example}
\end{frame}

\begin{frame}[fragile]
  \frametitle{Primary Key Definition Example}

  \begin{example}[ID is the primary key]
    \begin{lstlisting}
CREATE TABLE MOVIE (
  ID INTEGER PRIMARY KEY,
  TITLE VARCHAR(80) NOT NULL,
  ...
  VOTES INTEGER DEFAULT 0,
  CHECK (YEAR >= 1888),
  UNIQUE (TITLE, DIRECTOR)
)
    \end{lstlisting}
  \end{example}
\end{frame}

\subsection{Data Manipulation}

\begin{frame}[fragile]
  \frametitle{Inserting Rows}

  \begin{block}{Statement}
    \begin{lstlisting}
INSERT INTO table_name
  [ ( column_name [, ...] ) ]
  VALUES ( column_value [, ...] )
    \end{lstlisting}
  \end{block}

  \pause
  \begin{itemize}
    \item the order of values must match the order of columns
    \item if the column names are not specified,\\
      values must be written in the order used at table creation
    \item unspecified attributes will take their default values
  \end{itemize}
\end{frame}

\begin{frame}[fragile]
  \frametitle{Row Insertion Example}

  \begin{example}
    \begin{lstlisting}
INSERT INTO MOVIE VALUES (
  'Usual Suspects',
  1995,
  'Bryan Singer',
  'UK',
  8.7,
  35027
)
    \end{lstlisting}
  \end{example}
\end{frame}

\begin{frame}[fragile]
  \frametitle{Row Insertion Example}

  \begin{example}
    \begin{lstlisting}
INSERT INTO MOVIE (YEAR, TITLE) VALUES (
  1995,
  'Usual Suspects'
)
    \end{lstlisting}
  \end{example}
\end{frame}

\begin{frame}[fragile]
  \frametitle{Deleting Rows}

  \begin{block}{Statement}
    \begin{lstlisting}
DELETE FROM table_name
  [ WHERE condition ]
    \end{lstlisting}
  \end{block}

  \pause
  \begin{itemize}
    \item if no condition is specified, all rows will be deleted
  \end{itemize}
\end{frame}

\begin{frame}[fragile]
  \frametitle{Row Deletion Example}

  \begin{example}[delete all movies with scores less than 3]
    \begin{lstlisting}
DELETE FROM MOVIE
  WHERE (SCORE < 3)
    \end{lstlisting}
  \end{example}
\end{frame}

\begin{frame}[fragile]
  \frametitle{Row Deleting Example}

  \begin{example}[delete all movies with scores less than 3
                  and at least 5 votes]
    \begin{lstlisting}
DELETE FROM MOVIE
  WHERE ((SCORE < 3) AND (VOTES >= 5))
    \end{lstlisting}
  \end{example}
\end{frame}

\begin{frame}[fragile]
  \frametitle{Updating Rows}

  \begin{block}{Statement}
    \begin{lstlisting}
UPDATE table_name
  SET { column_name = column_value } [, ...]
  [ WHERE condition ]
    \end{lstlisting}
  \end{block}

  \pause
  \begin{itemize}
    \item if no condition is specified, all rows will be updated
  \end{itemize}
\end{frame}

\begin{frame}[fragile]
  \frametitle{Row Updating Example}

  \begin{example}[reset the scores and votes of all movies before 1997 to zero]
    \begin{lstlisting}
UPDATE MOVIE
  SET SCORE = 0, VOTES = 0
  WHERE (YEAR < 1997)
    \end{lstlisting}
  \end{example}
\end{frame}

\begin{frame}[fragile]
  \frametitle{Row Updating Example}

  \begin{example}[vote 9 for "Suspiria"]
    \begin{lstlisting}
UPDATE MOVIE
  SET SCORE = (SCORE * VOTES + 9) / (VOTES + 1),
      VOTES = VOTES + 1
  WHERE (TITLE = 'Suspiria')
    \end{lstlisting}
  \end{example}
\end{frame}

\subsection{Integrity}

\begin{frame}[fragile]
  \frametitle{Specifying Value Constraints}

  \begin{block}{Statement}
    \begin{lstlisting}
CREATE TABLE table_name (
  ...
  [ { CHECK ( expression ) } [, ...] ]
  ...
)
    \end{lstlisting}
  \end{block}
\end{frame}

\begin{frame}[fragile]
  \frametitle{Value Constraint Example}

  \begin{example}[YEAR values can not be smaller than 1888]
    \begin{lstlisting}
CREATE TABLE MOVIE (
  TITLE VARCHAR(80) NOT NULL,
  YEAR NUMERIC(4),
  ...,
  VOTES INTEGER DEFAULT 0,
  CHECK (YEAR >= 1888)
)
    \end{lstlisting}
  \end{example}
\end{frame}

\begin{frame}[fragile]
  \frametitle{Creating Domains}

  \begin{block}{Statement}
    \begin{lstlisting}
CREATE DOMAIN domain_name [ AS ] base_type
  [ DEFAULT default_value ]
  [ constraint_definition [, ...] ]
    \end{lstlisting}
  \end{block}

  \pause
  \begin{itemize}
    \item the created domain can be used as a column type
  \end{itemize}

  \pause
  \begin{block}{Deleting Domains}
    \begin{lstlisting}
DROP DOMAIN domain_name [, ...]
    \end{lstlisting}
  \end{block}
\end{frame}

\begin{frame}[fragile]
  \frametitle{Domain Example}

  \begin{example}[domain for year values]
    \begin{lstlisting}
CREATE DOMAIN YEARS AS NUMERIC(4)
  DEFAULT 2011
  CHECK (VALUE >= 1888)
    \end{lstlisting}

    \pause
    \begin{lstlisting}
CREATE TABLE MOVIE (
  TITLE VARCHAR(80) NOT NULL,
  YEAR YEARS,
  ...
)
    \end{lstlisting}
  \end{example}
\end{frame}

\begin{frame}[fragile]
  \frametitle{Constraint Management}

  \begin{block}{Statement}
    \begin{lstlisting}
ALTER TABLE table_name
  ADD [ CONSTRAINT constraint_name ]
    constraint_definition
    \end{lstlisting}
  \end{block}

  \pause
  \begin{itemize}
    \item what about existing tuples?
  \end{itemize}

  \pause
  \begin{block}{Removing Constraints}
    \begin{lstlisting}
ALTER TABLE table_name
  DROP [ CONSTRAINT ] constraint_name
    \end{lstlisting}
  \end{block}
\end{frame}

\begin{frame}[fragile]
  \frametitle{Constraint Management Examples}

  \begin{example}[SCORE values can not be greater than 10]
    \begin{lstlisting}
ALTER TABLE MOVIE
  ADD CHECK (SCORE <= 10)
    \end{lstlisting}
  \end{example}
\end{frame}

\begin{frame}[fragile]
  \frametitle{Constraint Management Examples}

  \begin{example}[SCORE values must be between 1 and 10]
    \begin{lstlisting}
ALTER TABLE MOVIE
  ADD CONSTRAINT SCORE_RANGE
      CHECK ((SCORE >= 1) AND (SCORE <= 10))
    \end{lstlisting}
  \end{example}

  \pause
  \begin{example}[SCORE values should not be checked for range]
    \begin{lstlisting}
ALTER TABLE MOVIE
  DROP CONSTRAINT SCORE_RANGE
    \end{lstlisting}
  \end{example}
\end{frame}

\begin{frame}[fragile]
  \frametitle{Creating the Example Tables}

  \begin{example}[creating the MOVIE table]
    \begin{lstlisting}
CREATE TABLE MOVIE (
  ID INTEGER PRIMARY KEY,
  TITLE VARCHAR(80) NOT NULL,
  YEAR NUMERIC(4),
  COUNTRY CHAR(2),
  SCORE FLOAT,
  VOTES INTEGER DEFAULT 0,
  DIRECTORID INTEGER
)
    \end{lstlisting}
  \end{example}
\end{frame}

\begin{frame}[fragile]
  \frametitle{Creating the Example Tables}

  \begin{example}[creating the PERSON table]
    \begin{lstlisting}
CREATE TABLE PERSON (
  ID INTEGER PRIMARY KEY,
  NAME VARCHAR(40) UNIQUE NOT NULL
)
    \end{lstlisting}
  \end{example}
\end{frame}

\begin{frame}[fragile]
  \frametitle{Creating the Example Tables}

  \begin{example}[creating the CASTING table]
    \begin{lstlisting}
CREATE TABLE CASTING (
  MOVIEID INTEGER,
  ACTORID INTEGER,
  ORD INTEGER,
  PRIMARY KEY (MOVIEID, ACTORID)
)
    \end{lstlisting}
  \end{example}
\end{frame}

\begin{frame}[fragile]
  \frametitle{Foreign Key Definition}

  \begin{block}{Statement}
    \begin{lstlisting}
CREATE TABLE table_name (
  ...
  [ { FOREIGN KEY ( column_name [, ...] )
      REFERENCES table_name
          [ ( column_name [, ...] ) ]
      [ ON DELETE option ]
      [ ON UPDATE option ] } [, ...] ]
  ...
)
    \end{lstlisting}
  \end{block}
\end{frame}

\begin{frame}[fragile]
  \frametitle{Foreign Key Definition Example}

  \begin{example}[creating the MOVIE table]
    \begin{lstlisting}
CREATE TABLE MOVIE (
  ID INTEGER PRIMARY KEY,
  TITLE VARCHAR(80) NOT NULL,
  ...
  DIRECTORID INTEGER,
  FOREIGN KEY DIRECTORID REFERENCES PERSON (ID)
    ON DELETE RESTRICT
    ON UPDATE CASCADE
)
    \end{lstlisting}
  \end{example}
\end{frame}

\begin{frame}[fragile]
  \frametitle{Defining Foreign Keys}

  \begin{itemize}
    \item if the foreign key is the primary key in the referenced table\\
      it does not have to be specified in the \lstinline!REFERENCES! part

    \pause
    \item if the foreign key consists of only one attribute,\\
      it can be specified at column definition:\\
      \lstinline!column_name data_type!\\
      ~~~~\lstinline!REFERENCES table_name [ ( column_name ) ]!
  \end{itemize}
\end{frame}

\begin{frame}[fragile]
  \frametitle{Creating the Example Tables}

  \begin{example}[creating the MOVIE table]
    \begin{lstlisting}
CREATE TABLE MOVIE (
  ID INTEGER PRIMARY KEY,
  TITLE VARCHAR(80) NOT NULL,
  YEAR NUMERIC(4),
  COUNTRY CHAR(2),
  SCORE FLOAT,
  VOTES INTEGER DEFAULT 0,
  DIRECTORID INTEGER REFERENCES PERSON
)
    \end{lstlisting}
  \end{example}
\end{frame}

\begin{frame}[fragile]
  \frametitle{Creating the Example Tables}

  \begin{example}[creating the PERSON table]
    \begin{lstlisting}
CREATE TABLE PERSON (
  ID INTEGER PRIMARY KEY,
  NAME VARCHAR(40) UNIQUE NOT NULL
)
    \end{lstlisting}
  \end{example}
\end{frame}

\begin{frame}[fragile]
  \frametitle{Creating the Example Tables}

  \begin{example}[creating the CASTING table]
    \begin{lstlisting}
CREATE TABLE CASTING (
  MOVIEID INTEGER REFERENCES MOVIE,
  ACTORID INTEGER REFERENCES PERSON,
  ORD INTEGER,
  PRIMARY KEY (MOVIEID, ACTORID)
)
    \end{lstlisting}
  \end{example}
\end{frame}

\begin{frame}
  \frametitle{Integrity Constraints}

  \begin{itemize}
    \item do not allow the operation
      (\lstinline!RESTRICT! | \lstinline!NO\_ACTION!)

    \pause
    \item reflect the change to affected tuples (\lstinline!CASCADE!)

    \pause
    \item assign null value (\lstinline!SET NULL!)
    \begin{itemize}
      \item if null values are allowed
    \end{itemize}

    \pause
    \item assign default value (\lstinline!SET DEFAULT!)
  \end{itemize}
\end{frame}

\begin{frame}[fragile]
  \frametitle{Referential Integrity Examples}

  \begin{example}[restrict when deleting]
    \begin{itemize}
      \item if the \texttt{ID} attribute of the director to be deleted\\
        from the \texttt{PERSON} relation is present among the current values\\
        of the \texttt{DIRECTORID} attribute of the \texttt{MOVIE} relation,\\
        do not allow deleting
    \end{itemize}
  \end{example}

  \pause
  \begin{example}[cascade when updating]
    \begin{itemize}
      \item when the \texttt{ID} attribute value of a director has been
        updated\\
        in the \texttt{PERSON} relation, also update the \texttt{DIRECTORID}
        attribute\\
        of the corresponding tuples in the \texttt{MOVIE} relation
    \end{itemize}
  \end{example}
\end{frame}

\begin{frame}[fragile]
  \frametitle{Referential Integrity Example}

  \begin{example}[cascade when deleting]
    \begin{itemize}
      \item delete a person from the \texttt{PERSON} relation

      \pause
      \item delete all tuples from the \texttt{CASTING} relation\\
        where the \texttt{ACTORID} attribute has the same value\\
        as the \texttt{ID} attribute of the deleted person

      \pause
      \item delete all tuples from the \texttt{MOVIE} relation\\
        where the \texttt{DIRECTORID} attribute has the same value\\
        as the \texttt{ID} attribute of the deleted person

      \pause
      \item for each deleted \texttt{MOVIE} tuple in the previous step,\\
        delete all tuples from the \texttt{CASTING} relation\\
        where the \texttt{MOVIEID} attribute has the same value\\
        as the \texttt{ID} attribute of the deleted movie

    \end{itemize}
  \end{example}
\end{frame}

\section*{References}

\begin{frame}
  \frametitle{References}

  \begin{block}{Required Reading: Date}
    \begin{itemize}
      \item Chapter 3: An Introduction to Relational Databases
      \begin{itemize}
        \item 3.2. \alert{An Informal Look at the Relational Model}
        \item 3.3. \alert{Relations and Relvars}
      \end{itemize}

      \item Chapter 6: \alert{Relations}

      \item Chapter 9: Integrity
      \begin{itemize}
        \item 9.10. \alert{Keys}
        \item 9.12. \alert{SQL Facilities}
      \end{itemize}
    \end{itemize}
  \end{block}
\end{frame}

\end{document}
