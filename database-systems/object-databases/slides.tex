% Copyright (c) 2005-2010
%       H. Turgut Uyar <uyar@itu.edu.tr>
%       Şule Gündüz Öğüdücü <sgunduz@itu.edu.tr>
%
% These notes are licensed using the
% "Creative Commons Attribution-NonCommercial-ShareAlike License".
% You are free to copy, distribute and transmit the work, and to adapt the work
% as long as you attribute the authors, do not use it for commercial purposes,
% and any derivative work is under the same or a similar license.
%
% Read the full legal code at:
% http://creativecommons.org/licenses/by-nc-sa/3.0/

\documentclass[dvipsnames]{beamer}

\usepackage{ae}
\usepackage[T1]{fontenc}
\usepackage[utf8]{inputenc}
\setbeamertemplate{navigation symbols}{}

\usepackage{listings}
\lstset{language=Java}

\mode<presentation> {
  \usetheme{Warsaw}
  \usecolortheme[named=ForestGreen]{structure}
  \setbeamercovered{transparent}
}

\title{Database Systems}
\subtitle{Object Databases}

\author{H. Turgut Uyar \and Şule Öğüdücü}
\date{2005-2010}

\AtBeginSubsection[]{
  \begin{frame}<beamer>
    \frametitle{Topics}
    \tableofcontents[currentsection,currentsubsection]
  \end{frame}
}

\theoremstyle{plain}

\pgfdeclareimage[width=2cm]{license}{../../license}

\pgfdeclareimage{object}{object}
\pgfdeclareimage{oid}{oid}

\begin{document}

\begin{frame}
  \titlepage
\end{frame}

\begin{frame}
  \frametitle{License}

  \pgfuseimage{license}\hfill
  \copyright 2005-2010 T. Uyar, Ş. Öğüdücü

  \vfill
  \begin{tiny}
    You are free:
    \begin{itemize}
      \item to Share — to copy, distribute and transmit the work
      \item to Remix — to adapt the work
    \end{itemize}

    Under the following conditions:
    \begin{itemize}
      \item Attribution — You must attribute the work in the manner specified by
        the author or licensor (but not in any way that suggests that they
        endorse you or your use of the work).

      \item Noncommercial — You may not use this work for commercial purposes.

      \item Share Alike — If you alter, transform, or build upon this work, you
        may distribute the resulting work only under the same or similar license
        to this one.
    \end{itemize}
  \end{tiny}

  \vfill
  Legal code (the full license):\\
  \url{http://creativecommons.org/licenses/by-nc-sa/3.0/}
\end{frame}

\begin{frame}
  \frametitle{Topics}
  \tableofcontents
\end{frame}

\section{Object Databases}

\subsection{Introduction}

\begin{frame}
  \frametitle{Object-Orientation}

  \begin{itemize}
    \item object oriented techniques are successful in programming
    \begin{itemize}
      \item could they be applied to data modelling?
    \end{itemize}

    \pause
    \bigskip
    \item mismatch between data modelling and software modelling
    \begin{itemize}
      \item data: relation, tuple, foreign key, \ldots
      \item software: object, method, \ldots
    \end{itemize}
  \end{itemize}
\end{frame}

\begin{frame}[fragile]
  \frametitle{Mismatch Example}

  \begin{example}[adding an actor to a movie]
    \begin{lstlisting}[language=SQL]
INSERT INTO MOVIE(ID,TITLE)
  VALUES(110,'Sleepy Hollow')

INSERT INTO PERSON(ID,NAME)
  VALUES(26,'Johnny Depp')

INSERT INTO CASTING(MOVIEID, ACTORID)
  VALUES(110,26)
    \end{lstlisting}
  \end{example}
\end{frame}

\begin{frame}[fragile]
  \frametitle{Mismatch Example}

  \begin{example}[adding an actor to a movie]
    \begin{lstlisting}
public class Movie {
  ...
  private Person director;
  private ArrayList<Person> cast;

  public void castActor(Person p) {
    this.cast.add(p);
  }
}
    \end{lstlisting}
  \end{example}
\end{frame}

\begin{frame}[fragile]
  \frametitle{Mismatch Example}

  \begin{example}[adding an actor to a movie]
    \begin{lstlisting}
Movie m = new Movie("Sleepy Hollow", ...);
Person p = new Person("Johnny Depp", ...);
m.castActor(p);
    \end{lstlisting}
  \end{example}
\end{frame}

\begin{frame}
  \frametitle{Persistence}

  \begin{itemize}
    \item \alert{persistence}:\\
      saving object state between different runs of the application

    \pause
    \medskip
    \item SQL is not the only way to achive persistence
    \begin{itemize}
      \item might even damage the software model
    \end{itemize}
  \end{itemize}
\end{frame}

\begin{frame}
  \frametitle{Nesneye Dayalı Veritabanları}

  \begin{itemize}
    \item persistent objects are stored in an object database
    \item use the host language to access these objects

    \pause
    \item single language for both application and data operations
    \begin{itemize}
      \item no need for data conversion
    \end{itemize}
  \end{itemize}
\end{frame}

\begin{frame}
  \frametitle{Object Ids}

  \begin{itemize}
    \item every object has an identifier
    \begin{itemize}
      \item identifiers don't change even if attributes change
    \end{itemize}

    \pause
    \item different from primary key
    \begin{itemize}
      \item primary key is visible (user-defined)
      \item value of primary key can change
    \end{itemize}

    \pause
    \item corresponds to references (pointers) in programming languages
    \begin{itemize}
      \item \emph{containment hierarchy}
    \end{itemize}
  \end{itemize}
\end{frame}

\begin{frame}
  \frametitle{Object Database Example}

  \begin{example}
    \begin{center}
      \pgfuseimage{object}
    \end{center}
  \end{example}
\end{frame}

\begin{frame}
  \frametitle{Object Database Example}

  \begin{example}
    \begin{center}
      \pgfuseimage{oid}
    \end{center}
  \end{example}
\end{frame}

\subsection{Sample System}

\begin{frame}
  \frametitle{DB4O}

  \begin{itemize}
    \item Java
    \item .NET
  \end{itemize}
\end{frame}

\begin{frame}[fragile]
  \frametitle{Database Connection}

  \begin{example}
    \begin{lstlisting}
ObjectContainer db = Db4o.openFile("dbfile");
try {
    // database operations
} catch (Exception e) {
    e.printStackTrace();
} finally {
    db.close();
}
    \end{lstlisting}
  \end{example}
\end{frame}

\begin{frame}[fragile]
  \frametitle{Example}

  \begin{example}
    \begin{lstlisting}
public class Movie {
  private String title;
  private float points;

  public Movie(String title, float points) {
    this.title = title;
    this.points = points;
  }

  // methods
}
    \end{lstlisting}
  \end{example}
\end{frame}

\begin{frame}[fragile]
  \frametitle{Example}

  \begin{example}
    \begin{lstlisting}
  public String getTitle() {
    return title;
  }

  public float getPoints() {
    return points;
  }

  public void addPoints(float points) {
    this.points += points;
  }
    \end{lstlisting}
  \end{example}
\end{frame}

\begin{frame}[fragile]
  \frametitle{Example}

  \begin{example}
    \begin{lstlisting}
Movie movie1 = new Movie("Casablanca", 5.2f);
db.set(movie1);
    \end{lstlisting}
  \end{example}
\end{frame}

\begin{frame}
  \frametitle{Queries}

  \begin{itemize}
    \item query by example
    \begin{itemize}
      \item create an example object of the same class
      \item select all objects with the same attribute values as the example
    \end{itemize}

    \medskip
    \item natural query
  \end{itemize}
\end{frame}

\begin{frame}[fragile]
  \frametitle{Query by Example}

  \begin{example}[all movies]
    \begin{lstlisting}
Movie proto = new Movie(null, 0f);
ObjectSet result = db.get(proto);

System.out.println(result.size());
while (result.hasNext()) {
  System.out.println(result.next());
}
    \end{lstlisting}
  \end{example}
\end{frame}

\begin{frame}[fragile]
  \frametitle{Update Example}

  \begin{example}[Casablanca filminin puanını güncelle]
    \begin{lstlisting}
Movie casablanca = new Movie("Casablanca", 0f);
ObjectSet result = db.get(casablanca);
Movie found = (Movie) result.next();
found.addPoints(1.1f);
db.set(found);
    \end{lstlisting}
  \end{example}
\end{frame}

\begin{frame}[fragile]
  \frametitle{Delete Example}

  \begin{example}[delete "Casablanca"]
    \begin{lstlisting}
Movie casablanca = new Movie("Casablanca", 0f);
ObjectSet result = db.get(casablanca);
Movie found = (Movie) result.next();
db.delete(found);
    \end{lstlisting}
  \end{example}
\end{frame}

\begin{frame}
  \frametitle{Problems}

  \begin{itemize}
    \item ad-hoc queries
    \item integrity
    \item referential integrity
    \item performance
  \end{itemize}
\end{frame}

\section{O/R Mapping}

\subsection{Introduction}

\begin{frame}
  \frametitle{Object/Relational Mapping}

  \begin{itemize}
    \item database employs relational model
    \item application is written in an object-oriented language

    \pause
    \item relation = class?
    \item domain = class?
  \end{itemize}
\end{frame}

\subsection{Sample System}

\begin{frame}
  \frametitle{BeanKeeper}

  \begin{itemize}
    \item transparent persistence
    \item connection to databases via JDBC
  \end{itemize}
\end{frame}

\begin{frame}
  \frametitle{Data Manipulation}

  \begin{itemize}
    \item database operations: \lstinline!Store! objects
    \item parameters: driver and JDBC URL
    \begin{itemize}
      \item \lstinline!save()!
      \item \lstinline!remove()!
      \item \lstinline!find()!
    \end{itemize}
  \end{itemize}
\end{frame}

\begin{frame}[fragile]
  \frametitle{Example}

  \begin{example}
    \begin{lstlisting}
public class Person {
  private String name;

  public String getName() {
    return name;
  }

  public void setName(String newName) {
    this.name = newName;
  }
}
    \end{lstlisting}
  \end{example}
\end{frame}

\begin{frame}[fragile]
  \frametitle{Example}

  \begin{example}
    \begin{lstlisting}
Store myStore = new Store("org.postgres.Driver",
                          "jdbc:postgres:mydb");

Public person = new Person();
person.setName(name);
myStore.save(person);
    \end{lstlisting}
  \end{example}
\end{frame}

\begin{frame}[fragile]
  \frametitle{Example}

  \begin{example}[all persons]
    \begin{lstlisting}
List persons = myStore.find("find person");
    \end{lstlisting}
  \end{example}
\end{frame}

\begin{frame}[fragile]
  \frametitle{SQL Queries}

  \begin{example}
    \begin{lstlisting}
Statement stmt = con.createStatement();
ResultSet rs = stmt.executeQuery(
    "select name from person"
    );
while (rs.next()) {
    name = rs.getString(1);
    System.out.println(name);
}
    \end{lstlisting}
  \end{example}
\end{frame}

% TODO: JPA

\section*{References}

\begin{frame}
  \frametitle{References}

  \begin{block}{Required text: Date}
    \begin{itemize}
      \item Chapter 25: \alert{Object Databases}
      \item Chapter 26: \alert{Object/Relational Databases}
    \end{itemize}
  \end{block}
\end{frame}

\end{document}
