% Copyright (c) 2005-2015
%       H. Turgut Uyar <uyar@itu.edu.tr>
%       Şule Gündüz Öğüdücü <sgunduz@itu.edu.tr>
%
% This work is licensed under a "Creative Commons
% Attribution-NonCommercial-ShareAlike 4.0 International License".
% For more information, please visit:
% https://creativecommons.org/licenses/by-nc-sa/4.0/

\documentclass[dvipsnames]{beamer}

\usepackage{ae}
\usepackage[scaled=0.88]{beramono}
\usepackage[T1]{fontenc}
\usepackage[utf8]{inputenc}
\setbeamersize{text margin left=2em, text margin right=2em}

\usepackage{listings}
\lstset{basicstyle=\ttfamily, keywordstyle=\color{ForestGreen},
        showstringspaces=false}

\mode<presentation> {
  \usetheme{Warsaw}
  \usecolortheme[named=ForestGreen]{structure}
  \setbeamercovered{transparent}
}

\title{Database Systems}
\subtitle{Non-Relational Databases}

\author{H. Turgut Uyar \and Şule Öğüdücü}
\date{2005-2015}

\AtBeginSubsection[]{
  \begin{frame}<beamer>
    \frametitle{Topics}
    \tableofcontents[currentsection,currentsubsection]
  \end{frame}
}

\theoremstyle{plain}

\pgfdeclareimage[width=2cm]{license}{../license}

\pgfdeclareimage[width=6.8cm]{casino-royale}{casino-royale}
\pgfdeclareimage[width=11cm]{sharding}{sharding}
\pgfdeclareimage[width=11cm]{cap}{cap}
\pgfdeclareimage[width=11cm]{key-value}{key-value}
\pgfdeclareimage[width=11cm]{image-store}{image-store}
\pgfdeclareimage[height=2cm]{column-family-store1}{column-family-store1}
\pgfdeclareimage[width=11cm]{column-family-store2}{column-family-store2}

\begin{document}

\begin{frame}
  \titlepage
\end{frame}

\begin{frame}
  \frametitle{License}

  \pgfuseimage{license}\hfill
  \copyright~2005-2015 T. Uyar, Ş. Öğüdücü

  \vfill
  \begin{footnotesize}
    You are free to:
    \begin{itemize}
      \itemsep0em
      \item Share -- copy and redistribute the material in any medium or format
      \item Adapt -- remix, transform, and build upon the material
    \end{itemize}

    Under the following terms:
    \begin{itemize}
      \itemsep0em
      \item Attribution -- You must give appropriate credit, provide a link to
        the license, and indicate if changes were made.

      \item NonCommercial -- You may not use the material for commercial
        purposes.

      \item ShareAlike -- If you remix, transform, or build upon the material,
        you must distribute your contributions under the same license as the
        original.
    \end{itemize}
  \end{footnotesize}

  \begin{small}
    For more information:\\
    \url{https://creativecommons.org/licenses/by-nc-sa/4.0/}

    \smallskip
    Read the full license:\\
    \url{https://creativecommons.org/licenses/by-nc-sa/4.0/legalcode}
  \end{small}
\end{frame}

\begin{frame}
  \frametitle{Topics}
  \tableofcontents
\end{frame}

\section{Non-Relational Databases}

\subsection{Introduction}

\begin{frame}
  \frametitle{Relational Model}

  \begin{itemize}
    \item relational model is not the best solution for all types of problems

    \bigskip
    \item storing user preferences
    \item processing data from Wikipedia pages
    \item building a social network
  \end{itemize}
\end{frame}

\begin{frame}
  \frametitle{Example: User Preferences}

  \begin{itemize}
    \item user, preference type, selected option

    \medskip
    \item example task:\\
      retrieve notification setting of a given user

    \pause
    \medskip
    \item no complex queries that would require SQL
  \end{itemize}
\end{frame}

\begin{frame}
  \frametitle{Example: Wikipedia Pages}

  \begin{columns}
    \column{.55\textwidth}
    \pgfuseimage{casino-royale}

    \column{.45\textwidth}
    \begin{itemize}
      \item combination of structured\\
        and unstructured data

      \medskip
      \item example task:\\
        retrieve first paragraph\\
        of all James Bond movies\\
        starring Daniel Craig

      \pause
      \medskip
      \item difficult to represent\\
        as a relation
    \end{itemize}
  \end{columns}
\end{frame}

\begin{frame}
  \frametitle{Example: Social Network}

  \begin{itemize}
    \item users: userid, name, age, gender, \ldots
    \item friends: userid1, userid2

    \medskip
    \item example tasks:\\
      find all friends of a given user\\
      find all friends of friends of a given user\\
      find all female friends of male friends of a given user\\
      find all friends of friends of ... friends of a given user\\

    \pause
    \medskip
    \item too many complicated joins
  \end{itemize}
\end{frame}

\begin{frame}
  \frametitle{Problems: Representation}

  \begin{itemize}
    \item difficult to handle unstructured and semistructured data
    \item difficult to represent hierarchy and neighborhood

    \pause
    \medskip
    \item rigid schemas: all rows need to store all fields
    \item even if not applicable
    \smallskip
    \item fixed in advance
    \item to make changes: shut down, alter table, restart
  \end{itemize}
\end{frame}

\begin{frame}
  \frametitle{Problems: Scaling}

  \begin{itemize}
    \item when volume of data increases:

    \bigskip
    \item scale up: faster processor
    \item works up to a point

    \pause
    \medskip
    \item scale out: more processors
    \item commodity hardware
  \end{itemize}
\end{frame}

\subsection{NoSQL}

\begin{frame}
  \frametitle{NoSQL Databases}

  \begin{itemize}
    \item NoSQL $\neq$ ``don't use SQL''
    \item Not Only SQL
    \item use relational for some parts and non-relational for other parts

    \pause
    \medskip
    \item key-value stores
    \item column family stores
    \item document stores
    \item graph databases
  \end{itemize}
\end{frame}

\begin{frame}
  \frametitle{NoSQL Principles}

  \begin{itemize}
    \item flexible schema

    \pause
    \medskip
    \item focus on performance
    \item no joins

    \pause
    \medskip
    \item massive scalability

    \pause
    \medskip
    \item focus on availability
    \item updates should always be allowed
  \end{itemize}
\end{frame}

\begin{frame}
  \frametitle{Availability vs Consistency}

  \begin{itemize}
    \item focus on availability $\rightarrow$ relaxed consistency
    \item fewer transactional guarantees

    \bigskip
    \item \alert{BASE} instead of ACID:
    \smallskip
    \item Basic availability
    \item Soft state
    \item Eventual consistency
  \end{itemize}
\end{frame}

\begin{frame}
  \frametitle{Sharding}

  \begin{itemize}
    \item when a server nears full capacity for data
    \item \alert{sharding}: break data into chunks
    \item spread chunks across distributed servers

    \medskip
    \item increases efficiency
    \item more servers $\rightarrow$ more points of failure
  \end{itemize}
\end{frame}

\begin{frame}
  \frametitle{Replication}

  \begin{itemize}
    \item replicate data between servers
    \item increases fault tolerance

    \medskip
    \item copies might diverge
    \item \alert{eventual consistency}: temporary inconsistency is allowed
    \item when system stops, all copies will be the same
  \end{itemize}
\end{frame}

\begin{frame}
  \frametitle{CAP Properties}

  \begin{itemize}
    \item \alert{C}onsistency:\\
      all clients can read a single, up-to-date version of data\\
      from replicated partitions

    \medskip
    \item \alert{A}vailability:\\
      internal communication failures between replicated data\\
      don't prevent updates

    \medskip
    \item \alert{P}artition tolerance:\\
      system keeps responding even if there is a communication failure\\
      between partitions
  \end{itemize}
\end{frame}

\begin{frame}
  \frametitle{CAP Theorem}

  \begin{itemize}
    \item Any distributed database can provide\\
      at most two of the three CAP properties.\\
      (Eric Brewer - 2000)
  \end{itemize}
\end{frame}

\begin{frame}
  \frametitle{Query Language}

  \begin{itemize}
    \item no declarative query language
    \item programmatic handling of data
  \end{itemize}
\end{frame}

\subsection{Serialization}

\begin{frame}
  \frametitle{Serialization}

  \begin{itemize}
    \item how to make objects persist?
    \item simple method: write to file
    \item in which format?
    \item simple format: string

    \pause
    \medskip
    \item on write: object $\rightarrow$ serial format (\alert{serialization})
    \item on read: serial format $\rightarrow$ object (deserialization)
  \end{itemize}
\end{frame}

\begin{frame}
  \frametitle{Serialization Formats}

  \begin{itemize}
    \item common formats: XML, JSON

    \smallskip
    \item human-readable
    \item useful for data interchange
    \item useful for representing semistructured data
  \end{itemize}
\end{frame}

\lstset{language=XML}

\begin{frame}
  \frametitle{XML}

  \begin{itemize}
    \item XML is not a language itself
    \item framework for defining languages

    \medskip
    \item XML-based languages:\\
      XHTML, DocBook, SVG, \ldots

    \smallskip
    \item XML processing languages:\\
      XPath, XQuery, XSL Transforms, \ldots
  \end{itemize}
\end{frame}

\begin{frame}
  \frametitle{XML Structure}

  \begin{itemize}
    \item an XML document forms a tree (hierarchy)

    \medskip
    \item nodes: \emph{elements}
    \item elements represented by opening and closing tags
    \item nesting determines hierarchy
    \item non-container elements: self-closing tags

    \medskip
    \item elements can have attributes
    \item elements can have text as child node: character data (CDATA)
  \end{itemize}
\end{frame}

\begin{frame}[fragile]
  \frametitle{XML Example: XHTML}

  \begin{lstlisting}
<html>
  <head>
    <title>...</title>
    <meta charset="utf-8" />
  </head>
  <body>
    <h1>...</h1>
    <p>...</p>
    <img src="..." alt="..." />
  </body>
</html>
  \end{lstlisting}
\end{frame}

\begin{frame}[fragile]
  \frametitle{XML Example: Movie}

  \begin{lstlisting}
<movie color="Color">
  <title>The Usual Suspects</title>
  <year>1995</year>
  <score>8.7</score>
  <votes>35027</votes>
  <director>Bryan Singer</director>
  <cast>
    <actor>Gabriel Byrne</actor>
    <actor>Benicio Del Toro</actor>
  </cast>
</movie>
  \end{lstlisting}
\end{frame}

\begin{frame}[fragile]
  \frametitle{XML Example: Movies}

  \begin{lstlisting}
<movies>
  <movie color="Color">
    <title>The Usual Suspects</title>
    <year>1995</year>
    ...
  </movie>
  <movie color="Color">
    <title>Being John Malkovich</title>
    <year>1999</year>
    ...
  </movie>
  ...
</movies>
  \end{lstlisting}
\end{frame}

\begin{frame}
  \frametitle{Well-Formed Documents}

  \begin{itemize}
    \item \alert{well-formed}: conforming to XML rules
    \item syntactically correct

    \medskip
    \item single root element
    \item proper nesting of elements: matched tags
    \item unique attributes within elements

    \pause
    \medskip
    \item XML parsers convert well-formed XML documents into\\
      \alert{DOM} objects (Document Object Model)
  \end{itemize}
\end{frame}

\begin{frame}
  \frametitle{Valid Documents}

  \begin{itemize}
    \item \alert{valid}: conforming to domain rules
    \item semantically correct

    \medskip
    \item DTD, XML Schema

    \pause
    \medskip
    \item validating XML parsers also check for validity
  \end{itemize}
\end{frame}

\lstset{language=Python}

\begin{frame}
  \frametitle{JSON}

  \begin{itemize}
    \item JavaScript Object Notation

    \medskip
    \item base values: number, string, \ldots
    \item objects: sets of key-value pairs
    \item arrays of values

    \medskip
    \item nested structure
  \end{itemize}
\end{frame}

\begin{frame}[fragile]
  \frametitle{JSON Example}

  \begin{lstlisting}
{
  "title": "The Usual Suspects",
  "year": 1995,
  "score": 8.7,
  "votes": 35027,
  "director": "Bryan Singer",
  "cast": [
    "Gabriel Byrne",
    "Benicio Del Toro"
  ]
}
  \end{lstlisting}
\end{frame}

\begin{frame}[fragile]
  \frametitle{JSON Example}

  \begin{lstlisting}
[
  {
    "title": "The Usual Suspects",
    "year": 1995,
    ...
  },
  ...
  {
    "title": "Being John Malkovich",
    "year": 1999,
    ...
  }
]
  \end{lstlisting}
\end{frame}

\begin{frame}
  \frametitle{Valid Documents}

  \begin{itemize}
    \item JSON Schema
  \end{itemize}
\end{frame}

\section{Data Models}

\subsection{Key-Value Stores}

\begin{frame}
  \frametitle{Key-Value Stores}

  \begin{itemize}
    \item model: (key, value) pairs
    \item indexed by keys
    \item keys are \alert{distinct}
    \item value is an arbitrary large blob of data

    \pause
    \medskip
    \item very simple interface: put, get, delete
    \item no queries on values

    \pause
    \bigskip
    \item products: Redis, Riak, Memcache, Amazon DynamoDB
  \end{itemize}
\end{frame}

\begin{frame}
  \frametitle{Key-Value Store Examples}

  \begin{itemize}
    \item web page caching
    \item key: URL, value: web page

    \pause
    \medskip
    \item image store
    \item key: path to image, value: image
  \end{itemize}
\end{frame}

\begin{frame}
  \frametitle{Key-Value Stores}

  \begin{itemize}
    \item distribute records to computing nodes based on key

    \medskip
    \item advanced: data structures in value
    \item not just a blob of data
  \end{itemize}
\end{frame}

\begin{frame}
  \frametitle{Column Family Stores}

  \begin{itemize}
    \item key is a (row, column) pair
    \item sparse matrix

    \pause
    \medskip
    \item advanced keys: (row, column family, column, timestamp)
    \item column family: groups of columns
    \item timestamp: store multiple values over time
  \end{itemize}

  \bigskip
  \begin{itemize}
    \item products: Apache Cassandra, Apache HBase, Google BigTable
  \end{itemize}
\end{frame}

\begin{frame}
  \frametitle{Column Family Store Example}

  \begin{itemize}
    \item user preferences
    \item privacy settings, contact information, notifications, \ldots
    \item typically under 100 fields, 1 KB

    \medskip
    \item only the associated user makes changes: no ACID requirements
    \item mostly read
    \item has to be fast and scalable
  \end{itemize}
\end{frame}

\subsection{Document Stores}

\begin{frame}
  \frametitle{Document Stores}

  \begin{itemize}
    \item model: (key, document) pairs
    \item document: JSON formatted data

    \medskip
    \item query based on document contents
    \item documents automatically indexed
    \item documents grouped into collections: hierarchical structure

    \bigskip
    \item products: MongoDB, CouchDB
    \item application example: content management systems
  \end{itemize}
\end{frame}

\begin{frame}[fragile]
  \frametitle{MongoDB Insert Example}

  \begin{lstlisting}
itucsdb.movies.insert(
  {
    "title": "Ed Wood",
    "year": 1994,
    "score": 7.8,
    "votes": 6587,
    "director": "Tim Burton",
    "cast": [
      "Johnny Depp"
    ]
  }
)
  \end{lstlisting}
\end{frame}

\begin{frame}[fragile]
  \frametitle{MongoDB Insert Example}

  \begin{lstlisting}
itucsdb.movies.insert(
  {
    "title": "Three Kings",
    "year": 1999,
    "score": 7.7,
    "votes": 10319,
    "cast": [
      "George Clooney",
      "Spike Jonze"
    ]
  }
)
  \end{lstlisting}
\end{frame}

\begin{frame}[fragile]
  \frametitle{MongoDB Find Example}

  \begin{lstlisting}
itucsdb.movies.find()

itucsdb.movies.find(
  {"year": 1999}
)

itucsdb.movies.find(
  {"year": {$gt 1999}}
)
  \end{lstlisting}
\end{frame}

\subsection{XML Databases}

\begin{frame}
  \frametitle{XML Databases}

  \begin{itemize}
    \item variant of document stores
    \item document: XML formatted data

    \medskip
    \item query using XPath

    \bigskip
    \item products: Oracle Berkeley DBXML, BaseX, eXist
  \end{itemize}
\end{frame}

\begin{frame}
  \frametitle{XPath}

  \begin{itemize}
    \item XPath: selecting nodes and data from XML documents

    \medskip
    \item path of nodes to find: chain of location steps
    \item starting from the root (absolute)
    \item starting from the current node (relative)
  \end{itemize}
\end{frame}

\begin{frame}
  \frametitle{XPath Examples}

  \begin{itemize}
    \item all movies: \lstinline|/movies/movie|
    \item actors of current movie: \lstinline|./cast/actor|
    \item \lstinline|../../year|
  \end{itemize}
\end{frame}

\begin{frame}
  \frametitle{Location Steps}

  \begin{itemize}
    \item location step structure:\\
      \lstinline|axis::node_selector[predicate]|

    \medskip
    \item axis: where to search
    \item selector: what to search
    \item predicate: under which conditions
  \end{itemize}
\end{frame}

\begin{frame}
  \frametitle{Axes}

  \begin{itemize}
    \item \lstinline|child|:
      all children, one level (default axis)
    \item \lstinline|descendant|:
      all children, recursively (shorthand: \lstinline|//|)
    \item \lstinline|parent|:
      parent node, one level
    \item \lstinline|ancestor|:
      parent nodes, up to document element
    \item \lstinline|attribute|:
      attributes (shorthand: \lstinline|@|)
    \item \lstinline|following-sibling|:
      siblings that come later
    \item \lstinline|preceding-sibling|:
      siblings that come earlier
    \item \ldots
  \end{itemize}
\end{frame}

\begin{frame}
  \frametitle{Node Selectors}

  \begin{itemize}
    \item node tag
    \item node attribute
    \item node text: \lstinline|text()|
    \item all children: \lstinline|*|
  \end{itemize}
\end{frame}

\begin{frame}
  \frametitle{XPath Examples}

  \begin{itemize}
    \item names of all directors:\\
      \lstinline|/movies/movie/director/text()|\\
      \lstinline|//director/text()|

    \pause
    \item all actors in this movie:\\
      \lstinline|./cast/actor|\\
      \lstinline|.//actor|

    \pause
    \item colors of all movies:\\
      \lstinline|//movie/@color|

    \pause
    \item scores of movies after this one:\\
      \lstinline|./following-sibling::movie/score|
  \end{itemize}
\end{frame}

\begin{frame}
  \frametitle{XPath Predicates}

  \begin{itemize}
    \item testing node position: \lstinline|[position]|

    \pause
    \medskip
    \item testing existence of a child: \lstinline|[child_tag]|
    \item testing value of a child: \lstinline|[child_tag="value"]|

    \pause
    \medskip
    \item testing existence of an attribute: \lstinline|[@attribute]|
    \item testing value of an attribute: \lstinline|[@attribute="value"]|
  \end{itemize}
\end{frame}

\begin{frame}
  \frametitle{XPath Examples}

  \begin{itemize}
    \item title of the first movie:\\
      \lstinline|/movies/movie[1]/title|

    \pause
    \item all movies in the year 1997:\\
      \lstinline|movie[year="1997"]|

    \pause
    \item black-and-white movies:\\
      \lstinline|movie[@color="BW"]|
  \end{itemize}
\end{frame}

\subsection{Graph Databases}

\begin{frame}
  \frametitle{Graph Databases}

  \begin{itemize}
    \item model: nodes and edges
    \item nodes have properties
    \item edges have labels

    \medskip
    \item for relationship intensive data: social networks, \ldots
    \item traversals instead of joins

    \bigskip
    \item products: Neo4J
  \end{itemize}
\end{frame}

\begin{frame}
  \frametitle{Graph Databases}

  \begin{itemize}
    \item better suited for tasks like:\\
      shortest path, friends of friends,\\
      neighboring nodes with specific properties

    \pause
    \medskip
    \item difficult to scale out

    \pause
    \bigskip
    \item declarative query languages: Cypher, Gremlin
  \end{itemize}
\end{frame}

\begin{frame}
  \frametitle{Cypher}

  \begin{itemize}
    \item locate the initial nodes
    \item select and traverse relationships
    \item change and/or return values
  \end{itemize}
\end{frame}

\begin{frame}[fragile]
  \frametitle{Cypher: Nodes}

  \begin{itemize}
    \item \lstinline|(name)|
    \item \lstinline|(name:Type)|
    \item \lstinline|(name:Type {attributes})|
  \end{itemize}

  \begin{lstlisting}
(matrix)
(matrix:Movie)
(matrix:Movie {title: "The Matrix"})
(matrix:Movie {title: "The Matrix", released: 1997})
  \end{lstlisting}
\end{frame}

\begin{frame}[fragile]
  \frametitle{Cypher: Relationships}

  \begin{itemize}
    \item undirected: \lstinline|--|
    \item directed: \lstinline|-->| \lstinline|<--|
    \item with details: \lstinline|-[]-|
  \end{itemize}

  \begin{lstlisting}
-[role]->
-[role:ACTED_IN]->
-[role:ACTED_IN {roles: ["Neo"]}]->
  \end{lstlisting}
\end{frame}

\begin{frame}[fragile]
  \frametitle{Cypher: Patterns}

  \begin{itemize}
    \item combine nodes and relationships
    \item give names to patterns
  \end{itemize}

  \begin{lstlisting}
(keanu:Person   {name:  "Keanu Reeves"} )
-[role:ACTED_IN {roles: ["Neo"] } ]->
(matrix:Movie   {title: "The Matrix"} )

acted_in = (:Person)-[:ACTED_IN]->(:Movie)
  \end{lstlisting}
\end{frame}

\begin{frame}[fragile]
  \frametitle{Cypher: Creating Data}

  \begin{lstlisting}
CREATE (:Movie {title: "The Matrix", released: 1997})

CREATE (p:Person {name: "Keanu Reeves", born: 1964})
RETURN p

CREATE (a:Person {name: "Tom Hanks", born:1956 })
  -[r:ACTED_IN {roles: ["Forrest"]}]->
  (m:Movie {title: "Forrest Gump", released: 1994})
CREATE (d:Person {name: "Robert Zemeckis", born: 1951})
  -[:DIRECTED]-> (m)
RETURN a, d, r, m
  \end{lstlisting}
\end{frame}

\begin{frame}[fragile]
  \frametitle{Cypher: Matching Patterns}

  \begin{lstlisting}
MATCH (m:Movie)
RETURN m

MATCH (p:Person {name:"Keanu Reeves"})
RETURN p

MATCH (p:Person {name:"Tom Hanks"})
  -[r:ACTED_IN]-> (m:Movie)
RETURN m.title, r.roles
  \end{lstlisting}
\end{frame}

\section*{References}

\begin{frame}
  \frametitle{References}

  \begin{block}{Supplementary Reading}
    \begin{itemize}
      \item Making Sense of NoSQL, by Dan McCreary and Ann Kelly,\\
        Manning Publications
      \item The Neo4J Manual: Tutorials\\
        \url{http://neo4j.com/docs/stable/tutorials.html}
    \end{itemize}
  \end{block}
\end{frame}

\end{document}
