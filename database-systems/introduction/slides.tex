% Copyright (c) 2002-2016
%       H. Turgut Uyar <uyar@itu.edu.tr>
%       Şule Gündüz Öğüdücü <sgunduz@itu.edu.tr>
%
% This work is licensed under a "Creative Commons
% Attribution-NonCommercial-ShareAlike 4.0 International License".
% For more information, please visit:
% https://creativecommons.org/licenses/by-nc-sa/4.0/

\documentclass[dvipsnames]{beamer}

\usepackage{ae}
\usepackage[scaled=0.88]{beramono}
\usepackage[T1]{fontenc}
\usepackage[utf8]{inputenc}
\setbeamersize{text margin left=2em, text margin right=2em}

\mode<presentation>
{
  \usetheme{Warsaw}
  \usecolortheme[named=ForestGreen]{structure}
  \setbeamercovered{transparent}
}

\title{Database Systems}
\subtitle{Introduction}

\author{H. Turgut Uyar \and Şule Öğüdücü}
\date{2002-2016}

\AtBeginSubsection[]{
  \begin{frame}<beamer>
    \frametitle{Topics}
    \tableofcontents[currentsection,currentsubsection]
  \end{frame}
}

\theoremstyle{plain}

\pgfdeclareimage[width=2cm]{license}{../license}

\pgfdeclareimage[width=5.5cm]{records}{records}
\pgfdeclareimage[width=6cm]{dbms}{dbms}
\pgfdeclareimage[height=6.5cm]{sparc}{sparc}
\pgfdeclareimage[height=6cm]{architecture1}{architecture1}
\pgfdeclareimage[height=5cm]{architecture2}{architecture2}

\begin{document}

\begin{frame}
  \titlepage
\end{frame}

\begin{frame}
  \frametitle{License}

  \pgfuseimage{license}\hfill
  \copyright~2002-2016 T. Uyar, Ş. Öğüdücü

  \vfill
  \begin{footnotesize}
    You are free to:
    \begin{itemize}
      \itemsep0em
      \item Share -- copy and redistribute the material in any medium or format
      \item Adapt -- remix, transform, and build upon the material
    \end{itemize}

    Under the following terms:
    \begin{itemize}
      \itemsep0em
      \item Attribution -- You must give appropriate credit, provide a link to
        the license, and indicate if changes were made.

      \item NonCommercial -- You may not use the material for commercial
        purposes.

      \item ShareAlike -- If you remix, transform, or build upon the material,
        you must distribute your contributions under the same license as the
        original.
    \end{itemize}
  \end{footnotesize}

  \begin{small}
    For more information:\\
    \url{https://creativecommons.org/licenses/by-nc-sa/4.0/}

    \smallskip
    Read the full license:\\
    \url{https://creativecommons.org/licenses/by-nc-sa/4.0/legalcode}
  \end{small}
\end{frame}

\begin{frame}
  \frametitle{Topics}
  \tableofcontents
\end{frame}

\section{Introduction}

\subsection{Problem}

\begin{frame}
  \frametitle{Problem}

  \begin{itemize}
    \item store and process large amounts of data effectively

    \medskip
    \item add new data
    \item change existing data
    \item delete data
    \item query data: planned - ad hoc

    \pause
    \smallskip
    \item \alert{CRUD}: create - read - update - delete
  \end{itemize}
\end{frame}

\begin{frame}
  \frametitle{Data}

  \begin{itemize}
    \item \alert{persistent} data: data that must be stored\\
      due to the nature of the information

    \bigskip
    \item temporary data
    \smallskip
    \item output data: data that can be derived from persistent data\\
      (query results, reports, etc.)
    \item input data: unprocessed data that just entered the system
  \end{itemize}
\end{frame}

\begin{frame}
  \frametitle{Example: University student data}

  \begin{columns}[t]
    \begin{column}<1->{.5\textwidth}
    \begin{itemize}
      \item Student Affairs:\\
	student name, number,
	department, courses taken,\\
	internships, \ldots
    \end{itemize}
    \end{column}

    \begin{column}<2->{.5\textwidth}
    \begin{itemize}
      \item common data:\\
	student name, number,\\
	department, \ldots
    \end{itemize}
    \end{column}
  \end{columns}

  \begin{columns}[t]
    \begin{column}<1->{.5\textwidth}
      \begin{itemize}
	\item Library:\\
	  student name, number,\\
	  department, books lent, \ldots
      \end{itemize}
    \end{column}

    \begin{column}<2->{.5\textwidth}
      \begin{itemize}
	\item application specific data:\\
	  courses, internships, books, \ldots
      \end{itemize}
    \end{column}
  \end{columns}
\end{frame}

\subsection{Record Files}

\begin{frame}
  \frametitle{Record Files}

  \begin{columns}[b]
    \column{.4\textwidth}
    \begin{center}
      \pgfuseimage{records}
    \end{center}

    \column{.6\textwidth}
    \begin{itemize}
      \item every application has its own data
      \item every application keeps its data\\
        in the files that it manages itself
    \end{itemize}
  \end{columns}
\end{frame}

\begin{frame}
  \frametitle{Redundancy}

  \begin{itemize}
    \item same data kept in multiple places
    \item waste of disk space
  \end{itemize}

  \medskip
  \begin{exampleblock}{example}
    \begin{itemize}
      \item names, numbers and departments of students are kept\\
        both in Student Affairs and in the Library
    \end{itemize}
  \end{exampleblock}
\end{frame}

\begin{frame}
  \frametitle{Inconsistency}

  \begin{itemize}
    \item multiple copies of the same data can become different
  \end{itemize}

  \medskip
  \begin{exampleblock}{example}
    \begin{itemize}
      \item the name of the same student can be recorded\\
        as ``Victoria Adams'' in Student Affairs and\\
        as ``Victoria Beckham'' in the Library
    \end{itemize}
  \end{exampleblock}
\end{frame}

\begin{frame}
  \frametitle{Integrity}

  \begin{itemize}
    \item it is difficult to keep the data correct
  \end{itemize}

  \medskip
  \begin{exampleblock}{example}
    \begin{itemize}
      \item ``Control and Computer Engineering'' department is closed\\
        but the department data of its students remains the same
    \end{itemize}
  \end{exampleblock}
\end{frame}

\begin{frame}
  \frametitle{Duplicated Work}

  \begin{itemize}
    \item a lot of work must be duplicated for every new application
  \end{itemize}

  \medskip
  \begin{exampleblock}{example}
    \begin{itemize}
      \item a new application will be developed for the Scholarship Office
    \end{itemize}
  \end{exampleblock}
\end{frame}

\begin{frame}
  \frametitle{Policy Gaps}

  \begin{itemize}
    \item no standards in the applications of the institution
    \item different paradigms, methods, programming languages
    \item data transfer between applications

    \pause
    \medskip
    \item each department considers only its own requirements
  \end{itemize}
\end{frame}

\begin{frame}
  \frametitle{Security}

  \begin{itemize}
    \item hard to define detailed security permissions
    \item security depends only on the operating system
  \end{itemize}
\end{frame}

\begin{frame}
  \frametitle{Data Dependence}

  \begin{itemize}
    \item \alert{data dependence}: application code depends\\
      on the organization of the data\\
      and the access method

    \medskip
    \item hard to make changes in the code
  \end{itemize}
\end{frame}

\begin{frame}
  \frametitle{Data Dependence Example}

  \begin{itemize}
    \item student number is a string in Student Affairs\\
      but a number in the Library

    \medskip
    \item Student Affairs application keeps a B-tree index\\
      on the student number
    \item B-tree search algorithms are used for queries
    \smallskip
    \item what if we decide to switch to a hashed index?
  \end{itemize}
\end{frame}

\section{Database Management Systems}

\subsection{Introduction}

\begin{frame}
  \frametitle{Database Management Systems}

  \begin{columns}[b]
    \column{.45\textwidth}
    \begin{center}
      \pgfuseimage{dbms}
    \end{center}

    \column{.55\textwidth}
    \begin{itemize}
      \item data is kept in a shared system
      \item applications access data\\
        over a common interface
    \end{itemize}
  \end{columns}
\end{frame}

\begin{frame}
  \frametitle{ANSI/SPARC Architecture}

  \begin{center}
    \pgfuseimage{sparc}
  \end{center}
\end{frame}

\begin{frame}
  \frametitle{External Level}

  \begin{itemize}
    \item external level from the end user's perspective:
    \item data needed by that end user
    \item interface of the application

    \pause
    \bigskip
    \item external level from the application programmer's perspective:
    \item programming language
    \item database extensions to this language: \alert{data sublanguage}
  \end{itemize}
\end{frame}

\begin{frame}
  \frametitle{Conceptual Level}

  \begin{itemize}
    \item conceptual level: the entire data
    \item where data independence is achieved

    \bigskip
    \item \alert{catalogue}: definitions that describe the data
    \item databases
    \item data types, integrity constraints
    \item users, privileges, security constraints
  \end{itemize}
\end{frame}

\begin{frame}
  \frametitle{Internal Level}

  \begin{itemize}
    \item internal level: implementation details

    \medskip
    \item how the data is represented
    \item files, records

    \smallskip
    \item how the data is accessed
    \item pointers, indexes, B-trees
  \end{itemize}
\end{frame}

\begin{frame}
  \frametitle{Conversions}

  \begin{itemize}
    \item conversions between levels for data independence
  \end{itemize}

  \medskip
  \begin{exampleblock}{example: conceptual - external}
    \begin{itemize}
      \item present the student number\\
        as a string to the Student Affairs application, and\\
        as a number to the Library application
    \end{itemize}
  \end{exampleblock}

  \pause
  \begin{exampleblock}{example: conceptual - internal}
    \begin{itemize}
      \item generate an index on the student number
    \end{itemize}
  \end{exampleblock}
\end{frame}

\begin{frame}
  \frametitle{Administrator Roles}

  \begin{itemize}
    \item data administator: makes the decisions
    \item which data will be stored?
    \item who can access which data?

    \pause
    \bigskip
    \item database administrator: applies the decisions
    \item defines the conceptual - external/internal conversions
    \item adjusts system performance
    \item guarantees system availability
  \end{itemize}
\end{frame}

\begin{frame}
  \frametitle{DBMS Functions}

  \begin{itemize}
    \item data definition language
    \item data manipulation language

    \pause
    \medskip
    \item checking data manipulation requests for security constraints
    \item checking data manipulation requests for integrity constraints
    \item processing simultaneous requests properly
    \item performance
  \end{itemize}
\end{frame}

\subsection{Client / Server}

\begin{frame}
  \frametitle{Client / Server Architecture}

  \begin{itemize}
    \item \alert{server}: provides the DBMS functions

    \bigskip
    \item \alert{client}: provides the interaction between the user and the server
    \item vendor supplied tools (query processors, report generators, \ldots)
    \item applications developed by application programmers
  \end{itemize}
\end{frame}

\begin{frame}
  \frametitle{Architecture}

  \begin{columns}
    \column{.4\textwidth}
    \begin{center}
      \pgfuseimage{architecture1}
    \end{center}

    \column{.6\textwidth}
    \begin{itemize}
      \item client and server can be\\
        on the same computer or\\
        on different computers
    \end{itemize}
  \end{columns}
\end{frame}

\begin{frame}
  \frametitle{Multiple Clients / Single Server}

  \begin{columns}
    \column{.45\textwidth}
    \begin{center}
      \pgfuseimage{architecture2}
    \end{center}

    \column{.55\textwidth}
    \begin{itemize}
      \item many clients can connect\\
        to a single server
    \end{itemize}

    \bigskip
    \begin{itemize}
      \item server is a bottleneck
      \item replication
    \end{itemize}
  \end{columns}
\end{frame}

\subsection{SQL}

\begin{frame}
  \frametitle{SQL}

  \begin{itemize}
    \item \alert{Structured Query Language}
    \item data definition language
    \item data manipulation language
    \item interaction with general purpose programming languages

    \bigskip
    \item started by IBM in the 1970s
    \item standards: 1992, 1999, 2003
  \end{itemize}
\end{frame}

\begin{frame}
  \frametitle{SQL Products}

  \begin{itemize}
    \item Oracle, IBM DB2, MS-SQL, \ldots
    \item open source: PostgreSQL, MySQL, \ldots
    \item embedded: SQLite, \ldots
  \end{itemize}
\end{frame}

\section*{References}

\begin{frame}
  \frametitle{References}

  \begin{block}{Required Reading: Date}
    \begin{itemize}
      \item Chapter 1: An Overview of Database Management
      \begin{itemize}
        \item 1.4. \alert{Why Database?}
        \item 1.5. \alert{Data Independence}
      \end{itemize}

      \item Chapter 2: \alert{Database System Architecture}
    \end{itemize}
  \end{block}
\end{frame}

\end{document}
