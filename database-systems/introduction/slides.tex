% Copyright (c) 2002-2010
%       H. Turgut Uyar <uyar@itu.edu.tr>
%       Şule Gündüz Öğüdücü <sgunduz@itu.edu.tr>
%
% These notes are licensed using the
% "Creative Commons Attribution-NonCommercial-ShareAlike License".
% You are free to copy, distribute and transmit the work, and to adapt the work
% as long as you attribute the authors, do not use it for commercial purposes,
% and any derivative work is under the same or a similar license.
%
% Read the full legal code at:
% http://creativecommons.org/licenses/by-nc-sa/3.0/

\documentclass[dvipsnames]{beamer}

\usepackage{ae}
\usepackage[T1]{fontenc}
\usepackage[utf8]{inputenc}
\setbeamertemplate{navigation symbols}{}

\mode<presentation>
{
  \usetheme{Warsaw}
  \usecolortheme[named=ForestGreen]{structure}
  \setbeamercovered{transparent}
}

\title{Database Systems}
\subtitle{Introduction}

\author{H. Turgut Uyar \and Şule Öğüdücü}
\date{2002-2010}

\AtBeginSubsection[]{
  \begin{frame}<beamer>
    \frametitle{Topics}
    \tableofcontents[currentsection,currentsubsection]
  \end{frame}
}

\theoremstyle{plain}

\pgfdeclareimage[width=2cm]{license}{../../license}

\pgfdeclareimage{records}{records}
\pgfdeclareimage{dbms}{dbms}
\pgfdeclareimage{sparc}{sparc}
\pgfdeclareimage{architecture1}{architecture1}
\pgfdeclareimage{architecture2}{architecture2}
\pgfdeclareimage{architecture3}{architecture3}

\begin{document}

\begin{frame}
  \titlepage
\end{frame}

\begin{frame}
  \frametitle{License}

  \pgfuseimage{license}\hfill
  \copyright 2002-2010 T. Uyar, Ş. Öğüdücü

  \vfill
  \begin{tiny}
    You are free:
    \begin{itemize}
      \item to Share — to copy, distribute and transmit the work
      \item to Remix — to adapt the work
    \end{itemize}

    Under the following conditions:
    \begin{itemize}
      \item Attribution — You must attribute the work in the manner specified by
        the author or licensor (but not in any way that suggests that they
        endorse you or your use of the work).

      \item Noncommercial — You may not use this work for commercial purposes.

      \item Share Alike — If you alter, transform, or build upon this work, you
        may distribute the resulting work only under the same or similar license
        to this one.
    \end{itemize}
  \end{tiny}

  \vfill
  Legal code (the full license):\\
  \url{http://creativecommons.org/licenses/by-nc-sa/3.0/}
\end{frame}

\begin{frame}
  \frametitle{Topics}
  \tableofcontents
\end{frame}

\section{Data Processing}

\subsection{Introduction}

\begin{frame}
  \frametitle{Processing Data}

  \begin{itemize}
    \item \emph{problem}:\\
      effectively storing and processing large amounts of data

    \pause
    \medskip
    \item processing data
    \begin{itemize}
      \item adding new data
      \item changing existing data
      \item deleting data
      \item querying: planned - ad hoc
    \end{itemize}
  \end{itemize}
\end{frame}

\begin{frame}
  \frametitle{Data Types}

  \begin{itemize}
    \item \emph{persistent data}:\\
      data that must be stored due to the nature of the information

    \pause
    \bigskip
    \item \emph{temporary data}
    \begin{itemize}
      \item \emph{output data}: data that can be derived from persistent data\\
        (query results, reports, etc.)

      \pause
      \medskip
      \item \emph{input data}: data which just entered the system
      \begin{itemize}
        \item can be added to persistent data
        \item can cause changes in the persistent data
        \item can be deleted without processing
      \end{itemize}
    \end{itemize}
  \end{itemize}
\end{frame}

\begin{frame}
  \frametitle{Roles}

  \begin{itemize}
    \item \emph{end users}:\\
      people who work on the data
    \begin{itemize}
      \item assumed not to have any technical knowledge
    \end{itemize}

    \pause
    \bigskip
    \item \emph{application programmers}:\\
      people who develop the programs that the end users use
  \end{itemize}
\end{frame}

\begin{frame}
  \frametitle{Application Example}

  \begin{example}[Student data]
    \begin{columns}[t]
      \begin{column}<1->{5.1cm}
      \begin{itemize}
        \item Student Affairs:\\
          name, number, department, courses taken, internships, etc.
      \end{itemize}
      \end{column}

      \begin{column}<2->{5.1cm}
      \begin{itemize}
        \item common data:\\
          name, number, department, etc.
      \end{itemize}
      \end{column}
    \end{columns}

    \begin{columns}[t]
      \begin{column}<1->{5.1cm}
      \begin{itemize}
        \item Library:\\
          name, number, department, books taken, etc.
      \end{itemize}
      \end{column}

      \begin{column}<2->{5.1cm}
      \begin{itemize}
        \item application specific data:\\
          courses taken, internships, books, etc.
      \end{itemize}
      \end{column}
    \end{columns}
  \end{example}
\end{frame}

\subsection{Record Files}

\begin{frame}
  \frametitle{Record Files}

  \begin{columns}
    \column{.5\textwidth}
    \begin{center}
      \pgfuseimage{records}
    \end{center}

    \column{.5\textwidth}
    \begin{itemize}
      \item each application has its own data
      \item the application keeps its data in the files that it manages itself
    \end{itemize}
  \end{columns}
\end{frame}

\begin{frame}
  \frametitle{Redundancy}

  \begin{itemize}
    \item the same data is kept in multiple places
    \begin{itemize}
      \item waste of disk space
    \end{itemize}
  \end{itemize}

  \pause
  \begin{example}
    \begin{itemize}
      \item the name, number and department of the student is kept both in the
        Student Affairs and in the Library
    \end{itemize}
  \end{example}
\end{frame}

\begin{frame}
  \frametitle{Inconsistency}

  \begin{itemize}
    \item if there are multiple copies of the same data, they can become
      different
  \end{itemize}

  \pause
  \begin{example}
    \begin{itemize}
      \item the name of the student can be recorded as "Victoria Adams" in
        Student Affairs and "Victoria Beckham" in the Library
    \end{itemize}
  \end{example}
\end{frame}

\begin{frame}
  \frametitle{Integrity}

  \begin{itemize}
    \item records can refer to invalid data
    \begin{itemize}
      \item especially in coded information
    \end{itemize}
  \end{itemize}

  \pause
  \begin{example}
    \begin{itemize}
      \item the department of the student can be recorded as "Control and
        Computer Engineering" but there might not be such a department anymore
    \end{itemize}
  \end{example}
\end{frame}

\begin{frame}
  \frametitle{New Applications}

  \begin{itemize}
    \item same work has to be done for each new application
  \end{itemize}

  \pause
  \begin{example}
    \begin{itemize}
      \item a new application will be developed for the Scholarship Office
    \end{itemize}
  \end{example}
\end{frame}

\begin{frame}
  \frametitle{Policy Gaps}

  \begin{itemize}
    \item no standards in the applications of the institution
    \begin{itemize}
      \item differences in paradigms, methods, programming languages, etc.
      \item data transfer between applications
    \end{itemize}

    \pause
    \item each department considers only its own requirements
  \end{itemize}
\end{frame}

\begin{frame}
  \frametitle{Security}

  \begin{itemize}
    \item hard to define detailed security arrangements
    \item security depends only on the operating system
  \end{itemize}
\end{frame}

\begin{frame}
  \frametitle{Data Dependence}

  \begin{definition}
    \alert{data dependence}:\\
      the application code depends on the organization of the data and the
      access method

    \begin{itemize}
      \item hard to make any changes in the code
    \end{itemize}
  \end{definition}
\end{frame}

\begin{frame}
  \frametitle{Data Dependence}

  \begin{example}
    \begin{itemize}
      \item the student number is a string in Student Affairs but a number in
        the Library

      \pause
      \item the Student Affairs application keeps a B-tree index on the student
        number and uses B-tree search algorithms in queries
    \end{itemize}
  \end{example}
\end{frame}

\section{Database Management Systems}

\subsection{Introduction}

\begin{frame}
  \frametitle{Database Management Systems}

  \begin{columns}
    \column{.5\textwidth}
    \begin{center}
      \pgfuseimage{dbms}
    \end{center}

    \column{.5\textwidth}
    \begin{itemize}
      \item data is kept in a shared system
      \item applications access data over a common interface
    \end{itemize}
  \end{columns}
\end{frame}

\begin{frame}
  \frametitle{ANSI/SPARC Architecture}

  \begin{center}
    \pgfuseimage{sparc}
  \end{center}
\end{frame}

\begin{frame}
  \frametitle{External Level}

  \begin{itemize}
    \item from the end user's perspective:
    \begin{itemize}
      \item the data needed by that end user
      \item the interface of the application that she is using
    \end{itemize}

    \pause
    \bigskip
    \item from the application programmer's perspective:
    \begin{itemize}
      \item the programming language she uses
      \item the extensions to this language for database operations:\\
        \alert{data sublanguage}
    \end{itemize}
  \end{itemize}
\end{frame}

\begin{frame}
  \frametitle{Conceptual Level}

  \begin{itemize}
    \item the entire data
    \item where the data independence is achieved

    \pause
    \bigskip
    \item \alert{catalogue}:\\
      definitions that describe the content of the data

    \begin{itemize}
      \item databases
      \item data types, integrity constraints
      \item users, privileges, security constraints
    \end{itemize}
  \end{itemize}
\end{frame}

\begin{frame}
  \frametitle{Internal Level}

  \begin{itemize}
    \item physical level

    \pause
    \item how the data is represented
    \begin{itemize}
      \item files, records
    \end{itemize}

    \pause
    \item how the data is accessed
    \begin{itemize}
      \item pointers, indexes, B-trees
    \end{itemize}
  \end{itemize}
\end{frame}

\begin{frame}
  \frametitle{Conversions}

  \begin{itemize}
    \item conversions between levels for abstraction
  \end{itemize}

  \pause
  \begin{columns}[t]
    \column{.5\textwidth}
    \begin{example}[conceptual - external]
      \begin{itemize}
        \item give the student number as a string to the Student Affairs
          application and as a number to the Library application
      \end{itemize}
    \end{example}

    \pause
    \column{.5\textwidth}
    \begin{example}[conceptual - internal]
      \begin{itemize}
        \item generate an index on the student number
      \end{itemize}
    \end{example}
  \end{columns}
\end{frame}

\begin{frame}
  \frametitle{Administrator Roles}

  \begin{itemize}
    \item \emph{data administator}:\\
      makes the decisions
    \begin{itemize}
      \item which data will be stored?
      \item who can access which data?
    \end{itemize}

    \pause
    \bigskip
    \item \emph{database administrator}:\\
      applies the decisions
    \begin{itemize}
      \item defines the conceptual-external/internal conversions
      \item adjusts system performance
      \item guarantees system availability
    \end{itemize}
  \end{itemize}
\end{frame}

\begin{frame}
  \frametitle{DBMS Functions}

  \begin{itemize}
    \item data definition language

    \pause
    \item data manipulation language

    \pause
    \item checking whether data manipulation requests conform to integrity
      and security constraints

    \pause
    \item processing simultaneous requests properly

    \pause
    \item performance
  \end{itemize}
\end{frame}

\subsection{Client/Server}

\begin{frame}
  \frametitle{Client/Server Architecture}

  \begin{itemize}
    \item \alert{server}:\\
      provides the DBMS functions

    \pause
    \bigskip
    \item \alert{client}:\\
      provides interaction between the user and the server
    \begin{itemize}
      \item vendor supplied tools (query processors, report generators, etc.)
      \item applications developed by application programmers
    \end{itemize}
  \end{itemize}
\end{frame}

\begin{frame}
  \frametitle{Architecture}

  \begin{columns}
    \column{.5\textwidth}
    \begin{center}
      \pgfuseimage{architecture1}
    \end{center}

    \column{.5\textwidth}
    \begin{itemize}
      \item the client and the server can be on the same computer
    \end{itemize}
  \end{columns}
\end{frame}

\begin{frame}
  \frametitle{Multiple Clients / Single Server}

  \begin{columns}
    \column{.5\textwidth}
    \begin{center}
      \pgfuseimage{architecture2}
    \end{center}

    \column{.5\textwidth}
    \begin{itemize}
      \item many clients can connect to a single server
    \end{itemize}

    \pause
    \bigskip
    \begin{example}[Bank]
      \begin{itemize}
        \item a server in the computer centre
        \item clients in branches
      \end{itemize}
    \end{example}
  \end{columns}
\end{frame}

\begin{frame}
  \frametitle{Multiple Clients / Multiple Servers}

  \begin{columns}
    \column{.5\textwidth}
    \begin{center}
      \pgfuseimage{architecture3}
    \end{center}

    \column{.5\textwidth}
    \begin{itemize}
      \item the servers can also be distributed
    \end{itemize}

    \pause
    \bigskip
    \begin{example}[Bank]
      \begin{itemize}
        \item each branch is the server (and client) for its own accounts
        \item a client for other branches' accounts
      \end{itemize}
    \end{example}
  \end{columns}
\end{frame}

\subsection{SQL}

\begin{frame}
  \frametitle{SQL}

  \begin{itemize}
    \item \emph{Structured Query Language}
    \begin{itemize}
      \item data definition language
      \item data manipulation language
      \item interaction with general purpose programming languages
    \end{itemize}

    \pause
    \bigskip
    \item history
    \begin{itemize}
      \item started by IBM in the 1970s
      \item standards: 1992, 1999, 2003
    \end{itemize}
  \end{itemize}
\end{frame}

\begin{frame}
  \frametitle{SQL Products}

  \begin{itemize}
    \item Oracle
    \item IBM DB2, Progress, MS-SQL, Sybase
    \item open source: PostgreSQL, MySQL, Firebird
    \item embedded: SQLite, MS Access
  \end{itemize}
\end{frame}

\section*{References}

\begin{frame}
  \frametitle{References}

  \begin{block}{Required text: Date}
    \begin{itemize}
      \item Chapter 1: An Overview of Database Management
      \begin{itemize}
        \item 1.4. \alert{Why Database?}
        \item 1.5. \alert{Data Independence}
      \end{itemize}

      \item Chapter 2: \alert{Database System Architecture}
    \end{itemize}
  \end{block}
\end{frame}

\end{document}
