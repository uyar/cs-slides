% Copyright (c) 2002-2012
%       H. Turgut Uyar <uyar@itu.edu.tr>
%       Şule Gündüz Öğüdücü <sgunduz@itu.edu.tr>
%
% These notes are licensed using the
% "Creative Commons Attribution-NonCommercial-ShareAlike License".
% You are free to copy, distribute and transmit the work, and to adapt the work
% as long as you attribute the authors, do not use it for commercial purposes,
% and any derivative work is under the same or a similar license.
%
% Read the full legal code at:
% http://creativecommons.org/licenses/by-nc-sa/3.0/

\documentclass[dvipsnames]{beamer}

\usepackage{ae}
\usepackage[T1]{fontenc}
\usepackage[utf8]{inputenc}
\setbeamertemplate{navigation symbols}{}

\mode<presentation>
{
  \usetheme{Warsaw}
  \usecolortheme[named=ForestGreen]{structure}
  \setbeamercovered{transparent}
}

\title{Database Systems}
\subtitle{Introduction}

\author{H. Turgut Uyar \and Şule Öğüdücü}
\date{2002-2012}

\AtBeginSubsection[]{
  \begin{frame}<beamer>
    \frametitle{Topics}
    \tableofcontents[currentsection,currentsubsection]
  \end{frame}
}

\theoremstyle{plain}

\pgfdeclareimage[width=2cm]{license}{../../license}

\pgfdeclareimage{records}{records}
\pgfdeclareimage{dbms}{dbms}
\pgfdeclareimage{sparc}{sparc}
\pgfdeclareimage{architecture1}{architecture1}
\pgfdeclareimage{architecture2}{architecture2}
\pgfdeclareimage{architecture3}{architecture3}

\begin{document}

\begin{frame}
  \titlepage
\end{frame}

\begin{frame}
  \frametitle{License}

  \pgfuseimage{license}\hfill
  \copyright 2002-2012 T. Uyar, Ş. Öğüdücü

  \vfill
  \begin{tiny}
    You are free:
    \begin{itemize}
      \item to Share -- to copy, distribute and transmit the work
      \item to Remix -- to adapt the work
    \end{itemize}

    Under the following conditions:
    \begin{itemize}
      \item Attribution -- You must attribute the work in the manner specified by
        the author or licensor (but not in any way that suggests that they
        endorse you or your use of the work).

      \item Noncommercial -- You may not use this work for commercial purposes.

      \item Share Alike -- If you alter, transform, or build upon this work, you
        may distribute the resulting work only under the same or similar license
        to this one.
    \end{itemize}
  \end{tiny}

  \vfill
  Legal code (the full license):\\
  \url{http://creativecommons.org/licenses/by-nc-sa/3.0/}
\end{frame}

\begin{frame}
  \frametitle{Topics}
  \tableofcontents
\end{frame}

\section{Data Processing}

\subsection{Introduction}

\begin{frame}
  \frametitle{Data Processing}

  \begin{itemize}
    \item storing and processing large amounts of data effectively

    \pause
    \medskip
    \item basic functions
    \begin{itemize}
      \item adding new data
      \item changing existing data
      \item deleting data
      \item querying: planned - ad hoc
    \end{itemize}
    \item \alert{CRUD}: create - read - update - delete
  \end{itemize}
\end{frame}

\begin{frame}
  \frametitle{Data Types}

  \begin{itemize}
    \item \emph{persistent data}:\\
      data that must be stored due to the nature of the information

    \pause
    \bigskip
    \item \emph{temporary data}
    \begin{itemize}
      \item \emph{output data}: data that can be derived from persistent data\\
        (query results, reports, etc.)

      \pause
      \medskip
      \item \emph{input data}: unprocessed data that just entered the system
      \begin{itemize}
        \item can be added to persistent data
        \item can cause changes in persistent data
        \item can be ignored completely
      \end{itemize}
    \end{itemize}
  \end{itemize}
\end{frame}

\begin{frame}
  \frametitle{Roles}

  \begin{itemize}
    \item \emph{end users}:\\
      people who work on the data
    \begin{itemize}
      \item assumed not to have any technical knowledge
    \end{itemize}

    \pause
    \bigskip
    \item \emph{application programmers}:\\
      people who develop the programs that the end users use
  \end{itemize}
\end{frame}

\begin{frame}
  \frametitle{Application Example}

  \begin{example}[student data]
    \begin{columns}[t]
      \begin{column}<1->{5.1cm}
      \begin{itemize}
        \item Student Affairs:\\
          student name, number,
          department, courses taken,\\
          internships, etc.
      \end{itemize}
      \end{column}

      \begin{column}<2->{5.1cm}
      \begin{itemize}
        \item common data:\\
          student name, number,\\
          department, etc.
      \end{itemize}
      \end{column}
    \end{columns}

    \begin{columns}[t]
      \begin{column}<1->{5.1cm}
      \begin{itemize}
        \item Library:\\
          student name, number,\\
          department, books lent, etc.
      \end{itemize}
      \end{column}

      \begin{column}<2->{5.1cm}
      \begin{itemize}
        \item application specific data:\\
          courses, internships, books, etc.
      \end{itemize}
      \end{column}
    \end{columns}
  \end{example}
\end{frame}

\subsection{Record Files}

\begin{frame}
  \frametitle{Record Files}

  \begin{columns}[b]
    \column{.4\textwidth}
    \begin{center}
      \pgfuseimage{records}
    \end{center}

    \column{.6\textwidth}
    \begin{itemize}
      \item every application has its own data
      \item every application keeps its data\\
	in the files that it manages itself
    \end{itemize}
  \end{columns}
\end{frame}

\begin{frame}
  \frametitle{Redundancy}

  \begin{itemize}
    \item the same data is kept in multiple places
    \begin{itemize}
      \item waste of disk space
    \end{itemize}
  \end{itemize}

  \pause
  \begin{example}
    \begin{itemize}
      \item the names, numbers and departments of students are kept\\
	both in Student Affairs and in the Library
    \end{itemize}
  \end{example}
\end{frame}

\begin{frame}
  \frametitle{Inconsistency}

  \begin{itemize}
    \item multiple copies of the same data can become different
  \end{itemize}

  \pause
  \begin{example}
    \begin{itemize}
      \item the name of the same student can be recorded\\
	as "Victoria Adams" in Student Affairs and\\
	as "Victoria Beckham" in the Library
    \end{itemize}
  \end{example}
\end{frame}

\begin{frame}
  \frametitle{Loss of Integrity}

  \begin{itemize}
    \item it is difficult to keep the data correct
  \end{itemize}

  \pause
  \begin{example}
    \begin{itemize}
      \item a student transfers from "Civil Engineering"\\
	to "Computer Engineering"
      \item Student Affairs data is updated, Library data is not
    \end{itemize}
  \end{example}
\end{frame}

\begin{frame}
  \frametitle{Difficulties in New Applications}

  \begin{itemize}
    \item a lot of work must be duplicated for every new application
  \end{itemize}

  \pause
  \begin{example}
    \begin{itemize}
      \item a new application will be developed for the Scholarship Office
    \end{itemize}
  \end{example}
\end{frame}

\begin{frame}
  \frametitle{Policy Gaps}

  \begin{itemize}
    \item no standards in the applications of the institution
    \begin{itemize}
      \item different paradigms, methods, programming languages
      \item data transfer between applications
    \end{itemize}

    \pause
    \item each department considers only its own requirements
  \end{itemize}
\end{frame}

\begin{frame}
  \frametitle{Security}

  \begin{itemize}
    \item hard to define detailed security permissions
    \item security depends only on the operating system
  \end{itemize}
\end{frame}

\begin{frame}
  \frametitle{Data Dependence}

  \begin{definition}
    \alert{data dependence}:\\
      application code depends on the organization of the data\\
      and the access method

    \begin{itemize}
      \item hard to make changes in the code
    \end{itemize}
  \end{definition}
\end{frame}

\begin{frame}
  \frametitle{Data Dependence}

  \begin{example}
    \begin{itemize}
      \item the student number is a string in Student Affairs\\
	but a number in the Library

      \pause
      \item the Student Affairs application keeps a B-tree index\\
	on the student number
      \begin{itemize}
	\item B-tree search algorithms are used in queries
      \end{itemize}
    \end{itemize}
  \end{example}
\end{frame}

\section{Database Management Systems}

\subsection{Introduction}

\begin{frame}
  \frametitle{Database Management Systems}

  \begin{columns}[b]
    \column{.45\textwidth}
    \begin{center}
      \pgfuseimage{dbms}
    \end{center}

    \column{.55\textwidth}
    \begin{itemize}
      \item data is kept in a shared system
      \item applications access data\\
	over a common interface
    \end{itemize}
  \end{columns}
\end{frame}

\begin{frame}
  \frametitle{ANSI/SPARC Architecture}

  \begin{center}
    \pgfuseimage{sparc}
  \end{center}
\end{frame}

\begin{frame}
  \frametitle{External Level}

  \begin{itemize}
    \item external level from the end user's perspective:
    \begin{itemize}
      \item the data needed by that end user
      \item the interface of the application that she is using
    \end{itemize}

    \pause
    \bigskip
    \item external level from the application programmer's perspective:
    \begin{itemize}
      \item the programming language she uses
      \item the extensions to this language for database operations:\\
        \alert{data sublanguage}
    \end{itemize}
  \end{itemize}
\end{frame}

\begin{frame}
  \frametitle{Conceptual Level}

  \begin{itemize}
    \item conceptual level: the entire data
    \item where data independence is achieved

    \pause
    \bigskip
    \item \alert{catalogue}:\\
      definitions that describe the content of the data
    \begin{itemize}
      \item databases
      \item data types, integrity constraints
      \item users, privileges, security constraints
    \end{itemize}
  \end{itemize}
\end{frame}

\begin{frame}
  \frametitle{Internal Level}

  \begin{itemize}
    \item internal level: implementation details

    \pause
    \item how the data is represented
    \begin{itemize}
      \item files, records
    \end{itemize}

    \pause
    \item how the data is accessed
    \begin{itemize}
      \item pointers, indexes, B-trees
    \end{itemize}
  \end{itemize}
\end{frame}

\begin{frame}
  \frametitle{Conversions}

  \begin{itemize}
    \item conversions between levels for data independence
  \end{itemize}

  \pause
  \begin{example}[conceptual - external]
    \begin{itemize}
      \item present the student number\\
        as a string to the Student Affairs application and\\
        as a number to the Library application
    \end{itemize}
  \end{example}

  \pause
  \begin{example}[conceptual - internal]
    \begin{itemize}
      \item generate an index on the student number
    \end{itemize}
  \end{example}
\end{frame}

\begin{frame}
  \frametitle{Administrator Roles}

  \begin{itemize}
    \item \emph{data administator}: makes the decisions
    \begin{itemize}
      \item which data will be stored?
      \item who can access which data?
    \end{itemize}

    \pause
    \bigskip
    \item \emph{database administrator}: applies the decisions
    \begin{itemize}
      \item defines the conceptual - external/internal conversions
      \item adjusts system performance
      \item guarantees system availability
    \end{itemize}
  \end{itemize}
\end{frame}

\begin{frame}
  \frametitle{DBMS Functions}

  \begin{itemize}
    \item data definition language

    \pause
    \item data manipulation language

    \pause
    \item checking whether data manipulation requests\\
      conform to integrity and security constraints

    \pause
    \item processing simultaneous requests properly

    \pause
    \item performance
  \end{itemize}
\end{frame}

\subsection{Client/Server}

\begin{frame}
  \frametitle{Client/Server Architecture}

  \begin{itemize}
    \item \alert{server}:\\
      provides the DBMS functions

    \pause
    \bigskip
    \item \alert{client}:\\
      provides the interaction between the user and the server
    \begin{itemize}
      \item vendor supplied tools (query processors, report generators, etc.)
      \item applications developed by application programmers
    \end{itemize}
  \end{itemize}
\end{frame}

\begin{frame}
  \frametitle{Architecture}

  \begin{columns}
    \column{.4\textwidth}
    \begin{center}
      \pgfuseimage{architecture1}
    \end{center}

    \column{.6\textwidth}
    \begin{itemize}
      \item the client and the server can be\\
	on the same computer or\\
	on different computers
    \end{itemize}
  \end{columns}
\end{frame}

\begin{frame}
  \frametitle{Multiple Clients / Single Server}

  \begin{columns}
    \column{.45\textwidth}
    \begin{center}
      \pgfuseimage{architecture2}
    \end{center}

    \column{.55\textwidth}
    \begin{itemize}
      \item many clients can connect\\
	to a single server
    \end{itemize}

    \pause
    \bigskip
    \begin{example}[Bank]
      \begin{itemize}
        \item server in the computer centre
        \item clients in branches
      \end{itemize}
    \end{example}
  \end{columns}
\end{frame}

\begin{frame}
  \frametitle{Multiple Clients / Multiple Servers}

  \begin{columns}
    \column{.45\textwidth}
    \begin{center}
      \pgfuseimage{architecture3}
    \end{center}

    \column{.55\textwidth}
    \begin{itemize}
      \item servers can be distributed too
    \end{itemize}

    \pause
    \bigskip
    \begin{example}[Bank]
      \begin{itemize}
        \item each branch is the server\\
	  (and client) for its own accounts
        \item each branch is a client\\
	  for other branches' accounts
      \end{itemize}
    \end{example}
  \end{columns}
\end{frame}

\subsection{SQL}

\begin{frame}
  \frametitle{SQL}

  \begin{itemize}
    \item \emph{Structured Query Language}
    \begin{itemize}
      \item data definition language
      \item data manipulation language
      \item interaction with general purpose programming languages
    \end{itemize}

    \pause
    \bigskip
    \item history
    \begin{itemize}
      \item started by IBM in the 1970s
      \item standards: 1992, 1999, 2003
    \end{itemize}
  \end{itemize}
\end{frame}

\begin{frame}
  \frametitle{SQL Products}

  \begin{itemize}
    \item Oracle
    \item IBM DB2, Progress, MS-SQL, Sybase
    \item open source: PostgreSQL, MySQL, Firebird
    \item embedded: SQLite, MS Access
  \end{itemize}
\end{frame}

\section*{References}

\begin{frame}
  \frametitle{References}

  \begin{block}{Required Reading: Date}
    \begin{itemize}
      \item Chapter 1: An Overview of Database Management
      \begin{itemize}
        \item 1.4. \alert{Why Database?}
        \item 1.5. \alert{Data Independence}
      \end{itemize}

      \item Chapter 2: \alert{Database System Architecture}
    \end{itemize}
  \end{block}
\end{frame}

\end{document}
