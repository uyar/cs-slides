% Copyright (c) 2005-2010
%       H. Turgut Uyar <uyar@itu.edu.tr>
%       Şule Gündüz Öğüdücü <sgunduz@itu.edu.tr>
%
% These notes are licensed using the
% "Creative Commons Attribution-NonCommercial-ShareAlike License".
% You are free to copy, distribute and transmit the work, and to adapt the work
% as long as you attribute the authors, do not use it for commercial purposes,
% and any derivative work is under the same or a similar license.
%
% Read the full legal code at:
% http://creativecommons.org/licenses/by-nc-sa/3.0/

\documentclass[dvipsnames]{beamer}

\usepackage{ae}
\usepackage[T1]{fontenc}
\usepackage[utf8]{inputenc}
\setbeamertemplate{navigation symbols}{}

\usepackage{listings}

\mode<presentation>
{
  \usetheme{Warsaw}
  \usecolortheme[named=ForestGreen]{structure}
  \setbeamercovered{transparent}
}

\title{Database Systems}
\subtitle{XML}

\author{H. Turgut Uyar \and Şule Öğüdücü}
\date{2005-2010}

\AtBeginSubsection[]{
  \begin{frame}<beamer>
    \frametitle{Topics}
    \tableofcontents[currentsection,currentsubsection]
  \end{frame}
}

\theoremstyle{plain}

\pgfdeclareimage[width=2cm]{license}{../../license}

\begin{document}

\begin{frame}
  \titlepage
\end{frame}

\begin{frame}
  \frametitle{License}

  \pgfuseimage{license}\hfill
  \copyright 2005-2010 T. Uyar, Ş. Öğüdücü

  \vfill
  \begin{tiny}
    You are free:
    \begin{itemize}
      \item to Share — to copy, distribute and transmit the work
      \item to Remix — to adapt the work
    \end{itemize}

    Under the following conditions:
    \begin{itemize}
      \item Attribution — You must attribute the work in the manner specified by
        the author or licensor (but not in any way that suggests that they
        endorse you or your use of the work).

      \item Noncommercial — You may not use this work for commercial purposes.

      \item Share Alike — If you alter, transform, or build upon this work, you
        may distribute the resulting work only under the same or similar license
        to this one.
    \end{itemize}
  \end{tiny}

  \vfill
  Legal code (the full license):\\
  \url{http://creativecommons.org/licenses/by-nc-sa/3.0/}
\end{frame}

\begin{frame}
  \frametitle{Topics}
  \tableofcontents
\end{frame}

\section{XML}

\subsection{Introduction}

\begin{frame}
  \frametitle{XML}

  \begin{itemize}
    \item not a language itself
    \begin{itemize}
      \item framework for defining languages
    \end{itemize}

    \pause
    \item XML-based languages
    \begin{itemize}
      \item XHTML, DocBook, SVG, MathML, WML, XMI, ...
    \end{itemize}

    \pause
    \item XML-related languages
    \begin{itemize}
      \item XPath, XQuery, XSL Transforms, SOAP, XLink, ...
    \end{itemize}
  \end{itemize}
\end{frame}

\begin{frame}
  \frametitle{XML Structure}

  \begin{itemize}
    \item an XML document forms a \emph{tree}

    \item nodes: elements, attributes, character data
    \begin{itemize}
      \item root: \emph{document element}
      \item leaves: character data, self-closing elements
    \end{itemize}
  \end{itemize}
\end{frame}

\begin{frame}[fragile]
  \frametitle{XML Example}

  \begin{example}[XHTML]
    \begin{lstlisting}[language=XML]
<html>
<head><title>ITU-SUNY ISE</title></head>
<body>
  <h1>Information Systems Engineering</h1>
  <p>You can get more information from the
    <a href="http://www.uolp.itu.edu.tr/">
      program page</a>.</p>
  <img src="logo.jpg" alt="ITU-SUNY ISE logo" />
</body>
</html>
    \end{lstlisting}
  \end{example}
\end{frame}

\begin{frame}[fragile]
  \frametitle{XML Example}

  \begin{example}[DocBook]
    \begin{lstlisting}[language=XML]
<book lang="en">
  <title>Database Systems Project</title>
  <bookinfo>...</bookinfo>
  <chapter>...</chapter>
  <chapter>...</chapter>
  ...
</book>
    \end{lstlisting}
  \end{example}
\end{frame}

\begin{frame}[fragile]
  \frametitle{XML Example}

  \begin{example}[DocBook]
    \begin{lstlisting}[language=XML]
  <bookinfo>
    <author>
      <firstname>Mehmet</firstname>
      <surname>Ascii</surname>
    </author>
    <date>2007</date>
  </bookinfo>
    \end{lstlisting}
  \end{example}
\end{frame}

\begin{frame}[fragile]
  \frametitle{XML Example}

  \begin{example}[DocBook]
    \begin{lstlisting}[language=XML]
  <chapter>
    <title>Introduction</title>
    <section>
      <title>Project Description</title>
      <para>This project ...</para>
    </section>
    ...
  </chapter>
    \end{lstlisting}
  \end{example}
\end{frame}

\begin{frame}[fragile]
  \frametitle{XML Example}

  \begin{example}[Movies]
    \begin{lstlisting}[language=XML]
<movies>
  <movie color="Color">
    <title>Usual Suspects</title>
    ...
  </movie>
  <movie color="Color">
    <title>Being John Malkovich</title>
    ...
  </movie>
  ...
</movies>
    \end{lstlisting}
  \end{example}
\end{frame}

\begin{frame}[fragile]
  \frametitle{XML Example}

  \begin{example}[Movies]
    \begin{lstlisting}[language=XML]
  <movie color="Color">
    <title>Usual Suspects</title>
    <year>1995</year>
    <score>8.7</score>
    <votes>35027</votes>
    <director>Bryan Singer</director>
    <cast>
      <actor>Gabriel Byrne</actor>
      <actor>Benicio Del Toro</actor>
    </cast>
  </movie>
    \end{lstlisting}
  \end{example}
\end{frame}

\begin{frame}
  \frametitle{Advantages of XML}

  \begin{itemize}
    \item text-based
    \begin{itemize}
      \item independent from platform and application
    \end{itemize}

    \pause
    \item easy to process
    \begin{itemize}
      \item ready-to-use parsers
    \end{itemize}

    \pause
    \item separate data from metadata
    \begin{itemize}
      \item content and presentation
    \end{itemize}
  \end{itemize}
\end{frame}

\subsection{Validity}

\begin{frame}
  \frametitle{XML Compliance}

  \begin{itemize}
    \item \alert{well-formed}: forms a regular tree

    \medskip
    \item \alert{valid}: conforms to a certain document structure
    \begin{itemize}
      \item DTD, XML Schema
    \end{itemize}
  \end{itemize}
\end{frame}

\begin{frame}[fragile]
  \frametitle{Document Type Definition Example}

  \begin{example}[Movie DTD]
    \begin{lstlisting}
<!DOCTYPE movies SYSTEM "movies1.dtd">
<movies>
  <movie color="Color">...</movie>
  <movie color="Color">...</movie>
  ...
</movies>
    \end{lstlisting}
  \end{example}
\end{frame}

\begin{frame}[fragile]
  \frametitle{DTD Example}

  \begin{example}[Movie DTD]
    \begin{lstlisting}
<!ELEMENT movies (movie*)>
<!ELEMENT movie (title,year,score,votes,
                 director?,cast?)>
<!ATTLIST movie color (Color | BW) "Color">
<!ELEMENT title (#PCDATA)>
<!ELEMENT year (#PCDATA)>
<!ELEMENT score (#PCDATA)>
<!ELEMENT votes (#PCDATA)>
<!ELEMENT director (#PCDATA)>
<!ELEMENT cast (actor)+>
<!ELEMENT actor (#PCDATA)>
    \end{lstlisting}
  \end{example}
\end{frame}

\subsection{Parsing}

\begin{frame}
  \frametitle{Parsing}

  \begin{itemize}
    \item \alert{SAX}: Simple API for XML
    \begin{itemize}
      \item event-based
      \item fast and less memory consuming
      \item hard to program
    \end{itemize}

    \pause
    \medskip
    \item \alert{DOM}: Document Object Model
    \begin{itemize}
      \item keeps document in its entirety
      \item easy to use in object oriented systems
    \end{itemize}
  \end{itemize}
\end{frame}

\begin{frame}
  \frametitle{Document Object Model}

  \begin{itemize}
    \item traversing the tree:
    \begin{itemize}
      \item \lstinline!getDocumentElement!
      \item \lstinline!getParentNode!
      \item \lstinline!getChildNodes!
      \item \lstinline!getNextSibling!
      \item \lstinline!getElementsByTagName!
    \end{itemize}

    \pause
    \item getting attributes and values:
    \begin{itemize}
      \item \lstinline!getAttribute!
      \item \lstinline!getTextContent!
    \end{itemize}

    \pause
    \item modifying the tree:
    \begin{itemize}
      \item \lstinline!appendChild removeChild!
      \item \lstinline!setAttribute!
    \end{itemize}
  \end{itemize}
\end{frame}

\begin{frame}[fragile]
  \frametitle{DOM Programming Example}

  \begin{example}[Movie class]
    \begin{lstlisting}[language=Java]
public class Movie {
  private String title;
  private boolean color;
  private int year;
  private float score;
  private int votes;
  private String director;
  private List cast;

  ...
}
    \end{lstlisting}
  \end{example}
\end{frame}

\begin{frame}[fragile]
  \frametitle{DOM Programming Example}

  \begin{example}[constructor from an XML element]
    \begin{lstlisting}[language=Java]
public Movie(Element me) {
  NodeList children = me.getChildNodes();
  this.title =
    children.item(0).getTextContent();
  this.year = Integer.parseInt(
    children.item(1).getTextContent());
  this.score = ...;
  this.votes = ...;
  this.director = ...;
  ...
}
    \end{lstlisting}
  \end{example}
\end{frame}

\begin{frame}[fragile]
  \frametitle{DOM Programming Example}

  \begin{example}[constructor from an XML element]
    \begin{lstlisting}[language=Java]
public Movie(Element me) {
  ...
  String me_color = me.getAttribute("color");
  if (me_color.compareTo("BW") == 0)
    this.color = false;
  else
    this.color = true;
  ...
}
    \end{lstlisting}
  \end{example}
\end{frame}

\begin{frame}[fragile]
  \frametitle{DOM Programming Example}

  \begin{example}[Constructor from an XML element]
    \begin{lstlisting}[language=Java]
public Movie(Element me) {
  ...
  this.cast = new ArrayList();
  Element ce = (Element) me.getLastChild();
  NodeList nodes =
    ce.getElementsByTagName("actor");
  for (int i = 0; i < nodes.getLength(); i++) {
    Element ae = (Element) nodes.item(i);
    this.cast.add(ae.getTextContent());
  }
}
    \end{lstlisting}
  \end{example}
\end{frame}

\begin{frame}[fragile]
  \frametitle{Validating Parser Example}

  \begin{example}[parse from file and validate]
    \begin{lstlisting}[language=Java]
try {
  DocumentBuilderFactory xmlFactory =
    DocumentBuilderFactory.newInstance();
  xmlFactory.setValidating(true);
  DocumentBuilder xmlBuilder =
    xmlFactory.newDocumentBuilder();
  Document xmlDocument =
    xmlBuilder.parse("imdb1.xml");
} catch (SAXException e) {
  System.err.println("Invalid document.");
  e.printStackTrace();
}
    \end{lstlisting}
  \end{example}
\end{frame}

\begin{frame}[fragile]
  \frametitle{DOM Programming Example}

  \begin{example}[Search movie by title]
    \begin{lstlisting}[language=Java]
/* get title of movie into movieTitle */
/* construct the documentElement */
NodeList nodes =
  documentElement.getElementsByTagName("title");
for (int i = 0; i < nodes.getLength(); i++) {
  Element te = (Element) nodes.item(i);
  String title = te.getTextContent();
  if (movieTitle.compareTo(title) == 0) {
    Element me = (Element) te.getParentNode();
    Movie m = new Movie(me);
  }
}
    \end{lstlisting}
  \end{example}
\end{frame}

\section{Querying}

\subsection{XPath}

\begin{frame}
  \frametitle{XPath Expressions}

  \begin{itemize}
    \item path of nodes to find: chain of location steps
    \begin{itemize}
      \item starting from the root (absolute)
      \item starting from the current node (relative)

      \medskip
      \item location steps are separated by \lstinline!/! symbols
    \end{itemize}

    \pause
    \begin{example}
      \begin{itemize}
       \item \lstinline!/movies/movie!
       \item \lstinline!cast/actor! or \lstinline!./cast/actor!
       \item \lstinline!../../year!
      \end{itemize}
    \end{example}
  \end{itemize}
\end{frame}

\begin{frame}
  \frametitle{Location Steps}

  \begin{itemize}
    \item location step structure:\\
      \lstinline!axis::node_selector[predicate]!

    \pause
    \medskip
    \item axis: where to search
    \item selector: what to search
    \item predicate: under which conditions
  \end{itemize}
\end{frame}

\begin{frame}
  \frametitle{Axes}

  \begin{itemize}
    \item \lstinline!child!:
      all children, one level (default axis)
    \item \lstinline!descendant!:
      all children, recursively (shorthand: \lstinline!//!)
    \item \lstinline!parent!:
      parent node, one level
    \item \lstinline!ancestor!:
      parent nodes, up to document element
    \item \lstinline!attribute!:
      attributes (shorthand: \lstinline!@!)
    \item \lstinline!following-sibling!:
      siblings that come later
    \item \lstinline!preceding-sibling!:
      siblings that come earlier
    \item ...
  \end{itemize}
\end{frame}

\begin{frame}
  \frametitle{Node Selectors}

  \begin{itemize}
    \item node tag
    \item node attribute
    \item node text: \lstinline!text()!
    \item all children: \lstinline!*!
  \end{itemize}
\end{frame}

\begin{frame}
  \frametitle{XPath Expression Examples}

  \begin{example}
    \begin{itemize}
      \item names of all directors:\\
        \lstinline!/child::movies/movie/director/text()!

      \pause
      \item all actors:\\
        \lstinline!movie/descendant::actor! or \lstinline!./movie//actor!

      \pause
      \item color info of all movies:\\
        \lstinline!//movie/attribute::color! or \lstinline!//movie/@color!

      \pause
      \item scores of movies after this one:\\
        \lstinline!./following-sibling::movie/score!

      \pause
      \item \lstinline!//actor/ancestor::movie/title!\\
        \lstinline!//actor/../preceding-sibling::title!
    \end{itemize}
  \end{example}
\end{frame}

\begin{frame}
  \frametitle{Predicates}

  \begin{itemize}
    \item at a certain position: \lstinline![index]!

    \pause
    \item existence of a child: \lstinline![element]!
    \item value of a child: \lstinline![element="value"]!

    \pause
    \item existence of an attribute: \lstinline![@attribute]!
    \item value of an attribute: \lstinline![@attribute="value"]!
  \end{itemize}
\end{frame}

\begin{frame}
  \frametitle{XPath Examples}

  \begin{example}
    \begin{itemize}
      \item the title of the first movie:\\
        \lstinline!/movies/movie[1]/title!

      \pause
      \item all movies in the year 1997:\\
        \lstinline!movie[year="1997"]!

      \pause
      \item black-and-white movies:\\
        \lstinline!movie[@color="BW"]!
    \end{itemize}
  \end{example}
\end{frame}

\subsection{XML Databases}

\begin{frame}
  \frametitle{XML Databases}

  \begin{itemize}
    \item storing XML documents in columns (XMLDOC type)
    \begin{itemize}
      \item make use of existing XML documents
      \item operators for this type of columns
      \item process documents in their entirety
      \item if updates are not frequent, searches are simple
    \end{itemize}

    \pause
    \item distributing XML documents into different tables and columns

    \pause
    \item native XML databases
  \end{itemize}
\end{frame}

\begin{frame}
  \frametitle{Special DTD Attributes}

  \begin{itemize}
    \item ID attributes
    \begin{itemize}
      \item similar to primary keys
    \end{itemize}

    \pause
    \item IDREF attributes
    \begin{itemize}
      \item similar to foreign keys
    \end{itemize}
  \end{itemize}
\end{frame}

\begin{frame}
  \frametitle{DTD Key Constraints}

  \begin{itemize}
    \item key values always string

    \pause
    \item keys always single-attribute

    \pause
    \item key values unique in all of document
    \begin{itemize}
      \item not just elements of the same type
    \end{itemize}

    \pause
    \item foreign keys refer only within the document
  \end{itemize}
\end{frame}

\begin{frame}[fragile]
  \frametitle{XML ID Attribute Example}

  \begin{example}[Movies]
    \begin{lstlisting}[language=XML]
<!DOCTYPE movies SYSTEM "movies2.dtd">
<movies>
  <movie color="Color">...  </movie>
  ...
  <person id="p302">Benicio Del Toro</person>
  <person id="p308">Gabriel Byrne</person>
  <person id="p639">Bryan Singer</person>
  ...
</movies>
    \end{lstlisting}
  \end{example}
\end{frame}

\begin{frame}[fragile]
  \frametitle{XML ID Attribute Example}

  \begin{example}[Movies]
    \begin{lstlisting}[language=XML]
  <movie color="Color">
    <title>Usual Suspects</title>
    ...
    <directorref id="p639" />
    <cast>
      <actorref id="p308" />
      <actorref id="p302" />
    </cast>
  </movie>
    \end{lstlisting}
  \end{example}
\end{frame}

\begin{frame}[fragile]
  \frametitle{DTD ID Attribute Example}

  \begin{example}
    \begin{lstlisting}[language=XML]
<!ELEMENT movies (movie*,person*)>
<!ELEMENT movie (title,...,directorref?,cast?)>
<!ATTLIST movie color (Color | BW) #IMPLIED>
<!ELEMENT title (#PCDATA)>
...
<!ELEMENT directorref EMPTY>
<!ATTLIST directorref id IDREF #REQUIRED>
<!ELEMENT cast (actorref)+>
<!ELEMENT actorref EMPTY>
<!ATTLIST actorref id IDREF #REQUIRED>
<!ELEMENT person (#PCDATA)>
<!ATTLIST person id ID #REQUIRED>
    \end{lstlisting}
  \end{example}
\end{frame}

\section*{References}

\begin{frame}
  \frametitle{References}

  \begin{block}{Required text: Date}
    \begin{itemize}
      \item Chapter 27: \alert{The World Wide Web and XML}
    \end{itemize}
  \end{block}
\end{frame}

\end{document}
