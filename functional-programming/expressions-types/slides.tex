% Copyright (c) 2013
%       H. Turgut Uyar <uyar@itu.edu.tr>
%
% These notes are licensed using the
% "Creative Commons Attribution-NonCommercial-ShareAlike License".
% You are free to copy, distribute and transmit the work, and to adapt the work
% as long as you attribute the authors, do not use it for commercial purposes,
% and any derivative work is under the same or a similar license.
%
% Read the full legal code at:
% http://creativecommons.org/licenses/by-nc-sa/3.0/

\documentclass[dvipsnames]{beamer}

\usepackage{ae}
\usepackage[T1]{fontenc}
\usepackage[utf8]{inputenc}
\usepackage{pythontex}
\setbeamertemplate{navigation symbols}{}

\mode<presentation>
{
  \usetheme{Boadilla}
  \setbeamercovered{transparent}
}

\title{Functional Programming}
\subtitle{Expressions and Types}

\author{H. Turgut Uyar}
\date{2013}

\AtBeginSubsection[]{
  \begin{frame}<beamer>
    \frametitle{Topics}
    \tableofcontents[currentsection,currentsubsection]
  \end{frame}
}

\theoremstyle{plain}

\pgfdeclareimage[width=2cm]{license}{../../license}

\begin{document}

\begin{frame}
  \titlepage
\end{frame}

\begin{frame}
  \frametitle{License}

  \pgfuseimage{license}\hfill
  \copyright 2013 T. Uyar

  \vfill
  \begin{tiny}
    You are free:
    \begin{itemize}
      \item to Share -- to copy, distribute and transmit the work
      \item to Remix -- to adapt the work
    \end{itemize}

    Under the following conditions:
    \begin{itemize}
      \item Attribution -- You must attribute the work in the manner specified by
        the author or licensor (but not in any way that suggests that they
        endorse you or your use of the work).

      \item Noncommercial -- You may not use this work for commercial purposes.

      \item Share Alike -- If you alter, transform, or build upon this work, you
        may distribute the resulting work only under the same or similar license
        to this one.
    \end{itemize}
  \end{tiny}

  \vfill
  Legal code (the full license):\\
  \url{http://creativecommons.org/licenses/by-nc-sa/3.0/}
\end{frame}

\begin{frame}
  \frametitle{Topics}
  \tableofcontents
\end{frame}

\section{Expressions}

\subsection{Expression Evaluation}

\begin{frame}
  \frametitle{Expressions}

  \begin{definition}
    \alert{expression}: a construct that will be \emph{evaluated} to yield a value
  \end{definition}

  \pause
  \medskip
  \begin{itemize}
    \item every programming language provides:

    \medskip
    \item primitive expressions: the simplest elements\\
    \begin{itemize}
      \item literals, constants, variables
    \end{itemize}
    \item means of combination: building compound elements from simpler ones\\
    \begin{itemize}
      \item operators
    \end{itemize}
    \item means of abstraction: naming compound elements\\
      and manipulating them as units
  \end{itemize}
\end{frame}

\begin{frame}[fragile]
  \frametitle{Expressions}

  \begin{example}[Haskell]
    \begin{pygments}{haskell}
Prelude> 6 * 7
42
Prelude> let r = 2.4
Prelude> 3.14159 * r * r
18.095558399999998
    \end{pygments}
  \end{example}
\end{frame}

\begin{frame}[fragile]
  \frametitle{Expressions}

  \begin{example}[Scala]
    \begin{pygments}{scala}
scala> 6 * 7
res2: Int = 42

scala> val r = 2.4
r: Double = 2.4

scala> 3.14159 * r * r
res3: Double = 18.095558399999998
    \end{pygments}
  \end{example}
\end{frame}

\begin{frame}[fragile]
  \frametitle{Expressions}

  \begin{example}[Python]
    \begin{pygments}{python}
>>> 6 * 7
42
>>> r = 2.4
>>> 3.14159 * r * r
18.095558399999998
    \end{pygments}
  \end{example}
\end{frame}

\begin{frame}
  \frametitle{Evaluating Expressions}

  \begin{itemize}
    \item take the operator with the highest precedence
    \item evaluate its operands (note the recursion)
    \item apply the operator to the operands

    \pause
    \medskip
    \item a name is evaluated by replacing it\\
      with the right-hand side of its definition
  \end{itemize}
\end{frame}

\begin{frame}[fragile]
  \frametitle{Evaluating Expressions}

  \begin{example}
    \pygment{python}{(3.14159 * r) * r}

    \pause
    \medskip
    \pygment{python}{(3.14159 * 2.4) * r}

    \pause
    \medskip
    \pygment{python}{7.539815999999999 * 2.4}

    \pause
    \medskip
    \pygment{python}{18.095558399999998}
  \end{example}
\end{frame}

\subsection{Function Evaluation}

\begin{frame}
  \frametitle{Functions}

  \begin{itemize}
    \item \alert{formal parameter}: an identifier\\
      through which a function can access an argument
    \begin{itemize}
      \item declared at function definition
    \end{itemize}

    \pause
    \medskip
    \item \alert{actual parameter}: an expression which yields an argument
    \begin{itemize}
      \item defined at function application
    \end{itemize}
  \end{itemize}
\end{frame}

\begin{frame}[fragile]
  \frametitle{Functions}

  \begin{example}[Haskell]
    \begin{pygments}{haskell}
square :: Int -> Int
square x = x * x

-- square 21 -> 441
-- square (2 + 5) -> 49

sum_of_squares :: (Int, Int) -> Int
sum_of_squares (x, y) = square x + square y

-- sum_of_squares (3, 4) -> 25
-- sum_of_squares (a + 1, a * 2)
    \end{pygments}
  \end{example}
\end{frame}

\begin{frame}[fragile]
  \frametitle{Functions}

  \begin{example}[Scala]
    \begin{pygments}{scala}
def square(x: Int): Int = x * x

// square(21) -> 441
// square(2 + 5) -> 49

def sumOfSquares(x: Int, y: Int): Int =
    square(x) + square(y)

// sumOfSquares(3, 4) -> 25
// sumOfSquares(a + 1, a * 2)
    \end{pygments}
  \end{example}
\end{frame}

\begin{frame}[fragile]
  \frametitle{Functions}

  \begin{example}[Python]
    \begin{pygments}{python}
def square(x):
    return x * x

# square(21) -> 441
# square(2 + 5) -> 49

def sum_of_squares(x, y):
    return square(x) + square(y)

# sum_of_squares(3, 4) -> 25
# sum_of_squares(a + 1, a * 2)
    \end{pygments}
  \end{example}
\end{frame}

\begin{frame}
  \frametitle{Evaluating Functions}

  \begin{itemize}
    \item evaluate all function arguments (left to right)
    \item replace the function application\\
      with the right-hand side of its definition
    \item replace the formal parameters of the function\\
      with the actual parameters

    \pause
    \bigskip
    \item \alert{substitution model}: reduces an expression to a value
  \end{itemize}
\end{frame}

\begin{frame}[fragile]
  \frametitle{Evaluating Functions}

  \begin{example}
    \pygment{python}{sum_of_squares(3, 2 + 2)}

    \pause
    \medskip
    \pygment{python}{sum_of_squares(3, 4)}

    \pause
    \medskip
    \pygment{python}{square(3) + square(4)}

    \pause
    \medskip
    \pygment{python}{3 * 3 + square(4)}

    \pause
    \medskip
    \pygment{python}{9 + square(4)}

    \pause
    \medskip
    \pygment{python}{9 + 4 * 4}

    \pause
    \medskip
    \pygment{python}{9 + 16}

    \pause
    \medskip
    \pygment{python}{25}
  \end{example}
\end{frame}

\begin{frame}
  \frametitle{Evaluation Strategies}

  \begin{itemize}
    \item \alert{applicative order evaluation}: first evaluate arguments\\
      and then apply the function to the resulting values\\
      (a.k.a. eager evaluation, strict evaluation, call by value)

    \pause
    \medskip
    \item \alert{normal order evaluation}: do not evaluate arguments\\
      until their values are needed\\
      (a.k.a. call by name)
  \end{itemize}
\end{frame}

\begin{frame}[fragile]
  \frametitle{Normal Order Evaluation}

  \begin{example}
    \pygment{python}{sum_of_squares(3, 2 + 2)}

    \pause
    \medskip
    \pygment{python}{square(3) + square(2 + 2)}

    \pause
    \medskip
    \pygment{python}{3 * 3 + square(2 + 2)}

    \pause
    \medskip
    \pygment{python}{9 + square(2 + 2)}

    \pause
    \medskip
    \pygment{python}{9 + (2 + 2) * (2 + 2)}

    \pause
    \medskip
    \pygment{python}{9 + 4 * (2 + 2)}

    \pause
    \medskip
    \pygment{python}{9 + 4 * 4}

    \pause
    \medskip
    \pygment{python}{9 + 16}

    \pause
    \medskip
    \pygment{python}{25}
  \end{example}
\end{frame}

\begin{frame}
  \frametitle{Church-Rosser Property}

  \begin{itemize}
    \item applicative order and normal order evaluations yield the same result\\
      as long as:

    \bigskip
    \item there are no side effects
    \item all evaluations terminate
  \end{itemize}
\end{frame}

\begin{frame}
  \frametitle{Lazy Evaluation}

  \begin{itemize}
    \item applicative order evaluation evaluates arguments only once
    \item but it might evaluate arguments which are not needed

    \pause
    \medskip
    \item normal order does not evaluate arguments which are not needed
    \item but it might evaluate others more than once

    \pause
    \medskip
    \item \alert{lazy evaluation}: evaluate an argument once\\
      when its value is \emph{first} needed
  \end{itemize}
\end{frame}

\begin{frame}[fragile]
  \frametitle{Evaluation Strategies}

  \begin{itemize}
    \item most languages use applicative order evaluation
  \end{itemize}

  \begin{example}[Python]
    \begin{pygments}{python}
def first(x, y):
    return x

# first(1, 1 / 0) -> division by zero
    \end{pygments}
  \end{example}
\end{frame}

\begin{frame}[fragile]
  \frametitle{Evaluation Strategies}

  \begin{itemize}
    \item Haskell uses lazy evaluation by default
  \end{itemize}

  \begin{example}[Haskell]
    \begin{pygments}{haskell}
first :: (Int, Int) -> Int
first (x, y) = x

-- first(1, 1 / 0) -> 1
    \end{pygments}
  \end{example}
\end{frame}

\begin{frame}[fragile]
  \frametitle{Evaluation Strategies}

  \begin{itemize}
    \item Scala uses applicative order evaluation by default
    \item but programmer can specify normal order evaluation
  \end{itemize}

  \begin{example}[Scala]
    \begin{pygments}{scala}
def first(x: Int, y: Int) = x

def first_normal(x: Int, y: => Int) = x

// first(1, 1 / 0) -> division by zero
// first_normal(1, 1 / 0) -> 1
    \end{pygments}
  \end{example}
\end{frame}

\subsection{Conditional Expressions}

\begin{frame}
  \frametitle{Conditional Expressions}

  \begin{itemize}
    \item perform different operations depending on the result of a test

    \pause
    \bigskip
    \item it's an expression, not a statement
    \item in C, not like:\\
      \pygment{c}{if (predicate) true_block; else false_block}
    \item but like:\\
      \pygment{c}{predicate ? true_expr : false_expr}
  \end{itemize}
\end{frame}

\begin{frame}[fragile]
  \frametitle{Conditional Expressions}

  \begin{example}[Haskell]
    \begin{pygments}{haskell}
-- if predicate then true_expr else false_expr
absval :: Int -> Int
absval x = if x < 0 then (-x) else x
    \end{pygments}
  \end{example}

  \pause
  \begin{example}[Scala]
    \begin{pygments}{scala}
// if (predicate) true_expr else false_expr
def absval(x: Int): Int =
    if (x < 0) -x else x
    \end{pygments}
  \end{example}

  \pause
  \begin{example}[Python]
    \begin{pygments}{python}
# true_expr if predicate else false_expr
def absval(x):
    return -x if x < 0 else x
    \end{pygments}
  \end{example}
\end{frame}

\begin{frame}
  \frametitle{Short-Circuit Evaluation}

  \begin{definition}
    \alert{short-circuit evaluation}: evaluation stops
      as soon as result is determined
  \end{definition}

  \pause
  \begin{example}
    \pygment{c}{(a >= 0) && (b < 10)}\\
    -- 2nd clause not evaluated if a < 0

    \bigskip
    \pygment{c}{(a >= 0) || (b < 10)}\\
    -- 2nd clause not evaluated if a >= 0
  \end{example}
\end{frame}

\begin{frame}[fragile]
  \frametitle{Short-Circuit Evaluation}

  \begin{itemize}
    \item code might depend on short-circuit evaluation
  \end{itemize}

  \begin{example}[Java]
    \begin{pygments}{java}
// find the index of a key in a list
index = 0;
while ((index < listlen) && (list[index] != key))
    index = index + 1;
    \end{pygments}
  \end{example}
\end{frame}

\begin{frame}[fragile]
  \frametitle{Short-Circuit Evaluation}

  \begin{itemize}
    \item using normal order evaluation, short-circuited Boolean operators\\
      can be expressed in terms of conditionals
  \end{itemize}

  \begin{visibleenv}<2->
  \begin{example}[Haskell]
    \begin{pygments}{haskell}
newAnd :: (Bool, Bool) -> Bool
newAnd (x, y) = if x then y else False

newOr :: (Bool, Bool) -> Bool
newOr (x, y) = if x then True else y

newNot :: Bool -> Bool
newNot x = if x then False else True
    \end{pygments}
  \end{example}
  \end{visibleenv}
\end{frame}

\section{Types}

\subsection{Type Systems}

\section*{References}

\begin{frame}
  \frametitle{References}

  \begin{block}{Required Reading: Abelson, Sussman}
    \begin{itemize}
      \item Chapter 1: Building Abstractions with Procedures
      \begin{itemize}
        \item 1.1. \alert{The Elements of Programming}
      \end{itemize}
    \end{itemize}
  \end{block}
\end{frame}

\end{document}
