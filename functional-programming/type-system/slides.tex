% Copyright (c) 2013-2015 H. Turgut Uyar <uyar@itu.edu.tr>
%
% This work is licensed under a "Creative Commons
% Attribution-NonCommercial-ShareAlike 4.0 International License".
% For more information, please visit:
% https://creativecommons.org/licenses/by-nc-sa/4.0/

\documentclass[dvipsnames]{beamer}

\usepackage[scaled=0.95]{cabin}
\usepackage[scaled=0.88]{beramono}
\usepackage[T1]{fontenc}
\usepackage[utf8]{inputenc}

\usepackage{listings}
\lstset{basicstyle=\ttfamily,
        keywordstyle=\color{blue},
        showstringspaces=false}
\lstdefinestyle{syntax}{frame=single}
\lstset{language=Haskell}

\mode<presentation>
{
  \usetheme{default}
  \useinnertheme{rounded}
  \usecolortheme{seahorse}
  \setbeamercovered{transparent}
}

\title{Functional Programming}
\subtitle{Type System}

\author{H. Turgut Uyar}
\date{2013-2015}

\AtBeginSubsection[]{
  \begin{frame}<beamer>
    \frametitle{Topics}
    \tableofcontents[currentsection,currentsubsection]
  \end{frame}
}

\theoremstyle{plain}

\pgfdeclareimage[height=1cm]{license}{../license}

\begin{document}

\begin{frame}
  \titlepage
\end{frame}

\begin{frame}
  \frametitle{License}

  \pgfuseimage{license}\hfill
  \copyright~2013-2015 H. Turgut Uyar

  \vfill
  \begin{footnotesize}
    You are free to:
    \begin{itemize}
      \itemsep0em
      \item Share -- copy and redistribute the material in any medium or format
      \item Adapt -- remix, transform, and build upon the material
    \end{itemize}

    Under the following terms:
    \begin{itemize}
      \itemsep0em
      \item Attribution -- You must give appropriate credit, provide a link to
        the license, and indicate if changes were made.

      \item NonCommercial -- You may not use the material for commercial
        purposes.

      \item ShareAlike -- If you remix, transform, or build upon the material,
        you must distribute your contributions under the same license as the
        original.
    \end{itemize}

    For more information:\\
    \url{https://creativecommons.org/licenses/by-nc-sa/4.0/}

    \smallskip
    Read the full license:\\
    \url{https://creativecommons.org/licenses/by-nc-sa/4.0/legalcode}
  \end{footnotesize}
\end{frame}

\begin{frame}
  \frametitle{Topics}
  \tableofcontents
\end{frame}

\section{Type System}

\subsection{Static vs Dynamic Typing}

\begin{frame}
  \frametitle{Type Errors}

  \begin{itemize}
    \item prevent operations that do not make sense
    \item check type of operands before operation

    \medskip
    \item \alert{strongly typed}:\\
      prohibit the application of any operation to any object\\
      that is not intended to support the operation
  \end{itemize}

  \pause
  \begin{exampleblock}{examples}
    \begin{itemize}
      \item \lstinline|mod|: both operands must be integers
      \item \lstinline|and|: operand must be a list of booleans
    \end{itemize}
  \end{exampleblock}
\end{frame}

\begin{frame}[fragile]
  \frametitle{Weak Typing Example}

  \begin{exampleblock}{C}
    \begin{lstlisting}[language=C]
void print_int(int i)
{
    printf("%d\n", i);
}

int main(int argc, char *argv[])
{
    float x = 51.2;     // char x[] = "itucs";
    print_int(x);
    return 0;
}
    \end{lstlisting}
  \end{exampleblock}
\end{frame}

\begin{frame}
  \frametitle{Type Checks}

  \begin{itemize}
    \item \alert{static} typing: checks at compile-time
    \item variables and expressions have fixed types

    \bigskip
    \item \alert{dynamic} typing: checks at run-time
    \item variables and expressions do not have fixed types
    \item values have fixed types
  \end{itemize}
\end{frame}

\begin{frame}[fragile]
  \frametitle{Static Typing Example}

  \begin{exampleblock}{static: Java}
    \begin{lstlisting}[language=Java]
public static boolean even(int n)
{
  return (n % 2 == 0);
}
    \end{lstlisting}
  \end{exampleblock}

  \begin{exampleblock}{dynamic: Python}
    \begin{lstlisting}[language=Python]
def even(n):
  return (n % 2 == 0)
    \end{lstlisting}
  \end{exampleblock}
\end{frame}

\begin{frame}
  \frametitle{Comparison}

  \begin{columns}[t]
    \column{.35\textwidth}
    \begin{itemize}
      \item static typing
      \smallskip
      \item more efficient
      \item more secure
      \item less flexible
    \end{itemize}

    \column{.65\textwidth}
    \begin{itemize}
      \item dynamic typing
      \smallskip
      \item requires tags: more memory
      \item poorer performance
      \item greater flexibility
    \end{itemize}
  \end{columns}
\end{frame}

\begin{frame}[fragile]
  \frametitle{Dynamic Typing Problem}

  \begin{exampleblock}{Python}
    \begin{lstlisting}[language=Python]
birthYear = 1985
if condition:
    birth_year = 1988
if birthYear < 1987:
    do_action()
    \end{lstlisting}
  \end{exampleblock}
\end{frame}

\subsection{Type Inference}

\begin{frame}
  \frametitle{Type Inference}

  \begin{itemize}
    \item type of entity not explicitly written
    \item inferred by the language processor

    \medskip
    \item assign every binding a type such that type checking succeeds
    \item fail if no such assignment can be found
  \end{itemize}
\end{frame}

\begin{frame}[fragile]
  \frametitle{Type Inference Examples}

  \begin{lstlisting}
nand x y = not (x && y)
-- nand :: Bool -> Bool -> Bool

limitedLast s n =
    if (length s > n) then s !! (n - 1) else (last s)
-- limitedLast :: [a] -> Int -> a

capitalize s = toUpper (s !! 0) : drop 1 s
-- capitalize :: [Char] -> [Char]

foo1 x y = if x then y else x + 1
-- type inference fails

foo2 f g x = (f x, g x)
-- (a -> b) -> (a -> c) -> a -> (b, c)
  \end{lstlisting}

\end{frame}

\begin{frame}[fragile]
  \frametitle{Type Inference Examples}

  \begin{lstlisting}
f (x, y) = (x, ['a' .. y])

g (m, zs) = m + length zs

h = g . f
  \end{lstlisting}

  \medskip
  \begin{itemize}
    \item exercise: what are the types of \lstinline|f|, \lstinline|g|,
      and \lstinline|h|?
  \end{itemize}
\end{frame}

\subsection{Type Classes}

\begin{frame}[fragile]
  \frametitle{Type Classes}

  \begin{exampleblock}{check whether an item is an element of a list}
    \begin{lstlisting}
elemChar :: Char -> [Char] -> Bool
elemChar _ []     = False
elemChar x (c:cs)
  | x == c        = True
  | otherwise     = elemChar x cs
    \end{lstlisting}
  \end{exampleblock}

  \pause
  \begin{itemize}
    \item a different function for every type?
    \item better if we can write it as:\\
      \lstinline|a -> [a] -> Bool|
    \item provided that type \lstinline|a| supports equality check
  \end{itemize}
\end{frame}

\begin{frame}[fragile]
  \frametitle{Type Classes}

  \begin{itemize}
    \item \alert{type class}: a collection of types\\
      over which some functions are defined
    \item every type belonging to the class must implement its functions
    \item members of a type class are called its \alert{instances}
  \end{itemize}

  \pause
  \begin{exampleblock}{equality class}
    \begin{lstlisting}
class Eq a where
  (==) :: a -> a -> Bool
    \end{lstlisting}
    \begin{itemize}
      \item \lstinline|Bool|, \lstinline|Char|, \lstinline|Integer|, \lstinline|Float|
        are instances of \lstinline|Eq|
      \item tuples and lists are also instances of \lstinline|Eq|
    \end{itemize}
  \end{exampleblock}
\end{frame}

\begin{frame}[fragile]
  \frametitle{Context}

  \begin{itemize}
    \item type signature can contain context (``provided that``)
    \item a type being an instance of a class
  \end{itemize}

  \begin{exampleblock}{}
    \begin{lstlisting}
elem :: Eq a => a -> [a] -> Bool
elem _ []     = False
elem x (c:cs)
  | x == c    = True
  | otherwise = elem x cs
    \end{lstlisting}
  \end{exampleblock}
\end{frame}

\begin{frame}[fragile]
  \frametitle{Instance Example}

  \begin{lstlisting}
data Move = Rock | Paper | Scissors

instance Eq Move where
  (==) Rock     Rock     = True  -- Rock == Rock = True
  (==) Paper    Paper    = True
  (==) Scissors Scissors = True
  (==) _        _        = False

elem Paper    [Rock, Paper, Rock] -- True
elem Scissors [Rock, Paper, Rock] -- False
  \end{lstlisting}
\end{frame}

\begin{frame}[fragile]
  \frametitle{Derived Instances}

  \begin{itemize}
    \item default equality check: derive from \lstinline|Eq|
  \end{itemize}

  \begin{exampleblock}{}
    \begin{lstlisting}
data Move = Rock | Paper | Scissors
            deriving Eq
    \end{lstlisting}
  \end{exampleblock}

  \pause
  \begin{itemize}
    \item default string conversion: derive from \lstinline|Show|
  \end{itemize}

  \begin{exampleblock}{}
    \begin{lstlisting}
class Show a where
  show :: a -> String

data Move = Rock | Paper | Scissors
            deriving (Eq, Show)
   \end{lstlisting}
  \end{exampleblock}
\end{frame}

\begin{frame}[fragile]
  \frametitle{Instance Example}

  \begin{exampleblock}{rational numbers}
    \begin{lstlisting}
data Rational = Fraction Integer Integer

instance Eq Rational where
  Fraction n1 d1 == Fraction n2 d2 = n1 * d2 == n2 * d1

instance Show Rational where
  show (Fraction n d) = show n ++ "/" ++ show d
    \end{lstlisting}
  \end{exampleblock}
\end{frame}

\begin{frame}[fragile]
  \frametitle{Default Definitions}

  \begin{itemize}
    \item classes can contain default definitions for functions
    \item they can be overridden by instances
  \end{itemize}

  \begin{exampleblock}{}
    \begin{lstlisting}
class Eq a where
  (==), (/=) :: a -> a -> Bool

  x /= y = not (x == y)
  x == y = not (x /= y)
    \end{lstlisting}
  \end{exampleblock}

  \begin{itemize}
    \item defining one of \lstinline|==| and \lstinline|/=| is enough
  \end{itemize}
\end{frame}

\begin{frame}[fragile]
  \frametitle{Derived Classes}

  \begin{itemize}
    \item type classes can depend on other type classes
  \end{itemize}

  \begin{exampleblock}{ordering class}
    \begin{lstlisting}
class Eq a => Ord a where
  (<), (<=), (>), (>=) :: a -> a -> Bool

  x <= y = (x < y || x == y)
  x >  y = y < x
    \end{lstlisting}
  \end{exampleblock}
\end{frame}

\begin{frame}[fragile]
  \frametitle{Type Class Example}

  \begin{lstlisting}
instance Ord Rational where
  Fraction n1 d1 < Fraction n2 d2 = n1 * d2 < n2 * d1

qSort :: Ord a => [a] -> [a]
qSort []     = []
qSort (x:xs) = qSort before ++ [x] ++ qSort after
  where
    before = [b | b <- xs, b <= x]
    after  = [a | a <- xs, a >  x]
  \end{lstlisting}
\end{frame}

\begin{frame}[fragile]
  \frametitle{Ordering}

  \begin{lstlisting}
data Ordering = LT | EQ | GT

class Eq a => Ord a where
  (<), (<=), (>), (>=) :: a -> a -> Bool
  max, min             :: a -> a -> a
  compare              :: a -> a -> Ordering

  compare x y
    | x == y    = EQ
    | x <= y    = LT
    | otherwise = GT
  \end{lstlisting}

  \begin{itemize}
    \item define one of \lstinline|compare| and \lstinline|<=|
  \end{itemize}
\end{frame}

\begin{frame}[fragile]
  \frametitle{Ordering Default Definitions}

  \begin{lstlisting}
  x <= y = compare x y /= GT
  x <  y = compare x y == LT
  x >= y = compare x y /= LT
  x >  y = compare x y == GT

  max x y
    | x <= y    = y
    | otherwise = x

  min x y
    | x < y     = x
    | otherwise = y
  \end{lstlisting}
\end{frame}

\begin{frame}[fragile]
  \frametitle{Numeric Class}

  \begin{lstlisting}
class (Eq a, Show a) => Num a where
  (+), (-), (*) :: a -> a -> a
  negate        :: a -> a
  abs, signum   :: a -> a
  fromInteger   :: Integer -> a

  x - y = x + negate y
  \end{lstlisting}

  \begin{itemize}
    \item exercise: make \lstinline|Rational| an instance of
      \lstinline|Num|
  \end{itemize}
\end{frame}

\section{Algebraic Types}

\subsection{Recursive Types}

\begin{frame}[fragile]
  \frametitle{Recursive Algebraic Types}

  \begin{itemize}
    \item types can be described in terms of themselves
  \end{itemize}

  \begin{exampleblock}{example: expressions}
    \begin{lstlisting}
data Expr = Lit Integer |
            Add Expr Expr |
            Mul Expr Expr
    \end{lstlisting}
  \end{exampleblock}
\end{frame}

\begin{frame}[fragile]
  \frametitle{Recursive Algebraic Types}

  \begin{exampleblock}{example: evaluating an expression}
    \begin{lstlisting}
eval :: Expr -> Integer
eval e = case e of
    Lit n     -> n
    Add e1 e2 -> (eval e1) + (eval e2)
    Mul e1 e2 -> (eval e1) * (eval e2)
    \end{lstlisting}
  \end{exampleblock}
\end{frame}

\begin{frame}[fragile]
  \frametitle{Recursive Algebraic Types}

  \begin{exampleblock}{example: converting an expression to a string}
    \begin{lstlisting}
instance Show Expr where
  show e = case e of
      Lit n     -> show n
      Add e1 e2 -> show e1 ++ "+" ++ show e2
      Mul e1 e2 -> show e1 ++ "*" ++ show e2
    \end{lstlisting}
  \end{exampleblock}

  \pause
  \begin{itemize}
    \item precedences incorrect, try with:\\
      \lstinline|Mul (Add (Lit 2) (Lit 3)) (Lit 5)|
    \item exercise: write a correct implementation\\
      where the string will contain the minimal number of parentheses
  \end{itemize}
\end{frame}

\subsection{Polymorphic Types}

\begin{frame}[fragile]
  \frametitle{Polymorphic Algebraic Types}

  \begin{exampleblock}{example: integer binary tree}
    \begin{lstlisting}
data BinTree = Nil | Node Int BinTree BinTree

depth :: BinTree -> Int
depth NilT            = 0
depth (NodeT _ t1 t2) = 1 + max (depth t1) (depth t2)

sumTree :: BinTree -> Int
sumTree Nil            = 0
sumTree (Node n t1 t2) = n + sumTree t1 + sumTree t2
    \end{lstlisting}
  \end{exampleblock}
\end{frame}

\begin{frame}[fragile]
  \frametitle{Polymorphic Algebraic Types}

  \begin{exampleblock}{example: polymorphic binary tree}
    \begin{lstlisting}
data BinTree a = Nil | Node a (BinTree a) (BinTree a)

depth :: BinTree a -> Integer
depth Nil            = 0
depth (Node _ t1 t2) = 1 + max (depth t1) (depth t2)

sumTree :: Num a => BinTree a -> a
sumTree Nil            = 0
sumTree (Node n t1 t2) = n + sumTree t1 + sumTree t2
    \end{lstlisting}
  \end{exampleblock}
\end{frame}

\subsection{Error Types}

\begin{frame}[fragile]
  \frametitle{The Maybe Type}

  \begin{itemize}
    \item returning an error value
  \end{itemize}

  \begin{lstlisting}
data Maybe a = Nothing | Just a
               deriving (Eq, Ord, Read, Show)
  \end{lstlisting}

  \begin{exampleblock}{example}
    \begin{lstlisting}
errDiv :: Integer -> Integer -> Maybe Integer
errDiv m n
  | n == 0    = Nothing
  | otherwise = Just (m `div` n)
    \end{lstlisting}
  \end{exampleblock}
\end{frame}

\begin{frame}[fragile]
  \frametitle{Maybe Example}

  \begin{exampleblock}{maximum of a list}
    \begin{lstlisting}
maxList :: Ord a => [a] -> Maybe a
maxList []     = Nothing
maxList (x:xs) = Just (foldl max x xs)
    \end{lstlisting}
  \end{exampleblock}
\end{frame}

\begin{frame}[fragile]
  \frametitle{The Maybe Type}

  \begin{itemize}
    \item transmitting an error through a function
  \end{itemize}

  \begin{exampleblock}{}
    \begin{lstlisting}
mapMaybe :: (a -> b) -> Maybe a -> Maybe b
mapMaybe g Nothing  = Nothing
mapMaybe g (Just x) = Just (g x)

mapMaybe length Nothing          -- Nothing
mapMaybe length (Just "haskell") -- Just 7
    \end{lstlisting}
  \end{exampleblock}
\end{frame}

\begin{frame}[fragile]
  \frametitle{The Maybe Type}

  \begin{itemize}
    \item trapping an error
  \end{itemize}

  \begin{exampleblock}{}
    \begin{lstlisting}
maybe :: b -> (a -> b) -> Maybe a -> b
maybe n f Nothing  = n
maybe n f (Just x) = f x

maybe 0 length Nothing          -- 0
maybe 0 length (Just "haskell") -- 7
    \end{lstlisting}
  \end{exampleblock}
\end{frame}

\begin{frame}[fragile]
  \frametitle{The Either Type}

  \begin{itemize}
    \item one of two types
  \end{itemize}

  \begin{lstlisting}
data Either a b = Left a | Right b
                  deriving (Eq, Ord, Read, Show)
  \end{lstlisting}

  \begin{exampleblock}{example}
    \begin{lstlisting}
isFalse :: Either Integer [a] -> Bool
isFalse (Left  0)  = True
isFalse (Right []) = True
isFalse _          = False
    \end{lstlisting}
  \end{exampleblock}
\end{frame}

\begin{frame}[fragile]
  \frametitle{Either Example}

  \begin{lstlisting}
either :: (a -> c) -> (b -> c) -> Either a b -> c
either f g (Left  x) = f x
either f g (Right y) = g y

either length abs (Left  "haskell") -- 7
either length abs (Right (-8))      -- 8
  \end{lstlisting}
\end{frame}

\begin{frame}[fragile]
  \frametitle{Either Example}

  \begin{lstlisting}
join f g (Left  x) = Left  (f x)
join f g (Right y) = Right (g y)
  \end{lstlisting}

  \medskip
  \begin{itemize}
    \item exercise: what is the type of \lstinline|join|?
    \item how can it be invoked?
    \item express \lstinline|join| in the form:\\
      \lstinline|join f g = either ___ ___|
  \end{itemize}
\end{frame}

\section*{References}

\begin{frame}
  \frametitle{References}

  \begin{block}{Required Reading: Thompson}
    \begin{itemize}
      \item Chapter 13: \alert{Overloading, type classes and type checking}
      \item Chapter 14: \alert{Algebraic types}
    \end{itemize}
  \end{block}
\end{frame}

\end{document}
