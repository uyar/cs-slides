% Copyright (c) 2013-2014 H. Turgut Uyar <uyar@itu.edu.tr>
%
% This work is licensed under a "Creative Commons
% Attribution-NonCommercial-ShareAlike 4.0 International License".
% For more information, please visit:
% https://creativecommons.org/licenses/by-nc-sa/4.0/

\documentclass[dvipsnames]{beamer}

\usepackage[scaled=0.95]{cabin}
\usepackage[scaled=0.88]{beramono}
\usepackage[T1]{fontenc}
\usepackage[utf8]{inputenc}

\usepackage{listings}
\lstset{basicstyle=\ttfamily,
        keywordstyle=\color{blue},
        showstringspaces=false}
\lstdefinestyle{syntax}{frame=single}
\lstset{language=Haskell}

\mode<presentation>
{
  \usetheme{default}
  \useinnertheme{rounded}
  \usecolortheme{seahorse}
  \setbeamercovered{transparent}
}

\title{Functional Programming}
\subtitle{Input/Output}

\author{H. Turgut Uyar}
\date{2013-2014}

\AtBeginSubsection[]{
  \begin{frame}<beamer>
    \frametitle{Topics}
    \tableofcontents[currentsection,currentsubsection]
  \end{frame}
}

\theoremstyle{plain}

\pgfdeclareimage[height=1cm]{license}{../license}

\begin{document}

\begin{frame}
  \titlepage
\end{frame}

\begin{frame}
  \frametitle{License}

  \pgfuseimage{license}\hfill
  \copyright~2013-2014 H. Turgut Uyar

  \vfill
  \begin{footnotesize}
    You are free to:
    \begin{itemize}
      \itemsep0em
      \item Share -- copy and redistribute the material in any medium or format
      \item Adapt -- remix, transform, and build upon the material
    \end{itemize}

    Under the following terms:
    \begin{itemize}
      \itemsep0em
      \item Attribution -- You must give appropriate credit, provide a link to
        the license, and indicate if changes were made.

      \item NonCommercial -- You may not use the material for commercial
        purposes.

      \item ShareAlike -- If you remix, transform, or build upon the material,
        you must distribute your contributions under the same license as the
        original.
    \end{itemize}

    For more information:\\
    \url{https://creativecommons.org/licenses/by-nc-sa/4.0/}

    \smallskip
    Read the full license:\\
    \url{https://creativecommons.org/licenses/by-nc-sa/4.0/legalcode}
  \end{footnotesize}
\end{frame}

\begin{frame}
  \frametitle{Topics}
  \tableofcontents
\end{frame}

\section{I/O Model}

\subsection{Introduction}

\begin{frame}
  \frametitle{I/O Model}

  \begin{itemize}
    \item how can I/O fit into the functional model?

    \bigskip
    \item how about a function that reads in a value of the desired type\\
      from the input?\\
      \lstinline|inputInt :: Integer|

    \pause
    \medskip
    \item breaks reasoning:\\
      \lstinline|inputDiff = inputInt - inputInt|

    \medskip
    \item any function might be affected:\\
      \lstinline|foo :: Integer -> Integer|\\
      \lstinline|foo n = inputInt + n|\\
  \end{itemize}
\end{frame}

\begin{frame}
  \frametitle{I/O Type}

  \begin{itemize}
    \item new type: \lstinline|IO a|\\
      a program which will do some I/O and return a value of type \texttt{a}

    \medskip
    \item instead of:\\
      \lstinline|inputInt :: Integer|
    \item we have:\\
      \lstinline|inputInt :: IO Integer|

    \pause
    \medskip
    \item no longer valid:\\
      \lstinline|inputInt - inputInt|\\
      \lstinline|inputInt + n|
  \end{itemize}
\end{frame}

\begin{frame}
  \frametitle{I/O Type}

  \begin{itemize}
    \item if I/O doesn't produce a result: \lstinline|IO ()|

    \medskip
    \item output:\\
      \lstinline|putStr :: String -> IO ()|\\
      \lstinline|putStrLn :: String -> IO ()|

    \smallskip
    \item input:\\
      \lstinline|getLine :: IO String|\\
  \end{itemize}
\end{frame}

\begin{frame}[fragile]
  \frametitle{Program Start}

  \begin{itemize}
    \item entry point of the program: \lstinline|main|
  \end{itemize}

  \medskip
  \begin{exampleblock}{example: Hello, world!}
    \begin{lstlisting}
main :: IO ()
main = putStrLn "Hello, world!"
    \end{lstlisting}
  \end{exampleblock}
\end{frame}

\subsection{String Conversions}

\begin{frame}[fragile]
  \frametitle{String Conversions}

  \begin{itemize}
    \item convert a type to string: \lstinline|show|
    \item convert a string to another type: \lstinline|read|
  \end{itemize}

  \begin{exampleblock}{examples}
    \begin{lstlisting}
show 42   ~> "42"
show 3.14 ~> "3.14"

read "42" :: Integer ~> 42
read "42" :: Float   ~> 42.0
read "3.14" :: Float ~> 3.14
    \end{lstlisting}
  \end{exampleblock}
\end{frame}

\subsection{Action Sequences}

\begin{frame}[fragile]
  \frametitle{Action Sequences}

  \begin{itemize}
    \item I/O consists of \alert{actions} happening in a sequence
    \item create an action sequence: \lstinline|do|
    \item small imperative programming language

    \medskip
    \begin{lstlisting}
do
    action1
    action2
    ...
    \end{lstlisting}
  \end{itemize}
\end{frame}

\begin{frame}[fragile]
  \frametitle{Sequence Example}

  \begin{exampleblock}{print a string 4 times}
    \begin{lstlisting}
put4times :: String -> IO ()
put4times str = do
    putStrLn str
    putStrLn str
    putStrLn str
    putStrLn str
    \end{lstlisting}
  \end{exampleblock}
\end{frame}

\begin{frame}[fragile]
  \frametitle{Capturing Values}

  \begin{itemize}
    \item capture value produced by the program: \lstinline|<-|
    \item can only be used within the sequence
  \end{itemize}

  \begin{exampleblock}{example: reverse and print the line read from the input}
    \begin{lstlisting}
reverseLine :: IO ()
reverseLine = do
    line <- getLine
    putStrLn (reverse line)
    \end{lstlisting}
  \end{exampleblock}
\end{frame}

\begin{frame}[fragile]
  \frametitle{Local Definitions}

  \begin{itemize}
    \item local definitions: \lstinline|let|
    \item can only be used within the sequence
  \end{itemize}

  \begin{exampleblock}{example: reverse two lines}
    \begin{lstlisting}
reverse2lines :: IO ()
reverse2lines = do
    line1 <- getLine
    line2 <- getLine
    let rev1 = reverse line1
    let rev2 = reverse line2
    putStrLn rev2
    putStrLn rev1
    \end{lstlisting}
  \end{exampleblock}
\end{frame}

\begin{frame}[fragile]
  \frametitle{Returning Values}

  \begin{itemize}
    \item returning result of sequence: \lstinline|return|
  \end{itemize}

  \begin{exampleblock}{example: read an integer from the input}
    \begin{lstlisting}
getInteger :: IO Integer
getInteger = do
    line <- getLine
    return (read line :: Integer)
    \end{lstlisting}
  \end{exampleblock}
\end{frame}

\begin{frame}[fragile]
  \frametitle{Recursion in Sequence}

  \begin{exampleblock}{copy input to output indefinitely}
    \begin{lstlisting}
copy :: IO ()
copy = do
    line <- getLine
    putStrLn line
    copy
    \end{lstlisting}
  \end{exampleblock}
\end{frame}

\begin{frame}[fragile]
  \frametitle{Conditional in Sequence}

  \begin{exampleblock}{copy input to output a number of times}
    \begin{lstlisting}
copyN :: Integer -> IO ()
copyN n =
    if n <= 0
        then return ()
        else do
            line <- getLine
            putStrLn line
            copyN (n - 1)
    \end{lstlisting}
  \end{exampleblock}
\end{frame}

\begin{frame}[fragile]
  \frametitle{Conditional in Sequence}

  \begin{exampleblock}{copy until input line is empty}
    \begin{lstlisting}
copyEmpty :: IO ()
copyEmpty = do
    line <- getLine
    if line == ""
        then return ()
        else do
            putStrLn line
            copyEmpty
    \end{lstlisting}
  \end{exampleblock}
\end{frame}

\section{Example: Rock - Paper - Scissors}

\subsection{Data Types}

\begin{frame}[fragile]
  \frametitle{Rock - Paper - Scissors}

  \begin{itemize}
    \item two players repeatedly play Rock-Paper-Scissors
  \end{itemize}

  \begin{exampleblock}{data types}
    \begin{lstlisting}
data Move = Rock | Paper | Scissors
            deriving Show

type Match = ([Move], [Move])
-- ex: ([Rock, Rock, Paper], [Scissors, Paper, Rock])
    \end{lstlisting}
  \end{exampleblock}
\end{frame}

\begin{frame}[fragile]
  \frametitle{Outcome}

  \begin{exampleblock}{outcome of one round}
    \begin{itemize}
      \item A wins $\mapsto$ 1, B wins $\mapsto$ -1, tied $\mapsto$ 0
    \end{itemize}
    \begin{lstlisting}
outcome :: Move -> Move -> Integer
outcome moveA moveB = case (moveA, moveB) of
    (Rock,     Scissors) ->  1
    (Scissors, Rock)     -> -1
    (Paper,    Rock)     ->  1
    (Rock,     Paper)    -> -1
    (Scissors, Paper)    ->  1
    (Paper,    Scissors) -> -1
    _                    ->  0
    \end{lstlisting}
  \end{exampleblock}

  \pause
  \vspace{-12pt}
  \begin{itemize}
    \item exercise: determine the outcome of a match\\
      \lstinline|matchOutcome [Rock, Paper] [Paper, Scissors] ~> -2|
  \end{itemize}
\end{frame}

\begin{frame}[fragile]
  \frametitle{String Conversions}

  \begin{exampleblock}{convert a round in the game to string}
    \begin{lstlisting}
showRound :: (Move, Move) -> String
showRound (moveA, moveB) = "A plays: " ++ show moveA
    ++ ", B plays: " ++ show moveB
    \end{lstlisting}
  \end{exampleblock}

  \pause
  \vspace{-12pt}
  \begin{itemize}
    \item exercise: convert match result to string\\
      \lstinline|showResult [Rock, Paper] [Paper, Scissors]|\\
      \lstinline|    ~> "Player B wins by 2"|

      \smallskip
      \lstinline|showResult [Rock, Paper] [Paper, Rock]|\\
      \lstinline|    ~> "It's a tie"|
  \end{itemize}
\end{frame}

\subsection{Strategies}

\begin{frame}[fragile]
  \frametitle{Strategies}

  \begin{itemize}
    \item a strategy is a function that selects a move\\
      based on the previous moves of the opponent:\\
      \lstinline|[Move] -> Move|
  \end{itemize}

  \begin{exampleblock}{always play the same move}
    \begin{lstlisting}
rock, paper, scissors :: [Move] -> Move
rock     _ = Rock
paper    _ = Paper
scissors _ = Scissors
    \end{lstlisting}
  \end{exampleblock}
\end{frame}

\begin{frame}[fragile]
  \frametitle{Strategies}

  \begin{exampleblock}{cycle through the possibilities}
    \begin{lstlisting}[deletekeywords={cycle}]
cycle :: [Move] -> Move
cycle moves = case (length moves) `mod` 3 of
    0 -> Rock
    1 -> Paper
    2 -> Scissors
    \end{lstlisting}
  \end{exampleblock}
\end{frame}

\begin{frame}[fragile]
  \frametitle{Strategies}

  \begin{exampleblock}{play whatever the opponent played last}
    \begin{lstlisting}
-- opponent moves kept in reverse order
echo :: [Move] -> Move
echo []            = Rock
echo (latest:rest) = latest
    \end{lstlisting}
  \end{exampleblock}
\end{frame}

\subsection{Game Play}

\begin{frame}[fragile]
  \frametitle{Game Play}

  \begin{itemize}
    \item play the game interactively
  \end{itemize}

  \begin{exampleblock}{convert a character into a move}
    \begin{lstlisting}
convertMove :: Char -> Move
convertMove c
  | c `elem` "rR" = Rock
  | c `elem` "pP" = Paper
  | c `elem` "sS" = Scissors
  | otherwise     = error "unknown move"
    \end{lstlisting}
  \end{exampleblock}
\end{frame}

\begin{frame}[fragile]
  \frametitle{Game Play}

  \begin{exampleblock}{play interactively against \lstinline|echo|}
    \begin{lstlisting}
play :: Match -> IO ()
play (movesA, movesB) = do
    ch <- getChar
    putStrLn ""
    if ch == '.'
        then putStrLn (showResult (movesA, movesB))
        else do
            let moveA = echo movesB
            let moveB = convertMove ch
            putStrLn (showRound (moveA, moveB))
            play (moveA:movesA, moveB:movesB)

playInteractive :: IO ()
playInteractive = play ([], [])
    \end{lstlisting}
  \end{exampleblock}
\end{frame}

\begin{frame}[fragile]
  \frametitle{Automatic Play}

  \begin{exampleblock}{generate a match:
      \lstinline[deletekeywords={cycle}]|cycle| versus \lstinline|echo|}
    \begin{lstlisting}[deletekeywords={cycle}]
generateMatch :: Integer -> Match
generateMatch 0 = ([], [])
generateMatch n = step (generateMatch (n - 1))
  where
    step :: Match -> Match
    step (movesA, movesB) = (cycle movesB : movesA,
                             echo movesA : movesB)
    \end{lstlisting}
  \end{exampleblock}
\end{frame}

\section*{References}

\begin{frame}
  \frametitle{References}

  \begin{block}{Required Reading: Thompson}
    \begin{itemize}
      \item Chapter 8: \alert{Playing the game: I/O in Haskell}
    \end{itemize}
  \end{block}
\end{frame}

\end{document}
