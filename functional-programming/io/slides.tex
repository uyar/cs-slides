% Copyright (c) 2013-2014 H. Turgut Uyar <uyar@itu.edu.tr>
%
% This work is licensed under a "Creative Commons
% Attribution-NonCommercial-ShareAlike 4.0 International License".
% For more information, please visit:
% https://creativecommons.org/licenses/by-nc-sa/4.0/

\documentclass[dvipsnames]{beamer}

\usepackage{ae}
\usepackage[T1]{fontenc}
\usepackage[utf8]{inputenc}
\usepackage{pythontex}
\setbeamertemplate{navigation symbols}{}
\setbeamersize{text margin left=2em, text margin right=2em}

\mode<presentation>
{
  \usetheme{default}
  \useinnertheme{rounded}
  \usecolortheme{seahorse}
  \setbeamercovered{transparent}
}

\title{Functional Programming}
\subtitle{Input/Output}

\author{H. Turgut Uyar}
\date{2013-2014}

\AtBeginSubsection[]{
  \begin{frame}<beamer>
    \frametitle{Topics}
    \tableofcontents[currentsection,currentsubsection]
  \end{frame}
}

\theoremstyle{plain}

\pgfdeclareimage[height=1cm]{license}{../license}

\begin{document}

\setpythontexfv[]{frame=single}

\begin{frame}
  \titlepage
\end{frame}

\begin{frame}
  \frametitle{License}

  \pgfuseimage{license}\hfill
  \copyright~2013-2014 H. Turgut Uyar

  \vfill
  \begin{tiny}
    You are free to:
    \begin{itemize}
      \item Share -- copy and redistribute the material in any medium or format
      \item Adapt -- remix, transform, and build upon the material
    \end{itemize}

    Under the following terms:
    \begin{itemize}
      \item Attribution -- You must give appropriate credit, provide a link to
        the license, and indicate if changes were made.\\
        You may do so in any reasonable manner, but not in any way
        that suggests the licensor endorses you or your use.

      \item Noncommercial -- You may not use the material for commercial
        purposes.

      \item Share Alike -- If you remix, transform, or build upon the material,
        you must distribute your contributions\\
        under the same license as the original.
    \end{itemize}
  \end{tiny}

  \vfill
  \begin{small}
    Legal code (the full license):\\
    \url{https://creativecommons.org/licenses/by-nc-sa/4.0/legalcode}
  \end{small}
\end{frame}

\begin{frame}
  \frametitle{Topics}
  \tableofcontents
\end{frame}

\section{I/O Model}

\subsection{The Problem}

\begin{frame}
  \frametitle{I/O Model}

  \begin{itemize}
    \item how can I/O fit into the functional model?

    \bigskip
    \item how about a function that reads in a value of the desired type\\
      from the input?\\
      \smallskip
      \pygment{haskell}{inputInt :: Integer}

    \pause
    \medskip
    \item breaks reasoning:\\
      \smallskip
      \pygment{haskell}{inputDiff = inputInt - inputInt}
    \item any function might be affected:\\
      \smallskip
      \pygment{haskell}{foo :: Integer -> Integer}\\
      \pygment{haskell}{foo n = inputInt + n}\\
  \end{itemize}
\end{frame}

\begin{frame}
  \frametitle{I/O Type}

  \begin{itemize}
    \item new type: \pygment{haskell}{IO a}\\
      a program which will do some I/O and return a value of type \texttt{a}

    \medskip
    \item instead of:\\
      \smallskip
      \pygment{haskell}{inputInt :: Integer}
    \item we have:\\
      \smallskip
      \pygment{haskell}{inputInt :: IO Integer}

    \pause
    \medskip
    \item no longer valid:\\
      \texttt{inputInt - inputInt}\\
      \texttt{inputInt + n}
  \end{itemize}
\end{frame}

\begin{frame}
  \frametitle{I/O Type}

  \begin{itemize}
    \item if the result of the I/O is not significant: \pygment{haskell}{IO ()}

    \medskip
    \item output:\\
      \smallskip
      \pygment{haskell}{putStr :: String -> IO ()}\\
      \pygment{haskell}{putStrLn :: String -> IO ()}

    \smallskip
    \item input:\\
      \smallskip
      \pygment{haskell}{getLine :: IO String}\\
  \end{itemize}
\end{frame}

\begin{frame}[fragile]
  \frametitle{Program Start}

  \begin{itemize}
    \item entry point of the program: \pygment{haskell}{main}
  \end{itemize}

  \begin{exampleblock}{Hello, world!}
    \begin{pygments}{haskell}
main :: IO ()
main = putStrLn "Hello, world!"
    \end{pygments}
  \end{exampleblock}
\end{frame}

\subsection{String Conversions}

\begin{frame}[fragile]
  \frametitle{String Conversions}

  \begin{itemize}
    \item to convert a type to string: \pygment{haskell}{show}
    \item to convert a string to another type: \pygment{haskell}{read}
  \end{itemize}

  \begin{exampleblock}{}
    \begin{pygments}{haskell}
show 42    -- "42"
show 3.14  -- "3.14"

read "42" :: Integer  -- 42
read "42" :: Float    -- 42.0
read "3.14" :: Float  -- 3.14
    \end{pygments}
  \end{exampleblock}
\end{frame}

\subsection{Action Sequences}

\begin{frame}[fragile]
  \frametitle{Action Sequences}

  \begin{itemize}
    \item I/O consists of \alert{actions} happening in a sequence
    \item to create an action sequence: \pygment{haskell}{do}
    \item small imperative programming language
  \end{itemize}

  \begin{block}{}
    \begin{pygments}{haskell}
do
    action1
    action2
    ...
    \end{pygments}
  \end{block}
\end{frame}

\begin{frame}[fragile]
  \frametitle{Sequence Example}

  \begin{exampleblock}{print a string 4 times}
    \begin{pygments}{haskell}
put4times :: String -> IO ()
put4times str = do
    putStrLn str
    putStrLn str
    putStrLn str
    putStrLn str
    \end{pygments}
  \end{exampleblock}
\end{frame}

\begin{frame}[fragile]
  \frametitle{Capturing Values}

  \begin{itemize}
    \item capturing value read from input: \pygment{haskell}{<-}
    \item can only be used within \pygment{haskell}{do}
  \end{itemize}

  \begin{exampleblock}{reverse and print the line read from the input}
    \begin{pygments}{haskell}
reverseLine :: IO ()
reverseLine = do
    line <- getLine
    putStrLn (reverse line)
    \end{pygments}
  \end{exampleblock}
\end{frame}

\begin{frame}[fragile]
  \frametitle{Local Definitions}

  \begin{itemize}
    \item local definitions: \pygment{haskell}{let}
    \item can only be used within \pygment{haskell}{do}
  \end{itemize}

  \begin{exampleblock}{}
    \begin{pygments}{haskell}
reverse2lines :: IO ()
reverse2lines = do
    line1 <- getLine
    line2 <- getLine
    let rev1 = reverse line1
    let rev2 = reverse line2
    putStrLn rev2
    putStrLn rev1
    \end{pygments}
  \end{exampleblock}
\end{frame}

\begin{frame}[fragile]
  \frametitle{Returning Values}

  \begin{itemize}
    \item returning result of sequence: \pygment{haskell}{return}
  \end{itemize}

  \begin{exampleblock}{}
    \begin{pygments}{haskell}
getInteger :: IO Integer
getInteger = do
    line <- getLine
    return (read line :: Integer)
    \end{pygments}
  \end{exampleblock}
\end{frame}

\begin{frame}[fragile]
  \frametitle{Recursion in Sequence}

  \begin{exampleblock}{copy input to output indefinitely}
    \begin{pygments}{haskell}
copy :: IO ()
copy = do
    line <- getLine
    putStrLn line
    copy
    \end{pygments}
  \end{exampleblock}
\end{frame}

\begin{frame}[fragile]
  \frametitle{Conditional in Sequence}

  \begin{exampleblock}{copy n times}
    \begin{pygments}{haskell}
copyN :: Integer -> IO ()
copyN n =
    if n <= 0
        then return ()
        else do
            line <- getLine
            putStrLn line
            copyN (n - 1)
    \end{pygments}
  \end{exampleblock}
\end{frame}

\begin{frame}[fragile]
  \frametitle{Conditional in Sequence}

  \begin{exampleblock}{copy until input line is empty}
    \begin{pygments}{haskell}
copyEmpty :: IO ()
copyEmpty = do
    line <- getLine
    if line == ""
        then return ()
        else do
            putStrLn line
            copyEmpty
    \end{pygments}
  \end{exampleblock}
\end{frame}

\section{Example: Rock - Paper - Scissors}

\subsection{Data Types}

\begin{frame}[fragile]
  \frametitle{Data Types}

  \begin{itemize}
    \item two players repeatedly play Rock - Paper - Scissors
  \end{itemize}

  \begin{exampleblock}{}
    \begin{pygments}{haskell}
data Move = Rock | Paper | Scissors
            deriving Show

type Match = ([Move], [Move])

-- ex: ([Rock, Rock, Paper], [Scissors, Paper, Rock])
    \end{pygments}
  \end{exampleblock}
\end{frame}

\begin{frame}[fragile]
  \frametitle{Rock - Paper - Scissors}

  \begin{itemize}
    \item a function that will determine the outcome of one round:\\
      $1$ if player A wins, $-1$ if player B wins, $0$ if tied
  \end{itemize}

  \begin{exampleblock}{}
    \begin{pygments}{haskell}
outcome :: Move -> Move -> Integer
outcome moveA moveB =
    case (moveA, moveB) of
      (Rock, Scissors)  -> 1
      (Scissors, Paper) -> 1
      (Paper, Rock)     -> 1
      (Rock, Paper)     -> -1
      (Paper, Scissors) -> -1
      (Scissors, Rock)  -> -1
      _                 -> 0
    \end{pygments}
  \end{exampleblock}

  \pause
  \begin{itemize}
    \item exercise: write a function to determine the outcome of a match
  \end{itemize}
\end{frame}

\begin{frame}<beamer>[fragile]
  \frametitle{Rock - Paper - Scissors}

  \begin{exampleblock}{outcome of a match}
    \begin{pygments}{haskell}
matchOutcome :: Match -> Integer
matchOutcome ([], []) = 0
matchOutcome (moveA:movesA, moveB:movesB) =
    outcome moveA moveB + matchOutcome (movesA, movesB)
    \end{pygments}
  \end{exampleblock}
\end{frame}

\subsection{Strategies}

\begin{frame}[fragile]
  \frametitle{Rock - Paper - Scissors}

  \begin{itemize}
    \item a strategy is a function that selects a move\\
      based on the previous moves of the opponent:\\
      \smallskip
      \pygment{haskell}{[Move] -> Move}
  \end{itemize}

  \pause
  \begin{exampleblock}{cycle through the possibilities}
    \begin{pygments}{haskell}
cycle' :: [Move] -> Move
cycle' moves =
    case (length moves) `mod` 3 of
      0 -> Rock
      1 -> Paper
      2 -> Scissors
    \end{pygments}
  \end{exampleblock}
\end{frame}

\begin{frame}[fragile]
  \frametitle{Rock - Paper - Scissors}

  \begin{itemize}
    \item strategy: play whatever the opponent played last
    \item keep the previous moves in reverse order (latest first)
  \end{itemize}

  \begin{exampleblock}{}
    \begin{pygments}{haskell}
echo :: [Move] -> Move
echo [] = Rock
echo (latest:rest) = latest
    \end{pygments}
  \end{exampleblock}

  \pause
  \begin{itemize}
    \item exercise: write the strategies where the player\\
      always plays the same move
  \end{itemize}
\end{frame}

\subsection{Game Play}

\begin{frame}[fragile]
  \frametitle{Rock - Paper - Scissors}

  \begin{itemize}
    \item write a function that will let the user play the game interactively
    \item let the computer play the ``echo'' strategy
  \end{itemize}

  \pause
  \begin{exampleblock}{}
    \begin{pygments}{haskell}
playInteractive :: Match -> IO ()
playInteractive (movesA, movesB) = do
    ch <- getChar
    if not (ch `elem` "rpsRPS")
        then putStrLn (showResult (movesA, movesB))
        else do
            let moveA = echo movesB
            let moveB = convertMove ch
            putStrLn ("\nA plays: " ++ show moveA ++
                      ", B plays: " ++ show moveB)
            playInteractive (moveA : movesA,
                             moveB : movesB)
    \end{pygments}
  \end{exampleblock}
\end{frame}

\begin{frame}
  \frametitle{Rock - Paper - Scissors}

  \begin{itemize}
    \item write a function to represent the outcome of a match:\\
      \smallskip
      \pygment{haskell}{showResult :: Match -> String}
    \item write a function to convert a character into a move:\\
      \smallskip
      \pygment{haskell}{convertMove :: Char -> Move}
  \end{itemize}
\end{frame}

\begin{frame}[fragile]
  \frametitle{Rock - Paper - Scissors}

  \begin{itemize}
    \item write a function that will let computer play the ``cycle'' strategy\\
      versus the ``echo'' strategy for a number of times
  \end{itemize}

  \pause
  \begin{exampleblock}{}
    \begin{pygments}{haskell}
generateMatchCycleVsEcho :: Integer -> Match
generateMatchCycleVsEcho 0 = ([], [])
generateMatchCycleVsEcho n =
    step (generateMatchCycleVsEcho (n - 1))
      where
        step :: Match -> Match
        step (movesA, movesB) =
            (cycle' movesB : movesA,
             echo movesA : movesB)
    \end{pygments}
  \end{exampleblock}
\end{frame}

\begin{frame}
  \frametitle{Rock - Paper - Scissors}

  \begin{itemize}
    \item write a function to represent all moves in a match:\\
      \smallskip
      \pygment{haskell}{showMoves :: Match -> String}
    \item write a function to represent the moves and result of a match:\\
      \smallskip
      \pygment{haskell}{showMatch :: Match -> String}
  \end{itemize}
\end{frame}

\section*{References}

\begin{frame}
  \frametitle{References}

  \begin{block}{Required Reading: Thompson}
    \begin{itemize}
      \item Chapter 8: \alert{Playing the game: I/O in Haskell}
    \end{itemize}
  \end{block}
\end{frame}

\end{document}
