% Copyright (c) 2015-2016 H. Turgut Uyar <uyar@itu.edu.tr>
%
% This work is licensed under a "Creative Commons
% Attribution-NonCommercial-ShareAlike 4.0 International License".
% For more information, please visit:
% https://creativecommons.org/licenses/by-nc-sa/4.0/

\documentclass[dvipsnames]{beamer}

\usepackage[scaled=0.95]{cabin}
\usepackage[scaled=0.88]{beramono}
\usepackage[T1]{fontenc}
\usepackage[utf8]{inputenc}

\usepackage{listings}
\lstset{basicstyle=\ttfamily,
        keywordstyle=\color{blue},
        showstringspaces=false}
\lstdefinestyle{syntax}{frame=single}
\lstset{language=Haskell,morekeywords={Applicative,fmap,pure}}

\mode<presentation>
{
  \usetheme{default}
  \useinnertheme{rounded}
  \usecolortheme{seahorse}
  \setbeamercovered{transparent}
}

\title{Functional Programming}
\subtitle{Monads}

\author{H. Turgut Uyar}
\date{2015-2016}

\AtBeginSubsection[]{
  \begin{frame}<beamer>
    \frametitle{Topics}
    \tableofcontents[currentsection,currentsubsection]
  \end{frame}
}

\theoremstyle{plain}

\pgfdeclareimage[height=1cm]{license}{../license}

\begin{document}

\begin{frame}
  \titlepage
\end{frame}

\begin{frame}
  \frametitle{License}

  \pgfuseimage{license}\hfill
  \copyright~2015-2016 H. Turgut Uyar

  \vfill
  \begin{footnotesize}
    You are free to:
    \begin{itemize}
      \itemsep0em
      \item Share -- copy and redistribute the material in any medium or format
      \item Adapt -- remix, transform, and build upon the material
    \end{itemize}

    Under the following terms:
    \begin{itemize}
      \itemsep0em
      \item Attribution -- You must give appropriate credit, provide a link to
        the license, and indicate if changes were made.

      \item NonCommercial -- You may not use the material for commercial
        purposes.

      \item ShareAlike -- If you remix, transform, or build upon the material,
        you must distribute your contributions under the same license as the
        original.
    \end{itemize}

    For more information:\\
    \url{https://creativecommons.org/licenses/by-nc-sa/4.0/}

    \smallskip
    Read the full license:\\
    \url{https://creativecommons.org/licenses/by-nc-sa/4.0/legalcode}
  \end{footnotesize}
\end{frame}

\begin{frame}
  \frametitle{Topics}
  \tableofcontents
\end{frame}

\section{Introduction}

\subsection{Functors}

\begin{frame}[fragile]
  \frametitle{Function Composition}

  \begin{exampleblock}{example}
    \begin{lstlisting}
-- show :: a -> String
--         a -> [Char]
-- length :: [a] -> Int

-- length . show $ 42   ~> 2
    \end{lstlisting}
  \end{exampleblock}
\end{frame}

\begin{frame}[fragile]
  \frametitle{Composition with IO}

  \begin{exampleblock}{example}
    \begin{lstlisting}
-- getLine :: IO String
--            IO [Char]
-- length :: [a] -> Int

-- length . getLine     ~> type error
    \end{lstlisting}
  \end{exampleblock}
\end{frame}

\begin{frame}[fragile]
  \frametitle{Composition with IO}

  \begin{exampleblock}{}
    \begin{lstlisting}
-- fmapIO length getLine

fmapIO f p = do x <- p
                return (f x)
    \end{lstlisting}

    \pause
    \medskip
    \begin{itemize}
      \item what is the type of \lstinline|fmapIO|?\\
        \lstinline|fmapIO :: ([a] -> Int) -> IO [Char] -> IO Int|
      \pause
      \smallskip
      \item more general:\\
        \lstinline|fmapIO :: (a -> b) -> IO a -> IO b|
    \end{itemize}
  \end{exampleblock}
\end{frame}

\begin{frame}[fragile]
  \frametitle{Type Examples}

  \begin{lstlisting}
data Box a = Box a
             deriving Show


data Maybe a = Nothing | Just a
               deriving Show
  \end{lstlisting}
\end{frame}

\begin{frame}[fragile]
  \frametitle{Functor Class}

  \begin{itemize}
    \item extract value, apply function, wrap result
    \item \alert{functor}
  \end{itemize}

  \begin{lstlisting}
class Functor f where
  fmap :: (a -> b) -> f a -> f b
  \end{lstlisting}

  \begin{itemize}
    \item infix \lstinline|fmap|: \lstinline|<$>|
  \end{itemize}
\end{frame}

\begin{frame}[fragile]
  \frametitle{Functor Example}

  \begin{lstlisting}
instance Functor Box where
  fmap f (Box x) = Box (f x)

-- (+3) (Box 2)         ~> type error
-- fmap (+3) (Box 2)    ~> Box 5
-- (+3) <$> Box 2       ~> Box 5
  \end{lstlisting}
\end{frame}

\begin{frame}[fragile]
  \frametitle{Functor Example}

  \begin{lstlisting}
instance Functor Maybe where
  fmap f Nothing  = Nothing
  fmap f (Just x) = Just (f x)

-- (+3) <$> Just 2      ~> Just 5
-- (+3) <$> Nothing     ~> Nothing
  \end{lstlisting}
\end{frame}

\begin{frame}
  \frametitle{Functor Laws}

  \begin{itemize}
    \item \lstinline|fmap id == id|

    \medskip
    \item \lstinline|fmap (f . g) == fmap f . fmap g|
  \end{itemize}
\end{frame}

\begin{frame}
  \frametitle{Functor Law Examples}

  \begin{itemize}
    \item \lstinline|fmap id (Box x)|\pause\\
      \lstinline|== Box (id x)|\pause\\
      \lstinline|== Box x|\pause\\
      \lstinline|== id (Box x)|

    \pause
    \bigskip
    \item \lstinline|fmap (f . g) (Box x)|\pause\\
      \lstinline|== Box ((f . g) x)|\pause\\
      \lstinline|== Box (f (g x))|
    \pause
    \item \lstinline|(fmap f . fmap g) (Box x)|\pause\\
      \lstinline|== (fmap f) (fmap g (Box x)|\pause\\
      \lstinline|== (fmap f) (Box (g x))|\pause\\
      \lstinline|== Box (f (g x))|
  \end{itemize}
\end{frame}

\begin{frame}[fragile]
  \frametitle{Lists as Functors}

  \begin{itemize}
    \item \lstinline|fmap :: (a -> b) -> f a -> f b|
    \item replace \lstinline|f| with \lstinline|[]|:\\
      \lstinline|fmap :: (a -> b) -> [a] -> [b]|
  \end{itemize}

  \pause
  \begin{lstlisting}
instance Functor [] where
  fmap = map
  \end{lstlisting}
\end{frame}

\begin{frame}[fragile]
  \frametitle{Functions as Functors}

  \begin{itemize}
    \item \lstinline|fmap :: (a -> b) -> f a -> f b|
    \item replace \lstinline|f| with \lstinline|(->) r|:\\
      \lstinline|fmap :: (a -> b) -> ((->) r) a -> ((->) r) b|
    \item same as:\\
      \lstinline|fmap :: (a -> b) -> (r -> a) -> (r -> b)|
  \end{itemize}

  \pause
  \begin{lstlisting}
instance Functor ((->) r) where
  fmap = (.)
  \end{lstlisting}
\end{frame}

\subsection{Applicative Functors}

\begin{frame}[fragile]
  \frametitle{Applicative Functors}

  \begin{itemize}
    \item extract function, extract value, apply, wrap result
    \item \alert{applicative functor}
  \end{itemize}

  \begin{lstlisting}
class Functor f => Applicative f where
  (<*>) :: f (a -> b) -> f a -> f b
  pure :: a -> f a
  \end{lstlisting}
\end{frame}

\begin{frame}[fragile]
  \frametitle{Applicative Functor Example}

  \begin{lstlisting}
instance Applicative Box where
  (Box f) <*> (Box x) = Box (f x)

  pure x = Box x

-- Box (+3) <$> Box 2   ~> type error
-- Box (+3) <*> Box 2   ~> Box 5
  \end{lstlisting}
\end{frame}

\begin{frame}[fragile]
  \frametitle{Applicative Functor Example}

  \begin{lstlisting}
instance Applicative Maybe where
  Nothing <*> _ = Nothing
  Just f  <*> v = fmap f v

  pure x = Just x

-- Just (+3) <*> Just 2 ~> Just 5
-- Nothing   <*> Just 2 ~> Nothing
  \end{lstlisting}
\end{frame}

\begin{frame}[fragile]
  \frametitle{Applicative Functor Example}

  \begin{itemize}
    \item how to add two \lstinline|Maybe| values?
  \end{itemize}

  \pause
  \medskip
  \begin{lstlisting}
addMaybe :: Num a => Maybe a -> Maybe a -> Maybe a
addMaybe (Just x1) (Just x2) = Just (x1 + x2)
addMaybe _        _          = Nothing
  \end{lstlisting}

  \pause
  \begin{lstlisting}
-- OR:
addMaybe v1 v2 = (+) <$> v1 <*> v2
  \end{lstlisting}
\end{frame}

\begin{frame}
  \frametitle{Applicative Functor Laws}

  \begin{itemize}
    \item identity:\\
      \lstinline|pure id <*> v == v|

    \medskip
    \item composition:\\
      \lstinline|pure (.) <*> u <*> v <*> w == u <*> (v <*> w)|

    \medskip
    \item homomorphism:\\
      \lstinline|pure f <*> pure x == pure (f x)|

    \medskip
    \item interchange:\\
      \lstinline|u <*> pure y == pure ($ y) <*> u|

    \pause
    \medskip
    \item as a consequence:\\
      \lstinline|f <$> x == pure f <*> x|
  \end{itemize}
\end{frame}

\subsection{Monads}

\begin{frame}[fragile]
  \frametitle{Composing with IO}

  \begin{exampleblock}{example}
    \begin{lstlisting}
-- getLine :: IO String
-- readFile :: String -> IO String

-- readFile . getLine   ~> type error
    \end{lstlisting}
  \end{exampleblock}
\end{frame}

\begin{frame}[fragile]
  \frametitle{IO Sequencing}

  \begin{itemize}
    \item sequence I/O operations
  \end{itemize}

  \begin{exampleblock}{}
    \begin{lstlisting}
-- bindIO getLine readFile

bindIO p q = do x <- p
                q x
    \end{lstlisting}

    \pause
    \medskip
    \begin{itemize}
      \item what is the type of \lstinline|bindIO|?\\
        \lstinline|bindIO :: IO String -> (String -> IO String)|\\
        \lstinline|                           -> IO String|
      \pause
      \smallskip
      \item more general:\\
        \lstinline|bindIO :: IO a -> (a -> IO b) -> IO b|
    \end{itemize}
  \end{exampleblock}
\end{frame}

\begin{frame}[fragile]
  \frametitle{Monads}

  \begin{itemize}
    \item pattern: extract value, apply function
    \item \alert{monad}
  \end{itemize}

  \begin{lstlisting}
class Monad m where
  (>>=) :: m a -> (a -> m b) -> m b
  return :: a -> m a
  \end{lstlisting}
\end{frame}

\begin{frame}[fragile]
  \frametitle{IO Monad}

  \begin{exampleblock}{example}
    \begin{lstlisting}
-- getLine :: IO String
-- readFile :: String -> IO String

readFileInteractive :: IO String
readFileInteractive = getLine >>= readFile


-- putStrLn :: String -> IO ()

catFileInteractive :: IO ()
catFileInteractive = getLine >>= readFile >>= putStrLn
    \end{lstlisting}
  \end{exampleblock}
\end{frame}

\begin{frame}[fragile]
  \frametitle{Monads}

  \begin{itemize}
    \item syntactic sugar for monads: \lstinline|do|
  \end{itemize}

  \begin{exampleblock}{example}
    \begin{lstlisting}
catUserFile' = do filename <- getLine
                  content <- readFile filename
                  putStrLn content
    \end{lstlisting}
  \end{exampleblock}
\end{frame}

\begin{frame}[fragile]
  \frametitle{Monad Example}

  \begin{lstlisting}
userAge = getLine >>=
    (\l -> putStrLn $ show $ 2016 - (read l :: Int))


-- OR:
userAge = do line <- getLine
             let age = 2016 - (read line :: Int)
             putStrLn $ show age
  \end{lstlisting}
\end{frame}

\begin{frame}[fragile]
  \frametitle{Monad Example}

  \begin{lstlisting}
instance Monad Box where
  Box x >>= f = f x

  return x = Box x

-- Box 18 >>= (\x -> Box (x `div` 2))      ~> Box 9
-- Box 18 >>= (\x -> return (x `div` 2))   ~> Box 9
  \end{lstlisting}
\end{frame}

\begin{frame}[fragile]
  \frametitle{Monad Example}

  \begin{lstlisting}
instance Monad Maybe where
  Nothing >>= f = Nothing
  Just x  >>= f = f x

  return x = Just x

-- Just 18 >>= (\x -> return (x `div` 2))   ~> Just 9
  \end{lstlisting}
\end{frame}

\begin{frame}[fragile]
  \frametitle{Monad Example}

  \begin{lstlisting}
half :: Monad m => Integer -> m Integer
half x = return (x `div` 2)

-- Box 18 >>= half              ~> Box 9
-- Just 18 >>= half             ~> Just 9

-- Just 18 >>= half >>= half    ~> Just 4
-- half <=< half $ Just 18      ~> Just 4
  \end{lstlisting}
\end{frame}

\section*{References}

\begin{frame}
  \frametitle{References}

  \begin{block}{Required Reading: Thompson}
    \begin{itemize}
      \item Chapter 18: \alert{Monads}
    \end{itemize}
  \end{block}
\end{frame}

\end{document}
