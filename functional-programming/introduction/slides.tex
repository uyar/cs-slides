% Copyright (c) 2013
%       H. Turgut Uyar <uyar@itu.edu.tr>
%
% These notes are licensed using the
% "Creative Commons Attribution-NonCommercial-ShareAlike License".
% You are free to copy, distribute and transmit the work, and to adapt the work
% as long as you attribute the authors, do not use it for commercial purposes,
% and any derivative work is under the same or a similar license.
%
% Read the full legal code at:
% http://creativecommons.org/licenses/by-nc-sa/3.0/

\documentclass[dvipsnames]{beamer}

\usepackage{ae}
\usepackage[T1]{fontenc}
\usepackage[utf8]{inputenc}
\usepackage{pythontex}
\setbeamertemplate{navigation symbols}{}

\mode<presentation>
{
  \usetheme{Boadilla}
  \setbeamercovered{transparent}
}

\title{Functional Programming}
\subtitle{Introduction}

\author{H. Turgut Uyar}
\date{2013}

\AtBeginSubsection[]{
  \begin{frame}<beamer>
    \frametitle{Topics}
    \tableofcontents[currentsection,currentsubsection]
  \end{frame}
}

\theoremstyle{plain}

\pgfdeclareimage[width=2cm]{license}{../../license}

\pgfdeclareimage[width=4cm]{turing}{turing}
\pgfdeclareimage[width=4cm]{church}{church}
\pgfdeclareimage[width=4cm]{backus}{backus}
\pgfdeclareimage[width=4cm]{mccarthy}{mccarthy}

\begin{document}

\begin{frame}
  \titlepage
\end{frame}

\begin{frame}
  \frametitle{License}

  \pgfuseimage{license}\hfill
  \copyright 2013 T. Uyar

  \vfill
  \begin{tiny}
    You are free:
    \begin{itemize}
      \item to Share -- to copy, distribute and transmit the work
      \item to Remix -- to adapt the work
    \end{itemize}

    Under the following conditions:
    \begin{itemize}
      \item Attribution -- You must attribute the work in the manner specified by
        the author or licensor (but not in any way that suggests that they
        endorse you or your use of the work).

      \item Noncommercial -- You may not use this work for commercial purposes.

      \item Share Alike -- If you alter, transform, or build upon this work, you
        may distribute the resulting work only under the same or similar license
        to this one.
    \end{itemize}
  \end{tiny}

  \vfill
  Legal code (the full license):\\
  \url{http://creativecommons.org/licenses/by-nc-sa/3.0/}
\end{frame}

\begin{frame}
  \frametitle{Topics}
  \tableofcontents
\end{frame}

\section{Paradigms}

\subsection{Programming Languages}

\begin{frame}
  \frametitle{Syntax and Semantics}

  \begin{itemize}
    \item \alert{syntax}: rules for writing a ``grammatically correct'' program
    \begin{itemize}
      \item how expressions, commands, declarations and other constructs\\
        must be arranged to make a well-formed program
    \end{itemize}

    \bigskip
    \item \alert{semantics}: how the program should be ``interpreted''
    \begin{itemize}
      \item how the program may be expected to behave\\
        when executed on a computer
    \end{itemize}
  \end{itemize}
\end{frame}

\begin{frame}
  \frametitle{Beyond Syntax and Semantics}

  \begin{itemize}
    \item mastering a programming language is not just knowing its syntax
      and understanding its semantics

    \bigskip
    \item standard library
    \item other libraries
    \item tools: debugging, testing, profiling
    \item documenting
    \item style: code formatting, identifiers, \ldots

    \pause
    \medskip
    \item idioms: patterns for using language features
  \end{itemize}
\end{frame}

\begin{frame}
  \frametitle{Universality}

  \begin{itemize}
    \item a programming language is \alert{universal}\\
      if it is capable of expressing any computation
  \end{itemize}

  \begin{block}{Church-Turing Thesis (from Wikipedia)}
    If some method (algorithm) exists to carry out a calculation,\\
    then the same calculation can also be carried out by a Turing machine\\
    (as well as by a recursively definable function).
  \end{block}
  
  \begin{itemize}
    \item any programming language that supports iteration or recursion\\
      is universal
  \end{itemize}
\end{frame}

\begin{frame}
  \frametitle{Programming Domains}

  \begin{itemize}
    \item no programming language is well suited to all kinds of applications

    \bigskip
    \item scientific applications: computation intensive
    \begin{itemize}
      \item floating-point arithmetics, arrays and matrices
    \end{itemize}
    
    \item business applications: input/output intensive
    \begin{itemize}
      \item text, file and database operations, strings, dates, decimals
    \end{itemize}
    
    \item artificial intelligence
    \begin{itemize}
      \item symbolic computations, lists, trees
    \end{itemize}
      
    \item systems programming
    \begin{itemize}
      \item speed, availability, low-level features
    \end{itemize}
      
    \item web applications
    \begin{itemize}
      \item security, reliability, scalability, easy deployment
    \end{itemize}
      
    \item scripting: glueing applications (probably written in different PLs)
    \begin{itemize}
      \item file and text processing, regular expressions
    \end{itemize}
  \end{itemize}
\end{frame}

\begin{frame}
  \frametitle{Evaluating Programming Languages}

  \begin{itemize}
    \item \alert{readability}: easy to understand written code
    \begin{itemize}
      \item ease of maintenance
    \end{itemize}

    \pause
    \medskip
    \item \alert{expressivity}: less lines of code
    \begin{itemize}
      \item increases programmer productivity
      \item if not sacrificing readability
    \end{itemize}

    \pause
    \medskip
    \item \alert{orthogonality}: combining features consistently
    \begin{itemize}
      \item small set of independent features
    \end{itemize}

    \pause
    \medskip
    \item \alert{reliability}
    \begin{itemize}
      \item type checking, safe memory management, exception handling
    \end{itemize}
  \end{itemize}
\end{frame}

\begin{frame}
  \frametitle{Paradigms}

  \begin{itemize}
    \item programming paradigm: style of programming
    \item thought patterns
  \end{itemize}
\end{frame}

\subsection{Imperative Paradigm}

\begin{frame}
  \frametitle{Imperative Programming}

  \begin{columns}
    \column{.4\textwidth}
    \begin{center}
      \pgfuseimage{turing}\\
      Alan Turing (1912-1954)
    \end{center}

    \column{.6\textwidth}
    \begin{itemize}
      \item based on the Turing machine
      \item a program is a list of instructions\\
        for a von Neumann computer
      \item contents of memory constitutes\\
        the program \alert{state}

      \pause
      \medskip
      \item statements update variables
        (\alert{mutation})
      \item assignment and control structures

      \pause
      \medskip
      \item natural model of hardware
    \end{itemize}
  \end{columns}
\end{frame}

\begin{frame}[fragile]
  \frametitle{Example: Greatest Common Divisor}

  \begin{example}[imperative style]
    \begin{pygments}[]{c}
int gcd(int x, int y)
{
    int r = 0;

    while (y > 0)
    {
        r = x % y;
        x = y;
        y = r;
    }
    return x;
}
    \end{pygments}
  \end{example}
\end{frame}

\begin{frame}
  \frametitle{Milestones in Imperative Programming Languages}

  \begin{columns}
    \column{.45\textwidth}
    \begin{center}
      \pgfuseimage{backus}\\
      John Backus (1924-2007)
    \end{center}

    \column{.55\textwidth}
    \begin{itemize}
      \item Fortran (1957)
      \item ALGOL (1960)
      \item Ada (1983)
      \item Java (1995)
    \end{itemize}
  \end{columns}
\end{frame}

\subsection{Functional Paradigm}

\begin{frame}
  \frametitle{Functional Programming}

  \begin{columns}
    \column{.4\textwidth}
    \begin{center}
      \pgfuseimage{church}\\
      Alonzo Church (1903-1995)
    \end{center}

    \column{.6\textwidth}
    \begin{itemize}
      \item based on $\lambda$-calculus by Alonzo Church
      \item a program is a function application
      \item same inputs should produce\\
        same output

      \pause
      \medskip
      \item the function also changes
        the context $\rightarrow$ \emph{side effect}
      \item \alert{avoid mutation}

      \pause
      \medskip
      \item functions as arguments
    \end{itemize}
  \end{columns}
\end{frame}

\begin{frame}[fragile]
  \frametitle{Example: Greatest Common Divisor}

  \begin{example}[functional style]
    \begin{pygments}[]{c}
# note that variables are not updated

int gcd(int x, int y)
{
    if (y == 0)
        return x;
    else
        return gcd(y, x % y);
}
    \end{pygments}
  \end{example}
\end{frame}

\begin{frame}[fragile]
  \frametitle{Sources of Side Effects}

  \begin{itemize}
    \item global variables
  \end{itemize}

  \begin{example}
    \begin{pygments}[]{c}
int x = 0;

int foo(int n)
{
    x++;
    return x * n;
}
    \end{pygments}
  \end{example}
\end{frame}

\begin{frame}[fragile]
  \frametitle{Sources of Side Effects}

  \begin{itemize}
    \item function state
    \item object state
  \end{itemize}

  \begin{example}
    \begin{pygments}[]{c}
int foo(int n)
{
    static int z[] = {3, 5, 2, 8};
    static int i = -1;

    i++;
    return z[i] * n;
}
    \end{pygments}
  \end{example}
\end{frame}

\begin{frame}[fragile]
  \frametitle{Sources of Side Effects}

  \begin{itemize}
    \item input/output
  \end{itemize}

  \begin{example}
    \begin{pygments}[]{c}
int foo(int n, FILE *fp)
{
    int value;

    fscanf(fp, "%d\n", &value);
    return value * n;
}
    \end{pygments}
  \end{example}
\end{frame}

\begin{frame}[fragile]
  \frametitle{Sources of Side Effects}

  \begin{itemize}
    \item randomness
  \end{itemize}

  \begin{example}
    \begin{pygments}[]{c}
int foo(int n)
{
    int r = rand() % 10;

    return r * n;
}
    \end{pygments}
  \end{example}
\end{frame}

\begin{frame}
  \frametitle{Milestones in Functional Programming Languages}

  \begin{columns}
    \column{.45\textwidth}
    \begin{center}
      \pgfuseimage{mccarthy}\\
      John McCarthy (1927-2011)
    \end{center}

    \column{.55\textwidth}
    \begin{itemize}
      \item Lisp (1957)
      \item ML (1973)
      \item Haskell (1990)
    \end{itemize}
  \end{columns}
\end{frame}

\subsection{Object Orientation}

\begin{frame}
  \frametitle{Object Orientation}

  \begin{itemize}
    \item data abstraction
    \item type hierarchies

    \bigskip
    \item these can be achieved in both imperative and functional paradigms
    \begin{itemize}
      \item Ocaml, F\#
      \item Scala
    \end{itemize}

    \bigskip
    \item modern imperative languages support functional features
    \begin{itemize}
      \item Ruby, Python
      \item Java 8, C\# (LINQ)
    \end{itemize}
  \end{itemize}
\end{frame}
% 
% \section*{References}
% 
% \begin{frame}
%   \frametitle{References}
% 
%   \begin{block}{Required Reading: Scott}
%     \begin{itemize}
%       \item Chapter 1: Introduction
%       \begin{itemize}
%         \item 1.1. \alert{The Art of Language Design}
%       \end{itemize}
%     \end{itemize}
%   \end{block}
% \end{frame}

\end{document}
