% Copyright (c) 2013-2015 H. Turgut Uyar <uyar@itu.edu.tr>
%
% This work is licensed under a "Creative Commons
% Attribution-NonCommercial-ShareAlike 4.0 International License".
% For more information, please visit:
% https://creativecommons.org/licenses/by-nc-sa/4.0/

\documentclass[dvipsnames]{beamer}

\usepackage[scaled=0.95]{cabin}
\usepackage[scaled=0.88]{beramono}
\usepackage[T1]{fontenc}
\usepackage[utf8]{inputenc}

\usepackage{listings}
\lstset{basicstyle=\ttfamily,
        keywordstyle=\color{blue},
        showstringspaces=false}
\lstdefinestyle{syntax}{frame=single}
\lstset{language=Haskell}

\mode<presentation>
{
  \usetheme{default}
  \useinnertheme{rounded}
  \usecolortheme{seahorse}
  \setbeamercovered{transparent}
}

\title{Functional Programming}
\subtitle{Function Closures}

\author{H. Turgut Uyar}
\date{2013-2015}

\AtBeginSubsection[]{
  \begin{frame}<beamer>
    \frametitle{Topics}
    \tableofcontents[currentsection,currentsubsection]
  \end{frame}
}

\theoremstyle{plain}

\pgfdeclareimage[height=1cm]{license}{../license}

\begin{document}

\begin{frame}
  \titlepage
\end{frame}

\begin{frame}
  \frametitle{License}

  \pgfuseimage{license}\hfill
  \copyright~2013-2015 H. Turgut Uyar

  \vfill
  \begin{footnotesize}
    You are free to:
    \begin{itemize}
      \itemsep0em
      \item Share -- copy and redistribute the material in any medium or format
      \item Adapt -- remix, transform, and build upon the material
    \end{itemize}

    Under the following terms:
    \begin{itemize}
      \itemsep0em
      \item Attribution -- You must give appropriate credit, provide a link to
        the license, and indicate if changes were made.

      \item NonCommercial -- You may not use the material for commercial
        purposes.

      \item ShareAlike -- If you remix, transform, or build upon the material,
        you must distribute your contributions under the same license as the
        original.
    \end{itemize}

    For more information:\\
    \url{https://creativecommons.org/licenses/by-nc-sa/4.0/}

    \smallskip
    Read the full license:\\
    \url{https://creativecommons.org/licenses/by-nc-sa/4.0/legalcode}
  \end{footnotesize}
\end{frame}

\begin{frame}
  \frametitle{Topics}
  \tableofcontents
\end{frame}

\section{Function Closures}

\subsection{Functions as Result}

\begin{frame}
  \frametitle{Functions as Result}

  \begin{itemize}
    \item higher-order functions can return functions as result
  \end{itemize}

  \begin{example}[body surface area]
    \begin{itemize}
      \item $h$: height ($cm$), $w$: weight ($kg$), result: area ($m^2$)
      \smallskip
      \item Du Bois formula:
        $0.007184 \cdot h^{0.725} \cdot w^{0.425}$\\
      \smallskip
      \item Boyd formula:
        $0.0333 \cdot h^{0.3} \cdot w^{0.6157 - 0.0188~log_{10}~w}$
      \smallskip
      \item Boyd formula is more accurate in infants
    \end{itemize}
  \end{example}
\end{frame}

\begin{frame}[fragile]
  \frametitle{Returning Functions}

  \begin{exampleblock}{body surface area}
    \begin{lstlisting}
duBois :: Float -> Float -> Float
duBois h w = 0.007184 * (h ** 0.725) * (w ** 0.425)

boyd :: Float -> Float -> Float
boyd h w = 0.0333 * (h ** 0.3)
         * (w ** (0.6157 - 0.0188 * (logBase 10 w)))

bsa :: Integer -> (Float -> Float -> Float)
bsa age = if age < 3 then boyd else duBois

-- (bsa 20) 180 75 ~> 1.9424062
-- (bsa 2)  86 13  ~> 0.58253276
    \end{lstlisting}
  \end{exampleblock}
\end{frame}

\begin{frame}[fragile]
  \frametitle{Function Closure}

  \begin{itemize}
    \item a function value has two parts:

    \medskip
    \item the code
    \item the environment that was current when the function was defined

    \medskip
    \item function \alert{closure}
  \end{itemize}
\end{frame}

\begin{frame}[fragile]
  \frametitle{Function Closure Example}

  \begin{exampleblock}{}
    \begin{lstlisting}
stepRange :: Integer ->
                (Integer -> Integer -> [Integer])
stepRange step = getRange
  where
    getRange :: Integer -> Integer -> [Integer]
    getRange m n = [m, m + step .. n]

step1 :: Integer -> Integer -> [Integer]
step1 = stepRange 1
-- step1 3 7 ~> [3, 4, 5, 6, 7]

step5 :: Integer -> Integer -> [Integer]
step5 = stepRange 5
-- step5 9 20 ~> [9, 14, 19]
    \end{lstlisting}
  \end{exampleblock}
\end{frame}

\begin{frame}[fragile]
  \frametitle{Function Closure Example}

  \begin{exampleblock}{Python}
    \begin{lstlisting}[language=Python]
def step_range(step):
    def get_range(m, n):
        return range(m, n + 1, step)
    return get_range

step1 = step_range(1)

# step1(3, 7) ~> [3, 4, 5, 6, 7]
    \end{lstlisting}
  \end{exampleblock}
\end{frame}

\subsection{Decorators}

\begin{frame}[fragile]
  \frametitle{Decorators (Python)}

  \begin{itemize}
    \item a decorator takes a function as parameter\\
      and returns a transformed version of it
  \end{itemize}

  \begin{exampleblock}{example: entry and exit messages}
    \begin{lstlisting}[language=Python]
def entry_exit(f):
    def wrapped(x):
        print("Entering with parameter: %s" % x)
        result = f(x)
        print("Exiting with result: %s" % result)
        return result
    return wrapped
    \end{lstlisting}
  \end{exampleblock}
\end{frame}

\begin{frame}[fragile]
  \frametitle{Decorator Example}

  \begin{exampleblock}{}
    \begin{lstlisting}[language=Python]
def fac(n):
    return 1 if n == 0 else n * fac(n - 1)

# entry_exit(fac)(5)
    \end{lstlisting}

    \pause
    \medskip
    \begin{lstlisting}[language=Python]
@entry_exit
def fac(n):
    return 1 if n == 0 else n * fac(n - 1)

# fac(5)
    \end{lstlisting}
  \end{exampleblock}
\end{frame}

\begin{frame}[fragile]
  \frametitle{Decorator Example}

  \begin{exampleblock}{memoization}
    \begin{lstlisting}[language=Python]
def memoize(f):
    cache = {}
    def wrapped(x):
        if x not in cache:
            cache[x] = f(x)
        return cache[x]
    return wrapped
    \end{lstlisting}
  \end{exampleblock}
\end{frame}

\begin{frame}[fragile]
  \frametitle{Decorator Example}

  \begin{exampleblock}{memoized Fibonacci sequence}
    \begin{lstlisting}[language=Python]
@memoize
def fib(n):
    if n == 1 or n == 2:
        return 1
    else:
        return fib(n - 2) + fib(n - 1)
    \end{lstlisting}
  \end{exampleblock}
\end{frame}

\subsection{Currying}

\begin{frame}
  \frametitle{Currying}

  \begin{itemize}
    \item a function which takes two parameters can be thought of\\
      as a function which takes one parameter\\
      and returns a function which takes one parameter
    \item generalize for $n$ parameters: \alert{currying}

    \pause
    \bigskip
    \item \alert{partial application}: call with fewer paratemers,\\
      obtain a function that expects the remaining parameters

    \item in function signatures, arrows associate to the right
    \item function application associates to the left
  \end{itemize}
\end{frame}

\begin{frame}[fragile]
  \frametitle{Currying Example}

  \begin{exampleblock}{}
    \begin{lstlisting}
add :: Integer -> Integer -> Integer
add x y = x + y

-- same as:
add :: Integer -> (Integer -> Integer)
add x = \y -> x + y

increment :: Integer -> Integer
increment = add 1
-- increment = \y -> 1 + y
-- increment y = 1 + y
    \end{lstlisting}
  \end{exampleblock}
\end{frame}

\begin{frame}[fragile]
  \frametitle{Currying Examples}

  \begin{exampleblock}{}
    \begin{lstlisting}
floorAll :: [Float] -> [Integer]
floorAll xs = map floor xs

-- same as:
floorAll :: [Float] -> [Integer]
floorAll = map floor
    \end{lstlisting}
  \end{exampleblock}

  \begin{exampleblock}{}
    \begin{lstlisting}
allOdds :: [Integer] -> [Integer]
allOdds xs = filter odd xs

-- same as:
allOdds :: [Integer] -> [Integer]
allOdds = filter odd
    \end{lstlisting}
  \end{exampleblock}
\end{frame}

\begin{frame}[fragile]
  \frametitle{Currying Example}

  \begin{exampleblock}{}
    \begin{lstlisting}
stepRange :: Integer -> Integer -> Integer ->
                 [Integer]
stepRange step m n = [m, m + step .. n]

step1 :: Integer -> Integer -> [Integer]
-- step1 m n = stepRange 1 m n
step1 = stepRange 1

naturals :: Integer -> [Integer]
-- naturals n = stepRange 1 0 n
naturals = stepRange 1 0
-- naturals n = step1 0 n
-- naturals = step1 0
    \end{lstlisting}
  \end{exampleblock}
\end{frame}

\begin{frame}[fragile]
  \frametitle{Currying Functions}

  \begin{itemize}
    \item \lstinline|curry|: convert a function that takes a pair\\
      into an equivalent function that takes two parameters
  \end{itemize}

  \begin{lstlisting}[deletekeywords={curry}]
curry :: ((a, b) -> c) -> (a -> b -> c)
curry f = \x y -> f (x, y)

-- same as:
curry :: ((a, b) -> c) -> a -> b -> c
curry f x y = f (x, y)
  \end{lstlisting}
\end{frame}

\begin{frame}[fragile]
  \frametitle{Currying Functions}

  \begin{exampleblock}{}
    \begin{lstlisting}
addT :: (Integer, Integer) -> Integer
addT (x, y) = x + y

addC = curry addT
    \end{lstlisting}
  \end{exampleblock}

  \begin{itemize}
    \item exercise: convert a function that takes two parameters\\
      into an equivalent function that takes a pair:\\
      \lstinline|uncurry addC ~> addT|
  \end{itemize}
\end{frame}

\section{Function Operators}

\subsection{Composition}

\begin{frame}[fragile]
  \frametitle{Function Composition}

  \begin{block}{function composition}
    \begin{lstlisting}
(f . g) x = f (g x)
    \end{lstlisting}
  \end{block}

  \pause
  \begin{itemize}
    \item what is the type of \lstinline|(.)|?
  \end{itemize}

  \begin{exampleblock}{}
    \begin{lstlisting}
(.) : (b -> c) -> (a -> b) -> a -> c
(.) f g x = f (g x)
infixr 9 .
    \end{lstlisting}
  \end{exampleblock}
\end{frame}

\begin{frame}[fragile]
  \frametitle{Function Composition Example}

  \begin{exampleblock}{test whether number is even}
    \begin{lstlisting}
even :: Integer -> Bool
even = not . odd
    \end{lstlisting}
  \end{exampleblock}

  \pause
  \begin{exampleblock}{second element of a list}
    \begin{lstlisting}
second :: [a] -> a
second = head . tail
    \end{lstlisting}
  \end{exampleblock}
\end{frame}

\begin{frame}[fragile]
  \frametitle{Function Composition Example}

  \begin{exampleblock}{last element of a list}
    \begin{lstlisting}
last :: [a] -> a
last = head . reverse
    \end{lstlisting}
  \end{exampleblock}

  \pause
  \begin{exampleblock}{length of a list}
    \begin{lstlisting}
length :: [a] -> Int
length = sum . map (\_ -> 1)
    \end{lstlisting}
  \end{exampleblock}
\end{frame}

\subsection{Application}

\begin{frame}[fragile]
  \frametitle{Function Application}

  \begin{block}{function application}
    \begin{lstlisting}
f $ x = f x
    \end{lstlisting}
  \end{block}

  \pause
  \begin{itemize}
    \item what is the type of \lstinline|($)|?
  \end{itemize}

  \begin{exampleblock}{}
    \begin{lstlisting}
($) :: (a -> b) -> a -> b
f $ x = f x
infixr 0 $
    \end{lstlisting}
  \end{exampleblock}

  \pause
  \begin{itemize}
    \item why?
  \end{itemize}
\end{frame}

\begin{frame}[fragile]
  \frametitle{Function Application}

  \begin{itemize}
    \item less parentheses, more readable
  \end{itemize}

  \begin{exampleblock}{}
    \begin{lstlisting}
sum (filter odd (map (floor . sqrt) [1 .. 100]))

-- same as:
sum $ filter odd $ map (floor . sqrt) [1 .. 100]
    \end{lstlisting}
  \end{exampleblock}
\end{frame}

\begin{frame}[fragile]
  \frametitle{Function Application}

  \begin{itemize}
    \item needed in some cases
  \end{itemize}

  \begin{exampleblock}{}
    \begin{lstlisting}
zipWith ($) [sum, product] [[1, 2], [3, 4]]
-- [3, 12]
    \end{lstlisting}
  \end{exampleblock}
\end{frame}

\subsection{Operator Sections}

\begin{frame}[fragile]
  \frametitle{Operator Sections}

  \begin{itemize}
    \item operators can be partially applied
    \item a function that expects the missing argument
  \end{itemize}

  \begin{exampleblock}{}
    \begin{lstlisting}
(+2) 5                      -- 7
(>2) 5                      -- True
(2>) 5                      -- False
filter (4>) [5, 2, 3, 7]    -- [2, 3]
map (`div` 2) [5, 2, 3, 7]  -- [2, 1, 1, 3]
(map (*2) . filter ((==1) . (`mod` 2))) [5, 2, 3, 6]
    \end{lstlisting}
  \end{exampleblock}
\end{frame}

% TODO: rock-paper-scissors strategy combinators (alternate)

% \begin{frame}<beamer>[fragile]
%   \frametitle{Average Damping}
%
%   \begin{itemize}
%     \item a general function for average damping
%   \end{itemize}
%
%   \begin{exampleblock}{}
%     \begin{lstlisting}
% averageDamp :: (Float -> Float) -> Float -> Float
% averageDamp f x = (x + f x) / 2.0
%
% sqrt' :: Float -> Float
% sqrt' x = fixedPoint (averageDamp (\y -> x / y)) 1.0
%     \end{lstlisting}
%   \end{exampleblock}
% \end{frame}

\section*{References}

\begin{frame}
  \frametitle{References}

  \begin{block}{Required Reading: Thompson}
    \begin{itemize}
      \item Chapter 11: \alert{Higher-order functions}
    \end{itemize}
  \end{block}
\end{frame}

\end{document}
