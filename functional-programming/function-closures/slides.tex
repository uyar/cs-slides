% Copyright (c) 2013-2014 H. Turgut Uyar <uyar@itu.edu.tr>
%
% This work is licensed under a "Creative Commons
% Attribution-NonCommercial-ShareAlike 4.0 International License".
% For more information, please visit:
% https://creativecommons.org/licenses/by-nc-sa/4.0/

\documentclass[dvipsnames]{beamer}

\usepackage{ae}
\usepackage[T1]{fontenc}
\usepackage[utf8]{inputenc}
\usepackage{pythontex}
\setbeamertemplate{navigation symbols}{}
\setbeamersize{text margin left=2em, text margin right=2em}

\mode<presentation>
{
  \usetheme{default}
  \useinnertheme{rounded}
  \usecolortheme{seahorse}
  \setbeamercovered{transparent}
}

\title{Functional Programming}
\subtitle{Function Closures}

\author{H. Turgut Uyar}
\date{2013-2014}

\AtBeginSubsection[]{
  \begin{frame}<beamer>
    \frametitle{Topics}
    \tableofcontents[currentsection,currentsubsection]
  \end{frame}
}

\theoremstyle{plain}

\pgfdeclareimage[height=1cm]{license}{../license}

\begin{document}

\setpythontexfv[]{frame=single}

\begin{frame}
  \titlepage
\end{frame}

\begin{frame}
  \frametitle{License}

  \pgfuseimage{license}\hfill
  \copyright~2013-2014 H. Turgut Uyar

  \vfill
  \begin{tiny}
    You are free to:
    \begin{itemize}
      \item Share -- copy and redistribute the material in any medium or format
      \item Adapt -- remix, transform, and build upon the material
    \end{itemize}

    Under the following terms:
    \begin{itemize}
      \item Attribution -- You must give appropriate credit, provide a link to
        the license, and indicate if changes were made.\\
        You may do so in any reasonable manner, but not in any way
        that suggests the licensor endorses you or your use.

      \item Noncommercial -- You may not use the material for commercial
        purposes.

      \item Share Alike -- If you remix, transform, or build upon the material,
        you must distribute your contributions\\
        under the same license as the original.
    \end{itemize}
  \end{tiny}

  \vfill
  \begin{small}
    Legal code (the full license):\\
    \url{https://creativecommons.org/licenses/by-nc-sa/4.0/legalcode}
  \end{small}
\end{frame}

\begin{frame}
  \frametitle{Topics}
  \tableofcontents
\end{frame}

\section{Function Closures}

\subsection{Returning Functions}

\begin{frame}
  \frametitle{Returning Functions}

  \begin{itemize}
    \item higher order functions can return functions as result
  \end{itemize}

  \begin{example}[body surface area]
    \begin{itemize}
      \item $h$: height ($cm$), $w$: weight ($kg$), result: area ($m^2$)
      \smallskip
      \item Du Bois formula:
        $0.007184 \cdot h^{0.725} \cdot w^{0.425}$\\
      \smallskip
      \item Boyd formula:
        $0.0333 \cdot h^{0.3} \cdot w^{0.6157 - 0.0188~log_{10}~w}$
      \smallskip
      \item Boyd formula is more accurate in infants
    \end{itemize}
  \end{example}
\end{frame}

\begin{frame}[fragile]
  \frametitle{Returning Functions}

  \begin{exampleblock}{body surface area}
    \begin{pygments}{haskell}
duBois :: Float -> Float -> Float
duBois h w = 0.007184 * (h ** 0.725) * (w ** 0.425)

boyd :: Float -> Float -> Float
boyd h w = 0.0333 * (h ** 0.3) *
           (w ** (0.6157 - 0.0188 * (logBase 10 w)))

bsa :: Integer -> (Float -> Float -> Float)
bsa age = if age < 3 then boyd else duBois

(bsa 20) 180 75  -- 1.9424062
(bsa 2) 86 13    -- 0.58253276
    \end{pygments}
  \end{exampleblock}
\end{frame}

\begin{frame}[fragile]
  \frametitle{Function Closure}

  \begin{itemize}
    \item a function value has two parts:

    \medskip
    \item the code
    \item the environment that was current when the function was defined
  \end{itemize}
\end{frame}

\begin{frame}[fragile]
  \frametitle{Function Closure Example}

  \begin{exampleblock}{}
    \begin{pygments}{haskell}
stepRange :: Integer ->
                (Integer -> Integer -> [Integer])
stepRange step = getRange
  where
    getRange :: Integer -> Integer -> [Integer]
    getRange m n = [m, m + step .. n]

step1 :: Integer -> Integer -> [Integer]
step1 = stepRange 1
-- step1 3 7 -> [3, 4, 5, 6, 7]

step5 :: Integer -> Integer -> [Integer]
step5 = stepRange 5
-- step5 9 20 -> [9, 14, 19]
    \end{pygments}
  \end{exampleblock}
\end{frame}

\begin{frame}[fragile]
  \frametitle{Function Closure Example}

  \begin{exampleblock}{Python}
    \begin{pygments}{python}
def step_range(step):
    def get_range(m, n):
        return range(m, n + 1, step)
    return get_range

step1 = step_range(1)
step1(3, 7)            # [3, 4, 5, 6, 7]
    \end{pygments}
  \end{exampleblock}
\end{frame}

\subsection{Decorators}

\begin{frame}[fragile]
  \frametitle{Decorators}

  \begin{itemize}
    \item in Python, a decorator returns a transformed version\\
      of its function parameter
  \end{itemize}

  \begin{exampleblock}{}
    \begin{pygments}{python}
def entry_exit(f):
    def wrapped(x):
        print("Entering with parameter: %s" % x)
        result = f(x)
        print("Exiting with result: %s" % result)
        return result
    return wrapped
    \end{pygments}
  \end{exampleblock}
\end{frame}

\begin{frame}[fragile]
  \frametitle{Decorator Example}

  \begin{exampleblock}{}
    \begin{pygments}{python}
def fact(n):
    return 1 if n == 0 else n * fact(n - 1)

entry_exit(fact)(5)
    \end{pygments}

    \pause
    \medskip
    \begin{pygments}{python}
@entry_exit
def fact(n):
    return 1 if n == 0 else n * fact(n - 1)

fact(5)
    \end{pygments}
  \end{exampleblock}
\end{frame}

\begin{frame}[fragile]
  \frametitle{Decorator Example}

  \begin{exampleblock}{memoization}
    \begin{pygments}{python}
def memoize(f):
    cache = {}
    def wrapped(x):
        if x not in cache:
            cache[x] = f(x)
        return cache[x]
    return wrapped
    \end{pygments}
  \end{exampleblock}
\end{frame}

\begin{frame}[fragile]
  \frametitle{Decorator Example}

  \begin{exampleblock}{Fibonacci sequence}
    \begin{pygments}{python}
@memoize
def fib(n):
    if n == 1 or n == 2:
        return 1
    else:
        return fib(n - 2) + fib(n - 1)
    \end{pygments}
  \end{exampleblock}
\end{frame}

\subsection{Currying}

\begin{frame}
  \frametitle{Currying}

  \begin{itemize}
    \item a function which takes two parameters can be thought of\\
      as a function which takes one parameter\\
      and returns a function which takes one parameter
    \item generalize for $n$ parameters: \alert{currying}

    \pause
    \bigskip
    \item \alert{partial application}: call with fewer paratemers,\\
      obtain a function that expects the remaining parameters
  \end{itemize}
\end{frame}

\begin{frame}[fragile]
  \frametitle{Currying Example}

  \begin{exampleblock}{}
    \begin{pygments}{haskell}
add :: Integer -> Integer -> Integer
add x y = x + y

-- same as: add :: Integer -> (Integer -> Integer)
    \end{pygments}

    \pause
    \medskip
    \begin{pygments}{haskell}
increment :: Integer -> Integer
increment = add 1

increment 7
    \end{pygments}
  \end{exampleblock}
\end{frame}

\begin{frame}[fragile]
  \frametitle{Currying}

  \begin{exampleblock}{}
    \begin{pygments}{haskell}
floorAll :: [Float] -> [Integer]
floorAll xs = map floor xs

allOdds :: [Integer] -> [Integer]
allOdds xs = filter odd xs

-- same as:
floorAll :: [Float] -> [Integer]
floorAll = map floor

allOdds :: [Integer] -> [Integer]
allOdds = filter odd
    \end{pygments}
  \end{exampleblock}
\end{frame}

\begin{frame}[fragile]
  \frametitle{Currying Example}

  \begin{exampleblock}{}
    \begin{pygments}{haskell}
stepRange' :: Integer -> Integer -> Integer ->
                  [Integer]
stepRange' step m n = [m, m + step .. n]

step1' :: Integer -> Integer -> [Integer]
-- step1' m n = stepRange' 1 m n
step1' = stepRange' 1

naturals :: Integer -> [Integer]
-- naturals n = stepRange' 1 0 n
naturals = stepRange' 1 0
    \end{pygments}
  \end{exampleblock}
\end{frame}

\begin{frame}[fragile]
  \frametitle{Currying vs Tupling}

  \begin{exampleblock}{}
    \begin{pygments}{haskell}
add1 :: Integer -> Integer -> Integer
add1 x y = x + y

add2 :: (Integer, Integer) -> Integer
add2 (x, y) = x + y
    \end{pygments}
  \end{exampleblock}
\end{frame}

\begin{frame}[fragile]
  \frametitle{Currying vs Tupling}

  \begin{itemize}
    \item a function that will convert a function that takes two parameters\\
      into an equivalent function that takes a pair
  \end{itemize}

  \begin{exampleblock}{}
    \begin{pygments}{haskell}
uncurry' :: (a -> b -> c) -> ((a, b) -> c)
uncurry' f = \(x, y) -> f x y

(uncurry' add1) (5, 8)
    \end{pygments}
  \end{exampleblock}

  \pause
  \begin{itemize}
    \item note that \pygment{haskell}{f = \x -> e}\\
      is equivalent to: \pygment{haskell}{f x = e}
  \end{itemize}

  \begin{exampleblock}{}
    \begin{pygments}{haskell}
uncurry' :: (a -> b -> c) -> ((a, b) -> c)
uncurry' f (x, y) = f x y
    \end{pygments}
  \end{exampleblock}
\end{frame}

\section{Function Operators}

\subsection{Composition}

\begin{frame}[fragile]
  \frametitle{Function Composition}

  \begin{block}{function composition}
    \begin{pygments}{haskell}
(f . g) = f (g x)
    \end{pygments}
  \end{block}

  \pause
  \begin{itemize}
    \item what is the type of \pygment{haskell}{(.)}?
  \end{itemize}

  \begin{exampleblock}{}
    \begin{pygments}{haskell}
(.) : (b -> c) -> (a -> b) -> a -> c
    \end{pygments}
  \end{exampleblock}
\end{frame}

\begin{frame}[fragile]
  \frametitle{Function Composition Example}

  \begin{exampleblock}{test whether number is even}
    \begin{pygments}{haskell}
even' :: Integer -> Bool
even' = not . odd
    \end{pygments}
  \end{exampleblock}

  \pause
  \begin{exampleblock}{second element of a list}
    \begin{pygments}{haskell}
second :: [a] -> a
second = head . tail
    \end{pygments}
  \end{exampleblock}
\end{frame}

\begin{frame}[fragile]
  \frametitle{Function Composition Example}

  \begin{exampleblock}{last element of a list}
    \begin{pygments}{haskell}
last' :: [a] -> a
last' = head . reverse
    \end{pygments}
  \end{exampleblock}

  \pause
  \begin{exampleblock}{length of a list}
    \begin{pygments}{haskell}
length' :: [a] -> Integer
length' = sum . map (\_ -> 1)
    \end{pygments}
  \end{exampleblock}
\end{frame}

\subsection{Application}

\begin{frame}[fragile]
  \frametitle{Function Application}

  \begin{block}{function application}
    \begin{pygments}{haskell}
f $ x = f x
    \end{pygments}
  \end{block}

  \begin{itemize}
    \item less parentheses, more readable
  \end{itemize}

  \begin{exampleblock}{}
    \begin{pygments}{haskell}
sum (filter odd (map (floor . sqrt) [1 .. 100]))

-- same as:
sum $ filter odd $ map (floor . sqrt) [1 .. 100]
    \end{pygments}
  \end{exampleblock}
\end{frame}

\begin{frame}[fragile]
  \frametitle{Function Application}

  \begin{itemize}
    \item needed in some cases
  \end{itemize}

  \begin{exampleblock}{}
    \begin{pygments}{haskell}
zipWith ($) [sum, product] [[1, 2], [3, 4]]
-- [3, 12]
    \end{pygments}
  \end{exampleblock}
\end{frame}

\subsection{Operator Sections}

\begin{frame}[fragile]
  \frametitle{Operator Sections}

  \begin{itemize}
    \item operators can be partially applied
    \item a function that expects the missing argument
  \end{itemize}

  \begin{exampleblock}{}
    \begin{pygments}{haskell}
(+2) 5                      -- 7
(>2) 5                      -- True
(2>) 5                      -- False
filter (4>) [5, 2, 3, 7]    -- [2, 3]
map (`div` 2) [5, 2, 3, 7]  -- [2, 1, 1, 3]
(map (*2) . filter ((==1) . (`mod` 2))) [5, 2, 3, 6]
    \end{pygments}
  \end{exampleblock}
\end{frame}

\subsection{Examples}

% TODO: edit distance

\begin{frame}
  \frametitle{Finding Fixed Points}

  \begin{itemize}
    \item $x$ is a \emph{fixed point} of $f$: $f(x)=x$
    \item repeatedly apply $f$ until value doesn't change:\\
      $x,f(x),f(f(x)),f(f(f(x))),\ldots$
  \end{itemize}
\end{frame}

\begin{frame}[fragile]
  \frametitle{Finding Fixed Points}

  \begin{exampleblock}{}
    \begin{pygments}{haskell}
fixedPoint :: (Float -> Float) -> Float -> Float
fixedPoint f firstGuess = fixedPointIter firstGuess
  where
    isCloseEnough :: Float -> Float -> Bool
    isCloseEnough x y = (abs (x - y) / x) < 0.001

    fixedPointIter :: Float -> Float
    fixedPointIter guess
      | isCloseEnough guess next = next
      | otherwise                = fixedPointIter next
      where
        next :: Float
        next = f guess
    \end{pygments}
  \end{exampleblock}
\end{frame}

\begin{frame}[fragile]
  \frametitle{Square Roots}

  \begin{itemize}
    \item use the fixed point algorithm for computing square roots
    \item $y = \sqrt{x} \Rightarrow y * y = x \Rightarrow y = x / y$
    \item fixed point of the function $f(y) = x / y$
  \end{itemize}

  \pause
  \begin{exampleblock}{}
    \begin{pygments}{haskell}
sqrt' :: Float -> Float
sqrt' x = fixedPoint (\y -> x / y) 1.0
    \end{pygments}
  \end{exampleblock}

  \pause
  \begin{itemize}
    \item this doesn't converge (try with $y = 2$)
    \item average successive values (average damping)
  \end{itemize}

  \begin{exampleblock}{}
    \begin{pygments}{haskell}
sqrt' x = fixedPoint (\y -> (y + x / y) / 2.0) 1.0
    \end{pygments}
  \end{exampleblock}
\end{frame}

\begin{frame}<beamer>[fragile]
  \frametitle{Average Damping}

  \begin{itemize}
    \item a general function for average damping
  \end{itemize}

  \begin{exampleblock}{}
    \begin{pygments}{haskell}
averageDamp :: (Float -> Float) -> Float -> Float
averageDamp f x = (x + f x) / 2.0

sqrt' :: Float -> Float
sqrt' x = fixedPoint (averageDamp (\y -> x / y)) 1.0
    \end{pygments}
  \end{exampleblock}
\end{frame}

\section*{References}

\begin{frame}
  \frametitle{References}

  \begin{block}{Required Reading: Thompson}
    \begin{itemize}
      \item Chapter 10: \alert{Generalization: patterns of computation}
      \item Chapter 11: \alert{Higher-order functions}
    \end{itemize}
  \end{block}
\end{frame}

\end{document}
