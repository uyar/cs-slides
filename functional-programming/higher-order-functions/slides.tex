% Copyright (c) 2013
%       H. Turgut Uyar <uyar@itu.edu.tr>
%
% These notes are licensed using the
% "Creative Commons Attribution-NonCommercial-ShareAlike License".
% You are free to copy, distribute and transmit the work, and to adapt the work
% as long as you attribute the authors, do not use it for commercial purposes,
% and any derivative work is under the same or a similar license.
%
% Read the full legal code at:
% http://creativecommons.org/licenses/by-nc-sa/3.0/

\documentclass[dvipsnames]{beamer}

\usepackage{ae}
\usepackage[T1]{fontenc}
\usepackage[utf8]{inputenc}
\usepackage{pythontex}
\setbeamertemplate{navigation symbols}{}

\mode<presentation>
{
  \usetheme{Boadilla}
  \setbeamercovered{transparent}
}

\title{Functional Programming}
\subtitle{Higher Order Functions}

\author{H. Turgut Uyar}
\date{2013}

\AtBeginSubsection[]{
  \begin{frame}<beamer>
    \frametitle{Topics}
    \tableofcontents[currentsection,currentsubsection]
  \end{frame}
}

\theoremstyle{plain}

\pgfdeclareimage[width=2cm]{license}{../../license}

\begin{document}

\begin{frame}
  \titlepage
\end{frame}

\begin{frame}
  \frametitle{License}

  \pgfuseimage{license}\hfill
  \copyright 2013 T. Uyar

  \vfill
  \begin{tiny}
    You are free:
    \begin{itemize}
      \item to Share -- to copy, distribute and transmit the work
      \item to Remix -- to adapt the work
    \end{itemize}

    Under the following conditions:
    \begin{itemize}
      \item Attribution -- You must attribute the work in the manner specified by
        the author or licensor (but not in any way that suggests that they
        endorse you or your use of the work).

      \item Noncommercial -- You may not use this work for commercial purposes.

      \item Share Alike -- If you alter, transform, or build upon this work, you
        may distribute the resulting work only under the same or similar license
        to this one.
    \end{itemize}
  \end{tiny}

  \vfill
  Legal code (the full license):\\
  \url{http://creativecommons.org/licenses/by-nc-sa/3.0/}
\end{frame}

\begin{frame}
  \frametitle{Topics}
  \tableofcontents
\end{frame}

\section{Higher Order Functions}

\subsection{First Class Values}

\begin{frame}
  \frametitle{First Class Values}

  \begin{itemize}
    \item \alert{first class values} can be:
    \begin{itemize}
      \item assigned
      \item composed with other values
      \item passed as arguments
      \item returned as function results
    \end{itemize}

    \pause
    \bigskip
    \item \emph{type completeness principle}:\\
      No operation should be arbitrarily restricted in the types\\
      of its operands.
  \end{itemize}
\end{frame}

\begin{frame}
  \frametitle{First Class Values}

  \begin{block}{C}
    \begin{table}
      \begin{tabular}{l||c|c|c|c}
                & primitive & structure &  array  & function\\\hline\hline
        assign  &  $\surd$  &  $\surd$  &   $?$   &    $?$  \\\hline
        compose &  $\surd$  &  $\surd$  & $\surd$ &    $?$  \\\hline
        pass    &  $\surd$  &  $\surd$  &   $?$   &    $?$  \\\hline
        return  &  $\surd$  &  $\surd$  &   $?$   &    $?$
      \end{tabular}
    \end{table}
  \end{block}
\end{frame}

\begin{frame}
  \frametitle{First Class Values}

  \begin{block}{Pascal}
    \begin{table}
      \begin{tabular}{l||c|c|c|c}
                & primitive & record  &  array  & function \\\hline\hline
        assign  &  $\surd$  & $\surd$ & $\surd$ & $\times$ \\\hline
        compose &  $\surd$  & $\surd$ & $\surd$ & $\times$ \\\hline
        pass    &  $\surd$  & $\surd$ & $\surd$ & $\surd$  \\\hline
        return  &  $\surd$  &   $?$   &   $?$   & $\times$
      \end{tabular}
    \end{table}
  \end{block}

  \begin{itemize}
    \item type definition needed for arrays and records to be returned
  \end{itemize}
\end{frame}

\begin{frame}
  \frametitle{First Class Values}

  \begin{block}{Haskell, Scala, Python}
    \begin{table}
      \begin{tabular}{l||c|c|c|c}
                & primitive &  tuple  &  list   & function\\\hline\hline
        assign  &  $\surd$  & $\surd$ & $\surd$ & $\surd$ \\\hline
        compose &  $\surd$  & $\surd$ & $\surd$ & $\surd$ \\\hline
        pass    &  $\surd$  & $\surd$ & $\surd$ & $\surd$ \\\hline
        return  &  $\surd$  & $\surd$ & $\surd$ & $\surd$
      \end{tabular}
    \end{table}
  \end{block}
\end{frame}

\subsection{Higher Order Functions}

\begin{frame}[fragile]
  \frametitle{Higher Order Functions}

  \begin{definition}
    \alert{first order functions}: functions that\\
    \begin{itemize}
      \item only accept data as parameter, and
      \item only return data as result
    \end{itemize}

    \bigskip
    \alert{higher order functions}: functions that
    \begin{itemize}
      \item take other functions as parameter, or
      \item return functions as result
    \end{itemize}
  \end{definition}
\end{frame}

\begin{frame}[fragile]
  \frametitle{First Order Function Examples}

  \begin{example}[Haskell]
    \begin{pygments}{haskell}
sum_integers a b =
    if a > b then 0 else a + sum_integers (a + 1) b
    \end{pygments}

    \pause
    \bigskip
    \begin{pygments}{haskell}
cube x = x * x * x

sum_cubes a b =
    if a > b then 0 else cube a + sum_cubes (a + 1) b
    \end{pygments}

    \pause
    \bigskip
    \begin{pygments}{haskell}
sum_factorials a b =
    if a > b then 0 else fact a + sum_factorials (a + 1) b
    \end{pygments}
  \end{example}
\end{frame}

\begin{frame}[fragile]
  \frametitle{Higher Order Function Example}

  \begin{example}[Haskell]
    \pause
    \begin{pygments}{haskell}
sum_func f a b =
    if a > b then 0 else f a + sum_func f (a + 1) b
    \end{pygments}

    \pause
    \bigskip
    \begin{pygments}{haskell}
sum_cubes a b = sum_func cube a b

sum_factorials a b = sum_func fact a b
    \end{pygments}

    \pause
    \smallskip
    \begin{pygments}{haskell}
sum_integers a b = sum_func id a b
    \end{pygments}

    \pause
    \smallskip
    \begin{pygments}{haskell}
-- what is the type of sum_func?
    \end{pygments}
  \end{example}
\end{frame}

\begin{frame}[fragile]
  \frametitle{Higher Order Functions}

  \begin{block}{Scala}
    \begin{pygments}{scala}
      fp: (t1, t2, ..., tn) => tr
    \end{pygments}
  \end{block}

  \pause
  \begin{example}[Scala]
    \pause
    \begin{pygments}{scala}
def sumFunc(f: Int => Int, a: Int, b: Int): Int =
    if (a > b) 0
    else f(a) + sumFunc(f, a + 1, b)

def cube(x: Int): Int = x * x * x

def sumCubes(a: Int, b: Int): Int = sumFunc(cube, a, b)
    \end{pygments}
  \end{example}
\end{frame}

\begin{frame}[fragile]
  \frametitle{Higher Order Function Example}

  \begin{example}[Python]
    \begin{pygments}{python}
def sum_func(f, a, b):
    total = 0
    while a <= b:
        total += f(a)
        a += 1
    return total

def cube(x):
    return x * x * x

def sum_cubes(a, b):
    return sum_func(cube, a, b)
    \end{pygments}
  \end{example}
\end{frame}

\begin{frame}[fragile]
  \frametitle{Higher Order Functions}

  \begin{block}{C}
    \begin{pygments}{c}
      tr (*fp)(t1, t2, ..., tn)
    \end{pygments}
  \end{block}

  \pause
  \begin{example}[C]
    \begin{pygments}{c}
int sum_func(int (*f)(int), int a, int b)
{
    int total = 0;
    while (a <= b)
    {
        total += f(a);
        a += 1;
    }
    return total;
}
    \end{pygments}
  \end{example}
\end{frame}

\begin{frame}[fragile]
  \frametitle{Higher Order Function Example}

  \begin{example}[C]
    \begin{pygments}{c}
int cube(int x)
{
    return x * x * x;
}

int sum_cubes(int a, int b)
{
    return sum_func(cube, a, b);
}
    \end{pygments}
  \end{example}
\end{frame}

\subsection{Anonymous Functions}

\begin{frame}[fragile]
  \frametitle{Anonymous Functions}

  \begin{itemize}
    \item no need to name small functions that are not used anywhere else\\
      $\rightarrow$ \alert{anonymous functions}
    \item anonymous functions can not be recursive
  \end{itemize}

  \begin{block}{Haskell}
    \begin{pygments}{haskell}
      \fp1, fp2, ... -> e
    \end{pygments}
  \end{block}

  \medskip
  \begin{example}
    \begin{pygments}{haskell}
sum_cubes a b = sum_func (\x -> x * x * x) a b
    \end{pygments}
  \end{example}
\end{frame}

\begin{frame}[fragile]
  \frametitle{Anonymous Functions}

  \begin{block}{Scala}
    \begin{pygments}{scala}
      (fp1: t1, fp2: t2, ...) => block
    \end{pygments}
  \end{block}

  \medskip
  \begin{example}
    \begin{pygments}{scala}
def sumCubes(a: Int, b: Int): Int =
    sumFunc((x: Int) => x * x * x, a, b)
    \end{pygments}
  \end{example}
\end{frame}

\begin{frame}[fragile]
  \frametitle{Anonymous Functions}

  \begin{block}{Python}
    \begin{pygments}{python}
      lambda fp1, fp2, ...: e
    \end{pygments}
  \end{block}

  \medskip
  \begin{example}
    \begin{pygments}{python}
def sum_cubes(a, b):
    sum_func(lambda x: x * x * x, a, b)
    \end{pygments}
  \end{example}
\end{frame}

% \begin{frame}[fragile]
%   \frametitle{Higher Order Function Example}
%
%   \begin{example}
%     \begin{itemize}
%       \item square roots with Newton's method:
%
%       \medskip
%       \item start with an initial estimate $y$ (say $y = 1$)
%       \item repeatedly improve the estimate by taking the mean of $y$
%         and $x / y$
%     \end{itemize}
%
%     \pause
%     \medskip
%     \begin{center}
%     \begin{tabular}{lll}
%       $y$      & $x / y$             & next guess\\\hline
%       1        & 2 / 1 = 2           & 1.5\\
%       1.5      & 2 / 1.5 = 1.333     & 1.4167\\
%       1.4167   & 2 / 1.4167 = 1.4118 & 1.4142\\
%       1.4142   & ...                 & ...
%     \end{tabular}
%     \end{center}
%   \end{example}
% \end{frame}
%
% \begin{frame}<beamer>[fragile]
%   \frametitle{Higher Order Function Example}
%
%   \begin{example}[Newton's method]
%     \begin{pygments}{haskell}
% newton x =
%     let
%         good_enough guess =
%             abs (guess * guess - x) < 0.001
%
%         improve guess =
%             (guess + (x / guess)) / 2.0
%
%         newton_iter guess =
%             if good_enough guess
%             then guess
%             else newton_iter (improve guess)
%     in
%         newton_iter 1.0
%     \end{pygments}
%   \end{example}
% \end{frame}
%
% \begin{frame}<beamer>[fragile]
%   \frametitle{Higher Order Function Example}
%
%   \begin{example}[Newton's method]
%     \begin{itemize}
%       \item doesn't work with too small and too large numbers
%     \end{itemize}
%
%     \bigskip
%     \begin{pygments}{haskell}
%         good_enough guess =
%             (abs (guess * guess - x)) / x < 0.001
%     \end{pygments}
%   \end{example}
% \end{frame}
%
% \begin{frame}[fragile]
%   \frametitle{Higher Order Function Example}
%
%   \begin{example}
%     \begin{itemize}
%       \item $x$ is a \emph{fixed point} of $f$: $f(x)=x$
%       \item find fixed point by repeatedly applying $f$ until value doesn't change:\\
%         $x,f(x),f(f(x)),f(f(f(x))),\ldots$
%
%       \medskip
%       \item $y = \sqrt{x} \Rightarrow y * y = x \Rightarrow y = x / y$
%       \item fixed point of the function $x / y$
%     \end{itemize}
%   \end{example}
% \end{frame}
%
% \begin{frame}<beamer>[fragile]
%   \frametitle{Higher Order Function Example}
%
%   \begin{example}[Newton's method]
%     \begin{pygments}{haskell}
% fixed_point f initial =
%     let
%         no_change x y = (abs (x - y) / x) < 0.001
%
%         fp_iter guess =
%             let
%                 next = f guess
%             in
%                 if no_change guess next
%                 then next
%                 else fp_iter next
%     in
%         fp_iter initial
%
% newton2 x = fixed_point (\y -> (y + x / y) / 2.0) 1.0
%     \end{pygments}
%   \end{example}
% \end{frame}
%
% \section{List Functions}
%
% \begin{frame}[fragile]
%   \frametitle{The map Function}
%
%   \begin{definition}
%     \alert{\pygment{haskell}{map}}:\\
%       apply a function to each element in a list
%   \end{definition}
%
%   \pause
%   \begin{example}[map]
%     \begin{pygments}{haskell}
% map fact [4,1,3] = [24,1,6]
%
% -- alternative definition for sumFacts
% sumFacts n = sum (map fact [1..n])
%     \end{pygments}
%   \end{example}
% \end{frame}
%
% \begin{frame}[fragile]
%   \frametitle{The filter Function}
%
%   \begin{definition}
%     \alert{\pygment{haskell}{filter}}:\\
%       select elements from a list using a function
%   \end{definition}
%
%   \pause
%   \begin{example}[filter]
%     \begin{pygments}{haskell}
% filter odd [4,1,3,2] = [1,3]
%     \end{pygments}
%   \end{example}
% \end{frame}
%
% \begin{frame}[fragile]
%   \frametitle{Function Composition}
%
%   \begin{block}{syntax}
%     \begin{pygments}
% f . g
%     \end{pygments}
%   \end{block}
%
%   \pause
%   \begin{example}[even]
%     \begin{pygments}{haskell}
% not :: Bool -> Bool
% odd :: Int -> Bool
%
% even :: Int -> Bool
% even = not . odd
%     \end{pygments}
%   \end{example}
% \end{frame}
%
% \section{Currying}
%
% \begin{frame}
%   \frametitle{Curried Functions}
%
%   \begin{definition}
%     \alert{curried function}:\\
%       partial application of a function
%   \end{definition}
% \end{frame}
%
% \begin{frame}[fragile]
%   \frametitle{Curried Function Example}
%
%   \begin{example}[power]
%     \begin{pygments}{haskell}
% powerc :: Int -> Float -> Float
% powerc n a =
%   if n==0 then 1.0 else a * powerc (n-1) a
%
% sqr :: Float -> Float
% sqr = powerc 2
%
% cube :: Float -> Float
% cube = powerc 3
%     \end{pygments}
%   \end{example}
% \end{frame}
%
% \begin{frame}
%   \frametitle{Lazy Lists}
%
%   \begin{definition}
%     \alert{lazy list}:\\
%       list members are generated when they are needed
%   \end{definition}
% \end{frame}
%
% \begin{frame}[fragile]
%   \frametitle{Lazy List Example}
%
%   \begin{example}
%     \begin{pygments}{haskell}
% primes :: [Int]
% primes = sieve [2..]
%
% sieve :: [Int] -> [Int]
% sieve (x:xs) =
%   x : sieve [y | y <- xs, y `mod` x > 0]
%     \end{pygments}
%   \end{example}
% \end{frame}
%
%
% \section*{References}
%
% \begin{frame}
%   \frametitle{References}
%
%   \begin{block}{Required Reading: Abelson, Sussman}
%     \begin{itemize}
%       \item Chapter 1: Building Abstractions with Procedures
%       \begin{itemize}
%         \item 1.1. \alert{The Elements of Programming}
%       \end{itemize}
%     \end{itemize}
%   \end{block}
% \end{frame}

\end{document}
