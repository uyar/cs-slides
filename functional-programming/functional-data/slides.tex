% Copyright (c) 2013-2014 H. Turgut Uyar <uyar@itu.edu.tr>
%
% This work is licensed under a "Creative Commons
% Attribution-NonCommercial-ShareAlike 4.0 International License".
% For more information, please visit:
% https://creativecommons.org/licenses/by-nc-sa/4.0/

\documentclass[dvipsnames]{beamer}

\usepackage{ae}
\usepackage[T1]{fontenc}
\usepackage[utf8]{inputenc}
\usepackage{pythontex}
\setbeamertemplate{navigation symbols}{}
\setbeamersize{text margin left=2em, text margin right=2em}

\mode<presentation>
{
  \usetheme{default}
  \useinnertheme{rounded}
  \usecolortheme{seahorse}
  \setbeamercovered{transparent}
}

\title{Functional Programming}
\subtitle{Functional Data Structures}

\author{H. Turgut Uyar}
\date{2013-2014}

\AtBeginSubsection[]{
  \begin{frame}<beamer>
    \frametitle{Topics}
    \tableofcontents[currentsection,currentsubsection]
  \end{frame}
}

\theoremstyle{plain}

\pgfdeclareimage[height=1cm]{license}{../license}

\begin{document}

\setpythontexfv[]{frame=single}

\begin{frame}
  \titlepage
\end{frame}

\begin{frame}
  \frametitle{License}

  \pgfuseimage{license}\hfill
  \copyright~2013-2014 H. Turgut Uyar

  \vfill
  \begin{tiny}
    You are free to:
    \begin{itemize}
      \item Share -- copy and redistribute the material in any medium or format
      \item Adapt -- remix, transform, and build upon the material
    \end{itemize}

    Under the following terms:
    \begin{itemize}
      \item Attribution -- You must give appropriate credit, provide a link to
        the license, and indicate if changes were made.\\
        You may do so in any reasonable manner, but not in any way
        that suggests the licensor endorses you or your use.

      \item Noncommercial -- You may not use the material for commercial
        purposes.

      \item Share Alike -- If you remix, transform, or build upon the material,
        you must distribute your contributions\\
        under the same license as the original.
    \end{itemize}
  \end{tiny}

  \vfill
  \begin{small}
    Legal code (the full license):\\
    \url{https://creativecommons.org/licenses/by-nc-sa/4.0/legalcode}
  \end{small}
\end{frame}

\begin{frame}
  \frametitle{Topics}
  \tableofcontents
\end{frame}

\section{Abstract Data Types}

\subsection{Introduction}

\begin{frame}
  \frametitle{Abstract Data Types}

  \begin{definition}
    \alert{abstract data type}:\\
      private representation equipped with public operations
  \end{definition}
\end{frame}

\begin{frame}[fragile]
  \frametitle{Sets}

  \begin{itemize}
    \item represent a set as an ordered list of elements without repetition
  \end{itemize}

  \begin{exampleblock}{}
    \begin{pygments}{haskell}
newtype Set a = Set [a]
    \end{pygments}
  \end{exampleblock}
\end{frame}

\begin{frame}[fragile]
  \frametitle{Sets}

  \begin{exampleblock}{}
    \begin{pygments}{haskell}
empty :: Set a
empty = Set []

sing :: a -> Set a
sing x = Set [x]
    \end{pygments}
  \end{exampleblock}
\end{frame}

\section*{References}

\begin{frame}
  \frametitle{References}

  \begin{block}{Required Reading: Thompson}
    \begin{itemize}
      \item Chapter 16: \alert{Abstract data types}
    \end{itemize}
  \end{block}
\end{frame}

\end{document}
