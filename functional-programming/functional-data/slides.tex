% Copyright (c) 2013-2016 H. Turgut Uyar <uyar@itu.edu.tr>
%
% This work is licensed under a "Creative Commons
% Attribution-NonCommercial-ShareAlike 4.0 International License".
% For more information, please visit:
% https://creativecommons.org/licenses/by-nc-sa/4.0/

\documentclass[dvipsnames]{beamer}

\usepackage[scaled=0.95]{cabin}
\usepackage[scaled=0.88]{beramono}
\usepackage[T1]{fontenc}
\usepackage[utf8]{inputenc}

\usepackage{listings}
\lstset{basicstyle=\ttfamily,
        keywordstyle=\color{blue},
        showstringspaces=false}
\lstdefinestyle{syntax}{frame=single}
\lstset{language=Haskell}

\mode<presentation>
{
  \usetheme{default}
  \useinnertheme{rounded}
  \usecolortheme{seahorse}
  \setbeamercovered{transparent}
}

\title{Functional Programming}
\subtitle{Functional Data Structures}

\author{H. Turgut Uyar}
\date{2013-2016}

\AtBeginSubsection[]{
  \begin{frame}<beamer>
    \frametitle{Topics}
    \tableofcontents[currentsection,currentsubsection]
  \end{frame}
}

\theoremstyle{plain}

\pgfdeclareimage[height=1cm]{license}{../license}

\pgfdeclareimage[width=10cm]{append1}{append1}
\pgfdeclareimage[width=10cm]{append2}{append2}
\pgfdeclareimage[width=10cm]{append}{append}

\begin{document}

\begin{frame}
  \titlepage
\end{frame}

\begin{frame}
  \frametitle{License}

  \pgfuseimage{license}\hfill
  \copyright~2013-2016 H. Turgut Uyar

  \vfill
  \begin{footnotesize}
    You are free to:
    \begin{itemize}
      \itemsep0em
      \item Share -- copy and redistribute the material in any medium or format
      \item Adapt -- remix, transform, and build upon the material
    \end{itemize}

    Under the following terms:
    \begin{itemize}
      \itemsep0em
      \item Attribution -- You must give appropriate credit, provide a link to
        the license, and indicate if changes were made.

      \item NonCommercial -- You may not use the material for commercial
        purposes.

      \item ShareAlike -- If you remix, transform, or build upon the material,
        you must distribute your contributions under the same license as the
        original.
    \end{itemize}

    For more information:\\
    \url{https://creativecommons.org/licenses/by-nc-sa/4.0/}

    \smallskip
    Read the full license:\\
    \url{https://creativecommons.org/licenses/by-nc-sa/4.0/legalcode}
  \end{footnotesize}
\end{frame}

\begin{frame}
  \frametitle{Topics}
  \tableofcontents
\end{frame}

\section{Functional Data}

\subsection{Immutability}

\begin{frame}[fragile]
  \frametitle{Appending Lists}

  \begin{itemize}
    \item append a list at the end another list, and get a third list
  \end{itemize}

  \begin{center}
    \pgfuseimage{append1}
  \end{center}

  \begin{exampleblock}{example: C}
    \begin{lstlisting}[language=C]
xs->last->next = ys->head;
zs->head = xs->head;
zs->last = ys->last;
    \end{lstlisting}
  \end{exampleblock}
\end{frame}

\begin{frame}
  \frametitle{Appending Lists}

  \begin{center}
    \pgfuseimage{append2}
  \end{center}

  \begin{itemize}
    \item very fast
    \item destroys both \lstinline|xs| and \lstinline|ys|
  \end{itemize}
\end{frame}

\begin{frame}[fragile]
  \frametitle{Appending Lists}

  \begin{lstlisting}
(++) :: [a] -> [a] -> [a]
[]     ++ ys = ys
(x:xs) ++ ys = x : (xs ++ ys)
  \end{lstlisting}

  \begin{center}
    \pgfuseimage{append}
  \end{center}

  \begin{itemize}
    \item copy some parts, share some parts
  \end{itemize}
\end{frame}

\begin{frame}[fragile]
  \frametitle{Updating Lists}

  \begin{itemize}
    \item update an element in a list:\\
      \lstinline|update [0,1,2,3,4] 2 7 ~> [0,1,7,3,4]|
  \end{itemize}

  \begin{lstlisting}
update :: [a] -> Int -> a -> [a]
update []     _ _ = error "index out of bounds"
update (_:xs) 0 y = y : xs
update (x:xs) n y = x : update xs (n - 1) y
  \end{lstlisting}

  \begin{itemize}
    \item exercise: draw data structures for above example values
  \end{itemize}
\end{frame}

\subsection{Abstract Data Types}

\begin{frame}
  \frametitle{Abstract Data Types}

  \begin{itemize}
    \item \alert{abstract data type}:
    \smallskip
    \item hidden representation
    \item public operations
  \end{itemize}
\end{frame}

\begin{frame}[fragile]
  \frametitle{Example: Natural Numbers}

  \begin{lstlisting}
module Nat (
  Nat,
  add,         -- Nat -> Nat -> Nat
  sub          -- Nat -> Nat -> Nat
) where
  \end{lstlisting}
\end{frame}

\begin{frame}[fragile]
  \frametitle{Example: Natural Numbers}

  \begin{lstlisting}
data Nat = Zero | Succ Nat
           deriving Show

add :: Nat -> Nat -> Nat
add n        Zero = n
add Zero     n    = n
add (Succ m) n    = Succ (add m n)

sub :: Nat -> Nat -> Nat
sub n         Zero      = n
sub Zero      _         = error "subtract from zero"
sub (Succ n1) (Succ n2) = sub n1 n2
  \end{lstlisting}
\end{frame}

\section{Example: Sets}

\subsection{Interface}

\begin{frame}[fragile]
  \frametitle{Set Interface}

  \begin{lstlisting}[deletekeywords={union}]
module Set (
  Set,
  empty,       -- Set a
  add,         -- Ord a => Set a -> a -> Set a
  makeSet,     -- Ord a => [a] -> Set a
  contains,    -- Ord a => Set a -> a -> Bool
  union,       -- Ord a => Set a -> Set a -> Set a
  card,        -- Set a -> Int
  mapSet       -- Ord b => (a -> b) -> Set a -> Set b
) where
  \end{lstlisting}
\end{frame}

\subsection{List Representation}

\begin{frame}[fragile]
  \frametitle{List Representation}

  \begin{itemize}
    \item using an ordered list of elements without repetition
  \end{itemize}

  \begin{exampleblock}{}
    \begin{lstlisting}
data Set a = OrderedList [a]
             deriving Show
    \end{lstlisting}
  \end{exampleblock}
\end{frame}

\begin{frame}[fragile]
  \frametitle{Empty Set}

  \begin{lstlisting}
empty :: Set a
empty = OrderedList []
  \end{lstlisting}
\end{frame}

\begin{frame}[fragile]
  \frametitle{Adding Elements}

  \begin{lstlisting}[deletekeywords={insert}]
add :: Ord a => Set a -> a -> Set a
add (OrderedList xs) x = OrderedList (insert xs x)

insert :: Ord a => [a] -> a -> [a]
insert []          y = [y]
insert xs@(x':xs') y
  | y < x'           = y  : xs
  | y > x'           = x' : insert xs' y
  | otherwise        = xs
  \end{lstlisting}
\end{frame}

\begin{frame}[fragile]
  \frametitle{Set from List}

  \begin{lstlisting}
makeSet :: Ord a => [a] -> Set a
makeSet = foldl add empty
  \end{lstlisting}
\end{frame}

\begin{frame}[fragile]
  \frametitle{Membership Check}

  \begin{lstlisting}
contains :: Ord a => Set a -> a -> Bool
contains (OrderedList xs) = search xs

search :: Ord a => [a] -> a -> Bool
search []     _ = False
search (x:xs) y
  | y == x      = True
  | y <  x      = False
  | otherwise   = search xs y
  \end{lstlisting}
\end{frame}

\begin{frame}[fragile]
  \frametitle{Set Union}

  \begin{lstlisting}[deletekeywords={union}]
union :: Ord a => Set a -> Set a -> Set a
union s1     (OrderedList []) = s1
union (OrderedList [])     s2 = s2
union (OrderedList (x:xs)) s2 =
    ((OrderedList xs) `union` s2) `add` x
  \end{lstlisting}
\end{frame}

\begin{frame}[fragile]
  \frametitle{Set Cardinality}

  \begin{lstlisting}
card :: Set a -> Int
card = length . makeList

makeList :: Set a -> [a]
makeList (OrderedList xs) = xs
  \end{lstlisting}
\end{frame}

\begin{frame}[fragile]
  \frametitle{Function Mapping}

  \begin{lstlisting}
mapSet :: Ord b => (a -> b) -> Set a -> Set b
mapSet f = makeSet . map f . makeList
  \end{lstlisting}
\end{frame}

\subsection{Tree Representation}

\begin{frame}[fragile]
  \frametitle{Tree Representation}

  \begin{itemize}
    \item using an ordered binary tree of elements without repetition
  \end{itemize}

  \begin{exampleblock}{}
    \begin{lstlisting}
data Set a = Nil | Node a (Set a) (Set a)
             deriving Show
    \end{lstlisting}
  \end{exampleblock}
\end{frame}

\begin{frame}[fragile]
  \frametitle{Empty Set}

  \begin{lstlisting}
empty :: Set a
empty = Nil
  \end{lstlisting}
\end{frame}

\begin{frame}[fragile]
  \frametitle{Adding Elements}

  \begin{lstlisting}[deletekeywords={insert}]
add :: Ord a => Set a -> a -> Set a
add Nil     y = Node y Nil Nil
add s@(Node x left right) y
  | y < x     = Node x (add left y) right
  | y > x     = Node x left (add right y)
  | otherwise = s
  \end{lstlisting}
\end{frame}

\begin{frame}[fragile]
  \frametitle{Set from List}

  \begin{lstlisting}
makeSet :: Ord a => [a] -> Set a
makeSet = foldl add empty
  \end{lstlisting}
\end{frame}

\begin{frame}[fragile]
  \frametitle{Membership Check}

  \begin{lstlisting}
contains :: Ord a => Set a -> a -> Bool
contains Nil _ = False
contains (Node x left right) y
  | y < x      = contains left x
  | y > x      = contains right x
  | otherwise  = True
  \end{lstlisting}
\end{frame}

\begin{frame}[fragile]
  \frametitle{Set Union}

  \begin{lstlisting}[deletekeywords={union}]
union :: Ord a => Set a -> Set a -> Set a
union s1  Nil = s1
union Nil s2  = s2
union (Node x left right) s2 =
    ((left `union` right) `union` s2) `add` x
  \end{lstlisting}
\end{frame}

\begin{frame}[fragile]
  \frametitle{Set Cardinality}

  \begin{lstlisting}
card :: Set a -> Int
card = length . makeList

makeList :: Set a -> [a]
makeList Nil                 = []
makeList (Node x left right) =
    makeList left ++ [x] ++ makeList right
  \end{lstlisting}
\end{frame}

\begin{frame}[fragile]
  \frametitle{Function Mapping}

  \begin{lstlisting}
mapSet :: Ord b => (a -> b) -> Set a -> Set b
mapSet f = makeSet . map f . makeList
  \end{lstlisting}
\end{frame}

\section*{References}

\begin{frame}
  \frametitle{References}

  \begin{block}{Required Reading: Thompson}
    \begin{itemize}
      \item Chapter 16: \alert{Abstract data types}
    \end{itemize}
  \end{block}

  \begin{block}{Recommended Reading: Okasaki}
    \begin{itemize}
      \item Purely Functional Data Structures
    \end{itemize}
  \end{block}
\end{frame}

\end{document}
