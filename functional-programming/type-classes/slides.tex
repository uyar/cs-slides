% Copyright (c) 2013-2014 H. Turgut Uyar <uyar@itu.edu.tr>
%
% This work is licensed under a "Creative Commons
% Attribution-NonCommercial-ShareAlike 4.0 International License".
% For more information, please visit:
% https://creativecommons.org/licenses/by-nc-sa/4.0/

\documentclass[dvipsnames]{beamer}

\usepackage{ae}
\usepackage[T1]{fontenc}
\usepackage[utf8]{inputenc}
\usepackage{pythontex}
\setbeamertemplate{navigation symbols}{}
\setbeamersize{text margin left=2em, text margin right=2em}

\mode<presentation>
{
  \usetheme{default}
  \useinnertheme{rounded}
  \usecolortheme{seahorse}
  \setbeamercovered{transparent}
}

\title{Functional Programming}
\subtitle{Type Classes}

\author{H. Turgut Uyar}
\date{2013-2014}

\AtBeginSubsection[]{
  \begin{frame}<beamer>
    \frametitle{Topics}
    \tableofcontents[currentsection,currentsubsection]
  \end{frame}
}

\theoremstyle{plain}

\pgfdeclareimage[height=1cm]{license}{../license}

\begin{document}

\setpythontexfv[]{frame=single}

\begin{frame}
  \titlepage
\end{frame}

\begin{frame}
  \frametitle{License}

  \pgfuseimage{license}\hfill
  \copyright~2013-2014 H. Turgut Uyar

  \vfill
  \begin{tiny}
    You are free to:
    \begin{itemize}
      \item Share -- copy and redistribute the material in any medium or format
      \item Adapt -- remix, transform, and build upon the material
    \end{itemize}

    Under the following terms:
    \begin{itemize}
      \item Attribution -- You must give appropriate credit, provide a link to
        the license, and indicate if changes were made.\\
        You may do so in any reasonable manner, but not in any way
        that suggests the licensor endorses you or your use.

      \item Noncommercial -- You may not use the material for commercial
        purposes.

      \item Share Alike -- If you remix, transform, or build upon the material,
        you must distribute your contributions\\
        under the same license as the original.
    \end{itemize}
  \end{tiny}

  \vfill
  \begin{small}
    Legal code (the full license):\\
    \url{https://creativecommons.org/licenses/by-nc-sa/4.0/legalcode}
  \end{small}
\end{frame}

\begin{frame}
  \frametitle{Topics}
  \tableofcontents
\end{frame}

\section{Type Classes}

\subsection{Introduction}

\begin{frame}[fragile]
  \frametitle{Overloading}

  \begin{itemize}
    \item check whether an item is an element of a list
  \end{itemize}

  \begin{exampleblock}{}
    \begin{pygments}{haskell}
elem' :: Char -> [Char] -> Bool
elem' _ []     = False
elem' x (c:cs) = if x == c then True else elem' x cs
    \end{pygments}
  \end{exampleblock}

  \pause
  \begin{itemize}
    \item a different function for every type?
    \item better to write it as:\\
      \pygment{haskell}{a -> [a] -> Bool}
    \item the type has to support equality checking
  \end{itemize}
\end{frame}

\begin{frame}[fragile]
  \frametitle{Type Classes}

  \begin{itemize}
    \item \alert{type class}: a collection of types\\
      over which some functions are defined
  \end{itemize}

  \begin{exampleblock}{}
    \begin{pygments}{haskell}
class Eq a where
  (==) :: a -> a -> Bool
    \end{pygments}
  \end{exampleblock}

  \pause
  \begin{itemize}
    \item every type belonging to the class has to implement the functions
    \item members of a type class are called its \alert{instances}
  \end{itemize}
\end{frame}

\begin{frame}[fragile]
  \frametitle{Type Classes}

  \begin{itemize}
    \item basic types like \pygment{haskell}{Bool}, \pygment{haskell}{Char},
      \pygment{haskell}{Integer}, \pygment{haskell}{Float} are instances\\
      of the type class \pygment{haskell}{Eq}
    \item tuples and lists are also instances of \pygment{haskell}{Eq}
  \end{itemize}

  \pause
  \begin{exampleblock}{}
    \begin{pygments}{haskell}
elem' :: Eq a => a -> [a] -> Bool
elem' _ []     = False
elem' x (c:cs) = if x == c then True else elem' x cs
    \end{pygments}
  \end{exampleblock}
\end{frame}

\section*{References}

\begin{frame}
  \frametitle{References}

  \begin{block}{Required Reading: Thompson}
    \begin{itemize}
      \item Chapter 13: \alert{Overloading, type classes and type checking}
    \end{itemize}
  \end{block}
\end{frame}

\end{document}
