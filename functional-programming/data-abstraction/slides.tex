% Copyright (c) 2013
%       H. Turgut Uyar <uyar@itu.edu.tr>
%
% These notes are licensed using the
% "Creative Commons Attribution-NonCommercial-ShareAlike License".
% You are free to copy, distribute and transmit the work, and to adapt the work
% as long as you attribute the authors, do not use it for commercial purposes,
% and any derivative work is under the same or a similar license.
%
% Read the full legal code at:
% http://creativecommons.org/licenses/by-nc-sa/3.0/

\documentclass[dvipsnames]{beamer}

\usepackage{ae}
\usepackage[T1]{fontenc}
\usepackage[utf8]{inputenc}
\usepackage{pythontex}
\setbeamertemplate{navigation symbols}{}

\mode<presentation>
{
  \usetheme{Boadilla}
  \setbeamercovered{transparent}
}

\title{Functional Programming}
\subtitle{Data Abstraction}

\author{H. Turgut Uyar}
\date{2013}

\AtBeginSubsection[]{
  \begin{frame}<beamer>
    \frametitle{Topics}
    \tableofcontents[currentsection,currentsubsection]
  \end{frame}
}

\theoremstyle{plain}

\pgfdeclareimage[width=2cm]{license}{../../license}

\begin{document}

\begin{frame}
  \titlepage
\end{frame}

\begin{frame}
  \frametitle{License}

  \pgfuseimage{license}\hfill
  \copyright 2013 T. Uyar

  \vfill
  \begin{tiny}
    You are free:
    \begin{itemize}
      \item to Share -- to copy, distribute and transmit the work
      \item to Remix -- to adapt the work
    \end{itemize}

    Under the following conditions:
    \begin{itemize}
      \item Attribution -- You must attribute the work in the manner specified by
        the author or licensor (but not in any way that suggests that they
        endorse you or your use of the work).

      \item Noncommercial -- You may not use this work for commercial purposes.

      \item Share Alike -- If you alter, transform, or build upon this work, you
        may distribute the resulting work only under the same or similar license
        to this one.
    \end{itemize}
  \end{tiny}

  \vfill
  Legal code (the full license):\\
  \url{http://creativecommons.org/licenses/by-nc-sa/3.0/}
\end{frame}

\begin{frame}
  \frametitle{Topics}
  \tableofcontents
\end{frame}

\section{Abstract Data Types}

\begin{frame}[fragile]
  \frametitle{Scala List Exercise}

  \begin{block}{Exercise}
    \begin{itemize}
      \item \pygment{scala}{pack}: pack consecutive duplicates of elements
        into sublists
      \item \pygment{scala}{encode}: encode n consecutive duplicates as a pair
    \end{itemize}
  \end{block}

  \begin{example}
    \begin{pygments}{scala}
val m = List("a", "a", "a", "b", "c", "c", "a")
pack(m)
// List(List(a, a, a), List(b), List(c, c), List(a))
encode(m)
// List((a, 3), (b, 1), (c, 2), (a, 1))
    \end{pygments}
  \end{example}
\end{frame}

\begin{frame}<beamer>[fragile]
  \frametitle{Scala List Examples}

  \begin{example}[Scala]
    \begin{pygments}{scala}
def pack[T](xs: List[T]): List[List[T]] =
    xs match {
        case Nil => Nil
        case x :: xs1 => {
            val (first, rest) = xs span ((y: T) => y == x)
            first :: pack(rest)
        }
    }
    \end{pygments}
  \end{example}
\end{frame}

\begin{frame}<beamer>[fragile]
  \frametitle{Scala List Examples}

  \begin{example}[Scala]
    \begin{pygments}{scala}
def encode[T](xs: List[T]): List[(T, Int)] =
    pack(xs) map (ys => (ys.head, ys.length))
}
    \end{pygments}
  \end{example}
\end{frame}

\end{document}
