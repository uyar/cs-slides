% Copyright (c) 2013-2014 H. Turgut Uyar <uyar@itu.edu.tr>
%
% This work is licensed under a "Creative Commons
% Attribution-NonCommercial-ShareAlike 4.0 International License".
% For more information, please visit:
% https://creativecommons.org/licenses/by-nc-sa/4.0/

\documentclass[dvipsnames]{beamer}

\usepackage{ae}
\usepackage[T1]{fontenc}
\usepackage[utf8]{inputenc}
\usepackage{pythontex}
\setbeamertemplate{navigation symbols}{}
\setbeamersize{text margin left=2em, text margin right=2em}

\mode<presentation>
{
  \usetheme{default}
  \useinnertheme{rounded}
  \usecolortheme{seahorse}
  \setbeamercovered{transparent}
}

\title{Functional Programming}
\subtitle{Lazy Evaluation}

\author{H. Turgut Uyar}
\date{2013-2014}

\AtBeginSubsection[]{
  \begin{frame}<beamer>
    \frametitle{Topics}
    \tableofcontents[currentsection,currentsubsection]
  \end{frame}
}

\theoremstyle{plain}

\pgfdeclareimage[height=1cm]{license}{../license}

\begin{document}

\begin{frame}
  \titlepage
\end{frame}

\begin{frame}
  \frametitle{License}

  \pgfuseimage{license}\hfill
  \copyright 2013 T. Uyar

  \vfill
  \begin{tiny}
    You are free:
    \begin{itemize}
      \item to Share -- to copy, distribute and transmit the work
      \item to Remix -- to adapt the work
    \end{itemize}

    Under the following conditions:
    \begin{itemize}
      \item Attribution -- You must attribute the work in the manner specified by
        the author or licensor (but not in any way that suggests that they
        endorse you or your use of the work).

      \item Noncommercial -- You may not use this work for commercial purposes.

      \item Share Alike -- If you alter, transform, or build upon this work, you
        may distribute the resulting work only under the same or similar license
        to this one.
    \end{itemize}
  \end{tiny}

  \vfill
  Legal code (the full license):\\
  \url{http://creativecommons.org/licenses/by-nc-sa/3.0/}
\end{frame}

\begin{frame}
  \frametitle{Topics}
  \tableofcontents
\end{frame}

\section{Expression Evaluation}

\subsection{Strategies}

\begin{frame}
  \frametitle{Expression Evaluation}

  \begin{itemize}
    \item take the operator with the highest precedence
    \item evaluate its operands (note the recursion)
    \item apply the operator to the operands

    \medskip
    \item a name is evaluated by replacing it with its definition
  \end{itemize}
\end{frame}

\begin{frame}
  \frametitle{Expression Evaluation Example}

  \begin{exampleblock}{}
    \pygment{python}{(3.14159 * r) * r}

    \pause
    \medskip
    \pygment{python}{(3.14159 * 2.4) * r}

    \pause
    \medskip
    \pygment{python}{7.539815999999999 * 2.4}

    \pause
    \medskip
    \pygment{python}{18.095558399999998}
  \end{exampleblock}
\end{frame}

\begin{frame}
  \frametitle{Function Evaluation}

  \begin{itemize}
    \item evaluate all function arguments (left to right)
    \item replace the function application\\
      with the right-hand side of its definition
    \item replace the formal parameters of the function\\
      with the actual parameters

    \pause
    \bigskip
    \item \alert{substitution model}
    \begin{itemize}
      \item reduces an expression to a value
    \end{itemize}
  \end{itemize}
\end{frame}

\begin{frame}
  \frametitle{Function Evaluation Example}

  \begin{exampleblock}{}
    \pygment{python}{sum_of_squares(3, 2 + 2)}

    \pause
    \medskip
    \pygment{python}{sum_of_squares(3, 4)}

    \pause
    \medskip
    \pygment{python}{square(3) + square(4)}

    \pause
    \medskip
    \pygment{python}{3 * 3 + square(4)}

    \pause
    \medskip
    \pygment{python}{9 + square(4)}

    \pause
    \medskip
    \pygment{python}{9 + 4 * 4}

    \pause
    \medskip
    \pygment{python}{9 + 16}

    \pause
    \medskip
    \pygment{python}{25}
  \end{exampleblock}
\end{frame}

\begin{frame}
  \frametitle{Evaluation Strategies}

  \begin{itemize}
    \item \alert{eager evaluation}: first evaluate arguments,\\
      then apply the function to the resulting values\\
      (a.k.a. ``call by value'')

    \pause
    \medskip
    \item \alert{normal order evaluation}: do not evaluate arguments\\
      until their values are needed\\
      (a.k.a. ``call by name'')
  \end{itemize}
\end{frame}

\begin{frame}
  \frametitle{Normal Order Evaluation Example}

  \begin{exampleblock}{}
    \pygment{python}{sum_of_squares(3, 2 + 2)}

    \pause
    \medskip
    \pygment{python}{square(3) + square(2 + 2)}

    \pause
    \medskip
    \pygment{python}{3 * 3 + square(2 + 2)}

    \pause
    \medskip
    \pygment{python}{9 + square(2 + 2)}

    \pause
    \medskip
    \pygment{python}{9 + (2 + 2) * (2 + 2)}

    \pause
    \medskip
    \pygment{python}{9 + 4 * (2 + 2)}

    \pause
    \medskip
    \pygment{python}{9 + 4 * 4}

    \pause
    \medskip
    \pygment{python}{9 + 16}

    \pause
    \medskip
    \pygment{python}{25}
  \end{exampleblock}
\end{frame}

\begin{frame}
  \frametitle{Church-Rosser Property}

  \begin{itemize}
    \item eager and normal order evaluations yield the same result\\
      as long as:

    \bigskip
    \item there are no side effects
    \item all evaluations terminate
  \end{itemize}
\end{frame}

\begin{frame}
  \frametitle{Lazy Evaluation}

  \begin{itemize}
    \item eager evaluation evaluates arguments only once
    \item but it might evaluate arguments which are not needed

    \medskip
    \item normal order does not evaluate arguments which are not needed
    \item but it might evaluate others more than once

    \pause
    \medskip
    \item \alert{lazy evaluation}: evaluate an argument once\\
      when its value is \emph{first} needed
  \end{itemize}
\end{frame}

\begin{frame}[fragile]
  \frametitle{Evaluation Strategies}

  \begin{itemize}
    \item most languages use eager evaluation
  \end{itemize}

  \begin{exampleblock}{Python}
    \begin{pygments}{python}
def first(x, y):
    return x

# first(1, 1 // 0) -> division by zero
    \end{pygments}
  \end{exampleblock}
\end{frame}

\begin{frame}[fragile]
  \frametitle{Evaluation Strategies}

  \begin{itemize}
    \item Haskell uses lazy evaluation by default
  \end{itemize}

  \begin{exampleblock}{Haskell}
    \begin{pygments}{haskell}
first :: Integer -> Integer -> Integer
first x y = x

-- first 1 (1 `div` 0) -> 1
    \end{pygments}
  \end{exampleblock}
\end{frame}

\subsection{Short-Circuit Evaluation}

\begin{frame}
  \frametitle{Short-Circuit Evaluation}

  \begin{itemize}
    \item \alert{short-circuit evaluation}:\\
      evaluation stops as soon as the result is determined
  \end{itemize}

  \pause
  \begin{exampleblock}{C}
    \pygment{c}{(a >= 0) && (b < 10)}\\
    \pygment{c}{// 2nd clause not evaluated if a < 0}

    \bigskip
    \pygment{c}{(a >= 0) || (b < 10)}\\
    \pygment{c}{// 2nd clause not evaluated if a >= 0}

    \pause
    \bigskip
    \pygment{c}{(a >= 0) || (b++ < 10)}\\
    \pygment{c}{// dangerous if side effects are allowed}
  \end{exampleblock}
\end{frame}

\begin{frame}[fragile]
  \frametitle{Short-Circuit Evaluation}

  \begin{itemize}
    \item code might depend on short-circuit evaluation
  \end{itemize}

  \begin{exampleblock}{Java}
    \begin{pygments}{java}
// find the index of a key item in a list
index = 0;
while ((index < items.length) && (items[index] != key))
    index++;
    \end{pygments}
  \end{exampleblock}
\end{frame}

\begin{frame}[fragile]
  \frametitle{Short-Circuit Evaluation Example}

  \begin{exampleblock}{}
    \begin{pygments}{haskell}
and' :: Bool -> Bool -> Bool
and' x y = if x then y else False

or' :: Bool -> Bool -> Bool
or' x y = if x then True else y

not' :: Bool -> Bool
not' x = if x then False else True
    \end{pygments}
  \end{exampleblock}
\end{frame}

\subsection{Strictness}

\begin{frame}[fragile]
  \frametitle{Space Behaviour}

  \begin{exampleblock}{}
    \begin{pygments}{haskell}
fact 0 = 1
fact n = n * fact (n - 1)
    \end{pygments}
  \end{exampleblock}

  \begin{itemize}
    \item creates new stack frames
  \end{itemize}
\end{frame}

\begin{frame}[fragile]
  \frametitle{Space Behaviour}

  \begin{exampleblock}{}
    \begin{pygments}{haskell}
factIter acc 0 = acc
factIter acc n = factIter (acc * n) (n - 1)
    \end{pygments}
  \end{exampleblock}

  \begin{itemize}
    \item doesn't compute the multiplication until the last moment\\
      because of lazy evaluation
  \end{itemize}
\end{frame}

\begin{frame}[fragile]
  \frametitle{Space Behaviour}

  \begin{itemize}
    \item possible solution: make the value needed
  \end{itemize}

  \begin{exampleblock}{}
    \begin{pygments}{haskell}
factIter acc 0 = acc
factIter acc n
  | acc == acc = factIter (acc * n) (n - 1)
    \end{pygments}
  \end{exampleblock}
\end{frame}

\begin{frame}[fragile]
  \frametitle{Strictness}

  \begin{itemize}
    \item function to force the evaluation of a parameter:
      \pyv[haskell]{seq}
  \end{itemize}

  \begin{exampleblock}{}
    \begin{pygments}{haskell}
seq :: a -> b -> b
seq x y -- evaluate x and return y

factIter' acc 0 = acc
factIter' acc n = seq acc (factIter' (acc * n) (n - 1))

strict :: (a -> b) -> a -> b
strict f x = seq x (f x)
    \end{pygments}
  \end{exampleblock}
\end{frame}

\section{Infinite Lists}

\subsection{Introduction}

\begin{frame}[fragile]
  \frametitle{Infinite Lists}

  \begin{itemize}
    \item lazy evaluation makes it possible to define infinite data structures
  \end{itemize}

  \begin{exampleblock}{}
    \begin{pygments}{haskell}
ones = 1 : ones

addFirstTwo :: [Integer] -> Integer
addFirstTwo (x1:x2:xs) = x1 + x2
-- addFirstTwo ones -> 2
    \end{pygments}
  \end{exampleblock}
\end{frame}

\begin{frame}[fragile]
  \frametitle{Infinite Lists}

  \begin{itemize}
    \item lazy evaluation makes it possible to define infinite data structures
  \end{itemize}

  \begin{exampleblock}{}
    \begin{pygments}{haskell}
from :: Integer -> [Integer]
from n = n : from (n + 1)

-- from 5 -> [5, 6, 7, 8, ...]
-- OR: [5 ..]
    \end{pygments}
  \end{exampleblock}
\end{frame}

\begin{frame}[fragile]
  \frametitle{Infinite List Example}

  \begin{exampleblock}{sieve of Eratosthenes}
    \begin{pygments}{haskell}
sieve :: [Integer] -> [Integer]
sieve (x:xs) = x : sieve [y | y <- xs, y `mod` x > 0]

-- sieve [2 ..] -> [2, 3, 5, 7, 11, ...]

isPrime :: Integer -> Bool
isPrime n = elem n (sieve [2 ..])
    \end{pygments}
  \end{exampleblock}
\end{frame}

\begin{frame}[fragile]
  \frametitle{Infinite List Example}

  \begin{exampleblock}{Fibonacci sequence}
    \begin{pygments}{haskell}
fibs :: [Integer]
fibs = 1 : 1 : zipWith (+) fibs (tail fibs)
    \end{pygments}
  \end{exampleblock}
\end{frame}

\subsection{Generators}

\subsection{Folding}

\begin{frame}[fragile]
  \frametitle{Folding}

  \begin{exampleblock}{}
    \begin{pygments}{haskell}
foldr f st []     = st
foldr f st (x:xs) = f x (foldr f st xs)
    \end{pygments}
  \end{exampleblock}

  \begin{itemize}
    \item not tail recursive
    \item if f is lazy on its second argument, prefer foldr
    \item \pyv[haskell]{foldr (&&) False (repeat False) -- False}
  \end{itemize}
\end{frame}

\begin{frame}[fragile]
  \frametitle{Folding}

  \begin{exampleblock}{}
    \begin{pygments}{haskell}
foldl f st []     = st
foldl f st (x:xs) = foldl f (f st x) xs
    \end{pygments}
  \end{exampleblock}

  \begin{itemize}
    \item tail recursive
    \item \pyv[haskell]{foldl (&&) False (repeat False) -- never terminates}
  \end{itemize}
\end{frame}

\section*{References}

\begin{frame}
  \frametitle{References}

  \begin{block}{Required Reading: Thompson}
    \begin{itemize}
      \item Chapter 17: \alert{Lazy programming}
      \item Chapter 20: Time and space behaviour
      \begin{itemize}
        \item 20.5: \alert{Folding revisited}
      \end{itemize}
    \end{itemize}
  \end{block}
\end{frame}

\end{document}
