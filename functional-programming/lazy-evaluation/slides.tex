% Copyright (c) 2013-2016 H. Turgut Uyar <uyar@itu.edu.tr>
%
% This work is licensed under a "Creative Commons
% Attribution-NonCommercial-ShareAlike 4.0 International License".
% For more information, please visit:
% https://creativecommons.org/licenses/by-nc-sa/4.0/

\documentclass[dvipsnames]{beamer}

\usepackage[scaled=0.95]{cabin}
\usepackage[scaled=0.88]{beramono}
\usepackage[T1]{fontenc}
\usepackage[utf8]{inputenc}

\usepackage{listings}
\lstset{basicstyle=\ttfamily,
        keywordstyle=\color{blue},
        showstringspaces=false}
\lstdefinestyle{syntax}{frame=single}
\lstset{language=Haskell}

\mode<presentation>
{
  \usetheme{default}
  \useinnertheme{rounded}
  \usecolortheme{seahorse}
  \setbeamercovered{transparent}
}

\title{Functional Programming}
\subtitle{Lazy Evaluation}

\author{H. Turgut Uyar}
\date{2013-2016}

\AtBeginSubsection[]{
  \begin{frame}<beamer>
    \frametitle{Topics}
    \tableofcontents[currentsection,currentsubsection]
  \end{frame}
}

\theoremstyle{plain}

\pgfdeclareimage[height=1cm]{license}{../license}

\begin{document}

\begin{frame}
  \titlepage
\end{frame}

\begin{frame}
  \frametitle{License}

  \pgfuseimage{license}\hfill
  \copyright~2013-2016 H. Turgut Uyar

  \vfill
  \begin{footnotesize}
    You are free to:
    \begin{itemize}
      \itemsep0em
      \item Share -- copy and redistribute the material in any medium or format
      \item Adapt -- remix, transform, and build upon the material
    \end{itemize}

    Under the following terms:
    \begin{itemize}
      \itemsep0em
      \item Attribution -- You must give appropriate credit, provide a link to
        the license, and indicate if changes were made.

      \item NonCommercial -- You may not use the material for commercial
        purposes.

      \item ShareAlike -- If you remix, transform, or build upon the material,
        you must distribute your contributions under the same license as the
        original.
    \end{itemize}

    For more information:\\
    \url{https://creativecommons.org/licenses/by-nc-sa/4.0/}

    \smallskip
    Read the full license:\\
    \url{https://creativecommons.org/licenses/by-nc-sa/4.0/legalcode}
  \end{footnotesize}
\end{frame}

\begin{frame}
  \frametitle{Topics}
  \tableofcontents
\end{frame}

\section{Expression Evaluation}

\subsection{Strategies}

\begin{frame}
  \frametitle{Expression Evaluation}

  \begin{itemize}
    \item reduce an expression to a value
    \item substitution model

    \pause
    \bigskip
    \item take operator with lowest precedence
    \item evaluate its operands (note the recursion)
    \item apply operator to operands
    \item substitute expression with value

    \medskip
    \item evaluating a name: substitute it with its definition
  \end{itemize}
\end{frame}

\begin{frame}
  \frametitle{Expression Evaluation Example}

  \lstinline|(3.14159 * r) * r|

  \medskip
  \lstinline|-- r = 2.4|

  \pause
  \bigskip
  \lstinline|(3.14159 * 2.4) * r|

  \pause
  \medskip
  \lstinline|7.539815999999999 * r|

  \pause
  \medskip
  \lstinline|7.539815999999999 * 2.4|

  \pause
  \medskip
  \lstinline|18.095558399999998|
\end{frame}

\begin{frame}
  \frametitle{Function Evaluation}

  \begin{itemize}
    \item evaluate all actual parameters (left to right)

    \medskip
    \item substitute function application with its definition
    \item substitute formal parameters with actual parameters
  \end{itemize}
\end{frame}

\begin{frame}[fragile]
  \frametitle{Function Evaluation Example}

  \begin{lstlisting}
sqr :: Int -> Int
sqr x = x * x

sumOfSquares :: Int -> Int -> Int
sumOfSquares x y = sqr x + sqr y

-- x = 3, y = 2 + 2
  \end{lstlisting}
\end{frame}

\begin{frame}
  \frametitle{Function Evaluation Example}

  \lstinline|sumOfSquares 3 (2 + 2)|

  \pause
  \medskip
  \lstinline|sumOfSquares 3 4|

  \pause
  \medskip
  \lstinline|sqr 3 + sqr 4|

  \pause
  \medskip
  \lstinline|(3 * 3) + sqr 4 |

  \pause
  \medskip
  \lstinline|9 + sqr 4|

  \pause
  \medskip
  \lstinline|9 + (4 * 4)|

  \pause
  \medskip
  \lstinline|9 + 16|

  \pause
  \medskip
  \lstinline|25|
\end{frame}

\begin{frame}
  \frametitle{Evaluation Strategies}

  \begin{itemize}
    \item \alert{strict}: evaluate parameters, apply function
      (``call by value'')

    \pause
    \medskip
    \item \alert{normal order}: evaluate parameters when needed
      (``call by name'')

    \pause
    \medskip
    \begin{block}{Church-Rosser property}
      result is the same as long as:
      \begin{itemize}
        \item there are no side effects
        \item all evaluations terminate
      \end{itemize}
    \end{block}
  \end{itemize}
\end{frame}

\begin{frame}
  \frametitle{Normal Order Evaluation Example}

  \lstinline|sumOfSquares 3 (2 + 2)|

  \pause
  \medskip
  \lstinline|sqr 3 + sqr (2 + 2)|

  \pause
  \medskip
  \lstinline|(3 * 3) + sqr (2 + 2)|

  \pause
  \medskip
  \lstinline|9 + sqr (2 + 2)|

  \pause
  \medskip
  \lstinline|9 + (2 + 2) * (2 + 2)|

  \pause
  \medskip
  \lstinline|9 + 4 * (2 + 2)|

  \pause
  \medskip
  \lstinline|9 + 4 * 4|

  \pause
  \medskip
  \lstinline|9 + 16|

  \pause
  \medskip
  \lstinline|25|
\end{frame}

\begin{frame}
  \frametitle{Lazy Evaluation}

  \begin{itemize}
    \item strict evaluation evaluates parameters only once
    \item but it might evaluate parameters which are not needed

    \medskip
    \item normal order evaluation doesn't evaluate parameters\\
      which are not needed
    \item but it might evaluate others more than once

    \pause
    \medskip
    \item \alert{lazy evaluation}: evaluate parameter once when
      \emph{first} needed
    \item \alert{memoization}
  \end{itemize}
\end{frame}

\begin{frame}
  \frametitle{Lazy Evaluation Example}

  \lstinline|sumOfSquares 3 (2 + 2)|

  \pause
  \medskip
  \lstinline|sqr 3 + sqr (2 + 2)|

  \pause
  \medskip
  \lstinline|(3 * 3) + sqr (2 + 2)|

  \pause
  \medskip
  \lstinline|9 + sqr (2 + 2)|

  \pause
  \medskip
  \lstinline|9 + (2 + 2) * (2 + 2)|

  \pause
  \medskip
  \lstinline|9 + 4|\pause~~\lstinline|* 4|

  \pause
  \medskip
  \lstinline|9 + 16|

  \pause
  \medskip
  \lstinline|25|
\end{frame}

\begin{frame}[fragile]
  \frametitle{Evaluation Strategies}

  \begin{itemize}
    \item most languages use strict evaluation
  \end{itemize}

  \begin{exampleblock}{Python}
    \begin{lstlisting}[language=Python]
def first(x, y):
    return x

# first(1, 1 // 0) ~> division by zero
    \end{lstlisting}
  \end{exampleblock}
\end{frame}

\begin{frame}[fragile]
  \frametitle{Evaluation Strategies}

  \begin{itemize}
    \item Haskell uses lazy evaluation by default
  \end{itemize}

  \begin{exampleblock}{Haskell}
    \begin{lstlisting}
first :: Int -> Int -> Int
first x y = x

-- first 1 (1 `div` 0) ~> 1
    \end{lstlisting}
  \end{exampleblock}
\end{frame}

\subsection{Short-Circuit Evaluation}

\begin{frame}
  \frametitle{Short-Circuit Evaluation}

  \begin{itemize}
    \item \alert{short-circuit evaluation}:\\
      evaluation stops as soon as result is determined
  \end{itemize}

  \pause
  \begin{exampleblock}{C}
    \lstinline[language=C]!(a >= 0) && (b < 10)!\\
    \lstinline[language=C]!// second clause not evaluated if a < 0!

    \bigskip
    \lstinline[language=C]!(a >= 0) || (b < 10)!\\
    \lstinline[language=C]|// second clause not evaluated if a >= 0|

    \pause
    \bigskip
    \lstinline[language=C]!(a >= 0) || (b++ < 10)!\\
    \lstinline[language=C]!// dangerous!
  \end{exampleblock}
\end{frame}

\begin{frame}[fragile]
  \frametitle{Short-Circuit Evaluation}

  \begin{itemize}
    \item code might depend on short-circuit evaluation
  \end{itemize}

  \begin{exampleblock}{Java}
    \begin{lstlisting}[language=Java]
// find the index of a key item in a list
index = 0;
while ((index < items.length) && (items[index] != key))
    index++;
    \end{lstlisting}
  \end{exampleblock}
\end{frame}

\begin{frame}[fragile]
  \frametitle{Short-Circuit Evaluation Examples}

  \begin{lstlisting}[deletekeywords={and,not,or}]
and :: Bool -> Bool -> Bool
and x y = if x then y else False

or :: Bool -> Bool -> Bool
or x y = if x then True else y
  \end{lstlisting}
\end{frame}

\subsection{Space Complexity}

\begin{frame}[fragile]
  \frametitle{Space Complexity}

  \begin{exampleblock}{}
    \begin{lstlisting}
fac :: Int -> Int
fac 0 = 1
fac n = n * fac (n - 1)
    \end{lstlisting}
  \end{exampleblock}

  \begin{itemize}
    \item not tail-recursive
    \item creates new stack frames
  \end{itemize}
\end{frame}

\begin{frame}[fragile]
  \frametitle{Space Complexity Example}

  \begin{exampleblock}{}
    \begin{lstlisting}
facIter :: Int -> Int -> Int
facIter acc 0 = acc
facIter acc n = facIter (acc * n) (n - 1)
    \end{lstlisting}
  \end{exampleblock}

  \begin{itemize}
    \item tail-recursive
    \item lazy evaluation: doesn't multiply until the last moment
  \end{itemize}
\end{frame}

\begin{frame}[fragile]
  \frametitle{Space Complexity Example}

  \begin{lstlisting}
facIter 1 n
~> facIter (1*n) (n-1)
~> facIter ((1*n)*(n-1)) (n-2)
~> facIter (((1*n)*(n-1))*(n-2)) (n-3)
~> ...
  \end{lstlisting}
\end{frame}

\begin{frame}[fragile]
  \frametitle{Space Complexity}

  \begin{itemize}
    \item possible solution: make the value needed
  \end{itemize}

  \begin{exampleblock}{}
    \begin{lstlisting}
facIter acc 0 = acc
facIter acc n
  | acc == acc = facIter (acc * n) (n - 1)
    \end{lstlisting}
  \end{exampleblock}
\end{frame}

\begin{frame}[fragile]
  \frametitle{Strictness}

  \begin{itemize}
    \item force the evaluation of a parameter:
      \lstinline|seq|
  \end{itemize}

  \begin{exampleblock}{}
    \begin{lstlisting}
seq :: a -> b -> b
seq x y
  | x == x = y  -- evaluate x and return y
    \end{lstlisting}

    \pause
    \medskip
    \begin{lstlisting}
facIter :: Int -> Int -> Int
facIter acc 0 = acc
facIter acc n = seq acc (facIter (acc * n) (n - 1))
    \end{lstlisting}
  \end{exampleblock}
\end{frame}

\begin{frame}[fragile]
  \frametitle{Strictness}

  \begin{itemize}
    \item make a function strict on a parameter:
      \lstinline|strict|
  \end{itemize}

  \begin{exampleblock}{}
    \begin{lstlisting}
strict :: (a -> b) -> a -> b
strict f x = seq x (f x)
    \end{lstlisting}

    \pause
    \medskip
    \begin{lstlisting}
fac :: Int -> Int
fac n = facIter 1 n
  where
    facIter :: Int -> Int -> Int
    facIter acc 0  = acc
    facIter acc n' = strict facIter (acc * n') (n' - 1)
    \end{lstlisting}
  \end{exampleblock}
\end{frame}

\section{Infinite Lists}

\subsection{Introduction}

\begin{frame}[fragile]
  \frametitle{Infinite Lists}

  \begin{itemize}
    \item lazy evaluation makes it possible to work with\\
      infinite data structures
    \item create a list with infinite copies of the same element:\\
      \lstinline|repeat 42 ~> [42, 42, 42, ...]|
  \end{itemize}

  \begin{exampleblock}{}
    \begin{lstlisting}[deletekeywords={repeat}]
repeat :: a -> [a]
repeat x = x : repeat x
    \end{lstlisting}

    \pause
    \medskip
    \begin{lstlisting}
addFirstTwo :: Num a => [a] -> a
addFirstTwo (x1:x2:_) = x1 + x2

-- addFirstTwo (repeat 42) ~> 84
    \end{lstlisting}
  \end{exampleblock}
\end{frame}

\begin{frame}[fragile]
  \frametitle{Infinite Ranges}

  \begin{lstlisting}
from :: Int -> [Int]
from n = n : from (n + 1)

-- from 5 -> [5, 6, 7, 8, ...]

-- OR: [5 ..]
-- addFirstTwo [7 ..] ~> 15
  \end{lstlisting}
\end{frame}

\begin{frame}[fragile]
  \frametitle{Infinite List Example}

  \begin{exampleblock}{Fibonacci sequence}
    \begin{lstlisting}
fibs :: [Int]
fibs = 1 : 1 : zipWith (+) fibs (tail fibs)

-- take 5 fibs ~> [1, 1, 2, 3, 5]
    \end{lstlisting}

% TODO: add drawing

  \end{exampleblock}
\end{frame}

\begin{frame}[fragile]
  \frametitle{Infinite List Example}

  \begin{exampleblock}{sieve of Eratosthenes}
    \begin{lstlisting}
sieve :: [Int] -> [Int]
sieve (x:xs) = x : sieve [y | y <- xs, y `mod` x > 0]

-- sieve [2 ..] -> [2, 3, 5, 7, 11, ...]

primes :: [Int]
primes = sieve [2 ..]
    \end{lstlisting}
  \end{exampleblock}
\end{frame}

\begin{frame}[fragile]
  \frametitle{Infinite List Example}

  \begin{exampleblock}{prime number test}
    \begin{lstlisting}
isPrime :: Int -> Bool
isPrime n = elemOrd n primes

elemOrd :: Ord a => a -> [a] -> Bool
elemOrd n (x:xs)
  | n <  x    = False
  | n == x    = True
  | otherwise = elemOrd n xs
    \end{lstlisting}
  \end{exampleblock}
\end{frame}

\subsection{Generators}

\lstset{language=Python, morekeywords={yield}}

\begin{frame}[fragile]
  \frametitle{Generators}

  \begin{itemize}
    \item Python uses generators for computing values when needed
    \item \lstinline|yield| instead of \lstinline|return|
  \end{itemize}

  \begin{exampleblock}{example}
    \begin{lstlisting}
def repeat(n):
    while True:
        yield n


answers = repeat(42)
for x in answers:
    print(x)    # 42, 42, 42, ...
    \end{lstlisting}
  \end{exampleblock}
\end{frame}

\begin{frame}[fragile]
  \frametitle{Generators}

  \begin{itemize}
    \item next call continues from where previous call left off
  \end{itemize}

  \begin{exampleblock}{example}
    \begin{lstlisting}
def from_(n):
    while True:
        yield n
        n += 1


from5 = from_(5)
for x in from5:
    print(x)    # 5, 6, 7, ...
    \end{lstlisting}
  \end{exampleblock}
\end{frame}

\begin{frame}[fragile]
  \frametitle{Generator Example}

  \begin{exampleblock}{Fibonacci sequence}
    \begin{lstlisting}
def fibs():
    yield 1
    yield 1
    back1, back2 = 1, 1
    while True:
        num = back2 + back1
        yield num
        back1 = back2
        back2 = num
    \end{lstlisting}
  \end{exampleblock}
\end{frame}

\lstset{language=Haskell}

\subsection{Folding}

\begin{frame}[fragile]
  \frametitle{Folding}

  \begin{lstlisting}
foldr :: (a -> b -> b) -> b -> [a] -> b
foldr f s []     = s
foldr f s (x:xs) = f x (foldr f s xs)
  \end{lstlisting}

  \begin{lstlisting}
foldl :: (b -> a -> b) -> b -> [a] -> b
foldl f s []     = s
foldl f s (x:xs) = foldl f (f s x) xs
  \end{lstlisting}

  \medskip
  \begin{itemize}
    \item \lstinline|foldr|: not tail recursive
    \item \lstinline|foldl|: tail recursive
  \end{itemize}
\end{frame}

\begin{frame}[fragile]
  \frametitle{Folding Example}

  \begin{lstlisting}
foldl (*) 1 [1 .. n]
~> foldl (*) (1*1) [2 .. n]
~> foldl (*) ((1*1)*2) [3 .. n]
~> foldl (*) (((1*1)*2)*4) [4 .. n]
~> ...
  \end{lstlisting}
\end{frame}

\begin{frame}[fragile]
  \frametitle{Folding}

  \begin{itemize}
    \item \lstinline|foldl'|: strict \lstinline|foldl|
  \end{itemize}

  \begin{lstlisting}
foldl' :: (b -> a -> b) -> b -> [a] -> b
foldl' f s []     = s
foldl' f s (x:xs) = strict (foldl' f) (f s x) xs
  \end{lstlisting}
\end{frame}

\begin{frame}[fragile]
  \frametitle{Folding Example}

  \begin{lstlisting}
foldl' (*) 1 [1 .. n]
~> foldl' (*) 1 [2 .. n]
~> foldl' (*) 2 [3 .. n]
~> foldl' (*) 6 [4 .. n]
~> ...
  \end{lstlisting}
\end{frame}

\begin{frame}[fragile]
  \frametitle{Folding Example}

  \begin{lstlisting}
foldl (&&) True (repeat False)
~> foldl (&&) True [False, False, False, ...]
~> foldl (&&) ((&&) True False) [False, False, ...]
~> foldl (&&) False [False, False, ...]
~> foldl (&&) ((&&) False False) [False, ...]
~> foldl (&&) False [False, ...]
~> ...
-- never terminates
  \end{lstlisting}
\end{frame}

\begin{frame}[fragile]
  \frametitle{Folding Example}

  \begin{lstlisting}
foldr (&&) True (repeat False)
~> foldr (&&) True [False, False, False, ...]
~> (&&) False (foldr (&&) True [False, False, ...])
~> False
  \end{lstlisting}
\end{frame}

\begin{frame}[fragile]
  \frametitle{Folding Example}

  \begin{lstlisting}
foldr ((:).(+2)) [] [1 .. n]
~> ((:).(+2)) 1 (foldr ((:).(+2)) [] [2 .. n])
~> (1+2) : (foldr ((:).(+2)) [] [2 .. n])
~> 3 : (foldr ((:).(+2)) [] [2 .. n])
~> 3 : ((:).(+2)) 2 (foldr ((:).(+2)) [] [3 .. n])
~> 3 : 4 : (foldr ((:).(+2)) [] [3 .. n])
~> ...
  \end{lstlisting}

  \begin{itemize}
    \item space complexity: $O(1)$
  \end{itemize}
\end{frame}

\begin{frame}[fragile]
  \frametitle{Folding Example}

  \begin{lstlisting}
foldr (*) 1 [1 .. n]
~> (*) 1 (foldr (*) 1 [2 .. n])
~> (*) 1 ((*) 2 (foldr (*) 1 [3 .. n]))
~> (*) 1 ((*) 2 ((*) 3 (foldr (*) 1 [4 .. n]))
~> ...
  \end{lstlisting}

  \begin{itemize}
    \item space complexity: $O(n)$
  \end{itemize}
\end{frame}

\begin{frame}[fragile]
  \frametitle{Folding Strategies}

  \begin{itemize}
    \item if f is lazy on its second argument, prefer \lstinline|foldr|
    \item if the whole list will be traversed, prefer \lstinline|foldl'|
  \end{itemize}
\end{frame}

\section*{References}

\begin{frame}
  \frametitle{References}

  \begin{block}{Required Reading: Thompson}
    \begin{itemize}
      \item Chapter 17: \alert{Lazy programming}
      \item Chapter 20: Time and space behaviour
      \begin{itemize}
        \item 20.5: \alert{Folding revisited}
      \end{itemize}
    \end{itemize}
  \end{block}
\end{frame}

\end{document}
