% Copyright (c) 2013
%       H. Turgut Uyar <uyar@itu.edu.tr>
%
% These notes are licensed using the
% "Creative Commons Attribution-NonCommercial-ShareAlike License".
% You are free to copy, distribute and transmit the work, and to adapt the work
% as long as you attribute the authors, do not use it for commercial purposes,
% and any derivative work is under the same or a similar license.
%
% Read the full legal code at:
% http://creativecommons.org/licenses/by-nc-sa/3.0/

\documentclass[dvipsnames]{beamer}

\usepackage{ae}
\usepackage[T1]{fontenc}
\usepackage[utf8]{inputenc}
\usepackage{pythontex}
\setbeamertemplate{navigation symbols}{}

\mode<presentation>
{
  \usetheme{Boadilla}
  \setbeamercovered{transparent}
}

\title{Functional Programming}
\subtitle{Pattern Matching}

\author{H. Turgut Uyar}
\date{2013}

\AtBeginSubsection[]{
  \begin{frame}<beamer>
    \frametitle{Topics}
    \tableofcontents[currentsection,currentsubsection]
  \end{frame}
}

\theoremstyle{plain}

\pgfdeclareimage[width=2cm]{license}{../../license}

\begin{document}

\begin{frame}
  \titlepage
\end{frame}

\begin{frame}
  \frametitle{License}

  \pgfuseimage{license}\hfill
  \copyright 2013 T. Uyar

  \vfill
  \begin{tiny}
    You are free:
    \begin{itemize}
      \item to Share -- to copy, distribute and transmit the work
      \item to Remix -- to adapt the work
    \end{itemize}

    Under the following conditions:
    \begin{itemize}
      \item Attribution -- You must attribute the work in the manner specified by
        the author or licensor (but not in any way that suggests that they
        endorse you or your use of the work).

      \item Noncommercial -- You may not use this work for commercial purposes.

      \item Share Alike -- If you alter, transform, or build upon this work, you
        may distribute the resulting work only under the same or similar license
        to this one.
    \end{itemize}
  \end{tiny}

  \vfill
  Legal code (the full license):\\
  \url{http://creativecommons.org/licenses/by-nc-sa/3.0/}
\end{frame}

\begin{frame}
  \frametitle{Topics}
  \tableofcontents
\end{frame}

\section{Pattern Matching}

\subsection{Guards}

\begin{frame}[fragile]
  \frametitle{Guards}

  \begin{itemize}
    \item \alert{guard}: check before evaluating function body
  \end{itemize}

  \begin{block}{Haskell}
    \begin{pygments}{haskell}
f fp1 fp2 ...
  | g1 = e1
  | g2 = e2
  | ...
  | gn = en
  | otherwise = eo
    \end{pygments}
  \end{block}
\end{frame}

\begin{frame}[fragile]
  \frametitle{Guards Example}

  \begin{example}[factorial]
    \begin{pygments}{haskell}
fact n
  | n < 0 = error "negative parameter"
  | n == 0 = 1
  | otherwise = n * fact (n - 1)
    \end{pygments}
  \end{example}
\end{frame}

\begin{frame}[fragile]
  \frametitle{Guards Example}

  \begin{example}
    \begin{itemize}
      \item a function that finds the maximum of a list
    \end{itemize}

    \pause
    \begin{pygments}{haskell}
max_ xs
  | null xs = error "empty list"
  | null (tail xs) = head xs
  | head xs > max_ (tail xs) = head xs
  | otherwise = max_ (tail xs)
    \end{pygments}
  \end{example}
\end{frame}

\subsection{Patterns}

\begin{frame}[fragile]
  \frametitle{Patterns}

  \begin{itemize}
    \item \alert{pattern}: check the shape of an expression
  \end{itemize}

  \begin{block}{Haskell case expressions}
    \begin{pygments}{haskell}
case e of
  p1 -> e1
  p2 -> e2
  ...
  pn -> en
  _ -> eo
    \end{pygments}
  \end{block}
\end{frame}

\begin{frame}[fragile]
  \frametitle{Pattern Example}

  \begin{example}[factorial]
    \begin{pygments}{haskell}
fact n =
    case n of
      0 -> 1
      _ -> n * fact (n - 1)
    \end{pygments}
  \end{example}
\end{frame}

\begin{frame}[fragile]
  \frametitle{List Patterns}

  \begin{itemize}
    \item list patterns:\\
      \pygment{haskell}{[] x:xs [n] n:[] [n1,n2]}
  \end{itemize}

  \pause
  \begin{example}
    \begin{itemize}
      \item a function that sums up a list
    \end{itemize}

    \begin{pygments}{haskell}
sum_ xs =
    case xs of
      [] -> 0
      x:xs' -> x + sum_ xs'
    \end{pygments}
  \end{example}
\end{frame}

\begin{frame}[fragile]
  \frametitle{List Pattern Example}

  \begin{example}
    \begin{itemize}
      \item a function that finds the maximum of a list
    \end{itemize}

    \pause
    \begin{pygments}{haskell}
max_ xs =
    case xs of
      [] -> error "empty list"
      [i] -> i
      x:xs' -> if x > max_ xs' then x else max_ xs'
    \end{pygments}
  \end{example}
\end{frame}

\begin{frame}[fragile]
  \frametitle{Pattern Wildcard}

  \begin{itemize}
    \item if binding not needed, use wildcard: \pygment{haskell}{_}
  \end{itemize}

  \begin{example}
    \begin{itemize}
      \item a function that adds the first and third elements of a list
    \end{itemize}

    \pause
    \begin{pygments}{haskell}
first_plus_third xs =
    case xs of
      [] -> 0
      [i] -> i
      [n1,_] -> n1
      x1:_:x3:_ -> x1 + x3
    \end{pygments}
  \end{example}
\end{frame}

\subsection{Multiple Patterns}

\begin{frame}[fragile]
  \frametitle{Multiple Patterns}

  \begin{itemize}
    \item expressions can be matched over multiple patterns
  \end{itemize}

  \begin{example}
    \begin{itemize}
      \item a function that converts three lists into a list of triples
    \end{itemize}

    \pause
    \begin{pygments}{haskell}
zip3_ xs ys zs =
    case (xs,ys,zs) of
      ([],[],[]) -> []
      (x:xs',y:ys',z:zs') -> (x,y,z) : zip3_ xs' ys' zs'
      _ -> error "list sizes don't match"
    \end{pygments}
  \end{example}

  \pause
  \begin{block}{Exercise}
    \begin{itemize}
      \item write a function that converts a list of triples into a triple of
        lists
    \end{itemize}
  \end{block}
\end{frame}

\begin{frame}<beamer>[fragile]
  \frametitle{Nested Pattern Example}

  \begin{example}
    \begin{pygments}{haskell}
unzip3_ xs =
    case xs of
      [] -> ([],[],[])
      (x1,x2,x3):xs' ->
          let
              (xs1,xs2,xs3) = unzip3_ xs'
          in
              (x1:xs1,x2:xs2,x3:xs3)
    \end{pygments}
  \end{example}
\end{frame}

\begin{frame}
  \frametitle{Pattern Exercise}

  \begin{block}{Exercise}
    \begin{itemize}
      \item write a function that checks whether a list is nondecreasing
    \end{itemize}
  \end{block}
\end{frame}

\begin{frame}<beamer>[fragile]
  \frametitle{Pattern Example}

  \begin{example}
    \begin{pygments}{haskell}
nondecreasing xs =
    case xs of
      [] -> True
      _:[] -> True
      x1:x2:xs' -> x1 <= x2 && nondecreasing (x2:xs')
    \end{pygments}
  \end{example}
\end{frame}

\begin{frame}[fragile]
  \frametitle{Patterns}

  \begin{block}{function patterns}
    \begin{pygments}{haskell}
f p1 = e1
f p2 = e2
    \end{pygments}
  \end{block}

  \pause
  \begin{example}[factorial]
    \begin{pygments}{haskell}
fact 0 = 1
fact n = n * fact (n - 1)

sum_ [] = 0
sum_ (x:xs) = x + sum_ xs
    \end{pygments}
  \end{example}
\end{frame}

\section{Algebraic Types}

\subsection{Type Definitions}

\begin{frame}[fragile]
  \frametitle{Type Definitions}

  \begin{block}{Haskell type definition}
    \begin{pygments}{haskell}
data TypeConstructor = ValueConstructor1 components1
                     | ValueConstructor2 components2
                     | ...
    \end{pygments}
  \end{block}

  \pause
  \begin{itemize}
    \item can be used to define enumerations, unions (discriminated records)
    \begin{itemize}
      \item type constructor $\rightarrow$ type name
      \item value constructor $\rightarrow$ tag
      \item getting value $\rightarrow$ projection
    \end{itemize}
    \item different value constructors can contain different components
    \item use pattern matching for tag test and projection
  \end{itemize}
\end{frame}

\begin{frame}[fragile]
  \frametitle{Type Definition Examples}

  \begin{example}
    \begin{pygments}{haskell}
data Month = Jan | Feb | Mar | Apr | May | Jun
           | Jul | Aug | Sep | Oct | Nov | Dec
    \end{pygments}

    \pause
    \bigskip
    \bigskip
    \begin{pygments}{haskell}
data TaggedNumber = Exact Integer | Inexact Float

rounded num =
    case num of
      Exact i -> i
      Inexact r -> round r
    \end{pygments}
  \end{example}
\end{frame}

\begin{frame}[fragile]
  \frametitle{Type Definition Example}

  \begin{example}[one-of type example]
    \begin{pygments}{haskell}
*Main> let fortytwo = Exact 42
*Main> rounded fortytwo
42
*Main> let p = Inexact 3.1415
*Main> rounded p
3
*Main> :t rounded
rounded :: TaggedNumber -> Integer
    \end{pygments}
  \end{example}
\end{frame}

\begin{frame}[fragile]
  \frametitle{Unions in C}

  \begin{itemize}
    \item no tag: \emph{free union}
  \end{itemize}

  \begin{example}
    \begin{pygments}{c}
union untagged_number {
    int ival;
    float rval;
};
    \end{pygments}
  \end{example}
\end{frame}

\begin{frame}[fragile]
  \frametitle{Unions in C}

  \begin{itemize}
    \item projection is unsafe $\rightarrow$ use in conjunction with structs
  \end{itemize}

  \begin{example}
    \begin{pygments}{c}
enum accuracy {EXACT, INEXACT};

struct tagged_number {
    enum accuracy acc;
    union {
        int ival;
        float rval;
    } val;
};
    \end{pygments}
  \end{example}
\end{frame}

\begin{frame}[fragile]
  \frametitle{Union Example}

  \begin{example}
    \begin{pygments}{c}
int rounded(struct tagged_number n)
{
    if (n.acc == EXACT)
        return (int) n.val.ival;
    else if (n.acc == INEXACT)
        return (int) round(n.val.rval);
}
    \end{pygments}
  \end{example}
\end{frame}

\subsection{Type Aliasing}

\begin{frame}[fragile]
  \frametitle{Type Aliasing}

  \begin{itemize}
    \item \alert{type aliasing}: giving an existing type a new name
  \end{itemize}

  \begin{block}{Haskell type aliasing}
    \begin{pygments}{haskell}
type NewName = OldName
    \end{pygments}
  \end{block}
\end{frame}

\begin{frame}[fragile]
  \frametitle{Type Aliasing Example}

  \begin{example}[playing cards]
    \begin{pygments}{haskell}
data Suit = Club | Diamond | Heart | Spade
            deriving Show

data Rank = Jack | Queen | King | Ace | Number Integer
            deriving Show

type Card = (Suit, Rank)
    \end{pygments}
  \end{example}
\end{frame}

\begin{frame}[fragile]
  \frametitle{Algebraic Type Example}

  \begin{example}[playing cards]
    \begin{itemize}
      \item a function that returns the color of a card
    \end{itemize}

    \pause
    \begin{pygments}{haskell}
data Color = Red | Black
             deriving Show

cardColor :: Card -> Color
cardColor card =
    case card of
      (Club, _) -> Black
      (Diamond, _) -> Red
      (Heart, _) -> Red
      (Spade, _) -> Black
    \end{pygments}
  \end{example}
\end{frame}

\begin{frame}
  \frametitle{Algebraic Type Exercise}

  \begin{block}{Exercise}
    \begin{itemize}
      \item write a function that will return the value of a card
      \begin{itemize}
        \item ace: 11, face cards: 10
      \end{itemize}
    \end{itemize}
  \end{block}
\end{frame}

\begin{frame}<beamer>[fragile]
  \frametitle{Algebraic Type Example}

  \begin{example}[value of playing cards]
    \begin{pygments}{haskell}
cardValue card =
    case card of
      (_, Ace) -> 11
      (_, King) -> 10
      (_, Queen) -> 10
      (_, Jack) -> 10
      (_, Number n) -> n
    \end{pygments}
  \end{example}
\end{frame}

\subsection{Parameterized Types}

\begin{frame}[fragile]
  \frametitle{Parameterized Types}

  \begin{example}[Haskell]
    \begin{pygments}{haskell}
type Triple a = (a, a, a)

type Coordinates = Triple Double

distance :: Coordinates -> Double
distance (x,y,z) = sqrt (x * x + y * y + z * z)

type Date = Triple Integer

year :: Date -> Integer
year (y,_,_) = y
    \end{pygments}
  \end{example}
\end{frame}

\begin{frame}[fragile]
  \frametitle{Uncertainty Types}

  \begin{itemize}
    \item some type or empty: \pygment{haskell}{Maybe}
    \item one of two types: \pygment{haskell}{Either}
  \end{itemize}

  \begin{block}{Haskell}
    \begin{pygments}{haskell}
data Maybe a = Nothing | Just a

data Either a b = Left a | Right b
    \end{pygments}
  \end{block}
\end{frame}

\begin{frame}
  \frametitle{Maybe Type Exercise}

  \begin{block}{Exercise}
    \begin{itemize}
      \item write a function that returns the maximum element of a list
      \begin{itemize}
        \item if empty: Nothing
        \item otherwise: Just x
      \end{itemize}
    \end{itemize}
  \end{block}
\end{frame}

\begin{frame}<beamer>[fragile]
  \frametitle{Maybe Type Example}

  \begin{example}[maximum of a list]
    \begin{pygments}{haskell}
max_ xs =
    let
       maxIter acc items =
           case items of
             [] -> acc
             item:items' -> if item > acc
                            then maxIter item items'
                            else maxIter acc items'
    in
        case xs of
          [] -> Nothing
          x:xs' -> Just (maxIter x xs')
    \end{pygments}
  \end{example}
\end{frame}

% \subsection{Recursive Types}
% 
% \begin{frame}[fragile]
%   \frametitle{Recursive Types}
% 
%   \begin{example}[Haskell]
%     \begin{pygments}{haskell}
% data List_ a = Empty | Cons a (List_ a)
%                deriving Show
%     \end{pygments}
% 
%     \pause
%     \begin{pygments}{haskell}
% null_ xs =
%     case xs of
%       Empty -> True
%       _ -> False
% 
% head_ xs =
%     case xs of
%       Empty -> Nothing
%       Cons x _ -> Just x
%     \end{pygments}
%   \end{example}
% \end{frame}
% 
% \begin{frame}
%   \frametitle{Recursive Type Exercise}
% 
%   \begin{block}{Exercise}
%     \begin{itemize}
%       \item write a function that appends two lists
%     \end{itemize}
%   \end{block}
% \end{frame}
% 
% \begin{frame}<beamer>[fragile]
%   \frametitle{Recursive Type Example}
% 
%   \begin{example}
%     \begin{pygments}{haskell}
% append xs ys =
%     case xs of
%       Empty -> ys
%       Cons x xs' -> Cons x (append xs' ys)
%     \end{pygments}
%   \end{example}
% \end{frame}
% 
% \section*{References}
% 
% \begin{frame}
%   \frametitle{References}
% 
%   \begin{block}{Required Reading: Abelson, Sussman}
%     \begin{itemize}
%       \item Chapter 1: Building Abstractions with Procedures
%       \begin{itemize}
%         \item 1.1. \alert{The Elements of Programming}
%       \end{itemize}
%     \end{itemize}
%   \end{block}
% \end{frame}

\end{document}
