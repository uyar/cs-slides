% Copyright (c) 2013-2015 H. Turgut Uyar <uyar@itu.edu.tr>
%
% This work is licensed under a "Creative Commons
% Attribution-NonCommercial-ShareAlike 4.0 International License".
% For more information, please visit:
% https://creativecommons.org/licenses/by-nc-sa/4.0/

\documentclass[dvipsnames]{beamer}

\usepackage[scaled=0.95]{cabin}
\usepackage[scaled=0.88]{beramono}
\usepackage[T1]{fontenc}
\usepackage[utf8]{inputenc}

\usepackage{listings}
\lstset{basicstyle=\ttfamily,
        keywordstyle=\color{blue},
        showstringspaces=false}
\lstdefinestyle{syntax}{frame=single}
\lstset{language=Haskell}

\mode<presentation>
{
  \usetheme{default}
  \useinnertheme{rounded}
  \usecolortheme{seahorse}
  \setbeamercovered{transparent}
}

\title{Functional Programming}
\subtitle{Pattern Matching}

\author{H. Turgut Uyar}
\date{2013-2015}

\AtBeginSubsection[]{
  \begin{frame}<beamer>
    \frametitle{Topics}
    \tableofcontents[currentsection,currentsubsection]
  \end{frame}
}

\theoremstyle{plain}

\pgfdeclareimage[height=1cm]{license}{../license}

\begin{document}

\begin{frame}
  \titlepage
\end{frame}

\begin{frame}
  \frametitle{License}

  \pgfuseimage{license}\hfill
  \copyright~2013-2015 H. Turgut Uyar

  \vfill
  \begin{footnotesize}
    You are free to:
    \begin{itemize}
      \itemsep0em
      \item Share -- copy and redistribute the material in any medium or format
      \item Adapt -- remix, transform, and build upon the material
    \end{itemize}

    Under the following terms:
    \begin{itemize}
      \itemsep0em
      \item Attribution -- You must give appropriate credit, provide a link to
        the license, and indicate if changes were made.

      \item NonCommercial -- You may not use the material for commercial
        purposes.

      \item ShareAlike -- If you remix, transform, or build upon the material,
        you must distribute your contributions under the same license as the
        original.
    \end{itemize}

    For more information:\\
    \url{https://creativecommons.org/licenses/by-nc-sa/4.0/}

    \smallskip
    Read the full license:\\
    \url{https://creativecommons.org/licenses/by-nc-sa/4.0/legalcode}
  \end{footnotesize}
\end{frame}

\begin{frame}
  \frametitle{Topics}
  \tableofcontents
\end{frame}

\lstset{deletekeywords={gcd}}

\section{Data Types}

\subsection{Tuples}

\begin{frame}[fragile]
  \frametitle{Tuples}

  \begin{itemize}
    \item \alert{tuple}: a combination of a fixed number of values
    \item different but fixed types

    \smallskip
    \begin{lstlisting}[style=syntax]
n :: (t1, t2, ..., tn)
n = (e1, e2, ..., en)
    \end{lstlisting}

    \medskip
    \item selector functions on pairs:\\
      \lstinline|fst|, \lstinline|snd|
  \end{itemize}
\end{frame}

\begin{frame}[fragile]
  \frametitle{Tuple Example}

  \begin{exampleblock}{representing a term in a polynomial: $2.4x^2$}
    \begin{lstlisting}
t :: (Float, Integer)
t = (2.4, 2)

-- fst t ~> 2.4
-- snd t ~> 2
    \end{lstlisting}
  \end{exampleblock}
\end{frame}

\begin{frame}
  \frametitle{Tuple Parameters}

  \begin{itemize}
    \item tuples can be sent as parameters
    \item not the same as multiple parameters

    \bigskip
    \item tuples can be returned as result
  \end{itemize}
\end{frame}

\begin{frame}[fragile]
  \frametitle{Tuple Parameter Example}

  \begin{exampleblock}{}
    \begin{lstlisting}
gcd :: Integer -> Integer -> Integer
gcd x y
  | y == 0    = x
  | otherwise = gcd y (x `mod` y)

gcd' :: (Integer, Integer) -> Integer
gcd' a
  | snd a == 0 = fst a
  | otherwise  = gcd' (snd a, (fst a) `mod` (snd a))

-- gcd   9702  945
-- gcd' (9702, 945)
    \end{lstlisting}
  \end{exampleblock}
\end{frame}

\begin{frame}[fragile]
  \frametitle{Tuple Result Example}

  \begin{exampleblock}{greatest common divisor and least common multiple}
    \begin{lstlisting}
gcd_lcm :: Integer -> Integer -> (Integer, Integer)
gcd_lcm x y = (d, m)
  where
    d = gcd x y
    m = (x * y) `div` d
    \end{lstlisting}
  \end{exampleblock}
\end{frame}

\subsection{Lists}

\begin{frame}[fragile]
  \frametitle{Lists}

  \begin{itemize}
    \item \alert{list}: a combination of an arbitrary number of values
    \item all of the same type

    \smallskip
    \begin{lstlisting}[style=syntax]
n :: [t]
n = [e1, e2, ..., en]
    \end{lstlisting}
  \end{itemize}
\end{frame}

\begin{frame}[fragile]
  \frametitle{List Example}

  \begin{exampleblock}{representing a polynomial: $2.4x^2 + 1.8x - 4.6$}
    \begin{lstlisting}
-- second degree polynomial
p1 :: (Float, Float, Float)
p1 = (2.4, 1.8, -4.6)

-- any degree polynomial, coefficients ordered
p2 :: [Float]
p2 = [2.4, 1.8, -4.6]

-- any degree polynomial, coefficients unordered
p3 :: [(Float, Integer)]
p3 = [(1.8, 1), (-4.6, 0), (2.4, 2)]
    \end{lstlisting}
  \end{exampleblock}
\end{frame}

\begin{frame}[fragile]
  \frametitle{Type Synonyms}

  \begin{itemize}
    \item type synonym: giving an existing type a new name
    \smallskip
    \begin{lstlisting}
type newName = oldName
    \end{lstlisting}
  \end{itemize}

  \pause
  \begin{exampleblock}{example: representing a polynomial}
    \begin{lstlisting}
type Term = (Float, Integer)
type Polynomial = [Term]

p4 :: Polynomial
p4 = [(1.8, 1), (-4.6, 0), (2.4, 2)]
    \end{lstlisting}
  \end{exampleblock}
\end{frame}

\begin{frame}
  \frametitle{Lists}

  \begin{itemize}
    \item a list consists of a first item (\alert{head})\\
      followed by a list of the remaining items (\alert{tail})

    \medskip
    \item check if empty: \lstinline|null|
    \item get the head: \lstinline|head|
    \item get the tail: \lstinline|tail|
    \item construct a list: \lstinline|item : sublist|

    \pause
    \medskip
    \item independent of type: \lstinline|[a]|
  \end{itemize}
\end{frame}

\begin{frame}[fragile]
  \frametitle{List Operation Examples}

  \begin{lstlisting}
null :: [a] -> Bool
-- null []           ~> True
-- null [1, 2, 3, 4] ~> False

head :: [a] -> a
-- head [1, 2, 3, 4] ~> 1
-- head []           ~> error
-- head [1]          ~> 1

tail :: [a] -> [a]
-- tail [1, 2, 3, 4] ~> [2, 3, 4]
-- tail []           ~> error
-- tail [1]          ~> []

(:) :: a -> [a] -> [a]
-- 1 : [2, 3]        ~> [1, 2, 3]
  \end{lstlisting}
\end{frame}

\begin{frame}[fragile]
  \frametitle{List Example}

  \begin{exampleblock}{number of elements}
    \begin{lstlisting}[deletekeywords={length}]
length :: [a] -> Int
length xs
  | null xs   = 0
  | otherwise = 1 + length (tail xs)
    \end{lstlisting}
  \end{exampleblock}
\end{frame}

\begin{frame}[fragile]
  \frametitle{List Example}

  \begin{exampleblock}{number of elements (tail recursive)}
    \begin{lstlisting}[deletekeywords={length}]
length :: [a] -> Int
length xs = lengthIter 0 xs
  where
    lengthIter :: Int -> [a] -> Int
    lengthIter acc xs'
      | null xs'  = acc
      | otherwise = lengthIter (acc + 1) (tail xs')
    \end{lstlisting}
  \end{exampleblock}
\end{frame}

\begin{frame}[fragile]
  \frametitle{List Example}

  \begin{exampleblock}{sum of elements}
    \begin{lstlisting}[deletekeywords={sum}]
sum :: [Integer] -> Integer
sum xs
  | null xs   = 0
  | otherwise = head xs + sum (tail xs)
    \end{lstlisting}
  \end{exampleblock}
\end{frame}

\begin{frame}[fragile]
  \frametitle{List Example}

  \begin{exampleblock}{sum of first two elements}
    \begin{lstlisting}
firstPlusSecond :: [Integer] -> Integer
firstPlusSecond xs
  | null xs        = 0
  | null (tail xs) = head xs
  | otherwise      = head xs + head (tail xs)
    \end{lstlisting}
  \end{exampleblock}
\end{frame}

\begin{frame}[fragile]
  \frametitle{Strings}

  \begin{itemize}
    \item a string is a list of characters
    \begin{lstlisting}
type String = [Char]
    \end{lstlisting}
  \end{itemize}

  \begin{exampleblock}{examples}
    \begin{lstlisting}
-- head "word"   ~> 'w'
-- tail "word"   ~> "ord"
-- null "word"   ~> False
-- null ""       ~> True
-- length "word" ~> 4
    \end{lstlisting}
  \end{exampleblock}
\end{frame}

\subsection{Algebraic Types}

\begin{frame}[fragile]
  \frametitle{Algebraic Types}

  \begin{itemize}
    \item \alert{algebraic types}: constructors and components
    \begin{lstlisting}[style=syntax]
data T = C1 t11 t12 ... t1m |
         C2 t21 t22 ... t2n |
         ...
    \end{lstlisting}
    \item value construction: \lstinline|Ci ei1 ei2 ... eik|
    \item constructors are functions
  \end{itemize}
\end{frame}

\begin{frame}[fragile]
  \frametitle{Algebraic Type Examples}

  \begin{exampleblock}{simple product type}
    \begin{lstlisting}
type Name = String
type Year = Integer

data Human = Person Name Year
             deriving Show

church :: Human
church = Person "Alonzo Church" 1903
    \end{lstlisting}
  \end{exampleblock}
\end{frame}

\begin{frame}[fragile]
  \frametitle{Algebraic Type Examples}

  \begin{exampleblock}{enumeration}
    \begin{lstlisting}
data Month = Jan | Feb | Mar | Apr | May | Jun |
             Jul | Aug | Sep | Oct | Nov | Dec
             deriving Show

m :: Month
m = Feb
    \end{lstlisting}
  \end{exampleblock}
\end{frame}

\begin{frame}[fragile]
  \frametitle{Algebraic Type Examples}

  \begin{exampleblock}{multiple options}
    \begin{lstlisting}
type Coords = (Float, Float)
type Length = Float

data Shape = Point Coords |
             Circle Coords Length |
             Rectangle Coords Length Length
             deriving Show

p, c, r :: Shape
p = Point (0.0, 0.0)
c = Circle (0.0, 0.0) 1.0
r = Rectangle (45.9, 87.6) 5.75 2.3
    \end{lstlisting}
  \end{exampleblock}
\end{frame}

\section{Pattern Matching}

\subsection{Patterns}

\begin{frame}[fragile]
  \frametitle{Patterns}

  \begin{itemize}
    \item expressions can be checked against patterns
    \item result is the expression for the first matched pattern
    \begin{lstlisting}[style=syntax]
case expr of
    p1 -> e1
    p2 -> e2
    ...
    pn -> en
    _  -> e
    \end{lstlisting}
    \item matched patterns generate bindings
  \end{itemize}
\end{frame}

\begin{frame}[fragile]
  \frametitle{Pattern Examples}

  \begin{exampleblock}{literal value}
    \begin{lstlisting}
gcd :: Integer -> Integer -> Integer
gcd x y = case y of
    0 -> x
    _ -> gcd y (x `mod` y)
    \end{lstlisting}
  \end{exampleblock}
\end{frame}

\begin{frame}[fragile]
  \frametitle{Pattern Examples}

  \begin{exampleblock}{tuple matching}
    \begin{lstlisting}
gcd' :: (Integer, Integer) -> Integer
gcd' a = case a of
    (x, 0) -> x
    (x, y) -> gcd' (y, x `mod` y)

-- gcd' (9702, 945)
-- second pattern, bindings: x <-> 9702, y <-> 945

-- gcd' (63, 0)
-- first pattern, bindings: x <-> 63
    \end{lstlisting}
  \end{exampleblock}
\end{frame}

\begin{frame}[fragile]
  \frametitle{Nested Patterns}

  \begin{itemize}
    \item patterns can be nested
  \end{itemize}

  \begin{exampleblock}{example}
    \begin{lstlisting}
shift :: ((a, b), c) -> (a, (b, c))
shift s = case s of
    ((x, y), z) -> (x, (y, z))
    \end{lstlisting}
  \end{exampleblock}
\end{frame}

\begin{frame}[fragile]
  \frametitle{Wildcards}

  \begin{itemize}
    \item if binding not needed, use wildcard: \lstinline|_|
  \end{itemize}

  \begin{exampleblock}{example: third component of a triple}
    \begin{lstlisting}
third :: (a, b, c) -> c
third t = case t of
    (x, y, z) -> z

-- OR:
third t = case t of
    (_, _, z) -> z
    \end{lstlisting}
  \end{exampleblock}
\end{frame}

\subsection{List Patterns}

\begin{frame}[fragile]
  \frametitle{List Patterns}

  \begin{itemize}
    \item empty list:\\
      \lstinline|[]|
    \item nonempty list:\\
      \lstinline|x:xs|
    \item list with exactly one element:\\
      \lstinline|[x]|
    \item list with exactly two elements:\\
      \lstinline|[x1, x2]|
    \item list with at least two elements:\\
      \lstinline|x1:x2:xs|
  \end{itemize}
\end{frame}

\begin{frame}[fragile]
  \frametitle{List Pattern Examples}

  \begin{exampleblock}{number of elements}
    \begin{lstlisting}[deletekeywords={length}]
length :: [a] -> Int
length xs = case xs of
    []    -> 0
    x:xs' -> 1 + length xs'
    \end{lstlisting}
  \end{exampleblock}
\end{frame}

\begin{frame}[fragile]
  \frametitle{List Pattern Examples}

  \begin{exampleblock}{sum of the first and third elements}
    \begin{lstlisting}
firstPlusThird :: [Integer] -> Integer
firstPlusThird xs = case xs of
    []        -> 0
    [x1]      -> x1
    [x1, _]   -> x1
    x1:_:x3:_ -> x1 + x3
    \end{lstlisting}
  \end{exampleblock}
\end{frame}

\subsection{Algebraic Type Patterns}

\begin{frame}
  \frametitle{Algebraic Type Patterns}

  \begin{itemize}
    \item patterns can match algebraic types
    \item use pattern matching to get values out of product types
  \end{itemize}
\end{frame}

\begin{frame}[fragile]
  \frametitle{Algebraic Type Pattern Examples}

  \begin{exampleblock}{get component out of product type}
    \begin{lstlisting}
birthYear :: Human -> Year
birthYear p = case p of
    Person _ y -> y

-- birthYear (Person "Alonzo Church" 1903) ~> 1903
-- binding: y <-> 1903
    \end{lstlisting}
  \end{exampleblock}
\end{frame}

\begin{frame}[fragile]
  \frametitle{Algebraic Type Pattern Examples}

  \begin{exampleblock}{number of days in a month}
    \begin{lstlisting}
daysInMonth :: Month -> Integer -> Integer
daysInMonth m y = case m of
    Apr -> 30
    Jun -> 30
    Sep -> 30
    Nov -> 30
    Feb -> if y `mod` 4 == 0 then 29 else 28
    _   -> 31

-- daysInMonth Jan 2014 ~> 31
-- daysInMonth Feb 2014 ~> 28
-- daysInMonth Feb 2016 ~> 29
    \end{lstlisting}
  \end{exampleblock}
\end{frame}

\begin{frame}[fragile]
  \frametitle{Algebraic Type Pattern Examples}

  \begin{exampleblock}{area of a geometric shape}
    \begin{lstlisting}
area :: Shape -> Float
area s = case s of
    Point     _     -> 0.0
    Circle    _ r   -> 3.14159 * r * r
    Rectangle _ h w -> h * w

-- area (Circle (0.0, 0.0) 3.0) ~> 28.274311
-- second pattern, binding: r <-> 3.0
    \end{lstlisting}
  \end{exampleblock}
\end{frame}

\subsection{Function Patterns}

\begin{frame}[fragile]
  \frametitle{Function Patterns}

  \begin{itemize}
    \item formal parameters are patterns
    \item components of the pattern will be matched\\
      with the components of the actual parameters

    \medskip
    \item in case of multiple patterns, the first match will be selected

    \smallskip
    \begin{lstlisting}[style=syntax]
n p1 = e1
n p2 = e2
...
    \end{lstlisting}
  \end{itemize}
\end{frame}

\begin{frame}[fragile]
  \frametitle{Function Pattern Example}

  \begin{lstlisting}
gcd :: Integer -> Integer -> Integer
gcd x y = case y of
    0 -> x
    _ -> gcd y (x `mod` y)

-- OR:
gcd :: Integer -> Integer -> Integer
gcd x 0 = x
gcd x y = gcd y (x `mod` y)
  \end{lstlisting}
\end{frame}

\begin{frame}[fragile]
  \frametitle{Function Pattern Example}

  \begin{lstlisting}
gcd' :: (Integer, Integer) -> Integer
gcd' a = case a of
    (x, 0) -> x
    (x, y) -> gcd' (y, x `mod` y)

-- OR:
gcd' :: (Integer, Integer) -> Integer
gcd' (x, 0) = x
gcd' (x, y) = gcd' (y, x `mod` y)
  \end{lstlisting}
\end{frame}

\begin{frame}[fragile]
  \frametitle{Function Pattern Example}

  \begin{lstlisting}
shift :: ((a, b), c) -> (a, (b, c))
shift s = case s of
    ((x, y), z) -> (x, (y, z))

-- OR:
shift :: ((a, b), c) -> (a, (b, c))
shift ((x, y), z) = (x, (y, z))
  \end{lstlisting}
\end{frame}

\begin{frame}[fragile]
  \frametitle{Function Pattern Example}

  \begin{lstlisting}
third :: (a, b, c) -> c
third t = case t of
    (_ , _, z) -> z

-- OR:
third :: (a, b, c) -> c
third (_, _, z) = z
  \end{lstlisting}
\end{frame}

\begin{frame}[fragile]
  \frametitle{Function Pattern Example}

  \begin{lstlisting}[deletekeywords={length}]
length :: [a] -> Int
length xs = case xs of
    []    -> 0
    x:xs' -> 1 + length xs'

-- OR:
length :: [a] -> Int
length []     = 0
length (x:xs) = 1 + length xs
  \end{lstlisting}
\end{frame}

\begin{frame}[fragile]
  \frametitle{Function Pattern Example}

  \begin{lstlisting}
birthYear :: Human -> Year
birthYear p = case p of
    Person _ y -> y

-- OR:
birthYear :: Human -> Year
birthYear (Person _ y) = y
  \end{lstlisting}
\end{frame}

\begin{frame}[fragile]
  \frametitle{Example}

  \begin{exampleblock}{efficient Fibonacci calculation}
    \begin{lstlisting}
fibStep :: (Integer, Integer) -> (Integer, Integer)
fibStep (u, v) = (v, u + v)

-- fibPair n ~> (fib n, fib (n + 1))
fibPair :: Integer -> (Integer, Integer)
fibPair 1 = (1, 1)
fibPair n = fibStep (fibPair (n - 1))

fastFib n = fst (fibPair n)
    \end{lstlisting}
  \end{exampleblock}
\end{frame}

\lstset{deletekeywords={Rational}}

\begin{frame}[fragile]
  \frametitle{Tuples or Algebraic Types?}

  \begin{exampleblock}{representing rational numbers}
    \begin{lstlisting}
type Rational = (Integer, Integer)

simplify :: Rational -> Rational
simplify (n, d) = (n `div` g, d `div` g)
  where
    g = gcd n d

type DayInYear = (Integer, Integer)

mar12 :: DayInYear
mar12 = (12, 3)
-- simplify mar12 ~> (4, 1)
    \end{lstlisting}
  \end{exampleblock}
\end{frame}

\begin{frame}[fragile]
  \frametitle{Tuples or Algebraic Types?}

  \begin{itemize}
    \item algebraic types give you better type checking
  \end{itemize}

  \bigskip
  \begin{lstlisting}
data Rational' = Fraction Integer Integer
                 deriving Show

simplify' :: Rational' -> Rational'
simplify' (Fraction n d) =
  Fraction (n `div` g) (d `div` g)
    where
      g = gcd n d
  \end{lstlisting}
\end{frame}

% TODO: add example for using case expression over function pattern

\section*{References}

\begin{frame}
  \frametitle{References}

  \begin{block}{Required Reading: Thompson}
    \begin{itemize}
      \item Chapter 5: \alert{Data types, tuples and lists}
    \end{itemize}
  \end{block}
\end{frame}

\end{document}
