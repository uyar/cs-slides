% Copyright (c) 2013
%       H. Turgut Uyar <uyar@itu.edu.tr>
%
% These notes are licensed using the
% "Creative Commons Attribution-NonCommercial-ShareAlike License".
% You are free to copy, distribute and transmit the work, and to adapt the work
% as long as you attribute the authors, do not use it for commercial purposes,
% and any derivative work is under the same or a similar license.
%
% Read the full legal code at:
% http://creativecommons.org/licenses/by-nc-sa/3.0/

\documentclass[dvipsnames]{beamer}

\usepackage{ae}
\usepackage[T1]{fontenc}
\usepackage[utf8]{inputenc}
\usepackage{pythontex}
\setbeamertemplate{navigation symbols}{}

\mode<presentation>
{
  \usetheme{Boadilla}
  \setbeamercovered{transparent}
}

\title{Functional Programming}
\subtitle{Pattern Matching}

\author{H. Turgut Uyar}
\date{2013}

\AtBeginSubsection[]{
  \begin{frame}<beamer>
    \frametitle{Topics}
    \tableofcontents[currentsection,currentsubsection]
  \end{frame}
}

\theoremstyle{plain}

\pgfdeclareimage[width=2cm]{license}{../../license}

\begin{document}

\begin{frame}
  \titlepage
\end{frame}

\begin{frame}
  \frametitle{License}

  \pgfuseimage{license}\hfill
  \copyright 2013 T. Uyar

  \vfill
  \begin{tiny}
    You are free:
    \begin{itemize}
      \item to Share -- to copy, distribute and transmit the work
      \item to Remix -- to adapt the work
    \end{itemize}

    Under the following conditions:
    \begin{itemize}
      \item Attribution -- You must attribute the work in the manner specified by
        the author or licensor (but not in any way that suggests that they
        endorse you or your use of the work).

      \item Noncommercial -- You may not use this work for commercial purposes.

      \item Share Alike -- If you alter, transform, or build upon this work, you
        may distribute the resulting work only under the same or similar license
        to this one.
    \end{itemize}
  \end{tiny}

  \vfill
  Legal code (the full license):\\
  \url{http://creativecommons.org/licenses/by-nc-sa/3.0/}
\end{frame}

\begin{frame}
  \frametitle{Topics}
  \tableofcontents
\end{frame}

\section{Pattern Matching}

\subsection{Guards}

\begin{frame}[fragile]
  \frametitle{Guards Example}

  \begin{example}[maximum of a list]
    \pause
    \begin{pygments}{haskell}
max' xs
  | null xs = error "empty list"
  | null (tail xs) = head xs
  | head xs > max' (tail xs) = head xs
  | otherwise = max' (tail xs)
    \end{pygments}
  \end{example}
\end{frame}

\subsection{Multiple Patterns}

\begin{frame}[fragile]
  \frametitle{Multiple Patterns}

  \begin{itemize}
    \item expressions can be matched over multiple patterns
  \end{itemize}

  \begin{example}[convert three lists into a list of triples]
    \pause
    \begin{pygments}{haskell}
zip3' xs ys zs =
    case (xs,ys,zs) of
      ([],[],[]) -> []
      (x:xs',y:ys',z:zs') -> (x,y,z) : zip3' xs' ys' zs'
      _ -> error "list sizes don't match"
    \end{pygments}
  \end{example}

  \pause
  \begin{block}{Exercise}
    \begin{itemize}
      \item write a function that converts a list of triples into a triple of
        lists
    \end{itemize}
  \end{block}
\end{frame}

\begin{frame}<beamer>[fragile]
  \frametitle{Nested Pattern Example}

  \begin{example}[convert a list of triples into a triple of lists]
    \begin{pygments}{haskell}
unzip3' xs =
    case xs of
      [] -> ([],[],[])
      (x1,x2,x3):xs' -> (x1:xs1,x2:xs2,x3:xs3)
                        where (xs1,xs2,xs3) = unzip3' xs'
    \end{pygments}
  \end{example}
\end{frame}

\begin{frame}
  \frametitle{Pattern Exercise}

  \begin{block}{Exercise}
    \begin{itemize}
      \item write a function that checks whether a list is nondecreasing
    \end{itemize}
  \end{block}
\end{frame}

\begin{frame}<beamer>[fragile]
  \frametitle{Pattern Example}

  \begin{example}[check whether a list is nondecreasing]
    \begin{pygments}{haskell}
nondecreasing xs =
    case xs of
      [] -> True
      _:[] -> True
      x1:x2:xs' -> x1 <= x2 && nondecreasing (x2:xs')
    \end{pygments}
  \end{example}
\end{frame}

\subsection{Parameterized Types}

\begin{frame}[fragile]
  \frametitle{Parameterized Types}

  \begin{itemize}
    \item type constructors can be parameterized

    \pause
    \bigskip
    \item already defined in Haskell standard library:
    \item some type or nothing: \pygment{haskell}{Maybe}
    \item one of two types: \pygment{haskell}{Either}
  \end{itemize}

  \begin{block}{Haskell}
    \begin{pygments}{haskell}
data Maybe a = Nothing | Just a

data Either a b = Left a | Right b
    \end{pygments}
  \end{block}
\end{frame}

\begin{frame}
  \frametitle{Maybe Type Exercise}

  \begin{block}{Exercise}
    \begin{itemize}
      \item write a function that returns the maximum element of a list
      \item if empty: \pygment{haskell}{Nothing}
      \item if not empty: \pygment{haskell}{Just x}
    \end{itemize}
  \end{block}
\end{frame}

\begin{frame}<beamer>[fragile]
  \frametitle{Maybe Type Example}

  \begin{example}[maximum of a list]
    \begin{pygments}{haskell}
maybeMax xs =
    let
       maxIter acc items =
           case items of
             [] -> acc
             item:items' -> if item > acc
                            then maxIter item items'
                            else maxIter acc items'
    in
        case xs of
          [] -> Nothing
          x:xs' -> Just (maxIter x xs')
    \end{pygments}
  \end{example}
\end{frame}

\subsection{Recursive Types}

\begin{frame}[fragile]
  \frametitle{Recursive Types}

  \begin{itemize}
    \item type constructors can refer to themselves
  \end{itemize}

  \pause
  \begin{example}[Haskell]
    \begin{pygments}{haskell}
data List' a = EmptyList | NonEmptyList a (List' a)
               deriving Show
    \end{pygments}
  \end{example}
\end{frame}

\begin{frame}[fragile]
  \frametitle{Recursive Type Example}

  \begin{example}[Haskell]
    \begin{pygments}{haskell}
null' xs =
    case xs of
      EmptyList -> True
      _ -> False
    \end{pygments}

    \pause
    \smallskip
    \begin{pygments}{haskell}
head' xs =
    case xs of
      EmptyList -> Nothing
      NonEmptyList x _ -> Just x
    \end{pygments}

    \pause
    \smallskip
    \begin{pygments}{haskell}
append' xs ys =
    case xs of
      EmptyList -> ys
      NonEmptyList x xs' -> NonEmptyList x (append' xs' ys)
    \end{pygments}
  \end{example}
\end{frame}

\begin{frame}[fragile]
  \frametitle{Recursive Type Example}

  \begin{example}[binary tree]
    \begin{pygments}{haskell}
data Tree' a = EmptyTree
             | NonEmptyTree a (Tree' a) (Tree' a)
             deriving Show
    \end{pygments}

    \pause
    \begin{pygments}{haskell}
depth' t =
    case t of
      EmptyTree -> 0
      NonEmptyTree _ t1 t2 -> 1 + max (depth' t1) (depth' t2)
    \end{pygments}
  \end{example}

  \pause
  \begin{block}{Exercise}
    \begin{itemize}
      \item write a function that returns the number of leaves in a tree
    \end{itemize}
  \end{block}
\end{frame}

\begin{frame}<beamer>[fragile]
  \frametitle{Recursive Type Example}

  \begin{example}[number of leaves in a tree]
    \pause
    \begin{pygments}{haskell}
leaves t =
    case t of
      EmptyTree -> 0
      NonEmptyTree _ EmptyTree EmptyTree -> 1
      NonEmptyTree _ t1 t2 -> leaves t1 + leaves t2
    \end{pygments}
  \end{example}
\end{frame}

\begin{frame}[fragile]
  \frametitle{Recursive Type Example}

  \begin{example}[binary tree with different types at nodes and leaves]
    \begin{pygments}{haskell}
data LeafNodeTree a b =
    Node a (LeafNodeTree a b) (LeafNodeTree a b)
  | Leaf b
  deriving Show
    \end{pygments}
  \end{example}
\end{frame}

\begin{frame}[fragile]
  \frametitle{Recursive Type Example}

  \begin{example}[expression tree]
    \begin{pygments}{haskell}
data Expression = Constant Integer
                | Negate Expression
                | Add Expression Expression
                | Multiply Expression Expression
                deriving Show
    \end{pygments}
  \end{example}
\end{frame}

\begin{frame}[fragile]
  \frametitle{Recursive Type Example}

  \begin{example}[evaluate an expression]
    \pause
    \begin{pygments}{haskell}
eval e =
    case e of
      Constant c -> c
      Negate e -> -(eval e)
      Add e1 e2 -> (eval e1) + (eval e2)
      Multiply e1 e2 -> (eval e1) * (eval e2)

-- eval (Constant 4) -> 4
-- eval (Add (Constant 4) (Constant 5)) -> 9
-- eval (Negate (Add (Constant 4) (Constant 5))) -> -9
-- eval (Multiply (Constant 2)
--         (Negate (Add (Constant 4) (Constant 5)))) -> -18
    \end{pygments}
  \end{example}
\end{frame}
% 
% \section*{References}
% 
% \begin{frame}
%   \frametitle{References}
% 
%   \begin{block}{Required Reading: Abelson, Sussman}
%     \begin{itemize}
%       \item Chapter 1: Building Abstractions with Procedures
%       \begin{itemize}
%         \item 1.1. \alert{The Elements of Programming}
%       \end{itemize}
%     \end{itemize}
%   \end{block}
% \end{frame}

\end{document}
