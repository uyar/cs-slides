% Copyright (c) 2013-2017 H. Turgut Uyar <uyar@itu.edu.tr>
%
% This work is licensed under a "Creative Commons
% Attribution-NonCommercial-ShareAlike 4.0 International License".
% For more information, please visit:
% https://creativecommons.org/licenses/by-nc-sa/4.0/

\documentclass[dvipsnames]{beamer}

\usepackage[scaled=0.95]{cabin}
\usepackage[scaled=0.88]{beramono}
\usepackage[T1]{fontenc}
\usepackage[utf8]{inputenc}

\usepackage{listings}
\lstset{basicstyle=\ttfamily,
        keywordstyle=\color{blue},
        showstringspaces=false}
\lstdefinestyle{syntax}{frame=single}
\lstdefinestyle{exclamfix}{deletestring=[b]{"}}
\lstset{language=Haskell}

\mode<presentation>
{
  \usetheme{default}
  \useinnertheme{rounded}
  \usecolortheme{seahorse}
  \setbeamercovered{transparent}
}

\title{Functional Programming}
\subtitle{Pattern Matching}

\author{H. Turgut Uyar}
\date{2013-2017}

\AtBeginSubsection[]{
  \begin{frame}<beamer>
    \frametitle{Topics}
    \tableofcontents[currentsection,currentsubsection]
  \end{frame}
}

\theoremstyle{plain}

\pgfdeclareimage[height=1cm]{license}{../license}

\begin{document}

\begin{frame}
  \titlepage
\end{frame}

\begin{frame}
  \frametitle{License}

  \pgfuseimage{license}\hfill
  \copyright~2013-2017 H. Turgut Uyar

  \vfill
  \begin{footnotesize}
    You are free to:
    \begin{itemize}
      \itemsep0em
      \item Share -- copy and redistribute the material in any medium or format
      \item Adapt -- remix, transform, and build upon the material
    \end{itemize}

    Under the following terms:
    \begin{itemize}
      \itemsep0em
      \item Attribution -- You must give appropriate credit, provide a link to
        the license, and indicate if changes were made.

      \item NonCommercial -- You may not use the material for commercial
        purposes.

      \item ShareAlike -- If you remix, transform, or build upon the material,
        you must distribute your contributions under the same license as the
        original.
    \end{itemize}

    For more information:\\
    \url{https://creativecommons.org/licenses/by-nc-sa/4.0/}

    \smallskip
    Read the full license:\\
    \url{https://creativecommons.org/licenses/by-nc-sa/4.0/legalcode}
  \end{footnotesize}
\end{frame}

\begin{frame}
  \frametitle{Topics}
  \tableofcontents
\end{frame}

\lstset{deletekeywords={gcd}}

\section{Data Types}

\subsection{Tuples}

\begin{frame}[fragile]
  \frametitle{Tuples}

  \begin{itemize}
    \item \alert{tuple}: a collection of a fixed number of values
    \item different but fixed types

    \smallskip
    \begin{lstlisting}[style=syntax]
n :: (t1, t2, ..., tn)
n = (e1, e2, ..., en)
    \end{lstlisting}

    \medskip
    \item selector functions on pairs:\\
      \lstinline|fst|, \lstinline|snd|
  \end{itemize}
\end{frame}

\begin{frame}[fragile]
  \frametitle{Tuple Example}

  \begin{exampleblock}{representing a term in a polynomial: $2.4x^2$}
    \begin{lstlisting}
t :: (Float, Integer)
t = (2.4, 2)

-- fst t ~> 2.4
-- snd t ~> 2
    \end{lstlisting}
  \end{exampleblock}
\end{frame}

\begin{frame}
  \frametitle{Tuple Parameters}

  \begin{itemize}
    \item tuples can be sent as parameters
    \item not the same as multiple parameters

    \bigskip
    \item tuples can be returned as result
  \end{itemize}
\end{frame}

\begin{frame}[fragile]
  \frametitle{Tuple Parameter Example}

  \begin{exampleblock}{}
    \begin{lstlisting}
gcd :: Integer -> Integer -> Integer
gcd x y
  | y == 0    = x
  | otherwise = gcd y (x `mod` y)

-- gcd 9702 945
    \end{lstlisting}

    \medskip
    \hrulefill
    \medskip

    \begin{lstlisting}
gcd' :: (Integer, Integer) -> Integer
gcd' a
  | snd a == 0 = fst a
  | otherwise  = gcd' (snd a, (fst a) `mod` (snd a))

-- gcd' (9702, 945)
    \end{lstlisting}
  \end{exampleblock}
\end{frame}

\begin{frame}[fragile]
  \frametitle{Tuple Result Example}

  \begin{exampleblock}{simplifying a fraction}
    \begin{lstlisting}
simplify :: (Integer, Integer) -> (Integer, Integer)
simplify f = (n `div` g, d `div` g)
  where
    n = fst f
    d = snd f
    g = gcd n d
    \end{lstlisting}
  \end{exampleblock}
\end{frame}

\begin{frame}[fragile]
  \frametitle{Type Synonyms}

  \begin{itemize}
    \item type synonym: giving an existing type a new name
    \smallskip
    \begin{lstlisting}
type NewName = ExistingType
    \end{lstlisting}
  \end{itemize}

  \pause
  \begin{exampleblock}{example}
    \begin{lstlisting}
type Term = (Float, Integer)

t :: Term
t = (2.4, 2)
    \end{lstlisting}
  \end{exampleblock}
\end{frame}

\begin{frame}[fragile]
  \frametitle{Example: Type Synonyms}

  \begin{lstlisting}
type Fraction = (Integer, Integer)

simplify :: Fraction -> Fraction
simplify f = (n `div` g, d `div` g)
  where
    n = fst f
    d = snd f
    g = gcd n d

x :: Fraction
x = (21, 14)

-- simplify x ~> (3, 2)
  \end{lstlisting}
\end{frame}

\begin{frame}[fragile]
  \frametitle{Example: Type Synonyms}

  \begin{lstlisting}
type DayInYear = (Integer, Integer)

dec4 :: DayInYear
dec4 = (4, 12)

-- simplify dec4 ~> (1, 3)
  \end{lstlisting}
\end{frame}

\subsection{Lists}

\begin{frame}[fragile]
  \frametitle{Lists}

  \begin{itemize}
    \item \alert{list}: a combination of an arbitrary number of values
    \item all of the same type

    \smallskip
    \begin{lstlisting}[style=syntax]
n :: [t]
n = [e1, e2, ..., en]
    \end{lstlisting}
  \end{itemize}
\end{frame}

\begin{frame}[fragile]
  \frametitle{List Example}

  \begin{exampleblock}{second degree polynomial: $2.4x^2 + 1.8x - 4.6$}
    \begin{lstlisting}
p1 :: (Float, Float, Float)
p1 = (-4.6, 1.8, 2.4)
    \end{lstlisting}
  \end{exampleblock}

  \begin{exampleblock}{any degree polynomial: $3.4x^3 - 7.1x + 0.5$}
    \begin{lstlisting}
p2 :: [Float]
p2 = [0.5, -7.1, 0.0, 3.4]
    \end{lstlisting}
  \end{exampleblock}

  \begin{exampleblock}{sparse terms: $72.3x^{9558} - 5.0x^3$}
    \begin{lstlisting}
p3 :: [Term]
p3 = [(-5.0, 3), (72.3, 9558)]
    \end{lstlisting}
  \end{exampleblock}
\end{frame}

\begin{frame}
  \frametitle{Lists}

  \begin{itemize}
    \item a list consists of a first item (\alert{head})\\
      followed by a list of the remaining items (\alert{tail})
    \item note the recursion in the definition

    \medskip
    \item check if empty: \lstinline|null|
    \item get the head: \lstinline|head|
    \item get the tail: \lstinline|tail|

    \pause
    \medskip
    \item independent of type: \lstinline|[a]|
  \end{itemize}
\end{frame}

\begin{frame}[fragile]
  \frametitle{List Operation Examples}

  \begin{lstlisting}
null :: [a] -> Bool
-- null []           ~> True
-- null [1, 2, 3, 4] ~> False

head :: [a] -> a
-- head [1, 2, 3, 4] ~> 1
-- head []           ~> error
-- head [1]          ~> 1

tail :: [a] -> [a]
-- tail [1, 2, 3, 4] ~> [2, 3, 4]
-- tail []           ~> error
-- tail [1]          ~> []
  \end{lstlisting}
\end{frame}

\begin{frame}[fragile]
  \frametitle{List Construction}

  \begin{itemize}
    \item list construction:\\
      \lstinline|item : sublist|
    \item associates from the right
  \end{itemize}

  \pause
  \begin{exampleblock}{examples}
    \begin{lstlisting}
(:) :: a -> [a] -> [a]
-- 1 : [2, 3]        ~> [1, 2, 3]
-- 1 : 2 : 3 : []    ~> [1, 2, 3]
    \end{lstlisting}
  \end{exampleblock}
\end{frame}

\begin{frame}[fragile]
  \frametitle{List Size}

  \begin{exampleblock}{number of elements in a list}
    \begin{lstlisting}[deletekeywords={length}]
length :: [a] -> Int
length xs
  | null xs   = 0
  | otherwise = 1 + length (tail xs)
    \end{lstlisting}
  \end{exampleblock}
\end{frame}

\begin{frame}[fragile]
  \frametitle{List Example}

  \begin{exampleblock}{sum of first two elements}
    \begin{lstlisting}
firstPlusSecond :: [Integer] -> Integer
firstPlusSecond xs
  | null xs        = 0
  | null (tail xs) = head xs
  | otherwise      = head xs + head (tail xs)
    \end{lstlisting}
  \end{exampleblock}
\end{frame}

\begin{frame}[fragile]
  \frametitle{Strings}

  \begin{itemize}
   \item a string is a list of characters
    \begin{lstlisting}
type String = [Char]
    \end{lstlisting}
  \end{itemize}

  \begin{exampleblock}{examples}
    \begin{lstlisting}
-- head "word"       ~> 'w'
-- tail "word"       ~> "ord"

-- null "word"       ~> False
-- null ""           ~> True

-- 'w' : "ord"       ~> "word"
-- length "word"     ~> 4
    \end{lstlisting}
  \end{exampleblock}
\end{frame}

\subsection{Algebraic Types}

\begin{frame}[fragile]
  \frametitle{Algebraic Types}

  \begin{itemize}
    \item \alert{algebraic types}: constructors and components
    \begin{lstlisting}[style=syntax]
data T = C1 t11 t12 ... t1m
       | C2 t21 t22 ....... t2n
       | ...
    \end{lstlisting}
    \item value construction: \lstinline|Ci ei1 ei2 ... eik|
    \item constructors are functions
    \item \lstinline|Ci| may be the same as, or different from \lstinline|T|
  \end{itemize}
\end{frame}

\begin{frame}[fragile]
  \frametitle{Algebraic Type Examples}

  \begin{exampleblock}{simple product type}
    \begin{lstlisting}
data Person = Person String Integer
              deriving Show

church :: Person
church = Person "Alonzo Church" 1903
    \end{lstlisting}
  \end{exampleblock}
\end{frame}

\begin{frame}[fragile]
  \frametitle{Algebraic Type Examples}

  \begin{exampleblock}{enumeration}
    \begin{lstlisting}
data Month = Jan | Feb | Mar | Apr | May | Jun
           | Jul | Aug | Sep | Oct | Nov | Dec
             deriving Show

m :: Month
m = Feb
    \end{lstlisting}
  \end{exampleblock}
\end{frame}

\begin{frame}[fragile]
  \frametitle{Algebraic Type Examples}

  \begin{exampleblock}{multiple options}
    \begin{lstlisting}
type Coords = (Float, Float)
type Length = Float

data Shape = Point     Coords
           | Circle    Coords Length
           | Rectangle Coords Length Length
             deriving Show

p, c, r :: Shape
p = Point     (0.0, 0.0)
c = Circle    (0.0, 0.0) 1.0
r = Rectangle (5.9, 7.6) 5.7 2.3
    \end{lstlisting}
  \end{exampleblock}
\end{frame}

\section{Pattern Matching}

\subsection{Patterns}

\begin{frame}[fragile]
  \frametitle{Patterns}

  \begin{itemize}
    \item expressions can be checked against patterns
    \item result is the expression for the first matched pattern
    \begin{lstlisting}[style=syntax]
case expr of
    p1 -> e1
    p2 -> e2
    ...
    pn -> en
    _  -> e
    \end{lstlisting}
    \item matched patterns generate bindings
  \end{itemize}
\end{frame}

\begin{frame}[fragile]
  \frametitle{Pattern Examples}

  \begin{exampleblock}{literal value pattern}
    \begin{lstlisting}
gcd :: Integer -> Integer -> Integer
gcd x y = case y of
    0 -> x
    _ -> gcd y (x `mod` y)
    \end{lstlisting}
  \end{exampleblock}
\end{frame}

\begin{frame}[fragile]
  \frametitle{Pattern Examples}

  \begin{exampleblock}{tuple pattern}
    \begin{lstlisting}
gcd' :: (Integer, Integer) -> Integer
gcd' a = case a of
    (x, 0) -> x
    (x, y) -> gcd' (y, x `mod` y)

-- gcd' (9702, 945)
-- second pattern, bindings: x <-> 9702, y <-> 945

-- gcd' (63, 0)
-- first pattern, bindings: x <-> 63
    \end{lstlisting}
  \end{exampleblock}
\end{frame}

\begin{frame}[fragile]
  \frametitle{Nested Patterns}

  \begin{itemize}
    \item patterns can be nested
  \end{itemize}

  \begin{exampleblock}{example}
    \begin{lstlisting}
shift :: ((a, b), c) -> (a, (b, c))
shift s = case s of
    ((x, y), z) -> (x, (y, z))
    \end{lstlisting}
  \end{exampleblock}
\end{frame}

\begin{frame}[fragile]
  \frametitle{Wildcards}

  \begin{itemize}
    \item if binding not needed, use wildcard: \lstinline|_|
  \end{itemize}

  \begin{exampleblock}{example: third component of a triple}
    \begin{lstlisting}
third :: (a, b, c) -> c
third t = case t of
    (x, y, z) -> z

-- OR:
third t = case t of
    (_, _, z) -> z
    \end{lstlisting}
  \end{exampleblock}
\end{frame}

\begin{frame}[fragile]
  \frametitle{List Patterns}

  \begin{itemize}
    \item empty list:\\
      \lstinline|[]|
    \item nonempty list:\\
      \lstinline|x:xs|
    \item list with exactly one element:\\
      \lstinline|[x]|
    \item list with exactly two elements:\\
      \lstinline|[x1,x2]|
    \item list with at least two elements:\\
      \lstinline|x1:x2:xs|
  \end{itemize}
\end{frame}

\begin{frame}[fragile]
  \frametitle{List Pattern Examples}

  \begin{exampleblock}{number of elements}
    \begin{lstlisting}[deletekeywords={length}]
length :: [a] -> Int
length xs = case xs of
    []    -> 0
    x:xs' -> 1 + length xs'
    \end{lstlisting}
  \end{exampleblock}
\end{frame}

\begin{frame}[fragile]
  \frametitle{List Pattern Examples}

  \begin{exampleblock}{sum of the first and third elements}
    \begin{lstlisting}
firstPlusThird :: [Integer] -> Integer
firstPlusThird xs = case xs of
    []        -> 0
    [x1]      -> x1
    [x1,_]    -> x1
    x1:_:x3:_ -> x1 + x3
    \end{lstlisting}
  \end{exampleblock}
\end{frame}

\begin{frame}[fragile]
  \frametitle{List Pattern Examples}

  \begin{exampleblock}{check whether a list is in nondecreasing order}
    \begin{lstlisting}
nondecreasing :: [Integer] -> Bool
nondecreasing xs = case xs of
  []       -> True
  [_]      -> True
  x1:x2:xs -> x1 <= x2 && nondecreasing (x2 : xs)
    \end{lstlisting}
  \end{exampleblock}

  \begin{itemize}
    \item reconstructing not necessary: \lstinline|@|
  \end{itemize}
\end{frame}

\begin{frame}[fragile]
  \frametitle{List Pattern Examples}

  \begin{exampleblock}{check whether list is in nondecreasing order}
    \begin{lstlisting}
nondecreasing :: [Integer] -> Bool
nondecreasing xs = case xs of
  []           -> True
  [_]          -> True
  x1:xs@(x2:_) -> x1 <= x2 && nondecreasing xs
    \end{lstlisting}
  \end{exampleblock}
\end{frame}

\begin{frame}
  \frametitle{Algebraic Type Patterns}

  \begin{itemize}
    \item patterns can match algebraic types
    \item use pattern matching to get values out of product types
  \end{itemize}
\end{frame}

\begin{frame}[fragile]
  \frametitle{Algebraic Type Pattern Examples}

  \begin{exampleblock}{get component out of product type}
    \begin{lstlisting}
birthYear :: Person -> Integer
birthYear p = case p of
    Person _ y -> y

-- birthYear (Person "Alonzo Church" 1903) ~> 1903
-- binding: y <-> 1903
    \end{lstlisting}
  \end{exampleblock}
\end{frame}

\begin{frame}[fragile]
  \frametitle{Algebraic Type Pattern Examples}

  \begin{exampleblock}{number of days in a month}
    \begin{lstlisting}
daysInMonth :: Month -> Integer -> Integer
daysInMonth m y = case m of
    Apr -> 30
    Jun -> 30
    Sep -> 30
    Nov -> 30
    Feb -> if y `mod` 4 == 0 then 29 else 28
    _   -> 31

-- daysInMonth Jan 2014 ~> 31
-- daysInMonth Feb 2014 ~> 28
-- daysInMonth Feb 2016 ~> 29
    \end{lstlisting}
  \end{exampleblock}
\end{frame}

\begin{frame}[fragile]
  \frametitle{Algebraic Type Pattern Examples}

  \begin{exampleblock}{area of a geometric shape}
    \begin{lstlisting}
area :: Shape -> Float
area s = case s of
    Point     _     -> 0.0
    Circle    _ r   -> 3.14159 * r * r
    Rectangle _ h w -> h * w

-- area (Circle (0.0, 0.0) 3.0) ~> 28.274311
-- second pattern, binding: r <-> 3.0
    \end{lstlisting}
  \end{exampleblock}
\end{frame}

\subsection{Parameter Patterns}

\begin{frame}[fragile]
  \frametitle{Parameter Patterns}

  \begin{itemize}
    \item formal parameters are patterns
    \item components of pattern matched with actual parameters

    \medskip
    \item in case of multiple patterns, first match will be selected

    \smallskip
    \begin{lstlisting}[style=syntax]
n p1 = e1
n p2 = e2
...
    \end{lstlisting}
  \end{itemize}
\end{frame}

\begin{frame}[fragile]
  \frametitle{Parameter Pattern Example}

  \begin{lstlisting}
gcd :: Integer -> Integer -> Integer
gcd x y = case y of
    0 -> x
    _ -> gcd y (x `mod` y)

-- OR:
gcd :: Integer -> Integer -> Integer
gcd x 0 = x
gcd x y = gcd y (x `mod` y)
  \end{lstlisting}
\end{frame}

\begin{frame}[fragile]
  \frametitle{Parameter Pattern Example}

  \begin{lstlisting}
gcd' :: (Integer, Integer) -> Integer
gcd' a = case a of
    (x, 0) -> x
    (x, y) -> gcd' (y, x `mod` y)

-- OR:
gcd' :: (Integer, Integer) -> Integer
gcd' (x, 0) = x
gcd' (x, y) = gcd' (y, x `mod` y)
  \end{lstlisting}
\end{frame}

\begin{frame}[fragile]
  \frametitle{Parameter Pattern Example}

  \begin{lstlisting}
shift :: ((a, b), c) -> (a, (b, c))
shift s = case s of
    ((x, y), z) -> (x, (y, z))

-- OR:
shift :: ((a, b), c) -> (a, (b, c))
shift ((x, y), z) = (x, (y, z))
  \end{lstlisting}
\end{frame}

\begin{frame}[fragile]
  \frametitle{Parameter Pattern Example}

  \begin{lstlisting}
third :: (a, b, c) -> c
third t = case t of
    (_ , _, z) -> z

-- OR:
third :: (a, b, c) -> c
third (_, _, z) = z
  \end{lstlisting}
\end{frame}

\begin{frame}[fragile]
  \frametitle{Parameter Pattern Example}

  \begin{lstlisting}[deletekeywords={length}]
length :: [a] -> Int
length xs = case xs of
    []    -> 0
    x:xs' -> 1 + length xs'

-- OR:
length :: [a] -> Int
length []     = 0
length (x:xs) = 1 + length xs
  \end{lstlisting}
\end{frame}

\begin{frame}[fragile]
  \frametitle{Parameter Pattern Example}

  \begin{lstlisting}
birthYear :: Person -> Year
birthYear p = case p of
    Person _ y -> y

-- OR:
birthYear :: Person -> Year
birthYear (Person _ y) = y
  \end{lstlisting}
\end{frame}

\begin{frame}[fragile]
  \frametitle{Record Types}

  \begin{itemize}
    \item give names to fields
    \item automatically creates functions to extract components

    \medskip
    \begin{lstlisting}[style=syntax]
data T = C1 { n11 :: t11,
              n12 :: t12,
              ...,
              n1m :: t1m }
       | ...
    \end{lstlisting}
  \end{itemize}
\end{frame}

\begin{frame}[fragile]
  \frametitle{Record Examples}

  \begin{exampleblock}{example}
    \begin{lstlisting}
data PersonR = PersonR { fullname :: String,
                         born :: Integer }
               deriving Show

church :: PersonR
church = PersonR "Alonzo Church" 1903
church = PersonR { born=1903, fullname="Alonzo Church" }

-- fullname church ~> "Alonzo Church"
-- born church     ~> 1903
    \end{lstlisting}
  \end{exampleblock}
\end{frame}

\begin{frame}[fragile]
  \frametitle{Example: Quadrants}

  \begin{itemize}
    \item when to use a \lstinline|case|?
  \end{itemize}

  \begin{lstlisting}
quadrant :: Coords -> Integer
quadrant (x, y) = case (x>=0, y>=0) of
    (True,  True)  -> 1
    (False, True)  -> 2
    (False, False) -> 3
    (True,  False) -> 4
  \end{lstlisting}
\end{frame}
% 
% \begin{frame}[fragile]
%   \frametitle{Example: Fibonacci}
% 
%   \begin{lstlisting}
% fibStep :: (Integer, Integer) -> (Integer, Integer)
% fibStep (u, v) = (v, u + v)
% 
% -- fibPair n ~> (fib n, fib (n + 1))
% fibPair :: Integer -> (Integer, Integer)
% fibPair 1 = (1, 1)
% fibPair n = fibStep (fibPair (n - 1))
% 
% fastFib n = fst (fibPair n)
%   \end{lstlisting}
% \end{frame}

\section{Lists}

\subsection{List Expressions}

\begin{frame}[fragile]
  \frametitle{List Operators}

  \begin{itemize}
    \item index: \lstinline|!!|
    \item append: \lstinline|++|
  \end{itemize}

  \pause
  \begin{exampleblock}{examples}
    \begin{lstlisting}
-- [1, 2, 3] !! 0       ~> 1
-- [1, 2, 3] !! 2       ~> 3
-- [1, 2, 3] !! 3       ~> error

-- [1, 2, 3] ++ [4, 5]  ~> [1, 2, 3, 4, 5]
    \end{lstlisting}
  \end{exampleblock}
\end{frame}

\begin{frame}[fragile]
  \frametitle{Example: Indexing Lists}

  \begin{lstlisting}
(!!) :: [a] -> Int -> a
(!!) []     _ = error "no such element"
(!!) (x:xs) 0 = x
(!!) (x:xs) n = (!!) xs (n - 1)
  \end{lstlisting}

  \pause
  \begin{itemize}
    \item use the infix operator notation:
  \end{itemize}

  \begin{lstlisting}
(!!) :: [a] -> Int -> a
[]     !! _ = error "no such element"
(x:xs) !! 0 = x
(x:xs) !! n = xs !! (n - 1)
  \end{lstlisting}
\end{frame}

\begin{frame}[fragile]
  \frametitle{Example: Appending Lists}

  \begin{lstlisting}
(++) :: [a] -> [a] -> [a]
[]     ++ ys = ys
(x:xs) ++ ys = x : (xs ++ ys)
  \end{lstlisting}
\end{frame}

\begin{frame}[fragile]
  \frametitle{Ranges}

  \begin{itemize}
    \item \lstinline|[n .. m]|: range with increment 1
    \item \lstinline|[n, p .. m]|: range with increment \texttt{p - n}
  \end{itemize}

  \begin{exampleblock}{examples}
    \begin{lstlisting}
[2 .. 7]          ~> [2, 3, 4, 5, 6, 7]
[3.1 .. 7.0]      ~> [3.1, 4.1, 5.1, 6.1, 7.1]
['a' .. 'm']      ~> "abcdefghijklm"

[7, 6 .. 3]       ~> [7, 6, 5, 4, 3]
[0.0, 0.3 .. 1.0] ~> [0.0, 0.3, 0.6, 0.8999999999999999]
['a', 'c' .. 'n'] ~> "acegikm"
    \end{lstlisting}
  \end{exampleblock}
\end{frame}

\subsection{Standard Functions}

\begin{frame}[fragile]
  \frametitle{Membership Check}

  \begin{itemize}
    \item check whether an element is a member of a list\\
      \lstinline[style=exclamfix]|elem 'r' "word" ~> True|\\
      \lstinline[style=exclamfix]|elem 'x' "word" ~> False|
  \end{itemize}

  \begin{exampleblock}{}
    \begin{lstlisting}[deletekeywords={elem}]
elem :: Char -> [Char] -> Bool
elem _ []     = False
elem x (c:cs) = if x == c then True else elem x cs
    \end{lstlisting}
  \end{exampleblock}

  \pause
  \begin{itemize}
    \item exercise: make a list of n copies of an item\\
      \lstinline|replicate 3 'c' ~> "ccc"|
  \end{itemize}
\end{frame}

\begin{frame}[fragile]
  \frametitle{Last Element}

  \begin{itemize}
    \item get the last element of a list\\
      \lstinline[style=exclamfix]|last "word" ~> 'd'|
  \end{itemize}

  \begin{exampleblock}{}
    \begin{lstlisting}[deletekeywords={last}]
last :: [a] -> a
last []     = error "empty list"
last [x]    = x
last (x:xs) = last xs
    \end{lstlisting}
  \end{exampleblock}

  \pause
  \begin{itemize}
    \item exercise: get all elements but the last of a list\\
      \lstinline[style=exclamfix]|init "word" ~> "wor"|
  \end{itemize}
\end{frame}

\begin{frame}[fragile]
  \frametitle{Split}

  \begin{itemize}
    \item take n elements from the front of a list\\
      \lstinline[style=exclamfix]|take 3 "Peccary" ~> "Pec"|
  \end{itemize}

  \begin{exampleblock}{}
    \begin{lstlisting}[deletekeywords={take}]
take :: Int -> [a] -> [a]
take 0 _      = []
take _ []     = []
take n (x:xs) = x : take (n - 1) xs
    \end{lstlisting}
  \end{exampleblock}

  \pause
  \begin{itemize}
    \item exercise: drop n elements from the front of a list\\
      \lstinline[style=exclamfix]|drop 3 "Peccary" ~> "cary"|

    \item exercise: split a list at a given position\\
      \lstinline[style=exclamfix]|splitAt 3 "Peccary" ~> ("Pec", "cary")|
  \end{itemize}
\end{frame}

\begin{frame}[fragile]
  \frametitle{Reverse}

  \begin{itemize}
    \item reverse a list\\
      \lstinline[style=exclamfix]|reverse "word" ~> "drow"|
  \end{itemize}

  \begin{exampleblock}{}
    \begin{lstlisting}[deletekeywords={reverse}]
reverse :: [a] -> [a]
reverse []     = []
reverse (x:xs) = (reverse xs) ++ [x]
    \end{lstlisting}
  \end{exampleblock}
\end{frame}

\begin{frame}[fragile]
  \frametitle{Concatenate}

  \begin{itemize}
    \item convert a list of lists of items into a list of items\\
      \lstinline|concat [[2, 3], [], [4] ~> [2, 3, 4]|
  \end{itemize}

  \begin{exampleblock}{}
    \begin{lstlisting}[deletekeywords={concat}]
concat :: [[a]] -> [a]
concat []       = []
concat (xs:xss) = xs ++ concat xss
    \end{lstlisting}
  \end{exampleblock}
\end{frame}

\begin{frame}[fragile]
  \frametitle{Zip}

  \begin{itemize}
    \item convert two lists into a list of pairs\\
      \lstinline[style=exclamfix]|zip [1, 2] "ab" ~> [(1, 'a'), (2, 'b')]|
  \end{itemize}

  \begin{exampleblock}{}
    \begin{lstlisting}[deletekeywords={zip}]
zip :: [a] -> [b] -> [(a, b)]
zip []     []     = []
zip (x:xs) (y:ys) = (x, y) : zip xs ys
    \end{lstlisting}
  \end{exampleblock}

  \pause
  \begin{itemize}
    \item not all cases are covered:\\
      \lstinline[style=exclamfix]|zip [1, 2] "abc" ~> [(1, 'a'), (2, 'b')]|
  \end{itemize}
\end{frame}

\begin{frame}[fragile]
  \frametitle{Zip}

  \begin{exampleblock}{}
    \begin{lstlisting}[deletekeywords={zip}]
zip :: [a] -> [b] -> [(a, b)]
zip (x:xs) (y:ys) = (x, y) : zip xs ys
zip _      _      = []
    \end{lstlisting}
  \end{exampleblock}

  \pause
  \begin{itemize}
    \item exercise: convert three lists into a list of triples\\
      \lstinline[style=exclamfix]|zip3 [1, 2] "abc" [7, 4]|\\
      \lstinline[style=exclamfix]|     ~> [(1, 'a', 7), (2, 'b', 4)]|
  \end{itemize}
\end{frame}

\begin{frame}[fragile]
  \frametitle{Unzip}

  \begin{itemize}
    \item convert a list of pairs into a pair of lists\\
      \lstinline[style=exclamfix]|unzip [(1, 'a'), (2, 'b')] ~> ([1, 2], "ab")|
  \end{itemize}

  \begin{exampleblock}{}
    \begin{lstlisting}[deletekeywords={unzip}]
unzip :: [(a, b)] -> ([a], [b])
unzip []           = ([], [])
unzip ((x, y):xys) = (x : xs, y : ys)
  where
    (xs, ys) = unzip xys
    \end{lstlisting}
  \end{exampleblock}

  \pause
  \begin{itemize}
    \item exercise: convert a list of triples into three lists\\
      \lstinline[style=exclamfix]|unzip3 [(1, 'a', 7), (2, 'b', 4)]|\\
      \lstinline[style=exclamfix]|     ~> ([1, 2], "ab", [7, 4])|
  \end{itemize}
\end{frame}

\subsection{Examples}

\begin{frame}[fragile]
  \frametitle{Example: Merging Lists}

  \begin{exampleblock}{merge two ordered lists}
    \begin{lstlisting}
merge :: [Integer] -> [Integer] -> [Integer]
merge xs     []     = xs
merge []     ys     = ys
merge (x:xs) (y:ys)
  | x <= y    = x : merge xs (y : ys)
  | otherwise = y : merge (x : xs) ys
    \end{lstlisting}
  \end{exampleblock}
\end{frame}

\begin{frame}[fragile]
  \frametitle{Merging Lists}

  \begin{exampleblock}{merge two ordered lists}
    \begin{lstlisting}
merge :: [Integer] -> [Integer] -> [Integer]
merge xs [] = xs
merge [] ys = ys
merge xs@(x':xs') ys@(y':ys')
  | x' <= y'  = x' : merge xs' ys
  | otherwise = y' : merge xs  ys'
    \end{lstlisting}
  \end{exampleblock}
\end{frame}

\begin{frame}[fragile]
  \frametitle{Roman Numeral Conversion}

  \begin{exampleblock}{convert an integer to Roman numerals}
    \begin{itemize}
      \item adapted from the book ``Dive into Python'' by Mark Pilgrim:\\
        \url{http://www.diveintopython.net/}
    \end{itemize}

    \medskip
    \begin{lstlisting}
romanNumerals =
  [("M", 1000), ("CM", 900), ("D", 500), ("CD", 400),
   ("C",  100), ("XC",  90), ("L",  50), ("XL",  40),
   ("X",   10), ("IX",   9), ("V",   5), ("IV",   4),
   ("I",    1)]
    \end{lstlisting}
  \end{exampleblock}
\end{frame}

\begin{frame}[fragile]
  \frametitle{Roman Numeral Conversion}

  \begin{exampleblock}{Python}
    \begin{lstlisting}[language=Python]
def toRoman(n):
    result = ""
    for numeral, integer in romanNumerals:
        while n >= integer:
            result += numeral
            n -= integer
    return result
    \end{lstlisting}
  \end{exampleblock}
\end{frame}

\begin{frame}[fragile]
  \frametitle{Roman Numeral Conversion}

  \begin{exampleblock}{}
    \begin{lstlisting}
toRoman :: Integer -> String
toRoman n = tR n romanNumerals
  where
    tR :: Integer -> [(String, Integer)] -> String
    tR n []              = ""
    tR n xs@((s, k):xs')
      | n >= k    = s ++ tR (n - k) xs
      | otherwise = tR n xs'
    \end{lstlisting}
  \end{exampleblock}

  \pause
  \begin{itemize}
    \item exercise: convert a Roman numeral string into an integer
  \end{itemize}
\end{frame}

\begin{frame}[fragile]
  \frametitle{Roman Numeral Conversion}

  \begin{exampleblock}{Python}
    \begin{lstlisting}[language=Python]
def fromRoman(s):
    result = 0
    index = 0
    for numeral, integer in romanNumerals:
        while s[index : index+len(numeral)] == numeral:
            result += integer
            index += len(numeral)
    return result
    \end{lstlisting}
  \end{exampleblock}
\end{frame}

\section*{References}

\begin{frame}
  \frametitle{References}

  \begin{block}{Required Reading: Thompson}
    \begin{itemize}
      \item Chapter 5: \alert{Data types, tuples and lists}
      \item Chapter 6: \alert{Programming with lists}
      \item Chapter 7: \alert{Defining functions over lists}
    \end{itemize}
  \end{block}
\end{frame}

\end{document}
