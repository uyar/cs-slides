% Copyright (c) 2013
%       H. Turgut Uyar <uyar@itu.edu.tr>
%
% These notes are licensed using the
% "Creative Commons Attribution-NonCommercial-ShareAlike License".
% You are free to copy, distribute and transmit the work, and to adapt the work
% as long as you attribute the authors, do not use it for commercial purposes,
% and any derivative work is under the same or a similar license.
%
% Read the full legal code at:
% http://creativecommons.org/licenses/by-nc-sa/3.0/

\documentclass[dvipsnames]{beamer}

\usepackage{ae}
\usepackage[T1]{fontenc}
\usepackage[utf8]{inputenc}
\usepackage{pythontex}
\setbeamertemplate{navigation symbols}{}

\mode<presentation>
{
  \usetheme{Boadilla}
  \setbeamercovered{transparent}
}

\title{Functional Programming}
\subtitle{Recursion}

\author{H. Turgut Uyar}
\date{2013}

\AtBeginSubsection[]{
  \begin{frame}<beamer>
    \frametitle{Topics}
    \tableofcontents[currentsection,currentsubsection]
  \end{frame}
}

\theoremstyle{plain}

\pgfdeclareimage[width=2cm]{license}{../../license}

\begin{document}

\begin{frame}
  \titlepage
\end{frame}

\begin{frame}
  \frametitle{License}

  \pgfuseimage{license}\hfill
  \copyright 2013 T. Uyar

  \vfill
  \begin{tiny}
    You are free:
    \begin{itemize}
      \item to Share -- to copy, distribute and transmit the work
      \item to Remix -- to adapt the work
    \end{itemize}

    Under the following conditions:
    \begin{itemize}
      \item Attribution -- You must attribute the work in the manner specified by
        the author or licensor (but not in any way that suggests that they
        endorse you or your use of the work).

      \item Noncommercial -- You may not use this work for commercial purposes.

      \item Share Alike -- If you alter, transform, or build upon this work, you
        may distribute the resulting work only under the same or similar license
        to this one.
    \end{itemize}
  \end{tiny}

  \vfill
  Legal code (the full license):\\
  \url{http://creativecommons.org/licenses/by-nc-sa/3.0/}
\end{frame}

\begin{frame}
  \frametitle{Topics}
  \tableofcontents
\end{frame}

\section{Recursion}

\subsection{Stack Frames}

\begin{frame}[fragile]
  \frametitle{Recursion Examples}

  \begin{itemize}
    \item consider two classic examples
  \end{itemize}

  \begin{example}[factorial]
    \begin{pygments}{haskell}
fact n =
    if n == 0
    then 1
    else n * fact (n - 1)
    \end{pygments}
  \end{example}

  \begin{example}[greatest common divisor]
    \begin{pygments}{haskell}
gcd' x y =
    if y == 0
    then x
    else gcd' y (x `mod` y)
    \end{pygments}
  \end{example}
\end{frame}

\begin{frame}[fragile]
  \frametitle{Stack Frame Example}

  \begin{example}[fact]
    \begin{pygments}{haskell}
fact n =
    if n == 0 then 1
    else n * fact (n - 1)
    \end{pygments}

    \pause
    \begin{verbatim}
fact 4
|- 4 * fact 3
       |- 3 * fact 2
              |- 2 * fact 1
                     |- 1 * fact 0
                            1
                     1
              2
       6
24
    \end{verbatim}
  \end{example}
\end{frame}

\begin{frame}[fragile]
  \frametitle{Stack Frame Example}

  \begin{example}[gcd']
    \begin{pygments}{haskell}
gcd' x y =
    if y == 0 then x
    else gcd' y (x `mod` y)
    \end{pygments}

    \pause
    \begin{verbatim}
gcd_ 9702 945
|- gcd_ 945 252
   |- gcd_ 252 189
      |- gcd_ 189 63
         |- gcd_ 63 0
         63
      63
   63
63
    \end{verbatim}
  \end{example}
\end{frame}

\subsection{Tail Recursion}

\begin{frame}
  \frametitle{Tail Recursion}

  \begin{itemize}
    \item if the result of the recursive call is also the result of the caller\\
      the function is said to be \alert{tail recursive}
    \item the recursive function call is the last action:\\
      nothing left for the caller to do

    \pause
    \medskip
    \item no need to keep the stack frame around\\
      $\rightarrow$ reuse the frame of the caller

    \pause
    \medskip
    \item to rearrange a function to be tail recursive
    \begin{itemize}
      \item define a helper function that takes an \alert{accumulator}
      \item base case: return accumulator
      \item recursive case: make recursive call with new accumulator value
    \end{itemize}
  \end{itemize}
\end{frame}

\begin{frame}[fragile]
  \frametitle{Tail Recursion Example}

  \begin{example}[factorial]
    \begin{pygments}{haskell}
fact_iter acc x =
    if x == 0
    then acc
    else fact_iter (acc * x) (x - 1)

fact n = fact_iter 1 n
    \end{pygments}
  \end{example}
\end{frame}

\begin{frame}[fragile]
  \frametitle{Stack Frame Example}

  \begin{example}[fact - tail recursive]
    \begin{pygments}{haskell}
fact_iter acc x =
    if x == 0 then acc
    else fact_iter (acc * x) (x - 1)
    \end{pygments}

    \pause
    \begin{verbatim}
fact 4
|- fact_iter 1 4
|- fact_iter 4 3
|- fact_iter 12 2
|- fact_iter 24 1
|- fact_iter 24 0
24
    \end{verbatim}
  \end{example}
\end{frame}

\subsection{Nested Functions}

\begin{frame}
  \frametitle{Nested Functions}

  \begin{itemize}
    \item functions can be defined locally within other functions
    \item they can use bindings from the environment\\
      in which they are defined
  \end{itemize}
\end{frame}

\begin{frame}[fragile]
  \frametitle{Nested Function Example}

  \begin{example}[factorial]
    \begin{itemize}
      \item no need for \pygment{haskell}{fact_iter} to be visible
        outside of \pygment{haskell}{fact}
    \end{itemize}

    \begin{pygments}{haskell}
fact n =
    let
        fact_iter acc x =
            if x == 0
            then acc
            else fact_iter (acc * x) (x - 1)
    in
        fact_iter 1 n
    \end{pygments}
  \end{example}
\end{frame}

\begin{frame}[fragile]
  \frametitle{Recursion Exercise}

  \begin{block}{Exercise}
    \begin{itemize}
      \item what is the type of \pygment{haskell}{fact}?
      \item what happens if called as \pygment{haskell}{fact 2.4}?
      \item what if \pygment{haskell}{fact_iter :: Integer -> Integer -> Integer}?
    \end{itemize}
  \end{block}
\end{frame}

\begin{frame}[fragile]
  \frametitle{Nested Function Example}

  \begin{example}[factorial - Scala]
    \begin{pygments}{scala}
def fact(n: Int): Int = {
    def factIter(acc: Int, x: Int): Int =
        if (x == 0) acc
        else factIter(acc * x, x - 1)
    factIter(1, n)
}
    \end{pygments}
  \end{example}
\end{frame}

\subsection{Errors}

\begin{frame}[fragile]
  \frametitle{Errors}

  \begin{block}{Haskell}
    \begin{itemize}
      \item an exception can be raised using \pygment{haskell}{error}
    \end{itemize}
  \end{block}

  \begin{example}[factorial]
    \begin{pygments}{haskell}
fact n =
    if n < 0
    then error "negative parameter"
    else if n == 0
         then 1
         else n * fact (n - 1)
    \end{pygments}
  \end{example}
\end{frame}

\begin{frame}[fragile]
  \frametitle{Errors}

  \begin{block}{Scala}
    \begin{itemize}
      \item an exception can be raised using \pygment{scala}{throw}
    \end{itemize}
  \end{block}

  \begin{example}[factorial]
    \begin{pygments}{scala}
def fact(n: Int): Int =
    if (n < 0) throw
        new IllegalArgumentException("negative parameter")
    else if (n == 0) 1
         else n * fact(n - 1)
    \end{pygments}
  \end{example}
\end{frame}

\section{Lists}

\subsection{Representation}

\begin{frame}
  \frametitle{Representation}

  \begin{itemize}
    \item a list of consists of an element (\alert{head})\\
      followed by the sublist of the remaining elements (\alert{tail})
  \end{itemize}

  \pause
  \medskip
  \begin{block}{Haskell list operations}
    \begin{itemize}
      \item check if empty: \pygment{haskell}{null}
      \item get the head: \pygment{haskell}{head}
      \item get the tail: \pygment{haskell}{tail}
      \item construct a list: \pygment{haskell}{item : sublist}
    \end{itemize}
  \end{block}
\end{frame}

\begin{frame}[fragile]
  \frametitle{List Operation Examples}

  \begin{example}[Haskell]
    \begin{pygments}{haskell}
Prelude> head [1,2,3,4]
1
Prelude> tail [1,2,3,4]
[2,3,4]
Prelude> null [1,2,3,4]
False
Prelude> 1 : [2,3]
[1,2,3]
Prelude> 1 : (2 : (3 : []))
[1,2,3]
Prelude> 1 : 2 : 3 : []
[1,2,3]
    \end{pygments}
  \end{example}
\end{frame}

\begin{frame}
  \frametitle{List Operations}

  \begin{block}{Scala list operations}
    \begin{itemize}
      \item check if empty: \pygment{scala}{.isEmpty}
      \item get the head: \pygment{haskell}{.head}
      \item get the tail: \pygment{haskell}{.tail}
      \item construct a list: \pygment{haskell}{item :: sublist}
    \end{itemize}
  \end{block}
\end{frame}

\subsection{List Recursion}

\begin{frame}[fragile]
  \frametitle{List Recursion Example}

  \begin{example}[length of a list]
    \begin{pygments}{haskell}
length' xs =
    if null xs
    then 0
    else 1 + length' (tail xs)
    \end{pygments}
  \end{example}

  \pause
  \begin{block}{Exercise}
    \begin{itemize}
      \item write a tail recursive version
    \end{itemize}
  \end{block}
\end{frame}

\begin{frame}<beamer>[fragile]
  \frametitle{List Recursion Example}

  \begin{example}[length of a list]
    \begin{pygments}{haskell}
length' xs =
    let
        length_iter acc items =
            if null items
            then acc
            else length_iter (acc + 1) (tail items)
    in
        length_iter 0 xs
    \end{pygments}
  \end{example}
\end{frame}

\begin{frame}[fragile]
  \frametitle{List Recursion Example}
  \begin{example}[sum of a list]
    \pause
    \begin{pygments}{haskell}
sum' xs =
    if null xs
    then 0
    else head xs + sum' (tail xs)
    \end{pygments}
  \end{example}

  \pause
  \begin{example}[Scala]
    \begin{pygments}{scala}
def sum(xs: List[Int]): Int =
    if (xs.isEmpty) 0
    else xs.head + sum(xs.tail)
    \end{pygments}
  \end{example}
\end{frame}

\begin{frame}[fragile]
  \frametitle{List Recursion Example}

  \begin{example}[append two lists]
    \pause
    \begin{pygments}{haskell}
append xs ys =
    if null xs
    then ys
    else head xs : append (tail xs) ys
    \end{pygments}
  \end{example}

  \pause
  \begin{itemize}
    \item Haskell list append operator: \pygment{haskell}{++}
  \end{itemize}
\end{frame}

\begin{frame}[fragile]
  \frametitle{List Recursion Example}

  \begin{example}[generate list from lower to upper limit]
    \pause
    \begin{pygments}{haskell}
count_up lower upper =
    if lower > upper
    then []
    else lower : count_up (lower + 1) upper
    \end{pygments}
  \end{example}

  \pause
  \begin{block}{Exercise}
    \begin{itemize}
      \item write a tail recursive version that generates the list\\
        from an upper limit down to a lower limit
    \end{itemize}
  \end{block}
\end{frame}

\begin{frame}<beamer>[fragile]
  \frametitle{List Recursion Example}

  \begin{example}[generate list from upper to lower limit]
    \begin{pygments}{haskell}
count_down upper lower =
    let
        count_down_iter acc u =
            if u < lower
            then acc
            else count_down_iter (acc ++ [u]) (u - 1)
    in
        count_down_iter [] upper
    \end{pygments}

% TODO: note the use of lower
  \end{example}
\end{frame}

\begin{frame}[fragile]
  \frametitle{List Recursion Example}

  \begin{example}[maximum of a list]
    \pause
    \begin{pygments}{haskell}
max' xs =
    if null xs
    then error "empty list"
    else if null (tail xs)
         then head xs
         else if head xs > max' (tail xs)
              then head xs
              else max' (tail xs)
    \end{pygments}
  \end{example}
\end{frame}

\begin{frame}[fragile]
  \frametitle{List Recursion Exercise}

  \begin{block}{Exercise}
    \begin{itemize}
      \item what happens if called as:\\
        \pygment{haskell}{max' (count_down 30 1)}\\
        \pygment{haskell}{max' (count_up 1 30)}
      \item fix the problem
   \end{itemize}
  \end{block}
\end{frame}

\begin{frame}<beamer>[fragile]
  \frametitle{List Recursion Example}

  \begin{example}[maximum of a list]
    \begin{pygments}{haskell}
max' xs =
    if null xs
    then error "empty list"
    else if null (tail xs)
         then head xs
         else
             let
                 max_rest = max' (tail xs)
             in
                 if head xs > max_rest
                 then head xs
                 else max_rest
    \end{pygments}
  \end{example}
\end{frame}

\begin{frame}
  \frametitle{List Recursion Exercises}

  \begin{block}{Exercises}
    \begin{itemize}
      \item write a function that will reverse a list
      \item write a tail recursive function that will reverse a list
    \end{itemize}
  \end{block}
\end{frame}

\begin{frame}<beamer>[fragile]
  \frametitle{List Recursion Example}

  \begin{example}[reverse a list]
    \begin{pygments}{haskell}
reverse' xs =
    if null xs
    then []
    else reverse' (tail xs) ++ [head xs]
    \end{pygments}
  \end{example}
\end{frame}

\begin{frame}<beamer>[fragile]
  \frametitle{List Recursion Example}

  \begin{example}[reverse a list]
    \begin{pygments}{haskell}
reverse' xs =
    let
        reverse_iter acc items =
            if null items
            then acc
            else reverse_iter (head items : acc) (tail items)
    in
        reverse_iter [] xs
    \end{pygments}
  \end{example}
\end{frame}
% 
% \section*{References}
% 
% \begin{frame}
%   \frametitle{References}
% 
%   \begin{block}{Required Reading: Abelson, Sussman}
%     \begin{itemize}
%       \item Chapter 1: Building Abstractions with Procedures
%       \begin{itemize}
%         \item 1.1. \alert{The Elements of Programming}
%       \end{itemize}
%     \end{itemize}
%   \end{block}
% \end{frame}

\end{document}
