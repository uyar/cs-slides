% Copyright (c) 2013
%       H. Turgut Uyar <uyar@itu.edu.tr>
%
% These notes are licensed using the
% "Creative Commons Attribution-NonCommercial-ShareAlike License".
% You are free to copy, distribute and transmit the work, and to adapt the work
% as long as you attribute the authors, do not use it for commercial purposes,
% and any derivative work is under the same or a similar license.
%
% Read the full legal code at:
% http://creativecommons.org/licenses/by-nc-sa/3.0/

\documentclass[dvipsnames]{beamer}

\usepackage{ae}
\usepackage[T1]{fontenc}
\usepackage[utf8]{inputenc}
\usepackage{pythontex}
\setbeamertemplate{navigation symbols}{}

\mode<presentation>
{
  \usetheme{Boadilla}
  \setbeamercovered{transparent}
}

\title{Functional Programming}
\subtitle{Recursion}

\author{H. Turgut Uyar}
\date{2013}

\AtBeginSubsection[]{
  \begin{frame}<beamer>
    \frametitle{Topics}
    \tableofcontents[currentsection,currentsubsection]
  \end{frame}
}

\theoremstyle{plain}

\pgfdeclareimage[width=2cm]{license}{../../license}

\begin{document}

\begin{frame}
  \titlepage
\end{frame}

\begin{frame}
  \frametitle{License}

  \pgfuseimage{license}\hfill
  \copyright 2013 T. Uyar

  \vfill
  \begin{tiny}
    You are free:
    \begin{itemize}
      \item to Share -- to copy, distribute and transmit the work
      \item to Remix -- to adapt the work
    \end{itemize}

    Under the following conditions:
    \begin{itemize}
      \item Attribution -- You must attribute the work in the manner specified by
        the author or licensor (but not in any way that suggests that they
        endorse you or your use of the work).

      \item Noncommercial -- You may not use this work for commercial purposes.

      \item Share Alike -- If you alter, transform, or build upon this work, you
        may distribute the resulting work only under the same or similar license
        to this one.
    \end{itemize}
  \end{tiny}

  \vfill
  Legal code (the full license):\\
  \url{http://creativecommons.org/licenses/by-nc-sa/3.0/}
\end{frame}

\begin{frame}
  \frametitle{Topics}
  \tableofcontents
\end{frame}

\section{Recursion}

\subsection{Stack Frames}

\begin{frame}[fragile]
  \frametitle{Recursion Examples}

  \begin{itemize}
    \item consider two classic examples
  \end{itemize}

  \begin{example}[factorial - Haskell]
    \begin{pygments}{haskell}
fact n =
    if n == 0
    then 1
    else n * fact (n - 1)
    \end{pygments}
  \end{example}

  \begin{example}[greatest common divisor - Haskell]
    \begin{pygments}{haskell}
gcd_ x y =
    if y == 0
    then x
    else gcd_ y (x `mod` y)
    \end{pygments}
  \end{example}
\end{frame}

\begin{frame}[fragile]
  \frametitle{Stack Frame Example}

  \begin{example}[fact]
    \begin{pygments}{haskell}
fact n =
    if n == 0 then 1
    else n * fact (n - 1)
    \end{pygments}

    \pause
    \begin{verbatim}
fact 4
|- 4 * fact 3
       |- 3 * fact 2
              |- 2 * fact 1
                     |- 1 * fact 0
                            1
                     1
              2
       6
24
    \end{verbatim}
  \end{example}
\end{frame}

\begin{frame}[fragile]
  \frametitle{Stack Frame Example}

  \begin{example}[gcd\_]
    \begin{pygments}{haskell}
gcd_ x y =
    if y == 0 then x
    else gcd_ y (x `mod` y)
    \end{pygments}

    \pause
    \begin{verbatim}
gcd_ 9702 945
|- gcd_ 945 252
   |- gcd_ 252 189
      |- gcd_ 189 63
         |- gcd_ 63 0
         63
      63
   63
63
    \end{verbatim}
  \end{example}
\end{frame}

\subsection{Tail Recursion}

\begin{frame}
  \frametitle{Tail Recursion}

  \begin{itemize}
    \item if the result of the recursive call is also the result of the caller\\
      the function is said to be \alert{tail recursive}
    \item the recursive function call is the last action:\\
      nothing left for the caller to do

    \pause
    \medskip
    \item no need to keep the stack frame around\\
      $\rightarrow$ reuse the frame of the caller

    \pause
    \medskip
    \item to rearrange a function to be tail recursive
    \begin{itemize}
      \item define a helper function that takes an \alert{accumulator}
      \item base case: return accumulator
      \item recursive case: make recursive call with new accumulator value
    \end{itemize}
  \end{itemize}
\end{frame}

\begin{frame}[fragile]
  \frametitle{Tail Recursion Example}

  \begin{example}[factorial]
    \begin{pygments}{haskell}
factIter acc x =
    if x == 0
    then acc
    else factIter (acc * x) (x - 1)

fact n = factIter 1 n
    \end{pygments}
  \end{example}
\end{frame}

\begin{frame}[fragile]
  \frametitle{Stack Frame Example}

  \begin{example}[fact - tail recursive]
    \begin{pygments}{haskell}
factIter acc x =
    if x == 0 then acc
    else factIter (acc * x) (x - 1)
    \end{pygments}

    \pause
    \begin{verbatim}
fact 4
|- factIter 1 4
|- factIter 4 3
|- factIter 12 2
|- factIter 24 1
|- factIter 24 0
24
    \end{verbatim}
  \end{example}
\end{frame}

\subsection{Nested Functions}

\begin{frame}
  \frametitle{Nested Functions}

  \begin{itemize}
    \item functions can be defined locally within other functions
    \item they can use bindings from the environment in which they are defined
  \end{itemize}
\end{frame}

\begin{frame}[fragile]
  \frametitle{Nested Function Example}

  \begin{example}[factorial - Haskell]
    \begin{itemize}
      \item no need for \pygment{haskell}{factIter} to be visible
        outside of \pygment{haskell}{fact}
    \end{itemize}

    \begin{pygments}{haskell}
fact n =
    let
        factIter acc x =
            if x == 0
            then acc
            else factIter (acc * x) (x - 1)
    in
        factIter 1 n
    \end{pygments}
  \end{example}
\end{frame}

\begin{frame}[fragile]
  \frametitle{Recursion Exercise}

  \begin{block}{Exercise}
    \begin{itemize}
      \item what is the type of \pygment{haskell}{fact}?
      \item what happens if called as \pygment{haskell}{fact 2.4}?
      \item what if \pygment{haskell}{factIter :: Integer -> Integer -> Integer}?
    \end{itemize}
  \end{block}
\end{frame}

\begin{frame}[fragile]
  \frametitle{Nested Function Example}

  \begin{example}[factorial - Scala]
    \begin{pygments}{scala}
def fact(n: Int): Int = {
    def factIter(acc: Int, x: Int): Int =
        if (x == 0) acc
        else factIter(acc * x, x - 1)
    factIter(1, n)
}
    \end{pygments}
  \end{example}
\end{frame}

\subsection{Errors}

\begin{frame}[fragile]
  \frametitle{Generating Errors}

  \begin{block}{Haskell}
    \begin{itemize}
      \item an exception can be raised using \pygment{haskell}{error}
    \end{itemize}
  \end{block}

  \begin{example}[factorial - Haskell]
    \begin{pygments}{haskell}
fact n =
    if n < 0
    then error "negative parameter"
    else if n == 0
         then 1
         else n * fact (n - 1)
    \end{pygments}
  \end{example}
\end{frame}

\begin{frame}[fragile]
  \frametitle{Generating Errors}

  \begin{block}{Scala}
    \begin{itemize}
      \item an exception can be raised using \pygment{scala}{throw}
    \end{itemize}
  \end{block}

  \begin{example}[factorial - Scala]
    \begin{pygments}{scala}
def fact(n: Int): Int =
    if (n < 0) throw
        new IllegalArgumentException("negative parameter")
    else if (n == 0) 1
         else n * fact(n - 1)
    \end{pygments}
  \end{example}
\end{frame}

\section{Lists}

\subsection{Model}

\begin{frame}
  \frametitle{Model}

  \begin{itemize}
    \item a list of type \texttt{$\alpha$}
      consists of an element of type \texttt{$\alpha$},\\
      followed by a sublist of type \texttt{$\alpha$}
    \item the first element is the \alert{head},\\
      the sublist of remaining elements is the \alert{tail}
  \end{itemize}

  \pause
  \medskip
  \begin{block}{Haskell}
    \begin{itemize}
      \item check if empty: \pygment{haskell}{null}
      \item get the head: \pygment{haskell}{head}
      \item get the tail: \pygment{haskell}{tail}
      \item construct a list: \pygment{haskell}{item : sublist}
    \end{itemize}
  \end{block}
\end{frame}

\begin{frame}[fragile]
  \frametitle{List Operation Examples}

  \begin{example}[Haskell]
    \begin{pygments}{haskell}
Prelude> head [1,2,3,4]
1
Prelude> tail [1,2,3,4]
[2,3,4]
Prelude> null [1,2,3,4]
False
Prelude> 1 : [2,3]
[1,2,3]
Prelude> 1 : (2 : (3 : []))
[1,2,3]
Prelude> 1 : 2 : 3 : []
[1,2,3]
    \end{pygments}
  \end{example}
\end{frame}

\begin{frame}
  \frametitle{Scala Lists}

  \begin{block}{Scala}
    \begin{itemize}
      \item check if empty: \pygment{scala}{.isEmpty}
      \item get the head: \pygment{haskell}{.head}
      \item get the tail: \pygment{haskell}{.tail}
      \item construct a list: \pygment{haskell}{item :: sublist}
    \end{itemize}
  \end{block}
\end{frame}

\subsection{List Recursion}

\begin{frame}[fragile]
  \frametitle{List Recursion Example}

  \begin{example}
    \begin{itemize}
      \item a function that computes the length of a list
    \end{itemize}
    
    \begin{pygments}{haskell}
length_ xs =
    if null xs
    then 0
    else 1 + length_ (tail xs)
    \end{pygments}
  \end{example}

  \pause
  \begin{block}{Exercise}
    \begin{itemize}
      \item write a tail recursive version
    \end{itemize}
  \end{block}
\end{frame}

\begin{frame}<beamer>[fragile]
  \frametitle{List Recursion Example}

  \begin{example}
    \begin{pygments}{haskell}
length_ xs =
    let
        lengthIter acc items =
            if null items
            then acc
            else lengthIter (acc + 1) (tail items)
    in
        lengthIter 0 xs
    \end{pygments}
  \end{example}
\end{frame}

\begin{frame}[fragile]
  \frametitle{List Recursion Example}
  \begin{example}
    \begin{itemize}
      \item a function that computes the sum of a list
    \end{itemize}

    \pause
    \begin{pygments}{haskell}
sum_ xs =
    if null xs
    then 0
    else head xs + sum_ (tail xs)
    \end{pygments}
  \end{example}

  \pause
  \begin{example}[Scala]
    \begin{pygments}{scala}
def sum_(xs: List[Int]): Int =
    if (xs.isEmpty) 0
    else xs.head + sum_(xs.tail)
    \end{pygments}
  \end{example}
\end{frame}

\begin{frame}[fragile]
  \frametitle{List Recursion Example}

  \begin{example}
    \begin{itemize}
      \item a function that appends a list at the end of another list
    \end{itemize}

    \pause
    \begin{pygments}{haskell}
append xs ys =
    if null xs
    then ys
    else head xs : append (tail xs) ys
    \end{pygments}
  \end{example}

  \pause
  \begin{itemize}
    \item Haskell list append operator: \pygment{haskell}{++}
  \end{itemize}
\end{frame}

\begin{frame}[fragile]
  \frametitle{List Recursion Example}

  \begin{example}
    \begin{itemize}
      \item a function that generates a list from a lower limit to an upper limit
    \end{itemize}

    \pause
    \begin{pygments}{haskell}
countUp lower upper =
    if lower > upper
    then []
    else lower : countUp (lower + 1) upper
    \end{pygments}
  \end{example}

  \pause
  \begin{block}{Exercise}
    \begin{itemize}
      \item write a tail recursive version that generates the list\\
        from an upper limit down to a lower limit
    \end{itemize}
  \end{block}
\end{frame}

\begin{frame}<beamer>[fragile]
  \frametitle{List Recursion Example}

  \begin{example}
    \begin{pygments}{haskell}
countDown upper lower =
    let
        countDownIter acc u =
            if u < lower
            then acc
            else countDownIter (acc ++ [u]) (u - 1)
    in
        countDownIter [] upper
    \end{pygments}

% TODO: note the use of lower
  \end{example}
\end{frame}

\begin{frame}[fragile]
  \frametitle{List Recursion Example}

  \begin{example}
    \begin{itemize}
      \item a function that finds the maximum of a list
    \end{itemize}

    \pause
    \begin{pygments}{haskell}
max_ xs =
    if null xs
    then error "empty list"
    else if null (tail xs)
         then head xs
         else if head xs > max_ (tail xs)
              then head xs
              else max_ (tail xs)
    \end{pygments}
  \end{example}
\end{frame}

\begin{frame}[fragile]
  \frametitle{List Recursion Exercise}

  \begin{block}{Exercise}
    \begin{itemize}
      \item what happens if called as:\\
        \pygment{haskell}{max_ (countDown 30 1)}\\
        \pygment{haskell}{max_ (countUp 1 30)}
      \item fix the problem
   \end{itemize}
  \end{block}
\end{frame}

\begin{frame}<beamer>[fragile]
  \frametitle{List Recursion Example}

  \begin{example}
    \begin{pygments}{haskell}
max_ xs =
    if null xs
    then error "empty list"
    else if null (tail xs)
         then head xs
         else
             let
                 maxRest = max_ (tail xs)
             in
                 if head xs > maxRest
                 then head xs
    \end{pygments}
  \end{example}
\end{frame}

\begin{frame}
  \frametitle{List Recursion Exercises}

  \begin{block}{Exercises}
    \begin{itemize}
      \item write a function that will reverse a list
      \item write a tail recursive function that will reverse a list
    \end{itemize}
  \end{block}
\end{frame}

\begin{frame}<beamer>[fragile]
  \frametitle{List Recursion Example}

  \begin{example}
    \begin{pygments}{haskell}
reverse_ xs =
    if null xs
    then []
    else reverse_ (tail xs) ++ [head xs]
    \end{pygments}
  \end{example}
\end{frame}

\begin{frame}<beamer>[fragile]
  \frametitle{List Recursion Example}

  \begin{example}
    \begin{pygments}{haskell}
reverse_ xs =
    let
        reverseIter acc items =
            if null items
            then acc
            else reverseIter (head items : acc) (tail items)
    in
        reverseIter [] xs
    \end{pygments}
  \end{example}
\end{frame}
% 
% \section*{References}
% 
% \begin{frame}
%   \frametitle{References}
% 
%   \begin{block}{Required Reading: Abelson, Sussman}
%     \begin{itemize}
%       \item Chapter 1: Building Abstractions with Procedures
%       \begin{itemize}
%         \item 1.1. \alert{The Elements of Programming}
%       \end{itemize}
%     \end{itemize}
%   \end{block}
% \end{frame}

\end{document}
