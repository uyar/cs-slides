% Copyright (c) 2013-2016 H. Turgut Uyar <uyar@itu.edu.tr>
%
% This work is licensed under a "Creative Commons
% Attribution-NonCommercial-ShareAlike 4.0 International License".
% For more information, please visit:
% https://creativecommons.org/licenses/by-nc-sa/4.0/

\documentclass[dvipsnames]{beamer}

\usepackage[scaled=0.95]{cabin}
\usepackage[scaled=0.88]{beramono}
\usepackage[T1]{fontenc}
\usepackage[utf8]{inputenc}

\usepackage{listings}
\lstset{basicstyle=\ttfamily,
        keywordstyle=\color{blue},
        showstringspaces=false}
\lstdefinestyle{syntax}{frame=single}
\lstset{language=Haskell}

\mode<presentation>
{
  \usetheme{default}
  \useinnertheme{rounded}
  \usecolortheme{seahorse}
  \setbeamercovered{transparent}
}

\title{Functional Programming}
\subtitle{Recursion}

\author{H. Turgut Uyar}
\date{2013-2016}

\AtBeginSubsection[]{
  \begin{frame}<beamer>
    \frametitle{Topics}
    \tableofcontents[currentsection,currentsubsection]
  \end{frame}
}

\theoremstyle{plain}

\pgfdeclareimage[height=1cm]{license}{../license}

\pgfdeclareimage[height=6cm]{fib}{fib}

\begin{document}

\begin{frame}
  \titlepage
\end{frame}

\begin{frame}
  \frametitle{License}

  \pgfuseimage{license}\hfill
  \copyright~2013-2016 H. Turgut Uyar

  \vfill
  \begin{footnotesize}
    You are free to:
    \begin{itemize}
      \itemsep0em
      \item Share -- copy and redistribute the material in any medium or format
      \item Adapt -- remix, transform, and build upon the material
    \end{itemize}

    Under the following terms:
    \begin{itemize}
      \itemsep0em
      \item Attribution -- You must give appropriate credit, provide a link to
        the license, and indicate if changes were made.

      \item NonCommercial -- You may not use the material for commercial
        purposes.

      \item ShareAlike -- If you remix, transform, or build upon the material,
        you must distribute your contributions under the same license as the
        original.
    \end{itemize}

    For more information:\\
    \url{https://creativecommons.org/licenses/by-nc-sa/4.0/}

    \smallskip
    Read the full license:\\
    \url{https://creativecommons.org/licenses/by-nc-sa/4.0/legalcode}
  \end{footnotesize}
\end{frame}

\begin{frame}
  \frametitle{Topics}
  \tableofcontents
\end{frame}

\section{Recursion}

\lstset{deletekeywords={gcd}}

\subsection{Primitive Recursion}

\begin{frame}[fragile]
  \frametitle{Recursion Examples}

  \begin{exampleblock}{greatest common divisor}
    \begin{lstlisting}
gcd :: Integer -> Integer -> Integer
gcd x y = if y == 0 then x else gcd y (x `mod` y)
    \end{lstlisting}
  \end{exampleblock}

  \pause
  \begin{exampleblock}{factorial}
    \begin{lstlisting}
fac :: Integer -> Integer
fac n
  | n <  0    = error "negative parameter"
  | n == 0    = 1
  | otherwise = n * fac (n - 1)
    \end{lstlisting}
  \end{exampleblock}
\end{frame}

\begin{frame}[fragile]
  \frametitle{Stack Frame Example}

  \begin{exampleblock}{}
    \begin{lstlisting}
gcd x y = if y == 0 then x else gcd y (x `mod` y)
    \end{lstlisting}

    \begin{lstlisting}[frame=single]
gcd 9702 945
~> gcd 945 252
   ~> gcd 252 189
      ~> gcd 189 63
         ~> gcd 63 0
            ~> 63
         ~> 63
      ~> 63
   ~> 63
~> 63
    \end{lstlisting}
  \end{exampleblock}
\end{frame}

\begin{frame}[fragile]
  \frametitle{Stack Frame Example}

  \begin{exampleblock}{}
    \begin{lstlisting}
fac n
  | n <  0    = error "negative parameter"
  | n == 0    = 1
  | otherwise = n * fac (n - 1)
    \end{lstlisting}

    \begin{lstlisting}[frame=single]
fac 4
~> 4 * fac 3
       ~> 3 * fac 2
              ~> 2 * fac 1
                     ~> 1 * fac 0
                            ~> 1
                     ~> 1
              ~> 2
       ~> 6
~> 24
    \end{lstlisting}
  \end{exampleblock}
\end{frame}

\subsection{Tail Recursion}

\begin{frame}
  \frametitle{Tail Recursion}

  \begin{itemize}
    \item \alert{tail recursive}: result of recursive call is also result of caller
    \item recursive call is last action, nothing left for caller to do

    \pause
    \bigskip
    \item no need to keep the stack frame, reuse frame of caller
    \item increased performance
  \end{itemize}
\end{frame}

\begin{frame}[fragile]
  \frametitle{Stack Frame Example}

  \begin{exampleblock}{}
    \begin{lstlisting}
gcd x y = if y == 0 then x else gcd y (x `mod` y)
    \end{lstlisting}

    \begin{lstlisting}[frame=single]
gcd 9702 945
~> gcd 945 252
~> gcd 252 189
~> gcd 189 63
~> gcd 63 0
~> 63
    \end{lstlisting}
  \end{exampleblock}
\end{frame}

\begin{frame}
  \frametitle{Tail Recursion}

  \begin{itemize}
    \item rearranging a function to be tail recursive:

    \medskip
    \item define a helper function that takes an accumulator
    \item base case: return accumulator
    \item recursive case: make recursive call with new accumulator value
  \end{itemize}
\end{frame}

\begin{frame}[fragile]
  \frametitle{Tail Recursion Example}

  \begin{exampleblock}{tail recursive factorial}
    \begin{lstlisting}
facIter :: Integer -> Integer -> Integer
facIter acc n
  | n <  0    = error "negative parameter"
  | n == 0    = acc
  | otherwise = facIter (acc * n) (n - 1)

fac :: Integer -> Integer
fac n = facIter 1 n
    \end{lstlisting}
  \end{exampleblock}
\end{frame}

\begin{frame}[fragile]
  \frametitle{Stack Frame Example}

  \begin{exampleblock}{}
  \begin{lstlisting}
facIter acc n
  | n <  0    = error "negative parameter"
  | n == 0    = acc
  | otherwise = facIter (acc * n) (n - 1)
  \end{lstlisting}

    \begin{lstlisting}[frame=single]
fac 4
~> facIter 1 4
   ~> facIter 4 3
   ~> facIter 12 2
   ~> facIter 24 1
   ~> facIter 24 0
   ~> 24
~> 24
    \end{lstlisting}
  \end{exampleblock}
\end{frame}

\begin{frame}[fragile]
  \frametitle{Tail Recursion Example}

  \begin{itemize}
    \item helper function can be local
    \item negativity check only once
  \end{itemize}

  \begin{lstlisting}
fac :: Integer -> Integer
fac n
  | n < 0     = error "negative parameter"
  | otherwise = facIter 1 n
      where
        facIter :: Integer -> Integer -> Integer
        facIter acc n'
          | n' == 0   = acc
          | otherwise = facIter (acc * n') (n' - 1)
  \end{lstlisting}
\end{frame}

\subsection{Tree Recursion}

\begin{frame}[fragile]
  \frametitle{Tree Recursion}

  \begin{exampleblock}{Fibonacci sequence}
    \[
      fib_n =
        \begin{cases}
          1                     & \mbox{if } n = 1\\
          1                     & \mbox{if } n = 2\\
          fib_{n-2} + fib_{n-1} & \mbox{if } n > 2
        \end{cases}
    \]

    \begin{lstlisting}
fib :: Integer -> Integer
fib n
  | n == 1    = 1
  | n == 2    = 1
  | otherwise = fib (n - 2) + fib (n - 1)
    \end{lstlisting}
  \end{exampleblock}
\end{frame}

\begin{frame}[fragile]
  \frametitle{Stack Frame Example}

  \begin{center}
    \pgfuseimage{fib}
  \end{center}
\end{frame}

\section{Examples}

\subsection{Exponentiation}

\begin{frame}[fragile]
  \frametitle{Exponentiation}

  \begin{lstlisting}
pow :: Integer -> Integer -> Integer
pow x y
  | y == 0    = 1
  | otherwise = x * pow x (y - 1)
  \end{lstlisting}

  \pause
  \begin{itemize}
    \item use the following definition:
  \end{itemize}
  \[
    x^y =
      \begin{cases}
        1               & \mbox{if y = 0}\\
        {(x^{y/2})}^2   & \mbox{if y is even}\\
        x \cdot x^{y-1} & \mbox{if y is odd}
      \end{cases}
  \]
\end{frame}

\begin{frame}[fragile]
  \frametitle{Fast Exponentiation}

  \begin{lstlisting}
pow :: Integer -> Integer -> Integer
pow x y
  | y == 0    = 1
  | even y    = sqr (pow x (y `div` 2))
  | otherwise = x * pow x (y - 1)
  where
    sqr :: Integer -> Integer
    sqr n = n * n
  \end{lstlisting}
\end{frame}

% \begin{frame}[fragile]
%   \frametitle{List Maximum}
%
% % TODO: find an example that doesn't use lists
%
%   \begin{exampleblock}{maximum of a list}
%     \begin{lstlisting}
% maxList :: [Integer] -> Integer
% maxList []         = error "empty list"
% maxList [x]        = x
% maxList (x:xs)
%   | x > maxList xs = x
%   | otherwise      = maxList xs
%     \end{lstlisting}
%   \end{exampleblock}
%
%   \pause
%   \begin{itemize}
%     \item what if called as:\\
%       \lstinline|maxList [30, 29 .. 1]|\\
%       \lstinline|maxList [1 .. 30]|
%   \end{itemize}
% \end{frame}
%
% \begin{frame}[fragile]
%   \frametitle{List Maximum}
%
%   \begin{exampleblock}{maximum of a list}
%     \begin{lstlisting}
% maxList :: [Integer] -> Integer
% maxList []      = error "empty list"
% maxList [x]     = x
% maxList (x:xs)
%   | x > maxTail = x
%   | otherwise   = maxTail
%   where
%     maxTail = maxList xs
%     \end{lstlisting}
%   \end{exampleblock}
% \end{frame}

\subsection{Counting Change}

\begin{frame}[fragile]
  \frametitle{Counting Change}

  \begin{itemize}
    \item how many different ways to change given amount of money?
    \item into units of 1, 5, 10, 25, 50
  \end{itemize}

  \begin{lstlisting}
countChange :: Integer -> Integer
countChange amount = cc amount 5
  where
    -- fd :: Integer -> Integer
    -- 1 ~> 1, 2 ~> 5, 3 ~> 10, 4 ~> 25, 5 ~> 50

    cc :: Integer -> Integer -> Integer
    cc a n
      | a == 0          = 1
      | a < 0 || n == 0 = 0
      | otherwise       = cc a (n - 1) + cc (a - fd n) n
  \end{lstlisting}
\end{frame}

\subsection{Square Roots}

\lstset{deletekeywords=sqrt}

\begin{frame}
  \frametitle{Square Roots with Newton's Method}

  \begin{itemize}
    \item start with an initial guess $y$ (say $y = 1$)
    \item repeatedly improve the guess by taking the mean of $y$
      and $x / y$
    \item until the guess is good enough ($\sqrt x \cdot \sqrt x = x$)
  \end{itemize}

  \medskip
  \begin{exampleblock}{example: $\sqrt{2}$}
    \begin{center}
      \begin{tabular}{lll}
      $y$      & $x / y$             & next guess\\\hline
      1        & 2 / 1 = 2           & 1.5\\
      1.5      & 2 / 1.5 = 1.333     & 1.4167\\
      1.4167   & 2 / 1.4167 = 1.4118 & 1.4142\\
      1.4142   & ...                 & ...
      \end{tabular}
    \end{center}
  \end{exampleblock}
\end{frame}

\begin{frame}[fragile]
  \frametitle{Square Roots with Newton's Method}

  \begin{lstlisting}
newton :: Float -> Float -> Float
newton guess x
  | isGoodEnough guess x = guess
  | otherwise            = newton (improve guess x) x
  \end{lstlisting}

  \pause
  \begin{lstlisting}
isGoodEnough :: Float -> Float -> Bool
isGoodEnough guess x = abs (guess*guess - x) < 0.001
  \end{lstlisting}

  \pause
  \begin{lstlisting}
improve :: Float -> Float -> Float
improve guess x = (guess + x/guess) / 2.0
  \end{lstlisting}

  \pause
  \begin{lstlisting}
sqrt :: Float -> Float
sqrt x = newton 1.0 x
  \end{lstlisting}
\end{frame}

\begin{frame}[fragile]
  \frametitle{Square Roots with Newton's Method}

  \begin{lstlisting}
sqrt :: Float -> Float
sqrt x = newton 1.0 x
  where
    newton :: Float -> Float -> Float
    newton guess x'
      | isGoodEnough guess x' = guess
      | otherwise             = newton (improve guess x') x'

    isGoodEnough :: Float -> Float -> Bool
    isGoodEnough guess x' =
        abs (guess*guess - x') < 0.001

    improve :: Float -> Float -> Float
    improve guess x' = (guess + x'/guess) / 2.0
  \end{lstlisting}
\end{frame}

\begin{frame}[fragile]
  \frametitle{Square Roots with Newton's Method}

  \begin{itemize}
    \item doesn't work with too small and too large numbers (why?)
  \end{itemize}

  \begin{lstlisting}
isGoodEnough guess x' =
    (abs (guess*guess - x')) / x' < 0.001
  \end{lstlisting}
\end{frame}

\begin{frame}[fragile]
  \frametitle{Square Roots with Newton's Method}

  \begin{itemize}
    \item no need to pass \lstinline|x| around, it's already in scope
  \end{itemize}

  \begin{lstlisting}
sqrt x = newton 1.0
  where
    newton :: Float -> Float
    newton guess
      | isGoodEnough guess = guess
      | otherwise          = newton (improve guess)

    isGoodEnough :: Float -> Bool
    isGoodEnough guess =
        (abs (guess*guess - x)) / x < 0.001

    improve :: Float -> Float
    improve guess = (guess + x/guess) / 2.0
  \end{lstlisting}
\end{frame}

\section*{References}

\begin{frame}
  \frametitle{References}

  \begin{block}{Required Reading: Thompson}
    \begin{itemize}
      \item Chapter 3: \alert{Basic types and definitions}
      \item Chapter 4: \alert{Designing and writing programs}
    \end{itemize}
  \end{block}
\end{frame}

\end{document}
