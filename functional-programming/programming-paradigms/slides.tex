% Copyright (c) 2013-2014 H. Turgut Uyar <uyar@itu.edu.tr>
%
% This work is licensed under a "Creative Commons
% Attribution-NonCommercial-ShareAlike 4.0 International License".
% For more information, please visit:
% https://creativecommons.org/licenses/by-nc-sa/4.0/

\documentclass[dvipsnames]{beamer}

\usepackage{ae}
\usepackage[T1]{fontenc}
\usepackage[utf8]{inputenc}
\usepackage{pythontex}
\setbeamertemplate{navigation symbols}{}
\setbeamersize{text margin left=2em, text margin right=2em}

\mode<presentation>
{
  \usetheme{default}
  \useinnertheme{rounded}
  \usecolortheme{seahorse}
  \setbeamercovered{transparent}
}

\title{Functional Programming}
\subtitle{Programming Paradigms}

\author{H. Turgut Uyar}
\date{2013-2014}

\AtBeginSubsection[]{
  \begin{frame}<beamer>
    \frametitle{Topics}
    \tableofcontents[currentsection,currentsubsection]
  \end{frame}
}

\theoremstyle{plain}

\pgfdeclareimage[height=1cm]{license}{../license}

\pgfdeclareimage[width=4cm]{turing}{turing}
\pgfdeclareimage[width=4cm]{church}{church}
\pgfdeclareimage[width=4cm]{backus}{backus}
\pgfdeclareimage[width=4cm]{mccarthy}{mccarthy}

\begin{document}

\setpythontexfv[]{frame=single}

\begin{frame}
  \titlepage
\end{frame}

\begin{frame}
  \frametitle{License}

  \pgfuseimage{license}\hfill
  \copyright~2013-2014 H. Turgut Uyar

  \vfill
  \begin{tiny}
    You are free to:
    \begin{itemize}
      \item Share -- copy and redistribute the material in any medium or format
      \item Adapt -- remix, transform, and build upon the material
    \end{itemize}

    Under the following terms:
    \begin{itemize}
      \item Attribution -- You must give appropriate credit, provide a link to
        the license, and indicate if changes were made.\\
        You may do so in any reasonable manner, but not in any way
        that suggests the licensor endorses you or your use.

      \item Noncommercial -- You may not use the material for commercial
        purposes.

      \item Share Alike -- If you remix, transform, or build upon the material,
        you must distribute your contributions\\
        under the same license as the original.
    \end{itemize}
  \end{tiny}

  \vfill
  \begin{small}
    Legal code (the full license):\\
    \url{https://creativecommons.org/licenses/by-nc-sa/4.0/legalcode}
  \end{small}
\end{frame}

\begin{frame}
  \frametitle{Topics}
  \tableofcontents
\end{frame}

\section{Programming Paradigms}

\subsection{Programming Languages}

\begin{frame}
  \frametitle{Programming Languages}

  \begin{itemize}
    \item \alert{syntax}: rules for writing a ``grammatically correct'' program
    \begin{itemize}
      \item how expressions, commands, declarations and other constructs\\
        must be arranged to make a well-formed program
    \end{itemize}

    \pause
    \bigskip
    \item \alert{semantics}: how the program should be ``interpreted''
    \begin{itemize}
      \item how the program may be expected to behave\\
        when executed on a computer
    \end{itemize}
  \end{itemize}
\end{frame}

\begin{frame}
  \frametitle{Idioms}

  \begin{itemize}
    \item knowing the syntax and understanding the semantics\\
      is not sufficient to master a programming language

    \bigskip
    \item standard library
    \item other libraries
    \item tools: debugging, testing, profiling
    \item documenting
    \item style: code formatting, variable naming, \ldots

    \pause
    \bigskip
    \item \alert{idioms}: patterns for using language features
  \end{itemize}
\end{frame}

\begin{frame}
  \frametitle{Universality}

  \begin{itemize}
    \item \alert{universal}: capable of expressing any computation
    \item any language that supports iteration or recursion is universal
  \end{itemize}

  \pause
  \medskip
  \begin{block}{Church-Turing Thesis (from Wolfram MathWorld)}
    Any real-world computation can be translated into\\
    an equivalent computation involving a Turing machine.\\
    \medskip
    It can also be calculated using general recursive functions.
  \end{block}
\end{frame}

\subsection{Imperative Programming}

\begin{frame}
  \frametitle{Imperative Programming}

  \begin{columns}
    \column{.4\textwidth}
    \begin{center}
      \pgfuseimage{turing}\\
      Alan Turing (1912-1954)
    \end{center}

    \column{.6\textwidth}
    \begin{itemize}
      \item based on the Turing machine
      \item a program is a list of instructions\\
        for a von Neumann computer
      \item contents of memory constitutes\\
        the program \alert{state}

      \pause
      \medskip
      \item statements update variables\\
        (\alert{mutation})
      \item assignment and control structures
      \item natural model of hardware
    \end{itemize}
  \end{columns}
\end{frame}

\begin{frame}[fragile]
  \frametitle{Imperative Programming Example}

  \begin{columns}
    \column{.55\textwidth}
    \begin{exampleblock}{greatest common divisor (Python)}
      \begin{pygments}{python}
def gcd(x, y):
    r = 0
    while y > 0:
        r = x % y
        x = y
        y = r
    return x
      \end{pygments}
    \end{exampleblock}

    \column{.35\textwidth}
    \begin{verbatim}
   x    y    r
--------------
9702  945    0
 945  252  252
 252  189  189
 189   63   63
  63    0    0

63
    \end{verbatim}
  \end{columns}
\end{frame}

\begin{frame}
  \frametitle{Milestones in Imperative Programming Languages}

  \begin{columns}
    \column{.45\textwidth}
    \begin{center}
      \pgfuseimage{backus}\\
      John Backus (1924-2007)
    \end{center}

    \column{.55\textwidth}
    \begin{itemize}
      \item Fortran (1957)
      \item ALGOL (1960)
      \item C (1972)
      \item Ada (1983)
      \item Java (1995)
    \end{itemize}
  \end{columns}
\end{frame}

\subsection{Functional Programming}

\begin{frame}
  \frametitle{Functional Programming}

  \begin{columns}
    \column{.4\textwidth}
    \begin{center}
      \pgfuseimage{church}\\
      Alonzo Church (1903-1995)
    \end{center}

    \column{.6\textwidth}
    \begin{itemize}
      \item based on $\lambda$-calculus
      \item a program is a function application
      \item the same inputs should produce\\
        the same output

      \pause
      \medskip
      \item the function also changes\\
        the context $\rightarrow$ \alert{side effect}
      \item avoid mutation

      \pause
      \medskip
      \item higher order functions
    \end{itemize}
  \end{columns}
\end{frame}

\begin{frame}[fragile]
  \frametitle{Functional Programming Example}

  \begin{columns}
    \column{.55\textwidth}
    \begin{exampleblock}{greatest common divisor (Python)}
      \begin{pygments}{python}
def gcd(x, y):
    if y == 0:
        return x
    else:
        return gcd(y, x % y)
      \end{pygments}
    \end{exampleblock}

    \column{.35\textwidth}
    \begin{verbatim}
gcd 9702 945
|- gcd 945 252
   |- gcd 252 189
      |- gcd 189 63
         |- gcd 63 0
            63
         63
      63
   63
63
    \end{verbatim}
  \end{columns}
\end{frame}

\begin{frame}[fragile]
  \frametitle{Side Effects}

  \begin{itemize}
    \item sources of side effects: global variables
  \end{itemize}

  \begin{exampleblock}{}
    \begin{pygments}{python}
factor = 0

def multiples(n):
    global factor
    factor = factor + 1
    return factor * n
      \end{pygments}
    \end{exampleblock}
\end{frame}

\begin{frame}[fragile]
  \frametitle{Side Effects}

  \begin{itemize}
    \item sources of side effects: function state, object state
  \end{itemize}

  \begin{exampleblock}{}
    \begin{pygments}{python}
class Multiplier:
    def __init__(self):
        self.factor = 0

    def multiples(self, n):
        self.factor = self.factor + 1
        return self.factor * n
    \end{pygments}
  \end{exampleblock}
\end{frame}

\begin{frame}[fragile]
  \frametitle{Side Effects}

  \begin{itemize}
    \item sources of side effects: input/output
  \end{itemize}

  \begin{exampleblock}{}
    \begin{pygments}{python}
def read_byte(f):
    return f.read(1)
    \end{pygments}
  \end{exampleblock}
\end{frame}

\begin{frame}[fragile]
  \frametitle{Side Effects}

  \begin{itemize}
    \item sources of side effects: randomness
  \end{itemize}

  \begin{exampleblock}{}
    \begin{pygments}{python}
import random

def get_random(n):
    return random.randrange(1, n + 1)
    \end{pygments}
  \end{exampleblock}
\end{frame}

\begin{frame}
  \frametitle{Problems with Side Effects}

  \begin{itemize}
    \item harder to reason about programs
    \item harder to test programs
    \item harder to parallelize programs

    \pause
    \bigskip
    \item could we write programs without relying on side effects?
    \item or, at least, could we constrain side effects?
  \end{itemize}
\end{frame}

\begin{frame}
  \frametitle{Milestones in Functional Programming Languages}

  \begin{columns}
    \column{.45\textwidth}
    \begin{center}
      \pgfuseimage{mccarthy}\\
      John McCarthy (1927-2011)
    \end{center}

    \column{.55\textwidth}
    \begin{itemize}
      \item Lisp (1957)
      \item ML (1973)
      \item Haskell (1990)
    \end{itemize}
  \end{columns}
\end{frame}

\begin{frame}
  \frametitle{Object Orientation}

  \begin{itemize}
    \item object orientation is about data abstraction
    \item type hierarchies

    \medskip
    \item these can be achieved in both paradigms
    \begin{itemize}
      \item Ocaml, F\#
      \item Scala
    \end{itemize}
    \item modern imperative languages support functional features
    \begin{itemize}
      \item Ruby, Python
      \item Java, C\#
    \end{itemize}

    \pause
    \bigskip
    \item what makes a language functional or imperative?
    \begin{itemize}
      \item higher order functions
      \item most code written in functional style
    \end{itemize}
  \end{itemize}
\end{frame}

\section{Basic Concepts}

\subsection{Expressions}

\begin{frame}
  \frametitle{Expressions}

  \medskip
  \begin{itemize}
    \item \alert{expression}: a construct that will be evaluated
      to yield a value
    \begin{itemize}
      \item primitive expressions: literals, constants, variables
      \item combining expressions: operators
    \end{itemize}

    \pause
    \medskip
    \item \alert{statement}: a construct that will be executed
      to update a variable
  \end{itemize}
\end{frame}

\begin{frame}[fragile]
  \frametitle{Expression Example}

  \begin{exampleblock}{conditional statement (Python)}
    \begin{pygments}{python}
if x < 0:
    abs_x = -x
else:
    abs_x = x
    \end{pygments}
  \end{exampleblock}

  \pause
  \begin{exampleblock}{conditional expression (Python)}
    \begin{pygments}{python}
abs_x = -x if x < 0 else x
    \end{pygments}
  \end{exampleblock}

  \pause
  \begin{exampleblock}{conditional expression (Haskell)}
    \begin{pygments}{haskell}
abs_x = if x < 0 then -x else x
    \end{pygments}
  \end{exampleblock}
\end{frame}

\begin{frame}<beamer>[fragile]
  \frametitle{Expression Example}

  \begin{pygments}{python}
if age < 18:
    minor = True
else:
    minor = False
  \end{pygments}

  \medskip
  \begin{pygments}{python}
minor = True if age < 18 else False
  \end{pygments}

  \pause
  \bigskip
  \begin{pygments}{python}
minor = age < 18
  \end{pygments}
\end{frame}

\subsection{Definitions}

\begin{frame}[fragile]
  \frametitle{Definitions}

  \medskip
  \begin{itemize}
    \item \alert{binding}: an association between an identifier and an entity
    \item \alert{environment}: a set of bindings
  \end{itemize}

  \begin{block}{definitions (Haskell)}
    \begin{pygments}{haskell}
name :: type
name = expression
    \end{pygments}
  \end{block}

  \pause
  \medskip
  \begin{itemize}
    \item in a pure functional language, redefining a variable is not allowed
  \end{itemize}
\end{frame}

\begin{frame}[fragile]
  \frametitle{Definition Examples}

  \begin{exampleblock}{}
    \begin{pygments}{haskell}
diameter :: Float
diameter = 4.8

circumference :: Float
circumference = 3.14159 * diameter

diameter = 15.62       -- multiple declarations
    \end{pygments}
  \end{exampleblock}
\end{frame}

\begin{frame}[fragile]
  \frametitle{Local Definitions}

  \medskip
  \begin{itemize}
    \item a \alert{local} definition can be used only within the expression\\
      where it is defined
  \end{itemize}

  \begin{columns}[b]
    \column{.53\textwidth}
    \begin{block}{}
      \begin{pygments}{haskell}
name = expression
  where
    name1 :: type1
    name1 = expression1

    name2 :: type2
    name2 = expression2
    ...
      \end{pygments}
    \end{block}

    \pause
    \column{.46\textwidth}
    \begin{exampleblock}{example}
      \begin{pygments}{haskell}
area = 3.14159 * r * r
  where
    r :: Float
    r = diameter / 2
      \end{pygments}
    \end{exampleblock}
  \end{columns}
\end{frame}

\subsection{Functions}

\begin{frame}
  \frametitle{Functions}

  \begin{itemize}
    \item in imperative languages, the body of a function is a block
    \begin{itemize}
      \item special construct for sending back the result: \alert{return}
    \end{itemize}

    \pause
    \bigskip
    \item in functional languages, the body of a function is an expression
  \end{itemize}
\end{frame}

\begin{frame}
  \frametitle{Function Parameters}

  \begin{itemize}
    \item \alert{formal parameter}: an identifier through which a function\\
      accesses an argument
    \begin{itemize}
      \item declared at function definition
    \end{itemize}

    \pause
    \medskip
    \item \alert{actual parameter}: an expression which yields an argument
    \begin{itemize}
      \item defined at function application
    \end{itemize}
  \end{itemize}
\end{frame}

\begin{frame}[fragile]
  \frametitle{Function Definitions}

  \begin{block}{}
    \begin{pygments}{haskell}
-- function definition
name :: t1 -> t2 -> ... -> tk -> t
name x1 x2 ... xk = expression

-- function application
name e1 e2 ... ek
    \end{pygments}
  \end{block}
\end{frame}

\begin{frame}[fragile]
  \frametitle{Function Examples}

  \begin{exampleblock}{}
    \begin{pygments}{haskell}
square :: Integer -> Integer
square x = x * x

-- square 21 -> 441
-- square (2 + 5) -> 49

sumOfSquares :: Integer -> Integer -> Integer
sumOfSquares x y = square x + square y

-- sumOfSquares 3 4 -> 25
-- sumOfSquares (a + 1) (a * 2)
    \end{pygments}
  \end{exampleblock}
\end{frame}

\begin{frame}[fragile]
  \frametitle{Function Example}

  \begin{exampleblock}{}
    \begin{pygments}{haskell}
sumOfCubes :: Integer -> Integer -> Integer
sumOfCubes x y = cube x + cube y
  where
    cube :: Integer -> Integer
    cube n = n * n * n
    \end{pygments}
  \end{exampleblock}
\end{frame}

\section*{References}

\begin{frame}
  \frametitle{References}

  \begin{block}{Required Reading: Thompson}
    \begin{itemize}
      \item Chapter 1: \alert{Introducing functional programming}
      \item Chapter 2: \alert{Getting started with Haskell and GHCi}
    \end{itemize}
  \end{block}
\end{frame}

\end{document}
