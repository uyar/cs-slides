% Copyright (c) 2013-2014 H. Turgut Uyar <uyar@itu.edu.tr>
%
% This work is licensed under a "Creative Commons
% Attribution-NonCommercial-ShareAlike 4.0 International License".
% For more information, please visit:
% https://creativecommons.org/licenses/by-nc-sa/4.0/

\documentclass[dvipsnames]{beamer}

\usepackage{ae}
\usepackage[T1]{fontenc}
\usepackage[utf8]{inputenc}
\usepackage{pythontex}
\setbeamertemplate{navigation symbols}{}
\setbeamersize{text margin left=2em, text margin right=2em}

\mode<presentation>
{
  \usetheme{default}
  \useinnertheme{rounded}
  \usecolortheme{seahorse}
  \setbeamercovered{transparent}
}

\title{Functional Programming}
\subtitle{Data Types and Pattern Matching}

\author{H. Turgut Uyar}
\date{2013-2014}

\AtBeginSubsection[]{
  \begin{frame}<beamer>
    \frametitle{Topics}
    \tableofcontents[currentsection,currentsubsection]
  \end{frame}
}

\theoremstyle{plain}

\pgfdeclareimage[height=1cm]{license}{../license}

\begin{document}

\setpythontexfv[]{frame=single}

\begin{frame}
  \titlepage
\end{frame}

\begin{frame}
  \frametitle{License}

  \pgfuseimage{license}\hfill
  \copyright~2013-2014 H. Turgut Uyar

  \vfill
  \begin{tiny}
    You are free to:
    \begin{itemize}
      \item Share -- copy and redistribute the material in any medium or format
      \item Adapt -- remix, transform, and build upon the material
    \end{itemize}

    Under the following terms:
    \begin{itemize}
      \item Attribution -- You must give appropriate credit, provide a link to
        the license, and indicate if changes were made.\\
        You may do so in any reasonable manner, but not in any way
        that suggests the licensor endorses you or your use.

      \item Noncommercial -- You may not use the material for commercial
        purposes.

      \item Share Alike -- If you remix, transform, or build upon the material,
        you must distribute your contributions\\
        under the same license as the original.
    \end{itemize}
  \end{tiny}

  \vfill
  \begin{small}
    Legal code (the full license):\\
    \url{https://creativecommons.org/licenses/by-nc-sa/4.0/legalcode}
  \end{small}
\end{frame}

\begin{frame}
  \frametitle{Topics}
  \tableofcontents
\end{frame}

\section{Data Types}

\subsection{Tuples}

\begin{frame}[fragile]
  \frametitle{Tuples}

  \begin{itemize}
    \item \alert{tuple}: a combination of a fixed number of values
      of fixed types
    \begin{block}{}
      \begin{pygments}{haskell}
name :: (t1, t2, ..., tn)
name = (v1,v2,...,vn)
      \end{pygments}
    \end{block}

    \medskip
    \item selector functions on pairs:\\
      \pygment{haskell}{fst}, \pygment{haskell}{snd}
  \end{itemize}
\end{frame}

\begin{frame}[fragile]
  \frametitle{Tuple Examples}

  \begin{exampleblock}{representing a term in a polynomial: $2.4x^2$}
    \begin{pygments}{haskell}
term1 :: (Float, Integer)
term1 = (2.4,2)

coeff1 = fst term1     -- 2.4
degree1 = snd term1    -- 2
    \end{pygments}
  \end{exampleblock}
\end{frame}

\begin{frame}[fragile]
  \frametitle{Tuple Parameters}

  \begin{itemize}
    \item tuple parameters are different from multiple parameters
  \end{itemize}

  \begin{exampleblock}{}
    \begin{pygments}{haskell}
gcd1 :: Integer -> Integer -> Integer
gcd1 x y
  | y == 0    = x
  | otherwise = gcd1 y (x `mod` y)
-- call as: gcd1 9702 945

gcd2 :: (Integer, Integer) -> Integer
gcd2 a
  | snd a == 0 = fst a
  | otherwise  = gcd2 (snd a,(fst a) `mod` (snd a))
-- call as: gcd2 (9702,945)
    \end{pygments}
  \end{exampleblock}
\end{frame}

\begin{frame}[fragile]
  \frametitle{Tuple Results}

  \begin{itemize}
    \item functions can return tuples
  \end{itemize}

  \begin{exampleblock}{example: greatest common divisor
                            and least common multiple}
    \begin{pygments}{haskell}
gcd_lcm :: Integer -> Integer -> (Integer, Integer)
gcd_lcm x y = (g, l)
  where
    gcd' :: Integer -> Integer -> Integer
    gcd' a b
      | b == 0    = a
      | otherwise = gcd' b (a `mod` b)

    g = gcd' x y
    l = (x * y) `div` g
    \end{pygments}
  \end{exampleblock}
\end{frame}

\subsection{Lists}

\begin{frame}[fragile]
  \frametitle{Lists}

  \begin{itemize}
    \item \alert{list}: a combination of an arbitrary number of values,\\
      all of the same type
    \begin{block}{}
      \begin{pygments}{haskell}
name :: [t]
name = [v1,v2,...,vn]
      \end{pygments}
    \end{block}
  \end{itemize}
\end{frame}

\begin{frame}[fragile]
  \frametitle{List Example}

  \begin{exampleblock}{representing polynomials}
    \begin{pygments}{haskell}
-- second degree polynomial
p1 :: (Float, Float, Float)
p1 = (2.4,1.8,-4.6)

-- any degree polynomial, coefficients ordered
p2 :: [Float]
p2 = [2.4,1.8,-4.6]

-- any degree polynomial
p3 :: [(Float, Integer)]
p3 = [(1.8,1),(-4.6,0),(2.4,2)]
    \end{pygments}
  \end{exampleblock}
\end{frame}

\begin{frame}[fragile]
  \frametitle{Type Synonyms}

  \begin{block}{type synonyms}
    \begin{pygments}{haskell}
type newName = oldName
    \end{pygments}
  \end{block}

  \begin{exampleblock}{example: representing polynomials}
    \begin{pygments}{haskell}
type Term = (Float, Integer)
type Polynomial = [Term]

p4 :: Polynomial
p4 = [(1.8,1),(-4.6,0),(2.4,2)]
    \end{pygments}
  \end{exampleblock}
\end{frame}

\begin{frame}
  \frametitle{Lists}

  \begin{itemize}
    \item a list consists of a first item (\alert{head})\\
      followed by a list of the remaining items (\alert{tail})

  \medskip
    \item basic list operations:
      \begin{itemize}
        \item check if empty: \pygment{haskell}{null}
        \item get the head: \pygment{haskell}{head}
        \item get the tail: \pygment{haskell}{tail}
        \item construct a list: \pygment{haskell}{item : sublist}
      \end{itemize}
  \end{itemize}
\end{frame}

\begin{frame}[fragile]
  \frametitle{List Operation Examples}

  \begin{exampleblock}{}
    \begin{pygments}{haskell}
-- null :: [a] -> Bool
null []        -- True
null [1,2,3,4] -- False

-- head :: [a] -> a
head [1,2,3,4] -- 1
head []        -- error

-- tail :: [a] -> [a]
tail [1,2,3,4] -- [2,3,4]
tail []        -- error
tail [1]       -- []

-- (:) :: a -> [a] -> [a]
1 : [2,3]      -- [1,2,3]
    \end{pygments}
  \end{exampleblock}
\end{frame}

\begin{frame}[fragile]
  \frametitle{List Example}

  \begin{exampleblock}{number of elements}
    \begin{pygments}{haskell}
length1 :: [a] -> Integer
length1 xs
  | null xs   = 0
  | otherwise = 1 + length1 (tail xs)
    \end{pygments}
  \end{exampleblock}

  \pause
  \begin{itemize}
    \item exercise: write a tail recursive version
    \item exercise: sum of elements
  \end{itemize}
\end{frame}

\begin{frame}<beamer>[fragile]
  \frametitle{List Example}

  \begin{exampleblock}{number of elements}
    \begin{pygments}{haskell}
length1' :: [a] -> Integer
length1' xs = lengthIter 0 xs
  where
    lengthIter :: Integer -> [a] -> Integer
    lengthIter acc xs'
      | null xs' = acc
      | otherwise = lengthIter (acc + 1) (tail xs')
    \end{pygments}
  \end{exampleblock}
\end{frame}

\begin{frame}<beamer>[fragile]
  \frametitle{List Example}

  \begin{exampleblock}{sum of elements}
    \begin{pygments}{haskell}
sum' :: [Integer] -> Integer
sum' xs
  | null xs   = 0
  | otherwise = head xs + sum' (tail xs)
    \end{pygments}
  \end{exampleblock}
\end{frame}

\subsection{Algebraic Types}

\begin{frame}[fragile]
  \frametitle{Algebraic Types}

  \begin{itemize}
    \item algebraic types can be used to define:
    \begin{itemize}
      \item enumerated types
      \item product types
      \item alternatives
    \end{itemize}
  \end{itemize}

  \begin{block}{}
    \begin{pygments}{haskell}
data Name = Constructor_1 t_1_1 t_1_2 ... |
            Constructor_2 t_2_1 t_2_2 ... |
            ...
            Constructor_n t_n_1 t_n_2 ...
    \end{pygments}
  \end{block}

  \begin{itemize}
    \item value construction:\\
      \pygment{haskell}{Constructor_i v_i_1 v_i_2 ...}
  \end{itemize}
\end{frame}

\begin{frame}[fragile]
  \frametitle{Enumerated Types}

  \begin{itemize}
    \item enumerated type: no components in constructors
  \end{itemize}

  \begin{exampleblock}{}
    \begin{pygments}{haskell}
data Month = Jan | Feb | Mar | Apr | May | Jun |
             Jul | Aug | Sep | Oct | Nov | Dec
             deriving Show

currentMonth :: Month
currentMonth = Feb
    \end{pygments}
  \end{exampleblock}
\end{frame}

\begin{frame}[fragile]
  \frametitle{Product Types}

  \begin{itemize}
    \item product type: one constructor with multiple components
  \end{itemize}

  \begin{exampleblock}{}
    \begin{pygments}{haskell}
data Pioneer = Person String Integer
               deriving Show
-- constructor: Person
-- components: name (String), birth year (Integer)

church :: Pioneer
church = Person "Alonzo Church" 1903

type Name = String
type BirthYear = Integer
data Pioneer = Person Name BirthYear
               deriving Show
    \end{pygments}
  \end{exampleblock}
\end{frame}

\begin{frame}[fragile]
  \frametitle{Alternative Types}

  \begin{itemize}
    \item alternative type: multiple constructors
  \end{itemize}

  \begin{exampleblock}{}
    \begin{pygments}{haskell}
data Shape = Point |
             Circle Float |
             Rectangle Float Float
             deriving Show

center, hole, box :: Shape
center = Point
hole = Circle 3.0
box = Rectangle 45.9 87.6
    \end{pygments}
  \end{exampleblock}
\end{frame}

\section{Pattern Matching}

\subsection{Function Parameters}

\begin{frame}[fragile]
  \frametitle{Pattern Matching}

  \begin{itemize}
    \item formal parameters are patterns
    \item components of the pattern will be matched\\
      with the components of the actual parameters
    \item a matched pattern generates bindings
  \end{itemize}

  \begin{exampleblock}{}
    \begin{pygments}{haskell}
gcd1 :: Integer -> Integer -> Integer
gcd1 x y
  | y == 0    = x
  | otherwise = gcd1 y (x `mod` y)

-- if called as: gcd1 9702 945
-- bindings: x <-> 9702, y <-> 945
    \end{pygments}
  \end{exampleblock}
\end{frame}

\begin{frame}[fragile]
  \frametitle{Multiple Patterns}

  \begin{itemize}
    \item in case of multiple patterns, the first match will be selected
    \item patterns can contain literals
  \end{itemize}

  \begin{block}{}
    \begin{pygments}{haskell}
name p1 = e1
name p2 = e2
...
    \end{pygments}
  \end{block}
\end{frame}

\begin{frame}[fragile]
  \frametitle{Multiple Pattern Examples}

  \begin{exampleblock}{}
    \begin{pygments}{haskell}
fact :: Integer -> Integer
fact 0 = 1
fact n = n * fact (n - 1)

gcd3 :: Integer -> Integer -> Integer
gcd3 x 0 = x
gcd3 x y = gcd3 y (x `mod` y)

-- if called as: gcd3 9702 945
-- second pattern, bindings: x <-> 9702, y <-> 945

-- if called as: gcd3 63 0
-- first pattern, bindings: x <-> 63
    \end{pygments}
  \end{exampleblock}
\end{frame}

\begin{frame}[fragile]
  \frametitle{Tuple Patterns}

  \begin{itemize}
    \item tuple selector functions work only with pairs
    \item pattern matching works with any tuple
    \item it is also more readable
  \end{itemize}

  \begin{exampleblock}{}
    \begin{pygments}{haskell}
gcd2 :: (Integer, Integer) -> Integer
gcd2 a
  | snd a == 0 = fst a
  | otherwise  = gcd2 (snd a,(fst a) `mod` (snd a))

gcd4 :: (Integer, Integer) -> Integer
gcd4 (x,0) = x
gcd4 (x,y) = gcd4 (y,x `mod` y)
    \end{pygments}
  \end{exampleblock}
\end{frame}

\begin{frame}<beamer>[fragile]
  \frametitle{Tuple Pattern Example}

  \begin{exampleblock}{efficient Fibonacci calculation}
    \begin{pygments}{haskell}
-- fibPair n = (fib n,fib (n + 1))

fibStep :: (Integer, Integer) -> (Integer, Integer)
fibStep (u,v) = (v,u+v)

fibPair :: Integer -> (Integer, Integer)
fibPair n
  | n == 1    = (1,1)
  | otherwise = fibStep (fibPair (n - 1))

fastFib n = fst (fibPair n)
    \end{pygments}
  \end{exampleblock}
\end{frame}

\begin{frame}[fragile]
  \frametitle{Nested Patterns}

  \begin{itemize}
    \item patterns can be nested
  \end{itemize}

  \begin{exampleblock}{}
    \begin{pygments}{haskell}
shift :: ((Integer, Integer), Integer)
            -> (Integer, (Integer, Integer))
shift ((x,y),z) = (x,(y,z))
    \end{pygments}
  \end{exampleblock}
\end{frame}

\begin{frame}[fragile]
  \frametitle{Wildcards}

  \begin{itemize}
    \item if binding is not needed, use wildcard: \pygment{haskell}{_}
  \end{itemize}

  \begin{exampleblock}{third component of a triple}
    \begin{pygments}{haskell}
third :: (a, b, c) -> c

-- instead of the following:
third (x,y,z) = z

-- write the following:
third (_,_,z) = z
    \end{pygments}
  \end{exampleblock}
\end{frame}

\subsection{List Patterns}

\begin{frame}[fragile]
  \frametitle{List Patterns}

  \begin{itemize}
    \item empty list:\\
      \pygment{haskell}{[]}
    \item nonempty list:\\
      \pygment{haskell}{x:xs}
    \item list with exactly one element:\\
      \pygment{haskell}{[x]}\\
      \pygment{haskell}{x:[]}
    \item list with exactly two elements:\\
      \pygment{haskell}{[x1,x2]}\\
      \pygment{haskell}{x1:x2:[]}
    \item list with at least two elements:\\
      \pygment{haskell}{x1:x2:xs}
  \end{itemize}
\end{frame}

\begin{frame}[fragile]
  \frametitle{List Pattern Examples}

  \begin{exampleblock}{number of elements}
    \begin{pygments}{haskell}
length2 :: [a] -> Integer
length2 [] = 0
length2 (x:xs) = 1 + length2 xs
    \end{pygments}
  \end{exampleblock}
\end{frame}

\begin{frame}[fragile]
  \frametitle{List Pattern Examples}

  \begin{exampleblock}{add the first and third elements of a list}
    \begin{pygments}{haskell}
firstPlusThird :: [Integer] -> Integer
firstPlusThird [] = 0
firstPlusThird [x1] = x1
firstPlusThird [x1,_] = x1
firstPlusThird (x1:_:x3:_) = x1 + x3
    \end{pygments}
  \end{exampleblock}
\end{frame}

\subsection{Algebraic Type Patterns}

\begin{frame}[fragile]
  \frametitle{Algebraic Type Patterns}

  \begin{itemize}
    \item pattern matching to get values out of algebraic types
  \end{itemize}

  \begin{exampleblock}{}
    \begin{pygments}{haskell}
data Pioneer = Person String Integer
               deriving Show

birthYear :: Pioneer -> Integer
birthYear (Person _ y) = y

church :: Pioneer
church = Person "Alonzo Church" 1903

birthYear church    -- 1903
-- binding: y <-> 1903
    \end{pygments}
  \end{exampleblock}
\end{frame}

\begin{frame}[fragile]
  \frametitle{Algebraic Type Example}

  \begin{exampleblock}{}
    \begin{pygments}{haskell}
data Shape = Point |
             Circle Float |
             Rectangle Float Float
             deriving Show

area :: Shape -> Float
area Point = 0.0
area (Circle r) = 3.14159 * r * r
area (Rectangle h w) = h * w

hole :: Shape
hole = Circle 3.0

area hole    -- 28.274311
-- second pattern, binding: r <-> 3.0
    \end{pygments}
  \end{exampleblock}
\end{frame}

\subsection{Case Expressions}

\begin{frame}[fragile]
  \frametitle{Pattern Matching}

  \begin{itemize}
    \item patterns can also be handled in case expressions
    \item the result is the expression for the first matched pattern
  \end{itemize}

  \begin{block}{}
    \begin{pygments}{haskell}
case v of
  p1 -> e1
  p2 -> e2
  ...
  pn -> en
  _ -> e
    \end{pygments}
  \end{block}
\end{frame}

\begin{frame}[fragile]
  \frametitle{Case Expression Examples}

  \begin{example}[number of days in a month]
    \begin{pygments}{haskell}
daysInMonth m y =
    case m of
      Apr -> 30
      Jun -> 30
      Sep -> 30
      Nov -> 30
      Feb -> if y `mod` 4 == 0 then 29 else 28
      _ -> 31

daysInMonth Jan 2014 -- 31
daysInMonth Feb 2014 -- 28
daysInMonth Feb 2016 -- 29
    \end{pygments}
  \end{example}
\end{frame}

\begin{frame}[fragile]
  \frametitle{Case Expression Example}

  \begin{exampleblock}{playing cards}
    \begin{pygments}{haskell}
data Suit = Club | Diamond | Heart | Spade
            deriving Show

data Rank = Jack | Queen | King | Ace | Number Integer
            deriving Show

type Card = (Suit, Rank)
    \end{pygments}
  \end{exampleblock}
\end{frame}

\begin{frame}[fragile]
  \frametitle{Case Expression Example}

  \begin{exampleblock}{color of a card}
    \begin{pygments}{haskell}
data Color = Red | Black
             deriving Show

cardColor :: Card -> Color
cardColor card =
    case card of
      (Club,_) -> Black
      (Diamond,_) -> Red
      (Heart,_) -> Red
      (Spade,_) -> Black
    \end{pygments}
  \end{exampleblock}

  \pause
  \begin{itemize}
    \item exercise: write a function that will return the value of a card
  \end{itemize}
\end{frame}

\begin{frame}<beamer>[fragile]
  \frametitle{Algebraic Type Example}

  \begin{exampleblock}{value of a card}
    \begin{pygments}{haskell}
cardValue :: Card -> Integer
cardValue card =
    case card of
      (_,Ace) -> 11
      (_,King) -> 10
      (_,Queen) -> 10
      (_,Jack) -> 10
      (_,Number n) -> n
    \end{pygments}
  \end{exampleblock}
\end{frame}

\section*{References}

\begin{frame}
  \frametitle{References}

  \begin{block}{Required Reading: Thompson}
    \begin{itemize}
      \item Chapter 5: \alert{Data types, tuples and lists}
    \end{itemize}
  \end{block}
\end{frame}

\end{document}
