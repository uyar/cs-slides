% Copyright (c) 2001-2011
%       H. Turgut Uyar <uyar@itu.edu.tr>
%       Ayşegül Gençata Yayımlı <gencata@itu.edu.tr>
%       Emre Harmancı <harmanci@itu.edu.tr>
%
% Bu notlar "Creative Commons Attribution-NonCommercial-ShareAlike License" ile
% lisanslanmıştır. Yazarının açıkça belirtilmesi koşuluyla ve ticari olmayan
% amaçlarla kullanılabilir ve dağıtılabilir. Bu notlardan yola çıkılarak
% oluşturulacak çalışmaların da aynı lisansa bağlı olmaları gerekir.
%
% Lisans ile ilgili ayrıntılı bilgi almak için şu sayfaya başvurabilirsiniz:
% http://creativecommons.org/licenses/by-nc-sa/3.0/

\documentclass[dvipsnames]{beamer}

\usepackage{ae}
\usepackage[T1]{fontenc}
\usepackage[utf8]{inputenc}
\setbeamertemplate{navigation symbols}{}

\mode<presentation>
{
  \usetheme{Rochester}
  \setbeamercovered{transparent}
}

\title{Discrete Mathematics}
\subtitle{Algebraic Structures}

\author{H. Turgut Uyar \and Ayşegül Gençata Yayımlı \and Emre Harmancı}
\date{2001-2011}

\AtBeginSubsection[]
{
  \begin{frame}<beamer>
    \frametitle{Topics}
    \tableofcontents[currentsection,currentsubsection]
  \end{frame}
}

\pgfdeclareimage[width=2cm]{license}{../../license}

\pgfdeclareimage{hasse2418}{hasse2418}
\pgfdeclareimage{enumeration}{enumeration}
\pgfdeclareimage{hasse36}{hasse36}
\pgfdeclareimage{lattice}{lattice}
\pgfdeclareimage{distributive1}{distributive1}
\pgfdeclareimage{distributive2}{distributive2}

\begin{document}

\begin{frame}
  \titlepage
\end{frame}

\begin{frame}
  \frametitle{License}

  \pgfuseimage{license}\hfill
  \copyright 2001-2010 T. Uyar, A. Yayımlı, E. Harmancı

  \vfill
  \begin{tiny}
    You are free:
    \begin{itemize}
      \item to Share — to copy, distribute and transmit the work
      \item to Remix — to adapt the work
    \end{itemize}

    Under the following conditions:
    \begin{itemize}
      \item Attribution — You must attribute the work in the manner specified by
        the author or licensor (but not in any way that suggests that they
        endorse you or your use of the work).

      \item Noncommercial — You may not use this work for commercial purposes.

      \item Share Alike — If you alter, transform, or build upon this work, you
        may distribute the resulting work only under the same or similar license
        to this one.
    \end{itemize}
  \end{tiny}

  \vfill
  Legal code (the full license):\\
  \url{http://creativecommons.org/licenses/by-nc-sa/3.0/}
\end{frame}

\begin{frame}
  \frametitle{Topics}
  \tableofcontents
\end{frame}

\section{Algebraic Structures}

\subsection{Introduction}

\begin{frame}
  \frametitle{Algebraic Structure}

  \begin{definition}
    \alert{algebraic structure}:
    \begin{itemize}
      \item carrier
      \item operations
      \item constants
    \end{itemize}
  \end{definition}

  \pause
  \begin{itemize}
    \item \emph{signature}: $<$carrier, operations, constants$>$
  \end{itemize}
\end{frame}

\begin{frame}
  \frametitle{Operation}

  \begin{itemize}
    \item binary operation:\\
      $\circ: S \times S \rightarrow T$

    \item unary operation:\\
      $\Delta: S \rightarrow T$

    \pause
    \medskip
    \item every operation is a function

    \pause
    \medskip
    \item \alert{closed}: $T \subseteq S$
  \end{itemize}
\end{frame}

\begin{frame}
  \frametitle{Closed Operation Examples}

  \begin{example}
    \begin{itemize}
      \item subtraction is closed on $\mathbb{Z}$

      \pause
      \item subtraction is not closed on $\mathbb{Z^+}$
    \end{itemize}
  \end{example}
\end{frame}

\begin{frame}
  \frametitle{Binary Operation Properties}

  \begin{definition}
    \alert{commutativity}:\\
    $\forall a,b \in S~a \circ b = b \circ a$
  \end{definition}

  \pause
  \begin{definition}
    \alert{associativity}:\\
    $\forall a,b,c \in S~(a \circ b) \circ c = a \circ (b \circ c)$
  \end{definition}
\end{frame}

\begin{frame}
  \frametitle{Binary Operation Example}

  \begin{example}
    $\circ: \mathbb{Z} \times \mathbb{Z} \rightarrow \mathbb{Z}$\\
    $a \circ b = a + b - 3ab$

    \pause
    \medskip
    \begin{itemize}
      \item commutative:\\
        $a \circ b = a + b - 3ab \pause
                   = b + a - 3ba \pause
                   = b \circ a$

      \pause
      \medskip
      \item associative:\\
        $\begin{array}{rcl}
          (a \circ b) \circ c & = & (a + b - 3ab) + c - 3 (a + b - 3ab) c\\ \pause
                              & = & a + b - 3ab + c - 3ac - 3bc + 9abc\\ \pause
                              & = & a + b + c - 3ab - 3ac - 3bc + 9abc\\ \pause
                              & = & a + (b + c - 3bc) - 3 a (b + c - 3bc)\\ \pause
                              & = & a \circ (b \circ c)
        \end{array}$
     \end{itemize}
 \end{example}
\end{frame}

\begin{frame}
  \frametitle{Constants}

  \begin{columns}
    \column{.5\textwidth}
    \begin{definition}
      \alert{identity}:\\
      $x \circ 1 = 1 \circ x = x$

      \pause
      \begin{itemize}
        \item left identity: $1_l \circ x = x$
        \item right identity: $x \circ 1_r = x$
      \end{itemize}
    \end{definition}

    \pause
    \column{.5\textwidth}
    \begin{definition}
      \alert{zero}:\\
      $x \circ 0 = 0 \circ x = 0$

      \pause
      \begin{itemize}
        \item left zero: $0_l \circ x = 0$
        \item right zero: $x \circ 0_r = 0$
      \end{itemize}
    \end{definition}
  \end{columns}
\end{frame}

\begin{frame}
  \frametitle{Examples of Constants}

  \begin{example}
    \begin{itemize}
      \item identity for $<\mathbb{N}, max>$ is $0$
      \item zero for $<\mathbb{N}, min>$ is $0$
    \end{itemize}
  \end{example}

  \pause
  \begin{example}
    \begin{columns}
      \column{.4\textwidth}
      \begin{tabular}{c||c|c|c}
        $\circ$ & a & b & c\\\hline\hline
              a & a & b & b\\\hline
              b & a & b & c\\\hline
              c & a & b & a
      \end{tabular}

      \column{.5\textwidth}
      \begin{itemize}
        \item $b$ is a left identity
        \item $a$ and $b$ are right zeros
      \end{itemize}
    \end{columns}
  \end{example}
\end{frame}

\begin{frame}
  \frametitle{Constants}

  \begin{columns}
  \column{.5\textwidth}
    \begin{theorem}
      $\exists 1_l \wedge \exists 1_r \Rightarrow 1_l = 1_r$
    \end{theorem}

    \pause
    \begin{proof}
      $1_l \circ 1_r \pause = 1_l \pause = 1_r$
    \end{proof}

    \pause
    \column{.5\textwidth}
    \begin{theorem}
      $\exists 0_l \wedge \exists 0_r \Rightarrow 0_l = 0_r$
    \end{theorem}

    \pause
    \begin{proof}
      $0_l \circ 0_r \pause = 0_l \pause = 0_r$
    \end{proof}
  \end{columns}
\end{frame}

\begin{frame}
  \frametitle{Inverse}

  \begin{definition}
    if $x \circ y = 1$:

    \begin{itemize}
      \item $x$ is a \emph{left inverse} of $y$
      \item $y$ is a \emph{right inverse} of $x$

      \pause
      \medskip
      \item if $x \circ y = y \circ x = 1$
        $x$ and $y$ are \alert{inverse}
    \end{itemize}
  \end{definition}
\end{frame}

\begin{frame}
  \frametitle{Inverse}

  \begin{theorem}
    if the operation $\circ$ is associative:\\
    $w \circ x = x \circ y = 1 \Rightarrow w = y$
  \end{theorem}

  \pause
  \begin{proof}
    $\begin{array}{ccl}
      w & = & w \circ 1\\ \pause
        & = & w \circ (x \circ y)\\ \pause
        & = & (w \circ x) \circ y\\ \pause
        & = & 1 \circ y\\ \pause
        & = & y
    \end{array}$
  \end{proof}
\end{frame}

\begin{frame}
  \frametitle{Algebraic Families}

  \begin{itemize}
    \item \emph{algebraic family}: signature + axioms
  \end{itemize}
\end{frame}

\begin{frame}
  \frametitle{Algebraic Family Examples}

  \begin{example}
    \begin{itemize}
      \item axioms:
      \begin{itemize}
        \item $x \circ y = y \circ x$
        \item $(x \circ y) \circ z = x \circ (y \circ z)$
        \item $x \circ 1 = x$
      \end{itemize}

      \pause
      \item structures obeying these axioms:
      \begin{itemize}
       \item $<\mathbb{Z},+,0>$
       \item $<\mathbb{Z},\cdot,1>$
       \item $<\mathcal{P}(S),\cup,\emptyset>$
      \end{itemize}
    \end{itemize}
  \end{example}
\end{frame}

\begin{frame}
  \frametitle{Subalgebra}

  \begin{definition}
    \alert{subalgebra}:\\
    let $A = <S,\circ,\Delta,k>~\wedge~A' = <S',\circ',\Delta',k'>$

    \pause
    \medskip
    \begin{itemize}
      \item $A'$ is a subalgebra of $A$ if:
      \begin{itemize}
        \item $S' \subseteq S$
        \item $\forall a,b \in S'~a \circ' b = a \circ b \in S'$
        \item $\forall a \in S'~\Delta' a = \Delta a \in S'$
        \item $k' = k$
      \end{itemize}
    \end{itemize}
  \end{definition}
\end{frame}

\begin{frame}
  \frametitle{Subalgebra Example}

  \begin{example}
    $<\mathbb{Z},+,0>$
    is a subalgebra of $<\mathbb{R},+,0>$
  \end{example}
\end{frame}

\subsection{Groups}

\begin{frame}
  \frametitle{Semigroups}

  \begin{definition}
    \alert{semigroup}: $<S,\circ>$
    \begin{itemize}
      \item $\forall a,b,c \in S~(a \circ b) \circ c = a \circ (b \circ c)$
    \end{itemize}
  \end{definition}
\end{frame}

\begin{frame}
  \frametitle{Semigroup Examples}

  \begin{example}
    $<\Sigma^+,\&>$

    \begin{itemize}
      \item $\Sigma$: alphabet, $\Sigma^+$: strings of length at least 1
      \item $\&$: string concatenation
    \end{itemize}
  \end{example}
\end{frame}

\begin{frame}
  \frametitle{Monoids}

  \begin{definition}
    \alert{monoid}: $<S,\circ,1>$

    \begin{itemize}
      \item $\forall a,b,c \in S~(a \circ b) \circ c = a \circ (b \circ c)$
      \item $\forall a \in S~a \circ 1 = 1 \circ a = a$
    \end{itemize}
  \end{definition}
\end{frame}

\begin{frame}
  \frametitle{Monoid Examples}

  \begin{example}
    $<\Sigma^*,\&,\epsilon>$

    \begin{itemize}
      \item $\Sigma$: alphabet, $\Sigma^*$: strings of any length
      \item $\&$: string concatenation
      \item $\epsilon$: empty string
    \end{itemize}
  \end{example}
\end{frame}

\begin{frame}
  \frametitle{Groups}

  \begin{definition}
    \alert{group}: $<S,\circ,1>$

    \begin{itemize}
      \item $\forall a,b,c \in S~(a \circ b) \circ c = a \circ (b \circ c)$
      \item $\forall a \in S~a \circ 1 = 1 \circ a = a$
      \item $\forall a \in S~\exists a^{-1} \in S~$
        $a \circ a^{-1} = a^{-1} \circ a = 1$

      \pause
      \medskip
      \item \emph{Abelian group}: $\forall a,b \in S~a \circ b = b \circ a$
    \end{itemize}
  \end{definition}
\end{frame}

\begin{frame}
  \frametitle{Group Examples}

  \begin{example}
    $<\mathbb{Z},+,0>$

    \begin{itemize}
      \item $x^{-1} = -x$
    \end{itemize}
  \end{example}

  \pause
  \begin{example}
    $<\mathbb{Q}-\{0\},\cdot,1>$

    \begin{itemize}
      \item $x^{-1} = \frac{1}{x}$
    \end{itemize}
  \end{example}
\end{frame}

\begin{frame}
  \frametitle{Permutations}

  \begin{itemize}
    \item permutation: a bijective function on a set

    \medskip
    $\left(
      \begin{array}{cccc}
         a_1   &  a_2   & \dots &  a_n\\
        p(a_1) & p(a_2) & \dots & p(a_n)
      \end{array}
    \right)$

    \medskip
    \item $n!$ permutations can be defined in a set of $n$ elements
  \end{itemize}
\end{frame}

\begin{frame}
  \frametitle{Permutation Examples}

  \begin{example}
    $A = \{1,2,3\}$

    \medskip
    $\begin{array}{cc}
      p_1 = \left(
        \begin{array}{ccc}
          1 & 2 & 3\\
          1 & 2 & 3
        \end{array}
      \right) &
      p_2 = \left(
        \begin{array}{ccc}
          1 & 2 & 3\\
          1 & 3 & 2
        \end{array}
      \right)\medskip\\
      p_3 = \left(
        \begin{array}{ccc}
          1 & 2 & 3\\
          2 & 1 & 3
        \end{array}
      \right) &
      p_4 = \left(
        \begin{array}{ccc}
          1 & 2 & 3\\
          2 & 3 & 1
        \end{array}
      \right)\medskip\\
      p_5 = \left(
        \begin{array}{ccc}
          1 & 2 & 3\\
          3 & 1 & 2
        \end{array}
      \right) &
      p_6 = \left(
        \begin{array}{ccc}
          1 & 2 & 3\\
          3 & 2 & 1
        \end{array}
      \right)
    \end{array}$
  \end{example}
\end{frame}

\begin{frame}
  \frametitle{Cyclic Permutation}

  \begin{itemize}
    \item \emph{cyclic permutation}:
    \begin{itemize}
      \item a subset of elements form a cycle
      \item the remaining elements do not change
    \end{itemize}

    \medskip
    $(a_i, a_j, a_k) = \left(
      \begin{array}{ccccccccc}
        \dots & a_i & \dots & a_n & \dots & a_j & \dots & a_k & \dots\\
        \dots & a_j & \dots & a_n & \dots & a_k & \dots & a_i & \dots
      \end{array}
    \right)$

    \pause
    \bigskip
    \item \emph{transposition}: a cyclic permutation of length 2
  \end{itemize}
\end{frame}

\begin{frame}
  \frametitle{Cyclic Permutation Examples}

  \begin{example}
    $A = \{1,2,3,4,5\}$

    \medskip
    $(1,3,5) = \left(
      \begin{array}{ccccc}
        1 & 2 & 3 & 4 & 5\\
        3 & 2 & 5 & 4 & 1
      \end{array}
    \right)$
  \end{example}
\end{frame}

\begin{frame}
  \frametitle{Permutation Composition}

  \begin{itemize}
    \item permutation composition is not commutative
  \end{itemize}

  \pause
  \begin{example}
    $A = \{1,2,3,4,5\}$

    \medskip
    \small{$\begin{array}{lll}
      (4,1,3,5) \diamond (5,2,3) & = & \left(
        \begin{array}{ccccc}
          1 & 2 & 3 & 4 & 5\\
          3 & 2 & 5 & 1 & 4
        \end{array}
      \right) \diamond \left(
        \begin{array}{ccccc}
          1 & 2 & 3 & 4 & 5\\
          1 & 3 & 5 & 4 & 2
        \end{array}
      \right)\smallskip\pause\\
      & = & \left(
        \begin{array}{ccccc}
          1 & 2 & 3 & 4 & 5\\
          5 & 3 & 2 & 1 & 4
        \end{array}
      \right)\medskip\pause\\
      (5,2,3) \diamond (4,1,3,5) & = & \left(
        \begin{array}{ccccc}
          1 & 2 & 3 & 4 & 5\\
          1 & 3 & 5 & 4 & 2
        \end{array}
      \right) \diamond \left(
        \begin{array}{ccccc}
          1 & 2 & 3 & 4 & 5\\
          3 & 2 & 5 & 1 & 4
      \end{array}
    \right)\smallskip\pause\\
    & = & \left(
      \begin{array}{ccccc}
        1 & 2 & 3 & 4 & 5\\
        3 & 5 & 4 & 1 & 2
      \end{array}
    \right)
    \end{array}$}
  \end{example}
\end{frame}

\begin{frame}
  \frametitle{Cyclic Permutation Composition}

  \begin{itemize}
    \item all permutations that are not cyclic can be written\\
      as a composition of disjoint cyclic permutations
  \end{itemize}

  \pause
  \begin{example}
    $A = \{1,2,3,4,5,6,7,8\}$

    \medskip
    $\left(
      \begin{array}{cccccccc}
        1 & 2 & 3 & 4 & 5 & 6 & 7 & 8\\
        3 & 4 & 6 & 5 & 2 & 1 & 8 & 7
      \end{array}
    \right) = (1,3,6) \diamond (2,4,5) \diamond (7,8)$
  \end{example}
\end{frame}

\begin{frame}
  \frametitle{Transposition Composition}

  \begin{itemize}
    \item all cyclic permutations can be written\\
      as a composition of transpositions
  \end{itemize}

  \pause
  \begin{example}
    $A = \{1,2,3,4,5\}$

    \medskip
    $(1,2,3,4,5) = (1,2) \diamond (1,3) \diamond (1,4) \diamond (1,5)$
  \end{example}
\end{frame}

\begin{frame}
  \frametitle{Group Examples}

  \begin{example}[composition of permutations]
    \begin{tiny}
    \[
      \begin{array}{c|cccccccccccccccccccccccc}
        A & 1_{A}  & p_{1}  & p_{2}  & p_{3}  & p_{4}  & p_{5}
          & p_{6}  & p_{7}  & p_{8}  & p_{9}  & p_{10} & p_{11}\\\hline
        1 &   1    &   1    &   1    &   1    &   1    &  1
          &   2    &   2    &   2    &   2    &   2    &  2\\
        2 &   2    &   2    &   3    &   3    &   4    &  4
          &   1    &   1    &   3    &   3    &   4    &  4\\
        3 &   3    &   4    &   2    &   4    &   2    &  3
          &   3    &   4    &   1    &   4    &   1    &  3\\
        4 &   4    &   3    &   4    &   2    &   3    &  2
          &   4    &   3    &   4    &   1    &   3    &  1
      \end{array}
    \]

    \[
      \begin{array}{c|cccccccccccccccccccccccc}
        A & p_{12} & p_{13} & p_{14} & p_{15} & p_{16} & p_{17}
          & p_{18} & p_{19} & p_{20} & p_{21} & p_{22} & p_{23}\\\hline
        1 &   3    &   3    &   3    &   3    &   3    &  3
          &   4    &   4    &   4    &   4    &   4    &  4\\
        2 &   1    &   1    &   2    &   2    &   4    &  4
          &   1    &   1    &   2    &   2    &   3    &  3\\
        3 &   2    &   4    &   1    &   4    &   1    &  2
          &   2    &   3    &   1    &   3    &   1    &  2\\
        4 &   4    &   2    &   4    &   1    &   2    &  1
          &   3    &   2    &   3    &   1    &   2    &  1
      \end{array}
    \]
    \end{tiny}

    \pause
    \medskip
    $p_8 \diamond p_{12}=1_A \Rightarrow p_{12} = p_8^{-1}$\\
    $p_{14} \diamond p_{14}=1_A \Rightarrow p_{14} = p_{14}^{-1}$\\

    \pause
    \bigskip
    $<\{1_A,p_1,\dots,p_{23}\},\diamond,\Delta^{-1},1_A>$
  \end{example}
\end{frame}

\begin{frame}
  \frametitle{Subgroup Example}

  \begin{example}[composition of permutations]
    \[
      \begin{array}{c||c|c|c|c|c|c}
        \diamond & 1_{A}  & p_{2}  & p_{6}  & p_{8}  & p_{12} & p_{14}\\\hline\hline
        1_{A}    & 1_{A}  & p_{2}  & p_{6}  & p_{8}  & p_{12} & p_{14}\\\hline
        p_{2}    & p_{2}  & 1_{A}  & p_{8}  & p_{6}  & p_{14} & p_{12}\\\hline
        p_{6}    & p_{6}  & p_{12} & 1_{A}  & p_{14} & p_{2}  & p_{8}\\\hline
        p_{8}    & p_{8}  & p_{14} & p_{2}  & p_{12} & 1_{A}  & p_{6}\\\hline
        p_{12}   & p_{12} & p_{6}  & p_{14} & 1_{A}  & p_{8}  & p_{2}\\\hline
        p_{14}   & p_{14} & p_{8}  & p_{12} & p_{2}  & p_{6}  & 1_{A}
      \end{array}
    \]
  \end{example}
\end{frame}

\begin{frame}
  \frametitle{Left and Right Cancellation}

  \begin{theorem}
    $a \circ c = b \circ c \Rightarrow a = b$

    $c \circ a = c \circ b \Rightarrow a = b$
  \end{theorem}

  \pause
  \begin{proof}
    $\begin{array}{crcl}
                  & a \circ c                & = & b \circ c\\ \pause
      \Rightarrow & (a \circ c) \circ c^{-1} & = & (b \circ c) \circ c^{-1}\\ \pause
      \Rightarrow & a \circ (c \circ c^{-1}) & = & b \circ (c \circ c^{-1})\\ \pause
      \Rightarrow & a \circ 1                & = & b \circ 1\\ \pause
      \Rightarrow & a                        & = & b
    \end{array}$

  \end{proof}
\end{frame}

\begin{frame}
  \frametitle{Basic Theorem of Groups}

  \begin{theorem}
    The unique solution of the equation $a \circ x = b$ is: $x = a^{-1} \circ b$.
  \end{theorem}

  \pause
  \begin{proof}
    $\begin{array}{crcl}
                & a \circ c                & = & b\\\pause
    \Rightarrow & a^{-1} \circ (a \circ c) & = & a^{-1} \circ b\\\pause
    \Rightarrow & 1 \circ c                & = & a^{-1} \circ b\\\pause
    \Rightarrow & c                        & = & a^{-1} \circ b
    \end{array}$
  \end{proof}
\end{frame}

\subsection{Rings}

\begin{frame}
  \frametitle{Ring}

  \begin{definition}
    \alert{ring}: $<S,+,\cdot,0>$

    \begin{itemize}
      \item $\forall a,b,c \in S~(a + b) + c = a + (b + c)$
      \item $\forall a \in S~a + 0 = 0 + a = a$
      \item $\forall a \in S~\exists (-a) \in S~a + (-a) = (-a) + a = 0$
      \item $\forall a,b \in S~a + b = b + a$

      \pause
      \item $\forall a,b,c \in S~(a \cdot b) \cdot c = a \cdot (b \cdot c)$

      \pause
      \item $\forall a,b,c \in S$
      \begin{itemize}
        \item $a \cdot (b + c) = a \cdot b + a \cdot c$
        \item $(b + c) \cdot a = b \cdot a + c \cdot a$
      \end{itemize}
    \end{itemize}
  \end{definition}
\end{frame}

\begin{frame}
  \frametitle{Field}

  \begin{definition}
    \alert{field}: $<S,+,\cdot,0,1>$
    \begin{itemize}
      \item all properties of a ring

      \pause
      \item $\forall a,b \in S~a \cdot b = b \cdot a$
      \item $\forall a \in S~a \cdot 1 = 1 \cdot a = a$
      \item $\forall a \in S~\exists a^{-1} \in S~a \cdot a^{-1} = a^{-1} \cdot a = 1$
    \end{itemize}
  \end{definition}
\end{frame}

\subsection*{References}

\begin{frame}
  \frametitle{References}

  \begin{block}{Grimaldi}
    \begin{itemize}
      \item Chapter 5: Relations and Functions
      \begin{itemize}
        \item 5.4. \alert{Special Functions}
      \end{itemize}

      \item Chapter 16: Groups, Coding Theory,\\
        and Polya's Method of Enumeration
      \begin{itemize}
        \item 16.1. \alert{Definitions, Examples, and Elementary Properties}
      \end{itemize}

      \item Chapter 14: Rings and Modular Arithmetic
      \begin{itemize}
        \item 14.1. \alert{The Ring Structure: Definition and Examples}
      \end{itemize}
    \end{itemize}
  \end{block}
\end{frame}

\section{Lattices}

\subsection{Partially Ordered Sets}

\begin{frame}
  \frametitle{Partially Ordered Set}

  \begin{definition}
    \alert{partial order relation}:
    \begin{itemize}
      \item reflexive
      \item anti-symmetric
      \item transitive
    \end{itemize}
  \end{definition}

  \pause
  \begin{itemize}
    \item \emph{partially ordered set (poset)}:\\
      a set with a partial order relation defined on its elements
  \end{itemize}
\end{frame}

\begin{frame}
  \frametitle{Poset Examples}

  \begin{example}[set of sets, $\subseteq$]
    \begin{itemize}
      \item $A \subseteq A$
      \item $A \subseteq B \wedge B \subseteq A \Rightarrow A = B$
      \item $A \subseteq B \wedge B \subseteq C \Rightarrow A \subseteq C$
    \end{itemize}
  \end{example}
\end{frame}

\begin{frame}
  \frametitle{Poset Examples}

  \begin{example}[$\mathbb{Z}$, $\leq$]
    \begin{itemize}
      \item $x \leq x$
      \item $x \leq y \wedge y \leq x \Rightarrow x = y$
      \item $x \leq y \wedge y \leq z \Rightarrow x \leq z$
    \end{itemize}
  \end{example}
\end{frame}

\begin{frame}
  \frametitle{Poset Examples}

  \begin{example}[$\mathbb{Z}^+$, $|$]
    \begin{itemize}
      \item $x | x$
      \item $x | y \wedge y | x \Rightarrow x = y$
      \item $x | y \wedge y | z \Rightarrow x | z$
    \end{itemize}
  \end{example}
\end{frame}

\begin{frame}
  \frametitle{Comparability}

  \begin{itemize}
    \item $a \preceq b$: \emph{a precedes b}

    \medskip
    \item $a \preceq b \vee b \preceq a$: \emph{a and b are comparable}

    \pause
    \bigskip
    \item \alert{total order} (linear order, chain):\\
      all elements are comparable with each other
  \end{itemize}
\end{frame}

\begin{frame}
  \frametitle{Comparability Examples}

  \begin{example}
    \begin{itemize}
      \item $\mathbb{Z}^+,|$: $3$ and $5$ are not comparable

      \pause
      \medskip
      \item $\mathbb{Z},\leq$: total order
    \end{itemize}
  \end{example}
\end{frame}

\begin{frame}
  \frametitle{Hasse Diagrams}

  \begin{itemize}
    \item $a \ll b$: \emph{a immediately precedes b}\\
      $\neg \exists x~ a \preceq x \preceq b$

    \pause
    \medskip
    \item Hasse diagram:
    \begin{itemize}
      \item draw a line between $a$ and $b$ if $a \ll b$
      \item preceding element is below
    \end{itemize}
  \end{itemize}
\end{frame}

\begin{frame}
  \frametitle{Hasse Diagram Examples}

  \begin{example}
    \begin{columns}
      \column{.45\textwidth}
      $\{1,2,3,4,6,8,9,12,18,24\}$\\
      $|$ relation

      \column{.45\textwidth}
      \begin{center}
        \pgfuseimage{hasse2418}
      \end{center}
    \end{columns}
  \end{example}
\end{frame}

\begin{frame}
  \frametitle{Consistent Enumeration}

  \begin{definition}
    \alert{consistent enumeration}:

    $f: S \rightarrow \mathbb{N}$\\
    $a \preceq b \Rightarrow f(a) \leq f(b)$
  \end{definition}

  \begin{itemize}
    \item there can be more than one consistent enumeration
  \end{itemize}
\end{frame}

\begin{frame}
  \frametitle{Consistent Enumeration}

  \begin{example}
    \begin{center}
      \pgfuseimage{enumeration}
    \end{center}

    \begin{itemize}
     \item $f(d)=1, f(e)=2, f(b)=3, f(c)=4, f(a)=5$

     \pause
     \item $f(e)=1, f(d)=2, f(c)=3, f(b)=4, f(a)=5$
    \end{itemize}
  \end{example}
\end{frame}

\begin{frame}
  \frametitle{Upper Bound - Lower Bound}

  \begin{definition}
    \alert{upper bound}: $max$\\
    $\forall x \in S~max \preceq x \Rightarrow x = max$
  \end{definition}

  \pause
  \begin{definition}
    \alert{lower bound}: $min$\\
    $\forall x \in S~x \preceq min \Rightarrow x = min$
  \end{definition}
\end{frame}

\begin{frame}
  \frametitle{Upper Bound - Lower Bound Examples}

  \begin{example}
    \begin{columns}
      \column{.45\textwidth}
      \begin{center}
        \pgfuseimage{hasse2418}
      \end{center}

      \column{.45\textwidth}
      $max:~18,24$\\
      $min:~1$
    \end{columns}
  \end{example}
\end{frame}

\begin{frame}
  \frametitle{Supremum}

  \begin{definition}
    $A \subseteq S$

    \medskip
    $M$ is an \alert{upper bound} of $A$:\\
    $\forall x \in A~x \preceq M$
  \end{definition}

  \pause
  \begin{definition}
    $M(A)$: set of upper bounds of $A$

    \medskip
    $sup(A)$ is the \alert{supremum} of $A$:\\
    $\forall M \in M(A)~sup(A) \preceq M$
  \end{definition}
\end{frame}

\begin{frame}
  \frametitle{Infimum}

  \begin{definition}
    $A \subseteq S$

    \medskip
    $m$ is a \alert{lower bound} of $A$:\\
    $\forall x \in S~m \preceq x$
  \end{definition}

  \pause
  \begin{definition}
    $m(A)$: set of lower bound of $A$

    \medskip
    $inf(A)$ is the \alert{infimum} of $A$:\\
    $\forall m \in m(A)~m \preceq inf(A)$
  \end{definition}
\end{frame}

\begin{frame}
  \frametitle{Bound Example}

  \begin{example}[factors of 36]
    \begin{columns}
      \column{.45\textwidth}
      \begin{center}
        \pgfuseimage{hasse36}
      \end{center}

      \column{.45\textwidth}
      inf = gcd\\
      sup = lcm
    \end{columns}
  \end{example}
\end{frame}

\subsection{Lattices}

\begin{frame}
  \frametitle{Lattice}

  \begin{definition}
    \alert{lattice}: $<L,\wedge,\vee>$\\
    $\wedge$: meet, $\vee$: join

    \pause
    \begin{itemize}
      \item $a \wedge b = b \wedge a$\\
        $a \vee b = b \vee a$
      \item$(a \wedge b) \wedge c = a \wedge (b \wedge c)$\\
        $(a \vee b) \vee c = a \vee (b \vee c)$
      \item $a \wedge (a \vee b) = a$\\
        $a \vee (a \wedge b) = a$
    \end{itemize}
  \end{definition}
\end{frame}

\begin{frame}
  \frametitle{Poset - Lattice Relationship}

  \begin{itemize}
    \item If $P$ is a poset, then $<P,inf,sup>$ is a lattice.
    \begin{itemize}
      \item $a \wedge b = inf(a,b)$
      \item $a \vee b = sup(a,b)$
    \end{itemize}

    \pause
    \medskip
    \item Every lattice is a poset where these definitions hold.
  \end{itemize}
\end{frame}

\begin{frame}
  \frametitle{Duality}

  \begin{definition}
    \alert{dual}:\\
    $\wedge$ instead of $\vee$, $\vee$ instead of $\wedge$
  \end{definition}

  \pause
  \begin{theorem}[Duality Theorem]
    Every theorem has a dual theorem in lattices.
  \end{theorem}
\end{frame}

\begin{frame}
  \frametitle{Lattice Theorems}

  \begin{theorem}
    $a \wedge a = a$
  \end{theorem}

  \pause
  \begin{proof}
    $a \wedge a = a \wedge (a \vee (a \wedge b))$
  \end{proof}
\end{frame}

\begin{frame}
  \frametitle{Lattice Theorems}

  \begin{theorem}
    $a \preceq b \Leftrightarrow a \wedge b$
    $ = a \Leftrightarrow a \vee b = b$
  \end{theorem}
\end{frame}

\begin{frame}
  \frametitle{Lattice Examples}

  \begin{example}
    \begin{columns}
      \column{.3\textwidth}
      \[
	<\mathcal{P}\{a,b,c\},\cap,\cup>
      \]
      $\subseteq$ relation

      \column{.6\textwidth}
      \begin{center}
        \pgfuseimage{lattice}
      \end{center}
    \end{columns}
  \end{example}
\end{frame}

\begin{frame}
  \frametitle{Bounded Lattice}

  \begin{columns}[t]
    \column{.5\textwidth}
    \begin{definition}
      lower bound of lattice $L$: $0$\\
      $\forall x \in L~0 \preceq x$
    \end{definition}

    \pause
    \column{.5\textwidth}
    \begin{definition}
      upper bound of lattice $L$: $I$\\
      $\forall x \in L~x \preceq I$
    \end{definition}
  \end{columns}

  \pause
  \bigskip
  \begin{theorem}
    Every finite lattice is bounded.
  \end{theorem}
\end{frame}

\begin{frame}
  \frametitle{Distributive Lattice}

  \begin{itemize}
    \item \emph{distributive lattice}:
    \begin{itemize}
      \item $\forall a,b,c \in L~a \wedge (b \vee c) = (a \wedge b) \vee (a \wedge c)$
      \item $\forall a,b,c \in L~a \vee (b \wedge c) = (a \vee b) \wedge (a \vee c)$
    \end{itemize}
  \end{itemize}
\end{frame}

\begin{frame}
  \frametitle{Counterexamples}

  \begin{example}
    \begin{columns}
      \column{.45\textwidth}
      \begin{center}
        \pgfuseimage{distributive1}
      \end{center}

      \pause
      \column{.45\textwidth}
      $a \vee (b \wedge c)$ \pause $= a \vee 0$ \pause $= a$

      \pause
      $(a \vee b) \wedge (a \vee c)$ \pause $= I \wedge c$ \pause $= c$
    \end{columns}
  \end{example}
\end{frame}

\begin{frame}
  \frametitle{Counterexamples}

  \begin{example}
    \begin{columns}
      \column{.45\textwidth}
      \begin{center}
        \pgfuseimage{distributive2}
      \end{center}

      \pause
      \column{.45\textwidth}
      $a \vee (b \wedge c)$ \pause $= a \vee 0$ \pause $= a$

      \pause
      $(a \vee b) \wedge (a \vee c)$ \pause $= I \wedge I$ \pause $= I$
    \end{columns}
  \end{example}
\end{frame}

\begin{frame}
  \frametitle{Distributive Lattice}

  \begin{theorem}
    A lattice is nondistributive if and only if it has a sublattice\\
    isomorphic to any of these two structures.
  \end{theorem}
\end{frame}

\begin{frame}
  \frametitle{Join Irreducible}

  \begin{definition}
    \alert{join irreducible element}:\\
    $a = x \vee y \Rightarrow a = x \vee a = y$
  \end{definition}

  \pause
  \medskip
  \begin{itemize}
    \item \emph{atom}: a join irreducible element\\
      which immediately succeeds the minimum
  \end{itemize}
\end{frame}

\begin{frame}
  \frametitle{Join Irreducible Example}

  \begin{example}[Divisibility Relation]
    \begin{itemize}
      \item prime numbers and 1 are join irreducible

      \pause
      \medskip
      \item 1 is the minimum, the prime numbers are the atoms
    \end{itemize}
  \end{example}
\end{frame}

\begin{frame}
  \frametitle{Join Irreducible}

  \begin{theorem}
    Every element in a lattice can be written\\
    as the join of join irreducible elements.
  \end{theorem}
\end{frame}

\begin{frame}
  \frametitle{Complement}

  \begin{definition}
    $a$ and $x$ are \alert{complements}:\\
    $a \wedge x = 0$ and $a \vee x = I$
  \end{definition}
\end{frame}

\begin{frame}
  \frametitle{Complemented Lattice}

  \begin{theorem}
    In a bounded, distributive lattice\\
    the complement is unique, if it exists.
  \end{theorem}

  \pause
  \begin{proof}
    $a \wedge x = 0, a \vee x = I$, $a \wedge y = 0, a \vee y = I$

    \pause
    \medskip
    \begin{eqnarray*}
  & x & = x \vee 0 \pause = x \vee (a \wedge y) \pause = (x \vee a) \wedge (x \vee y) \pause = I \wedge (x \vee y)\\
  &   & \pause = x \vee y \pause = y \vee x \pause = I \wedge (y \vee x)\\
  &   & \pause = (y \vee a) \wedge (y \vee x) \pause = y \vee (a \wedge x) \pause = y \vee 0 \pause = y\\
    \end{eqnarray*}
  \end{proof}
\end{frame}

\subsection{Boolean Algebra}

\begin{frame}
  \frametitle{Boolean Algebra}

  \begin{definition}
    \alert{Boolean algebra}:\\
    $<B,+,\cdot,\overline{x},1,0>$

    \pause
    \[\begin{array}{ll}
      a + b = b + a &
      a \cdot b = b \cdot a\\ \pause
      (a + b) + c = a + (b + c) &
      (a \cdot b) \cdot c = a \cdot (b \cdot c)\\ \pause
      a + 0 = a &
      a \cdot 1 = a\\ \pause
      a + \overline{a} = 1 &
      a \cdot \overline{a} = 0
    \end{array}\]
  \end{definition}
\end{frame}

\begin{frame}
  \frametitle{Boolean Algebra - Lattice Relationship}

  \begin{definition}
    A Boolean algebra is a finite, distributive, complemented lattice.
  \end{definition}
\end{frame}

\begin{frame}
  \frametitle{Duality}

  \begin{definition}
    \alert{dual}:\\
    $+$ instead of $\cdot$, $\cdot$ instead of $+$\\
    0 instead of 1, 1 instead of 0
  \end{definition}

  \pause
  \begin{example}
    $(1 + a) \cdot (b + 0) = b$

    dual of the theorem:

    $(0 \cdot a) + (b \cdot 1) = b$
  \end{example}
\end{frame}

\begin{frame}
  \frametitle{Boolean Algebra Examples}

  \begin{example}
    $B = \{0,1\}, + = \vee, \cdot = \wedge$
  \end{example}

  \pause
  \begin{example}
    $B = \{$ factors of $70$ $\}$, $+ = lcm, \cdot = gcd$
  \end{example}
\end{frame}

\begin{frame}
  \frametitle{Boolean Algebra Theorems}

    \[\begin{array}{ll}
      a + a = a &
      a \cdot a = a\\ \pause
      a + 1 = 1 &
      a \cdot 0 = 0\\ \pause
      a + (a \cdot b) = a &
      a \cdot (a + b) = a\\ \pause
      (a + b) + c = a + (b + c) &
      (a \cdot b) \cdot c = a \cdot (b \cdot c)\bigskip\\ \pause
      \overline{\overline{a}} = a & \\ \pause
      \overline{a + b} = \overline{a} \cdot \overline{b} &
      \overline{a \cdot b} = \overline{a} + \overline{b}
    \end{array}\]
\end{frame}

\subsection*{References}

\begin{frame}
  \frametitle{References}

  \begin{block}{To read: Grimaldi}
    \begin{itemize}
      \item Chapter 7: Relations: The Second Time Around
      \begin{itemize}
        \item 7.3. \alert{Partial Orders: Hasse Diagrams}
      \end{itemize}

      \item Chapter 15: Boolean Algebra and Switching Functions
      \begin{itemize}
        \item 15.4. \alert{The Structure of a Boolean Algebra}
      \end{itemize}
    \end{itemize}
  \end{block}
\end{frame}

\end{document}
