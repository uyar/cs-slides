% Copyright (c) 2001-2012
%       H. Turgut Uyar <uyar@itu.edu.tr>
%       Ayşegül Gençata Yayımlı <gencata@itu.edu.tr>
%       Emre Harmancı <harmanci@itu.edu.tr>
%
% These notes are licensed using the
% "Creative Commons Attribution-NonCommercial-ShareAlike License".
% You are free to copy, distribute and transmit the work, and to adapt the work
% as long as you attribute the authors, do not use it for commercial purposes,
% and any derivative work is under the same or a similar license.
%
% Read the full legal code at:
% http://creativecommons.org/licenses/by-nc-sa/3.0/

\documentclass[dvipsnames]{beamer}

\usepackage{ae}
\usepackage[T1]{fontenc}
\usepackage[utf8]{inputenc}
\usepackage{textcomp}
\setbeamertemplate{navigation symbols}{}

\mode<presentation>
{
  \usetheme{Rochester}
  \setbeamercovered{transparent}
}

\title{Discrete Mathematics}
\subtitle{Propositions}

\author{H. Turgut Uyar \and Ayşegül Gençata Yayımlı \and Emre Harmancı}
\date{2001-2012}

\AtBeginSubsection[]
{
  \begin{frame}<beamer>
    \frametitle{Topics}
    \tableofcontents[currentsection,currentsubsection]
  \end{frame}
}

%\beamerdefaultoverlayspecification{<+->}

\pgfdeclareimage[width=2cm]{license}{../../license}

\begin{document}

\begin{frame}
  \titlepage
\end{frame}

\begin{frame}
  \frametitle{Licence}

  \pgfuseimage{license}\hfill
  \copyright 2001-2012 T. Uyar, A. Yayımlı, E. Harmancı

  \vfill
  \begin{tiny}
    You are free:
    \begin{itemize}
      \item to Share — to copy, distribute and transmit the work
      \item to Remix — to adapt the work
    \end{itemize}

    Under the following conditions:
    \begin{itemize}
      \item Attribution — You must attribute the work in the manner specified by
        the author or licensor (but not in any way that suggests that they
        endorse you or your use of the work).

      \item Noncommercial — You may not use this work for commercial purposes.

      \item Share Alike — If you alter, transform, or build upon this work, you
        may distribute the resulting work only under the same or similar license
        to this one.
    \end{itemize}
  \end{tiny}

  \vfill
  Legal code (the full license):\\
  \url{http://creativecommons.org/licenses/by-nc-sa/3.0/}
\end{frame}

\begin{frame}
  \frametitle{Topics}
  \tableofcontents
\end{frame}

\section{Propositions}

\subsection{Introduction}

\begin{frame}
  \frametitle{Proposition}

  \begin{definition}
    \alert{proposition} (or \alert{statement}):\\
      a declarative sentence that is either true or false
  \end{definition}

  \pause
  \begin{itemize}
    \item \alert{law of the excluded middle}:\\
      a proposition cannot be partially true or partially false
  \end{itemize}

  \pause
  \begin{itemize}
    \item \alert{law of contradiction}:\\
      a proposition cannot be both true and false
  \end{itemize}
\end{frame}

\begin{frame}
  \frametitle{Proposition Examples}

  \begin{columns}[t]
    \column{.55\textwidth}
    \begin{example}[proposition]
      \begin{itemize}
        \item The Moon revolves around\\
          the Earth.
        \item Elephants can fly.
        \item $3+8=11$
      \end{itemize}
    \end{example}

    \pause
    \column{.45\textwidth}
    \begin{example}[not proposition]
      \begin{itemize}
        \item What time is it?
        \item Ali, throw the ball!
        \item $x<43$
      \end{itemize}
    \end{example}
  \end{columns}
\end{frame}

\begin{frame}
  \frametitle{Proposition Variable}

  \begin{definition}
    \alert{proposition variable}:\\
      a name that represents the proposition

    \begin{itemize}
      \item can take on the values \emph {True} ($T$) or \emph{False} ($F$)
    \end{itemize}
  \end{definition}

  \pause
  \begin{example}
    \begin{itemize}
      \item $p_1$: The Moon revolves around the Earth. ($T$)
      \item $p_2$: Elephants can fly. ($F$)
      \item $p_3$: $3+8=11$ ($T$)
    \end{itemize}
  \end{example}
\end{frame}

\subsection{Compound Propositions}

\begin{frame}
  \frametitle{Compound Propositions}

  \begin{itemize}
    \item \alert{compound propositions} are obtained by
    \begin{itemize}
      \item negating a proposition, or
      \item combining two or more propositions using \alert{logical connectives}
    \end{itemize}
    \item \alert{primitive propositions} can not be decomposed\\
      into smaller units
  \end{itemize}

  \pause
  \begin{itemize}
    \item \alert{truth table}:\\
      a table that lists the truth value of the compound proposition\\
      for all possible values of its primitive propositions
  \end{itemize}
\end{frame}

\begin{frame}
  \frametitle{Negation (NOT)}

  \begin{columns}
    \column{.3\textwidth}
    \begin{table}
      \caption{$\neg p$}
      \begin{tabular}{|c||c|}\hline
        $p$ & $\neg p$\\\hline\hline
        $T$ & $F$\\\hline
        $F$ & $T$\\\hline
      \end{tabular}
    \end{table}

    \pause
    \column{.7\textwidth}
    \begin{example}
      \begin{itemize}
        \item $\neg p_1$: The Moon does not revolve around the Earth.\\
          $\neg T$: \emph{False}
        \item $\neg p_2$: Elephants cannot fly.\\
          $\neg F$: \emph{True}
      \end{itemize}
    \end{example}
  \end{columns}
\end{frame}

\begin{frame}
  \frametitle{Conjunction (AND)}

  \begin{columns}
    \column{.38\textwidth}
    \begin{table}
      \caption{$p \wedge q$}
      \begin{tabular}{|c|c||c|}\hline
        $p$ & $q$ & $p \wedge q$\\\hline\hline
        $T$ & $T$ & $T$\\\hline
        $T$ & $F$ & $F$\\\hline
        $F$ & $T$ & $F$\\\hline
        $F$ & $F$ & $F$\\\hline
      \end{tabular}
    \end{table}

    \pause
    \column{.62\textwidth}
    \begin{example}
      \begin{itemize}
        \item $p_1 \wedge p_2$: The Moon revolves around the Earth and elephants
          can fly.\\
          $T \wedge F$: \emph{False}
      \end{itemize}
    \end{example}
  \end{columns}
\end{frame}

\begin{frame}
  \frametitle{Disjunction (OR)}

  \begin{columns}
    \column{.4\textwidth}
    \begin{table}
      \caption{$p \vee q$}
      \begin{tabular}{|c|c||c|}\hline
        $p$ & $q$ & $p \vee q$\\\hline\hline
        $T$ & $T$ & $T$\\\hline
        $T$ & $F$ & $T$\\\hline
        $F$ & $T$ & $T$\\\hline
        $F$ & $F$ & $F$\\\hline
      \end{tabular}
    \end{table}

    \pause
    \column{.6\textwidth}
    \begin{example}
      \begin{itemize}
        \item $p_1 \vee p_2$: The Moon revolves around the Earth or elephants
          can fly.\\
          $T \vee F$: \emph{True}
      \end{itemize}
    \end{example}
  \end{columns}
\end{frame}

\begin{frame}
  \frametitle{Exclusive Disjunction (XOR)}

  \begin{columns}
    \column{.35\textwidth}
    \begin{table}
      \caption{$p \veebar q$}
      \begin{tabular}{|c|c||c|}\hline
        $p$ & $q$ & $p \veebar q$\\\hline\hline
        $T$ & $T$ & $F$\\\hline
        $T$ & $F$ & $T$\\\hline
        $F$ & $T$ & $T$\\\hline
        $F$ & $F$ & $F$\\\hline
      \end{tabular}
    \end{table}

    \pause
    \column{.65\textwidth}
    \begin{example}
      \begin{itemize}
        \item $p_1 \veebar p_2$: Either the Moon revolves around the Earth or
          elephants can fly.\\
          $T \veebar F$: \emph{True}
      \end{itemize}
    \end{example}
  \end{columns}
\end{frame}

\begin{frame}
  \frametitle{Implication (IF)}

  \begin{columns}
    \column{.4\textwidth}
    \begin{table}
      \caption{$p \rightarrow q$}
      \begin{tabular}{|c|c||c|}\hline
        $p$ & $q$ & $p \rightarrow q$\\\hline\hline
        $T$ & $F$ & $F$\\\hline
        $F$ & $T$ & $T$\\\hline
        $F$ & $F$ & $T$\\\hline
        $T$ & $T$ & $T$\\\hline
      \end{tabular}
    \end{table}

    \pause
    \column{.6\textwidth}
    \begin{itemize}
      \item $p$: \alert{hypothesis}
      \item $q$: \alert{conclusion}

      \item read:
      \begin{itemize}
        \item if $p$ then $q$
        \item $p$ is sufficient for $q$
        \item $q$ is necessary for $p$
      \end{itemize}

      \pause
      \item $\neg p \vee q$
    \end{itemize}
  \end{columns}
\end{frame}

\begin{frame}
  \frametitle{Implication Examples}

  \begin{example}
    \begin{itemize}
      \item $p_4$: $3<8$, $p_5$: $3<14$, $p_6$: $3<2$
      \item $p_7$: The Sun revolves around the Earth.
    \end{itemize}

    \pause
    \begin{columns}[t]
      \column{.48\textwidth}
        \begin{itemize}
          \item $p_4 \rightarrow p_5$: If 3 is less than 8,\\
            then 3 is less than 14.\\
            $T \rightarrow T$: \emph{True}
          \pause
          \item $p_4 \rightarrow p_6$: If 3 is less than 8,\\
            then 3 is less than 2.\\
            $T \rightarrow F$: \emph{False}
        \end{itemize}

      \pause
      \column{.5\textwidth}
        \begin{itemize}
          \item $p_2 \rightarrow p_1$: If elephants can fly then the Moon
            revolves around the Earth.\\
            $F \rightarrow T$: \emph{True}
          \pause
          \item $p_2 \rightarrow p_7$: If elephants can fly then the Sun
            revolves around the Earth.
            \\
            $F \rightarrow F$: \emph{True}
        \end{itemize}
    \end{columns}
  \end{example}
\end{frame}

\begin{frame}
  \frametitle{Implication Examples}

  \begin{example}
    \begin{itemize}
      \item "If I weigh over 70 kg, then I will exercise."
    \end{itemize}

    \pause
    \begin{columns}
      \column{.6\textwidth}
      \begin{itemize}
        \item $p$: I weigh over 70 kg.
        \item $q$: I exercise.
      \end{itemize}

      \pause
      \begin{itemize}
        \item when is this claim false?
      \end{itemize}

      \column{.4\textwidth}
      \begin{table}
        \caption{$p \rightarrow q$}
        \begin{tabular}{|c|c||c|}\hline
          $p$ & $q$ & $p \rightarrow q$\\\hline\hline
          $T$ & $T$ & $T$\\\hline
          $T$ & $F$ & $F$\\\hline
          $F$ & $T$ & $T$\\\hline
          $F$ & $F$ & $T$\\\hline
        \end{tabular}
      \end{table}
    \end{columns}
  \end{example}
\end{frame}

\begin{frame}
  \frametitle{Biconditional (IFF)}

  \begin{columns}
    \column{.4\textwidth}
    \begin{table}
      \caption{$p \leftrightarrow q$}
      \begin{tabular}{|c|c||c|}\hline
        $p$ & $q$ & $p \leftrightarrow q$\\\hline\hline
        $T$ & $F$ & $F$\\\hline
        $T$ & $T$ & $T$\\\hline
        $F$ & $T$ & $F$\\\hline
        $F$ & $F$ & $T$\\\hline
      \end{tabular}
    \end{table}

    \pause
    \column{.6\textwidth}
    \begin{itemize}
      \item read:
      \begin{itemize}
        \item $p$ if and only if $q$
        \item $p$ is necessary and sufficient for $q$
      \end{itemize}

      \pause
      \item $(p \rightarrow q) \wedge (q \rightarrow p)$
      \item $\neg (p \veebar q)$
    \end{itemize}
  \end{columns}
\end{frame}

\begin{frame}
  \frametitle{Example}

  \begin{example}
    \begin{itemize}
      \item The parent tells the child:\\
        "If you do your homework, you can play computer games."

      \pause
      \medskip
      \item $s$: The child does her homework.
      \item $t$: The child plays computer games.

      \pause
      \medskip
      \item which one does the parent mean?
      \begin{itemize}
        \item $s \rightarrow t$
        \item $\neg s \rightarrow \neg t$
        \item $s \leftrightarrow t$
      \end{itemize}
    \end{itemize}
  \end{example}
\end{frame}

\subsection{Well-Formed Formulas}

\begin{frame}
  \frametitle{Well-Formed Formula}

  \begin{block}{syntax}
    \begin{itemize}
      \item which rules will be used to form compound propositions?
      \item formula that obeys these rules: \alert{well-formed formula} (WFF)
    \end{itemize}
  \end{block}

  \pause
  \begin{block}{semantics}
    \begin{itemize}
      \item \emph{interpretation}: calculating the value\\
        of a compound proposition\\
        by assigning values to its primitive propositions
      \item truth table: all interpretations of a proposition
    \end{itemize}
  \end{block}
\end{frame}

\begin{frame}
  \frametitle{Formula Examples}

  \begin{example}[not well-formed]
    \begin{itemize}
      \item $\vee p$
      \item $p \wedge \neg$
      \item $p \neg \wedge q$
    \end{itemize}
  \end{example}
\end{frame}

\begin{frame}
  \frametitle{Precedence}

  \begin{enumerate}
    \item $\neg$
    \item $\wedge$
    \item $\vee$
    \item $\rightarrow$
    \item $\leftrightarrow$
  \end{enumerate}

  \begin{itemize}
    \item parentheses are used to change precedence
  \end{itemize}
\end{frame}

\begin{frame}
  \frametitle{Precedence Examples}

  \begin{example}
    \begin{itemize}
      \item $s$: Phyllis goes out for a walk.
      \item $t$: The Moon is out.
      \item $u$: It is snowing.
    \end{itemize}

    \medskip
    \begin{itemize}
      \item what do the following WFFs mean?

      \pause
      \begin{itemize}
        \item $t \wedge \neg u \rightarrow s$
        \pause
        \item $t \rightarrow (\neg u \rightarrow s)$
        \pause
        \item $\neg (s \leftrightarrow (u \vee t))$
        \pause
        \item $\neg s \leftrightarrow u \vee t$
      \end{itemize}
    \end{itemize}
  \end{example}
\end{frame}

% \begin{frame}
%   \frametitle{Example: The Fork in the Road}
%
%   \bigskip
%   \hyperlink{formattr}{\beamergotobutton{skip example}}
%
%   \begin{example}
%     \begin{itemize}
%       \item from the book "Mathematical Puzzles and Diversions"\\
%         by Martin Gardner
%
%       \pause
%       \medskip
%       \item on a remote island, a logician arrives at a fork in the road
%       \item two tribes living on the island: truth-tellers (tt) and liars (l)
%       \begin{itemize}
%         \item truth-tellers always tell the truth, liars always lie
%       \end{itemize}
%
%       \item there is a native bystander at the fork
%       \begin{itemize}
%         \item no way of telling which tribe the native is from
%       \end{itemize}
%
%       \pause
%       \item to figure out which road leads to the village,\\
%         what question does the logician ask?
%       \begin{itemize}
%         \item only one question with a yes/no answer
%       \end{itemize}
%     \end{itemize}
%   \end{example}
% \end{frame}
%
% \begin{frame}
%   \frametitle{Example: The Fork in the Road}
%
%   \begin{example}
%     \begin{itemize}
%       \item $p$: The road leads to the village.
%       \item $q$: The native is a truth-teller.
%     \end{itemize}
%   \end{example}
% \end{frame}
%
% \begin{frame}
%   \frametitle{Example: The Fork in the Road}
%
%   \begin{example}
%     \begin{itemize}
%       \item \emph{If I were to ask you if this road leads to the village,\\
%         would you say "Yes"?}
%
%       \pause
%       \medskip
%       \begin{table}
%         \begin{tabular}{|l|l||l|}\hline
%           $p$ & $q$ & Answer\\\hline\hline
%           $T$ & $T$ & Yes\\\hline
%           $T$ & $F$ & Yes\\\hline
%           $F$ & $T$ & No\\\hline
%           $F$ & $F$ & No\\\hline
%         \end{tabular}
%       \end{table}
%
%       \pause
%       \medskip
%       \item "Yes" if the road leads to the village, "No" if not
%     \end{itemize}
%   \end{example}
% \end{frame}
%
% \begin{frame}
%   \frametitle{Example: The Fork in the Road}
%
%   \begin{example}
%     \begin{itemize}
%       \item \emph{What is the result of $p \veebar \neg q$?}
%
%       \pause
%       \medskip
%       \begin{table}
%         \begin{tabular}{|c|c|c|c||c|c|}\hline
%               &     &          &                    & tt  & l\\
%           $p$ & $q$ & $\neg q$ & $p \veebar \neg q$ & $A$ & $\neg A$\\
%               &     &          & ($A$)              &     &\\\hline\hline
%           $T$ & $T$ & $F$ & $T$ & $T$ & $-$\\\hline
%           $T$ & $F$ & $T$ & $F$ & $-$ & $T$\\\hline
%           $F$ & $T$ & $F$ & $F$ & $F$ & $-$\\\hline
%           $F$ & $F$ & $T$ & $T$ & $-$ & $F$\\\hline
%         \end{tabular}
%       \end{table}
%
%       \pause
%       \medskip
%       \item "$T$" if the road leads to the village, "$F$" if not
%     \end{itemize}
%   \end{example}
% \end{frame}
%
% \begin{frame}
%   \frametitle{Example: The Fork in the Road}
%
%   \begin{example}
%     \begin{itemize}
%       \item an honest liar also lies to himself
%
%       \pause
%       \medskip
%       \item simple liar (sl): compute, negate
%       \item honest liar (hl): negate, compute, negate
%     \end{itemize}
%   \end{example}
% \end{frame}
%
% \begin{frame}
%   \frametitle{Example: The Fork in the Road}
%
%   \begin{example}
%     \begin{itemize}
%       \item \emph{What is the result of $p \rightarrow \neg q$?}
%
%       \pause
%       \medskip
%       \begin{table}
%         \begin{tabular}{|c|c|c|c|c|c||c|c|c|}\hline
%               &     &          &          &                        &
%               & tt  & sl       & hl\\
%           $p$ & $q$ & $\neg p$ & $\neg q$ & $p \rightarrow \neg q$ & $\neg p \rightarrow q$
%               & $A$ & $\neg A$ & $\neg B$\\
%               &     &          &          & ($A$)                  & ($B$)
%               &     &          &\\\hline\hline
%           $T$ & $T$ & $F$ & $F$ & $F$ & $T$ & $F$ & $-$ & $-$\\\hline
%           $T$ & $F$ & $F$ & $T$ & $T$ & $T$ & $-$ & $F$ & $F$\\\hline
%           $F$ & $T$ & $T$ & $F$ & $T$ & $T$ & $T$ & $-$ & $-$\\\hline
%           $F$ & $F$ & $T$ & $T$ & $T$ & $F$ & $-$ & $F$ & $T$\\\hline
%         \end{tabular}
%       \end{table}
%
%       \pause
%       \medskip
%       \item honest liar: "$F$" if the road leads to the village, "$T$" if not
%       \item fails in the simple liar case
%     \end{itemize}
%   \end{example}
% \end{frame}
%
% \begin{frame}
%   \frametitle{Example: The Fork in the Road}
%
%   \begin{example}
%     \begin{itemize}
%       \item \emph{What is the result of $p \leftrightarrow \neg q$?}
%
%       \pause
%       \medskip
%       \begin{table}
%         \begin{tabular}{|c|c|c|c|c|c||c|c|c|}\hline
%             &     &          &          &                            &
%             & tt  & sl       & hl\\
%         $p$ & $q$ & $\neg p$ & $\neg q$ & $p \leftrightarrow \neg q$ & $\neg p \leftrightarrow q$
%             & $A$ & $\neg A$ & $\neg B$\\
%             &     &          &        & ($A$)                        & ($B$)
%             &     &          &\\\hline\hline
%         $T$ & $T$ & $F$ & $F$ & $F$ & $F$ & $F$ & $-$ & $-$\\\hline
%         $T$ & $F$ & $F$ & $T$ & $T$ & $T$ & $-$ & $F$ & $F$\\\hline
%         $F$ & $T$ & $T$ & $F$ & $T$ & $T$ & $T$ & $-$ & $-$\\\hline
%         $F$ & $F$ & $T$ & $T$ & $F$ & $F$ & $-$ & $T$ & $T$\\\hline
%         \end{tabular}
%       \end{table}
%
%       \pause
%       \medskip
%       \item "$F$" if the road leads to the village, "$T$" if not
%     \end{itemize}
%   \end{example}
% \end{frame}

\begin{frame}
  \frametitle{Formula Attributes}

  \begin{enumerate}
    \item \emph{tautology}: True for all interpretations
    \item \emph{contradiction}: False for all interpretations
    \item \emph{valid}: True for some interpretations
  \end{enumerate}
\end{frame}

\begin{frame}
  \frametitle{Tautology Example}

  \begin{example}
    \begin{table}
      \caption{$p \wedge (p \rightarrow q) \rightarrow q$}
      \begin{tabular}{|c|c|c|c||c|}\hline
        $p$ & $q$ & $p \rightarrow q$ & $p \wedge A$ & $B \rightarrow q$\\
            &     & $(A)$             & $(B)$        &\\\hline\hline
        $T$ & $T$ & $T$ & $T$ & $T$\\\hline
        $T$ & $F$ & $F$ & $F$ & $T$\\\hline
        $F$ & $T$ & $T$ & $F$ & $T$\\\hline
        $F$ & $F$ & $T$ & $F$ & $T$\\\hline
      \end{tabular}
    \end{table}
  \end{example}
\end{frame}

\begin{frame}
  \frametitle{Contradiction Example}

  \begin{example}
    \begin{table}
      \caption{$p \wedge (\neg p \wedge q)$}
      \begin{tabular}{|c|c|c|c||c|}\hline
        $p$ & $q$ & $\neg p$ & $\neg p \wedge q$ & $p \wedge A$\\
            &     &          & ($A$)             &\\\hline\hline
        $T$ & $T$ & $F$ & $F$ & $F$\\\hline
        $T$ & $F$ & $F$ & $F$ & $F$\\\hline
        $F$ & $T$ & $T$ & $T$ & $F$\\\hline
        $F$ & $F$ & $T$ & $F$ & $F$\\\hline
      \end{tabular}
    \end{table}
  \end{example}
\end{frame}

\subsection{Metalanguage}

\begin{frame}
  \frametitle{Metalanguage}

  \begin{definition}
    \alert{target language}:\\
      the language being worked on
  \end{definition}

  \pause
  \begin{definition}
    \alert{metalanguage}:\\
      the language used when talking about the properties\\
      of the target language
  \end{definition}

  \pause
  \begin{itemize}
    \item validity, contradiction and tautology are defined\\
      in the metalanguage
  \end{itemize}
\end{frame}

\begin{frame}
  \frametitle{Metalanguage Examples}

  \begin{example}[for a Turk who is learning English]
    \begin{itemize}
      \item target language: English
      \item metalanguage: Turkish
    \end{itemize}
  \end{example}

  \pause
  \begin{example}[in an introductory programming course]
    \begin{itemize}
      \item target language: C, Python, Java, \ldots
      \item metalanguage: English, Turkish, \ldots
    \end{itemize}
  \end{example}
\end{frame}

\begin{frame}
  \frametitle{Metalogic}

  \begin{itemize}
    \item $P_1,P_2,\dots,P_n \vdash Q$\\
      There is a proof which infers the conclusion $Q$\\
      from the assumptions $P_1,P_2,\dots,P_n$.

    \pause
    \medskip
    \item $P_1,P_2,\dots,P_n \vDash Q$\\
      $Q$ must be true if $P_1,P_2,\dots,P_n$ are all true.
  \end{itemize}
\end{frame}

\begin{frame}
  \frametitle{Formal Systems}

  \begin{definition}
    \alert{consistent}: for all well-formed formulas $P$ and $Q$\\
      if $P \vdash Q$ then $P \vDash Q$

    \begin{itemize}
      \item each provable proposition is actually true
    \end{itemize}
  \end{definition}

  \pause
  \begin{definition}
    \alert{complete}: for all well-formed formulas $P$ and $Q$\\
      if $P \vDash Q$ then $P \vdash Q$
    \begin{itemize}
      \item every true proposition can be proven
    \end{itemize}
  \end{definition}
\end{frame}

\begin{frame}
  \frametitle{Gödel's Theorem}

  \begin{itemize}
    \item Propositional logic is consistent and complete.
  \end{itemize}

  \pause
  \begin{block}{Gödel's Theorem}
    \begin{itemize}
      \item Any logical system that is powerful enough\\
        to express ordinary arithmetic\\
        must be either inconsistent or incomplete.
    \end{itemize}
  \end{block}
\end{frame}

\section{Propositional Calculus}

\subsection{Introduction}

\begin{frame}
  \frametitle{Approaches in Propositional Calculus}

  \begin{enumerate}
    \item semantic approach: \emph{truth tables}
    \begin{itemize}
      \item too complicated when the number of primitive statements grow
    \end{itemize}

    \pause
    \item syntactic approach: \emph{rules of inference}
    \begin{itemize}
      \item obtaining new propositions from existing propositions\\
        using logical implications
    \end{itemize}

    \pause
    \item axiomatic approach: \emph{Boolean algebra}
    \begin{itemize}
      \item substituting equivalent formulas in equations
    \end{itemize}
  \end{enumerate}
\end{frame}

\begin{frame}
  \frametitle{Truth Table Example}

  \begin{example}[$p \rightarrow q$]
    \begin{center}
      \begin{tabular}{|c|c||c|c|c|c|}\hline
        $p$ & $q$ & $p \rightarrow q$ & $\neg q \rightarrow \neg p$
            & $q \rightarrow p$ & $\neg p \rightarrow \neg q$\\\hline\hline
        $T$ & $T$ & $T$ & $T$ & $T$ & $T$\\\hline
        $T$ & $F$ & $F$ & $F$ & $T$ & $T$\\\hline
        $F$ & $T$ & $T$ & $T$ & $F$ & $F$\\\hline
        $F$ & $F$ & $T$ & $T$ & $T$ & $T$\\\hline
      \end{tabular}
    \end{center}

    \pause
    \begin{itemize}
      \item \emph{contrapositive}: $\neg q \rightarrow \neg p$

      \pause
      \item \emph{converse}: $q \rightarrow p$

      \pause
      \item \emph{inverse}: $\neg p \rightarrow \neg q$
    \end{itemize}
  \end{example}
\end{frame}

\subsection{Laws of Logic}

\begin{frame}
  \frametitle{Logical Equivalence}

  \begin{definition}
    if $P \leftrightarrow Q$ is a tautology, then $P$ and $Q$ are
    \alert{logically equivalent}:\\
    $P \Leftrightarrow Q$
  \end{definition}
\end{frame}

\begin{frame}
  \frametitle{Logical Equivalence Example}

  \begin{example}
    \begin{itemize}
      \item $\neg p \Leftrightarrow p \rightarrow F$
    \end{itemize}

    \begin{table}
      \caption{$\neg p \leftrightarrow p \rightarrow F$}
      \begin{tabular}{|c|c|c||c|}\hline
        $p$ & $\neg p$ & $p \rightarrow F$ & $\neg p \leftrightarrow A$\\
            &          & ($A$)             &\\\hline\hline
        $T$ & $F$ & $F$ & $T$\\\hline
        $F$ & $T$ & $T$ & $T$\\\hline
      \end{tabular}
    \end{table}
  \end{example}
\end{frame}

\begin{frame}
  \frametitle{Logical Equivalence Example}

  \begin{example}
    \begin{itemize}
      \item $p \rightarrow q \Leftrightarrow \neg p \vee q$
    \end{itemize}

    \begin{table}
      \caption{$(p \rightarrow q) \leftrightarrow (\neg p \vee q)$}
      \begin{tabular}{|c|c|c|c|c||c|}\hline
        $p$ & $q$ & $p \rightarrow q$ & $\neg p$ & $\neg p \vee q$ & $A \leftrightarrow B$\\
            &     & ($A$)             &          & ($B$)           &\\\hline\hline
        $T$ & $T$ & $T$ & $F$ & $T$ & $T$\\\hline
        $T$ & $F$ & $F$ & $F$ & $F$ & $T$\\\hline
        $F$ & $T$ & $T$ & $T$ & $T$ & $T$\\\hline
        $F$ & $F$ & $T$ & $T$ & $T$ & $T$\\\hline
      \end{tabular}
    \end{table}
  \end{example}
\end{frame}

\begin{frame}
  \frametitle{Laws of Logic}

  \begin{tabular}{ll}
  \alert{Double Negation (DN)} &\\
    $\neg (\neg p) \Leftrightarrow p$ &\\\\
  \pause
  \alert{Commutativity (Co)} &\\
    $p \wedge q \Leftrightarrow q \wedge p$ &
    $p \vee q \Leftrightarrow q \vee p$\\\\
  \pause
  \alert{Associativity (As)} &\\
    $(p \wedge q) \wedge r \Leftrightarrow p \wedge (q \wedge r)$ &
    $(p \vee q) \vee r \Leftrightarrow p \vee (q \vee r)$\\\\
  \pause
  \alert{Idempotence (Ip)} &\\
    $p \wedge p \Leftrightarrow p$ &
    $p \vee p \Leftrightarrow p$\\\\
  \pause
  \alert{Inverse (In)} &\\
    $p \wedge \neg p \Leftrightarrow F$ &
    $p \vee \neg p \Leftrightarrow T$
  \end{tabular}
\end{frame}

\begin{frame}
  \frametitle{Laws of Logic}

  \begin{tabular}{ll}
  \alert{Identity (Id)} &\\
    $p \wedge T \Leftrightarrow p$ &
    $p \vee F \Leftrightarrow p$\\\\
  \pause
  \alert{Domination (Do)} &\\
    $p \wedge F \Leftrightarrow F$ &
    $p \vee T \Leftrightarrow T$\\\\
  \pause
  \alert{Distributivity (Di)} &\\
    $p \wedge (q \vee r) \Leftrightarrow (p \wedge q) \vee (p \wedge r)$ &
    $p \vee (q \wedge r) \Leftrightarrow (p \vee q) \wedge (p \vee r)$\\\\
  \pause
  \alert{Absorption (Ab)} &\\
    $p \wedge (p \vee q) \Leftrightarrow p$ &
    $p \vee (p \wedge q) \Leftrightarrow p$\\\\
  \pause
  \alert{DeMorgan's Laws (DM)} &\\
    $\neg (p \wedge q) \Leftrightarrow \neg p \vee \neg q$ &
    $\neg (p \vee q) \Leftrightarrow \neg p \wedge \neg q$
  \end{tabular}
\end{frame}

\begin{frame}
  \frametitle{Equivalence Example}

  \begin{example}
    \begin{eqnarray*}
                      & p \rightarrow q           &   \\
      \pause
      \Leftrightarrow & \neg p \vee q             &   \\
      \pause
      \Leftrightarrow & q \vee \neg p             & Co\\
      \pause
      \Leftrightarrow & \neg \neg q \vee \neg p   & DN\\
      \pause
      \Leftrightarrow & \neg q \rightarrow \neg p &
    \end{eqnarray*}
  \end{example}
\end{frame}

\begin{frame}
  \frametitle{Equivalence Example}

  \begin{example}
    \begin{eqnarray*}
                      & \neg (\neg ((p \vee q) \wedge r) \vee \neg q)      &   \\
      \pause
      \Leftrightarrow & \neg \neg ((p \vee q) \wedge r) \wedge \neg \neg q & DM\\
      \pause
      \Leftrightarrow & ((p \vee q) \wedge r) \wedge q                     & DN\\
      \pause
      \Leftrightarrow & (p \vee q) \wedge (r \wedge q)                     & As\\
      \pause
      \Leftrightarrow & (p \vee q) \wedge (q \wedge r)                     & Co\\
      \pause
      \Leftrightarrow & ((p \vee q) \wedge q) \wedge r                     & As\\
      \pause
      \Leftrightarrow & q \wedge r                                         & Ab
    \end{eqnarray*}
  \end{example}
\end{frame}

\begin{frame}
  \frametitle{Duality}

  \begin{definition}
    If $s$ contains no logical connectives other than $\wedge$ and $\vee$,\\
    then the \alert{dual} of $s$, denoted $s^d$, is the statement obtained
    from $s$\\
    by replacing each occurrence of\\
    $\wedge$ by $\vee$, $\vee$ by $\wedge$, $T$ by $F$, and $F$ by $T$.
  \end{definition}

  \pause
  \begin{example}[dual proposition]
    \begin{eqnarray*}
      s:   & (p \wedge \neg q) \vee (r \wedge T)\\
      s^d: & (p \vee \neg q) \wedge (r \vee F)
    \end{eqnarray*}
  \end{example}
\end{frame}

\begin{frame}
  \frametitle{Principle of Duality}

  \begin{block}{principle of duality}
    Let $s$ and $t$ be statements that contain no logical connectives\\
    other than $\wedge$ and $\vee$.\\
    If $s \Leftrightarrow t$ then $s^d \Leftrightarrow t^d$.
  \end{block}
\end{frame}

\subsection{Rules of Inference}

\begin{frame}
  \frametitle{Rules of Inference}

  \begin{definition}
    if $P \rightarrow Q$ is a tautology, then $P$ \alert{logically implies} $Q$:\\
    $P \Rightarrow Q$
  \end{definition}
\end{frame}

\begin{frame}
  \frametitle{Logical Implication Example}

  \begin{example}
    \begin{itemize}
      \item $p \wedge (p \rightarrow q) \Rightarrow q$
    \end{itemize}

    \begin{table}
      \caption{$p \wedge (p \rightarrow q) \rightarrow q$}
      \begin{tabular}{|c|c|c|c||c|}\hline
        $p$ & $q$ & $p \rightarrow q$ & $p \wedge A$ & $B \rightarrow q$\\
            &     & ($A$)             & ($B$)        &\\\hline\hline
        $T$ & $T$ & $T$ & $T$ & $T$\\\hline
        $T$ & $F$ & $F$ & $F$ & $T$\\\hline
        $F$ & $T$ & $T$ & $F$ & $T$\\\hline
        $F$ & $F$ & $T$ & $F$ & $T$\\\hline
      \end{tabular}
    \end{table}
  \end{example}
\end{frame}

\begin{frame}
  \frametitle{Inference}

  \begin{itemize}
    \item establishing the validity of an argument,\\
      starting from a set of propositions\\
      which are assumed or proven to be true
  \end{itemize}

  \pause
  \begin{block}{notation}
    \begin{columns}
      \column{.5\textwidth}
      \[
      \frac
        {
          \begin{array}{c}
            p_1\\
            p_2\\
            \dots\\
            p_n
          \end{array}
        }
        {
          \therefore q
        }
      \]

      \column{.5\textwidth}
      $p_1 \wedge p_2 \wedge \cdots \wedge p_n \Rightarrow q$
    \end{columns}
  \end{block}
\end{frame}

\begin{frame}
  \frametitle{Basic Rules}

  \begin{block}{Identity (ID)}
    \[
    \frac
      {
        \begin{array}{c}
          p
        \end{array}
      }
      {
        \therefore p
      }
    \]
  \end{block}

  \pause
  \begin{block}{Contradiction (CTR)}
  \[
  \frac
    {
      \begin{array}{c}
        F
      \end{array}
    }
    {
      \therefore p
    }
  \]
  \end{block}
\end{frame}

\begin{frame}
  \frametitle{Basic Rules}

  \begin{block}{Implication Introduction (ImpI)}
    \[
    \frac
      {
        \begin{array}{c}
          p \vdash q
        \end{array}
      }
      {
        \therefore ~ \vdash p \rightarrow q
      }
    \]
  \end{block}

  \begin{itemize}
    \item if it can be shown that $q$ is true assuming $p$ is true,\\
      then $p \rightarrow q$ is true \emph{without assuming $p$ is true}

    \pause
    \medskip
    \item $p$ is a \alert{provisional assumption} (PA)
    \item provisional assumptions have to be \alert{discharged} at some point
  \end{itemize}
\end{frame}

\begin{frame}
  \frametitle{Basic Rules}

  \begin{block}{AND Introduction (AndI)}
    \[
    \frac
      {
        \begin{array}{c}
          p\\
          q
        \end{array}
      }
      {
        \therefore p \wedge q
      }
    \]
  \end{block}

  \pause
  \begin{block}{AND Elimination (AndE)}
  \[
  \frac
    {
      \begin{array}{c}
        p \wedge q
      \end{array}
    }
    {
      \therefore p
    }
  \]
  \end{block}
\end{frame}

\begin{frame}
  \frametitle{Basic Rules}

  \begin{block}{OR Introduction (OrI)}
    \[
    \frac
      {
        \begin{array}{c}
          p
        \end{array}
      }
      {
        \therefore p \vee q
      }
    \]
  \end{block}

  \pause
  \begin{block}{OR Elimination (OrE)}
  \[
  \frac
    {
      \begin{array}{c}
        p \vee q\\
        p \vdash r\\
        q \vdash r
      \end{array}
    }
    {
      \therefore ~ \vdash r
    }
  \]
  \end{block}
\end{frame}

\begin{frame}
  \frametitle{Basic Rules}

  \begin{block}{Modus Ponens (Implication Elimination - ImpE)}
    \[
    \frac
      {
        \begin{array}{c}
          p \rightarrow q\\
          p
        \end{array}
      }
      {
        \therefore q
      }
    \]
  \end{block}

  \pause
  \begin{block}{Modus Tollens (MT)}
    \[
    \frac
      {
        \begin{array}{c}
          p \rightarrow q\\
          \neg q
        \end{array}
      }
      {
        \therefore \neg p
      }
    \]
  \end{block}
\end{frame}

\begin{frame}
  \frametitle{Modus Tollens}

  \begin{example}
    \begin{columns}
      \column{.3\textwidth}
      \[
      \frac
        {
          \begin{array}{c}
            p \rightarrow q\\
            \neg q
          \end{array}
        }
        {
          \therefore \neg p
        }
      \]

      \pause
      \column{.65\textwidth}
      \begin{eqnarray*}
        1. & p \rightarrow q           & A\\
        \pause
        2. & \neg q \rightarrow \neg p & 1\\
        \pause
        3. & \neg q                    & A\\
        \pause
        4. & \neg p                    & ImpE:2,3\\
      \end{eqnarray*}
    \end{columns}
  \end{example}
\end{frame}

\begin{frame}
  \frametitle{Modus Ponens Example}

  \begin{example}
    \begin{itemize}
      \item If Ali wins the lottery, he will buy a car.
      \item Ali has won the lottery.

      \pause
      \medskip
      \item Therefore, Ali will buy a car.
    \end{itemize}
  \end{example}
\end{frame}

\begin{frame}
  \frametitle{Modus Tollens Example}

  \begin{example}
    \begin{itemize}
      \item If Ali wins the lottery, he will buy a car.
      \item Ali did not buy a car.

      \pause
      \medskip
      \item Therefore, Ali did not win the lottery.
    \end{itemize}
  \end{example}
\end{frame}

\begin{frame}
  \frametitle{Fallacies}

  \begin{block}{fallacy of affirming the conclusion}
    \[
    \frac
      {
      \begin{array}{c}
        p \rightarrow q\\
        q
        \end{array}
      }
      {
        \therefore p
      }
    \]
  \end{block}

  \pause
  \begin{itemize}
    \item $(p \rightarrow q) \wedge q \rightarrow p$ is not a tautology:\\
      if $p=F,q=T$: $(F \rightarrow T) \wedge T \rightarrow F$
  \end{itemize}
\end{frame}

\begin{frame}
  \frametitle{Example of Affirming the Conclusion}

  \begin{example}
    \begin{itemize}
      \item If Ali wins the lottery, he will buy a car.
      \item Ali has bought a car.

      \pause
      \medskip
      \item Therefore, Ali has won the lottery.
    \end{itemize}
  \end{example}
\end{frame}

\begin{frame}
  \frametitle{Fallacies}

  \begin{block}{fallacy of denying the hypothesis}
    \[
    \frac
      {
        \begin{array}{c}
          p \rightarrow q\\
          \neg p
        \end{array}
      }
      {
        \therefore \neg q
      }
    \]
  \end{block}

  \pause
  \begin{itemize}
    \item $(p \rightarrow q) \wedge \neg p \rightarrow \neg q$ is not a tautology:\\
      if $p=F,q=T$: $(F \rightarrow T) \wedge T \rightarrow F$
  \end{itemize}
\end{frame}

\begin{frame}
  \frametitle{Example of Denying the Hyphothesis}

  \begin{example}
    \begin{itemize}
      \item If Ali wins the lottery, he will buy a car.
      \item Ali has not won the lottery.

      \pause
      \medskip
      \item Therefore, Ali will not buy a car.
    \end{itemize}
  \end{example}
\end{frame}

\begin{frame}
  \frametitle{Disjunctive Syllogism}

  \begin{columns}
    \column{.5\textwidth}
    \begin{block}{Disjunctive Syllogism (DS)}
      \[
      \frac
        {
          \begin{array}{c}
            p \vee q\\
            \neg p
          \end{array}
        }
        {
          \therefore q
        }
      \]
    \end{block}

    \pause
    \column{.5\textwidth}
    \begin{eqnarray*}
      1.   & p \vee q        & A\\
      \pause
      2.   & \neg p          & A\\
      \pause
      3.   & p \rightarrow F & 2\\
      \pause
      4a1. & p               & PA\\
      \pause
      4a2. & F               & ImpE:3,4a1\\
      \pause
      4a.  & q               & CTR:4a2\\
      \pause
      4b1. & q               & PA\\
      \pause
      4b.  & q               & ID:4b1\\
      \pause
      5.   & q               & OrE:1,4a,4b
    \end{eqnarray*}
  \end{columns}
\end{frame}

\begin{frame}
  \frametitle{Disjunctive Syllogism Example}

  \begin{example}
    \begin{itemize}
      \item Ali's wallet is either in his pocket or on his desk.
      \item Ali's wallet is not in his pocket.

      \pause
      \medskip
      \item Therefore, Ali's wallet is on his desk.
    \end{itemize}
  \end{example}
\end{frame}

\begin{frame}
  \frametitle{Hypothetical Syllogism}

  \begin{columns}
    \column{.5\textwidth}
    \begin{block}{Hypothetical Syllogism (HS)}
      \[
      \frac
        {
          \begin{array}{c}
            p \rightarrow q\\
            q \rightarrow r
          \end{array}}
        {
          \therefore p \rightarrow r
        }
      \]
    \end{block}

    \pause
    \column{.5\textwidth}
    \begin{eqnarray*}
      1. & p               & PA\\
      \pause
      2. & p \rightarrow q & A\\
      \pause
      3. & q               & ImpE:2,1\\
      \pause
      4. & q \rightarrow r & A\\
      \pause
      5. & r               & ImpE:4,3\\
      \pause
      6. & p \rightarrow r & ImpI:1,5\\
    \end{eqnarray*}
  \end{columns}
\end{frame}

\begin{frame}
  \frametitle{Hypotetical Syllogism Example}

  \begin{example}[Star Trek]
    Spock to Lieutenant Decker:
    \begin{quote}
      It would be a suicide to attack the enemy ship now.\\
      Someone who attempts suicide is not psychologically fit\\
      to command the Enterprise.\\
      Therefore, I am obliged to relieve you from duty.
    \end{quote}
  \end{example}
\end{frame}

\begin{frame}
  \frametitle{Hypotetical Syllogism Example}

  \begin{example}[Star Trek]
    \begin{itemize}
      \item $p$: Decker attacks the enemy ship.
      \item $q$: Decker attempts suicide.
      \item $r$: Decker is not psychologically fit to command the Enterprise.
      \item $s$: Spock relieves Decker from duty.
    \end{itemize}
  \end{example}
\end{frame}

\begin{frame}
  \frametitle{Hypotetical Syllogism Example}

  \begin{example}
    \begin{columns}
      \column{.3\textwidth}
      \[
      \frac
        {
          \begin{array}{c}
            p\\
            p \rightarrow q\\
            q \rightarrow r\\
            r \rightarrow s
          \end{array}
        }
        {
          \therefore s
        }
      \]

      \pause
      \column{.65\textwidth}
      \begin{eqnarray*}
        1. & p \rightarrow q & A\\
        \pause
        2. & q \rightarrow r & A\\
        \pause
        3. & p \rightarrow r & HS:1,2\\
        \pause
        4. & r \rightarrow s & A\\
        \pause
        5. & p \rightarrow s & HS:3,4\\
        \pause
        6. & p               & A\\
        \pause
        7. & s               & ImpE:5,6
      \end{eqnarray*}
    \end{columns}
  \end{example}
\end{frame}

% \begin{frame}
%   \frametitle{Dilemma}
%
%   \begin{columns}[t]
%     \column{.5\textwidth}
%     \begin{block}{Constructive Dilemma}
%       \[
%       \frac
%         {
%           \begin{array}{c}
%             p \rightarrow q\\
%             r \rightarrow s\\
%             p \vee r
%           \end{array}}
%         {
%           \therefore q \vee s
%         }
%       \]
%     \end{block}
%
%     \pause
%     \column{.5\textwidth}
%     \begin{block}{Destructive Dilemma}
%       \[
%       \frac
%         {
%           \begin{array}{c}
%             p \rightarrow q\\
%             r \rightarrow s\\
%             \neg q \vee \neg s
%           \end{array}
%           }
%           {
%             \therefore \neg p \vee \neg r
%           }
%       \]
%     \end{block}
%   \end{columns}
% \end{frame}

\begin{frame}
  \frametitle{Inference Examples}

  \begin{example}
    \begin{columns}[t]
      \column{.25\textwidth}
      \[
      \frac
        {
          \begin{array}{c}
            p \rightarrow r\\
            r \rightarrow s\\
            x \vee \neg s\\
            u \vee \neg x\\
            \neg u
          \end{array}
        }
        {
          \therefore \neg p
        }
      \]

      \pause
      \column{.3\textwidth}
      \begin{eqnarray*}
        1. & u \vee \neg x   & A\\
        \pause
        2. & \neg u          & A\\
        \pause
        3. & \neg x          & DS:1,2\\
        \pause
        4. & x \vee \neg s   & A\\
        \pause
        5. & \neg s          & DS:4,3\\
      \end{eqnarray*}

      \pause
      \column{.45\textwidth}
      \begin{eqnarray*}
        6. & r \rightarrow s & A\\
        \pause
        7. & \neg r          & MT:6,5\\
        \pause
        8. & p \rightarrow r & A\\
        \pause
        9. & \neg p          & MT:8,7\\
      \end{eqnarray*}
    \end{columns}
  \end{example}
\end{frame}

\begin{frame}
  \frametitle{Inference Examples}

  \begin{example}
    \[
    \frac
      {
        \begin{array}{c}
          (\neg p \vee \neg q) \rightarrow (r \wedge s)\\
          r \rightarrow x\\
          \neg x
        \end{array}
      }
      {
        \therefore p
      }
    \]

    \pause
    \begin{columns}[t]
      \column{.3\textwidth}
      \begin{eqnarray*}
        1. & r \rightarrow x                               & A\\
        \pause
        2. & \neg x                                        & A\\
        \pause
        3. & \neg r                                        & MT:1,2\\
        \pause
        4. & \neg r \vee \neg s                            & OrI:3\\
        \pause
        5. & \neg (r \wedge s)                             & DM:4
      \end{eqnarray*}

      \pause
      \column{.6\textwidth}
      \begin{eqnarray*}
        6. & (\neg p \vee \neg q) \rightarrow (r \wedge s) & A\\
        \pause
        7. & \neg (\neg p \vee \neg q)                     & MT:6,5\\
        \pause
        8. & p \wedge q                                    & DM:7\\
        \pause
        9. & p                                             & AndE:8
      \end{eqnarray*}
    \end{columns}
  \end{example}
\end{frame}

\begin{frame}
  \frametitle{Inference Examples}

  \begin{example}
    \begin{columns}
      \column{.3\textwidth}
      \[
      \frac
        {
          \begin{array}{c}
            p \rightarrow (q \vee r)\\
            s \rightarrow \neg r\\
            q \rightarrow \neg p\\
            p\\
            s
          \end{array}
        }
        {
          \therefore F
        }
      \]

      \pause
      \column{.6\textwidth}
      \begin{eqnarray*}
        1. & q \rightarrow \neg p     & A\\
       \pause
        2. & p                        & A\\
       \pause
        3. & \neg q                   & MT:1,2\\
       \pause
        4. & s                        & A\\
       \pause
        5. & s \rightarrow \neg r     & A\\
       \pause
        6. & \neg r                   & ImpE:5,4\\
       \pause
        7. & p \rightarrow (q \vee r) & A\\
       \pause
        8. & q \vee r                 & ImpE:7,2\\
       \pause
        9. & q                        & DS:8,6\\
       \pause
       10. & q \wedge \neg q : F      & AndI:9,3
      \end{eqnarray*}
    \end{columns}
  \end{example}
\end{frame}

\begin{frame}
  \frametitle{Inference Examples}

  \begin{example}
    If there is a chance of rain or her red headband is missing,\\
    then Lois will not mow her lawn. Whenever the temperature is\\
    over 20\textcelsius, there is no chance for rain. Today the temperature is\\
    22\textcelsius ~ and Lois is wearing her red headband. Therefore,\\
    Lois will mow her lawn.
  \end{example}
\end{frame}

\begin{frame}
  \frametitle{Inference Examples}

  \begin{example}
    \begin{itemize}
      \item $p$: There is a chance of rain.
      \item $q$: Lois' red headband is lost.
      \item $r$: Lois mows her lawn.
      \item $s$: The temperature is over 20\textcelsius.
    \end{itemize}
  \end{example}
\end{frame}

\begin{frame}
  \frametitle{Inference Examples}

  \begin{example}
    \begin{columns}
      \column{.3\textwidth}
      \[
      \frac
        {
          \begin{array}{c}
            (p \vee q) \rightarrow \neg r\\
            s \rightarrow \neg p\\
            s \wedge \neg q
          \end{array}
        }
        {
          \therefore r
        }
      \]

      \pause
      \column{.65\textwidth}
      \begin{eqnarray*}
        1. & s \wedge \neg q                & A\\
        \pause
        2. & s                              & AndE:1\\
        \pause
        3. & s \rightarrow \neg p           & A\\
        \pause
        4. & \neg p                         & ImpE:3,2\\
        \pause
        5. & \neg q                         & AndE:1\\
        \pause
        6. & \neg p \wedge \neg q           & AndI:4,5\\
        \pause
        7. & \neg (p \vee q)                & DM:6\\
        \pause
        8. & (p \vee q) \rightarrow \neg r  & A\\
        \pause
        9. & ?                              & 7,8
      \end{eqnarray*}
    \end{columns}
  \end{example}
\end{frame}

\section*{References}

\begin{frame}
  \frametitle{References}

  \begin{block}{Required Text: Grimaldi}
    \begin{itemize}
      \item Chapter 2: Fundamentals of Logic
      \begin{itemize}
        \item 2.1. \alert{Basic Connectives and Truth Tables}
        \item 2.2. \alert{Logical Equivalence: The Laws of Logic}\\
        \item 2.3. \alert{Logical Implication: Rules of Inference}
      \end{itemize}
    \end{itemize}
  \end{block}

  \begin{block}{Supplementary Text: O'Donnell, Hall, Page}
    \begin{itemize}
      \item Chapter 6: Propositional Logic
    \end{itemize}
  \end{block}
\end{frame}

\end{document}
