% Copyright (c) 2001-2015
%       H. Turgut Uyar <uyar@itu.edu.tr>
%       Ayşegül Gençata Yayımlı <gencata@itu.edu.tr>
%       Emre Harmancı <harmanci@itu.edu.tr>
%
% This work is licensed under a "Creative Commons
% Attribution-NonCommercial-ShareAlike 4.0 International License".
% For more information, please visit:
% https://creativecommons.org/licenses/by-nc-sa/4.0/

\documentclass[dvipsnames]{beamer}

\usepackage{ae}
\usepackage[T1]{fontenc}
\usepackage[utf8]{inputenc}
\usepackage{textcomp}
\setbeamertemplate{navigation symbols}{}
\setbeamersize{text margin left=2em, text margin right=2em}

\mode<presentation>
{
  \usetheme{Rochester}
  \setbeamercovered{transparent}
}

\title{Discrete Mathematics}
\subtitle{Propositions}

\author{H. Turgut Uyar \and Ayşegül Gençata Yayımlı \and Emre Harmancı}
\date{2001-2015}

\AtBeginSubsection[]
{
  \begin{frame}<beamer>
    \frametitle{Topics}
    \tableofcontents[currentsection,currentsubsection]
  \end{frame}
}

%\beamerdefaultoverlayspecification{<+->}

\pgfdeclareimage[height=1cm]{license}{../license}

\begin{document}

\begin{frame}
  \titlepage
\end{frame}

\begin{frame}
  \frametitle{License}

  \pgfuseimage{license}\hfill
  \copyright~2001-2015 T. Uyar, A. Yayımlı, E. Harmancı

  \vfill
  \begin{footnotesize}
    You are free to:
    \begin{itemize}
      \itemsep0em
      \item Share -- copy and redistribute the material in any medium or format
      \item Adapt -- remix, transform, and build upon the material
    \end{itemize}

    Under the following terms:
    \begin{itemize}
      \itemsep0em
      \item Attribution -- You must give appropriate credit, provide a link to
        the license, and indicate if changes were made.

      \item NonCommercial -- You may not use the material for commercial
        purposes.

      \item ShareAlike -- If you remix, transform, or build upon the material,
        you must distribute your contributions under the same license as the
        original.
    \end{itemize}
  \end{footnotesize}

  \begin{small}
    For more information:\\
    \url{https://creativecommons.org/licenses/by-nc-sa/4.0/}

    \smallskip
    Read the full license:\\
    \url{https://creativecommons.org/licenses/by-nc-sa/4.0/legalcode}
  \end{small}
\end{frame}

\begin{frame}
  \frametitle{Topics}
  \tableofcontents
\end{frame}

\section{Propositions}

\subsection{Introduction}

\begin{frame}
  \frametitle{Proposition}

  \begin{definition}
    \alert{proposition} (or \alert{statement}):\\
      a declarative sentence that is either true or false
  \end{definition}

  \pause
  \begin{itemize}
    \item \alert{law of the excluded middle}:\\
      a proposition cannot be partially true or partially false
  \end{itemize}

  \begin{itemize}
    \item \alert{law of contradiction}:\\
      a proposition cannot be both true and false
  \end{itemize}
\end{frame}

\begin{frame}
  \frametitle{Proposition Examples}

  \begin{columns}[t]
    \column{.5\textwidth}
    \begin{exampleblock}{propositions}
      \begin{itemize}
        \item The Moon revolves\\
          around the Earth.
        \item Elephants can fly.
        \item $3+8=11$
      \end{itemize}
    \end{exampleblock}

    \pause
    \column{.5\textwidth}
    \begin{exampleblock}{not propositions}
      \begin{itemize}
        \item What time is it?
        \item Exterminate!
        \item $x<43$
      \end{itemize}
    \end{exampleblock}
  \end{columns}
\end{frame}

\begin{frame}
  \frametitle{Propositional Variable}

  \begin{itemize}
    \item \alert{propositional variable}:\\
      a name that represents the proposition
  \end{itemize}

  \begin{exampleblock}{examples}
    \begin{itemize}
      \item $p_1$: The Moon revolves around the Earth. ($T$)
      \item $p_2$: Elephants can fly. ($F$)
      \item $p_3$: $3+8=11$ ($T$)
    \end{itemize}
  \end{exampleblock}
\end{frame}

\subsection{Logical Operators}

\begin{frame}
  \frametitle{Compound Propositions}

  \begin{itemize}
    \item compound propositions are obtained by applying
      \alert{logical operators}
  \end{itemize}

  \pause
  \bigskip
  \begin{itemize}
    \item \alert{truth table}:\\
      a table that lists the truth value of the compound proposition\\
      for all possible values of its variables
  \end{itemize}
\end{frame}

\begin{frame}
  \frametitle{Negation (NOT)}

  \begin{columns}
    \column{.3\textwidth}
    \begin{table}
      \caption{$\neg p$}
      \begin{tabular}{|c||c|}\hline
        $p$ & $\neg p$\\\hline\hline
        $T$ & $F$\\\hline
        $F$ & $T$\\\hline
      \end{tabular}
    \end{table}

    \pause
    \column{.7\textwidth}
    \begin{exampleblock}{examples}
      \begin{itemize}
        \item $\neg p_1$: The Moon does not revolve\\
          around the Earth.\\
          $\neg T : F$
        \item $\neg p_2$: Elephants cannot fly.\\
          $\neg F : T$
      \end{itemize}
    \end{exampleblock}
  \end{columns}
\end{frame}

\begin{frame}
  \frametitle{Conjunction (AND)}

  \begin{columns}
    \column{.38\textwidth}
    \begin{table}
      \caption{$p \wedge q$}
      \begin{tabular}{|c|c||c|}\hline
        $p$ & $q$ & $p \wedge q$\\\hline\hline
        $T$ & $T$ & $T$\\\hline
        $T$ & $F$ & $F$\\\hline
        $F$ & $T$ & $F$\\\hline
        $F$ & $F$ & $F$\\\hline
      \end{tabular}
    \end{table}

    \pause
    \column{.62\textwidth}
    \begin{exampleblock}{examples}
      \begin{itemize}
        \item $p_1 \wedge p_2$: The Moon revolves around the Earth and elephants
          can fly.\\
          $T \wedge F : F$
        \item $p_1 \wedge p_3$: The Moon revolves around the Earth and $3+8=11$.\\
          $T \wedge T : T$
      \end{itemize}
    \end{exampleblock}
  \end{columns}
\end{frame}

\begin{frame}
  \frametitle{Disjunction (OR)}

  \begin{columns}
    \column{.4\textwidth}
    \begin{table}
      \caption{$p \vee q$}
      \begin{tabular}{|c|c||c|}\hline
        $p$ & $q$ & $p \vee q$\\\hline\hline
        $T$ & $T$ & $T$\\\hline
        $T$ & $F$ & $T$\\\hline
        $F$ & $T$ & $T$\\\hline
        $F$ & $F$ & $F$\\\hline
      \end{tabular}
    \end{table}

    \pause
    \column{.6\textwidth}
    \begin{exampleblock}{example}
      \begin{itemize}
        \item $p_1 \vee p_2$: The Moon revolves around the Earth or elephants
          can fly.\\
          $T \vee F : T$
      \end{itemize}
    \end{exampleblock}
  \end{columns}
\end{frame}

\begin{frame}
  \frametitle{Exclusive Disjunction (XOR)}

  \begin{columns}
    \column{.35\textwidth}
    \begin{table}
      \caption{$p \veebar q$}
      \begin{tabular}{|c|c||c|}\hline
        $p$ & $q$ & $p \veebar q$\\\hline\hline
        $T$ & $T$ & $F$\\\hline
        $T$ & $F$ & $T$\\\hline
        $F$ & $T$ & $T$\\\hline
        $F$ & $F$ & $F$\\\hline
      \end{tabular}
    \end{table}

    \pause
    \column{.65\textwidth}
    \begin{exampleblock}{examples}
      \begin{itemize}
        \item $p_1 \veebar p_2$: Either the Moon revolves around the Earth or
          elephants can fly.\\
          $T \veebar F : T$
        \item $p_1 \veebar p_3$: Either the Moon revolves around the Earth or
          $3+8=11$.\\
          $T \veebar T : F$
      \end{itemize}
    \end{exampleblock}
  \end{columns}
\end{frame}

\begin{frame}
  \frametitle{Implication (IF)}

  \begin{columns}
    \column{.4\textwidth}
    \begin{table}
      \caption{$p \rightarrow q$}
      \begin{tabular}{|c|c||c|}\hline
        $p$ & $q$ & $p \rightarrow q$\\\hline\hline
        $T$ & $T$ & $T$\\\hline
        $T$ & $F$ & $F$\\\hline
        $F$ & $T$ & $T$\\\hline
        $F$ & $F$ & $T$\\\hline
      \end{tabular}
    \end{table}

    \column{.6\textwidth}
    \begin{itemize}
      \item also called \alert{conditional}
      \item if $p$ then $q$
      \smallskip
      \item $p$ is sufficient for $q$
      \item $q$ is necessary for $p$

      \medskip
      \item $p$: \alert{hypothesis}
      \item $q$: \alert{conclusion}
    \end{itemize}
  \end{columns}
\end{frame}

\begin{frame}
  \frametitle{Implication Examples}

  \begin{itemize}
    \item $p_4$: $3<8$, $p_5$: $3<14$, $p_6$: $3<2$, $p_7$: $8<6$
  \end{itemize}

  \pause
  \begin{columns}[t]
    \column{.5\textwidth}
      \begin{itemize}
        \item $p_4 \rightarrow p_5$:\\
          ~~if $3 < 8$, then $3 < 14$\\
          ~~$T \rightarrow T : T$
        \pause
        \item $p_4 \rightarrow p_6$:\\
          ~~if $3 < 8$, then $3 < 2$\\
          ~~$T \rightarrow F : F$
      \end{itemize}

    \pause
    \column{.5\textwidth}
      \begin{itemize}
        \item $p_6 \rightarrow p_4$:\\
        ~~if $3 < 2$, then $3 < 8$\\
        ~~$F \rightarrow T : T$
        \pause
        \item $p_6 \rightarrow p_7$:\\
        ~~if $3 < 2$, then $8 < 6$\\
        ~~$F \rightarrow F : T$
      \end{itemize}
  \end{columns}
\end{frame}

\begin{frame}
  \frametitle{Implication Example}

  \begin{itemize}
    \item "If I weigh over 70 kg, then I will exercise."
  \end{itemize}

  \pause
  \begin{columns}
    \column{.6\textwidth}
    \begin{itemize}
      \item $p$: I weigh over 70 kg.
      \item $q$: I exercise.
    \end{itemize}

    \pause
    \begin{itemize}
      \item when is this claim false?
    \end{itemize}

    \column{.4\textwidth}
    \begin{table}
      \caption{$p \rightarrow q$}
      \begin{tabular}{|c|c||c|}\hline
        $p$ & $q$ & $p \rightarrow q$\\\hline\hline
        $T$ & $T$ & $T$\\\hline
        $T$ & $F$ & $F$\\\hline
        $F$ & $T$ & $T$\\\hline
        $F$ & $F$ & $T$\\\hline
      \end{tabular}
    \end{table}
  \end{columns}
\end{frame}

\begin{frame}
  \frametitle{Biconditional (IFF)}

  \begin{columns}
    \column{.4\textwidth}
    \begin{table}
      \caption{$p \leftrightarrow q$}
      \begin{tabular}{|c|c||c|}\hline
        $p$ & $q$ & $p \leftrightarrow q$\\\hline\hline
        $T$ & $T$ & $T$\\\hline
        $T$ & $F$ & $F$\\\hline
        $F$ & $T$ & $F$\\\hline
        $F$ & $F$ & $T$\\\hline
      \end{tabular}
    \end{table}

    \column{.6\textwidth}
    \begin{itemize}
      \item $p$ if and only if $q$
      \smallskip
      \item $p$ is necessary and sufficient for $q$
    \end{itemize}
  \end{columns}
\end{frame}

\begin{frame}
  \frametitle{Example}

  \begin{itemize}
    \item mother tells child:\\
      "If you do your homework, you can play computer games."

    \pause
    \medskip
    \item $h$: The child does her homework.
    \item $p$: The child plays computer games.
    \item what does the mother mean?

    \pause
    \medskip
    \item $h \rightarrow p$
    \pause
    \item $\neg h \rightarrow \neg p$
    \pause
    \item $h \leftrightarrow p$
  \end{itemize}
\end{frame}

% \subsection{Well-Formed Formula}

\begin{frame}
  \frametitle{Well-Formed Formula}

  syntax
  \begin{itemize}
    \item which rules will be used to form compound propositions?
    \item a formula that obeys these rules: \alert{well-formed formula} (WFF)
  \end{itemize}

  \pause
  \bigskip
  semantics
  \begin{itemize}
    \item \emph{interpretation}: calculating the value of a
      compound proposition\\
      by assigning values to its variables
    \item truth table: all interpretations of a proposition
  \end{itemize}
\end{frame}

\begin{frame}
  \frametitle{Formula Examples}

  \begin{exampleblock}{not well-formed}
    \begin{itemize}
      \item $\vee p$
      \item $p \wedge \neg$
      \item $p~\neg \wedge q$
    \end{itemize}
  \end{exampleblock}
\end{frame}

\begin{frame}
  \frametitle{Operator Precedence}

  \begin{enumerate}
    \item $\neg$
    \item $\wedge$
    \item $\vee$
    \item $\rightarrow$
    \item $\leftrightarrow$
  \end{enumerate}

  \begin{itemize}
    \item parentheses are used to change the order of calculation
    \item implication associates from the right:\\
      $p \rightarrow q \rightarrow r$ \hspace{2em} means
      \hspace{2em} $p \rightarrow (q \rightarrow r)$
  \end{itemize}
\end{frame}

\begin{frame}
  \frametitle{Precedence Examples}

  \begin{itemize}
    \item $s$: Phyllis goes out for a walk.
    \item $t$: The Moon is out.
    \item $u$: It is snowing.
    \item what do the following WFFs mean?

    \pause
    \medskip
    \item $t \wedge \neg u \rightarrow s$
    \pause
    \item $t \rightarrow (\neg u \rightarrow s)$
    \pause
    \item $\neg (s \leftrightarrow (u \vee t))$
    \pause
    \item $\neg s \leftrightarrow u \vee t$
  \end{itemize}
\end{frame}

% \begin{frame}
%   \frametitle{Example: The Fork in the Road}
%
%   \bigskip
%   \hyperlink{formattr}{\beamergotobutton{skip example}}
%
%   \begin{example}
%     \begin{itemize}
%       \item from the book "Mathematical Puzzles and Diversions"\\
%         by Martin Gardner
%
%       \pause
%       \medskip
%       \item on a remote island, a logician arrives at a fork in the road
%       \item two tribes living on the island: truth-tellers (tt) and liars (l)
%       \begin{itemize}
%         \item truth-tellers always tell the truth, liars always lie
%       \end{itemize}
%
%       \item there is a native bystander at the fork
%       \begin{itemize}
%         \item no way of telling which tribe the native is from
%       \end{itemize}
%
%       \pause
%       \item to figure out which road leads to the village,\\
%         what question does the logician ask?
%       \begin{itemize}
%         \item only one question with a yes/no answer
%       \end{itemize}
%     \end{itemize}
%   \end{example}
% \end{frame}
%
% \begin{frame}
%   \frametitle{Example: The Fork in the Road}
%
%   \begin{example}
%     \begin{itemize}
%       \item $p$: The road leads to the village.
%       \item $q$: The native is a truth-teller.
%     \end{itemize}
%   \end{example}
% \end{frame}
%
% \begin{frame}
%   \frametitle{Example: The Fork in the Road}
%
%   \begin{example}
%     \begin{itemize}
%       \item \emph{If I were to ask you if this road leads to the village,\\
%         would you say "Yes"?}
%
%       \pause
%       \medskip
%       \begin{table}
%         \begin{tabular}{|l|l||l|}\hline
%           $p$ & $q$ & Answer\\\hline\hline
%           $T$ & $T$ & Yes\\\hline
%           $T$ & $F$ & Yes\\\hline
%           $F$ & $T$ & No\\\hline
%           $F$ & $F$ & No\\\hline
%         \end{tabular}
%       \end{table}
%
%       \pause
%       \medskip
%       \item "Yes" if the road leads to the village, "No" if not
%     \end{itemize}
%   \end{example}
% \end{frame}
%
% \begin{frame}
%   \frametitle{Example: The Fork in the Road}
%
%   \begin{example}
%     \begin{itemize}
%       \item \emph{What is the result of $p \veebar \neg q$?}
%
%       \pause
%       \medskip
%       \begin{table}
%         \begin{tabular}{|c|c|c|c||c|c|}\hline
%               &     &          &                    & tt  & l\\
%           $p$ & $q$ & $\neg q$ & $p \veebar \neg q$ & $A$ & $\neg A$\\
%               &     &          & ($A$)              &     &\\\hline\hline
%           $T$ & $T$ & $F$ & $T$ & $T$ & $-$\\\hline
%           $T$ & $F$ & $T$ & $F$ & $-$ & $T$\\\hline
%           $F$ & $T$ & $F$ & $F$ & $F$ & $-$\\\hline
%           $F$ & $F$ & $T$ & $T$ & $-$ & $F$\\\hline
%         \end{tabular}
%       \end{table}
%
%       \pause
%       \medskip
%       \item "$T$" if the road leads to the village, "$F$" if not
%     \end{itemize}
%   \end{example}
% \end{frame}
%
% \begin{frame}
%   \frametitle{Example: The Fork in the Road}
%
%   \begin{example}
%     \begin{itemize}
%       \item an honest liar also lies to himself
%
%       \pause
%       \medskip
%       \item simple liar (sl): compute, negate
%       \item honest liar (hl): negate, compute, negate
%     \end{itemize}
%   \end{example}
% \end{frame}
%
% \begin{frame}
%   \frametitle{Example: The Fork in the Road}
%
%   \begin{example}
%     \begin{itemize}
%       \item \emph{What is the result of $p \rightarrow \neg q$?}
%
%       \pause
%       \medskip
%       \begin{table}
%         \begin{tabular}{|c|c|c|c|c|c||c|c|c|}\hline
%               &     &          &          &                        &
%               & tt  & sl       & hl\\
%           $p$ & $q$ & $\neg p$ & $\neg q$ & $p \rightarrow \neg q$ & $\neg p \rightarrow q$
%               & $A$ & $\neg A$ & $\neg B$\\
%               &     &          &          & ($A$)                  & ($B$)
%               &     &          &\\\hline\hline
%           $T$ & $T$ & $F$ & $F$ & $F$ & $T$ & $F$ & $-$ & $-$\\\hline
%           $T$ & $F$ & $F$ & $T$ & $T$ & $T$ & $-$ & $F$ & $F$\\\hline
%           $F$ & $T$ & $T$ & $F$ & $T$ & $T$ & $T$ & $-$ & $-$\\\hline
%           $F$ & $F$ & $T$ & $T$ & $T$ & $F$ & $-$ & $F$ & $T$\\\hline
%         \end{tabular}
%       \end{table}
%
%       \pause
%       \medskip
%       \item honest liar: "$F$" if the road leads to the village, "$T$" if not
%       \item fails in the simple liar case
%     \end{itemize}
%   \end{example}
% \end{frame}
%
% \begin{frame}
%   \frametitle{Example: The Fork in the Road}
%
%   \begin{example}
%     \begin{itemize}
%       \item \emph{What is the result of $p \leftrightarrow \neg q$?}
%
%       \pause
%       \medskip
%       \begin{table}
%         \begin{tabular}{|c|c|c|c|c|c||c|c|c|}\hline
%             &     &          &          &                            &
%             & tt  & sl       & hl\\
%         $p$ & $q$ & $\neg p$ & $\neg q$ & $p \leftrightarrow \neg q$ & $\neg p \leftrightarrow q$
%             & $A$ & $\neg A$ & $\neg B$\\
%             &     &          &        & ($A$)                        & ($B$)
%             &     &          &\\\hline\hline
%         $T$ & $T$ & $F$ & $F$ & $F$ & $F$ & $F$ & $-$ & $-$\\\hline
%         $T$ & $F$ & $F$ & $T$ & $T$ & $T$ & $-$ & $F$ & $F$\\\hline
%         $F$ & $T$ & $T$ & $F$ & $T$ & $T$ & $T$ & $-$ & $-$\\\hline
%         $F$ & $F$ & $T$ & $T$ & $F$ & $F$ & $-$ & $T$ & $T$\\\hline
%         \end{tabular}
%       \end{table}
%
%       \pause
%       \medskip
%       \item "$F$" if the road leads to the village, "$T$" if not
%     \end{itemize}
%   \end{example}
% \end{frame}

\subsection{Metalanguage}

\begin{frame}
  \frametitle{Metalanguage}

  \begin{itemize}
    \item \alert{target language}: the language being worked on

    \pause
    \medskip
    \item \alert{metalanguage}: the language used when talking\\
      \emph{about} the properties of the target language
  \end{itemize}
\end{frame}

\begin{frame}
  \frametitle{Metalanguage Examples}

  \begin{itemize}
    \item a native Turkish speaker learning English
    \item target language: English
    \item metalanguage: Turkish
  \end{itemize}

  \pause
  \begin{itemize}
    \item a student learning programming
    \item target language: C, Python, Java, \ldots
    \item metalanguage: English, Turkish, \ldots
  \end{itemize}
\end{frame}

\begin{frame}
  \frametitle{Formula Properties}

  \begin{itemize}
    \item WFF is true for all interpretations: \alert{tautology}
    \item WFF is false for all interpretations: \alert{contradiction}

    \pause
    \bigskip
    \item these are concepts of the metalanguage
  \end{itemize}
\end{frame}

\begin{frame}
  \frametitle{Tautology Example}

  \begin{table}
    \caption{$p \wedge (p \rightarrow q) \rightarrow q$}
    \begin{tabular}{|c|c|c|c||c|}\hline
      $p$ & $q$ & $p \rightarrow q$ & $p \wedge A$ & $B \rightarrow q$\\
          &     & $(A)$             & $(B)$        &\\\hline\hline
      $T$ & $T$ & $T$ & $T$ & $T$\\\hline
      $T$ & $F$ & $F$ & $F$ & $T$\\\hline
      $F$ & $T$ & $T$ & $F$ & $T$\\\hline
      $F$ & $F$ & $T$ & $F$ & $T$\\\hline
    \end{tabular}
  \end{table}
\end{frame}

\begin{frame}
  \frametitle{Contradiction Example}

  \begin{table}
    \caption{$p \wedge (\neg p \wedge q)$}
    \begin{tabular}{|c|c|c|c||c|}\hline
      $p$ & $q$ & $\neg p$ & $\neg p \wedge q$ & $p \wedge A$\\
          &     &          & ($A$)             &\\\hline\hline
      $T$ & $T$ & $F$ & $F$ & $F$\\\hline
      $T$ & $F$ & $F$ & $F$ & $F$\\\hline
      $F$ & $T$ & $T$ & $T$ & $F$\\\hline
      $F$ & $F$ & $T$ & $F$ & $F$\\\hline
    \end{tabular}
  \end{table}
\end{frame}

\begin{frame}
  \frametitle{Logical Implication and Equivalence}

  \begin{itemize}
    \item if $P \rightarrow Q$ is a tautology,
      then $P$ \alert{logically implies} $Q$:\\
      $P \Rightarrow Q$

    \pause
    \medskip
    \item if $P \leftrightarrow Q$ is a tautology,
      then $P$ and $Q$ are \alert{logically equivalent}:\\
      $P \Leftrightarrow Q$
  \end{itemize}
\end{frame}

\begin{frame}
  \frametitle{Logical Implication Example}

  $(p \rightarrow q) \wedge p \Rightarrow q$

  \begin{table}
    \caption{$(p \rightarrow q) \wedge p \rightarrow q$}
    \begin{tabular}{|c|c|c|c||c|}\hline
      $p$ & $q$ & $p \rightarrow q$ & $A \wedge p$ & $B \rightarrow q$\\
          &     & ($A$)             & ($B$)        &\\\hline\hline
      $T$ & $T$ & $T$ & $T$ & $T$\\\hline
      $T$ & $F$ & $F$ & $F$ & $T$\\\hline
      $F$ & $T$ & $T$ & $F$ & $T$\\\hline
      $F$ & $F$ & $T$ & $F$ & $T$\\\hline
    \end{tabular}
  \end{table}
\end{frame}

\begin{frame}
  \frametitle{Logical Equivalence Example}

  $\neg p \Leftrightarrow p \rightarrow F$

  \begin{table}
    \caption{$\neg p \leftrightarrow p \rightarrow F$}
    \begin{tabular}{|c|c|c||c|}\hline
      $p$ & $\neg p$ & $p \rightarrow F$ & $\neg p \leftrightarrow A$\\
          &          & ($A$)             &\\\hline\hline
      $T$ & $F$ & $F$ & $T$\\\hline
      $F$ & $T$ & $T$ & $T$\\\hline
    \end{tabular}
  \end{table}
\end{frame}

\begin{frame}
  \frametitle{Logical Equivalence Example}

  $p \rightarrow q \Leftrightarrow \neg p \vee q$

  \begin{table}
    \caption{$(p \rightarrow q) \leftrightarrow (\neg p \vee q)$}
    \begin{tabular}{|c|c|c|c|c||c|}\hline
      $p$ & $q$ & $p \rightarrow q$ & $\neg p$ & $\neg p \vee q$ & $A \leftrightarrow B$\\
          &     & ($A$)             &          & ($B$)           &\\\hline\hline
      $T$ & $T$ & $T$ & $F$ & $T$ & $T$\\\hline
      $T$ & $F$ & $F$ & $F$ & $F$ & $T$\\\hline
      $F$ & $T$ & $T$ & $T$ & $T$ & $T$\\\hline
      $F$ & $F$ & $T$ & $T$ & $T$ & $T$\\\hline
    \end{tabular}
  \end{table}
\end{frame}

\begin{frame}
  \frametitle{Logical Equivalence Example}

  \begin{itemize}
    \item implication: $p \rightarrow q$
    \item \emph{contrapositive}: $\neg q \rightarrow \neg p$
    \item \emph{converse}: $q \rightarrow p$
    \item \emph{inverse}: $\neg p \rightarrow \neg q$
  \end{itemize}

  \pause
  \medskip
  $p \rightarrow q \Leftrightarrow \neg q \rightarrow \neg p$

  \vspace{-12pt}
  \begin{table}
    \caption{$(p \rightarrow q) \leftrightarrow (\neg q \rightarrow \neg p)$}
    \begin{tabular}{|c|c|c|c|c|c||c|}\hline
      $p$ &   $q$    & $p \rightarrow q$
          & $\neg q$ & $\neg p$ & $\neg q \rightarrow \neg p$ & $A \leftrightarrow B$\\
          &     & (A)               &          &          & (B)
          &\\\hline\hline
      $T$ & $T$ & $T$ & $F$ & $F$ & $T$ & $T$\\\hline
      $T$ & $F$ & $F$ & $T$ & $F$ & $F$ & $T$\\\hline
      $F$ & $T$ & $T$ & $F$ & $T$ & $T$ & $T$\\\hline
      $F$ & $F$ & $T$ & $T$ & $T$ & $T$ & $T$\\\hline
    \end{tabular}
  \end{table}
\end{frame}

\begin{frame}
  \frametitle{Metalogic}

  \begin{itemize}
    \item $P_1,P_2,\dots,P_n \vdash Q$\\
      \smallskip
      There is a proof which infers the conclusion $Q$\\
      from the assumptions $P_1,P_2,\dots,P_n$.

    \pause
    \medskip
    \item $P_1,P_2,\dots,P_n \vDash Q$\\
      \smallskip
      $Q$ must be true if $P_1,P_2,\dots,P_n$ are all true.
  \end{itemize}
\end{frame}

\begin{frame}
  \frametitle{Formal Systems}

  \begin{itemize}
    \item a formal system is \alert{consistent} if for all WFFs $P$ and $Q$:\\
      if $P \vdash Q$ then $P \vDash Q$
    \item if every provable proposition is actually true

    \pause
    \bigskip
    \item a formal system is \alert{complete} if for all WFFs $P$ and $Q$:\\
        if $P \vDash Q$ then $P \vdash Q$
    \item if every true proposition can be proven
  \end{itemize}
\end{frame}

\begin{frame}
  \frametitle{Gödel's Theorem}

  \begin{itemize}
    \item propositional logic is consistent and complete
  \end{itemize}

  \pause
  \begin{theorem}[Gödel's Theorem]
    Any logical system that is powerful enough to express arithmetic\\
    must be either inconsistent or incomplete.
  \end{theorem}

  \pause
  \medskip
  \begin{itemize}
    \item liar's paradox: ``This statement is false.''
  \end{itemize}
\end{frame}

\subsection{Laws of Logic}

\begin{frame}
  \frametitle{Propositional Calculus}

  \begin{enumerate}
    \item semantic approach: \emph{truth tables}\\
      too complicated when the number of primitive statements grow

    \pause
    \medskip
    \item syntactic approach: \emph{rules of inference}\\
      obtain new propositions from known propositions\\
      using logical implications

    \pause
    \medskip
    \item axiomatic approach: \emph{Boolean algebra}\\
      substitute logically equivalent formulas for one another
  \end{enumerate}
\end{frame}

\begin{frame}
  \frametitle{Laws of Logic}

  \begin{tabular}{ll}
  \alert{Double Negation (DN)} &\\
    $\neg (\neg p) \Leftrightarrow p$ &\\\\
  \pause
  \alert{Commutativity (Co)} &\\
    $p \wedge q \Leftrightarrow q \wedge p$ &
    $p \vee q \Leftrightarrow q \vee p$\\\\
  \pause
  \alert{Associativity (As)} &\\
    $(p \wedge q) \wedge r \Leftrightarrow p \wedge (q \wedge r)$ &
    $(p \vee q) \vee r \Leftrightarrow p \vee (q \vee r)$\\\\
  \pause
  \alert{Idempotence (Ip)} &\\
    $p \wedge p \Leftrightarrow p$ &
    $p \vee p \Leftrightarrow p$\\\\
  \pause
  \alert{Inverse (In)} &\\
    $p \wedge \neg p \Leftrightarrow F$ &
    $p \vee \neg p \Leftrightarrow T$
  \end{tabular}
\end{frame}

\begin{frame}
  \frametitle{Laws of Logic}

  \begin{tabular}{ll}
  \alert{Identity (Id)} &\\
    $p \wedge T \Leftrightarrow p$ &
    $p \vee F \Leftrightarrow p$\\\\
  \pause
  \alert{Domination (Do)} &\\
    $p \wedge F \Leftrightarrow F$ &
    $p \vee T \Leftrightarrow T$\\\\
  \pause
  \alert{Distributivity (Di)} &\\
    $p \wedge (q \vee r) \Leftrightarrow (p \wedge q) \vee (p \wedge r)$ &
    $p \vee (q \wedge r) \Leftrightarrow (p \vee q) \wedge (p \vee r)$\\\\
  \pause
  \alert{Absorption (Ab)} &\\
    $p \wedge (p \vee q) \Leftrightarrow p$ &
    $p \vee (p \wedge q) \Leftrightarrow p$\\\\
  \pause
  \alert{DeMorgan's Laws (DM)} &\\
    $\neg (p \wedge q) \Leftrightarrow \neg p \vee \neg q$ &
    $\neg (p \vee q) \Leftrightarrow \neg p \wedge \neg q$
  \end{tabular}
\end{frame}

\begin{frame}
  \frametitle{Equivalence Example}

  \begin{eqnarray*}
                    & p \rightarrow q           &   \\
    \pause
    \Leftrightarrow & \neg p \vee q             &   \\
    \pause
    \Leftrightarrow & q \vee \neg p             & Co\\
    \pause
    \Leftrightarrow & \neg \neg q \vee \neg p   & DN\\
    \pause
    \Leftrightarrow & \neg q \rightarrow \neg p &
  \end{eqnarray*}
\end{frame}

\begin{frame}
  \frametitle{Equivalence Example}

  \begin{eqnarray*}
                    & \neg (\neg ((p \vee q) \wedge r) \vee \neg q)      &   \\
    \pause
    \Leftrightarrow & \neg \neg ((p \vee q) \wedge r) \wedge \neg \neg q & DM\\
    \pause
    \Leftrightarrow & ((p \vee q) \wedge r) \wedge q                     & DN\\
    \pause
    \Leftrightarrow & (p \vee q) \wedge (r \wedge q)                     & As\\
    \pause
    \Leftrightarrow & (p \vee q) \wedge (q \wedge r)                     & Co\\
    \pause
    \Leftrightarrow & ((p \vee q) \wedge q) \wedge r                     & As\\
    \pause
    \Leftrightarrow & q \wedge r                                         & Ab
  \end{eqnarray*}
\end{frame}

\begin{frame}
  \frametitle{Duality}

  \begin{itemize}
    \item \alert{dual} of $s$: $s^d$\\
      replace: $\wedge$ with $\vee$, $\vee$ with $\wedge$, $T$ with $F$, $F$ with $T$

    \pause
    \medskip
    \item \alert{principle of duality}:
      if $s \Leftrightarrow t$ then $s^d \Leftrightarrow t^d$
  \end{itemize}

  \pause
  \begin{exampleblock}{example}
    \begin{eqnarray*}
      s:   & (p \wedge \neg q) \vee (r \wedge T)\\
      s^d: & (p \vee \neg q) \wedge (r \vee F)
    \end{eqnarray*}
  \end{exampleblock}
\end{frame}

\section{Rules of Inference}

\subsection{Introduction}

\begin{frame}
  \frametitle{Inference}

  \begin{itemize}
    \item establish the validity of an argument
    \item starting from a set of propositions
    \item which are assumed or proven to be true
  \end{itemize}

  \pause
  \begin{block}{notation}
    \begin{columns}
      \column{.5\textwidth}
      \[
      \frac
        {
          \begin{array}{c}
            p_1\\
            p_2\\
            \dots\\
            p_n
          \end{array}
        }
        {
          \therefore q
        }
      \]

      \column{.5\textwidth}
      $p_1 \wedge p_2 \wedge \cdots \wedge p_n \Rightarrow q$
    \end{columns}
  \end{block}
\end{frame}

\subsection{Basic Rules}

\begin{frame}
  \frametitle{Trivial Rules}

  \begin{columns}
    \column{.45\textwidth}
    \begin{block}{Identity (ID)}
      \[
      \frac
        {
          \begin{array}{c}
            p
          \end{array}
        }
        {
          \therefore p
        }
      \]
    \end{block}

    \pause
    \column{.45\textwidth}
    \begin{block}{Contradiction (CTR)}
    \[
    \frac
      {
        \begin{array}{c}
          F
        \end{array}
      }
      {
        \therefore p
      }
    \]
    \end{block}
  \end{columns}
\end{frame}

\begin{frame}
  \frametitle{Basic Rules}

  \begin{columns}[t]
    \column{.45\textwidth}
    \begin{block}{OR Introduction (OrI)}
      \[
      \frac
        {
          \begin{array}{c}
            p
          \end{array}
        }
        {
          \therefore p \vee q
        }
      \]
    \end{block}

    \pause
    \column{.45\textwidth}
    \begin{block}{AND Elimination (AndE)}
    \[
    \frac
      {
        \begin{array}{c}
          p \wedge q
        \end{array}
      }
      {
        \therefore p
      }
    \]
    \end{block}

    \pause
    \begin{block}{AND Introduction (AndI)}
      \[
      \frac
        {
          \begin{array}{c}
            p\\
            q
          \end{array}
        }
        {
          \therefore p \wedge q
        }
      \]
    \end{block}
  \end{columns}
\end{frame}

\subsection{Modus Ponens}

\begin{frame}
  \frametitle{Modus Ponens}

  \begin{block}{Implication Elimination (ImpE)}
    \[
    \frac
      {
        \begin{array}{c}
          p \rightarrow q\\
          p
        \end{array}
      }
      {
        \therefore q
      }
    \]
  \end{block}

  \pause
  \begin{exampleblock}{example}
    \begin{itemize}
      \item If Lydia wins the lottery, she will buy a car.
      \item Lydia has won the lottery.

      \medskip
      \item Therefore, Lydia will buy a car.
    \end{itemize}
  \end{exampleblock}
\end{frame}

\begin{frame}
  \frametitle{Modus Tollens}

  \begin{block}{Modus Tollens (MT)}
    \[
    \frac
      {
        \begin{array}{c}
          p \rightarrow q\\
          \neg q
        \end{array}
      }
      {
        \therefore \neg p
      }
    \]
  \end{block}

  \pause
  \begin{exampleblock}{example}
    \begin{itemize}
      \item If Lydia wins the lottery, she will buy a car.
      \item Lydia did not buy a car.

      \medskip
      \item Therefore, Lydia did not win the lottery.
    \end{itemize}
  \end{exampleblock}
\end{frame}

\begin{frame}
  \frametitle{Modus Tollens}

  \begin{columns}
    \column{.3\textwidth}
    \[
    \frac
      {
        \begin{array}{c}
          p \rightarrow q\\
          \neg q
        \end{array}
      }
      {
        \therefore \neg p
      }
    \]

    \pause
    \column{.65\textwidth}
    \begin{eqnarray*}
      1. & p \rightarrow q           & A\\
      \pause
      2. & \neg q \rightarrow \neg p & EQ: 1\\
      \pause
      3. & \neg q                    & A\\
      \pause
      4. & \neg p                    & ImpE:2,3\\
    \end{eqnarray*}
  \end{columns}
\end{frame}

\begin{frame}
  \frametitle{Fallacies}

  \begin{columns}
    \column{.5\textwidth}
    \[
    \frac
      {
      \begin{array}{c}
        p \rightarrow q\\
        q
        \end{array}
      }
      {
        \therefore p
      }
    \]

    \medskip
    \column{.5\textwidth}
    $(p \rightarrow q) \wedge q \nRightarrow p$
    \begin{itemize}
     \item $p: F$, $q: T$
      \smallskip
      $(F \rightarrow T) \wedge T \rightarrow F: F$
    \end{itemize}
  \end{columns}

  \pause
  \begin{exampleblock}{example}
    \begin{itemize}
      \item If Lydia wins the lottery, she will buy a car.
      \item Lydia has bought a car.

      \medskip
      \item Therefore, Lydia has won the lottery.
    \end{itemize}
  \end{exampleblock}
\end{frame}

\begin{frame}
  \frametitle{Fallacies}

  \begin{columns}
    \column{.5\textwidth}
    \[
    \frac
      {
        \begin{array}{c}
          p \rightarrow q\\
          \neg p
        \end{array}
      }
      {
        \therefore \neg q
      }
    \]

    \column{.5\textwidth}
    $(p \rightarrow q) \wedge \neg p \nRightarrow \neg q$
    \begin{itemize}
     \item $p: F$, $q: T$
      \smallskip
      $(F \rightarrow T) \wedge T \rightarrow F: F$
    \end{itemize}
  \end{columns}

  \pause
  \begin{exampleblock}{example}
    \begin{itemize}
      \item If Lydia wins the lottery, she will buy a car.
      \item Lydia has not won the lottery.

      \medskip
      \item Therefore, Lydia will not buy a car.
    \end{itemize}
  \end{exampleblock}
\end{frame}

\subsection{Provisional Assumptions}

\begin{frame}
  \frametitle{Implication Introduction}

  \begin{block}{Implication Introduction (ImpI)}
    \[
    \frac
      {
        \begin{array}{c}
          p \vdash q
        \end{array}
      }
      {
        \therefore ~ \vdash p \rightarrow q
      }
    \]
  \end{block}

  \begin{itemize}
    \item if it can be shown that $q$ is true assuming $p$ is true
    \item then $p \rightarrow q$ is true \emph{without assuming $p$ is true}

    \pause
    \medskip
    \item $p$ is a \alert{provisional assumption} (PA)
    \item provisional assumptions have to be \alert{discharged}
  \end{itemize}
\end{frame}

\begin{frame}
  \frametitle{Implication Introduction Example}

  \begin{columns}
    \column{.3\textwidth}
    \[
    \frac
      {
        \begin{array}{c}
          p \rightarrow q\\
          \neg q
        \end{array}
      }
      {
        \therefore \neg p
      }
    \]

    \pause
    \column{.65\textwidth}
    \begin{eqnarray*}
      1. & p                         & PA\\
      \pause
      2. & p \rightarrow q           & A\\
      \pause
      3. & q                         & ImpE:2,1\\
      \pause
      4. & \neg q                    & A\\
      \pause
      5. & q \rightarrow F           & EQ:4\\
      \pause
      6. & F                         & ImpE:5,3\\
      \pause
      7. & p \rightarrow F           & ImpI:1,6\\
      \pause
      8. & \neg p                    & EQ:7\\
    \end{eqnarray*}
  \end{columns}
\end{frame}

\begin{frame}
  \frametitle{OR Elimination}

  \begin{block}{OR Elimination (OrE)}
    \[
    \frac
      {
        \begin{array}{c}
          p \vee q\\
          p \vdash r\\
          q \vdash r
        \end{array}
      }
      {
        \therefore ~ \vdash r
      }
    \]
  \end{block}

  \begin{itemize}
    \item $p$ and $q$ are provisional assumptions
  \end{itemize}
\end{frame}

\begin{frame}
  \frametitle{Disjunctive Syllogism}

  \begin{block}{Disjunctive Syllogism (DS)}
    \[
    \frac
      {
        \begin{array}{c}
          p \vee q\\
          \neg p
        \end{array}
      }
      {
        \therefore q
      }
    \]
  \end{block}

  \pause
  \begin{exampleblock}{example}
    \begin{itemize}
      \item Bart's wallet is either in his pocket or on his desk.
      \item Bart's wallet is not in his pocket.

      \medskip
      \item Therefore, Bart's wallet is on his desk.
    \end{itemize}
  \end{exampleblock}
\end{frame}

\begin{frame}
  \frametitle{Disjunctive Syllogism}

  \begin{columns}
    \column{.5\textwidth}
    \[
    \frac
      {
        \begin{array}{c}
          p \vee q\\
          \neg p
        \end{array}
      }
      {
        \therefore q
      }
    \]

    \pause
    \bigskip
    applying OrE:
    \[
    \frac
      {
        \begin{array}{c}
          p \vee q\\
          p \vdash q\\
          q \vdash q\\
        \end{array}
      }
      {
        \therefore q
      }
    \]

    \pause
    \column{.5\textwidth}
    \begin{eqnarray*}
      1.   & p \vee q        & A\\
      \pause
      2.   & \neg p          & A\\
      \pause
      3.   & p \rightarrow F & EQ:2\\
      \pause
      4a1. & p               & PA\\
      \pause
      4a2. & F               & ImpE:3,4a1\\
      \pause
      4a.  & q               & CTR:4a2\\
      \pause
      4b1. & q               & PA\\
      \pause
      4b.  & q               & ID:4b1\\
      \pause
      5.   & q               & OrE:1,4a,4b
    \end{eqnarray*}
  \end{columns}
\end{frame}

\begin{frame}
  \frametitle{Hypothetical Syllogism}

  \begin{columns}
    \column{.5\textwidth}
    \begin{block}{Hypothetical Syllogism (HS)}
      \[
      \frac
        {
          \begin{array}{c}
            p \rightarrow q\\
            q \rightarrow r
          \end{array}}
        {
          \therefore p \rightarrow r
        }
      \]
    \end{block}

    \pause
    \column{.5\textwidth}
    \begin{eqnarray*}
      1. & p               & PA\\
      \pause
      2. & p \rightarrow q & A\\
      \pause
      3. & q               & ImpE:2,1\\
      \pause
      4. & q \rightarrow r & A\\
      \pause
      5. & r               & ImpE:4,3\\
      \pause
      6. & p \rightarrow r & ImpI:1,5\\
    \end{eqnarray*}
  \end{columns}
\end{frame}

\begin{frame}
  \frametitle{Hypotetical Syllogism Example}

  Spock to Lieutenant Decker:
  \begin{quote}
    It would be a suicide to attack the enemy ship now.\\
    Someone who attempts suicide is not psychologically fit\\
    to command the Enterprise.\\
    Therefore, I am obliged to relieve you from duty.
  \end{quote}

  \pause
  \begin{itemize}
    \item $p$: Decker attacks the enemy ship.
    \item $q$: Decker attempts suicide.
    \item $r$: Decker is not psychologically fit to command the Enterprise.
    \item $s$: Spock relieves Decker from duty.
  \end{itemize}
\end{frame}

\begin{frame}
  \frametitle{Hypotetical Syllogism Example}

  \begin{columns}
    \column{.3\textwidth}
    \[
    \frac
      {
        \begin{array}{c}
          p\\
          p \rightarrow q\\
          q \rightarrow r\\
          r \rightarrow s
        \end{array}
      }
      {
        \therefore s
      }
    \]

    \pause
    \column{.65\textwidth}
    \begin{eqnarray*}
      1. & p \rightarrow q & A\\
      \pause
      2. & q \rightarrow r & A\\
      \pause
      3. & p \rightarrow r & HS:1,2\\
      \pause
      4. & r \rightarrow s & A\\
      \pause
      5. & p \rightarrow s & HS:3,4\\
      \pause
      6. & p               & A\\
      \pause
      7. & s               & ImpE:5,6
    \end{eqnarray*}
  \end{columns}
\end{frame}

% \begin{frame}
%   \frametitle{Dilemma}
%
%   \begin{columns}[t]
%     \column{.5\textwidth}
%     \begin{block}{Constructive Dilemma}
%       \[
%       \frac
%         {
%           \begin{array}{c}
%             p \rightarrow q\\
%             r \rightarrow s\\
%             p \vee r
%           \end{array}}
%         {
%           \therefore q \vee s
%         }
%       \]
%     \end{block}
%
%     \pause
%     \column{.5\textwidth}
%     \begin{block}{Destructive Dilemma}
%       \[
%       \frac
%         {
%           \begin{array}{c}
%             p \rightarrow q\\
%             r \rightarrow s\\
%             \neg q \vee \neg s
%           \end{array}
%           }
%           {
%             \therefore \neg p \vee \neg r
%           }
%       \]
%     \end{block}
%   \end{columns}
% \end{frame}

\begin{frame}
  \frametitle{Inference Examples}

  \begin{columns}[t]
    \column{.25\textwidth}
    \[
    \frac
      {
        \begin{array}{c}
          p \rightarrow r\\
          r \rightarrow s\\
          x \vee \neg s\\
          u \vee \neg x\\
          \neg u
        \end{array}
      }
      {
        \therefore \neg p
      }
    \]

    \pause
    \column{.3\textwidth}
    \begin{eqnarray*}
      1. & \neg u          & A\\
      \pause
      2. & u \vee \neg x   & A\\
      \pause
      3. & \neg x          & DS:2,1\\
      \pause
      4. & x \vee \neg s   & A\\
      \pause
      5. & \neg s          & DS:4,3\\
    \end{eqnarray*}

    \pause
    \column{.45\textwidth}
    \begin{eqnarray*}
      6. & r \rightarrow s & A\\
      \pause
      7. & \neg r          & MT:6,5\\
      \pause
      8. & p \rightarrow r & A\\
      \pause
      9. & \neg p          & MT:8,7\\
    \end{eqnarray*}
  \end{columns}
\end{frame}

\begin{frame}
  \frametitle{Inference Examples}

  \[
  \frac
    {
      \begin{array}{c}
        (\neg p \vee \neg q) \rightarrow (r \wedge s)\\
        r \rightarrow x\\
        \neg x
      \end{array}
    }
    {
      \therefore p
    }
  \]

  \pause
  \begin{columns}[t]
    \column{.3\textwidth}
    \begin{eqnarray*}
      1. & \neg x                                        & A\\
      \pause
      2. & r \rightarrow x                               & A\\
      \pause
      3. & \neg r                                        & MT:2,1\\
      \pause
      4. & \neg r \vee \neg s                            & OrI:3\\
      \pause
      5. & \neg (r \wedge s)                             & DM:4
    \end{eqnarray*}

    \pause
    \column{.6\textwidth}
    \begin{eqnarray*}
      6. & (\neg p \vee \neg q) \rightarrow (r \wedge s) & A\\
      \pause
      7. & \neg (\neg p \vee \neg q)                     & MT:6,5\\
      \pause
      8. & p \wedge q                                    & DM:7\\
      \pause
      9. & p                                             & AndE:8
    \end{eqnarray*}
  \end{columns}
\end{frame}

\begin{frame}
  \frametitle{Inference Examples}

  \begin{columns}
    \column{.3\textwidth}
    \[
    \frac
      {
        \begin{array}{c}
          p \rightarrow (q \vee r)\\
          s \rightarrow \neg r\\
          q \rightarrow \neg p\\
          p\\
          s
        \end{array}
      }
      {
        \therefore F
      }
    \]

    \pause
    \column{.6\textwidth}
    \begin{eqnarray*}
      1. & p                        & A\\
      \pause
      2. & q \rightarrow \neg p     & A\\
      \pause
      3. & \neg q                   & MT:2,1\\
      \pause
      4. & s                        & A\\
      \pause
      5. & s \rightarrow \neg r     & A\\
      \pause
      6. & \neg r                   & ImpE:5,4\\
      \pause
      7. & p \rightarrow (q \vee r) & A\\
      \pause
      8. & q \vee r                 & ImpE:7,2\\
      \pause
      9. & q                        & DS:8,6\\
      \pause
      10. & q \wedge \neg q : F      & AndI:9,3
    \end{eqnarray*}
  \end{columns}
\end{frame}

\begin{frame}
  \frametitle{Inference Examples}

  If there is a chance of rain or her red headband is missing,\\
  then Lois will not mow her lawn. Whenever the temperature is\\
  over 20\textcelsius, there is no chance for rain. Today the temperature is\\
  22\textcelsius ~ and Lois is wearing her red headband. Therefore,\\
  Lois will mow her lawn.

  \pause
  \medskip
  \begin{itemize}
    \item $p$: There is a chance of rain.
    \item $q$: Lois' red headband is lost.
    \item $r$: Lois mows her lawn.
    \item $s$: The temperature is over 20\textcelsius.
  \end{itemize}
\end{frame}

\begin{frame}
  \frametitle{Inference Examples}

  \begin{columns}
    \column{.3\textwidth}
    \[
    \frac
      {
        \begin{array}{c}
          (p \vee q) \rightarrow \neg r\\
          s \rightarrow \neg p\\
          s \wedge \neg q
        \end{array}
      }
      {
        \therefore r
      }
    \]

    \pause
    \column{.65\textwidth}
    \begin{eqnarray*}
      1. & s \wedge \neg q                & A\\
      \pause
      2. & s                              & AndE:1\\
      \pause
      3. & s \rightarrow \neg p           & A\\
      \pause
      4. & \neg p                         & ImpE:3,2\\
      \pause
      5. & \neg q                         & AndE:1\\
      \pause
      6. & \neg p \wedge \neg q           & AndI:4,5\\
      \pause
      7. & \neg (p \vee q)                & DM:6\\
      \pause
      8. & (p \vee q) \rightarrow \neg r  & A\\
      \pause
      9. & ?                              & 7,8
    \end{eqnarray*}
  \end{columns}
\end{frame}

\section*{References}

\begin{frame}
  \frametitle{References}

  \begin{block}{Required Reading: Grimaldi}
    \begin{itemize}
      \item Chapter 2: Fundamentals of Logic
      \begin{itemize}
        \item 2.1. \alert{Basic Connectives and Truth Tables}
        \item 2.2. \alert{Logical Equivalence: The Laws of Logic}\\
        \item 2.3. \alert{Logical Implication: Rules of Inference}
      \end{itemize}
    \end{itemize}
  \end{block}

  \begin{block}{Supplementary Reading: O'Donnell, Hall, Page}
    \begin{itemize}
      \item Chapter 6: Propositional Logic
    \end{itemize}
  \end{block}
\end{frame}

\end{document}
