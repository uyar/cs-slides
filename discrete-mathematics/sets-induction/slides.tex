% Copyright (c) 2001-2014
%       H. Turgut Uyar <uyar@itu.edu.tr>
%       Ayşegül Gençata Yayımlı <gencata@itu.edu.tr>
%       Emre Harmancı <harmanci@itu.edu.tr>
%
% This work is licensed under a "Creative Commons
% Attribution-NonCommercial-ShareAlike 4.0 International License".
% For more information, please visit:
% https://creativecommons.org/licenses/by-nc-sa/4.0/

\documentclass[dvipsnames]{beamer}

\usepackage{ae}
\usepackage[T1]{fontenc}
\usepackage[utf8]{inputenc}
\setbeamertemplate{navigation symbols}{}
\setbeamersize{text margin left=2em, text margin right=2em}

\mode<presentation>
{
  \usetheme{Rochester}
  \setbeamercovered{transparent}
}

\title{Discrete Mathematics}
\subtitle{Sets and Induction}

\author{H. Turgut Uyar \and Ayşegül Gençata Yayımlı \and Emre Harmancı}
\date{2001-2014}

\AtBeginSubsection[]
{
  \begin{frame}<beamer>
    \frametitle{Topics}
    \tableofcontents[currentsection,currentsubsection]
  \end{frame}
}

%\beamerdefaultoverlayspecification{<+->}

\pgfdeclareimage[height=1cm]{license}{../license}

\pgfdeclareimage{induction}{induction}
\pgfdeclareimage{error1}{error1}
\pgfdeclareimage{error2}{error2}

\begin{document}

\begin{frame}
  \titlepage
\end{frame}

\begin{frame}
  \frametitle{License}

  \pgfuseimage{license}\hfill
  \copyright~2001-2014 T. Uyar, A. Yayımlı, E. Harmancı

  \vfill
  \begin{footnotesize}
    You are free to:
    \begin{itemize}
      \itemsep0em
      \item Share -- copy and redistribute the material in any medium or format
      \item Adapt -- remix, transform, and build upon the material
    \end{itemize}

    Under the following terms:
    \begin{itemize}
      \itemsep0em
      \item Attribution -- You must give appropriate credit, provide a link to
        the license, and indicate if changes were made.

      \item NonCommercial -- You may not use the material for commercial
        purposes.

      \item ShareAlike -- If you remix, transform, or build upon the material,
        you must distribute your contributions under the same license as the
        original.
    \end{itemize}
  \end{footnotesize}

  \begin{small}
    For more information:\\
    \url{https://creativecommons.org/licenses/by-nc-sa/4.0/}

    \smallskip
    Read the full license:\\
    \url{https://creativecommons.org/licenses/by-nc-sa/4.0/legalcode}
  \end{small}
\end{frame}

\begin{frame}
  \frametitle{Topics}
  \tableofcontents
\end{frame}

\section{Sets}

\subsection{Introduction}

\begin{frame}
  \frametitle{Set}

  \begin{definition}
    \alert{set}: a collection of elements that are
    \begin{itemize}
      \item distinct
      \item unordered
      \item non-repeating
    \end{itemize}
  \end{definition}
\end{frame}

\begin{frame}
  \frametitle{Set Representation}

  \begin{itemize}
    \item \emph{explicit representation}\\
      elements are listed within braces: $\{a_1,a_2,\dots,a_n\}$

    \pause
    \medskip
    \item \emph{implicit representation}\\
      elements that validate a predicate: $\{x | x \in G, p(x)\}$

    \pause
    \medskip
    \item $\emptyset$: empty set

    \pause
    \medskip
    \item let $S$ be a set, and $a$ be an element
    \begin{itemize}
      \item $a \in S$: $a$ is an element of $S$
      \item $a \notin S$: $a$ is not an element of $S$
    \end{itemize}

    \pause
    \medskip
    \item $|S|$: number of elements in $S$ (\alert{cardinality})
  \end{itemize}
\end{frame}

\begin{frame}
  \frametitle{Explicit Representation Example}

  \begin{example}
    $\{3,8,2,11,5\}$\\
    $11 \in \{3,8,2,11,5\}$\\
    $|\{3,8,2,11,5\}|=5$
  \end{example}
\end{frame}

\begin{frame}
  \frametitle{Implicit Representation Examples}

  \begin{example}
    $\{x | x \in \mathbb{Z}^+, 20 < x^3 < 100\} \equiv \{3,4\}$

    $\{2x-1 | x \in \mathbb{Z}^+, 20 < x^3 < 100\} \equiv \{5,7\}$
  \end{example}

  \pause
  \begin{example}
    $A = \{ x | x \in \mathbb{R}, 1 \leq x \leq 5 \}$
  \end{example}

  \pause
  \begin{example}
    $E = \{ n | n \in \mathbb{N}, \exists k \in \mathbb{N}~[n=2k] \}$\\
    $A = \{ x | x \in E, 1 \leq x \leq 5 \}$
  \end{example}
\end{frame}

\begin{frame}
  \frametitle{Set Dilemma}

  \begin{itemize}
    \item There is a barber who lives in a small town.\\
      He shaves all those men who don't shave themselves.\\
      He doesn't shave those men who shave themselves.

    \smallskip
    \begin{quote}
      Does the barber shave himself?
    \end{quote}

    \pause
    \item yes $\rightarrow$ but he doesn't shave men who shave themselves\\
      $\rightarrow$ no

    \pause
    \item no $\rightarrow$ but he shaves all men who don't shave themselves\\
      $\rightarrow$ yes
  \end{itemize}
\end{frame}

\begin{frame}
  \frametitle{Set Dilemma}

  \begin{itemize}
    \item let $S$ be the set of sets that are not an element of themselves\\
      $S = \{A | A \notin A\}$

    \pause
    \begin{quote}
      Is $S$ an element of itself?
    \end{quote}

    \pause
    \item yes $\rightarrow$ but the predicate is not valid $\rightarrow$ no

    \pause
    \item no $\rightarrow$ but the predicate is valid $\rightarrow$ yes
  \end{itemize}
\end{frame}

% \subsection{Subset}

\begin{frame}
  \frametitle{Subset}

  \begin{definition}
    $A \subseteq B \Leftrightarrow \forall x~[x \in A \rightarrow x \in B]$
  \end{definition}

  \pause
  \begin{itemize}
    \item \alert{set equality}:\\
      $A = B \Leftrightarrow (A \subseteq B) \wedge (B \subseteq A)$

    \pause
    \item \alert{proper subset}:\\
      $A \subset B \Leftrightarrow (A \subseteq B) \wedge (A \neq B)$

    \pause
    \item $\forall S~[\emptyset \subseteq S]$
  \end{itemize}
\end{frame}

\begin{frame}
  \frametitle{Subset}

  \begin{block}{not a subset}
    \begin{eqnarray*}
      A \nsubseteq B & \Leftrightarrow
                     & \neg \forall x~[x \in A \rightarrow x \in B]\\\pause
                     & \Leftrightarrow
                     & \exists x~\neg [x \in A \rightarrow x \in B]\\\pause
                     & \Leftrightarrow
                     & \exists x~\neg [\neg (x \in A) \vee (x \in B)]\\\pause
                     & \Leftrightarrow
                     & \exists x~[(x \in A) \wedge \neg (x \in B)]\\\pause
                     & \Leftrightarrow
                     & \exists x~[(x \in A) \wedge (x \notin B)]
    \end{eqnarray*}
  \end{block}
\end{frame}

\begin{frame}
  \frametitle{Power Set}

  \begin{definition}
    \alert{power set}: $\mathcal{P}(S)$\\
    the set of all subsets of a set,\\
    including the empty set and the set itself
  \end{definition}

  \begin{itemize}
    \item if a set has $n$ elements, its power set has $2^n$ elements
  \end{itemize}
\end{frame}

\begin{frame}
  \frametitle{Example of Power Set}

  \begin{example}
    \begin{eqnarray*}
      \mathcal{P}(\{1,2,3\}) = & \{ &\\
                               &    & \emptyset,\\
                               &    & \{1\},\{2\},\{3\},\\
                               &    & \{1,2\},\{1,3\},\{2,3\},\\
                               &    & \{1,2,3\}\\
                               & \} &
    \end{eqnarray*}
  \end{example}
\end{frame}

\subsection{Set Operations}

\begin{frame}
  \frametitle{Set Operations}

  \begin{block}{complement}
    $\overline{A} = \{ x | x \notin A \} $
  \end{block}

  \pause
  \begin{block}{intersection}
    $A \cap B = \{ x | (x \in A) \wedge (x \in B) \}$

    \begin{itemize}
      \item if $A \cap B = \emptyset$ then $A$ and $B$ are \alert{disjoint}
    \end{itemize}
  \end{block}

  \pause
  \begin{block}{union}
    $A \cup B = \{ x | (x \in A) \vee (x \in B) \}$
  \end{block}
\end{frame}

\begin{frame}
  \frametitle{Set Operations}

  \begin{block}{difference}
    $A - B = \{ x | (x \in A) \wedge (x \notin B) \}$

    \pause
    \begin{itemize}
      \item $A-B = A \cap \overline{B}$

      \pause
      \item \emph{symmetric difference}:\\
        $A \bigtriangleup B = \{ x | (x \in A \cup B)
                              \wedge (x \notin A \cap B) \}$
    \end{itemize}
  \end{block}
\end{frame}

\begin{frame}
  \frametitle{Cartesian Product}

  \begin{definition}
    \alert{Cartesian product}:\\
      $A \times B = \{(a,b) | a \in A, b \in B\}$

      \medskip
      $A \times B \times C \times \dots \times N =
        \{(a,b,\ldots,n) | a \in A, b \in B, \ldots, n \in N\}$
  \end{definition}

  \medskip
  \begin{itemize}
    \item $|A \times B \times C \times \dots \times N| =
      |A| \cdot |B| \cdot |C| \cdots|N|$
  \end{itemize}
\end{frame}

\begin{frame}
  \frametitle{Cartesian Product Example}

  \begin{example}
    $A = \{a_1.a_2,a_3,a_4\}$

    $B = \{b_1,b_2,b_3\}$

    \medskip
    \begin{eqnarray*}
      A \times B = & \{ & \\
                   &    & (a_1,b_1),(a_1,b_2),(a_1,b_3),\\
                   &    & (a_2,b_1),(a_2,b_2),(a_2,b_3),\\
                   &    & (a_3,b_1),(a_3,b_2),(a_3,b_3),\\
                   &    & (a_4,b_1),(a_4,b_2),(a_4,b_3)\\
                   & \} &
    \end{eqnarray*}
  \end{example}
\end{frame}

\begin{frame}
  \frametitle{Equivalences}

  \begin{tabular}{ll}
    \alert{Double Complement} &\\
      $\overline{\overline{A}} = A$\\\\
    \pause
    \alert{Commutativity} &\\
      $A \cap B = B \cap A$ &
      $A \cup B = B \cup A$\\\\
    \pause
    \alert{Associativity} &\\
      $(A \cap B) \cap C = A \cap (B \cap C)$ &
      $(A \cup B) \cup C = A \cup (B \cup C)$\\\\
    \pause
    \alert{Idempotence} &\\
      $A \cap A = A$ &
      $A \cup A = A$\\\\
    \pause
    \alert{Inverse} &\\
      $A \cap \overline{A} = \emptyset$ &
      $A \cup \overline{A} = \mathcal{U}$\\\\
  \end{tabular}
\end{frame}

\begin{frame}
  \frametitle{Equivalences}

  \begin{tabular}{ll}
    \alert{Identity} &\\
      $A \cap \mathcal{U} = A$ &
      $A \cup \emptyset = A$\\\\
    \pause
    \alert{Domination} &\\
      $A \cap \emptyset = \emptyset$ &
      $A \cup \mathcal{U} = \mathcal{U}$\\\\
    \pause
    \alert{Distributivity} &\\
      $A \cap (B \cup C) = (A \cap B) \cup (A \cap C)$ &
      $A \cup (B \cap C) = (A \cup B) \cap (A \cup C)$\\\\
    \pause
    \alert{Absorption} &\\
      $A \cap (A \cup B) = A$ &
      $A \cup (A \cap B) = A$\\\\
    \pause
    \alert{DeMorgan's Laws} &\\
      $\overline{A \cap B} = \overline{A} \cup \overline{B}$ &
      $\overline{A \cup B} = \overline{A} \cap \overline{B}$\\\\
  \end{tabular}
\end{frame}

\begin{frame}
  \frametitle{DeMorgan's Laws}

  \begin{proof}
    \begin{eqnarray*}
      \overline{A \cap B} & = & \{x | x \notin (A \cap B)\}\\\pause
                          & = & \{x | \neg (x \in (A \cap B))\}\\\pause
                          & = & \{x | \neg ((x \in A) \wedge (x \in B))\}\\\pause
                          & = & \{x | \neg (x \in A) \vee \neg (x \in B)\}\\\pause
                          & = & \{x | (x \notin A) \vee (x \notin B)\}\\\pause
                          & = & \{x | (x \in \overline{A}) \vee (x \in \overline{B})\}\\\pause
                          & = & \{x | x \in \overline{A} \cup \overline{B}\}\\\pause
                          & = & \overline{A} \cup \overline{B}
    \end{eqnarray*}
  \end{proof}
\end{frame}

\begin{frame}
  \frametitle{Example of Equivalence}

  \begin{theorem}
    $A \cap (B-C) = (A \cap B) - (A \cap C)$
  \end{theorem}
\end{frame}

\begin{frame}
  \frametitle{Equivalence Example}

  \begin{proof}
    \begin{eqnarray*}
      (A \cap B) - (A \cap C)
          & = & (A \cap B) \cap \overline{(A \cap C)}\\\pause
          & = & (A \cap B) \cap (\overline{A} \cup \overline{C})\\\pause
          & = & ((A \cap B) \cap \overline{A}) \cup ((A \cap B) \cap \overline{C}))\\\pause
          & = & \emptyset \cup ((A \cap B) \cap \overline{C}))\\\pause
          & = & (A \cap B) \cap \overline{C}\\\pause
          & = & A \cap (B \cap \overline{C})\\\pause
          & = & A \cap (B - C)
    \end{eqnarray*}
  \end{proof}
\end{frame}

\begin{frame}
  \frametitle{Equivalence Proof Example}

  \begin{theorem}
    \begin{eqnarray*}
      &                 & A \subseteq B\\
      & \Leftrightarrow & A \cup B = B\\
      & \Leftrightarrow & A \cap B = A\\
      & \Leftrightarrow & \overline{B} \subseteq \overline{A}
    \end{eqnarray*}
  \end{theorem}
\end{frame}

\begin{frame}
  \frametitle{Equivalence Proof Example}

  \begin{proof}[$A \subseteq B \Rightarrow A \cup B = B$]
    $A \cup B = B \Leftrightarrow
      A \cup B \subseteq B \wedge B \subseteq A \cup B$

    \pause
    \bigskip
    \begin{columns}
      \column{.4\textwidth}
      $B \subseteq A \cup B$

      \pause
      \medskip
      \column{.5\textwidth}
      \begin{eqnarray*}
        x \in A \cup B & \Rightarrow & x \in A \vee x \in B\\\pause
        A \subseteq B  & \Rightarrow & x \in B\\\pause
                       & \Rightarrow & A \cup B \subseteq B
      \end{eqnarray*}
    \end{columns}
  \end{proof}
\end{frame}

\begin{frame}
  \frametitle{Equivalence Proof Example}

  \begin{proof}[$A \cup B = B \Rightarrow A \cap B = A$]
    $A \cap B = A \Leftrightarrow
      A \cap B \subseteq A \wedge A \subseteq A \cap B$

    \pause
    \bigskip
    \begin{columns}
      \column{.4\textwidth}
      $A \cap B \subseteq A$

      \pause
      \medskip
      \column{.5\textwidth}
      \begin{eqnarray*}
        y \in A      & \Rightarrow & y \in A \cup B\\\pause
        A \cup B = B & \Rightarrow & y \in B\\\pause
                     & \Rightarrow & y \in A \cap B\\\pause
                     & \Rightarrow & A \subseteq A \cap B
      \end{eqnarray*}
    \end{columns}
  \end{proof}
\end{frame}

\begin{frame}
  \frametitle{Equivalence Proof Example}

  \begin{proof}[$A \cap B = A \Rightarrow \overline{B} \subseteq \overline{A}$]
    \begin{eqnarray*}
      z \in \overline{B} & \Rightarrow & z \notin B\\\pause
                         & \Rightarrow & z \notin A \cap B\\\pause
      A \cap B = A       & \Rightarrow & z \notin A\\\pause
                         & \Rightarrow & z \in \overline{A}\\\pause
                         & \Rightarrow & \overline{B} \subseteq \overline{A}
    \end{eqnarray*}
  \end{proof}
\end{frame}

\begin{frame}
  \frametitle{Equivalence Proof Example}

  \begin{proof}[$\overline{B} \subseteq \overline{A} \Rightarrow A \subseteq B$]
    \begin{eqnarray*}
      \neg(A \subseteq B)
        & \Rightarrow & \exists w~[w \in A \wedge w \notin B]\\\pause
        & \Rightarrow & \exists w~[w \notin \overline{A} \wedge w \in \overline{B}]\\\pause
        & \Rightarrow & \neg(\overline{B} \subseteq \overline{A})
    \end{eqnarray*}
  \end{proof}
\end{frame}

\subsection{Inclusion-Exclusion}

\begin{frame}
  \frametitle{Principle of Inclusion-Exclusion}

  \begin{itemize}
    \item $|A \cup B| = |A| + |B| - |A \cap B|$

    \pause
    \item $|A \cup B \cup C| = |A| + |B| + |C|
      - (|A \cap B| + |A \cap C| + |B \cap C|)
      + |A \cap B \cap C|$
  \end{itemize}

  \pause
  \begin{theorem}
    \begin{eqnarray*}
      |A_1 \cup A_2 \cup \cdots \cup A_n| & = & \sum_i{|A_i|}
          - \sum_{i,j}{|A_i \cap A_j|}\\
      & & + \sum_{i,j,k}{|A_i \cap A_j \cap A_k|}\\
      & & \cdots + -1^{n-1} {|A_i \cap A_j \cap \cdots \cap A_n|}
    \end{eqnarray*}
  \end{theorem}
\end{frame}

\begin{frame}
  \frametitle{Inclusion-Exclusion Example}

  \begin{example}[sieve of Eratosthenes]
    \begin{itemize}
      \item a method to identify prime numbers
    \end{itemize}

    \pause
    \begin{tiny}
    \begin{tabular}{ccccccccccccccccccccccc}
  2 &  3 &  4 &  5 &  6 &  7 &  8 &  9 & 10 & 11 & 12 & 13 & 14 & 15 & 16 & 17\\
 18 & 19 & 20 & 21 & 22 & 23 & 24 & 25 & 26 & 27 & 28 & 29 & 30\\
\\ \pause
  2 &  3 &    &  5 &    &  7 &    &  9 &    & 11 &    & 13 &    & 15 &    & 17\\
    & 19 &    & 21 &    & 23 &    & 25 &    & 27 &    & 29 & \\
\\  \pause
  2 &  3 &    &  5 &    &  7 &    &    &    & 11 &    & 13 &    &    &    & 17\\
    & 19 &    &    &    & 23 &    & 25 &    &    &    & 29 & \\
\\  \pause
  2 &  3 &    &  5 &    &  7 &    &    &    & 11 &    & 13 &    &    &    & 17\\
    & 19 &    &    &    & 23 &    &    &    &    &    & 29 & \\
    \end{tabular}
    \end{tiny}
  \end{example}
\end{frame}

\begin{frame}
  \frametitle{Inclusion-Exclusion Example}

  \begin{example}[sieve of Eratosthenes]
    \begin{itemize}
      \item number of primes between 1 and 100
      \medskip

      \pause
      \item numbers that are not divisible by 2, 3, 5 and 7
      \begin{itemize}
        \item $A_2$: set of numbers divisible by 2
        \item $A_3$: set of numbers divisible by 3
        \item $A_5$: set of numbers divisible by 5
        \item $A_7$: set of numbers divisible by 7
      \end{itemize}

      \pause
      \item $|A_2 \cup A_3 \cup A_5 \cup A_7|$
    \end{itemize}
  \end{example}
\end{frame}

\begin{frame}
  \frametitle{Inclusion-Exclusion Example}

  \begin{example}[sieve of Eratosthenes]
    \begin{columns}[t]
      \column{.5\textwidth}
      \begin{itemize}
        \item $|A_2| = \left\lfloor 100/2 \right\rfloor = 50$
        \item $|A_3| = \left\lfloor 100/3 \right\rfloor = 33$
        \item $|A_5| = \left\lfloor 100/5 \right\rfloor = 20$
        \item $|A_7| = \left\lfloor 100/7 \right\rfloor = 14$
      \end{itemize}

      \pause
      \column{.5\textwidth}
      \begin{itemize}
        \item $|A_2 \cap A_3| = \left\lfloor 100/6  \right\rfloor = 16$
        \item $|A_2 \cap A_5| = \left\lfloor 100/10 \right\rfloor = 10$
        \item $|A_2 \cap A_7| = \left\lfloor 100/14 \right\rfloor = 7$
        \item $|A_3 \cap A_5| = \left\lfloor 100/15 \right\rfloor = 6$
        \item $|A_3 \cap A_7| = \left\lfloor 100/21 \right\rfloor = 4$
        \item $|A_5 \cap A_7| = \left\lfloor 100/35 \right\rfloor = 2$
      \end{itemize}
    \end{columns}
  \end{example}
\end{frame}

\begin{frame}
  \frametitle{Inclusion-Exclusion Example}

  \begin{example}[sieve of Eratosthenes]
    \begin{itemize}
      \item $|A_2 \cap A_3 \cap A_5| = \left\lfloor 100/30  \right\rfloor = 3$
      \item $|A_2 \cap A_3 \cap A_7| = \left\lfloor 100/42  \right\rfloor = 2$
      \item $|A_2 \cap A_5 \cap A_7| = \left\lfloor 100/70  \right\rfloor = 1$
      \item $|A_3 \cap A_5 \cap A_7| = \left\lfloor 100/105 \right\rfloor = 0$
    \end{itemize}

    \pause
    \begin{itemize}
      \item $|A_2 \cap A_3 \cap A_5 \cap A_7| = \left\lfloor 100/210 \right\rfloor = 0$
    \end{itemize}
  \end{example}
\end{frame}

\begin{frame}
  \frametitle{Inclusion-Exclusion Example}

  \begin{example}[sieve of Eratosthenes]
    \begin{eqnarray*}
      |A_2 \cup A_3 \cup A_5 \cup A_7| & = & (50 + 33 + 20 +14)\\
                                       & - & (16 + 10 + 7 + 6 + 4 + 2)\\
                                       & + & (3 + 2 + 1 + 0)\\
                                       & - & (0)\\
                                       & = & 78
    \end{eqnarray*}

    \pause
    \begin{itemize}
      \item number of primes: $(100 - 78) + 4 - 1 = 25$
    \end{itemize}
  \end{example}
\end{frame}

\subsection{Infinite Sets}

\begin{frame}
  \frametitle{Subset Cardinality}

  \begin{itemize}
    \item $A \subset B \Rightarrow |A| < |B|$

    \pause
    \item this doesn't necessarily hold for infinite sets
  \end{itemize}

  \begin{example}
$\mathbb{Z}^+ \subset \mathbb{N}$

\smallskip
but

\smallskip
$|\mathbb{Z}^+| = |\mathbb{N}|$
  \end{example}

  \pause
  \begin{itemize}
    \item how can we compare the cardinalities of infinite sets?
  \end{itemize}
\end{frame}

\begin{frame}
  \frametitle{Infinite Sets}

  \begin{itemize}
    \item in order to compare the cardinalities two sets,\\
      let's pair off the elements of the sets
    \item if every element can be paired,\\
      then they have the same cardinality
  \end{itemize}

  \pause
  \begin{block}{$|\mathbb{Z}^+| = |\mathbb{N}|$}
    \begin{tabular}{lcccccccc}
      $\mathbb{Z}^+$ & $1$ & $2$ & $3$ & $4$ & $5$ & $6$ & $7$ & \ldots\\
      $\mathbb{N}$   & $0$ & $1$ & $2$ & $3$ & $4$ & $5$ & $6$ & \ldots\\
    \end{tabular}
  \end{block}
\end{frame}

\begin{frame}
  \frametitle{Infinite Sets}

  \begin{footnotesize}
  \begin{block}{$|\mathbb{Z}^+| = |\mathbb{Q}|$}
    \[
    \begin{array}{l|ccccccc|}
             &  $1$   &  $2$   &  $3$   &  $4$   &  $5$   & \ldots\\\hline
      $1$    & $1/1$  & $2/1$  & $3/1$  & $4/1$  & $5/1$  & \ldots\\
      $2$    & $1/2$  &        & $3/2$  &        & $5/2$  & \ldots\\
      $3$    & $1/3$  & $2/3$  &        & $4/3$  & $5/3$  & \ldots\\
      $4$    & $1/4$  &        & $3/4$  &        & $5/4$  & \ldots\\
      $5$    & $1/5$  & $2/5$  & $3/5$  & $4/5$  &        & \ldots\\
      \vdots & \vdots & \vdots & \vdots & \vdots & \vdots &
    \end{array}
    \]

    \begin{itemize}
      \item pair off row-wise:\\
      \begin{tabular}{llllll}
        $1/1 \rightarrow 1$ & $2/1 \rightarrow 2$ & $3/1 \rightarrow 3$
                            & $4/1 \rightarrow 4$ & $5/1 \rightarrow 5$ & \ldots\\
      \end{tabular}

      \pause
      \item pair off diagonally:\\
      \begin{tabular}{llllll}
        $1/1 \rightarrow 1$ & $2/1 \rightarrow 2$ & $1/2 \rightarrow 3$
                            & $3/1 \rightarrow 4$ & $1/3 \rightarrow 5$ & \\
        $4/1 \rightarrow 6$ & $3/2 \rightarrow 7$ & $2/3 \rightarrow 8$
                            & $1/4 \rightarrow 9$ & $5/1 \rightarrow 10$ & \ldots
      \end{tabular}
    \end{itemize}
  \end{block}
  \end{footnotesize}
\end{frame}

\begin{frame}
  \frametitle{Infinite Sets}

  \begin{block}{$|\mathbb{Z}^+| \stackrel{?}{=} |\mathbb{R}|$}
    \begin{itemize}
      \item consider the set $\{x | x \in \mathbb{R}, 0 < x \leq 1\}$
      \item no element is represented by an expansion that terminates:\\
        $0.4999...$ instead of $0.5$
    \end{itemize}

    \vspace{-2em}
    \begin{columns}[t]
      \column{.42\textwidth}
      \[
      \begin{array}{lcl}
        0.a_{11}a_{12}a_{13}a_{14}\cdots & \rightarrow & 1\\
        0.a_{21}a_{22}a_{23}a_{24}\cdots & \rightarrow & 2\\
        0.a_{31}a_{32}a_{33}a_{34}\cdots & \rightarrow & 3\\
        \vdots                           &             &\\
        0.a_{n1}a_{n2}a_{n3}a_{n4}\cdots & \rightarrow & n\\
        \vdots                           &             &
      \end{array}
      \]

      \pause
      \column{.55\textwidth}
      \begin{itemize}
        \item consider the number\\
          $0.b_1b_2b_3\cdots$ where
        \begin{equation*}
          b_k = \left\{
            \begin{array}{ll}
              3 & if~a_{kk} \neq 3\\
              7 & if~a_{kk} = 3
            \end{array}\right.
        \end{equation*}

        \pause
        \item $\forall k \in \mathbb{Z}^+~r \neq r_k$
        \item \emph{Cantor's Diagonal Construction}
      \end{itemize}
    \end{columns}
  \end{block}
\end{frame}

\begin{frame}
  \frametitle{Infinite Sets}

  \begin{itemize}
    \item result: $|\mathbb{N}| < |\mathbb{R}|$

    \pause
    \medskip
    \item let $C$ be the set of all possible computer programs
    \item let $P$ be the set of all possible problems
    \item $|C| = |\mathbb{N}|$
    \item $|P| = |\mathbb{R}|$

    \pause
    \medskip
    \item there are problems which cannot be solved using computers
  \end{itemize}
\end{frame}

\subsection*{References}

\begin{frame}
  \frametitle{References}

  \begin{block}{Required Reading: Grimaldi}
    \begin{itemize}
      \item Chapter 3: Set Theory
      \begin{itemize}
        \item 3.1. \alert{Sets and Subsets}
        \item 3.2. \alert{Set Operations and the Laws of Set Theory}
      \end{itemize}

      \item Chapter 8: The Principle of Inclusion and Exclusion
      \begin{itemize}
        \item 8.1. \alert{The Principle of Inclusion and Exclusion}
      \end{itemize}

      \item Appendix 3: \alert{Countable and Uncountable Sets}
    \end{itemize}
  \end{block}

  \begin{block}{Supplementary Reading: O'Donnell, Hall, Page}
    \begin{itemize}
      \item Chapter 8: Set Theory
    \end{itemize}
  \end{block}
\end{frame}

\section{Induction}

\subsection{Introduction}

\begin{frame}
  \frametitle{Induction}

  \begin{definition}
    $S(n)$: a predicate defined on $n \in \mathbb{Z}^+$

    \pause
    \medskip
    $S(n_0) \wedge (\forall k \geq n_0~[S(k) \Rightarrow S(k+1)])
      \Rightarrow \forall n \geq n_0~S(n)$
  \end{definition}

  \pause
  \medskip
  \begin{itemize}
    \item $S(n_0)$: \emph{base step}
    \item $\forall k \geq n_0~[S(k) \Rightarrow S(k+1)]$: \emph{induction step}
  \end{itemize}
\end{frame}

\begin{frame}
  \frametitle{Induction}

  \begin{center}
    \pgfuseimage{induction}
  \end{center}
\end{frame}

\begin{frame}
  \frametitle{Induction Example}

  \begin{theorem}
    $\forall n \in \mathbb{Z}^+~[1+3+5+\cdots+(2n-1)=n^2]$
  \end{theorem}

  \pause
  \begin{proof}
    \begin{itemize}
      \item $n=1$: $1=1^2$

      \pause
      \item $n=k$: assume $1+3+5+\cdots+(2k-1)=k^2$

      \pause
      \item $n=k+1$:
      \begin{eqnarray*}
        &   & 1+3+5+\cdots+(2k-1)+(2k+1)\\\pause
        & = & k^2+2k+1\\\pause
        & = & (k+1)^2
      \end{eqnarray*}
    \end{itemize}
  \end{proof}
\end{frame}

\begin{frame}
  \frametitle{Induction Example}

  \begin{theorem}
    $\forall n \in \mathbb{Z}^+, n \geq 4~[2^n < n!]$
  \end{theorem}

  \pause
  \begin{proof}
    \begin{itemize}
      \item $n=4$: $2^4=16<24=4!$

      \pause
      \item $n=k$: assume $2^k < k!$

      \pause
      \item $n=k+1$:\\
        $2^{k+1} = 2 \cdot 2^k < 2 \cdot k! < (k+1) \cdot k! = (k+1)!$
    \end{itemize}
  \end{proof}
\end{frame}

\begin{frame}
  \frametitle{Induction Example}

  \begin{theorem}
    $\forall n \in \mathbb{Z}^+, n \geq 14~\exists i,j \in \mathbb{N}~[n=3i+8j]$
  \end{theorem}

  \pause
  \begin{proof}
    \begin{itemize}
      \item $n=14$: $14=3 \cdot 2 + 8 \cdot 1$

      \pause
      \item $n=k$: assume $k=3i+8j$

      \pause
      \item $n=k+1$:
      \begin{itemize}
        \item $k=3i+8j, j>0 \Rightarrow k+1=k-8+3 \cdot 3$\\
          $\Rightarrow k+1=3(i+3)+8(j-1)$
        \item $k=3i+8j, j=0, i \geq 5 \Rightarrow k+1=k-5 \cdot 3+2 \cdot 8$\\
          $\Rightarrow k+1=3(i-5)+8(j+2)$
      \end{itemize}
    \end{itemize}
  \end{proof}
\end{frame}

\subsection{Strong Induction}

\begin{frame}
  \frametitle{Strong Induction}

  \begin{definition}
    $S(n_0) \wedge
      (\forall k \geq n_0~[(\forall i \leq k~S(i)) \Rightarrow S(k+1)])
      \Rightarrow \forall n \geq n_0~S(n)$
  \end{definition}
\end{frame}

\begin{frame}
  \frametitle{Strong Induction Example}

  \begin{theorem}
    $\forall n \in \mathbb{Z}^+, n \geq 2$\\
      n can be written as the product of prime numbers.
  \end{theorem}

  \pause
  \begin{proof}
    \begin{itemize}
      \item $n=2$: $2=2$

      \pause
      \item assume that the theorem is true for $\forall i \leq k$

      \pause
      \item $n=k+1$:
      \begin{enumerate}
        \item if prime: $n=n$

        \pause
        \item if not prime: $n=u \cdot v$\\
          $u < k \wedge v < k \Rightarrow$ both $u$ and $v$ can be written\\
          as the product of prime numbers
      \end{enumerate}
    \end{itemize}
  \end{proof}
\end{frame}

\begin{frame}
  \frametitle{Strong Induction Example}

  \begin{theorem}
    $\forall n \in \mathbb{Z}^+, n \geq 14~\exists i,j \in \mathbb{N}~[n=3i+8j]$
  \end{theorem}

  \pause
  \begin{proof}
    \begin{itemize}
      \item $n=14$: $14=3 \cdot 2 + 8 \cdot 1$\\
        $n=15$: $15=3 \cdot 5 + 8 \cdot 0$\\
        $n=16$: $16=3 \cdot 0 + 8 \cdot 2$

      \pause
      \item $n \leq k$: assume $k=3i+8j$

      \pause
      \item $n=k+1$: $k+1=(k-2)+3$
    \end{itemize}
  \end{proof}
\end{frame}

\begin{frame}
  \frametitle{Flawed Induction Example}

  \begin{theorem}
    $\forall n \in \mathbb{Z}^+~[1+2+3+\cdots+n=\frac{n^2+n+2}{2}]$
  \end{theorem}

  \pause
  \begin{block}{invalid base step}
    \begin{itemize}
      \item $n=k$: assume $1+2+3+\cdots+k=\frac{k^2+k+2}{2}$

      \pause
      \item $n=k+1$:
      \begin{eqnarray*}
        &   & 1+2+3+\cdots+k+(k+1)\\\pause
        & = & \frac{k^2+k+2}{2}+k+1
          =   \frac{k^2+k+2}{2}+\frac{2k+2}{2}\\\pause
        & = & \frac{k^2+3k+4}{2}
          =   \frac{(k+1)^2+(k+1)+2}{2}
      \end{eqnarray*}

      \pause
      \item $n=1$: $1 \neq \frac{1^2+1+2}{2}=2$
    \end{itemize}
  \end{block}
\end{frame}

\begin{frame}
  \frametitle{Flawed Induction Example}

  \begin{center}
    \pgfuseimage{error1}
  \end{center}
\end{frame}

\begin{frame}
  \frametitle{Flawed Induction Example}

  \begin{theorem}
    All horses are of the same color.\\
    \pause
    \bigskip
    $A(n)$: All horses in sets of $n$ horses are of the same color.

    \medskip
    $\forall n \in \mathbb{N^+}~A(n)$
  \end{theorem}
\end{frame}

\begin{frame}
  \frametitle{Flawed Induction Example}

  \begin{block}{invalid induction step}
    \begin{itemize}
      \item $n=1$: $A(1)$\\
        All horses in sets of $1$~horse are of the same color.

      \pause
      \medskip
      \item $n=k$: assume $A(k)$ is true\\
        All horses in sets of $k$~horses are of the same color.
      \pause
      \medskip
      \item $A(k+1)=\{a_1,a_2,\dots,a_k\} \cup \{a_2,a_3,\dots,a_{k+1}\}$
      \begin{itemize}
        \item All horses in set $\{a_1,a_2,\dots,a_k\}$
          are of the same color ($a_2$).
        \item All horses in set $\{a_2,a_3,\dots,a_{k+1}\}$
          are of the same color ($a_2$).
      \end{itemize}
    \end{itemize}
  \end{block}
\end{frame}

\begin{frame}
  \frametitle{Flawed Induction Examples}

  \begin{center}
    \pgfuseimage{error2}
  \end{center}
\end{frame}

\section*{References}

\begin{frame}
  \frametitle{References}

  \begin{block}{Required Reading: Grimaldi}
    \begin{itemize}
      \item Chapter 4: Properties of Integers: Mathematical Induction
      \begin{itemize}
        \item 4.1. \alert{The Well-Ordering Principle: Mathematical Induction}
      \end{itemize}
    \end{itemize}
  \end{block}

  \begin{block}{Supplementary Reading: O'Donnell, Hall, Page}
    \begin{itemize}
      \item Chapter 4: Induction
    \end{itemize}
  \end{block}
\end{frame}

\end{document}
