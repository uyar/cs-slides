% Copyright (c) 2001-2016
%       H. Turgut Uyar <uyar@itu.edu.tr>
%       Ayşegül Gençata Yayımlı <gencata@itu.edu.tr>
%       Emre Harmancı <harmanci@itu.edu.tr>
%
% This work is licensed under a "Creative Commons
% Attribution-NonCommercial-ShareAlike 4.0 International License".
% For more information, please visit:
% https://creativecommons.org/licenses/by-nc-sa/4.0/

\documentclass[dvipsnames]{beamer}

\usepackage{ae}
\usepackage[scaled=0.88]{beramono}
\usepackage[T1]{fontenc}
\usepackage[utf8]{inputenc}
\setbeamersize{text margin left=2em, text margin right=2em}

\mode<presentation>
{
  \usetheme{Rochester}
  \setbeamercovered{transparent}
}

\title{Discrete Mathematics}
\subtitle{Graphs}

\author{H. Turgut Uyar \and Ayşegül Gençata Yayımlı \and Emre Harmancı}
\date{2001-2016}

\AtBeginSubsection[]
{
  \begin{frame}<beamer>
    \frametitle{Topics}
    \tableofcontents[currentsection,currentsubsection]
  \end{frame}
}

\pgfdeclareimage[height=1cm]{license}{../license}

\pgfdeclareimage[width=6cm]{plain}{plain}
\pgfdeclareimage[width=4cm]{directed}{directed}
\pgfdeclareimage[width=6cm]{multi}{multi}
\pgfdeclareimage[width=4cm]{matrix}{matrix}
\pgfdeclareimage[height=4cm]{isomorphicf}{isomorphicf}
\pgfdeclareimage[height=4cm]{isomorphict}{isomorphict}
\pgfdeclareimage[height=3.5cm]{petersen1}{petersen1}
\pgfdeclareimage[height=3.5cm]{petersen2}{petersen2}
\pgfdeclareimage[height=4cm]{homeomorphict}{homeomorphict}
\pgfdeclareimage[height=3cm]{regular}{regular}
\pgfdeclareimage[width=4cm]{k4}{k4}
\pgfdeclareimage[width=4cm]{k5}{k5}
\pgfdeclareimage[height=2cm]{bipartite}{bipartite}
\pgfdeclareimage[width=4cm]{k23}{k23}
\pgfdeclareimage[width=4cm]{k33}{k33}
\pgfdeclareimage[width=6cm]{disconnected}{disconnected}
\pgfdeclareimage[width=5cm]{distance}{distance}
\pgfdeclareimage[width=5cm]{cutpoint}{cutpoint}
\pgfdeclareimage[width=3cm]{weak}{weak}
\pgfdeclareimage[width=3cm]{unilateral}{unilateral}
\pgfdeclareimage[width=3cm]{strong}{strong}
\pgfdeclareimage{konigsberg}{konigsberg}
\pgfdeclareimage[width=3cm]{envelope}{envelope}
\pgfdeclareimage{konigsgraph}{konigsgraph}
\pgfdeclareimage[width=3.5cm]{euler}{euler}
\pgfdeclareimage[width=3.5cm]{hamilton}{hamilton}
\pgfdeclareimage[width=4.5cm]{k4planar}{k4planar}
\pgfdeclareimage[width=6cm]{region}{region}
\pgfdeclareimage[width=3cm]{eulerproof1}{eulerproof1}
\pgfdeclareimage[width=3cm]{eulerproof2}{eulerproof2}
\pgfdeclareimage{tetrahedron}{tetrahedron}
\pgfdeclareimage{planartetra}{planartetra}
\pgfdeclareimage[width=5cm]{hexahedron}{hexahedron}
\pgfdeclareimage{planarhexa}{planarhexa}
\pgfdeclareimage{octahedron}{octahedron}
\pgfdeclareimage{planarocta}{planarocta}
\pgfdeclareimage{dodecahedron}{dodecahedron}
\pgfdeclareimage{planardodeca}{planardodeca}
\pgfdeclareimage{icosahedron}{icosahedron}
\pgfdeclareimage[width=5cm]{warshall}{warshall}
\pgfdeclareimage[width=5cm]{warshall1}{warshall1}
\pgfdeclareimage[width=5cm]{warshall2}{warshall2}
\pgfdeclareimage[width=5cm]{warshall3}{warshall3}
\pgfdeclareimage[width=4cm]{fivesubstances}{fivesubstances}
\pgfdeclareimage[height=4cm]{coloring1}{coloring1}
\pgfdeclareimage[height=4cm]{coloring2}{coloring2}
\pgfdeclareimage[height=4cm]{coloring3}{coloring3}
\pgfdeclareimage[height=4cm]{coloring4}{coloring4}
\pgfdeclareimage[height=4cm]{coloring5}{coloring5}
\pgfdeclareimage[height=4cm]{coloring6}{coloring6}
\pgfdeclareimage[height=4cm]{coloring7}{coloring7}
\pgfdeclareimage[height=4cm]{herschel}{herschel}
\pgfdeclareimage[width=4cm]{sudoku}{sudoku}
\pgfdeclareimage[width=6cm]{dijkstra}{dijkstra}

\begin{document}

\begin{frame}
  \titlepage
\end{frame}

\begin{frame}
  \frametitle{License}

  \pgfuseimage{license}\hfill
  \copyright~2001-2016 T. Uyar, A. Yayımlı, E. Harmancı

  \vfill
  \begin{footnotesize}
    You are free to:
    \begin{itemize}
      \itemsep0em
      \item Share -- copy and redistribute the material in any medium or format
      \item Adapt -- remix, transform, and build upon the material
    \end{itemize}

    Under the following terms:
    \begin{itemize}
      \itemsep0em
      \item Attribution -- You must give appropriate credit, provide a link to
        the license, and indicate if changes were made.

      \item NonCommercial -- You may not use the material for commercial
        purposes.

      \item ShareAlike -- If you remix, transform, or build upon the material,
        you must distribute your contributions under the same license as the
        original.
    \end{itemize}
  \end{footnotesize}

  \begin{small}
    For more information:\\
    \url{https://creativecommons.org/licenses/by-nc-sa/4.0/}

    \smallskip
    Read the full license:\\
    \url{https://creativecommons.org/licenses/by-nc-sa/4.0/legalcode}
  \end{small}
\end{frame}

\begin{frame}
  \frametitle{Topics}
  \tableofcontents
\end{frame}

\section{Graphs}

\subsection{Introduction}

\begin{frame}
  \frametitle{Graphs}

  \begin{definition}
    \alert{graph}: $G=(V,E)$

    \begin{itemize}
      \item $V$: \alert{node} (or \emph{vertex}) set
      \item $E \subseteq V \times V$: \alert{edge} set
    \end{itemize}
  \end{definition}

  \pause
  \begin{itemize}
    \item $e=(v_1,v_2) \in E$:
    \smallskip
    \item $v_1$ and $v_2$ are \emph{endnodes} of $e$
    \item $e$ is \emph{incident} to $v_1$ and $v_2$
    \item $v_1$ and $v_2$ are \emph{adjacent}
  \end{itemize}
\end{frame}

\begin{frame}
  \frametitle{Graph Example}

  \begin{columns}
    \column{.58\textwidth}
    \begin{center}
      \pgfuseimage{plain}
    \end{center}

    \column{.4\textwidth}
    $\begin{array}{lcl}
      V & = & \{a,b,c,d,e,f\}\\
      E & = & \{(a,b),(a,c),\\
        &   & ~(a,d),(a,e),\\
        &   & ~(a,f),(b,c),\\
        &   & ~(d,e),(e,f)\}
    \end{array}$
  \end{columns}
\end{frame}

\begin{frame}
  \frametitle{Directed Graphs}

  \begin{definition}
    \alert{directed graph} (\emph{digraph}): $D=(V,A)$

    \begin{itemize}
      \item $V$: node set
      \item $A \subseteq V \times V$: \alert{arc} set
    \end{itemize}
  \end{definition}

  \begin{itemize}
    \item $a=(v_1,v_2) \in A$:
    \smallskip
    \item $v_1$: \emph{origin} node of $a$
    \item $v_2$: \emph{terminating} node of $a$
  \end{itemize}
\end{frame}

\begin{frame}
  \frametitle{Directed Graph Example}

  \begin{center}
    \pgfuseimage{directed}
  \end{center}
\end{frame}

\begin{frame}
  \frametitle{Weighted Graphs}

  \begin{itemize}
    \item weighted graph: labels assigned to edges

    \medskip
    \item weight
    \item length, distance
    \item cost, delay
    \item probability
    \item $\ldots$
  \end{itemize}

%   TODO: add example for weighted graph
\end{frame}

\begin{frame}
  \frametitle{Multigraphs}

  \begin{itemize}
    \item \alert{parallel edges}: edges between same node pair
    \item \alert{loop}: edge starting and ending in same node

    \bigskip
    \item \alert{plain} graph: no loops, no parallel edges
    \item \alert{multigraph}: a graph which is not plain
  \end{itemize}
\end{frame}

\begin{frame}
  \frametitle{Multigraph Example}

  \begin{columns}
    \column{.55\textwidth}
    \begin{center}
      \pgfuseimage{multi}
    \end{center}

    \column{.45\textwidth}
    \begin{itemize}
      \item parallel edges: $(a,b)$
      \item loop: $(e,e)$
    \end{itemize}
  \end{columns}
\end{frame}

\begin{frame}
  \frametitle{Subgraph}

  \begin{definition}
    $G'=(V',E')$ is a \alert{subgraph} of $G=(V,E)$:\\
      $V' \subseteq V \wedge~ E' \subseteq E \wedge~
      \forall (v_1,v_2) \in E'~[v_1,v_2 \in V']$
  \end{definition}
\end{frame}

\begin{frame}
  \frametitle{Incidence Matrix}

  \begin{itemize}
    \item rows: nodes, columns: edges
    \item 1 if edge incident on node, 0 otherwise
  \end{itemize}

  \begin{exampleblock}{example}
    \begin{columns}
      \column{.38\textwidth}
      \begin{center}
        \pgfuseimage{matrix}
      \end{center}

      \column{.58\textwidth}
      \[
        \begin{array}{c|cccccccc}
              & e_1 & e_2 & e_3 & e_4 & e_5 & e_6 & e_7 & e_8\\\hline
          v_1 & 1 & 1 & 1 & 0 & 1 & 0 & 0 & 0\\
          v_2 & 1 & 0 & 0 & 1 & 0 & 0 & 0 & 0\\
          v_3 & 0 & 0 & 1 & 1 & 0 & 0 & 1 & 1\\
          v_4 & 0 & 0 & 0 & 0 & 1 & 1 & 0 & 1\\
          v_5 & 0 & 1 & 0 & 0 & 0 & 1 & 1 & 0
        \end{array}
      \]
    \end{columns}
  \end{exampleblock}
\end{frame}

\begin{frame}
  \frametitle{Adjacency Matrix}

  \begin{itemize}
    \item rows: nodes, columns: nodes
    \item 1 if nodes are adjacent, 0 otherwise
  \end{itemize}

  \begin{exampleblock}{example}
    \begin{columns}
      \column{.38\textwidth}
      \begin{center}
        \pgfuseimage{matrix}
      \end{center}

      \column{.58\textwidth}
      \[
        \begin{array}{c|ccccc}
                & v_1 & v_2 & v_3 & v_4 & v_5\\\hline
            v_1 & 0 & 1 & 1 & 1 & 1\\
            v_2 & 1 & 0 & 1 & 0 & 0\\
            v_3 & 1 & 1 & 0 & 1 & 1\\
            v_4 & 1 & 0 & 1 & 0 & 1\\
            v_5 & 1 & 0 & 1 & 1 & 0
        \end{array}
      \]
    \end{columns}
  \end{exampleblock}
\end{frame}

\begin{frame}
  \frametitle{Adjacency Matrix}

  \begin{itemize}
    \item multigraph: number of edges between nodes
  \end{itemize}

  \begin{exampleblock}{example}
    \begin{columns}
    \column{.5\textwidth}
    \begin{center}
      \pgfuseimage{directed}
    \end{center}

    \column{.5\textwidth}
      \[
        \begin{array}{c|cccc}
              & a & b & c & d\\\hline
            a & 0 & 0 & 0 & 1\\
            b & 2 & 1 & 1 & 0\\
            c & 0 & 0 & 0 & 0\\
            d & 0 & 1 & 1 & 0
        \end{array}
      \]
    \end{columns}
  \end{exampleblock}
\end{frame}

\begin{frame}
  \frametitle{Adjacency Matrix}

  \begin{itemize}
    \item weighted graph: weight of edge
  \end{itemize}

%   TODO: add example for weighted graph adjacency matrix
\end{frame}

\begin{frame}
  \frametitle{Degree}

  \begin{itemize}
    \item \alert{degree} of node: number of incident edges
  \end{itemize}

  \pause
  \begin{theorem}
    $d_i$: degree of $v_i$

    \[ |E| = \frac{\sum_i d_i}{2} \]
  \end{theorem}
\end{frame}

\begin{frame}
  \frametitle{Degree Example}

  \begin{columns}
    \column{.58\textwidth}
    \begin{center}
      \pgfuseimage{plain}
    \end{center}

    \column{.4\textwidth}
    $\begin{array}{rcr}
    d_a & = & 5\\
    d_b & = & 2\\
    d_c & = & 2\\
    d_d & = & 2\\
    d_e & = & 3\\
    d_f & = & 2\\\hline
        &   & 16\\
    \medskip
    |E| & = & 8
    \end{array}$
  \end{columns}
\end{frame}

\begin{frame}
  \frametitle{Multigraph Degree Example}

  \begin{columns}
    \column{.58\textwidth}
    \begin{center}
      \pgfuseimage{multi}
    \end{center}

    \column{.4\textwidth}
    $\begin{array}{rcr}
    d_a & = & 6\\
    d_b & = & 3\\
    d_c & = & 2\\
    d_d & = & 2\\
    d_e & = & 5\\
    d_f & = & 2\\\hline
        &   & 20\\
    \medskip
    |E| & = & 10
    \end{array}$
  \end{columns}
\end{frame}

\begin{frame}
  \frametitle{Degree in Directed Graphs}

  \begin{itemize}
    \item \emph{in-degree}: ${d_v}^i$
    \item \emph{out-degree}: ${d_v}^o$

    \pause
    \medskip
    \item node with in-degree 0: \emph{source}
    \item node with out-degree  0: \emph{sink}

    \pause
    \bigskip
    \item $\sum_{v \in V} {d_v}^i = \sum_{v \in V} {d_v}^o = |A|$
  \end{itemize}
\end{frame}

\begin{frame}
  \frametitle{Degree}

  \begin{theorem}
    In an undirected graph, there is an even number of nodes\\
    which have an odd degree.
  \end{theorem}

  \pause
  \begin{proof}
    \begin{itemize}
      \item $t_i$: number of nodes of degree $i$

      \pause
$2|E| = \sum_i d_i = 1t_1 + 2t_2 + 3t_3 + 4t_4 + 5t_5 + \dots$

\pause
$2|E| - 2t_2 - 4t_4 - \dots = t_1 + t_3 + t_5 + \dots + 2t_3 + 4t_5 + \dots$

\pause
$2|E| - 2t_2 - 4t_4 - \dots - 2t_3 - 4t_5 - \dots = t_1 + t_3 + t_5 + \dots$

      \pause
      \item left-hand side even $\Rightarrow$ right-hand side even
    \end{itemize}
  \end{proof}
\end{frame}

\begin{frame}
  \frametitle{Isomorphism}

  \begin{definition}
    $G=(V,E)$ and $G^\star=(V^\star,E^\star)$ are \alert{isomorphic}:\\
      $\exists f: V \rightarrow V^\star~[(u,v) \in E \Rightarrow (f(u),f(v)) \in E^\star] \wedge
      f$ is bijective
  \end{definition}

  \pause
  \begin{itemize}
    \item $G$ and $G^\star$ can be drawn the same way
  \end{itemize}
\end{frame}

\begin{frame}
  \frametitle{Isomorphism Example}

  \begin{columns}
    \column{.4\textwidth}
    \begin{center}
      \pgfuseimage{isomorphicf}
    \end{center}

    \column{.6\textwidth}
    \begin{center}
      \pgfuseimage{isomorphict}
    \end{center}
  \end{columns}

  \pause
  \bigskip
  \begin{itemize}
    \item $f = (a \mapsto d, b \mapsto e, c \mapsto b, d \mapsto c,
      e \mapsto a)$
  \end{itemize}
\end{frame}

\begin{frame}
  \frametitle{Petersen Graph}

  \begin{columns}
    \column{.4\textwidth}
    \begin{center}
      \pgfuseimage{petersen1}
    \end{center}

    \column{.6\textwidth}
    \begin{center}
      \pgfuseimage{petersen2}
    \end{center}
  \end{columns}

  \bigskip
  \begin{itemize}
    \item $f = (a \mapsto q, b \mapsto v, c \mapsto u, d \mapsto y,
        e \mapsto r,$\\
      $~~~~~~~f \mapsto w, g \mapsto x, h \mapsto t, i \mapsto z,
        j \mapsto s)$
  \end{itemize}
\end{frame}

\begin{frame}
  \frametitle{Homeomorphism}

  \begin{definition}
    $G=(V,E)$ and $G^\star=(V^\star,E^\star)$ are \alert{homeomorphic}:
    \begin{itemize}
      \item $G$ and $G^\star$ isomorphic, except that\\
      \item some edges in $E^\star$ are divided with additional nodes
    \end{itemize}
  \end{definition}
\end{frame}

\begin{frame}
  \frametitle{Homeomorphism Example}

  \begin{columns}
    \column{.5\textwidth}
    \begin{center}
      \pgfuseimage{isomorphict}
    \end{center}

    \column{.5\textwidth}
    \begin{center}
      \pgfuseimage{homeomorphict}
    \end{center}
  \end{columns}
\end{frame}

\begin{frame}
  \frametitle{Regular Graphs}

  \begin{itemize}
    \item \alert{regular} graph: all nodes have the same degree
    \item $n$-regular: all nodes have degree $n$
  \end{itemize}

  \begin{exampleblock}{examples}
    \begin{center}
      \pgfuseimage{regular}
    \end{center}
  \end{exampleblock}
\end{frame}

\begin{frame}
  \frametitle{Completely Connected Graphs}

  \begin{itemize}
    \item $G=(V,E)$ is \alert{completely connected}:\\
      $\forall v_1,v_2 \in V~(v_1,v_2) \in E$
    \item every pair of nodes are adjacent

    \medskip
    \item $K_n$: completely connected graph with $n$ nodes
  \end{itemize}
\end{frame}

\begin{frame}
  \frametitle{Completely Connected Graph Examples}

  \begin{columns}[t]
    \column{.5\textwidth}
    \begin{center}
      $K_4$

      \bigskip
      \pgfuseimage{k4}
    \end{center}

    \column{.5\textwidth}
    \begin{center}
      $K_5$

      \bigskip
      \pgfuseimage{k5}
    \end{center}
  \end{columns}
\end{frame}

\begin{frame}
  \frametitle{Bipartite Graphs}

  \begin{itemize}
    \item $G=(V,E)$ is \alert{bipartite}:
      $V_1 \cup V_2 = V$, $V_1 \cap V_2 = \emptyset$\\
      $\forall (v_1,v_2) \in E~[v_1 \in V_1 \wedge v_2 \in V_2$]

  \begin{exampleblock}{example}
    \begin{center}
      \pgfuseimage{bipartite}
    \end{center}
  \end{exampleblock}

    \medskip
    \item \emph{complete bipartite}:
      $\forall v_1 \in V_1~\forall v_2 \in V_2~(v_1,v_2) \in E$
    \item $K_{m,n}$: $|V_1|=m$, $|V_2|=n$
  \end{itemize}
\end{frame}

\begin{frame}
  \frametitle{Complete Bipartite Graph Examples}

  \begin{columns}[t]
    \column{.5\textwidth}
    \begin{center}
      $K_{2,3}$

      \bigskip
      \pgfuseimage{k23}
    \end{center}

    \column{.5\textwidth}
    \begin{center}
      $K_{3,3}$

      \bigskip
      \pgfuseimage{k33}
    \end{center}
  \end{columns}
\end{frame}

\subsection{Walks}

\begin{frame}
  \frametitle{Walk}

  \begin{itemize}
    \item \alert{walk}: sequence of nodes and edges\\
      from a starting node ($v_0$) to an ending node ($v_n$)

    \[
      v_0 \xrightarrow{e_1} v_1 \xrightarrow{e_2} v_2 \xrightarrow{e_3} v_3
      \xrightarrow{} \cdots \xrightarrow{} v_{n-1} \xrightarrow{e_n} v_n
    \]

    where $e_i=(v_{i-1},v_i)$

    \item no need to write the edges if not weighted

    \pause
    \medskip
    \item \alert{length}: number of edges
    \item $v_0 = v_n$: \alert{closed}
  \end{itemize}
\end{frame}

\begin{frame}
  \frametitle{Walk Example}

  \begin{columns}
    \column{.58\textwidth}
    \begin{center}
      \pgfuseimage{plain}
    \end{center}

    \column{.4\textwidth}
    $c \xrightarrow{(c,b)} b \xrightarrow{(b,a)} a \xrightarrow{(a,d)} d$\\
    $~~\xrightarrow{(d,e)} e \xrightarrow{(e,f)} f \xrightarrow{(f,a)} a$\\
    $~~\xrightarrow{(a,b)} b$

    \medskip
    $c \rightarrow b \rightarrow a \rightarrow d \rightarrow e$\\
    $~~\rightarrow f \rightarrow a \rightarrow b$

    \bigskip
    length: 7
  \end{columns}
\end{frame}

\begin{frame}
  \frametitle{Trail}

  \begin{itemize}
    \item \alert{trail}: edges not repeated
    \item \alert{circuit}: closed trail
    \item \alert{spanning} trail: covers all edges
  \end{itemize}
\end{frame}

\begin{frame}
  \frametitle{Trail Example}

  \begin{columns}
    \column{.58\textwidth}
    \begin{center}
      \pgfuseimage{plain}
    \end{center}

    \column{.4\textwidth}
    $c \xrightarrow{(c,b)} b \xrightarrow{(b,a)} a \xrightarrow{(a,e)} e$\\
    $~~\xrightarrow{(e,d)} d \xrightarrow{(d,a)} a \xrightarrow{(a,f)} f$

    \medskip
    $c \rightarrow a \rightarrow e \rightarrow d \rightarrow a
        \rightarrow f$
  \end{columns}
\end{frame}

\begin{frame}
  \frametitle{Path}

  \begin{itemize}
    \item \alert{path}: nodes not repeated
    \item \alert{cycle}: closed path
    \item \alert{spanning} path: visits all nodes
  \end{itemize}
\end{frame}

\begin{frame}
  \frametitle{Path Example}

  \begin{columns}
    \column{.58\textwidth}
    \begin{center}
      \pgfuseimage{plain}
    \end{center}

    \column{.4\textwidth}
    $c \xrightarrow{(c,b)} b \xrightarrow{(b,a)} a \xrightarrow{(a,d)} d$\\
    $~~\xrightarrow{(d,e)} e \xrightarrow{(e,f)} f$

    \medskip
    $c \rightarrow b \rightarrow a \rightarrow d \rightarrow e \rightarrow f$
  \end{columns}
\end{frame}

\begin{frame}
  \frametitle{Connected Graphs}

  \begin{itemize}
    \item \alert{connected}: a path between every pair of nodes
    \item a disconnected graph can be divided\\
      into connected components
  \end{itemize}
\end{frame}

\begin{frame}
  \frametitle{Connected Components Example}

  \begin{columns}
    \column{.54\textwidth}
    \begin{center}
      \pgfuseimage{disconnected}
    \end{center}

    \column{.45\textwidth}
    \begin{itemize}
      \item disconnected:\\
        no path between $a$ and $c$
      \item connected components:\\
        $a,d,e$\\
        $b,c$\\
        $f$
    \end{itemize}
  \end{columns}
\end{frame}

\begin{frame}
  \frametitle{Distance}

  \begin{itemize}
    \item \alert{distance} between $v_i$ and $v_j$:\\
      length of shortest path between $v_i$ and $v_j$
    \item \alert{diameter} of graph: largest distance in graph
  \end{itemize}
\end{frame}

\begin{frame}
  \frametitle{Distance Example}

  \begin{columns}
    \column{.48\textwidth}
    \begin{center}
      \pgfuseimage{distance}
    \end{center}

    \column{.52\textwidth}
    \begin{itemize}
      \item distance between $a$ and $e$: 2\\
      \item diameter: 3
    \end{itemize}
  \end{columns}
\end{frame}

\begin{frame}
  \frametitle{Cut-Points}

  \begin{itemize}
    \item \alert{$G - v$}: delete $v$ and all its incident edges from $G$

    \item $v$ is a \alert{cut-point} for $G$:\\
      $G$ is connected but $G - v$ is not
  \end{itemize}
\end{frame}

\begin{frame}
  \frametitle{Cut-Point Example}

  \begin{columns}
    \column{.5\textwidth}
    \begin{center}
      $G$

      \bigskip
      \pgfuseimage{distance}
    \end{center}

    \column{.5\textwidth}
    \begin{center}
      $G-d$

      \bigskip
      \pgfuseimage{cutpoint}
    \end{center}
  \end{columns}
\end{frame}

\begin{frame}
  \frametitle{Directed Walks}

  \begin{itemize}
    \item ignoring directions on arcs:
      \emph{semi-walk}, \emph{semi-trail}, \emph{semi-path}

    \bigskip
    \item if between every pair of nodes there is:
    \smallskip
    \item a semi-path: \alert{weakly} connected
    \item a path from one to the other: \alert{unilaterally} connected
    \item a path: \alert{strongly} connected
  \end{itemize}
\end{frame}

\begin{frame}
  \frametitle{Directed Graph Examples}

  \begin{columns}
    \column{.3\textwidth}
    \begin{exampleblock}{weakly}
      \begin{center}
        \pgfuseimage{weak}
      \end{center}
    \end{exampleblock}

    \column{.3\textwidth}
    \begin{exampleblock}{unilaterally}
      \begin{center}
        \pgfuseimage{unilateral}
      \end{center}
    \end{exampleblock}

    \column{.3\textwidth}
    \begin{exampleblock}{strongly}
      \begin{center}
        \pgfuseimage{strong}
      \end{center}
    \end{exampleblock}
  \end{columns}
\end{frame}

\subsection{Traversable Graphs}

\begin{frame}
  \frametitle{Bridges of Königsberg}

  \begin{center}
    \pgfuseimage{konigsberg}
  \end{center}

  \begin{itemize}
    \item cross each bridge exactly once\\
      and return to the starting point
  \end{itemize}
\end{frame}

\begin{frame}
  \frametitle{Graphs}

  \begin{center}
    \pgfuseimage{konigsgraph}
  \end{center}
\end{frame}

\begin{frame}
  \frametitle{Traversable Graphs}

  \begin{itemize}
    \item $G$ is \alert{traversable}: $G$ contains a spanning trail

    \medskip
    \item a node with an odd degree must be either the starting node\\
      or the ending node of the trail
    \item all nodes except the starting node and the ending node\\
      must have even degrees
  \end{itemize}
\end{frame}

\begin{frame}
  \frametitle{Bridges of Königsberg}

  \begin{center}
    \pgfuseimage{konigsgraph}
  \end{center}

  \begin{itemize}
    \item all nodes have odd degrees: not traversable
  \end{itemize}
\end{frame}

\begin{frame}
  \frametitle{Traversable Graph Example}

  \begin{columns}
    \column{.4\textwidth}
    \begin{center}
      \pgfuseimage{envelope}
    \end{center}

    \column{.6\textwidth}
    \begin{itemize}
      \item $a,b,c$: even
      \item $d,e$: odd

      \medskip
      \item start from $d$, end at $e$:\\
        $d \rightarrow b \rightarrow a \rightarrow c \rightarrow e$
        $~~\rightarrow d \rightarrow c \rightarrow b \rightarrow e$
    \end{itemize}
  \end{columns}
\end{frame}

\begin{frame}
  \frametitle{Euler Graphs}

  \begin{itemize}
    \item \alert{Euler graph}: contains closed spanning trail
  \end{itemize}

  \begin{itemize}
    \item $G$ is an Euler graph $\Leftrightarrow$
      all nodes in $G$ have even degrees
  \end{itemize}
\end{frame}

\begin{frame}
  \frametitle{Euler Graph Examples}

  \begin{columns}
    \column{.5\textwidth}
    \begin{center}
      Euler

      \bigskip
      \pgfuseimage{euler}
    \end{center}

    \column{.5\textwidth}
    \begin{center}
      not Euler

      \bigskip
      \pgfuseimage{hamilton}
    \end{center}
  \end{columns}
\end{frame}

\begin{frame}
  \frametitle{Hamilton Graphs}

  \begin{itemize}
    \item \alert{Hamilton graph}: contains a closed spanning path
  \end{itemize}
\end{frame}

\begin{frame}
  \frametitle{Hamilton Graph Examples}

  \begin{columns}
    \column{.5\textwidth}
    \begin{center}
      Hamilton

      \bigskip
      \pgfuseimage{hamilton}
    \end{center}

    \column{.5\textwidth}
    \begin{center}
      not Hamilton

      \bigskip
      \pgfuseimage{euler}
    \end{center}
  \end{columns}
\end{frame}

\subsection{Planar Graphs}

\begin{frame}
  \frametitle{Planar Graphs}

  \begin{definition}
    $G$ is \alert{planar}:\\
    $G$ can be drawn on a plane without intersecting its edges
  \end{definition}

  \begin{itemize}
    \item a \alert{map} of $G$: a planar drawing of $G$
  \end{itemize}
\end{frame}

\begin{frame}
  \frametitle{Planar Graph Example}

  \begin{columns}
    \column{.5\textwidth}
    \begin{center}
      \pgfuseimage{k4}
    \end{center}

    \column{.5\textwidth}
    \begin{center}
      \pgfuseimage{k4planar}
    \end{center}
  \end{columns}
\end{frame}

\begin{frame}
  \frametitle{Regions}

  \begin{itemize}
    \item map divides plane into \alert{regions}
    \item degree of region: length of closed trail that surrounds region
  \end{itemize}

  \pause
  \begin{theorem}
    $d_{r_i}$: degree of region $r_i$

    \[ |E| = \frac{\sum_i d_{r_i}}{2} \]
  \end{theorem}
\end{frame}

\begin{frame}
  \frametitle{Region Example}

  \begin{columns}
    \column{.58\textwidth}
    \begin{center}
      \pgfuseimage{region}
    \end{center}

    \column{.4\textwidth}
    $\begin{array}{rcr}
      d_{r_1} & = & 3\\
      d_{r_2} & = & 3\\
      d_{r_3} & = & 5\\
      d_{r_4} & = & 4\\
      d_{r_5} & = & 3\\\hline
              & = & 18\\
      \medskip
      |E|     & = & 9
    \end{array}$
  \end{columns}
\end{frame}

\begin{frame}
  \frametitle{Euler's Formula}

  \begin{theorem}[Euler's Formula]
    $G=(V,E)$: planar, connected graph\\
    $R$: set of regions in a map of $G$

    \[|V| - |E| + |R| = 2\]
  \end{theorem}
\end{frame}

\begin{frame}
  \frametitle{Euler's Formula Example}

  \begin{center}
    \pgfuseimage{region}
  \end{center}

  \begin{itemize}
    \item $|V| = 6$, $|E| = 9$, $|R| = 5$
  \end{itemize}
\end{frame}
%
% \begin{frame}
%   \frametitle{Proof of Euler's Formula}
%
%   \begin{block}{Proof}
%     method: induction on $|E|$
%
%     \pause
%     \begin{itemize}
%       \item base step: one node, no edges\\
%         $|V| = 1$, $|E| = 0$, $|R| = 1$
%
%       \pause
%       \item assume it holds for a connected, planar graph with $k$ edges
%     \end{itemize}
%   \end{block}
% \end{frame}
%
% \begin{frame}
%   \frametitle{Proof of Euler's Formula}
%
%   \begin{proof}[Induction Step]
%     \begin{columns}[t]
%       \column{.5\textwidth}
%       \begin{itemize}
%         \item connect a new node\\
% 	  to an existing node:
%
%         \medskip
%         \pgfuseimage{eulerproof1}
%       \end{itemize}
%
%       \column{.5\textwidth}
%       \begin{itemize}
%         \item add an edge between\\
% 	  two existing nodes:
%
%         \medskip
%         \pgfuseimage{eulerproof2}
%       \end{itemize}
%     \end{columns}
%
%     \pause
%     \begin{columns}
%       \column{.5\textwidth}
%       \begin{itemize}
%         \item $|V|$ is increased by 1,\\
% 	  $|E|$ is increased by 1,\\
%           $|R|$ remains the same
%       \end{itemize}
%
%       \pause
%       \column{.5\textwidth}
%       \begin{itemize}
%         \item $|V|$ remains the same,\\
% 	  $|E|$ is increased by 1,\\
%           $|R|$ is increased by 1
%       \end{itemize}
%     \end{columns}
%   \end{proof}
% \end{frame}

\begin{frame}
  \frametitle{Planar Graph Theorems}

  \begin{theorem}
    $G=(V,E)$: connected, planar graph where $|V| \geq 3$

    $|E| \leq 3 |V| - 6$
  \end{theorem}

  \pause
  \begin{proof}
    \begin{itemize}
      \item sum of region degrees: $2 |E|$

      \pause
      \item degree of a region $\geq 3$\\
        \pause
        $\Rightarrow 2 |E| \geq 3 |R|$
        \pause
        $\Rightarrow |R| \leq \frac{2}{3} |E|$

      \pause
      \item $|V| - |E| + |R| = 2$\\
        \pause
        $\Rightarrow |V| - |E| + \frac{2}{3} |E| \geq 2$
        \pause
        $\Rightarrow |V| - \frac{1}{3} |E| \geq 2$\\
        \pause
        $\Rightarrow 3 |V| - |E| \geq 6$
        \pause
        $\Rightarrow |E| \leq 3 |V| - 6$\\
    \end{itemize}
  \end{proof}
\end{frame}

\begin{frame}
  \frametitle{Planar Graph Theorems}

  \begin{theorem}
    $G=(V,E)$: connected, planar graph where $|V| \geq 3$:

    $\exists v \in V~[d_v \leq 5]$
  \end{theorem}

  \pause
  \begin{proof}
    \begin{itemize}
      \item assume: $\forall v \in V~[d_v \geq 6]$\\
        \pause
        $\Rightarrow 2 |E| \geq 6 |V|$\\
        \pause
        $\Rightarrow |E| \geq 3 |V|$\\
        \pause
        $\Rightarrow |E| > 3 |V| - 6$
    \end{itemize}
  \end{proof}
\end{frame}

\begin{frame}
  \frametitle{Nonplanar Graphs}

  \begin{columns}
    \column{.45\textwidth}
    \begin{theorem}
      $K_5$ is not planar.

      \medskip
      \begin{center}
        \pgfuseimage{k5}
      \end{center}
    \end{theorem}

    \pause
    \column{.55\textwidth}
    \begin{proof}
      \begin{itemize}
        \item $|V| = 5$

        \pause
        \item $3 |V| - 6 = 3 \cdot 5 - 6 = 9$

        \pause
        \item $|E| \leq 9$ should hold

        \pause
        \item but $|E| = 10$
      \end{itemize}
    \end{proof}
  \end{columns}
\end{frame}

\begin{frame}
  \frametitle{Nonplanar Graphs}

  \begin{columns}
    \column{.45\textwidth}
    \begin{theorem}
      $K_{3,3}$ is not planar.

      \medskip
      \begin{center}
        \pgfuseimage{k33}
      \end{center}
    \end{theorem}

    \pause
    \column{.55\textwidth}
    \begin{proof}
      \begin{itemize}
        \item $|V| = 6, |E| = 9$

        \pause
        \item if planar then $|R| = 5$

        \pause
        \item degree of a region $\geq 4$\\
          $\Rightarrow \sum_{r \in R} d_r \geq 20$

        \pause
        \item $|E| \geq 10$ should hold

        \pause
        \item but $|E| = 9$
      \end{itemize}
    \end{proof}
  \end{columns}
\end{frame}

\begin{frame}
  \frametitle{Kuratowski's Theorem}

  \begin{theorem}
    \begin{center}
      $G$ contains a subgraph homeomorphic to $K_5$ or $K_{3,3}$.\\
      $\Leftrightarrow$\\
      $G$ is not planar.
    \end{center}
  \end{theorem}
\end{frame}

\begin{frame}
  \frametitle{Platonic Solids}

  \begin{itemize}
    \item \emph{regular polyhedron}: a 3-dimensional solid\\
      where faces are identical regular polygons

    \bigskip
    \item projection of a regular polyhedron onto the plane:\\
      a planar graph
    \smallskip
    \item corners: nodes
    \item sides: edges
    \item faces: regions
  \end{itemize}
\end{frame}

\begin{frame}
  \frametitle{Platonic Solid Example}

  \begin{columns}
    \column{.7\textwidth}
    \begin{center}
      \pgfuseimage{hexahedron}
    \end{center}

    \column{.3\textwidth}
    \begin{center}
      \pgfuseimage{planarhexa}
    \end{center}
  \end{columns}
\end{frame}

\begin{frame}
  \frametitle{Platonic Solids}

  \begin{itemize}
    \item $|V|$: number of corners (nodes)
    \item $|E|$: number of sides (edges)
    \item $|R|$: number of faces (regions)
    \item $n$: number of faces meeting at a corner (node degree)
    \item $m$: number of sides of a face (region degree)
  \end{itemize}

  \pause
  \begin{itemize}
    \item $m,n \geq 3$
    \item $2|E| = n \cdot |V|$
    \item $2|E| = m \cdot |R|$
  \end{itemize}
\end{frame}

\begin{frame}
  \frametitle{Platonic Solids}

    \begin{itemize}
      \item from Euler's formula:
      \[
        2 = |V| - |E| + |R| = \frac{2|E|}{n} - |E| + \frac{2|E|}{m}
        = |E| \Big( \frac{2m-mn+2n}{mn} \Big) > 0
      \]

      \pause
      \item $|E|,m,n > 0$:
      \begin{eqnarray*}
        2m - mn + 2n > 0 \Rightarrow mn - 2m -2n < 0 \\\pause
        \Rightarrow mn - 2m - 2n + 4 < 4 \pause \Rightarrow (m - 2)(n - 2) < 4
      \end{eqnarray*}

      \pause
      \item only 5 solutions
    \end{itemize}
\end{frame}

\begin{frame}
  \frametitle{Tetrahedron}

  \begin{columns}
    \column{.7\textwidth}
    \begin{center}
      \pgfuseimage{tetrahedron}
    \end{center}

    \column{.3\textwidth}
    \begin{center}
      \pgfuseimage{planartetra}

      $m=3, n=3$
    \end{center}
  \end{columns}
\end{frame}

\begin{frame}
  \frametitle{Hexahedron}

  \begin{columns}
    \column{.7\textwidth}
    \begin{center}
      \pgfuseimage{hexahedron}
    \end{center}

    \column{.3\textwidth}
    \begin{center}
      \pgfuseimage{planarhexa}

      $m=4, n=3$
    \end{center}
  \end{columns}
\end{frame}

\begin{frame}
  \frametitle{Octahedron}

  \begin{columns}
    \column{.7\textwidth}
    \begin{center}
      \pgfuseimage{octahedron}
    \end{center}

    \column{.3\textwidth}
    \begin{center}
      \pgfuseimage{planarocta}

      $m=3, n=4$
    \end{center}
  \end{columns}
\end{frame}

\begin{frame}
  \frametitle{Dodecahedron}

  \begin{columns}
    \column{.7\textwidth}
    \begin{center}
      \pgfuseimage{dodecahedron}
    \end{center}

    \column{.3\textwidth}
    \begin{center}
      \pgfuseimage{planardodeca}

      $m=5, n=3$
    \end{center}
  \end{columns}
\end{frame}

\begin{frame}
  \frametitle{Icosahedron}

  \begin{columns}
    \column{.7\textwidth}
    \begin{center}
      \pgfuseimage{icosahedron}
    \end{center}

    \column{.3\textwidth}
    $m=3, n=5$
  \end{columns}
\end{frame}

\section{Graph Problems}

\subsection{Connectivity}

\begin{frame}
  \frametitle{Connectivity Matrix}

  \begin{itemize}
    \item $A$: adjacency matrix of $G=(V,E)$
    \item $A^k_{ij}$: number of walks of length $k$ between $v_i$ and $v_j$

    \medskip
    \item maximum distance between two nodes: $|V|-1$

    \pause
    \medskip
    \item connectivity matrix:\\
      $C = A^1 + A^2 + A^3 + \dots + A^{|V|-1}$
    \item connected: all elements of $C$ are non-zero
  \end{itemize}
\end{frame}

\begin{frame}
  \frametitle{Warshall's Algorithm}

  \begin{itemize}
    \item very expensive to compute the connectivity matrix
    \item easier to find whether there is a walk between two nodes\\
      rather than finding the number of walks

    \pause
    \bigskip
    \item for each node:
    \smallskip
    \item from all nodes which can reach the current node\\
      (rows that contain 1 in current column)

    \item to all nodes which can be reached from the current node\\
      (columns that contain 1 in current row)
  \end{itemize}
\end{frame}

\begin{frame}
  \frametitle{Warshall's Algorithm Example}

  \begin{columns}
    \column{.5\textwidth}
    \begin{center}
      \pgfuseimage{warshall}
    \end{center}

    \column{.5\textwidth}
    \[
      \begin{array}{c|cccc}
            & a & b & c & d\\\hline
          a & 0 & \alert{1} & 0 & 0\\
          b & 0 & 1 & 0 & 0\\
          c & 0 & 0 & 0 & 1\\
          d & \alert{1} & 0 & 1 & 0
      \end{array}
    \]
  \end{columns}
\end{frame}

\begin{frame}
  \frametitle{Warshall's Algorithm Example}

  \begin{columns}
    \column{.5\textwidth}
    \begin{center}
      \pgfuseimage{warshall1}
    \end{center}

    \column{.5\textwidth}
    \[
      \begin{array}{c|cccc}
            & a & b & c & d\\\hline
          a & 0 & \alert{1} & 0 & 0\\
          b & 0 & 1 & 0 & 0\\
          c & 0 & 0 & 0 & 1\\
          d & 1 & \alert{1} & 1 & 0
      \end{array}
    \]
  \end{columns}
\end{frame}

\begin{frame}
  \frametitle{Warshall's Algorithm Example}

  \begin{columns}
    \column{.5\textwidth}
    \begin{center}
      \pgfuseimage{warshall1}
    \end{center}

    \column{.5\textwidth}
    \[
      \begin{array}{c|cccc}
            & a & b & c & d\\\hline
          a & 0 & 1 & 0 & 0\\
          b & 0 & 1 & 0 & 0\\
          c & 0 & 0 & 0 & \alert{1}\\
          d & 1 & 1 & \alert{1} & 0
      \end{array}
    \]
  \end{columns}
\end{frame}

\begin{frame}
  \frametitle{Warshall's Algorithm Example}

  \begin{columns}
    \column{.5\textwidth}
    \begin{center}
      \pgfuseimage{warshall2}
    \end{center}

    \column{.5\textwidth}
    \[
      \begin{array}{c|cccc}
            & a & b & c & d\\\hline
          a & 0 & 1 & 0 & 0\\
          b & 0 & 1 & 0 & 0\\
          c & 0 & 0 & 0 & \alert{1}\\
          d & \alert{1} & \alert{1} & \alert{1} & \alert{1}
      \end{array}
    \]
  \end{columns}
\end{frame}

\begin{frame}
  \frametitle{Warshall's Algorithm Example}

  \begin{columns}
    \column{.5\textwidth}
    \begin{center}
      \pgfuseimage{warshall3}
    \end{center}

    \column{.5\textwidth}
    \[
      \begin{array}{c|cccc}
            & a & b & c & d\\\hline
          a & 0 & 1 & 0 & 0\\
          b & 0 & 1 & 0 & 0\\
          c & 1 & 1 & 1 & 1\\
          d & 1 & 1 & 1 & 1
      \end{array}
    \]
  \end{columns}
\end{frame}

\subsection{Graph Coloring}

\begin{frame}
  \frametitle{Graph Coloring}

  \begin{itemize}
    \item $G=(V,E)$, $C$: set of colors
    \item \alert{proper coloring} of $G$: find an $f: V \rightarrow C$,
      such that\\
      $\forall (v_i,v_j) \in E~[f(v_i) \neq f(v_j)]$

    \pause
    \medskip
    \item \alert{chromatic number} of $G$: $\chi (G)$\\
      minimum $|C|$
     \item finding $\chi (G)$ is a very difficult problem

     \pause
     \smallskip
     \item $\chi (K_n) = n$
  \end{itemize}
\end{frame}

\begin{frame}
  \frametitle{Chromatic Number Example}

  \begin{center}
    \pgfuseimage{herschel}
  \end{center}

  \begin{itemize}
    \item Herschel graph: $\chi (G)=2$
  \end{itemize}
\end{frame}

\begin{frame}
  \frametitle{Graph Coloring Solution}

  \begin{itemize}
    \item pick a node and assign a color
    \item assign same color to all nodes with no conflict
    \item pick an uncolored node and assign a second color
    \item assign same color to all uncolored nodes with no conflict
    \item pick an uncolored node and assign a third color
    \item \ldots
  \end{itemize}
\end{frame}

\begin{frame}
  \frametitle{Heuristic Solutions}

  \begin{itemize}
    \item \alert{heuristic} solution: based on intuition
    \item \alert{greedy} solution: doesn't look ahead
    \item doesn't produce optimal results
  \end{itemize}
\end{frame}

\begin{frame}
  \frametitle{Graph Coloring Example}

  \begin{itemize}
    \item a company produces chemical compounds
    \item some compounds cannot be stored together
    \item such compounds must be placed in separate storage areas

    \pause
    \medskip
    \item store compounds using minimum number of storage areas
  \end{itemize}
\end{frame}

\begin{frame}
  \frametitle{Graph Coloring Example}

  \begin{itemize}
    \item every compound is a node
    \item two compounds that cannot be stored together are adjacent
  \end{itemize}

  \begin{center}
    \pgfuseimage{fivesubstances}
  \end{center}
\end{frame}

\begin{frame}
  \frametitle{Graph Coloring Example}

  \begin{center}
    \pgfuseimage{coloring1}
  \end{center}
\end{frame}

\begin{frame}
  \frametitle{Graph Coloring Example}

  \begin{columns}
    \column{.5\textwidth}
    \begin{center}
      \pgfuseimage{coloring2}
    \end{center}

    \column{.5\textwidth}
    \begin{center}
      \pgfuseimage{coloring3}
    \end{center}
  \end{columns}
\end{frame}

\begin{frame}
  \frametitle{Graph Coloring Example}

  \begin{columns}
    \column{.5\textwidth}
    \begin{center}
      \pgfuseimage{coloring4}
    \end{center}

    \column{.5\textwidth}
    \begin{center}
      \pgfuseimage{coloring5}
    \end{center}
  \end{columns}
\end{frame}

\begin{frame}
  \frametitle{Graph Coloring Example}

  \begin{columns}
    \column{.5\textwidth}
    \begin{center}
      \pgfuseimage{coloring6}
    \end{center}

    \column{.5\textwidth}
    \begin{center}
      \pgfuseimage{coloring7}
    \end{center}
  \end{columns}
\end{frame}

\begin{frame}
  \frametitle{Graph Coloring Example: Sudoku}

  \begin{columns}[t]
    \column{.4\textwidth}
    \begin{center}
      \pgfuseimage{sudoku}
    \end{center}

    \column{.55\textwidth}
    \begin{itemize}
      \item every cell is a node
      \item cells of the same row\\
        are adjacent
      \item cells of the same column\\
        are adjacent
      \item cells of the same $3 \times 3$ block\\
        are adjacent
      \item every number is a color
    \end{itemize}

    \pause
    \begin{itemize}
      \item problem: properly color a graph\\
        that is partially colored
    \end{itemize}
  \end{columns}
\end{frame}

\begin{frame}
  \frametitle{Region Coloring}

  \begin{itemize}
    \item coloring a map by assigning different colors to adjacent regions
  \end{itemize}

  \medskip
  \begin{theorem}[Four~Color Theorem]
    The regions in a map can be colored using four colors.
  \end{theorem}
\end{frame}

\subsection{Shortest Path}

\begin{frame}
  \frametitle{Shortest Path}

  \begin{itemize}
    \item finding shortest paths from a starting node\\
      to all other nodes:
      Dijkstra's algorithm
  \end{itemize}
\end{frame}

\begin{frame}
  \frametitle{Dijkstra's Algorithm Example}

  \begin{columns}
    \column{.5\textwidth}
    \begin{center}
      \pgfuseimage{dijkstra}
    \end{center}

    \column{.45\textwidth}
    \begin{itemize}
      \item starting node: $c$
    \end{itemize}

    \begin{table}
      \begin{tabular}{r|l}
        a & $(\infty,-)$ \\\hline
        b & $(\infty,-)$ \\\hline
        c & $(0,-)$      \\\hline
        f & $(\infty,-)$ \\\hline
        g & $(\infty,-)$ \\\hline
        h & $(\infty,-)$
      \end{tabular}
    \end{table}
  \end{columns}
\end{frame}

\begin{frame}
  \frametitle{Dijkstra's Algorithm Example}

  \begin{columns}
    \column{.5\textwidth}
    \begin{center}
      \pgfuseimage{dijkstra}
    \end{center}

    \column{.45\textwidth}
    \begin{itemize}
      \item from $c$: base distance=$0$
      \item $c \rightarrow f: 6, 6 < \infty$
      \item $c \rightarrow h: 11, 11 < \infty$

      \pause
      \begin{table}
        \begin{tabular}{r|l|c}
          a & $(\infty,-)$ & \\\hline
          b & $(\infty,-)$ & \\\hline
          c & $(0,-)$      & $\surd$ \\\hline
          f & $(6,cf)$     & \\\hline
          g & $(\infty,-)$ & \\\hline
          h & $(11,ch)$    &
        \end{tabular}
      \end{table}

      \pause
      \item closest node: $f$
    \end{itemize}
  \end{columns}
\end{frame}

\begin{frame}
  \frametitle{Dijkstra's Algorithm Example}

  \begin{columns}
    \column{.5\textwidth}
    \begin{center}
      \pgfuseimage{dijkstra}
    \end{center}

    \column{.45\textwidth}
    \begin{itemize}
      \item from $f$: base distance=$6$
      \item $f \rightarrow a: 6+11, 17 < \infty$
      \item $f \rightarrow g: 6+9, 15 < \infty$
      \item $f \rightarrow h: 6+4, 10 < 11$

      \pause
      \begin{table}
        \begin{tabular}{r|l|c}
          a & $(17,cfa)$   & \\\hline
          b & $(\infty,-)$ & \\\hline
          c & $(0,-)$      & $\surd$ \\\hline
          f & $(6,cf)$     & $\surd$ \\\hline
          g & $(15,cfg)$   & \\\hline
          h & $(10,cfh)$   &
        \end{tabular}
      \end{table}

      \pause
      \item closest node: $h$
    \end{itemize}
  \end{columns}
\end{frame}

\begin{frame}
  \frametitle{Dijkstra's Algorithm Example}

  \begin{columns}
    \column{.5\textwidth}
    \begin{center}
      \pgfuseimage{dijkstra}
    \end{center}

    \column{.45\textwidth}
    \begin{itemize}
      \item from $h$: base distance=$10$
      \item $h \rightarrow a: 10+11, 21 \nless 17$
      \item $h \rightarrow g: 10+4, 14 < 15$

      \pause
      \begin{table}
        \begin{tabular}{r|l|c}
          a & $(17,cfa)$   & \\\hline
          b & $(\infty,-)$ & \\\hline
          c & $(0,-)$      & $\surd$ \\\hline
          f & $(6,cf)$     & $\surd$ \\\hline
          g & $(14,cfhg)$  & \\\hline
          h & $(10,cfh)$   & $\surd$
        \end{tabular}
      \end{table}

      \pause
      \item closest node: $g$
    \end{itemize}
  \end{columns}
\end{frame}

\begin{frame}
  \frametitle{Dijkstra's Algorithm Example}

  \begin{columns}
    \column{.5\textwidth}
    \begin{center}
      \pgfuseimage{dijkstra}
    \end{center}

    \column{.45\textwidth}
    \begin{itemize}
      \item from $g$: base distance=$14$
      \item $g \rightarrow a: 14+17, 31 \nless 17$

      \pause
      \begin{table}
        \begin{tabular}{r|l|c}
          a & $(17,cfa)$   & \\\hline
          b & $(\infty,-)$ & \\\hline
          c & $(0,-)$      & $\surd$ \\\hline
          f & $(6,cf)$     & $\surd$ \\\hline
          g & $(14,cfhg)$  & $\surd$ \\\hline
          h & $(10,cfh)$   & $\surd$
        \end{tabular}
      \end{table}

      \pause
      \item closest node: $a$
    \end{itemize}
  \end{columns}
\end{frame}

\begin{frame}
  \frametitle{Dijkstra's Algorithm Example}

  \begin{columns}
    \column{.5\textwidth}
    \begin{center}
      \pgfuseimage{dijkstra}
    \end{center}

    \column{.45\textwidth}
    \begin{itemize}
      \item from $a$: base distance=$17$
      \item $a \rightarrow b: 17+5, 22 < \infty$

      \pause
      \begin{table}
        \begin{tabular}{r|l|c}
          a & $(17,cfa)$   & $\surd$ \\\hline
          b & $(22,cfab)$  & \\\hline
          c & $(0,-)$      & $\surd$ \\\hline
          f & $(6,cf)$     & $\surd$ \\\hline
          g & $(14,cfhg)$  & $\surd$ \\\hline
          h & $(10,cfh)$   & $\surd$
        \end{tabular}
      \end{table}

      \pause
      \item last node: $b$
    \end{itemize}
  \end{columns}
\end{frame}

\subsection{TSP}

\begin{frame}
  \frametitle{Traveling Salesperson Problem}

  \begin{itemize}
    \item start from a home town
    \item visit every city exactly once
    \item return to the home town
    \item minimum total distance

    \pause
    \medskip
    \item find Hamiltonian cycle
    \item very difficult problem
  \end{itemize}
\end{frame}

\begin{frame}
  \frametitle{TSP Solution}

  \begin{itemize}
    \item heuristic: nearest-neighbor
  \end{itemize}
\end{frame}

% TODO: add example for TSP

\subsection{Searching Graphs}

\begin{frame}
  \frametitle{Searching Graphs}

  \begin{itemize}
    \item searching nodes of graph $G=(V,E)$ starting from node $v_1$

    \bigskip
    \item depth-first
    \item breadth-first
  \end{itemize}
\end{frame}

\begin{frame}
  \frametitle{Depth-First Search}

  \begin{enumerate}
    \item $v \leftarrow v_1, T=\emptyset$, $D=\{v_1\}$

    \pause
    \smallskip
    \item find smallest $i$ in $2 \leq i \leq |V|$
      such that $(v,v_i) \in E$ and $v_i \notin D$
    \begin{itemize}
      \item if no such $i$: go to step 3
      \item if found: $T=T \cup \{(v,v_i)\}$, $D=D \cup \{v_i\}$,
        $v \leftarrow v_i$,\\
        go to step 2
    \end{itemize}

    \pause
    \smallskip
    \item if $v=v_1$: result is $T$

    \pause
    \smallskip
    \item if $v \neq v_1$: $v \leftarrow backtrack(v)$, go to step 2
  \end{enumerate}
\end{frame}

% TODO: add example for depth-first search

\begin{frame}
  \frametitle{Breadth-First Search}

  \begin{enumerate}
    \item $T=\emptyset$, $D=\{v_1\}$, $Q=(v_1)$

    \pause
    \smallskip
    \item if $Q$ empty: result is $T$

    \smallskip
    \item if $Q$ not empty: $v \leftarrow front(Q)$, $Q \leftarrow Q - v$\\
      for $2 \leq i \leq |V|$ check edges $(v,v_i) \in E$:
    \begin{itemize}
      \item if $v_i \notin D$ : $Q = Q + v_i$, $T = T \cup \{(v,v_i)\}$,
        $D=D \cup \{v_i\}$
       \item go to step 3
    \end{itemize}
  \end{enumerate}
\end{frame}

% TODO: add example for breadth-first search

\section*{References}

\begin{frame}
  \frametitle{References}

  \begin{block}{Required Reading: Grimaldi}
    \begin{itemize}
      \item Chapter 11: \alert{An Introduction to Graph Theory}

      \item Chapter 7: Relations: The Second Time Around
      \begin{itemize}
        \item 7.2. \alert{Computer Recognition: Zero-One Matrices\\
                          and Directed Graphs}
      \end{itemize}

      \item Chapter 13: Optimization and Matching
      \begin{itemize}
        \item 13.1. \alert{Dijkstra's Shortest Path Algorithm}
      \end{itemize}
    \end{itemize}
  \end{block}
\end{frame}

\end{document}
