% Copyright (c) 2001-2012
%       H. Turgut Uyar <uyar@itu.edu.tr>
%       Ayşegül Gençata Yayımlı <gencata@itu.edu.tr>
%       Emre Harmancı <harmanci@itu.edu.tr>
%
% These notes are licensed using the
% "Creative Commons Attribution-NonCommercial-ShareAlike License".
% You are free to copy, distribute and transmit the work, and to adapt the work
% as long as you attribute the authors, do not use it for commercial purposes,
% and any derivative work is under the same or a similar license.
%
% Read the full legal code at:
% http://creativecommons.org/licenses/by-nc-sa/3.0/

\documentclass[dvipsnames]{beamer}

\usepackage{ae}
\usepackage[T1]{fontenc}
\usepackage[utf8]{inputenc}
\setbeamertemplate{navigation symbols}{}

\mode<presentation>
{
  \usetheme{Rochester}
  \setbeamercovered{transparent}
}

\title{Discrete Mathematics}
\subtitle{Graphs}

\author{H. Turgut Uyar \and Ayşegül Gençata Yayımlı \and Emre Harmancı}
\date{2001-2012}

\AtBeginSubsection[]
{
  \begin{frame}<beamer>
    \frametitle{Topics}
    \tableofcontents[currentsection,currentsubsection]
  \end{frame}
}

%\beamerdefaultoverlayspecification{<+->}

\pgfdeclareimage[width=2cm]{license}{../../license}

\pgfdeclareimage[width=6cm]{plain}{plain}
\pgfdeclareimage[width=4cm]{directed}{directed}
\pgfdeclareimage[width=6cm]{multi}{multi}
\pgfdeclareimage[width=4cm]{matrix}{matrix}
\pgfdeclareimage[height=4.5cm]{regular}{regular}
\pgfdeclareimage[width=4cm]{k4}{k4}
\pgfdeclareimage[width=4cm]{k5}{k5}
\pgfdeclareimage[width=4cm]{k23}{k23}
\pgfdeclareimage[width=4cm]{k33}{k33}
\pgfdeclareimage[height=4cm]{isomorphicf}{isomorphicf}
\pgfdeclareimage[height=4cm]{isomorphict}{isomorphict}
\pgfdeclareimage[height=3.5cm]{petersen1}{petersen1}
\pgfdeclareimage[height=3.5cm]{petersen2}{petersen2}
\pgfdeclareimage[height=4cm]{homeomorphict}{homeomorphict}
\pgfdeclareimage[width=6cm]{disconnected}{disconnected}
\pgfdeclareimage[width=5cm]{distance}{distance}
\pgfdeclareimage[width=5cm]{cutpoint}{cutpoint}
\pgfdeclareimage[width=4cm]{weak}{weak}
\pgfdeclareimage[width=4cm]{unilateral}{unilateral}
\pgfdeclareimage[width=4cm]{strong}{strong}
\pgfdeclareimage{konigsberg}{konigsberg}
\pgfdeclareimage[width=3cm]{envelope}{envelope}
\pgfdeclareimage{konigsgraph}{konigsgraph}
\pgfdeclareimage[width=3.5cm]{euler}{euler}
\pgfdeclareimage[width=3.5cm]{hamilton}{hamilton}
\pgfdeclareimage[width=5cm]{warshall}{warshall}
\pgfdeclareimage[width=5cm]{warshall1}{warshall1}
\pgfdeclareimage[width=5cm]{warshall2}{warshall2}
\pgfdeclareimage[width=5cm]{warshall3}{warshall3}
\pgfdeclareimage[width=4.5cm]{k4planar}{k4planar}
\pgfdeclareimage[width=6cm]{region}{region}
\pgfdeclareimage[width=3cm]{eulerproof1}{eulerproof1}
\pgfdeclareimage[width=3cm]{eulerproof2}{eulerproof2}
\pgfdeclareimage[width=5cm]{hexahedron}{hexahedron}
\pgfdeclareimage{tetrahedron}{tetrahedron}
\pgfdeclareimage{planartetra}{planartetra}
\pgfdeclareimage{planarhexa}{planarhexa}
\pgfdeclareimage{octahedron}{octahedron}
\pgfdeclareimage{planarocta}{planarocta}
\pgfdeclareimage{dodecahedron}{dodecahedron}
\pgfdeclareimage{planardodeca}{planardodeca}
\pgfdeclareimage{icosahedron}{icosahedron}
\pgfdeclareimage[width=4cm]{fivesubstances}{fivesubstances}
\pgfdeclareimage[height=4cm]{coloring1}{coloring1}
\pgfdeclareimage[height=4cm]{coloring2}{coloring2}
\pgfdeclareimage[height=4cm]{coloring3}{coloring3}
\pgfdeclareimage[height=4cm]{coloring4}{coloring4}
\pgfdeclareimage[height=4cm]{coloring5}{coloring5}
\pgfdeclareimage[height=4cm]{coloring6}{coloring6}
\pgfdeclareimage[height=4cm]{coloring7}{coloring7}
\pgfdeclareimage[height=4cm]{herschel}{herschel}
\pgfdeclareimage[width=4cm]{sudoku}{sudoku}
\pgfdeclareimage[width=6cm]{dijkstra}{dijkstra}

\pgfdeclareimage{tree}{tree}
\pgfdeclareimage{cycle}{cycle}
\pgfdeclareimage{nodecount}{nodecount}
\pgfdeclareimage[height=5cm]{rooted}{rooted}
\pgfdeclareimage[height=4cm]{book}{book}
\pgfdeclareimage{dictionary}{dictionary}
\pgfdeclareimage{expr1a}{expr1a}
\pgfdeclareimage{expr1b}{expr1b}
\pgfdeclareimage{expr2a}{expr2a}
\pgfdeclareimage{expr2b}{expr2b}
\pgfdeclareimage{expr3}{expr3}
\pgfdeclareimage{expr}{expr}
\pgfdeclareimage{scale1}{scale1}
\pgfdeclareimage{scale2}{scale2}
\pgfdeclareimage[height=4cm]{spanning}{spanning}
\pgfdeclareimage[height=4cm]{kruskal1}{kruskal1}
\pgfdeclareimage[height=4cm]{kruskal2}{kruskal2}
\pgfdeclareimage[height=4cm]{kruskal3}{kruskal3}
\pgfdeclareimage[height=4cm]{kruskal4}{kruskal4}
\pgfdeclareimage[height=4cm]{kruskal5}{kruskal5}
\pgfdeclareimage[height=4cm]{kruskal}{kruskal}
\pgfdeclareimage[height=4cm]{prim1}{prim1}
\pgfdeclareimage[height=4cm]{prim2}{prim2}
\pgfdeclareimage[height=4cm]{prim3}{prim3}
\pgfdeclareimage[height=4cm]{prim4}{prim4}
\pgfdeclareimage[height=4cm]{prim5}{prim5}
\pgfdeclareimage[height=4cm]{prim}{prim}

\begin{document}

\begin{frame}
  \titlepage
\end{frame}

\begin{frame}
  \frametitle{License}

  \pgfuseimage{license}\hfill
  \copyright 2001-2012 T. Uyar, A. Yayımlı, E. Harmancı

  \vfill
  \begin{tiny}
    You are free:
    \begin{itemize}
      \item to Share -- to copy, distribute and transmit the work
      \item to Remix -- to adapt the work
    \end{itemize}

    Under the following conditions:
    \begin{itemize}
      \item Attribution -- You must attribute the work in the manner specified by
        the author or licensor (but not in any way that suggests that they
        endorse you or your use of the work).

      \item Noncommercial -- You may not use this work for commercial purposes.

      \item Share Alike -- If you alter, transform, or build upon this work, you
        may distribute the resulting work only under the same or similar license
        to this one.
    \end{itemize}
  \end{tiny}

  \vfill
  Legal code (the full license):\\
  \url{http://creativecommons.org/licenses/by-nc-sa/3.0/}
\end{frame}

\begin{frame}
  \frametitle{Topics}
  \tableofcontents
\end{frame}

\section{Graphs}

\subsection{Introduction}

\begin{frame}
  \frametitle{Graphs}

  \begin{definition}
    \alert{graph}: $G=(V,E)$

    \begin{itemize}
      \item $V$: \alert{node} (or \emph{vertex}) set
      \item $E \subseteq V \times V$: \alert{edge} set
    \end{itemize}
  \end{definition}

  \pause
  \begin{itemize}
    \item if $e=(v_1,v_2) \in E$:
    \begin{itemize}
      \item $v_1$ and $v_2$ are \emph{endnodes} of $e$
      \item $e$ is \emph{incident} to $v_1$ and $v_2$
      \item $v_1$ and $v_2$ are \emph{adjacent}
    \end{itemize}

    \pause
    \item node with no incident edge: \emph{isolated node}
  \end{itemize}
\end{frame}

\begin{frame}
  \frametitle{Graph Example}

  \begin{example}
    \begin{columns}
      \column{.58\textwidth}
      \begin{center}
        \pgfuseimage{plain}
      \end{center}

      \pause
      \column{.4\textwidth}
      $\begin{array}{lcl}
        V & = & \{a,b,c,d,e,f\}\\
        E & = & \{(a,b),(a,c),\\
          &   & ~(a,d),(a,e),\\
          &   & ~(a,f),(b,c),\\
          &   & ~(d,e),(e,f)\}
      \end{array}$
    \end{columns}
  \end{example}
\end{frame}

\begin{frame}
  \frametitle{Directed Graphs}

  \begin{definition}
    \alert{directed graph} (or \emph{digraph}): edges have directions

    \pause
    \begin{itemize}
      \item directed edge: \alert{arc}
      \item \alert{origin} and \alert{terminating} nodes
    \end{itemize}
  \end{definition}
\end{frame}

\begin{frame}
  \frametitle{Directed Graph Example}

  \begin{example}
    \begin{center}
      \pgfuseimage{directed}
    \end{center}
  \end{example}
\end{frame}

\begin{frame}
  \frametitle{Multigraphs}

  \begin{definition}
    \alert{parallel edges}:\\
    edges between the same node pair

    \pause
    \bigskip
    \alert{loop}:\\
    an edge whose ends are incident to the same node

    \pause
    \bigskip
    \alert{plain graph}:\\
    a graph which does not contain any loops or parallel edges

    \pause
    \bigskip
    \alert{multigraph}:\\
    a graph which is not plain
  \end{definition}
\end{frame}

\begin{frame}
  \frametitle{Multigraph Example}

  \begin{example}
    \begin{columns}
      \column{.58\textwidth}
      \begin{center}
        \pgfuseimage{multi}
      \end{center}

      \column{.4\textwidth}
      \begin{itemize}
        \item parallel edges:\\
          $(a,b)$
        \item loop:\\
          $(e,e)$
      \end{itemize}
    \end{columns}
  \end{example}
\end{frame}

\begin{frame}
  \frametitle{Representation}

  \begin{itemize}
    \item \emph{incidence matrix}:
    \begin{itemize}
      \item rows represent nodes, columns represent edges
      \item cell: 1 if the edge is incident to the node, 0 otherwise
    \end{itemize}

    \pause
    \medskip
    \item \emph{adjacency matrix}:
    \begin{itemize}
      \item rows and columns represent nodes
      \item cells represent the number of edges between the nodes
    \end{itemize}
  \end{itemize}
\end{frame}

\begin{frame}
  \frametitle{Incidence Matrix Example}

  \begin{example}
    \begin{columns}
      \column{.38\textwidth}
      \begin{center}
        \pgfuseimage{matrix}
      \end{center}

      \column{.58\textwidth}
      \[
        \begin{array}{c|cccccccc}
              & e_1 & e_2 & e_3 & e_4 & e_5 & e_6 & e_7 & e_8\\\hline
          v_1 & 1 & 1 & 1 & 0 & 1 & 0 & 0 & 0\\
          v_2 & 1 & 0 & 0 & 1 & 0 & 0 & 0 & 0\\
          v_3 & 0 & 0 & 1 & 1 & 0 & 0 & 1 & 1\\
          v_4 & 0 & 0 & 0 & 0 & 1 & 1 & 0 & 1\\
          v_5 & 0 & 1 & 0 & 0 & 0 & 1 & 1 & 0
        \end{array}
      \]
    \end{columns}
  \end{example}
\end{frame}

\begin{frame}
  \frametitle{Adjacency Matrix Example}

  \begin{example}
    \begin{columns}
      \column{.38\textwidth}
      \begin{center}
        \pgfuseimage{matrix}
      \end{center}

      \column{.58\textwidth}
      \[
        \begin{array}{c|ccccc}
                & v_1 & v_2 & v_3 & v_4 & v_5\\\hline
            v_1 & 0 & 1 & 1 & 1 & 1\\
            v_2 & 1 & 0 & 1 & 0 & 0\\
            v_3 & 1 & 1 & 0 & 1 & 1\\
            v_4 & 1 & 0 & 1 & 0 & 1\\
            v_5 & 1 & 0 & 1 & 1 & 0
        \end{array}
      \]
    \end{columns}
  \end{example}
\end{frame}

\begin{frame}
  \frametitle{Adjacency Matrix Example}

  \begin{example}
    \begin{columns}
    \column{.5\textwidth}
    \begin{center}
      \pgfuseimage{directed}
    \end{center}

    \column{.5\textwidth}
      \[
        \begin{array}{c|cccc}
              & a & b & c & d\\\hline
            a & 0 & 0 & 0 & 1\\
            b & 2 & 1 & 1 & 0\\
            c & 0 & 0 & 0 & 0\\
            d & 0 & 1 & 1 & 0
        \end{array}
      \]
    \end{columns}
  \end{example}
\end{frame}

\begin{frame}
  \frametitle{Degree}

  \begin{definition}
    \alert{degree}: number of edges incident to the node
  \end{definition}

  \pause
  \begin{theorem}
    if the degree of node $v_i$ is $d_i$:

    \[ |E| = \frac{\sum_i d_i}{2} \]
  \end{theorem}
\end{frame}

\begin{frame}
  \frametitle{Degree Example}

  \begin{example}[plain]
    \begin{columns}
      \column{.58\textwidth}
      \begin{center}
        \pgfuseimage{plain}
      \end{center}

      \column{.4\textwidth}
      $\begin{array}{ccc}
      d_a & = & 5\\
      d_b & = & 2\\
      d_c & = & 2\\
      d_d & = & 2\\
      d_e & = & 3\\
      d_f & = & 2\\
      \medskip
      Total & = & 16\\
      \medskip
      |E| & = & 8
      \end{array}$
    \end{columns}
  \end{example}
\end{frame}

\begin{frame}
  \frametitle{Degree Example}

  \begin{example}[multigraph]
    \begin{columns}
      \column{.58\textwidth}
      \begin{center}
        \pgfuseimage{multi}
      \end{center}

      \column{.4\textwidth}
      $\begin{array}{ccc}
      d_a & = & 6\\
      d_b & = & 3\\
      d_c & = & 2\\
      d_d & = & 2\\
      d_e & = & 5\\
      d_f & = & 2\\
      \medskip
      Total & = & 20\\
      \medskip
      |E| & = & 10
      \end{array}$
    \end{columns}
  \end{example}
\end{frame}

\begin{frame}
  \frametitle{Degree in Directed Graphs}

  \begin{itemize}
    \item two types of degree
    \begin{itemize}
      \item \emph{in-degree}: ${d_v}^i$
      \item \emph{out-degree}: ${d_v}^o$
    \end{itemize}

    \pause
    \medskip
    \item node with in-degree 0: \emph{source}
    \item node with out-degree  0: \emph{sink}

    \pause
    \item $\sum_{v \in V} {d_v}^i = \sum_{v \in V} {d_v}^o = |A|$
  \end{itemize}
\end{frame}

\begin{frame}
  \frametitle{Degree}

  \begin{theorem}
    In an undirected graph, the number of nodes with an odd degree is even.
  \end{theorem}

  \pause
  \begin{proof}
    \begin{itemize}
      \item $t_i$: number of nodes of degree $i$

      \pause
$2|E| = \sum_i d_i = 1t_1 + 2t_2 + 3t_3 + 4t_4 + 5t_5 + \dots$

\pause
$2|E| - 2t_2 - 4t_4 - \dots = t_1 + t_3 + \dots + 2t_3 + 4t_5 + \dots$

\pause
$2|E| - 2t_2 - 4t_4 - \dots - 2t_3 - 4t_5 - \dots = t_1 + t_3 + t_5 + \dots$

      \pause
      \item since the left-hand side is even, the right-hand side is also even
    \end{itemize}
  \end{proof}
\end{frame}

\begin{frame}
  \frametitle{Regular Graphs}

  \begin{definition}
    \alert{regular graph}:\\
    all nodes have the same degree

    \begin{itemize}
      \item $n$-regular: all nodes have degree $n$
    \end{itemize}
  \end{definition}
\end{frame}

\begin{frame}
  \frametitle{Regular Graph Example}

  \begin{example}
    \begin{center}
      \pgfuseimage{regular}
    \end{center}
  \end{example}
\end{frame}

\begin{frame}
  \frametitle{Completely Connected Graphs}

  \begin{definition}
    \alert{completely connected graph}:\\
    $\forall v_1,v_2 \in V~(v_1,v_2) \in E$

    \begin{itemize}
      \item $K_n$: a complete graph of $n$ nodes
    \end{itemize}
  \end{definition}
\end{frame}

\begin{frame}
  \frametitle{Completely Connected Graph Examples}

  \begin{columns}
    \column{.5\textwidth}
    \begin{example}[$K_4$]
      \begin{center}
        \pgfuseimage{k4}
      \end{center}
    \end{example}

    \column{.5\textwidth}
    \begin{example}[$K_5$]
      \begin{center}
        \pgfuseimage{k5}
      \end{center}
    \end{example}
  \end{columns}
\end{frame}

\begin{frame}
  \frametitle{Bipartite Graphs}

  \begin{definition}
    \alert{bipartite graph}:\\
    $V = V_1 \cup V_2 \wedge V_1 \cap V_2 = \emptyset$\\
    $\forall (v_1,v_2) \in E~v_1 \in V_1 \wedge v_2 \in V_2$

    \pause
    \begin{itemize}
      \item \emph{complete bipartite graph}:\\
      $\forall v_1 \in V_1 \forall v_2 \in V_2~(v_1,v_2) \in E$
      \begin{itemize}
        \item $K_{m,n}$: $|V_1|=m$, $|V_2|=n$
      \end{itemize}
    \end{itemize}
  \end{definition}
\end{frame}

\begin{frame}
  \frametitle{Bipartite Graph Examples}

  \begin{columns}[t]
    \column{.5\textwidth}
    \begin{example}[$K_{2,3}$]
      \begin{center}
        \pgfuseimage{k23}
      \end{center}
    \end{example}

    \column{.5\textwidth}
    \begin{example}[$K_{3,3}$]
      \begin{center}
        \pgfuseimage{k33}
      \end{center}
    \end{example}
  \end{columns}
\end{frame}

\subsection{Isomorphism}

\begin{frame}
  \frametitle{Subgraph}

  \begin{definition}
    \alert{subgraph}:\\
      if $G'=(V',E')$ is a subgraph of $G=(V,E)$

    \begin{itemize}
      \item $V' \subseteq V$
      \item $E' \subseteq E$
      \item $\forall (v_1,v_2) \in E'~v_1 \in V' \wedge v_2 \in V'$
    \end{itemize}
  \end{definition}
\end{frame}

\begin{frame}
  \frametitle{Isomorphism}

  \begin{definition}
    \alert{isomorphic graphs}:\\
    if $G=(V,E)$ and $G^\star=(V^\star,E^\star)$ are isomorphic\\
    $\exists f: V \rightarrow V^\star~(u,v) \in E \Rightarrow (f(u),f(v)) \in E^\star$

    \begin{itemize}
      \item $f$ is bijective
    \end{itemize}
  \end{definition}

  \pause
  \begin{itemize}
    \item can be drawn the same way
  \end{itemize}
\end{frame}

\begin{frame}
  \frametitle{Isomorphism Example}

  \begin{example}
    \begin{columns}
      \column{.4\textwidth}
      \begin{center}
        \pgfuseimage{isomorphicf}
      \end{center}

      \column{.6\textwidth}
      \begin{center}
        \pgfuseimage{isomorphict}
      \end{center}
    \end{columns}

    \pause
    \bigskip
    \begin{itemize}
      \item $f = \{(a,d),(b,e),(c,b),(d,c),(e,a)\}$
    \end{itemize}
  \end{example}
\end{frame}

\begin{frame}
  \frametitle{Isomorphism Example}

  \begin{example}[Petersen graph]
    \begin{columns}
      \column{.4\textwidth}
      \begin{center}
        \pgfuseimage{petersen1}
      \end{center}

      \column{.6\textwidth}
      \begin{center}
        \pgfuseimage{petersen2}
      \end{center}
    \end{columns}

    \pause
    \bigskip
    \begin{itemize}
      \item $f = \{(a,q),(b,v),(c,u),(d,y),(e,r),$\\
        $~~~~~~~(f,w),(g,x),(h,t),(i,z),(j,s)\}$
    \end{itemize}
  \end{example}
\end{frame}

\begin{frame}
  \frametitle{Homeomorphism}

  \begin{definition}
    \alert{homeomorphic graph}:\\
    graph obtained by dividing an edge in an isomorphic graph\\
    with additional nodes
  \end{definition}
\end{frame}

\begin{frame}
  \frametitle{Homeomorphism Example}

  \begin{example}
    \begin{columns}
      \column{.5\textwidth}
      \begin{center}
        \pgfuseimage{isomorphict}
      \end{center}

      \column{.5\textwidth}
      \begin{center}
        \pgfuseimage{homeomorphict}
      \end{center}
    \end{columns}
  \end{example}
\end{frame}

\subsection{Connectivity}

\begin{frame}
  \frametitle{Walk}

  \begin{definition}
    \alert{walk}:\\
    a sequence of nodes and edges starting at node ($v_0$) and\\
    ending at node ($v_n$) in the form

    \[
      v_0,e_1,v_1,e_2,v_2,e_3,v_3,\dots,e_{n-1},v_{n-1},e_n,v_n
    \]

    where $e_i=(v_{i-1},v_i)$
  \end{definition}

  \pause
  \begin{itemize}
    \item no need to write the edges

    \pause
    \medskip
    \item \alert{length}: number of edges
    \item if $v_0 \neq v_n$ \alert{open}, if $v_0 = v_n$ \alert{closed}
  \end{itemize}
\end{frame}

\begin{frame}
  \frametitle{Walk Example}

  \begin{example}
    \begin{columns}
      \column{.58\textwidth}
      \begin{center}
        \pgfuseimage{plain}
      \end{center}

      \column{.4\textwidth}
      $(c,b),(b,a),(a,d),(d,e),$\\
      $(e,f),(f,a),(a,b)$

      \medskip
      $c,b,a,d,e,f,a,b$

      \bigskip
      length: 7
    \end{columns}
  \end{example}
\end{frame}

\begin{frame}
  \frametitle{Trail}

  \begin{definition}
    \alert{trail}: a walk where edges are not repeated

    \pause
    \begin{itemize}
      \item closed trail: \alert{circuit}
      \item \alert{spanning trail}: a trail that visits all the edges in the
	graph
    \end{itemize}
  \end{definition}
\end{frame}

\begin{frame}
  \frametitle{Trail Example}

  \begin{example}
    \begin{columns}
      \column{.58\textwidth}
      \begin{center}
        \pgfuseimage{plain}
      \end{center}

      \column{.4\textwidth}
      $(c,b),(b,a),(a,e),(e,d),$\\
      $(d,a),(a,f)$

      \medskip
      $c,b,a,e,d,a,f$
    \end{columns}
  \end{example}
\end{frame}

\begin{frame}
  \frametitle{Path}

  \begin{definition}
    \alert{path}: a walk where nodes are not repeated

    \pause
    \begin{itemize}
      \item closed path: \alert{cycle}
      \item \alert{spanning path}: a path that visits all the nodes in the graph
    \end{itemize}
  \end{definition}
\end{frame}

\begin{frame}
  \frametitle{Path Example}

  \begin{example}
    \begin{columns}
      \column{.58\textwidth}
      \begin{center}
        \pgfuseimage{plain}
      \end{center}

      \column{.4\textwidth}
      $(c,b),(b,a),(a,d),(d,e),$\\
      $(e,f)$

      \medskip
      $c,b,a,d,e,f$
    \end{columns}
  \end{example}
\end{frame}

\begin{frame}
  \frametitle{Connectivity}

  \begin{definition}
    \alert{connected graph}:\\
    there is a path between all node pairs
  \end{definition}

  \pause
  \begin{itemize}
    \item a disconnected graph can be separated\\
      into connected components
  \end{itemize}
\end{frame}

\begin{frame}
  \frametitle{Connected Components Example}

  \begin{example}
    \begin{columns}
      \column{.55\textwidth}
      \begin{center}
        \pgfuseimage{disconnected}
      \end{center}

      \column{.43\textwidth}
      \pause
      \begin{itemize}
        \item graph is disconnected:\\
          no path between\\
          $a$ and $c$
        \item connected components:\\
          $a,d,e$\\
          $b,c$\\
          $f$
      \end{itemize}
    \end{columns}
  \end{example}
\end{frame}

\begin{frame}
  \frametitle{Distance}

  \begin{definition}
    \alert{distance}: length of the shortest path between two nodes
  \end{definition}

  \pause
  \begin{definition}
    \alert{diameter}: largest distance in the graph
  \end{definition}
\end{frame}

\begin{frame}
  \frametitle{Distance Example}

  \begin{example}
    \begin{columns}
      \column{.5\textwidth}
      \begin{center}
        \pgfuseimage{distance}
      \end{center}

      \column{.5\textwidth}
      \pause
      \begin{itemize}
        \item distance between $a$ and $e$: 2\\
        \item diameter: 3
      \end{itemize}
    \end{columns}
  \end{example}
\end{frame}

\begin{frame}
  \frametitle{Cut-Point}

  \begin{definition}
    \alert{$G - v$}:\\
    graph obtained by deleting node $v$ and all its incident edges\\
    from graph $G$
  \end{definition}

  \pause
  \begin{definition}
    \alert{cut-point}:\\
    if $G$ is connected but $G - v$ is disconnected then $v$ is a cut-point
  \end{definition}
\end{frame}

\begin{frame}
  \frametitle{Cut-Point Example}

  \begin{columns}
    \column{.5\textwidth}
    \begin{block}{$G$}
      \begin{center}
        \pgfuseimage{distance}
      \end{center}
    \end{block}

    \column{.5\textwidth}
    \begin{block}{$G - d$}
      \begin{center}
        \pgfuseimage{cutpoint}
      \end{center}
    \end{block}
  \end{columns}
\end{frame}

\begin{frame}
  \frametitle{Directed Walks}

  \begin{itemize}
    \item similar to undirected graphs

    \pause
    \item assuming the arcs as undirected edges:\\
      \emph{semi-walk}, \emph{semi-trail}, \emph{semi-path}
  \end{itemize}
\end{frame}

\begin{frame}
  \frametitle{Weakly Connected Graph}

  \begin{columns}
    \column{.5\textwidth}
    \begin{definition}
      \emph{weakly connected}:\\
      there is a semi-path\\
      between each node pair
    \end{definition}

    \column{.5\textwidth}
    \begin{example}
      \begin{center}
        \pgfuseimage{weak}
      \end{center}
    \end{example}
  \end{columns}
\end{frame}

\begin{frame}
  \frametitle{Unilaterally Connected Graph}

  \begin{columns}
    \column{.5\textwidth}
    \begin{definition}
      \emph{unilaterally connected}:\\
      for each node pair, there is\\
      a path from one to the other
    \end{definition}

    \column{.5\textwidth}
    \begin{example}
      \begin{center}
        \pgfuseimage{unilateral}
      \end{center}
    \end{example}
  \end{columns}
\end{frame}

\begin{frame}
  \frametitle{Strongly Connected Graph}

  \begin{columns}
    \column{.5\textwidth}
    \begin{definition}
      \emph{strongly connected}:\\
      there is a path\\
      between each node pair
    \end{definition}

    \column{.5\textwidth}
    \begin{example}
      \begin{center}
        \pgfuseimage{strong}
      \end{center}
    \end{example}
  \end{columns}
\end{frame}
%
% \begin{frame}
%   \frametitle{Directed Connectivity}
%
%   \begin{theorem}
%     It is necessary and sufficient for a finite and directed graph $D$
%
%     \begin{itemize}
%       \item to contain a spanning semi-walk to be weakly connected.
%       \item<2-> to contain a spanning open walk to be unilaterally connected.
%       \item<3-> to contain a spanning closed walk to be strongly connected.
%     \end{itemize}
%   \end{theorem}
% \end{frame}

\begin{frame}
  \frametitle{Bridges of Königsberg}

  \begin{center}
    \pgfuseimage{konigsberg}
  \end{center}

  \begin{itemize}
    \item cross each bridge exactly once\\
      and return to the starting point
  \end{itemize}
\end{frame}

\begin{frame}
  \frametitle{Traversable Graph}

  \begin{definition}
    \alert{traversable graph}:\\
    a graph which contains a spanning trail
  \end{definition}

  \begin{itemize}
    \pause
    \item an odd-degree node must be either the initial\\
      or the terminal node of the trail

    \pause
    \item all nodes except the initial and the terminal nodes\\
      must have even degrees
  \end{itemize}
\end{frame}

\begin{frame}
  \frametitle{Traversable Graph Example}

  \begin{example}
    \begin{columns}
      \column{.4\textwidth}
      \begin{center}
        \pgfuseimage{envelope}
      \end{center}

      \pause
      \column{.6\textwidth}
      \begin{itemize}
        \item degrees of $a$, $b$ and $c$ are even
        \item degrees of $d$ and $e$ are odd
        \pause
        \item a spanning trail can be formed\\
          starting from node $d$ and\\
          ending at node $e$ (or vice versa):\\
          $d,b,a,c,e,d,c,b,e$
      \end{itemize}
    \end{columns}
  \end{example}
\end{frame}

\begin{frame}
  \frametitle{Bridges of Königsberg}

  \begin{center}
    \pgfuseimage{konigsgraph}
  \end{center}

  \pause
  \begin{itemize}
    \item all node degrees are odd: not traversable
  \end{itemize}
\end{frame}

\begin{frame}
  \frametitle{Euler Graphs}

  \begin{definition}
    \alert{Euler graph}:\\
      a graph which contains a spanning circuit

    \pause
    \begin{itemize}
      \item Euler graph $\Leftrightarrow$ degrees of all nodes are even
    \end{itemize}
  \end{definition}
\end{frame}

\begin{frame}
  \frametitle{Euler Graph Examples}

  \begin{columns}
    \column{.5\textwidth}
    \begin{example}[Euler graph]
      \begin{center}
        \pgfuseimage{euler}
      \end{center}
    \end{example}

    \column{.5\textwidth}
    \begin{example}[not an Euler graph]
      \begin{center}
        \pgfuseimage{hamilton}
      \end{center}
    \end{example}
  \end{columns}
\end{frame}

\begin{frame}
  \frametitle{Hamilton Graphs}

  \begin{definition}
    \alert{Hamilton graph}:\\
      a graph which contains a spanning cycle
  \end{definition}
\end{frame}

\begin{frame}
  \frametitle{Hamilton Graph Examples}

  \begin{columns}
    \column{.5\textwidth}
    \begin{example}[Hamilton graph]
      \begin{center}
        \pgfuseimage{hamilton}
      \end{center}
    \end{example}

    \column{.5\textwidth}
    \begin{example}[not a Hamilton graph]
      \begin{center}
        \pgfuseimage{euler}
      \end{center}
    \end{example}
  \end{columns}
\end{frame}

\begin{frame}
  \frametitle{Connectivity Matrix}

  \begin{itemize}
    \item if the adjacency matrix of the graph is $A$,\\
      the $(i,j)$ element of $A^k$ shows the number of walks of length $k$\\
      between the nodes $i$ and $j$

    \pause
    \item in an undirected graph with $n$ nodes,\\
      the distance between two nodes is at most $n-1$

    \pause
    \medskip
    \item \alert{connectivity matrix}:\\
      $C = A^1 + A^2 + A^3 + \dots + A^{n-1}$
    \begin{itemize}
      \item if all elements are non-zero, then the graph is connected
    \end{itemize}
  \end{itemize}
\end{frame}

\begin{frame}
  \frametitle{Warshall's Algorithm}

  \begin{itemize}
    \item it is easier to find whether there is a walk between two nodes\\
      instead of finding the number of walks

    \pause
    \medskip
    \item for each node:
    \begin{itemize}
      \item from all nodes which can reach the chosen node\\
        (the rows that contain 1 in the chosen column)

      \item to the nodes which can be reached from the chosen node\\
        (the columns that contain 1 in the chosen row)
    \end{itemize}
  \end{itemize}
\end{frame}

\begin{frame}
  \frametitle{Warshall's Algorithm Example}

  \begin{example}
    \begin{columns}
      \column{.5\textwidth}
      \begin{center}
        \pgfuseimage{warshall}
      \end{center}

      \column{.5\textwidth}
      \[
        \begin{array}{c|cccc}
              & a & b & c & d\\\hline
            a & 0 & \alert{1} & 0 & 0\\
            b & 0 & 1 & 0 & 0\\
            c & 0 & 0 & 0 & 1\\
            d & \alert{1} & 0 & 1 & 0
        \end{array}
      \]
    \end{columns}
  \end{example}
\end{frame}

\begin{frame}
  \frametitle{Warshall's Algorithm Example}

  \begin{example}
    \begin{columns}
      \column{.5\textwidth}
      \begin{center}
        \pgfuseimage{warshall1}
      \end{center}

      \column{.5\textwidth}
      \[
        \begin{array}{c|cccc}
              & a & b & c & d\\\hline
            a & 0 & \alert{1} & 0 & 0\\
            b & 0 & 1 & 0 & 0\\
            c & 0 & 0 & 0 & 1\\
            d & 1 & \alert{1} & 1 & 0
        \end{array}
      \]
    \end{columns}
  \end{example}
\end{frame}

\begin{frame}
  \frametitle{Warshall's Algorithm Example}

  \begin{example}
    \begin{columns}
      \column{.5\textwidth}
      \begin{center}
        \pgfuseimage{warshall1}
      \end{center}

      \column{.5\textwidth}
      \[
        \begin{array}{c|cccc}
              & a & b & c & d\\\hline
            a & 0 & 1 & 0 & 0\\
            b & 0 & 1 & 0 & 0\\
            c & 0 & 0 & 0 & \alert{1}\\
            d & 1 & 1 & \alert{1} & 0
        \end{array}
      \]
    \end{columns}
  \end{example}
\end{frame}

\begin{frame}
  \frametitle{Warshall's Algorithm Example}

  \begin{example}
    \begin{columns}
      \column{.5\textwidth}
      \begin{center}
        \pgfuseimage{warshall2}
      \end{center}

      \column{.5\textwidth}
      \[
        \begin{array}{c|cccc}
              & a & b & c & d\\\hline
            a & 0 & 1 & 0 & 0\\
            b & 0 & 1 & 0 & 0\\
            c & 0 & 0 & 0 & \alert{1}\\
            d & \alert{1} & \alert{1} & \alert{1} & \alert{1}
        \end{array}
      \]
    \end{columns}
  \end{example}
\end{frame}

\begin{frame}
  \frametitle{Warshall's Algorithm Example}

  \begin{example}
    \begin{columns}
      \column{.5\textwidth}
      \begin{center}
        \pgfuseimage{warshall3}
      \end{center}

      \column{.5\textwidth}
      \[
        \begin{array}{c|cccc}
              & a & b & c & d\\\hline
            a & 0 & 1 & 0 & 0\\
            b & 0 & 1 & 0 & 0\\
            c & 1 & 1 & 1 & 1\\
            d & 1 & 1 & 1 & 1
        \end{array}
      \]
    \end{columns}
  \end{example}
\end{frame}

\subsection{Planar Graphs}

\begin{frame}
  \frametitle{Planar Graphs}

  \begin{definition}
    \alert{planar graph}:\\
    a graph that can be drawn on a plane\\
    without any intersection of its edges

    \begin{itemize}
      \item \alert{map}: a planar drawing of a graph
    \end{itemize}
  \end{definition}
\end{frame}

\begin{frame}
  \frametitle{Planar Graph Example}

  \begin{example}[$K_4$]
    \begin{columns}
      \column{.5\textwidth}
      \begin{center}
        \pgfuseimage{k4}
      \end{center}

      \column{.5\textwidth}
      \begin{center}
        \pgfuseimage{k4planar}
      \end{center}
    \end{columns}
  \end{example}
\end{frame}

\begin{frame}
  \frametitle{Regions}

  \begin{itemize}
    \item a map divides the plane into \emph{regions}
    \item \emph{degree of a region}:\\
      length of the circuit surrounding the region
  \end{itemize}

  \pause
  \begin{theorem}
    if the degree of region $r_i$ is $d_{r_i}$:

    \[ |E| = \frac{\sum_i d_{r_i}}{2} \]
  \end{theorem}
\end{frame}

\begin{frame}
  \frametitle{Region Example}

  \begin{example}
    \begin{columns}
      \column{.58\textwidth}
      \begin{center}
        \pgfuseimage{region}
      \end{center}

      \pause
      \column{.4\textwidth}
      $d_{r_1} = 3$ (abda)\\
      $d_{r_2} = 3$ (bcdb)\\
      $d_{r_3} = 5$ (cdefec)\\
      $d_{r_4} = 4$ (abcea)\\
      $d_{r_5} = 3$ (adea)

      \medskip
      $\sum_r d_r = 18$\\
      $|E| = 9$
    \end{columns}
  \end{example}
\end{frame}

\begin{frame}
  \frametitle{Euler's Formula}

  \begin{theorem}[Euler's Formula]
    In a planar and connected graph $|V| - |E| + |R| = 2$.
  \end{theorem}
\end{frame}

\begin{frame}
  \frametitle{Euler's Formula Example}

  \begin{example}
    \begin{center}
      \pgfuseimage{region}
    \end{center}

    \begin{itemize}
     \item $|V| = 6$, $|E| = 9$, $|R| = 5$
    \end{itemize}
  \end{example}
\end{frame}

\begin{frame}
  \frametitle{Proof of Euler's Formula}

  \begin{block}{Proof}
    method: induction on $|E|$

    \pause
    \begin{itemize}
      \item base step: one node, no edges\\
        $|V| = 1$, $|E| = 0$, $|R| = 1$

      \pause
      \item assume it holds for a connected, planar graph with $k$ nodes
    \end{itemize}
  \end{block}
\end{frame}

\begin{frame}
  \frametitle{Proof of Euler's Formula}

  \begin{proof}[Induction Step]
    \begin{columns}[t]
      \column{.5\textwidth}
      \begin{itemize}
        \item connect a new node\\
	  to an existing node:

        \medskip
        \pgfuseimage{eulerproof1}
      \end{itemize}

      \column{.5\textwidth}
      \begin{itemize}
        \item add an edge between\\
	  two existing nodes:

        \medskip
        \pgfuseimage{eulerproof2}
      \end{itemize}
    \end{columns}

    \pause
    \begin{columns}
      \column{.5\textwidth}
      \begin{itemize}
        \item $|V|$ is increased by 1,\\
	  $|E|$ is increased by 1,\\
          $|R|$ remains the same
      \end{itemize}

      \pause
      \column{.5\textwidth}
      \begin{itemize}
        \item $|V|$ remains the same,\\
	  $|E|$ is increased by 1,\\
          $|R|$ is increased by 1
      \end{itemize}
    \end{columns}
  \end{proof}
\end{frame}

\begin{frame}
  \frametitle{Planar Graph Theorems}

  \begin{theorem}
    In a plain, planar graph:\\
    $|V| \geq 3 \Rightarrow |E| \leq 3 |V| - 6$
  \end{theorem}

  \pause
  \begin{proof}
    \begin{itemize}
      \item the sum of region degrees: $2 |E|$

      \pause
      \item degree of a region is at least $3$\\
        \pause
        $\Rightarrow 2 |E| \geq 3 |R|$
        \pause
        $\Rightarrow |R| \leq \frac{2}{3} |E|$

      \pause
      \item $|V| - |E| + |R| = 2$\\
        \pause
        $\Rightarrow |V| - |E| + \frac{2}{3} |E| \geq 2$
        \pause
        $\Rightarrow |V| - \frac{1}{3} |E| \geq 2$\\
        \pause
        $\Rightarrow 3 |V| - |E| \geq 6$
        \pause
        $\Rightarrow |E| \leq 3 |V| - 6$\\
    \end{itemize}
  \end{proof}
\end{frame}

\begin{frame}
  \frametitle{Planar Graph Theorems}

  \begin{theorem}
    In a connected, plain and planar graph\\
    $|V| \geq 3 \Rightarrow \exists v \in V~d_v \leq 5$
  \end{theorem}

  \pause
  \begin{proof}
    \begin{itemize}
      \item let $\forall v \in V~d_v \geq 6$\\
        \pause
        $\Rightarrow 2 |E| \geq 6 |V|$\\
        \pause
        $\Rightarrow |E| \geq 3 |V|$\\
        \pause
        $\Rightarrow |E| > 3 |V| - 6$: \alert{contradiction}
    \end{itemize}
  \end{proof}
\end{frame}

\begin{frame}
  \frametitle{Nonplanar Graphs}

  \begin{columns}
    \column{.45\textwidth}
    \begin{theorem}
      \begin{center}
        \pgfuseimage{k5}
      \end{center}

      $K_5$ is not planar.
    \end{theorem}

    \pause
    \column{.55\textwidth}
    \begin{proof}
      \begin{itemize}
        \item $|V| = 5$

        \pause
        \item $3 |V| - 6 = 3 \cdot 5 - 6 = 9$

        \pause
        \item so $|E| \leq 9$ \\

        \pause
        \item but $|E| = 10$: \alert{contradiction}
      \end{itemize}
    \end{proof}
  \end{columns}
\end{frame}

\begin{frame}
  \frametitle{Nonplanar Graphs}

  \begin{columns}
    \column{.45\textwidth}
    \begin{theorem}
      \begin{center}
        \pgfuseimage{k33}
      \end{center}

      $K_{3,3}$ is not planar.
    \end{theorem}

    \pause
    \column{.55\textwidth}
    \begin{proof}
      \begin{itemize}
        \item $|V| = 6, |E| = 9$

        \pause
        \item if planar then $|R| = 5$

        \pause
        \item degree of a region is at least $4$\\
          $\Rightarrow \sum_{r \in R} d_r \geq 20$

        \pause
        \item so $|E| \geq 10$\\

        \pause
        \item but $|E| = 9$: \alert{contradiction}
      \end{itemize}
    \end{proof}
  \end{columns}
\end{frame}

\begin{frame}
  \frametitle{Kuratowski's Theorem}

  \begin{theorem}
    The graph has a subgraph hemeomorphic to $K_5$ or $K_{3,3}$\\
    $\Leftrightarrow$ the graph is not planar
  \end{theorem}
\end{frame}

\begin{frame}
  \frametitle{Platonic Solids}

  \begin{definition}
    \alert{regular polyhedron}:\\
    a 3-dimensional solid where the faces are identical regular polygons
  \end{definition}

  \pause
  \begin{itemize}
    \item the projection of a regular polyhedron onto the plane\\
      is a planar graph
    \begin{itemize}
      \item every corner is a node
      \item every side is an edge
    \end{itemize}
  \end{itemize}
\end{frame}

\begin{frame}
  \frametitle{Platonic Solids}

  \begin{example}[cube: regular hexahedron]
    \begin{center}
      \pgfuseimage{hexahedron}
    \end{center}
  \end{example}
\end{frame}

\begin{frame}
  \frametitle{Platonic Solids}

  \begin{itemize}
    \item $v$: number of nodes (corners)
    \item $e$: number of edges (side)
    \item $r$: number of regions (face)
    \item $n$: number of faces incident to a corner = node degree
    \item $m$: number of edges surrounding a face = region degree
  \end{itemize}

  \pause
  \begin{itemize}
    \item $m,n \geq 3$
    \item $2e = m \cdot r$
    \item $2e = n \cdot v$
  \end{itemize}
\end{frame}

\begin{frame}
  \frametitle{Platonic Solids}

    \begin{itemize}
      \item from Euler's formula:
      \[
        0 < 2 = v - e + r = \frac{2e}{n} - e + \frac{2e}{m}
        = e \Big( \frac{2m-mn+2n}{mn} \Big)
      \]

      \pause
      \item Since $e,m,n > 0$:
      \begin{eqnarray*}
        2m - mn + 2n > 0 \Rightarrow mn - 2m -2n < 0 \\\pause
        \Rightarrow mn - 2m - 2n + 4 < 4 \pause \Rightarrow (m - 2)(n - 2) < 4
      \end{eqnarray*}

      \pause
      \item values satisfying the inequation:
      \begin{enumerate}
        \item $m=3, n=3$
        \item $m=4, n=3$
        \item $m=3, n=4$
        \item $m=5, n=3$
        \item $m=3, n=5$
      \end{enumerate}
    \end{itemize}
\end{frame}

\begin{frame}
  \frametitle{Tetrahedron}

  \begin{columns}
    \column{.7\textwidth}
    \begin{center}
      \pgfuseimage{tetrahedron}
    \end{center}

    \column{.3\textwidth}
    \begin{center}
      \pgfuseimage{planartetra}

      $m=3, n=3$
    \end{center}
  \end{columns}
\end{frame}

\begin{frame}
  \frametitle{Hexahedron - Cube}

  \begin{columns}
    \column{.7\textwidth}
    \begin{center}
      \pgfuseimage{hexahedron}
    \end{center}

    \column{.3\textwidth}
    \begin{center}
      \pgfuseimage{planarhexa}

      $m=4, n=3$
    \end{center}
  \end{columns}
\end{frame}

\begin{frame}
  \frametitle{Octahedron}

  \begin{columns}
    \column{.7\textwidth}
    \begin{center}
      \pgfuseimage{octahedron}
    \end{center}

    \column{.3\textwidth}
    \begin{center}
      \pgfuseimage{planarocta}

      $m=3, n=4$
    \end{center}
  \end{columns}
\end{frame}

\begin{frame}
  \frametitle{Dodecahedron}

  \begin{columns}
    \column{.7\textwidth}
    \begin{center}
      \pgfuseimage{dodecahedron}
    \end{center}

    \column{.3\textwidth}
    \begin{center}
      \pgfuseimage{planardodeca}

      $m=5, n=3$
    \end{center}
  \end{columns}
\end{frame}

\begin{frame}
  \frametitle{Icosahedron}

  \begin{columns}
    \column{.7\textwidth}
    \begin{center}
      \pgfuseimage{icosahedron}
    \end{center}

    \column{.3\textwidth}
    $m=3, n=5$
  \end{columns}
\end{frame}

\begin{frame}
  \frametitle{Graph Coloring}

  \begin{definition}
    \alert{proper coloring}:\\
      assign colors to all nodes in a graph $G=(V,E)$\\
      so that for each $(v_1,v_2) \in E$ the colors of $v_1$ and $v_2$ are different

    \pause
    \begin{itemize}
      \item using the minimum number of colors
    \end{itemize}
  \end{definition}
\end{frame}

\begin{frame}
  \frametitle{Graph Coloring Example}

  \begin{example}
    \begin{itemize}
      \item a company produce chemical compounds
      \item some compounds cannot be stored together
      \item such compounds must be placed in different storage areas

      \pause
      \medskip
      \item store the compounds using the least number of storage areas
    \end{itemize}
  \end{example}
\end{frame}

\begin{frame}
  \frametitle{Graph Coloring}

  \begin{example}
    \begin{itemize}
      \item every compound is a node
      \item two compounds that cannot be stored together are adjacent
    \end{itemize}

    \begin{center}
      \pgfuseimage{fivesubstances}
    \end{center}
  \end{example}
\end{frame}

\begin{frame}
  \frametitle{Graph Coloring Example}

  \begin{example}
    \begin{center}
      \pgfuseimage{coloring1}
    \end{center}
  \end{example}
\end{frame}

\begin{frame}
  \frametitle{Graph Coloring Example}

  \begin{example}
    \begin{columns}
      \column{.5\textwidth}
      \begin{center}
        \pgfuseimage{coloring2}
      \end{center}

      \column{.5\textwidth}
      \begin{center}
        \pgfuseimage{coloring3}
      \end{center}
    \end{columns}
  \end{example}
\end{frame}

\begin{frame}
  \frametitle{Graph Coloring Example}

  \begin{example}
    \begin{columns}
      \column{.5\textwidth}
      \begin{center}
        \pgfuseimage{coloring4}
      \end{center}

      \column{.5\textwidth}
      \begin{center}
        \pgfuseimage{coloring5}
      \end{center}
    \end{columns}
  \end{example}
\end{frame}

\begin{frame}
  \frametitle{Graph Coloring Example}

  \begin{example}
    \begin{columns}
      \column{.5\textwidth}
      \begin{center}
        \pgfuseimage{coloring6}
      \end{center}

      \column{.5\textwidth}
      \begin{center}
        \pgfuseimage{coloring7}
      \end{center}
    \end{columns}
  \end{example}
\end{frame}

\begin{frame}
  \frametitle{Chromatic Number}

  \begin{definition}
    \alert{chromatic number}:\\
      Minimum number of colors needed to properly color the graph G: $\chi (G)$
  \end{definition}

  \pause
  \begin{itemize}
     \item Calculating $\chi (G)$ is a very difficult problem
     \item for $n \geq 1$, $\chi (K_n) = n$
  \end{itemize}
\end{frame}

\begin{frame}
  \frametitle{Example of Chromatic Number}

  \begin{example}[Herschel graph]
    \begin{center}
      \pgfuseimage{herschel}
    \end{center}

    \begin{itemize}
      \item chromatic number: 2
    \end{itemize}
  \end{example}
\end{frame}

\begin{frame}
  \frametitle{Graph Coloring Example}

  \begin{example}[Sudoku]
    \begin{columns}[t]
      \column{.4\textwidth}
      \begin{center}
        \pgfuseimage{sudoku}
      \end{center}

      \column{.55\textwidth}
      \begin{itemize}
        \item every cell is a node
        \item cells of the same row are adjacent
        \item cells of the same column are adjacent
        \item cells of the same $3 \times 3$ block are adjacent
        \item every number is a color
      \end{itemize}

      \pause
      \begin{itemize}
        \item problem: properly color a graph that is partially-colored
      \end{itemize}
    \end{columns}
  \end{example}
\end{frame}

\begin{frame}
  \frametitle{Region Coloring}

  \begin{itemize}
    \item coloring a map by assigning different colors to adjacent regions
  \end{itemize}

  \pause
  \medskip
  \begin{theorem}[4~Color Theorem]
    The regions in a planar map can be colored using 4 colors.
  \end{theorem}
\end{frame}

\subsection*{References}

\begin{frame}
  \frametitle{References}

  \begin{block}{Required Reading: Grimaldi}
    \begin{itemize}
      \item Chapter 11: \alert{An Introduction to Graph Theory}

      \item Chapter 7: Relations: The Second Time Around
      \begin{itemize}
        \item 7.2. \alert{Computer Recognition: Zero-One Matrices\\
                          and Directed Graphs}
      \end{itemize}
    \end{itemize}
  \end{block}
\end{frame}

\section{Trees}

\subsection{Introduction}

\begin{frame}
  \frametitle{Tree}

  \begin{definition}
    \alert{tree}: $T=(V,E)$\\
    a connected graph which contains no cycle
  \end{definition}

  \pause
  \begin{itemize}
    \item a graph where the connected components are trees: \emph{forest}
  \end{itemize}
\end{frame}

\begin{frame}
  \frametitle{Tree Examples}

  \begin{example}
    \begin{center}
      \pgfuseimage{tree}
    \end{center}
  \end{example}
\end{frame}

\begin{frame}
  \frametitle{Tree Theorems}

  \begin{theorem}
    In a tree, there is a unique path between any two distinct nodes.
  \end{theorem}

  \begin{itemize}
    \item there is a path because the tree is connected
    \item if there were more than one path, it would cause a cycle:
    \begin{center}
      \pgfuseimage{cycle}
    \end{center}
  \end{itemize}
\end{frame}

\begin{frame}
  \frametitle{Tree Theorems}

  \begin{theorem}
    In a tree $T = (V, E)$: $|V| = |E| + 1$
  \end{theorem}

  \begin{itemize}
    \item proof method: induction on the number of edges
  \end{itemize}
\end{frame}

\begin{frame}
  \frametitle{Tree Theorems}

  \begin{block}{Proof: Base step}
    \begin{itemize}
      \item $|E|=0 \Rightarrow |V|=1$
      \item $|E|=1 \Rightarrow |V|=2$
      \item $|E|=2 \Rightarrow |V|=3$

      \pause
      \medskip
      \item assume that it is true for $|E| \leq k$
    \end{itemize}
  \end{block}
\end{frame}

\begin{frame}
  \frametitle{Tree Theorems}

  \begin{proof}[Proof: Induction step]
    \begin{itemize}
      \item $|E|=k+1$
    \end{itemize}

    \begin{columns}[t]
      \column{.4\textwidth}
      \begin{center}
        \pgfuseimage{nodecount}
      \end{center}

      \pause
      \column{.55\textwidth}
      \begin{itemize}
        \item delete edge $(y,z)$:\\
          $T_1=(V_1,E_1)$, $T_2=(V_2,E_2)$
      \end{itemize}
      \pause
      \begin{eqnarray*}
        |V| & = & |V_1|+|V_2|\\\pause
            & = & |E_1|+1+|E_2|+1\\\pause
            & = & (|E_1|+|E_2|+1)+1\\\pause
            & = & |E|+1
      \end{eqnarray*}
    \end{columns}
  \end{proof}
\end{frame}

\begin{frame}
  \frametitle{Tree Theorems}

  \begin{theorem}
    In a tree, there are at least two nodes with degree 1.
  \end{theorem}

  \pause
  \begin{proof}
    \begin{itemize}
      \item $2 |E| = \sum_{v \in V} d_v$

      \pause
      \item assume that there is only 1 node with degree 1:\\
        \pause
        $\Rightarrow 2 |E| \geq 2 (|V| - 1) + 1$\\
        \pause
        $\Rightarrow 2 |E| \geq 2 |V| - 1$\\
        \pause
        $\Rightarrow |E| \geq |V| - \frac{1}{2}$
        \pause
        $> |V| - 1$ \alert{contradiction}
    \end{itemize}
  \end{proof}
\end{frame}

\begin{frame}
  \frametitle{Tree Theorems}

  \begin{theorem}
    The following statements are equivalent:

    \begin{enumerate}
      \item $T$ is a tree ($T$ is connected and contains no cycle).
      \item There is a unique path between every pair of nodes in $T$.
      \item $T$ is connected, but if any edge is removed\\
	it will no longer be connected.
      \item $T$ contains no cycle, but if an edge is added between\\
	any pair of nodes a unique cycle will be formed.
    \end{enumerate}
  \end{theorem}
\end{frame}

\begin{frame}[label=theoremset2]
  \frametitle{Tree Theorems}

  \begin{theorem}
    The following statements are equivalent:

    \begin{enumerate}
      \item $T$ is a tree ($T$ is connected and contains no cycle).
      \item $T$ is connected and $|E| = |V| - 1$.
      \item $T$ contains no cycle and $|E| = |V| - 1$.
    \end{enumerate}
  \end{theorem}
\end{frame}

\subsection{Rooted Trees}

\begin{frame}
  \frametitle{Rooted Tree}

  \begin{itemize}
    \item there is a hierarchy between nodes

    \pause
    \item natural direction on edges $\Rightarrow$ in and out degrees
    \begin{itemize}
      \item node with in-degree 0 (top of the hierarchy): \alert{root}
      \item nodes with out-degree 0: \alert{leaf}
      \item nodes that are not leaves: \alert{internal node}
    \end{itemize}
  \end{itemize}
\end{frame}

\begin{frame}
  \frametitle{Node Level}

  \begin{definition}
    \alert{level}: distance from the root
  \end{definition}

  \pause
  \begin{itemize}
    \item \alert{parent}: nearest node on the path from the root
    \item \alert{children}: neighboring nodes in the next level
    \item \alert{sibling}: nodes with the same parent
  \end{itemize}
\end{frame}

\begin{frame}
  \frametitle{Rooted Tree Example}

  \begin{example}
    \begin{columns}
      \column{.4\textwidth}
      \begin{center}
        \pgfuseimage{rooted}
      \end{center}

      \pause
      \column{.58\textwidth}
      \begin{itemize}
        \item root: $r$
        \item leaves: $x ~ y ~ z ~ u ~ v$
        \item internal nodes: $r ~ p ~ n ~ t ~ s ~ q ~ w$
        \item parent of $y$: $w$\\
          children of $w$: $y$ and $z$\\
	\item $y$ and $z$ are siblings
      \end{itemize}
    \end{columns}
  \end{example}
\end{frame}

\begin{frame}
  \frametitle{Rooted Tree Example}

  \begin{example}[book order]
    \begin{columns}
      \column{.65\textwidth}
      \begin{center}
        \pgfuseimage{book}
      \end{center}

      \pause
      \column{.33\textwidth}
      Book
      \begin{itemize}
        \item C1
        \begin{itemize}
          \item S1.1
          \item S1.2
        \end{itemize}
        \item C2
        \item C3
        \begin{itemize}
          \item S3.1
          \item S3.2
          \begin{itemize}
            \item S3.2.1
            \item S3.2.2
          \end{itemize}
          \item S3.3
        \end{itemize}
      \end{itemize}
    \end{columns}
  \end{example}
\end{frame}

\begin{frame}
  \frametitle{Ordered Rooted Tree}

  \begin{itemize}
    \item sibling nodes are ordered from left to right

    \pause
    \medskip
    \item \alert{universal address system}
    \begin{itemize}
      \item assign the address $0$ to the root
      \item assign the positive integers $1,2,3,\dots$ to the nodes at level 1,\\
        from left to right
      \item let $v$ be an internal node with address $a$,\\
        assign the addresses $a.1,a.2,a.3,\dots$ to the children of $v$\\
        from left to right
    \end{itemize}
  \end{itemize}
\end{frame}

\begin{frame}
  \frametitle{Lexicographic Order}

  \begin{itemize}
    \item let $b$ and $c$ be two addresses
  \end{itemize}

  \begin{definition}
    for \alert{$b < c$}:
    \begin{enumerate}
      \item $b=a_1.a_2. \dots .a_m$\\
        $c=a_1.a_2. \dots .a_m.a_{m+1} \dots a_n$
      \pause
      \item $b=a_1.a_2. \dots .a_m.x_1 \dots y$\\
        $c=a_1.a_2. \dots .a_m.x_2 \dots z$\\
        $x_1 < x_2$
    \end{enumerate}
  \end{definition}
\end{frame}

\begin{frame}
  \frametitle{Lexicographic Order Example}

  \begin{example}
    \begin{columns}
      \column{.57\textwidth}
      \begin{center}
        \pgfuseimage{dictionary}
      \end{center}

      \pause
      \column{.4\textwidth}
      \begin{itemize}
        \item 0 - 1 - 1.1 - 1.2\\
          - 1.2.1 - 1.2.2 - 1.2.3\\
          - 1.2.3.1 - 1.2.3.2\\
          - 1.3 - 1.4 - 2\\
          - 2.1 - 2.2 - 2.2.1\\
          - 3 - 3.1 - 3.2
      \end{itemize}
    \end{columns}
  \end{example}
\end{frame}

\begin{frame}
  \frametitle{Binary Trees}

  \begin{definition}
    \alert{binary tree}:\\
    $\forall v \in V~{d_v}^o \in \{0,1,2\}$
  \end{definition}

  \pause
  \begin{definition}
    \alert{complete binary tree}:\\
    $\forall v \in V~{d_v}^o \in \{0,2\}$
  \end{definition}
\end{frame}

\begin{frame}
  \frametitle{Expression Tree}

  \begin{itemize}
    \item binary operations can be represented by complete binary trees
    \begin{itemize}
      \item operator as the root, operands as the children
    \end{itemize}

    \pause
    \medskip
    \item every mathematical expression can be represented\\
      as a binary tree
    \begin{itemize}
      \item operators at internal nodes, variables and values at the leaves
      \item \emph{does not have to be a complete binary tree}
    \end{itemize}
  \end{itemize}
\end{frame}

\begin{frame}
  \frametitle{Expression Tree Examples}

  \begin{columns}[t]
    \column{.5\textwidth}
    \begin{example}[$7-a$]
      \begin{center}
        \pgfuseimage{expr1a}
      \end{center}
    \end{example}

    \column{.5\textwidth}
    \begin{example}[$a+b$]
      \begin{center}
        \pgfuseimage{expr1b}
      \end{center}
    \end{example}
  \end{columns}
\end{frame}

\begin{frame}
  \frametitle{Expression Tree Examples}

  \begin{columns}[t]
    \column{.5\textwidth}
    \begin{example}[$(7-a)/5$]
      \begin{center}
        \pgfuseimage{expr2a}
      \end{center}
    \end{example}

    \column{.5\textwidth}
    \begin{example}[$(a+b) \uparrow 3$]
      \begin{center}
        \pgfuseimage{expr2b}
      \end{center}
    \end{example}
  \end{columns}
\end{frame}

\begin{frame}
  \frametitle{Expression Tree Examples}

  \begin{example}[$((7-a)/5)*((a+b) \uparrow 3)$]
    \begin{center}
      \pgfuseimage{expr3}
    \end{center}
  \end{example}
\end{frame}

\begin{frame}
  \frametitle{Expression Tree Examples}

  \begin{example}[$t+(u*v)/(w+x-y \uparrow z)$]
    \begin{center}
      \pgfuseimage{expr}
    \end{center}
  \end{example}
\end{frame}

\begin{frame}
  \frametitle{Expression Tree Traversals}

  \begin{enumerate}
    \item \alert{inorder traversal}: traverse the left subtree, visit the root,\\
      traverse the right subtree

    \pause
    \medskip
    \item \alert{preorder traversal}: visit the root, traverse the left subtree,\\
      traverse the right subtree

    \pause
    \medskip
    \item \alert{postorder traversal}: traverse the left subtree,\\
      traverse the right subtree, visit the root
    \begin{itemize}
      \item \emph{reverse Polish notation}
    \end{itemize}
  \end{enumerate}
\end{frame}

\begin{frame}
  \frametitle{Preorder Traversal Example}

  \begin{example}
    \begin{columns}
      \column{.4\textwidth}
      \begin{center}
        \pgfuseimage{expr}
      \end{center}

      \pause
      \column{.6\textwidth}
      $+ ~ t ~ / ~ * ~ u ~ v ~ + ~ w ~ - ~ x ~ \uparrow ~ y ~ z$
    \end{columns}
  \end{example}
\end{frame}

\begin{frame}
  \frametitle{Inorder Traversal Example}

  \begin{example}
    \begin{columns}
      \column{.4\textwidth}
      \begin{center}
        \pgfuseimage{expr}
      \end{center}

      \pause
      \column{.6\textwidth}
      $t ~ + ~ u ~ * ~ v ~ / ~ w ~ + ~ x ~ - ~ y ~ \uparrow ~ z$
    \end{columns}
  \end{example}
\end{frame}

\begin{frame}
  \frametitle{Postorder Traversal Example}

  \begin{example}
    \begin{columns}
      \column{.4\textwidth}
      \begin{center}
        \pgfuseimage{expr}
      \end{center}

      \pause
      \column{.6\textwidth}
      $t ~ u ~ v ~ * ~ w ~ x ~ y ~ z ~ \uparrow ~ - ~ + ~ / ~ +$
    \end{columns}
  \end{example}
\end{frame}

\begin{frame}
  \frametitle{Expression Tree Evaluation}

  \begin{itemize}
    \item precedence in an expression tree:
    \begin{itemize}
      \item inorder traversal requires parantheses
      \item preorder and postorder traversals do not require parantheses
    \end{itemize}
  \end{itemize}
\end{frame}

\begin{frame}
  \frametitle{Postorder Evaluation Example}

  \begin{example}[$t ~ u ~ v ~ * ~ w ~ x ~ y ~ z ~ \uparrow ~ - ~ + ~ / ~ +$]
    $4 ~ 2 ~ 3 ~ * ~ 1 ~ 9 ~ 2 ~ 3 ~ \uparrow ~ - ~ + ~ / ~ +$

    \pause
    \medskip
    \[
      \begin{array}{ccccccc}
  4 & \pause 2 & \pause 3 & \pause * &          &          &                \\
  4 &        6 & \pause 1 & \pause 9 & \pause 2 & \pause 3 & \pause \uparrow\\
  4 &        6 &        1 &        9 &        8 & \pause - &                \\
  4 &        6 &        1 &        1 & \pause + &          &                \\
  4 &        6 &        2 & \pause / &          &          &                \\
  4 &        3 & \pause + &          &          &          &                \\
  7 &          &          &          &          &          &                \\
      \end{array}
    \]
  \end{example}
\end{frame}

\subsection{Searching Graphs}

\begin{frame}
  \frametitle{Searching Graphs}

  \begin{itemize}
    \item searching nodes of a graph $G=(V,E)$\\
     starting from node $v_1$
    \begin{itemize}
      \item depth-first
      \item breadth-first
    \end{itemize}
  \end{itemize}
\end{frame}

\begin{frame}
  \frametitle{Depth-First Search}

  \begin{enumerate}
    \item $v \leftarrow v_1, T=\emptyset$, $D=\{v_1\}$

    \pause
    \item find smallest $i$ in $2 \leq i \leq |V|$
      such that $(v,v_i) \in E$ and $v_i \notin D$
      \begin{itemize}
        \item if no such $i$ exists: go to step 3
        \item if found: $T=T \cup \{(v,v_i)\}$, $D=D \cup \{v_i\}$,
          $v \leftarrow v_i$,\\
          go to step 2
      \end{itemize}

    \pause
    \item if $v=v_1$ then the result is $T$

    \pause
    \item if $v \neq v_1$ then $v \leftarrow parent(v)$, go to step 2
  \end{enumerate}
\end{frame}

\begin{frame}
  \frametitle{Breadth-First Search}

  \begin{enumerate}
    \item $T=\emptyset$, $D=\{v_1\}$, $Q=(v_1)$

    \pause
    \item if $Q$ is empty: the result is $T$
    \item if $Q$ not empty: $v \leftarrow front(Q)$, $Q \leftarrow Q - v$\\
      for $2 \leq i \leq |V|$ check the edges $(v,v_i) \in E$:
    \begin{itemize}
      \item if $v_i \notin D$ : $Q = Q + v_i$, $T = T \cup \{(v,v_i)\}$,
        $D=D \cup \{v_i\}$
       \item go to step 3
    \end{itemize}
  \end{enumerate}
\end{frame}

\subsection{Regular Trees}

\begin{frame}
  \frametitle{Regular Tree}

  \begin{definition}
    \alert{$m$-ary tree}:\\
    all internal nodes have out-degree $m$
  \end{definition}
\end{frame}

\begin{frame}
  \frametitle{Regular Tree Theorems}

  \begin{theorem}
    in an $m$-ary tree

    \begin{itemize}
      \item $n$: number of nodes
      \item $l$: number of leaves
      \item $i$: number of ınternal nodes
    \end{itemize}

    then

    \begin{itemize}
      \item $n = m \cdot i + 1$

      \pause
      \item  $l = n - i = \pause m \cdot i + 1 - i
        \pause = (m - 1) \cdot i + 1$

      \pause
      \[
        i = \frac{l - 1}{m - 1}
      \]
    \end{itemize}
  \end{theorem}
\end{frame}

\begin{frame}
  \frametitle{Regular Tree Examples}

  \begin{example}
    How many matches are played in a tennis tournament\\
    with 27 players?

    \pause
    \bigskip
    \begin{itemize}
      \item every player is a leaf: $l = 27$
      \item every match is an internal node: $m = 2$

      \pause
      \item number of matches: $i = \frac{l - 1}{m - 1} = \frac{27 - 1}{2 - 1} = 26$
    \end{itemize}
  \end{example}
\end{frame}

\begin{frame}
  \frametitle{Regular Tree Examples}

  \begin{example}
    How many extension cords with 4 outlets are required\\
    to connect 25 computers to a wall socket?

    \pause
    \bigskip
    \begin{itemize}
      \item every computer is a leaf: $l = 25$
      \item every extension cord is an internal node: $m = 4$

      \pause
      \item number of cords: $i = \frac{l - 1}{m - 1} = \frac{25 - 1}{4 - 1} = 8$
    \end{itemize}
  \end{example}
\end{frame}

\begin{frame}
  \frametitle{Decision Trees}

  \begin{example}[counterfeit coin problem]
    \begin{itemize}
      \item one of 8 coins is counterfeit (is heavier)
      \item find the counterfeit coin using a beam balance
    \end{itemize}
  \end{example}
\end{frame}

\begin{frame}
  \frametitle{Decision Trees}

  \begin{example}[in 3 weighings]
    \begin{center}
      \pgfuseimage{scale1}
    \end{center}
  \end{example}
\end{frame}

\begin{frame}
  \frametitle{Decision Trees}

  \begin{example}[in 2 weighings]
    \begin{center}
      \pgfuseimage{scale2}
    \end{center}
  \end{example}
\end{frame}

\subsection*{References}

\begin{frame}
  \frametitle{References}

  \begin{block}{Required Reading: Grimaldi}
    \begin{itemize}
      \item Chapter 12: Trees
      \begin{itemize}
        \item 12.1. \alert{Definitions and Examples}
        \item 12.2. \alert{Rooted Trees}
      \end{itemize}
    \end{itemize}
  \end{block}
\end{frame}

\section{Weighted Graphs}

\subsection{Shortest Path}

\begin{frame}
  \frametitle{Shortest Path}

  \begin{itemize}
    \item Dijkstra's algorithm finds the shortest paths\\
      from a node to all other nodes
  \end{itemize}
\end{frame}

\begin{frame}
  \frametitle{Dijkstra's Algorithm Example}

  \begin{example}[initialization]
    \begin{columns}
      \column{.5\textwidth}
      \begin{center}
        \pgfuseimage{dijkstra}
      \end{center}

      \column{.45\textwidth}
      \begin{itemize}
        \item starting node: $c$
      \end{itemize}

      \begin{table}
        \begin{tabular}{r|l}
          a & $(\infty,-)$ \\\hline
          b & $(\infty,-)$ \\\hline
          c & $(0,-)$      \\\hline
          f & $(\infty,-)$ \\\hline
          g & $(\infty,-)$ \\\hline
          h & $(\infty,-)$
        \end{tabular}
      \end{table}
    \end{columns}
  \end{example}
\end{frame}

\begin{frame}
  \frametitle{Dijkstra's Algorithm Example}

  \begin{example}[From node $c$ - base distance=$0$]
    \begin{columns}
      \column{.5\textwidth}
      \begin{center}
        \pgfuseimage{dijkstra}
      \end{center}

      \column{.45\textwidth}
      \begin{itemize}
        \item $c \rightarrow f: 6, 6 < \infty$
        \item $c \rightarrow h: 11, 11 < \infty$
      \end{itemize}

      \pause
      \begin{table}
        \begin{tabular}{r|l|c}
          a & $(\infty,-)$ & \\\hline
          b & $(\infty,-)$ & \\\hline
          c & $(0,-)$      & $\surd$ \\\hline
          f & $(6,cf)$     & \\\hline
          g & $(\infty,-)$ & \\\hline
          h & $(11,ch)$    &
        \end{tabular}
      \end{table}

      \pause
      \begin{itemize}
        \item closest node: $f$
      \end{itemize}
    \end{columns}
  \end{example}
\end{frame}

\begin{frame}
  \frametitle{Dijkstra's Algorithm Example}

  \begin{example}[from node $f$ - base distance=$6$]
    \begin{columns}
      \column{.5\textwidth}
      \begin{center}
        \pgfuseimage{dijkstra}
      \end{center}

      \column{.45\textwidth}
      \begin{itemize}
        \item $f \rightarrow a: 6+11, 17 < \infty$
        \item $f \rightarrow g: 6+9, 15 < \infty$
        \item $f \rightarrow h: 6+4, 10 < 11$
      \end{itemize}

      \pause
      \begin{table}
        \begin{tabular}{r|l|c}
          a & $(17,cfa)$   & \\\hline
          b & $(\infty,-)$ & \\\hline
          c & $(0,-)$      & $\surd$ \\\hline
          f & $(6,cf)$     & $\surd$ \\\hline
          g & $(15,cfg)$   & \\\hline
          h & $(10,cfh)$   &
        \end{tabular}
      \end{table}

      \pause
      \begin{itemize}
        \item closest node: $h$
      \end{itemize}
    \end{columns}
  \end{example}
\end{frame}

\begin{frame}
  \frametitle{Dijkstra's Algorithm Example}

  \begin{example}[from node $h$ - base distance=$10$]
    \begin{columns}
      \column{.5\textwidth}
      \begin{center}
        \pgfuseimage{dijkstra}
      \end{center}

      \column{.45\textwidth}
      \begin{itemize}
        \item $h \rightarrow a: 10+11, 21 \nless 17$
        \item $h \rightarrow g: 10+4, 14 < 15$
      \end{itemize}

      \pause
      \begin{table}
        \begin{tabular}{r|l|c}
          a & $(17,cfa)$   & \\\hline
          b & $(\infty,-)$ & \\\hline
          c & $(0,-)$      & $\surd$ \\\hline
          f & $(6,cf)$     & $\surd$ \\\hline
          g & $(14,cfhg)$  & \\\hline
          h & $(10,cfh)$   & $\surd$
        \end{tabular}
      \end{table}

      \pause
      \begin{itemize}
        \item closest node: $g$
      \end{itemize}
    \end{columns}
  \end{example}
\end{frame}

\begin{frame}
  \frametitle{Dijkstra's Algorithm Example}

  \begin{example}[from node $g$ - base distance=$14$]
    \begin{columns}
      \column{.5\textwidth}
      \begin{center}
        \pgfuseimage{dijkstra}
      \end{center}

      \column{.45\textwidth}
      \begin{itemize}
        \item $g \rightarrow a: 14+17, 31 \nless 17$
      \end{itemize}

      \pause
      \begin{table}
        \begin{tabular}{r|l|c}
          a & $(17,cfa)$   & \\\hline
          b & $(\infty,-)$ & \\\hline
          c & $(0,-)$      & $\surd$ \\\hline
          f & $(6,cf)$     & $\surd$ \\\hline
          g & $(14,cfhg)$  & $\surd$ \\\hline
          h & $(10,cfh)$   & $\surd$
        \end{tabular}
      \end{table}

      \pause
      \begin{itemize}
        \item closest node: $a$
      \end{itemize}
    \end{columns}
  \end{example}
\end{frame}

\begin{frame}
  \frametitle{Dijkstra's Algorithm Example}

  \begin{example}[from node $a$ - base distance=$17$]
    \begin{columns}
      \column{.5\textwidth}
      \begin{center}
        \pgfuseimage{dijkstra}
      \end{center}

      \column{.45\textwidth}
      \begin{itemize}
        \item $a \rightarrow b: 17+5, 22 < \infty$
      \end{itemize}

      \pause
      \begin{table}
        \begin{tabular}{r|l|c}
          a & $(17,cfa)$   & $\surd$ \\\hline
          b & $(22,cfab)$  & \\\hline
          c & $(0,-)$      & $\surd$ \\\hline
          f & $(6,cf)$     & $\surd$ \\\hline
          g & $(14,cfhg)$  & $\surd$ \\\hline
          h & $(10,cfh)$   & $\surd$
        \end{tabular}
      \end{table}

      \pause
      \begin{itemize}
        \item last node: $b$
      \end{itemize}
    \end{columns}
  \end{example}
\end{frame}

\subsection{Minimum Spanning Tree}

\begin{frame}
  \frametitle{Spanning Tree}

  \begin{definition}
    \alert{spanning tree}:\\
    a subgraph which is a tree and contains all the nodes of the graph
  \end{definition}

  \pause
  \begin{definition}
    \alert{minimum spanning tree}:\\
    a spanning tree for which the total weight of edges is minimal
  \end{definition}
\end{frame}

\begin{frame}
  \frametitle{Kruskal's Algorithm}

  \begin{block}{Kruskal's algorithm}
    \begin{enumerate}
      \item $i \leftarrow 1$, $e_1 \in E$, $wt(e_1)$ is minimal

      \pause
      \item for $1 \leq i \leq n-2$:\\
        the selected edges are $e_1,e_2,\dots,e_i$\\
        select a new edge $e_{i+1}$ from the remaining edges such that:
      \begin{itemize}
        \item $wt(e_{i+1})$ is minimal
        \item $e_1,e_2,\dots,e_i,e_{i+1}$ contains no cycle
      \end{itemize}

      \pause
      \item $i \leftarrow i+1$
      \begin{itemize}
        \item $i=n-1$ $\Rightarrow$ the subgraph $G$ containing the edges\\
         $e_1,e_2,\dots,e_{n-1}$ is a minimum spanning tree
        \item $i<n-1$ $\Rightarrow$ go to step 2
      \end{itemize}
    \end{enumerate}
  \end{block}
\end{frame}

\begin{frame}
  \frametitle{Kruskal's Algorithm Example}

  \begin{example}[initialization]
    \begin{columns}
      \column{.4\textwidth}
      \begin{center}
        \pgfuseimage{spanning}
      \end{center}

      \pause
      \column{.6\textwidth}
      \begin{itemize}
        \item $i \leftarrow 1$
        \item minimum weight: $1$\\
          $(e,g)$

        \pause
        \item $T = \{ (e,g) \}$
      \end{itemize}
    \end{columns}
  \end{example}
\end{frame}

\begin{frame}
  \frametitle{Kruskal's Algorithm Example}

  \begin{example}[$1 < 6$]
    \begin{columns}
      \column{.4\textwidth}
      \begin{center}
        \pgfuseimage{kruskal1}
      \end{center}

      \pause
      \column{.6\textwidth}
      \begin{itemize}
        \item minimum weight: $2$\\
          $(d,e), (d,f), (f,g)$

        \pause
        \item $T = \{ (e,g), (d,f) \}$
        \item $i \leftarrow 2$
      \end{itemize}
    \end{columns}
  \end{example}
\end{frame}

\begin{frame}
  \frametitle{Kruskal's Algorithm Example}

  \begin{example}[$2 < 6$]
    \begin{columns}
      \column{.4\textwidth}
      \begin{center}
        \pgfuseimage{kruskal2}
      \end{center}

      \pause
      \column{.6\textwidth}
      \begin{itemize}
        \item minimum weight: $2$\\
          $(d,e), (f,g)$

        \pause
        \item $T = \{ (e,g), (d,f), (d,e) \}$
        \item $i \leftarrow 3$
      \end{itemize}
    \end{columns}
  \end{example}
\end{frame}

\begin{frame}
  \frametitle{Kruskal's Algorithm Example}

  \begin{example}[$3 < 6$]
    \begin{columns}
      \column{.4\textwidth}
      \begin{center}
        \pgfuseimage{kruskal3}
      \end{center}

      \pause
      \column{.6\textwidth}
      \begin{itemize}
        \item minimum weight: $2$\\
          $(f,g)$ forms a cycle

        \pause
        \item minimum weight: $3$\\
          $(c,e), (c,g), (d,g)$\\
          $(d,g)$ forms a cycle

        \pause
        \item $T = \{ (e,g), (d,f), (d,e), (c,e) \}$
        \item $i \leftarrow 4$
      \end{itemize}
    \end{columns}
  \end{example}
\end{frame}

\begin{frame}
  \frametitle{Kruskal's Algorithm Example}

  \begin{example}[$4 < 6$]
    \begin{columns}
      \column{.4\textwidth}
      \begin{center}
        \pgfuseimage{kruskal4}
      \end{center}

      \pause
      \column{.6\textwidth}
      \begin{itemize}
        \item $T = \{$\\
          $~~(e,g), (d,f), (d,e),$\\
          $~~(c,e), (b,e)$\\
          $\}$
        \item $i \leftarrow 5$
      \end{itemize}
    \end{columns}
  \end{example}
\end{frame}

\begin{frame}
  \frametitle{Kruskal's Algorithm Example}

  \begin{example}[$5 < 6$]
    \begin{columns}
      \column{.4\textwidth}
      \begin{center}
        \pgfuseimage{kruskal5}
      \end{center}

      \pause
      \column{.6\textwidth}
      \begin{itemize}
        \item $T = \{$\\
          $~~(e,g), (d,f), (d,e),$\\
          $~~(c,e), (b,e), (a,b)$\\
          $\}$
        \item $i \leftarrow 6$
      \end{itemize}
    \end{columns}
  \end{example}
\end{frame}

\begin{frame}
  \frametitle{Kruskal's Algorithm Example}

  \begin{example}[$6 \nless 6$]
    \begin{columns}
      \column{.4\textwidth}
      \begin{center}
        \pgfuseimage{kruskal}
      \end{center}

      \column{.6\textwidth}
      \begin{itemize}
        \item total weight: $17$
      \end{itemize}
    \end{columns}
  \end{example}
\end{frame}

\begin{frame}
  \frametitle{Prim's Algorithm}

  \begin{block}{Prim's algorithm}
    \begin{enumerate}
      \item $i \leftarrow 1$, $v_1 \in V$, $P=\{v_1\}$, $N=V-\{v_1\}$,
        $T=\emptyset$

      \pause
      \item for $1 \leq i \leq n-1$:\\
        $P=\{v_1,v_2,\dots,v_i\}$, $T=\{e_1,e_2,\dots,e_{i-1}\}$, $N=V-P$\\
        select a node $v_{i+1} \in N$ such that for a node $x \in P$\\
        $e=(x,v_{i+1}) \notin T$, $wt(e)$ is minimal\\
        $P \leftarrow P+\{v_{i+1}\}$, $N \leftarrow N-\{v_{i+1}\}$,
        $T \leftarrow T+\{e\}$

      \pause
      \item $i \leftarrow i+1$
      \begin{itemize}
        \item $i=n$ $\Rightarrow$: the subgraph $G$ containing the edges\\
	  $e_1,e_2,\dots,e_{n-1}$ is a minimum spanning tree
        \item $i<n$ $\Rightarrow$ go to step 2

      \end{itemize}
    \end{enumerate}
  \end{block}
\end{frame}

\begin{frame}
  \frametitle{Prim's Algorithm Example}

  \begin{example}[initialization]
    \begin{columns}
      \column{.4\textwidth}
      \begin{center}
        \pgfuseimage{spanning}
      \end{center}

      \pause
      \column{.6\textwidth}
      \begin{itemize}
        \item $i \leftarrow 1$
        \item $P = \{ a \}$
        \item $N = \{ b, c, d, e, f, g \}$
        \item $T = \emptyset$
      \end{itemize}
    \end{columns}
  \end{example}
\end{frame}

\begin{frame}
  \frametitle{Prim's Algorithm Example}

  \begin{example}[$1 < 7$]
    \begin{columns}
      \column{.4\textwidth}
      \begin{center}
        \pgfuseimage{spanning}
      \end{center}

      \pause
      \column{.6\textwidth}
      \begin{itemize}
        \item $T = \{ (a,b) \}$
        \item $P = \{ a, b \}$
        \item $N = \{ c, d, e, f, g \}$
        \item $i \leftarrow 2$
      \end{itemize}
    \end{columns}
  \end{example}
\end{frame}

\begin{frame}
  \frametitle{Prim's Algorithm Example}

  \begin{example}[$2 < 7$]
    \begin{columns}
      \column{.4\textwidth}
      \begin{center}
        \pgfuseimage{prim1}
      \end{center}

      \pause
      \column{.6\textwidth}
      \begin{itemize}
        \item $T = \{ (a,b), (b,e) \}$
        \item $P = \{ a, b, e \}$
        \item $N = \{ c, d, f, g \}$
        \item $i \leftarrow 3$
      \end{itemize}
    \end{columns}
  \end{example}
\end{frame}

\begin{frame}
  \frametitle{Prim's Algorithm Example}

  \begin{example}[$3 < 7$]
    \begin{columns}
      \column{.4\textwidth}
      \begin{center}
        \pgfuseimage{prim2}
      \end{center}

      \pause
      \column{.6\textwidth}
      \begin{itemize}
        \item $T = \{ (a,b), (b,e), (e,g) \}$
        \item $P = \{ a, b, e, g \}$
        \item $N = \{ c, d, f \}$
        \item $i \leftarrow 4$
      \end{itemize}
    \end{columns}
  \end{example}
\end{frame}

\begin{frame}
  \frametitle{Prim's Algorithm Example}

  \begin{example}[$4 < 7$]
    \begin{columns}
      \column{.4\textwidth}
      \begin{center}
        \pgfuseimage{prim3}
      \end{center}

      \pause
      \column{.6\textwidth}
      \begin{itemize}
        \item $T = \{ (a,b), (b,e), (e,g), (d,e) \}$
        \item $P = \{ a, b, e, g, d \}$
        \item $N = \{ c, f \}$
        \item $i \leftarrow 5$
      \end{itemize}
    \end{columns}
  \end{example}
\end{frame}

\begin{frame}
  \frametitle{Prim's Algorithm Example}

  \begin{example}[$5 < 7$]
    \begin{columns}
      \column{.4\textwidth}
      \begin{center}
        \pgfuseimage{prim4}
      \end{center}

      \pause
      \column{.6\textwidth}
      \begin{itemize}
        \item $T = \{$\\
          $~~(a,b), (b,e), (e,g),$\\
          $~~(d,e), (f,g)$\\
          $\}$
        \item $P = \{ a, b, e, g, d, f \}$
        \item $N = \{ c \}$
        \item $i \leftarrow 6$
      \end{itemize}
    \end{columns}
  \end{example}
\end{frame}

\begin{frame}
  \frametitle{Prim's Algorithm Example}

  \begin{example}[$6 < 7$]
    \begin{columns}
      \column{.4\textwidth}
      \begin{center}
        \pgfuseimage{prim5}
      \end{center}

      \pause
      \column{.6\textwidth}
      \begin{itemize}
        \item $T = \{$\\
          $~~(a,b), (b,e), (e,g),$\\
          $~~(d,e), (f,g), (c,g)$\\
          $\}$
        \item $P = \{ a, b, e, g, d, f, c \}$
        \item $N = \emptyset$
        \item $i \leftarrow 7$
      \end{itemize}
    \end{columns}
  \end{example}
\end{frame}

\begin{frame}
  \frametitle{Prim's Algorithm Example}

  \begin{example}[$7 \nless 7$]
    \begin{columns}
      \column{.4\textwidth}
      \begin{center}
        \pgfuseimage{prim}
      \end{center}

      \column{.6\textwidth}
      \begin{itemize}
        \item total weight: $17$
      \end{itemize}
    \end{columns}
  \end{example}
\end{frame}

\subsection*{References}

\begin{frame}
  \frametitle{References}

  \begin{block}{Required Reading: Grimaldi}
    \begin{itemize}
      \item Chapter 13: Optimization and Matching
      \begin{itemize}
        \item 13.1. \alert{Dijkstra's Shortest Path Algorithm}
        \item 13.2. \alert{Minimal Spanning Trees:\\
			   The Algorithms of Kruskal and Prim}
      \end{itemize}
    \end{itemize}
  \end{block}
\end{frame}

\end{document}
