% Copyright (c) 2001-2014
%       H. Turgut Uyar <uyar@itu.edu.tr>
%       Ayşegül Gençata Yayımlı <gencata@itu.edu.tr>
%       Emre Harmancı <harmanci@itu.edu.tr>
%
% This work is licensed under a "Creative Commons
% Attribution-NonCommercial-ShareAlike 4.0 International License".
% For more information, please visit:
% https://creativecommons.org/licenses/by-nc-sa/4.0/

\documentclass[dvipsnames]{beamer}

\usepackage{ae}
\usepackage[T1]{fontenc}
\usepackage[utf8]{inputenc}
\setbeamertemplate{navigation symbols}{}
\setbeamersize{text margin left=2em, text margin right=2em}

\mode<presentation>
{
  \usetheme{Rochester}
  \setbeamercovered{transparent}
}

\title{Discrete Mathematics}
\subtitle{Graphs}

\author{H. Turgut Uyar \and Ayşegül Gençata Yayımlı \and Emre Harmancı}
\date{2001-2014}

\AtBeginSubsection[]
{
  \begin{frame}<beamer>
    \frametitle{Topics}
    \tableofcontents[currentsection,currentsubsection]
  \end{frame}
}

%\beamerdefaultoverlayspecification{<+->}

\pgfdeclareimage[width=2cm]{license}{../license}

\pgfdeclareimage[width=6cm]{plain}{plain}
\pgfdeclareimage[width=4cm]{directed}{directed}
\pgfdeclareimage[width=6cm]{multi}{multi}
\pgfdeclareimage[width=4cm]{matrix}{matrix}
\pgfdeclareimage[height=4cm]{isomorphicf}{isomorphicf}
\pgfdeclareimage[height=4cm]{isomorphict}{isomorphict}
\pgfdeclareimage[height=3.5cm]{petersen1}{petersen1}
\pgfdeclareimage[height=3.5cm]{petersen2}{petersen2}
\pgfdeclareimage[height=4cm]{homeomorphict}{homeomorphict}
\pgfdeclareimage[height=4.5cm]{regular}{regular}
\pgfdeclareimage[width=4cm]{k4}{k4}
\pgfdeclareimage[width=4cm]{k5}{k5}
\pgfdeclareimage[width=4cm]{k23}{k23}
\pgfdeclareimage[width=4cm]{k33}{k33}
\pgfdeclareimage[width=6cm]{disconnected}{disconnected}
\pgfdeclareimage[width=5cm]{distance}{distance}
\pgfdeclareimage[width=5cm]{cutpoint}{cutpoint}
\pgfdeclareimage[width=4cm]{weak}{weak}
\pgfdeclareimage[width=4cm]{unilateral}{unilateral}
\pgfdeclareimage[width=4cm]{strong}{strong}
\pgfdeclareimage[width=5cm]{warshall}{warshall}
\pgfdeclareimage[width=5cm]{warshall1}{warshall1}
\pgfdeclareimage[width=5cm]{warshall2}{warshall2}
\pgfdeclareimage[width=5cm]{warshall3}{warshall3}
\pgfdeclareimage{konigsberg}{konigsberg}
\pgfdeclareimage[width=3cm]{envelope}{envelope}
\pgfdeclareimage{konigsgraph}{konigsgraph}
\pgfdeclareimage[width=3.5cm]{euler}{euler}
\pgfdeclareimage[width=3.5cm]{hamilton}{hamilton}
\pgfdeclareimage[width=4.5cm]{k4planar}{k4planar}
\pgfdeclareimage[width=6cm]{region}{region}
\pgfdeclareimage[width=3cm]{eulerproof1}{eulerproof1}
\pgfdeclareimage[width=3cm]{eulerproof2}{eulerproof2}
\pgfdeclareimage{tetrahedron}{tetrahedron}
\pgfdeclareimage{planartetra}{planartetra}
\pgfdeclareimage[width=5cm]{hexahedron}{hexahedron}
\pgfdeclareimage{planarhexa}{planarhexa}
\pgfdeclareimage{octahedron}{octahedron}
\pgfdeclareimage{planarocta}{planarocta}
\pgfdeclareimage{dodecahedron}{dodecahedron}
\pgfdeclareimage{planardodeca}{planardodeca}
\pgfdeclareimage{icosahedron}{icosahedron}
\pgfdeclareimage[width=4cm]{fivesubstances}{fivesubstances}
\pgfdeclareimage[height=4cm]{coloring1}{coloring1}
\pgfdeclareimage[height=4cm]{coloring2}{coloring2}
\pgfdeclareimage[height=4cm]{coloring3}{coloring3}
\pgfdeclareimage[height=4cm]{coloring4}{coloring4}
\pgfdeclareimage[height=4cm]{coloring5}{coloring5}
\pgfdeclareimage[height=4cm]{coloring6}{coloring6}
\pgfdeclareimage[height=4cm]{coloring7}{coloring7}
\pgfdeclareimage[height=4cm]{herschel}{herschel}
\pgfdeclareimage[width=4cm]{sudoku}{sudoku}
\pgfdeclareimage[width=6cm]{dijkstra}{dijkstra}

\begin{document}

\begin{frame}
  \titlepage
\end{frame}

\begin{frame}
  \frametitle{License}

  \pgfuseimage{license}\hfill
  \copyright~2001-2014 T. Uyar, A. Yayımlı, E. Harmancı

  \vfill
  \begin{footnotesize}
    You are free to:
    \begin{itemize}
      \itemsep0em
      \item Share -- copy and redistribute the material in any medium or format
      \item Adapt -- remix, transform, and build upon the material
    \end{itemize}

    Under the following terms:
    \begin{itemize}
      \itemsep0em
      \item Attribution -- You must give appropriate credit, provide a link to
        the license, and indicate if changes were made.

      \item NonCommercial -- You may not use the material for commercial
        purposes.

      \item ShareAlike -- If you remix, transform, or build upon the material,
        you must distribute your contributions under the same license as the
        original.
    \end{itemize}
  \end{footnotesize}

  \begin{small}
    For more information:\\
    \url{https://creativecommons.org/licenses/by-nc-sa/4.0/}

    \smallskip
    Read the full license:\\
    \url{https://creativecommons.org/licenses/by-nc-sa/4.0/legalcode}
  \end{small}
\end{frame}

\begin{frame}
  \frametitle{Topics}
  \tableofcontents
\end{frame}

\section{Graphs}

\subsection{Introduction}

\begin{frame}
  \frametitle{Graphs}

  \begin{definition}
    \alert{graph}: $G=(V,E)$

    \begin{itemize}
      \item $V$: \alert{node} (or \emph{vertex}) set
      \item $E \subseteq V \times V$: \alert{edge} set
    \end{itemize}
  \end{definition}

  \pause
  \begin{itemize}
    \item if $e=(v_1,v_2) \in E$:
    \begin{itemize}
      \item $v_1$ and $v_2$ are \emph{endnodes} of $e$
      \item $e$ is \emph{incident} to $v_1$ and $v_2$
      \item $v_1$ and $v_2$ are \emph{adjacent}
    \end{itemize}

    \item node with no incident edge: \emph{isolated node}
  \end{itemize}
\end{frame}

\begin{frame}
  \frametitle{Graph Example}

  \begin{example}
    \begin{columns}
      \column{.58\textwidth}
      \begin{center}
        \pgfuseimage{plain}
      \end{center}

      \column{.4\textwidth}
      $\begin{array}{lcl}
        V & = & \{a,b,c,d,e,f\}\\
        E & = & \{(a,b),(a,c),\\
          &   & ~(a,d),(a,e),\\
          &   & ~(a,f),(b,c),\\
          &   & ~(d,e),(e,f)\}
      \end{array}$
    \end{columns}
  \end{example}
\end{frame}

\begin{frame}
  \frametitle{Directed Graphs}

  \begin{definition}
    \alert{directed graph} (or \emph{digraph}): $D=(V,A)$

    \begin{itemize}
      \item $V$: node set
      \item $A \subseteq V \times V$: \alert{arc} set
    \end{itemize}
  \end{definition}

  \begin{itemize}
    \item if $a=(v_1,v_2) \in A$:
    \begin{itemize}
      \item $v_1$: \emph{origin} node of $a$
      \item $v_2$: \emph{terminating} node of $a$
    \end{itemize}
  \end{itemize}
\end{frame}

\begin{frame}
  \frametitle{Directed Graph Example}

  \begin{example}
    \begin{center}
      \pgfuseimage{directed}
    \end{center}
  \end{example}
\end{frame}

\begin{frame}
  \frametitle{Weighted Graphs}

  \begin{itemize}
    \item in a weighted graph, labels are assigned to edges:\\
      weight, length, cost, delay, probability, $\ldots$
  \end{itemize}
\end{frame}

\begin{frame}
  \frametitle{Multigraphs}

  \begin{itemize}
    \item \alert{parallel edges}: edges between the same pair of nodes
    \item \alert{loop}: an edge starting and ending in the same node

    \pause
    \bigskip
    \item \alert{plain} graph: a graph without any loops or parallel edges
    \item \alert{multigraph}: a graph which is not plain
  \end{itemize}
\end{frame}

\begin{frame}
  \frametitle{Multigraph Example}

  \begin{example}
    \begin{columns}
      \column{.58\textwidth}
      \begin{center}
        \pgfuseimage{multi}
      \end{center}

      \column{.4\textwidth}
      \begin{itemize}
        \item parallel edges:\\
          $(a,b)$
        \item loop:\\
          $(e,e)$
      \end{itemize}
    \end{columns}
  \end{example}
\end{frame}

\begin{frame}
  \frametitle{Subgraph}

  \begin{definition}
    $G'=(V',E')$ is a \alert{subgraph} of $G=(V,E)$:

    \begin{itemize}
      \item $V' \subseteq V$, and
      \item $E' \subseteq E$, and
      \item $\forall (v_1,v_2) \in E'~v_1,v_2 \in V'$
    \end{itemize}
  \end{definition}
\end{frame}

\begin{frame}
  \frametitle{Representation}

  \begin{itemize}
    \item incidence matrix
    \begin{itemize}
      \item rows represent nodes, columns represent edges
      \item cell: 1 if the edge is incident to the node, 0 otherwise
    \end{itemize}

    \medskip
    \item adjacency matrix
    \begin{itemize}
      \item rows and columns represent nodes
      \item cell: 1 if the nodes are adjacent, 0 otherwise
      \item in a multigraph, the cells can represent\\
        the number of edges between the nodes
      \item in a weighted graph, the cells can represent\\
        the labels assigned to the edges
    \end{itemize}
  \end{itemize}
\end{frame}

\begin{frame}
  \frametitle{Incidence Matrix Example}

  \begin{example}
    \begin{columns}
      \column{.38\textwidth}
      \begin{center}
        \pgfuseimage{matrix}
      \end{center}

      \column{.58\textwidth}
      \[
        \begin{array}{c|cccccccc}
              & e_1 & e_2 & e_3 & e_4 & e_5 & e_6 & e_7 & e_8\\\hline
          v_1 & 1 & 1 & 1 & 0 & 1 & 0 & 0 & 0\\
          v_2 & 1 & 0 & 0 & 1 & 0 & 0 & 0 & 0\\
          v_3 & 0 & 0 & 1 & 1 & 0 & 0 & 1 & 1\\
          v_4 & 0 & 0 & 0 & 0 & 1 & 1 & 0 & 1\\
          v_5 & 0 & 1 & 0 & 0 & 0 & 1 & 1 & 0
        \end{array}
      \]
    \end{columns}
  \end{example}
\end{frame}

\begin{frame}
  \frametitle{Adjacency Matrix Example}

  \begin{example}
    \begin{columns}
      \column{.38\textwidth}
      \begin{center}
        \pgfuseimage{matrix}
      \end{center}

      \column{.58\textwidth}
      \[
        \begin{array}{c|ccccc}
                & v_1 & v_2 & v_3 & v_4 & v_5\\\hline
            v_1 & 0 & 1 & 1 & 1 & 1\\
            v_2 & 1 & 0 & 1 & 0 & 0\\
            v_3 & 1 & 1 & 0 & 1 & 1\\
            v_4 & 1 & 0 & 1 & 0 & 1\\
            v_5 & 1 & 0 & 1 & 1 & 0
        \end{array}
      \]
    \end{columns}
  \end{example}
\end{frame}

\begin{frame}
  \frametitle{Adjacency Matrix Example}

  \begin{example}
    \begin{columns}
    \column{.5\textwidth}
    \begin{center}
      \pgfuseimage{directed}
    \end{center}

    \column{.5\textwidth}
      \[
        \begin{array}{c|cccc}
              & a & b & c & d\\\hline
            a & 0 & 0 & 0 & 1\\
            b & 2 & 1 & 1 & 0\\
            c & 0 & 0 & 0 & 0\\
            d & 0 & 1 & 1 & 0
        \end{array}
      \]
    \end{columns}
  \end{example}
\end{frame}

\begin{frame}
  \frametitle{Degree}

  \begin{definition}
    \alert{degree}: number of edges incident to the node
  \end{definition}

  \pause
  \begin{theorem}
    let $d_i$ be the degree of node $v_i$:

    \[ |E| = \frac{\sum_i d_i}{2} \]
  \end{theorem}
\end{frame}

\begin{frame}
  \frametitle{Degree Example}

  \begin{example}[plain graph]
    \begin{columns}
      \column{.58\textwidth}
      \begin{center}
        \pgfuseimage{plain}
      \end{center}

      \column{.4\textwidth}
      $\begin{array}{ccc}
      d_a & = & 5\\
      d_b & = & 2\\
      d_c & = & 2\\
      d_d & = & 2\\
      d_e & = & 3\\
      d_f & = & 2\\
      \medskip
      Total & = & 16\\
      \medskip
      |E| & = & 8
      \end{array}$
    \end{columns}
  \end{example}
\end{frame}

\begin{frame}
  \frametitle{Degree Example}

  \begin{example}[multigraph]
    \begin{columns}
      \column{.58\textwidth}
      \begin{center}
        \pgfuseimage{multi}
      \end{center}

      \column{.4\textwidth}
      $\begin{array}{ccc}
      d_a & = & 6\\
      d_b & = & 3\\
      d_c & = & 2\\
      d_d & = & 2\\
      d_e & = & 5\\
      d_f & = & 2\\
      \medskip
      Total & = & 20\\
      \medskip
      |E| & = & 10
      \end{array}$
    \end{columns}
  \end{example}
\end{frame}

\begin{frame}
  \frametitle{Degree in Directed Graphs}

  \begin{itemize}
    \item two types of degree
    \begin{itemize}
      \item \emph{in-degree}: ${d_v}^i$
      \item \emph{out-degree}: ${d_v}^o$
    \end{itemize}

    \pause
    \medskip
    \item node with in-degree 0: \emph{source}
    \item node with out-degree  0: \emph{sink}

    \pause
    \bigskip
    \item $\sum_{v \in V} {d_v}^i = \sum_{v \in V} {d_v}^o = |A|$
  \end{itemize}
\end{frame}

\begin{frame}
  \frametitle{Degree}

  \begin{theorem}
    In an undirected graph, there is an even number of nodes\\
    which have an odd degree.
  \end{theorem}

  \pause
  \begin{proof}
    \begin{itemize}
      \item $t_i$: number of nodes of degree $i$

      \pause
$2|E| = \sum_i d_i = 1t_1 + 2t_2 + 3t_3 + 4t_4 + 5t_5 + \dots$

\pause
$2|E| - 2t_2 - 4t_4 - \dots = t_1 + t_3 + t_5 + \dots + 2t_3 + 4t_5 + \dots$

\pause
$2|E| - 2t_2 - 4t_4 - \dots - 2t_3 - 4t_5 - \dots = t_1 + t_3 + t_5 + \dots$

      \pause
      \item since the left-hand side is even, the right-hand side is also even
    \end{itemize}
  \end{proof}
\end{frame}

\begin{frame}
  \frametitle{Isomorphism}

  \begin{definition}
    $G=(V,E)$ and $G^\star=(V^\star,E^\star)$ are \alert{isomorphic}:
    \begin{itemize}
      \item $\exists f: V \rightarrow V^\star~(u,v) \in E \Rightarrow (f(u),f(v)) \in E^\star$
      \item $f$ is bijective
    \end{itemize}
  \end{definition}

  \pause
  \begin{itemize}
    \item $G$ and $G^\star$ can be drawn the same way
  \end{itemize}
\end{frame}

\begin{frame}
  \frametitle{Isomorphism Example}

  \begin{example}
    \begin{columns}
      \column{.4\textwidth}
      \begin{center}
        \pgfuseimage{isomorphicf}
      \end{center}

      \column{.6\textwidth}
      \begin{center}
        \pgfuseimage{isomorphict}
      \end{center}
    \end{columns}

    \pause
    \bigskip
    \begin{itemize}
      \item $f = (a \mapsto d, b \mapsto e, c \mapsto b, d \mapsto c,
        e \mapsto a)$
    \end{itemize}
  \end{example}
\end{frame}

\begin{frame}
  \frametitle{Isomorphism Example}

  \begin{example}[Petersen graph]
    \begin{columns}
      \column{.4\textwidth}
      \begin{center}
        \pgfuseimage{petersen1}
      \end{center}

      \column{.6\textwidth}
      \begin{center}
        \pgfuseimage{petersen2}
      \end{center}
    \end{columns}

    \bigskip
    \begin{itemize}
      \item $f = (a \mapsto q, b \mapsto v, c \mapsto u, d \mapsto y,
          e \mapsto r,$\\
        $~~~~~~~f \mapsto w, g \mapsto x, h \mapsto t, i \mapsto z,
          j \mapsto s)$
    \end{itemize}
  \end{example}
\end{frame}

\begin{frame}
  \frametitle{Homeomorphism}

  \begin{definition}
    $G=(V,E)$ and $G^\star=(V^\star,E^\star)$ are \alert{homeomorphic}:
    \begin{itemize}
      \item $G$ and $G^\star$ are isomorphic except that\\
        some edges in $E^\star$ are divided with additional nodes
    \end{itemize}
  \end{definition}
\end{frame}

\begin{frame}
  \frametitle{Homeomorphism Example}

  \begin{example}
    \begin{columns}
      \column{.5\textwidth}
      \begin{center}
        \pgfuseimage{isomorphict}
      \end{center}

      \column{.5\textwidth}
      \begin{center}
        \pgfuseimage{homeomorphict}
      \end{center}
    \end{columns}
  \end{example}
\end{frame}

\begin{frame}
  \frametitle{Regular Graphs}

  \begin{definition}
    \alert{regular} graph: all nodes have the same degree

    \begin{itemize}
      \item $n$-regular: all nodes have degree $n$
    \end{itemize}
  \end{definition}
\end{frame}

\begin{frame}
  \frametitle{Regular Graph Examples}

  \begin{example}
    \begin{center}
      \pgfuseimage{regular}
    \end{center}
  \end{example}
\end{frame}

\begin{frame}
  \frametitle{Completely Connected Graphs}

  \begin{definition}
    $G=(V,E)$ is \alert{completely connected}:
    \begin{itemize}
      \item $\forall v_1,v_2 \in V~(v_1,v_2) \in E$
    \end{itemize}
  \end{definition}

  \pause
  \begin{itemize}
    \item there is an edge between every pair of nodes
    \item $K_n$: the completely connected graph with $n$ nodes
  \end{itemize}
\end{frame}

\begin{frame}
  \frametitle{Completely Connected Graph Examples}

  \begin{columns}
    \column{.5\textwidth}
    \begin{example}[$K_4$]
      \begin{center}
        \pgfuseimage{k4}
      \end{center}
    \end{example}

    \column{.5\textwidth}
    \begin{example}[$K_5$]
      \begin{center}
        \pgfuseimage{k5}
      \end{center}
    \end{example}
  \end{columns}
\end{frame}

\begin{frame}
  \frametitle{Bipartite Graphs}

  \begin{definition}
    $G=(V,E)$ is \alert{bipartite}:
    \begin{itemize}
      \item $\forall (v_1,v_2) \in E~v_1 \in V_1 \wedge v_2 \in V_2$
      \item $V_1 \cup V_2 = V$, $V_1 \cap V_2 = \emptyset$
    \end{itemize}
  \end{definition}

  \pause
  \begin{itemize}
    \item \emph{complete bipartite}:
      $\forall v_1 \in V_1~\forall v_2 \in V_2~(v_1,v_2) \in E$
    \item $K_{m,n}$: $|V_1|=m$, $|V_2|=n$
  \end{itemize}
\end{frame}

\begin{frame}
  \frametitle{Complete Bipartite Graph Examples}

  \begin{columns}[t]
    \column{.5\textwidth}
    \begin{example}[$K_{2,3}$]
      \begin{center}
        \pgfuseimage{k23}
      \end{center}
    \end{example}

    \column{.5\textwidth}
    \begin{example}[$K_{3,3}$]
      \begin{center}
        \pgfuseimage{k33}
      \end{center}
    \end{example}
  \end{columns}
\end{frame}

\subsection{Connectivity}

\begin{frame}
  \frametitle{Walk}

  \begin{definition}
    \alert{walk}: a sequence of nodes and edges\\
      from a starting node ($v_0$) to an ending node ($v_n$)

    \[
      v_0 \xrightarrow{e_1} v_1 \xrightarrow{e_2} v_2 \xrightarrow{e_3} v_3
      \xrightarrow{} \cdots \xrightarrow{} v_{n-1} \xrightarrow{e_n} v_n
    \]

    where $e_i=(v_{i-1},v_i)$
  \end{definition}

  \pause
  \begin{itemize}
    \item no need to write the edges

    \medskip
    \item \alert{length}: number of edges in the walk
    \item if $v_0 \neq v_n$ \alert{open}, if $v_0 = v_n$ \alert{closed}
  \end{itemize}
\end{frame}

\begin{frame}
  \frametitle{Walk Example}

  \begin{example}
    \begin{columns}
      \column{.58\textwidth}
      \begin{center}
        \pgfuseimage{plain}
      \end{center}

      \column{.4\textwidth}
      $c \xrightarrow{(c,b)} b \xrightarrow{(b,a)} a \xrightarrow{(a,d)} d$\\
      $~~\xrightarrow{(d,e)} e \xrightarrow{(e,f)} f \xrightarrow{(f,a)} a$\\
      $~~\xrightarrow{(a,b)} b$

      \medskip
      $c \rightarrow b \rightarrow a \rightarrow d \rightarrow e$\\
      $~~\rightarrow f \rightarrow a \rightarrow b$

      \bigskip
      length: 7
    \end{columns}
  \end{example}
\end{frame}

\begin{frame}
  \frametitle{Trail}

  \begin{definition}
    \alert{trail}: a walk where edges are not repeated
  \end{definition}

  \begin{itemize}
    \item \alert{circuit}: closed trail
    \item \alert{spanning} trail: a trail that covers all the edges in the
      graph
  \end{itemize}
\end{frame}

\begin{frame}
  \frametitle{Trail Example}

  \begin{example}
    \begin{columns}
      \column{.58\textwidth}
      \begin{center}
        \pgfuseimage{plain}
      \end{center}

      \column{.4\textwidth}
      $c \xrightarrow{(c,b)} b \xrightarrow{(b,a)} a \xrightarrow{(a,e)} e$\\
      $~~\xrightarrow{(e,d)} d \xrightarrow{(d,a)} a \xrightarrow{(a,f)} f$

      \medskip
      $c \rightarrow a \rightarrow e \rightarrow d \rightarrow a
         \rightarrow f$
    \end{columns}
  \end{example}
\end{frame}

\begin{frame}
  \frametitle{Path}

  \begin{definition}
    \alert{path}: a walk where nodes are not repeated
  \end{definition}

  \begin{itemize}
    \item \alert{cycle}: closed path
    \item \alert{spanning} path: a path that visits all the nodes in the graph
  \end{itemize}
\end{frame}

\begin{frame}
  \frametitle{Path Example}

  \begin{example}
    \begin{columns}
      \column{.58\textwidth}
      \begin{center}
        \pgfuseimage{plain}
      \end{center}

      \column{.4\textwidth}
      $c \xrightarrow{(c,b)} b \xrightarrow{(b,a)} a \xrightarrow{(a,d)} d$\\
      $~~\xrightarrow{(d,e)} e \xrightarrow{(e,f)} f$

      \medskip
      $c \rightarrow b \rightarrow a \rightarrow d \rightarrow e \rightarrow f$
    \end{columns}
  \end{example}
\end{frame}

\begin{frame}
  \frametitle{Connectivity}

  \begin{definition}
    \alert{connected} graph:\\
    there is a path between every pair of nodes
  \end{definition}

  \begin{itemize}
    \item a disconnected graph can be divided\\
      into connected components
  \end{itemize}
\end{frame}

\begin{frame}
  \frametitle{Connected Components Example}

  \begin{example}
    \begin{columns}
      \column{.54\textwidth}
      \begin{center}
        \pgfuseimage{disconnected}
      \end{center}

      \column{.45\textwidth}
      \begin{itemize}
        \item graph is disconnected:\\
          no path between $a$ and $c$
        \item connected components:\\
          $a,d,e$\\
          $b,c$\\
          $f$
      \end{itemize}
    \end{columns}
  \end{example}
\end{frame}

\begin{frame}
  \frametitle{Distance}

  \begin{definition}
    \alert{distance} between nodes $v_i$ and $v_j$:\\
    the length of the shortest path between $v_i$ and $v_j$
  \end{definition}

  \begin{itemize}
    \item \alert{diameter}: the largest distance in the graph
  \end{itemize}
\end{frame}

\begin{frame}
  \frametitle{Distance Example}

  \begin{example}
    \begin{columns}
      \column{.48\textwidth}
      \begin{center}
        \pgfuseimage{distance}
      \end{center}

      \column{.52\textwidth}
      \begin{itemize}
        \item distance between $a$ and $e$: 2\\
        \item diameter: 3
      \end{itemize}
    \end{columns}
  \end{example}
\end{frame}

\begin{frame}
  \frametitle{Cut-Points}

    \begin{itemize}
      \item the graph \alert{$G - v$} is obtained by deleting the node $v$\\
        and all its incident edges from the graph $G$
    \end{itemize}

  \begin{definition}
    $v$ is a \alert{cut-point} for $G$:\\
      $G$ is connected but $G - v$ is not
  \end{definition}
\end{frame}

\begin{frame}
  \frametitle{Cut-Point Example}

  \begin{columns}
    \column{.5\textwidth}
    \begin{block}{$G$}
      \begin{center}
        \pgfuseimage{distance}
      \end{center}
    \end{block}

    \column{.5\textwidth}
    \begin{block}{$G - d$}
      \begin{center}
        \pgfuseimage{cutpoint}
      \end{center}
    \end{block}
  \end{columns}
\end{frame}

\begin{frame}
  \frametitle{Directed Walks}

  \begin{itemize}
    \item same as in undirected graphs
    \item ignoring the directions on the arcs:\\
      \emph{semi-walk}, \emph{semi-trail}, \emph{semi-path}
  \end{itemize}
\end{frame}

\begin{frame}
  \frametitle{Weakly Connected Graph}

  \begin{columns}
    \column{.5\textwidth}
    \begin{definition}
      \alert{weakly} connected:\\
      there is a semi-path\\
      between every pair of nodes
    \end{definition}

    \column{.5\textwidth}
    \begin{example}
      \begin{center}
        \pgfuseimage{weak}
      \end{center}
    \end{example}
  \end{columns}
\end{frame}

\begin{frame}
  \frametitle{Unilaterally Connected Graph}

  \begin{columns}
    \column{.5\textwidth}
    \begin{definition}
      \alert{unilaterally} connected:\\
      for every pair of nodes, there is\\
      a path from one to the other
    \end{definition}

    \column{.5\textwidth}
    \begin{example}
      \begin{center}
        \pgfuseimage{unilateral}
      \end{center}
    \end{example}
  \end{columns}
\end{frame}

\begin{frame}
  \frametitle{Strongly Connected Graph}

  \begin{columns}
    \column{.5\textwidth}
    \begin{definition}
      \alert{strongly} connected:\\
      there is a path in both directions\\
      between every pair of nodes
    \end{definition}

    \column{.5\textwidth}
    \begin{example}
      \begin{center}
        \pgfuseimage{strong}
      \end{center}
    \end{example}
  \end{columns}
\end{frame}

\begin{frame}
  \frametitle{Connectivity Matrix}

  \begin{itemize}
    \item let $A$ be the adjacency matrix of an undirected graph $G=(V,E)$
    \item $A^k_{ij}$: number of walks of length $k$
      between nodes $i$ and $j$

    \item the distance between two nodes is at most $|V|-1$

    \pause
    \medskip
    \item connectivity matrix:\\
      $C = A^1 + A^2 + A^3 + \dots + A^{|V|-1}$
    \item if all elements of $C$ are non-zero, then $G$ is connected
  \end{itemize}
\end{frame}

\begin{frame}
  \frametitle{Warshall's Algorithm}

  \begin{itemize}
    \item it is easier to find whether there is a walk between two nodes\\
      rather than finding the number of walks

    \pause
    \medskip
    \item for each node:
    \begin{itemize}
      \item from all nodes which can reach the chosen node\\
        (the rows that contain 1 in the chosen column)

      \item to the nodes which can be reached from the chosen node\\
        (the columns that contain 1 in the chosen row)
    \end{itemize}
  \end{itemize}
\end{frame}

\begin{frame}
  \frametitle{Warshall's Algorithm Example}

  \begin{example}
    \begin{columns}
      \column{.5\textwidth}
      \begin{center}
        \pgfuseimage{warshall}
      \end{center}

      \column{.5\textwidth}
      \[
        \begin{array}{c|cccc}
              & a & b & c & d\\\hline
            a & 0 & \alert{1} & 0 & 0\\
            b & 0 & 1 & 0 & 0\\
            c & 0 & 0 & 0 & 1\\
            d & \alert{1} & 0 & 1 & 0
        \end{array}
      \]
    \end{columns}
  \end{example}
\end{frame}

\begin{frame}
  \frametitle{Warshall's Algorithm Example}

  \begin{example}
    \begin{columns}
      \column{.5\textwidth}
      \begin{center}
        \pgfuseimage{warshall1}
      \end{center}

      \column{.5\textwidth}
      \[
        \begin{array}{c|cccc}
              & a & b & c & d\\\hline
            a & 0 & \alert{1} & 0 & 0\\
            b & 0 & 1 & 0 & 0\\
            c & 0 & 0 & 0 & 1\\
            d & 1 & \alert{1} & 1 & 0
        \end{array}
      \]
    \end{columns}
  \end{example}
\end{frame}

\begin{frame}
  \frametitle{Warshall's Algorithm Example}

  \begin{example}
    \begin{columns}
      \column{.5\textwidth}
      \begin{center}
        \pgfuseimage{warshall1}
      \end{center}

      \column{.5\textwidth}
      \[
        \begin{array}{c|cccc}
              & a & b & c & d\\\hline
            a & 0 & 1 & 0 & 0\\
            b & 0 & 1 & 0 & 0\\
            c & 0 & 0 & 0 & \alert{1}\\
            d & 1 & 1 & \alert{1} & 0
        \end{array}
      \]
    \end{columns}
  \end{example}
\end{frame}

\begin{frame}
  \frametitle{Warshall's Algorithm Example}

  \begin{example}
    \begin{columns}
      \column{.5\textwidth}
      \begin{center}
        \pgfuseimage{warshall2}
      \end{center}

      \column{.5\textwidth}
      \[
        \begin{array}{c|cccc}
              & a & b & c & d\\\hline
            a & 0 & 1 & 0 & 0\\
            b & 0 & 1 & 0 & 0\\
            c & 0 & 0 & 0 & \alert{1}\\
            d & \alert{1} & \alert{1} & \alert{1} & \alert{1}
        \end{array}
      \]
    \end{columns}
  \end{example}
\end{frame}

\begin{frame}
  \frametitle{Warshall's Algorithm Example}

  \begin{example}
    \begin{columns}
      \column{.5\textwidth}
      \begin{center}
        \pgfuseimage{warshall3}
      \end{center}

      \column{.5\textwidth}
      \[
        \begin{array}{c|cccc}
              & a & b & c & d\\\hline
            a & 0 & 1 & 0 & 0\\
            b & 0 & 1 & 0 & 0\\
            c & 1 & 1 & 1 & 1\\
            d & 1 & 1 & 1 & 1
        \end{array}
      \]
    \end{columns}
  \end{example}
\end{frame}

\subsection{Traversable Graphs}

\begin{frame}
  \frametitle{Bridges of Königsberg}

  \begin{center}
    \pgfuseimage{konigsberg}
  \end{center}

  \begin{itemize}
    \item cross each bridge exactly once\\
      and return to the starting point
  \end{itemize}
\end{frame}

\begin{frame}
  \frametitle{Traversable Graphs}

  \begin{definition}
    $G$ is \alert{traversable}: $G$ contains a spanning trail
  \end{definition}

  \begin{center}
    \pgfuseimage{konigsgraph}
  \end{center}
\end{frame}

\begin{frame}
  \frametitle{Traversable Graphs}

  \begin{itemize}
    \item a node with an odd degree must be either the starting node\\
      or the ending node of the trail
    \item all nodes except the starting node and the ending node\\
      must have even degrees
  \end{itemize}
\end{frame}

\begin{frame}
  \frametitle{Traversable Graph Example}

  \begin{example}
    \begin{columns}
      \column{.4\textwidth}
      \begin{center}
        \pgfuseimage{envelope}
      \end{center}

      \column{.6\textwidth}
      \begin{itemize}
        \item degrees of $a$, $b$ and $c$ are even
        \item degrees of $d$ and $e$ are odd

        \pause
        \medskip
        \item a spanning trail can be formed\\
          starting from node $d$ and\\
          ending at node $e$ (or vice versa):\\
          $d \rightarrow b \rightarrow a \rightarrow c \rightarrow e$
          $~~\rightarrow d \rightarrow c \rightarrow b \rightarrow e$
      \end{itemize}
    \end{columns}
  \end{example}
\end{frame}

\begin{frame}
  \frametitle{Bridges of Königsberg}

  \begin{center}
    \pgfuseimage{konigsgraph}
  \end{center}

  \begin{itemize}
    \item all node have odd degrees: not traversable
  \end{itemize}
\end{frame}

\begin{frame}
  \frametitle{Euler Graphs}

  \begin{definition}
    \alert{Euler graph}: a graph that contains a closed spanning trail
  \end{definition}

  \begin{itemize}
    \item $G$ is an Euler graph $\Leftrightarrow$
      all nodes in $G$ have even degrees
  \end{itemize}
\end{frame}

\begin{frame}
  \frametitle{Euler Graph Examples}

  \begin{columns}
    \column{.5\textwidth}
    \begin{example}[Euler graph]
      \begin{center}
        \pgfuseimage{euler}
      \end{center}
    \end{example}

    \column{.5\textwidth}
    \begin{example}[not an Euler graph]
      \begin{center}
        \pgfuseimage{hamilton}
      \end{center}
    \end{example}
  \end{columns}
\end{frame}

\begin{frame}
  \frametitle{Hamilton Graphs}

  \begin{definition}
    \alert{Hamilton graph}: a graph that contains a closed spanning path
  \end{definition}
\end{frame}

\begin{frame}
  \frametitle{Hamilton Graph Examples}

  \begin{columns}
    \column{.5\textwidth}
    \begin{example}[Hamilton graph]
      \begin{center}
        \pgfuseimage{hamilton}
      \end{center}
    \end{example}

    \column{.5\textwidth}
    \begin{example}[not a Hamilton graph]
      \begin{center}
        \pgfuseimage{euler}
      \end{center}
    \end{example}
  \end{columns}
\end{frame}

\subsection{Planar Graphs}

\begin{frame}
  \frametitle{Planar Graphs}

  \begin{definition}
    $G$ is \alert{planar}:\\
    $G$ can be drawn on a plane without intersecting its edges
  \end{definition}

  \begin{itemize}
    \item a \alert{map} of $G$: a planar drawing of $G$
  \end{itemize}
\end{frame}

\begin{frame}
  \frametitle{Planar Graph Example}

  \begin{example}[$K_4$]
    \begin{columns}
      \column{.5\textwidth}
      \begin{center}
        \pgfuseimage{k4}
      \end{center}

      \column{.5\textwidth}
      \begin{center}
        \pgfuseimage{k4planar}
      \end{center}
    \end{columns}
  \end{example}
\end{frame}

\begin{frame}
  \frametitle{Regions}

  \begin{itemize}
    \item a map divides the plane into \alert{regions}
    \item degree of a region:\\
      length of the closed trail that surrounds the region
  \end{itemize}

  \pause
  \begin{theorem}
    let $d_{r_i}$ be the degree of region $r_i$:

    \[ |E| = \frac{\sum_i d_{r_i}}{2} \]
  \end{theorem}
\end{frame}

\begin{frame}
  \frametitle{Region Example}

  \begin{example}
    \begin{columns}
      \column{.58\textwidth}
      \begin{center}
        \pgfuseimage{region}
      \end{center}

      \column{.4\textwidth}
      $d_{r_1} = 3$ (abda)\\
      $d_{r_2} = 3$ (bcdb)\\
      $d_{r_3} = 5$ (cdefec)\\
      $d_{r_4} = 4$ (abcea)\\
      $d_{r_5} = 3$ (adea)

      \medskip
      $\sum_r d_r = 18$\\
      $|E| = 9$
    \end{columns}
  \end{example}
\end{frame}

\begin{frame}
  \frametitle{Euler's Formula}

  \begin{theorem}[Euler's Formula]
    let $G=(V,E)$ be a planar, connected graph, and\\
    let $R$ be the set of regions in a map of $G$:

    \[|V| - |E| + |R| = 2\]
  \end{theorem}
\end{frame}

\begin{frame}
  \frametitle{Euler's Formula Example}

  \begin{example}
    \begin{center}
      \pgfuseimage{region}
    \end{center}

    \begin{itemize}
     \item $|V| = 6$, $|E| = 9$, $|R| = 5$
    \end{itemize}
  \end{example}
\end{frame}
% 
% \begin{frame}
%   \frametitle{Proof of Euler's Formula}
% 
%   \begin{block}{Proof}
%     method: induction on $|E|$
% 
%     \pause
%     \begin{itemize}
%       \item base step: one node, no edges\\
%         $|V| = 1$, $|E| = 0$, $|R| = 1$
% 
%       \pause
%       \item assume it holds for a connected, planar graph with $k$ edges
%     \end{itemize}
%   \end{block}
% \end{frame}
% 
% \begin{frame}
%   \frametitle{Proof of Euler's Formula}
% 
%   \begin{proof}[Induction Step]
%     \begin{columns}[t]
%       \column{.5\textwidth}
%       \begin{itemize}
%         \item connect a new node\\
% 	  to an existing node:
% 
%         \medskip
%         \pgfuseimage{eulerproof1}
%       \end{itemize}
% 
%       \column{.5\textwidth}
%       \begin{itemize}
%         \item add an edge between\\
% 	  two existing nodes:
% 
%         \medskip
%         \pgfuseimage{eulerproof2}
%       \end{itemize}
%     \end{columns}
% 
%     \pause
%     \begin{columns}
%       \column{.5\textwidth}
%       \begin{itemize}
%         \item $|V|$ is increased by 1,\\
% 	  $|E|$ is increased by 1,\\
%           $|R|$ remains the same
%       \end{itemize}
% 
%       \pause
%       \column{.5\textwidth}
%       \begin{itemize}
%         \item $|V|$ remains the same,\\
% 	  $|E|$ is increased by 1,\\
%           $|R|$ is increased by 1
%       \end{itemize}
%     \end{columns}
%   \end{proof}
% \end{frame}

\begin{frame}
  \frametitle{Planar Graph Theorems}

  \begin{theorem}
    let $G=(V,E)$ be a connected, planar graph where $|V| \geq 3$:\\
    $|E| \leq 3 |V| - 6$
  \end{theorem}

  \pause
  \begin{proof}
    \begin{itemize}
      \item sum of region degrees: $2 |E|$

      \pause
      \item degree of a region is at least $3$\\
        \pause
        $\Rightarrow 2 |E| \geq 3 |R|$
        \pause
        $\Rightarrow |R| \leq \frac{2}{3} |E|$

      \pause
      \item $|V| - |E| + |R| = 2$\\
        \pause
        $\Rightarrow |V| - |E| + \frac{2}{3} |E| \geq 2$
        \pause
        $\Rightarrow |V| - \frac{1}{3} |E| \geq 2$\\
        \pause
        $\Rightarrow 3 |V| - |E| \geq 6$
        \pause
        $\Rightarrow |E| \leq 3 |V| - 6$\\
    \end{itemize}
  \end{proof}
\end{frame}

\begin{frame}
  \frametitle{Planar Graph Theorems}

  \begin{theorem}
    let $G=(V,E)$ be a connected, planar graph where $|V| \geq 3$:\\
    $\exists v \in V~d_v \leq 5$
  \end{theorem}

  \pause
  \begin{proof}
    \begin{itemize}
      \item let $\forall v \in V~d_v \geq 6$\\
        \pause
        $\Rightarrow 2 |E| \geq 6 |V|$\\
        \pause
        $\Rightarrow |E| \geq 3 |V|$\\
        \pause
        $\Rightarrow |E| > 3 |V| - 6$
    \end{itemize}
  \end{proof}
\end{frame}

\begin{frame}
  \frametitle{Nonplanar Graphs}

  \begin{columns}
    \column{.45\textwidth}
    \begin{theorem}
      $K_5$ is not planar.

      \medskip
      \begin{center}
        \pgfuseimage{k5}
      \end{center}
    \end{theorem}

    \pause
    \column{.55\textwidth}
    \begin{proof}
      \begin{itemize}
        \item $|V| = 5$

        \pause
        \item $3 |V| - 6 = 3 \cdot 5 - 6 = 9$

        \pause
        \item $|E| \leq 9$ should hold

        \pause
        \item but $|E| = 10$
      \end{itemize}
    \end{proof}
  \end{columns}
\end{frame}

\begin{frame}
  \frametitle{Nonplanar Graphs}

  \begin{columns}
    \column{.45\textwidth}
    \begin{theorem}
      $K_{3,3}$ is not planar.

      \medskip
      \begin{center}
        \pgfuseimage{k33}
      \end{center}
    \end{theorem}

    \pause
    \column{.55\textwidth}
    \begin{proof}
      \begin{itemize}
        \item $|V| = 6, |E| = 9$

        \pause
        \item if planar then $|R| = 5$

        \pause
        \item degree of a region is at least $4$\\
          $\Rightarrow \sum_{r \in R} d_r \geq 20$

        \pause
        \item $|E| \geq 10$ should hold

        \pause
        \item but $|E| = 9$
      \end{itemize}
    \end{proof}
  \end{columns}
\end{frame}

\begin{frame}
  \frametitle{Kuratowski's Theorem}

  \begin{theorem}
    \begin{center}
      $G$ contains a subgraph homeomorphic to $K_5$ or $K_{3,3}$.\\
      $\Leftrightarrow$\\
      $G$ is not planar.
    \end{center}
  \end{theorem}
\end{frame}

\begin{frame}
  \frametitle{Platonic Solids}

  \begin{itemize}
    \item \emph{regular polyhedron}: a 3-dimensional solid\\
      where the faces are identical regular polygons

    \pause
    \item the projection of a regular polyhedron onto the plane\\
      is a planar graph
    \begin{itemize}
      \item every corner is a node
      \item every side is an edge
      \item every face is a region
    \end{itemize}
  \end{itemize}
\end{frame}

\begin{frame}
  \frametitle{Platonic Solids}

  \begin{example}[cube]
    \begin{center}
      \pgfuseimage{hexahedron}
    \end{center}
  \end{example}
\end{frame}

\begin{frame}
  \frametitle{Platonic Solids}

  \begin{itemize}
    \item $v$: number of corners (nodes)
    \item $e$: number of sides (edges)
    \item $r$: number of faces (regions)
    \item $n$: number of faces meeting at a corner (node degree)
    \item $m$: number of sides of a face (region degree)
  \end{itemize}

  \pause
  \begin{itemize}
    \item $m,n \geq 3$
    \item $2e = n \cdot v$
    \item $2e = m \cdot r$
  \end{itemize}
\end{frame}

\begin{frame}
  \frametitle{Platonic Solids}

    \begin{itemize}
      \item from Euler's formula:
      \[
        2 = v - e + r = \frac{2e}{n} - e + \frac{2e}{m}
        = e \Big( \frac{2m-mn+2n}{mn} \Big) > 0
      \]

      \pause
      \item $e,m,n > 0$:
      \begin{eqnarray*}
        2m - mn + 2n > 0 \Rightarrow mn - 2m -2n < 0 \\\pause
        \Rightarrow mn - 2m - 2n + 4 < 4 \pause \Rightarrow (m - 2)(n - 2) < 4
      \end{eqnarray*}

      \pause
      \item the values that satisfy this inequation:
      \begin{enumerate}
        \item $m=3, n=3$
        \item $m=4, n=3$
        \item $m=3, n=4$
        \item $m=5, n=3$
        \item $m=3, n=5$
      \end{enumerate}
    \end{itemize}
\end{frame}

\begin{frame}
  \frametitle{Tetrahedron}

  \begin{columns}
    \column{.7\textwidth}
    \begin{center}
      \pgfuseimage{tetrahedron}
    \end{center}

    \column{.3\textwidth}
    \begin{center}
      \pgfuseimage{planartetra}

      $m=3, n=3$
    \end{center}
  \end{columns}
\end{frame}

\begin{frame}
  \frametitle{Hexahedron}

  \begin{columns}
    \column{.7\textwidth}
    \begin{center}
      \pgfuseimage{hexahedron}
    \end{center}

    \column{.3\textwidth}
    \begin{center}
      \pgfuseimage{planarhexa}

      $m=4, n=3$
    \end{center}
  \end{columns}
\end{frame}

\begin{frame}
  \frametitle{Octahedron}

  \begin{columns}
    \column{.7\textwidth}
    \begin{center}
      \pgfuseimage{octahedron}
    \end{center}

    \column{.3\textwidth}
    \begin{center}
      \pgfuseimage{planarocta}

      $m=3, n=4$
    \end{center}
  \end{columns}
\end{frame}

\begin{frame}
  \frametitle{Dodecahedron}

  \begin{columns}
    \column{.7\textwidth}
    \begin{center}
      \pgfuseimage{dodecahedron}
    \end{center}

    \column{.3\textwidth}
    \begin{center}
      \pgfuseimage{planardodeca}

      $m=5, n=3$
    \end{center}
  \end{columns}
\end{frame}

\begin{frame}
  \frametitle{Icosahedron}

  \begin{columns}
    \column{.7\textwidth}
    \begin{center}
      \pgfuseimage{icosahedron}
    \end{center}

    \column{.3\textwidth}
    $m=3, n=5$
  \end{columns}
\end{frame}

\section{Graph Problems}

\subsection{Graph Coloring}

\begin{frame}
  \frametitle{Graph Coloring}

  \begin{definition}
    \alert{proper coloring} of $G=(V,E)$: $f: V \rightarrow C$\\
      where $C$ is a set of colors
    \begin{itemize}
      \item $\forall (v_i,v_j) \in E~f(v_i) \neq f(v_j)$
      \item minimizing $|C|$
    \end{itemize}
  \end{definition}
\end{frame}

\begin{frame}
  \frametitle{Graph Coloring Example}

  \begin{example}
    \begin{itemize}
      \item a company produces chemical compounds
      \item some compounds cannot be stored together
      \item such compounds must be placed in separate storage areas

      \pause
      \medskip
      \item store the compounds using the least number of storage areas
    \end{itemize}
  \end{example}
\end{frame}

\begin{frame}
  \frametitle{Graph Coloring}

  \begin{example}
    \begin{itemize}
      \item every compound is a node
      \item two compounds that cannot be stored together are adjacent
    \end{itemize}

    \begin{center}
      \pgfuseimage{fivesubstances}
    \end{center}
  \end{example}
\end{frame}

\begin{frame}
  \frametitle{Graph Coloring Example}

  \begin{example}
    \begin{center}
      \pgfuseimage{coloring1}
    \end{center}
  \end{example}
\end{frame}

\begin{frame}
  \frametitle{Graph Coloring Example}

  \begin{example}
    \begin{columns}
      \column{.5\textwidth}
      \begin{center}
        \pgfuseimage{coloring2}
      \end{center}

      \column{.5\textwidth}
      \begin{center}
        \pgfuseimage{coloring3}
      \end{center}
    \end{columns}
  \end{example}
\end{frame}

\begin{frame}
  \frametitle{Graph Coloring Example}

  \begin{example}
    \begin{columns}
      \column{.5\textwidth}
      \begin{center}
        \pgfuseimage{coloring4}
      \end{center}

      \column{.5\textwidth}
      \begin{center}
        \pgfuseimage{coloring5}
      \end{center}
    \end{columns}
  \end{example}
\end{frame}

\begin{frame}
  \frametitle{Graph Coloring Example}

  \begin{example}
    \begin{columns}
      \column{.5\textwidth}
      \begin{center}
        \pgfuseimage{coloring6}
      \end{center}

      \column{.5\textwidth}
      \begin{center}
        \pgfuseimage{coloring7}
      \end{center}
    \end{columns}
  \end{example}
\end{frame}

\begin{frame}
  \frametitle{Chromatic Number}

  \begin{definition}
    \alert{chromatic number} of $G$: $\chi (G)$\\
    minimum number of colors needed to color G
  \end{definition}

  \begin{itemize}
     \item finding $\chi (G)$ is a very difficult problem
     \item $\chi (K_n) = n$
  \end{itemize}
\end{frame}

\begin{frame}
  \frametitle{Chromatic Number Example}

  \begin{example}[Herschel graph]
    \begin{center}
      \pgfuseimage{herschel}
    \end{center}

    \begin{itemize}
      \item chromatic number: 2
    \end{itemize}
  \end{example}
\end{frame}

\begin{frame}
  \frametitle{Graph Coloring Example}

  \begin{example}[Sudoku]
    \begin{columns}[t]
      \column{.4\textwidth}
      \begin{center}
        \pgfuseimage{sudoku}
      \end{center}

      \column{.55\textwidth}
      \begin{itemize}
        \item every cell is a node
        \item cells of the same row\\
          are adjacent
        \item cells of the same column\\
          are adjacent
        \item cells of the same $3 \times 3$ block\\
          are adjacent
        \item every number is a color
      \end{itemize}

      \pause
      \begin{itemize}
        \item problem: properly color a graph\\
          that is partially colored
      \end{itemize}
    \end{columns}
  \end{example}
\end{frame}

\begin{frame}
  \frametitle{Region Coloring}

  \begin{itemize}
    \item coloring a map by assigning different colors to adjacent regions
  \end{itemize}

  \medskip
  \begin{theorem}[Four~Color Theorem]
    The regions in a map can be colored using four colors.
  \end{theorem}
\end{frame}

\subsection{Shortest Path}

\begin{frame}
  \frametitle{Shortest Path}

  \begin{itemize}
    \item finding the shortest paths from a starting node\\
      to all other nodes:
      Dijkstra's algorithm
  \end{itemize}
\end{frame}

\begin{frame}
  \frametitle{Dijkstra's Algorithm Example}

  \begin{example}[initialization]
    \begin{columns}
      \column{.5\textwidth}
      \begin{center}
        \pgfuseimage{dijkstra}
      \end{center}

      \column{.45\textwidth}
      \begin{itemize}
        \item starting node: $c$
      \end{itemize}

      \begin{table}
        \begin{tabular}{r|l}
          a & $(\infty,-)$ \\\hline
          b & $(\infty,-)$ \\\hline
          c & $(0,-)$      \\\hline
          f & $(\infty,-)$ \\\hline
          g & $(\infty,-)$ \\\hline
          h & $(\infty,-)$
        \end{tabular}
      \end{table}
    \end{columns}
  \end{example}
\end{frame}

\begin{frame}
  \frametitle{Dijkstra's Algorithm Example}

  \begin{example}[from node $c$ - base distance=$0$]
    \begin{columns}
      \column{.5\textwidth}
      \begin{center}
        \pgfuseimage{dijkstra}
      \end{center}

      \column{.45\textwidth}
      \begin{itemize}
        \item $c \rightarrow f: 6, 6 < \infty$
        \item $c \rightarrow h: 11, 11 < \infty$
      \end{itemize}

      \pause
      \begin{table}
        \begin{tabular}{r|l|c}
          a & $(\infty,-)$ & \\\hline
          b & $(\infty,-)$ & \\\hline
          c & $(0,-)$      & $\surd$ \\\hline
          f & $(6,cf)$     & \\\hline
          g & $(\infty,-)$ & \\\hline
          h & $(11,ch)$    &
        \end{tabular}
      \end{table}

      \pause
      \begin{itemize}
        \item closest node: $f$
      \end{itemize}
    \end{columns}
  \end{example}
\end{frame}

\begin{frame}
  \frametitle{Dijkstra's Algorithm Example}

  \begin{example}[from node $f$ - base distance=$6$]
    \begin{columns}
      \column{.5\textwidth}
      \begin{center}
        \pgfuseimage{dijkstra}
      \end{center}

      \column{.45\textwidth}
      \begin{itemize}
        \item $f \rightarrow a: 6+11, 17 < \infty$
        \item $f \rightarrow g: 6+9, 15 < \infty$
        \item $f \rightarrow h: 6+4, 10 < 11$
      \end{itemize}

      \pause
      \begin{table}
        \begin{tabular}{r|l|c}
          a & $(17,cfa)$   & \\\hline
          b & $(\infty,-)$ & \\\hline
          c & $(0,-)$      & $\surd$ \\\hline
          f & $(6,cf)$     & $\surd$ \\\hline
          g & $(15,cfg)$   & \\\hline
          h & $(10,cfh)$   &
        \end{tabular}
      \end{table}

      \pause
      \begin{itemize}
        \item closest node: $h$
      \end{itemize}
    \end{columns}
  \end{example}
\end{frame}

\begin{frame}
  \frametitle{Dijkstra's Algorithm Example}

  \begin{example}[from node $h$ - base distance=$10$]
    \begin{columns}
      \column{.5\textwidth}
      \begin{center}
        \pgfuseimage{dijkstra}
      \end{center}

      \column{.45\textwidth}
      \begin{itemize}
        \item $h \rightarrow a: 10+11, 21 \nless 17$
        \item $h \rightarrow g: 10+4, 14 < 15$
      \end{itemize}

      \pause
      \begin{table}
        \begin{tabular}{r|l|c}
          a & $(17,cfa)$   & \\\hline
          b & $(\infty,-)$ & \\\hline
          c & $(0,-)$      & $\surd$ \\\hline
          f & $(6,cf)$     & $\surd$ \\\hline
          g & $(14,cfhg)$  & \\\hline
          h & $(10,cfh)$   & $\surd$
        \end{tabular}
      \end{table}

      \pause
      \begin{itemize}
        \item closest node: $g$
      \end{itemize}
    \end{columns}
  \end{example}
\end{frame}

\begin{frame}
  \frametitle{Dijkstra's Algorithm Example}

  \begin{example}[from node $g$ - base distance=$14$]
    \begin{columns}
      \column{.5\textwidth}
      \begin{center}
        \pgfuseimage{dijkstra}
      \end{center}

      \column{.45\textwidth}
      \begin{itemize}
        \item $g \rightarrow a: 14+17, 31 \nless 17$
      \end{itemize}

      \pause
      \begin{table}
        \begin{tabular}{r|l|c}
          a & $(17,cfa)$   & \\\hline
          b & $(\infty,-)$ & \\\hline
          c & $(0,-)$      & $\surd$ \\\hline
          f & $(6,cf)$     & $\surd$ \\\hline
          g & $(14,cfhg)$  & $\surd$ \\\hline
          h & $(10,cfh)$   & $\surd$
        \end{tabular}
      \end{table}

      \pause
      \begin{itemize}
        \item closest node: $a$
      \end{itemize}
    \end{columns}
  \end{example}
\end{frame}

\begin{frame}
  \frametitle{Dijkstra's Algorithm Example}

  \begin{example}[from node $a$ - base distance=$17$]
    \begin{columns}
      \column{.5\textwidth}
      \begin{center}
        \pgfuseimage{dijkstra}
      \end{center}

      \column{.45\textwidth}
      \begin{itemize}
        \item $a \rightarrow b: 17+5, 22 < \infty$
      \end{itemize}

      \pause
      \begin{table}
        \begin{tabular}{r|l|c}
          a & $(17,cfa)$   & $\surd$ \\\hline
          b & $(22,cfab)$  & \\\hline
          c & $(0,-)$      & $\surd$ \\\hline
          f & $(6,cf)$     & $\surd$ \\\hline
          g & $(14,cfhg)$  & $\surd$ \\\hline
          h & $(10,cfh)$   & $\surd$
        \end{tabular}
      \end{table}

      \pause
      \begin{itemize}
        \item last node: $b$
      \end{itemize}
    \end{columns}
  \end{example}
\end{frame}

\subsection{Searching Graphs}

\begin{frame}
  \frametitle{Searching Graphs}

  \begin{itemize}
    \item searching nodes of graph $G=(V,E)$ starting from node $v_1$

    \bigskip
    \item depth-first
    \item breadth-first
  \end{itemize}
\end{frame}

\begin{frame}
  \frametitle{Depth-First Search}

  \begin{enumerate}
    \item $v \leftarrow v_1, T=\emptyset$, $D=\{v_1\}$

    \pause
    \item find smallest $i$ in $2 \leq i \leq |V|$
      such that $(v,v_i) \in E$ and $v_i \notin D$
      \begin{itemize}
        \item if no such $i$ exists: go to step 3
        \item if found: $T=T \cup \{(v,v_i)\}$, $D=D \cup \{v_i\}$,
          $v \leftarrow v_i$,\\
          go to step 2
      \end{itemize}

    \pause
    \item if $v=v_1$ then the result is $T$

    \pause
    \item if $v \neq v_1$ then $v \leftarrow backtrack(v)$, go to step 2
  \end{enumerate}
\end{frame}

% TODO: add example for depth-first search

\begin{frame}
  \frametitle{Breadth-First Search}

  \begin{enumerate}
    \item $T=\emptyset$, $D=\{v_1\}$, $Q=(v_1)$

    \pause
    \item if $Q$ is empty: the result is $T$
    \item if $Q$ not empty: $v \leftarrow front(Q)$, $Q \leftarrow Q - v$\\
      for $2 \leq i \leq |V|$ check the edges $(v,v_i) \in E$:
    \begin{itemize}
      \item if $v_i \notin D$ : $Q = Q + v_i$, $T = T \cup \{(v,v_i)\}$,
        $D=D \cup \{v_i\}$
       \item go to step 3
    \end{itemize}
  \end{enumerate}
\end{frame}

% TODO: add example for depth-first search

% TODO: traveling salesman problem

\section*{References}

\begin{frame}
  \frametitle{References}

  \begin{block}{Required Reading: Grimaldi}
    \begin{itemize}
      \item Chapter 11: \alert{An Introduction to Graph Theory}

      \item Chapter 7: Relations: The Second Time Around
      \begin{itemize}
        \item 7.2. \alert{Computer Recognition: Zero-One Matrices\\
                          and Directed Graphs}
      \end{itemize}

      \item Chapter 13: Optimization and Matching
      \begin{itemize}
        \item 13.1. \alert{Dijkstra's Shortest Path Algorithm}
      \end{itemize}
    \end{itemize}
  \end{block}
\end{frame}

\end{document}
