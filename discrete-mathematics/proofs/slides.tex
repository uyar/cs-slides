% Copyright (c) 2001-2012
%       H. Turgut Uyar <uyar@itu.edu.tr>
%       Ayşegül Gençata Yayımlı <gencata@itu.edu.tr>
%       Emre Harmancı <harmanci@itu.edu.tr>
%
% These notes are licensed using the
% "Creative Commons Attribution-NonCommercial-ShareAlike License".
% You are free to copy, distribute and transmit the work, and to adapt the work
% as long as you attribute the authors, do not use it for commercial purposes,
% and any derivative work is under the same or a similar license.
%
% Read the full legal code at:
% http://creativecommons.org/licenses/by-nc-sa/3.0/

\documentclass[dvipsnames]{beamer}

\usepackage{ae}
\usepackage[T1]{fontenc}
\usepackage[utf8]{inputenc}
\setbeamertemplate{navigation symbols}{}

\mode<presentation>
{
  \usetheme{Rochester}
  \setbeamercovered{transparent}
}

\title{Discrete Mathematics}
\subtitle{Theorem Proving}

\author{H. Turgut Uyar \and Ayşegül Gençata Yayımlı \and Emre Harmancı}
\date{2001-2012}

\AtBeginSubsection[]
{
  \begin{frame}<beamer>
    \frametitle{Topics}
    \tableofcontents[currentsection,currentsubsection]
  \end{frame}
}

%\beamerdefaultoverlayspecification{<+->}

\pgfdeclareimage[width=2cm]{license}{../../license}

\pgfdeclareimage{induction}{induction}
\pgfdeclareimage{error1}{error1}
\pgfdeclareimage{error2}{error2}

\begin{document}

\begin{frame}
  \titlepage
\end{frame}

\begin{frame}
  \frametitle{License}

  \pgfuseimage{license}\hfill
  \copyright 2001-2012 T. Uyar, A. Yayımlı, E. Harmancı

  \vfill
  \begin{tiny}
    You are free:
    \begin{itemize}
      \item to Share — to copy, distribute and transmit the work
      \item to Remix — to adapt the work
    \end{itemize}

    Under the following conditions:
    \begin{itemize}
      \item Attribution — You must attribute the work in the manner specified by
        the author or licensor (but not in any way that suggests that they
        endorse you or your use of the work).

      \item Noncommercial — You may not use this work for commercial purposes.

      \item Share Alike — If you alter, transform, or build upon this work, you
        may distribute the resulting work only under the same or similar license
        to this one.
    \end{itemize}
  \end{tiny}

  \vfill
  Legal code (the full license):\\
  \url{http://creativecommons.org/licenses/by-nc-sa/3.0/}
\end{frame}

\begin{frame}
  \frametitle{Topics}
  \tableofcontents
\end{frame}

\section{Basic Techniques}

\subsection{Introduction}

\begin{frame}
  \frametitle{Brute Force Method}

  \begin{itemize}
    \item examining all possible cases one by one
  \end{itemize}

  \pause
  \begin{theorem}
    Every number from the set $\{2,4,6,\dots,26\}$ can be written\\
    as the sum of at most 3 square numbers.
  \end{theorem}

  \pause
  \begin{proof}
    \begin{tabular}{lll}
      2 = 1+1   & 10 = 9+1    & 20 = 16+4\\
      4 = 4     & 12 = 4+4+4  & 22 = 9+9+4\\
      6 = 4+1+1 & 14 = 9+4+1  & 24 = 16+4+4\\
      8 = 4+4   & 16 = 16     & 26 = 25+1\\
                & 18 = 9+9    &
    \end{tabular}\\
  \end{proof}
\end{frame}

\begin{frame}
  \frametitle{Basic Rules}

  \begin{block}{Universal Specification (US)}
    $\forall x~p(x) \Rightarrow p(a)$
  \end{block}

  \pause
  \begin{block}{Universal Generalization (UG)}
    $p(a)$ for an \alert{arbitrarily chosen} $a$
      $\Rightarrow$ $\forall x~p(x)$
  \end{block}
\end{frame}

\begin{frame}
  \frametitle{Universal Specification Example}

  \begin{example}
    \begin{quote}
      All humans are mortal. Socrates is human.\\
      Therefore, Socrates is mortal.
    \end{quote}

    \pause
    \begin{itemize}
      \item $\mathcal{U}$: all humans
      \item $p(x)$: $x$ is mortal
      \item $\forall x~p(x)$: All humans are mortal.
      \item $a$: Socrates, $a \in \mathcal{U}$: Socrates is human.
      \item therefore, $p(a)$: Socrates is mortal.
    \end{itemize}
  \end{example}
\end{frame}

\begin{frame}
  \frametitle{Universal Specification Example}

  \begin{example}
    \begin{columns}
      \column{.4\textwidth}
      \[
      \frac
        {
          \begin{array}{c}
            \forall x~[j(x) \vee s(x) \rightarrow \neg p(x)]\\
            p(m)
          \end{array}
        }
        {
          \therefore \neg s(m)
        }
      \]

      \pause
      \column{.55\textwidth}
      \begin{eqnarray*}
        1. & \forall x~[j(x) \vee s(x) \rightarrow \neg p(x)] & A\\\pause
        2. & p(m)                                             & A\\\pause
        3. & j(m) \vee s(m) \rightarrow \neg p(m)             & US:1\\\pause
        4. & \neg (j(m) \vee s(m))                            & MT:3,2\\\pause
        5. & \neg j(m) \wedge \neg s(m)                       & DM:4\\\pause
        6. & \neg s(m)                                        & AndE:5
      \end{eqnarray*}
    \end{columns}
  \end{example}
\end{frame}

\begin{frame}
  \frametitle{Universal Generalization Example}

  \begin{example}
    \begin{columns}
      \column{.35\textwidth}
      \[
      \frac
        {
          \begin{array}{c}
            \forall x~[p(x) \rightarrow q(x)]\\
            \forall x~[q(x) \rightarrow r(x)]
          \end{array}
        }
        {
          \therefore \forall x~[p(x) \rightarrow r(x)]
        }
      \]

      \pause
      \column{.6\textwidth}
      \begin{eqnarray*}
        1. & \forall x~[p(x) \rightarrow q(x)] & A\\\pause
        2. & p(c) \rightarrow q(c)             & US:1\\\pause
        3. & \forall x~[q(x) \rightarrow r(x)] & A\\\pause
        4. & q(c) \rightarrow r(c)             & US:3\\\pause
        5. & p(c) \rightarrow r(c)             & HS:2,4\\\pause
        6. & \forall x~[p(x) \rightarrow r(x)] & UG:5
      \end{eqnarray*}
    \end{columns}
  \end{example}
\end{frame}

\begin{frame}
  \frametitle{Vacuous Proof}

  \begin{block}{vacuous proof}
    to prove $P \Rightarrow Q$, show that $P$ is false
  \end{block}
\end{frame}

\begin{frame}
  \frametitle{Vacuous Proof Example}

  \begin{theorem}
    $\forall S~[\emptyset \subseteq S]$
  \end{theorem}

  \pause
  \begin{proof}
    $\emptyset \subseteq S \Leftrightarrow
      \forall x~[x \in \emptyset \rightarrow x \in S]$\\\pause
    $\forall x~[x \notin \emptyset]$
  \end{proof}
\end{frame}

\begin{frame}
  \frametitle{Trivial Proof}

  \begin{block}{trivial proof}
    to prove $P \Rightarrow Q$, show that $Q$ is true
  \end{block}
\end{frame}

\begin{frame}
  \frametitle{Trivial Proof Example}

  \begin{theorem}
    $\forall x \in \mathbb{R}~[x \geq 0 \Rightarrow x^2 \geq 0]$
  \end{theorem}

  \pause
  \begin{proof}
    $\forall x \in \mathbb{R}~[x^2 \geq 0]$
  \end{proof}
\end{frame}

\subsection{Direct Proof}

\begin{frame}
  \frametitle{Direct Proof}

  \begin{block}{direct proof}
    to prove $P \Rightarrow Q$, show that $P \vdash Q$
  \end{block}
\end{frame}

\begin{frame}
  \frametitle{Direct Proof Example}

  \begin{theorem}
    $\forall a \in \mathbb{Z}~[3 | (a-2) \Rightarrow 3 | (a^2-1)]$
  \end{theorem}

  \pause
  \begin{proof}
    \begin{eqnarray*}
      3 | (a-2) & \Rightarrow & a-2 = 3k\\\pause
                & \Rightarrow & a+1 = a-2 + 3 = 3k+3 = 3(k+1)\\\pause
                & \Rightarrow & a^2-1 = (a+1)(a-1) = 3(k+1)(a-1)
    \end{eqnarray*}
  \end{proof}
\end{frame}

\begin{frame}
  \frametitle{Indirect Proof}

  \begin{block}{indirect proof}
    to prove $P \Rightarrow Q$, show that $\neg Q \vdash \neg P$
  \end{block}
\end{frame}

\begin{frame}
  \frametitle{Indirect Proof Example}

  \begin{theorem}
    $\forall x,y \in \mathbb{N}~[x \cdot y > 25
      \Rightarrow (x > 5) \vee (y > 5)]$
  \end{theorem}

  \pause
  \begin{proof}
    \begin{itemize}
      \item $\neg Q \Leftrightarrow (0 \leq x \leq 5) \wedge (0 \leq y \leq 5)$

      \pause
      \item $0 = 0 \cdot 0 \leq x \cdot y \leq 5 \cdot 5 = 25$
    \end{itemize}
  \end{proof}
\end{frame}

\begin{frame}
  \frametitle{Indirect Proof Example}

  \begin{theorem}
    $(\exists k~a,b,k \in \mathbb{N}~[ab=2k]) \Rightarrow
      (\exists i \in \mathbb{N}~[a=2i]) \vee
      (\exists j \in \mathbb{N}~[b=2j])$
  \end{theorem}

  \pause
  \begin{proof}
    \begin{itemize}
      \item $\neg Q \Leftrightarrow (\neg \exists i \in \mathbb{N}~[a=2i])
                          \wedge (\neg \exists j \in \mathbb{N}~[b=2j])$
    \end{itemize}

    \pause
    \begin{eqnarray*}
      & \Rightarrow & (\exists x \in \mathbb{N}~[a=2x+1])
               \wedge (\exists y \in \mathbb{N}~[b=2y+1])\\\pause
      & \Rightarrow & ab=(2x+1)(2y+1)\\\pause
      & \Rightarrow & ab=4xy+2(x+y)+1\\\pause
      & \Rightarrow & \neg (\exists a,b,k \in \mathbb{N}~[ab=2k])
    \end{eqnarray*}
  \end{proof}
\end{frame}

\subsection{Proof by Contradiction}

\begin{frame}
  \frametitle{Proof by Contradiction}

  \begin{block}{proof by contradiction}
    to prove $P$, show that $\neg P \vdash Q \wedge \neg Q$
  \end{block}
\end{frame}

\begin{frame}
  \frametitle{Proof by Contradiction Example}

  \begin{theorem}
    There is no largest prime number.
  \end{theorem}

  \pause
  \begin{proof}
    \begin{itemize}
      \item $\neg P$: There is a largest prime number.

      \pause
      \item $Q$: The largest prime number is $S$.

      \pause
      \item prime numbers: $2,3,5,7,11,\dots,S$

      \pause
      \item $2 \cdot 3 \cdot 5 \cdot 7 \cdot 11 \cdots S + 1$ is not divisible\\
        by a prime number between $2..S$
      \pause
      \begin{enumerate}
        \item either it is prime itself: $\neg Q$

        \pause
        \item or it is divisible by a prime number greater than $S$: $\neg Q$
      \end{enumerate}
    \end{itemize}
  \end{proof}
\end{frame}

\begin{frame}
  \frametitle{Proof by Contradiction Example}

  \begin{theorem}
    $\neg \exists a,b \in \mathbb{Z}^+~[\sqrt{2}=\frac{a}{b}]$
  \end{theorem}

  \pause
  \begin{proof}
    \begin{itemize}
      \item $\neg P$: $\exists a,b \in \mathbb{Z}^+~[\sqrt{2}=\frac{a}{b}]$
      \item $Q$: $gcd(a,b)=1$
    \end{itemize}

    \pause
    \vspace{-0.7cm}
    \begin{columns}[t]
      \column{.5\textwidth}
      \begin{eqnarray*}
        & \Rightarrow & 2 = \frac{a^2}{b^2}\\\pause
        & \Rightarrow & a^2 = 2b^2\\\pause
        & \Rightarrow & \exists i \in \mathbb{Z}^+~[a^2=2i]\\\pause
        & \Rightarrow & \exists j \in \mathbb{Z}^+~[a=2j]
      \end{eqnarray*}

      \pause
      \column{.5\textwidth}
      \begin{eqnarray*}
        & \Rightarrow & 4j^2 = 2b^2\\\pause
        & \Rightarrow & b^2 = 2j^2\\\pause
        & \Rightarrow & \exists k \in \mathbb{Z}^+~[b^2=2k]\\\pause
        & \Rightarrow & \exists l \in \mathbb{Z}^+~[b=2l]\\\pause
        & \Rightarrow & gcd(a,b) \geq 2: \neg Q
      \end{eqnarray*}
    \end{columns}
  \end{proof}
\end{frame}

\subsection{Equivalence Proofs}

\begin{frame}
  \frametitle{Equivalence Proofs}

  \begin{itemize}
    \item to prove $P \Leftrightarrow Q$, both $P \Rightarrow Q$ and
      $Q \Rightarrow P$ must be proven

    \pause
    \medskip
    \item a method to prove
      $P_1 \Leftrightarrow P_2 \Leftrightarrow \cdots \Leftrightarrow P_n$:\\
      $P_1 \Rightarrow P_2 \Rightarrow \cdots \Rightarrow P_n \Rightarrow P_1$
  \end{itemize}
\end{frame}

\begin{frame}
  \frametitle{Equivalence Proof Example}

  \begin{theorem}
    $a,b,n,q_1,r_1,q_2,r_2 \in \mathbb{Z}^+$\\
    $a = q_1 \cdot n + r_1$\\
    $b = q_2 \cdot n + r_2$\\

    \bigskip
    $r_1 = r_2 \Leftrightarrow n | (a - b)$
  \end{theorem}
\end{frame}

\begin{frame}
  \frametitle{Equivalence Proof Example}

  \begin{columns}[t]
    \column{.55\textwidth}
    \begin{proof}[$r_1 = r_2 \Rightarrow n | (a - b)$]
      \begin{eqnarray*}
        a - b & = & (q_1 \cdot n + r_1)\\
              &   & -(q_2 \cdot n + r_2)\\\pause
              & = & (q_1 - q_2) \cdot n\\
              &   & + (r_1 - r_2)\\\pause
        r_1 = r_2 & \Rightarrow & r_1 - r_2 = 0\\\pause
                  & \Rightarrow & a - b = (q_1 - q_2) \cdot n
      \end{eqnarray*}
    \end{proof}

    \pause
    \column{.45\textwidth}
    \begin{proof}[$n | (a - b) \Rightarrow r_1 = r_2$]
      \begin{eqnarray*}
        a - b & = & (q_1 \cdot n + r_1)\\
              &   & -(q_2 \cdot n + r_2)\\\pause
              & = & (q_1 - q_2) \cdot n\\
              &   & + (r_1 - r_2)\\\pause
        n | (a - b) & \Rightarrow & r_1 - r_2 = 0\\\pause
                    & \Rightarrow & r_1 = r_2
      \end{eqnarray*}
    \end{proof}
  \end{columns}
\end{frame}

\begin{frame}
  \frametitle{Equivalence Proof Example}

  \begin{theorem}
    \begin{eqnarray*}
      &                 & A \subseteq B\\
      & \Leftrightarrow & A \cup B = B\\
      & \Leftrightarrow & A \cap B = A\\
      & \Leftrightarrow & \overline{B} \subseteq \overline{A}
    \end{eqnarray*}
  \end{theorem}
\end{frame}

\begin{frame}
  \frametitle{Equivalence Proof Example}

  \begin{proof}[$A \subseteq B \Rightarrow A \cup B = B$]
    $A \cup B = B \Leftrightarrow
      A \cup B \subseteq B \wedge B \subseteq A \cup B$

    \pause
    \bigskip
    \begin{columns}
      \column{.4\textwidth}
      $B \subseteq A \cup B$

      \pause
      \medskip
      \column{.5\textwidth}
      \begin{eqnarray*}
        x \in A \cup B & \Rightarrow & x \in A \vee x \in B\\\pause
        A \subseteq B  & \Rightarrow & x \in B\\\pause
                       & \Rightarrow & A \cup B \subseteq B
      \end{eqnarray*}
    \end{columns}
  \end{proof}
\end{frame}

\begin{frame}
  \frametitle{Equivalence Proof Example}

  \begin{proof}[$A \cup B = B \Rightarrow A \cap B = A$]
    $A \cap B = A \Leftrightarrow
      A \cap B \subseteq A \wedge A \subseteq A \cap B$

    \pause
    \bigskip
    \begin{columns}
      \column{.4\textwidth}
      $A \cap B \subseteq A$

      \pause
      \medskip
      \column{.5\textwidth}
      \begin{eqnarray*}
        y \in A      & \Rightarrow & y \in A \cup B\\\pause
        A \cup B = B & \Rightarrow & y \in B\\\pause
                     & \Rightarrow & y \in A \cap B\\\pause
                     & \Rightarrow & A \subseteq A \cap B
      \end{eqnarray*}
    \end{columns}
  \end{proof}
\end{frame}

\begin{frame}
  \frametitle{Equivalence Proof Example}

  \begin{proof}[$A \cap B = A \Rightarrow \overline{B} \subseteq \overline{A}$]
    \begin{eqnarray*}
      z \in \overline{B} & \Rightarrow & z \notin B\\\pause
                         & \Rightarrow & z \notin A \cap B\\\pause
      A \cap B = A       & \Rightarrow & z \notin A\\\pause
                         & \Rightarrow & z \in \overline{A}\\\pause
                         & \Rightarrow & \overline{B} \subseteq \overline{A}
    \end{eqnarray*}
  \end{proof}
\end{frame}

\begin{frame}
  \frametitle{Equivalence Proof Example}

  \begin{proof}[$\overline{B} \subseteq \overline{A} \Rightarrow A \subseteq B$]
    \begin{eqnarray*}
      \neg(A \subseteq B)
        & \Rightarrow & \exists w~[w \in A \wedge w \notin B]\\\pause
        & \Rightarrow & \exists w~[w \notin \overline{A} \wedge w \in \overline{B}]\\\pause
        & \Rightarrow & \neg(\overline{B} \subseteq \overline{A})
    \end{eqnarray*}
  \end{proof}
\end{frame}

\section{Induction}

\subsection{Introduction}

\begin{frame}
  \frametitle{Induction}

  \begin{definition}
    $S(n)$: a predicate defined on $n \in \mathbb{Z}^+$

    \pause
    \medskip
    $S(n_0) \wedge (\forall k \geq n_0~[S(k) \Rightarrow S(k+1)])
      \Rightarrow \forall n \geq n_0~S(n)$
  \end{definition}

  \pause
  \medskip
  \begin{itemize}
    \item $S(n_0)$: \emph{base step}
    \item $\forall k \geq n_0~[S(k) \Rightarrow S(k+1)]$: \emph{induction step}
  \end{itemize}
\end{frame}

\begin{frame}
  \frametitle{Induction}

  \begin{center}
    \pgfuseimage{induction}
  \end{center}
\end{frame}

\begin{frame}
  \frametitle{Induction Example}

  \begin{theorem}
    $\forall n \in \mathbb{Z}^+~[1+3+5+\cdots+(2n-1)=n^2]$
  \end{theorem}

  \pause
  \begin{proof}
    \begin{itemize}
      \item $n=1$: $1=1^2$

      \pause
      \item $n=k$: assume $1+3+5+\cdots+(2k-1)=k^2$

      \pause
      \item $n=k+1$:
      \begin{eqnarray*}
        &   & 1+3+5+\cdots+(2k-1)+(2k+1)\\\pause
        & = & k^2+2k+1\\\pause
        & = & (k+1)^2
      \end{eqnarray*}
    \end{itemize}
  \end{proof}
\end{frame}

\begin{frame}
  \frametitle{Induction Example}

  \begin{theorem}
    $\forall n \in \mathbb{Z}^+, n \geq 4~[2^n < n!]$
  \end{theorem}

  \pause
  \begin{proof}
    \begin{itemize}
      \item $n=4$: $2^4=16<24=4!$

      \pause
      \item $n=k$: assume $2^k < k!$

      \pause
      \item $n=k+1$:\\
        $2^{k+1} = 2 \cdot 2^k < 2 \cdot k! < (k+1) \cdot k! = (k+1)!$
    \end{itemize}
  \end{proof}
\end{frame}

\begin{frame}
  \frametitle{Induction Example}

  \begin{theorem}
    $\forall n \in \mathbb{Z}^+, n \geq 14~\exists i,j \in \mathbb{N}~[n=3i+8j]$
  \end{theorem}

  \pause
  \begin{proof}
    \begin{itemize}
      \item $n=14$: $14=3 \cdot 2 + 8 \cdot 1$

      \pause
      \item $n=k$: assume $k=3i+8j$

      \pause
      \item $n=k+1$:
      \begin{itemize}
        \item $k=3i+8j, j>0 \Rightarrow k+1=k-8+3 \cdot 3$\\
          $\Rightarrow k+1=3(i+3)+8(j-1)$
        \item $k=3i+8j, j=0, i \geq 5 \Rightarrow k+1=k-5 \cdot 3+2 \cdot 8$\\
          $\Rightarrow k+1=3(i-5)+8(j+2)$
      \end{itemize}
    \end{itemize}
  \end{proof}
\end{frame}

\subsection{Strong Induction}

\begin{frame}
  \frametitle{Strong Induction}

  \begin{definition}
    $S(n_0) \wedge
      (\forall k \geq n_0~[(\forall i \leq k~S(i)) \Rightarrow S(k+1)])
      \Rightarrow \forall n \geq n_0~S(n)$
  \end{definition}
\end{frame}

\begin{frame}
  \frametitle{Strong Induction Example}

  \begin{theorem}
    $\forall n \in \mathbb{Z}^+, n \geq 2$\\
      n can be written as the product of prime numbers
  \end{theorem}

  \pause
  \begin{proof}
    \begin{itemize}
      \item $n=2$: $2=2$

      \pause
      \item assume that the theorem is true for $\forall i \leq k$

      \pause
      \item $n=k+1$:
      \begin{enumerate}
        \item if prime: $n=n$

        \pause
        \item if not prime: $n=u \cdot v$\\
          $u < k \wedge v < k \Rightarrow$ both $u$ and $v$ can be written\\
          as the product of prime numbers
      \end{enumerate}
    \end{itemize}
  \end{proof}
\end{frame}

\begin{frame}
  \frametitle{Strong Induction Example}

  \begin{theorem}
    $\forall n \in \mathbb{Z}^+, n \geq 14~\exists i,j \in \mathbb{N}~[n=3i+8j]$
  \end{theorem}

  \pause
  \begin{proof}
    \begin{itemize}
      \item $n=14$: $14=3 \cdot 2 + 8 \cdot 1$\\
        $n=15$: $15=3 \cdot 5 + 8 \cdot 0$\\
        $n=16$: $16=3 \cdot 0 + 8 \cdot 2$

      \pause
      \item $n \leq k$: assume $k=3i+8j$

      \pause
      \item $n=k+1$: $k+1=(k-2)+3$
    \end{itemize}
  \end{proof}
\end{frame}

\begin{frame}
  \frametitle{Flawed Induction Examples}

  \begin{theorem}
    $\forall n \in \mathbb{Z}^+~[1+2+3+\cdots+n=\frac{n^2+n+2}{2}]$
  \end{theorem}

  \pause
  \begin{block}{invalid base step}
    \begin{itemize}
      \item $n=k$: assume $1+2+3+\cdots+k=\frac{k^2+k+2}{2}$

      \pause
      \item $n=k+1$:
      \begin{eqnarray*}
        &   & 1+2+3+\cdots+k+(k+1)\\\pause
        & = & \frac{k^2+k+2}{2}+k+1
          =   \frac{k^2+k+2}{2}+\frac{2k+2}{2}\\\pause
        & = & \frac{k^2+3k+4}{2}
          =   \frac{(k+1)^2+(k+1)+2}{2}
      \end{eqnarray*}

      \pause
      \item $n=1$: $1 \neq \frac{1^2+1+2}{2}=2$
    \end{itemize}
  \end{block}
\end{frame}

\begin{frame}
  \frametitle{Flawed Induction Examples}

  \begin{center}
    \pgfuseimage{error1}
  \end{center}
\end{frame}

\begin{frame}
  \frametitle{Flawed Induction Examples}

  \begin{theorem}
    All horses are of the same color.\\
    \pause
    \bigskip
    $A(n)$: All horses in sets of $n$ cardinality are of the same color.

    \medskip
    $\forall n \in \mathbb{N^+}~A(n)$
  \end{theorem}
\end{frame}

\begin{frame}
  \frametitle{Flawed Induction Examples}

  \begin{block}{Flawed induction over $n$}
    \begin{itemize}
      \item $n=1$: $A(1)$\\
        All horses in sets of $1$~horse are of the same color.

      \pause
      \medskip
      \item $n=k$: assume $A(k)$ is true\\
        All horses in sets of $k$~horses are of the same color.
      \pause
      \medskip
      \item $A(k+1)=\{a_1,a_2,\dots,a_k\} \cup \{a_2,a_3,\dots,a_{k+1}\}$
      \begin{itemize}
        \item All horses in set $\{a_1,a_2,\dots,a_k\}$ are of the same color ($a_2$)
        \item All horses in set $\{a_2,a_3,\dots,a_{k+1}\}$ are of the same color
          ($a_2$)
      \end{itemize}
    \end{itemize}
  \end{block}
\end{frame}

\begin{frame}
  \frametitle{Flawed Induction Examples}

  \begin{center}
    \pgfuseimage{error2}
  \end{center}
\end{frame}

\section*{References}

\begin{frame}
  \frametitle{References}

  \begin{block}{Required Text: Grimaldi}
    \begin{itemize}
      \item Chapter 2: Fundamentals of Logic
      \begin{itemize}
        \item 2.5. \alert{Quantifiers, Definitions, and the Proofs of Theorems}
      \end{itemize}

      \item Chapter 4: Properties of Integers: Mathematical Induction
      \begin{itemize}
        \item 4.1. \alert{The Well-Ordering Principle: Mathematical Induction}
      \end{itemize}
    \end{itemize}
  \end{block}

  \begin{block}{Supplementary Text: O'Donnell, Hall, Page}
    \begin{itemize}
      \item Chapter 4: Induction
    \end{itemize}
  \end{block}
\end{frame}

\end{document}
