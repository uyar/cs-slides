% Copyright (c) 2001-2014
%       H. Turgut Uyar <uyar@itu.edu.tr>
%       Ayşegül Gençata Yayımlı <gencata@itu.edu.tr>
%       Emre Harmancı <harmanci@itu.edu.tr>
%
% This work is licensed under a "Creative Commons
% Attribution-NonCommercial-ShareAlike 4.0 International License".
% For more information, please visit:
% https://creativecommons.org/licenses/by-nc-sa/4.0/

\documentclass[dvipsnames]{beamer}

\usepackage{ae}
\usepackage[T1]{fontenc}
\usepackage[utf8]{inputenc}
\setbeamertemplate{navigation symbols}{}
\setbeamersize{text margin left=2em, text margin right=2em}

\mode<presentation>
{
  \usetheme{Rochester}
  \setbeamercovered{transparent}
}

\title{Discrete Mathematics}
\subtitle{Relations and Functions}

\author{H. Turgut Uyar \and Ayşegül Gençata Yayımlı \and Emre Harmancı}
\date{2001-2014}

\AtBeginSubsection[]
{
  \begin{frame}<beamer>
    \frametitle{Topics}
    \tableofcontents[currentsection,currentsubsection]
  \end{frame}
}

%\beamerdefaultoverlayspecification{<+->}

\pgfdeclareimage[height=1cm]{license}{../license}

\pgfdeclareimage{relation}{relation}
\pgfdeclareimage{composite1}{composite1}
\pgfdeclareimage{composite2}{composite2}
\pgfdeclareimage{compatible1}{compatible1}
\pgfdeclareimage{compatible2}{compatible2}
\pgfdeclareimage{compatible3}{compatible3}

\begin{document}

\begin{frame}
  \titlepage
\end{frame}

\begin{frame}
  \frametitle{License}

  \pgfuseimage{license}\hfill
  \copyright~2001-2014 T. Uyar, A. Yayımlı, E. Harmancı

  \vfill
  \begin{footnotesize}
    You are free to:
    \begin{itemize}
      \itemsep0em
      \item Share -- copy and redistribute the material in any medium or format
      \item Adapt -- remix, transform, and build upon the material
    \end{itemize}

    Under the following terms:
    \begin{itemize}
      \itemsep0em
      \item Attribution -- You must give appropriate credit, provide a link to
        the license, and indicate if changes were made.

      \item NonCommercial -- You may not use the material for commercial
        purposes.

      \item ShareAlike -- If you remix, transform, or build upon the material,
        you must distribute your contributions under the same license as the
        original.
    \end{itemize}
  \end{footnotesize}

  \begin{small}
    For more information:\\
    \url{https://creativecommons.org/licenses/by-nc-sa/4.0/}

    \smallskip
    Read the full license:\\
    \url{https://creativecommons.org/licenses/by-nc-sa/4.0/legalcode}
  \end{small}
\end{frame}

\begin{frame}
  \frametitle{Topics}
  \tableofcontents
\end{frame}

\section{Relations}

\subsection{Introduction}

\begin{frame}
  \frametitle{Relation}

  \begin{definition}
    \alert{relation}: $\alpha \subseteq A \times B \times C \times \dots \times N$
  \end{definition}

  \begin{itemize}
    \item \alert{tuple}: an element of a relation
  \end{itemize}

  \pause
  \medskip
  \begin{itemize}
    \item $\alpha \subseteq A \times B$: \emph{binary relation}
    \item $a \alpha b$ is the same as $(a,b) \in \alpha$
  \end{itemize}
\end{frame}

\begin{frame}
  \frametitle{Relation Example}

  \begin{example}
    $A=\{a_1,a_2,a_3,a_4\}, B=\{b_1,b_2,b_3\}$\\
    $\alpha = \{(a_1,b_1),(a_1,b_3),(a_2,b_2),(a_2,b_3),
                (a_3,b_1),(a_3,b_3),(a_4,b_1)\}$

    \pause
    \medskip
    \begin{columns}
      \column{.3\textwidth}
      \begin{center}
        \pgfuseimage{relation}
      \end{center}

      \column{.3\textwidth}
      \[
        \begin{array}{c|ccc|}
              & b_1 & b_2 & b_3\\\hline
          a_1 &  1  &  0  &  1\\
          a_2 &  0  &  1  &  1\\
          a_3 &  1  &  0  &  1\\
          a_4 &  1  &  0  &  0\\\hline
        \end{array}
      \]

      \column{.3\textwidth}
      \[
        M_\alpha =
          \begin{array}{|ccc|}
            1 & 0 & 1\\
            0 & 1 & 1\\
            1 & 0 & 1\\
            1 & 0 & 0
          \end{array}
      \]
    \end{columns}
  \end{example}
\end{frame}

\begin{frame}
  \frametitle{Relation Composition}

  \begin{definition}
    \alert{relation composition}:\\
      let $\alpha \subseteq A \times B$, $\beta \subseteq B \times C$\\
      $\alpha \beta = \{(a,c)~|~a \in A, c \in C,
                \exists b \in B~[a \alpha b \wedge b \beta c]\}$
  \end{definition}

  \begin{example}
    \begin{columns}
      \column{.45\textwidth}
      \begin{center}
        \pgfuseimage{composite1}
      \end{center}

      \column{.45\textwidth}
      \begin{center}
        \pgfuseimage{composite2}
      \end{center}
    \end{columns}
  \end{example}
\end{frame}

\begin{frame}
  \frametitle{Relation Composition}

  \begin{itemize}
    \item $M_{\alpha \beta} = M_{\alpha} \times M_{\beta}$
    \item using logical operations:\\
      $1:T~~~~0:F~~~~\cdot:\wedge~~~~+:\vee$
  \end{itemize}

  \pause
  \begin{example}
    \begin{columns}
      \column{.28\textwidth}
      \[
        M_\alpha =
          \begin{array}{|ccc|}
            1 & 0 & 0\\
            0 & 0 & 1\\
            0 & 1 & 1\\
            0 & 1 & 0\\
            1 & 0 & 1
         \end{array}
      \]

      \column{.28\textwidth}
      \[
        M_\beta =
          \begin{array}{|cccc|}
            1 & 1 & 0 & 0\\
            0 & 0 & 1 & 1\\
            0 & 1 & 1 & 0
          \end{array}
      \]

      \column{.37\textwidth}
      \[
        M_{\alpha \beta} =
          \begin{array}{|cccc|}
            1 & 1 & 0 & 0\\
            0 & 1 & 1 & 0\\
            0 & 1 & 1 & 1\\
            0 & 0 & 1 & 1\\
            1 & 1 & 1 & 0
          \end{array}
      \]
    \end{columns}
  \end{example}
\end{frame}

\begin{frame}
  \frametitle{Relation Composition Associativity}

  \begin{proof}[$(\alpha \beta) \gamma = \alpha (\beta \gamma)$]
    \pause
    \begin{eqnarray*}
      &                 & (a,d) \in (\alpha \beta)\gamma\\\pause
      & \Leftrightarrow & \exists c~[(a,c) \in \alpha \beta
                              \wedge (c,d) \in \gamma]\\\pause
      & \Leftrightarrow & \exists c~[\exists b~[(a,b) \in \alpha
                                         \wedge (b,c) \in \beta]
                                         \wedge (c,d) \in \gamma]\\\pause
      & \Leftrightarrow & \exists b~[(a,b) \in \alpha
                              \wedge \exists c~[(b,c) \in \beta
                              \wedge (c,d) \in \gamma]]\\\pause
      & \Leftrightarrow & \exists b~[(a,b) \in \alpha
                              \wedge (b,d) \in \beta \gamma]\\\pause
      & \Leftrightarrow & (a,d) \in \alpha (\beta \gamma)
    \end{eqnarray*}
  \end{proof}
\end{frame}

\begin{frame}
  \frametitle{Relation Composition Theorems}

  \begin{itemize}
    \item let $\alpha , \delta \subseteq A \times B$, and\\
          let $\beta , \gamma \subseteq B \times C$

    \medskip
    \item $\alpha (\beta \cup \gamma) = \alpha \beta \cup \alpha \gamma$

    \item $\alpha (\beta \cap \gamma)
      \subseteq \alpha \beta \cap \alpha \gamma$

    \item $(\alpha \cup \delta) \beta = \alpha \beta \cup \delta \beta$

    \item $(\alpha \cap \delta) \beta
      \subseteq \alpha \beta \cap \delta \beta$

    \item $(\alpha \subseteq \delta \wedge \beta \subseteq \gamma)
      \Rightarrow \alpha \beta \subseteq \delta \gamma$
  \end{itemize}
\end{frame}

\begin{frame}
  \frametitle{Relation Composition Theorems}

  \begin{proof}[$\alpha (\beta \cup \gamma) = \alpha \beta \cup \alpha \gamma$]
    \pause
    \begin{eqnarray*}
      &                 & (a,c) \in \alpha (\beta \cup \gamma)\\\pause
      & \Leftrightarrow & \exists b~[(a,b) \in \alpha
                              \wedge (b,c) \in (\beta \cup \gamma)]\\\pause
      & \Leftrightarrow & \exists b~[(a,b) \in \alpha
                             \wedge ((b,c) \in \beta
                                \vee (b,c) \in \gamma)]\\\pause
      & \Leftrightarrow & \exists b~[((a,b) \in \alpha \wedge (b,c) \in \beta)\\
      &                 &     ~~\vee ((a,b) \in \alpha \wedge (b,c) \in \gamma)]\\\pause
      & \Leftrightarrow & (a,c) \in \alpha \beta \vee (a,c) \in \alpha \gamma\\\pause
      & \Leftrightarrow & (a,c) \in \alpha \beta \cup \alpha \gamma
    \end{eqnarray*}
  \end{proof}
\end{frame}

\begin{frame}
  \frametitle{Converse Relation}

  \begin{definition}
    $\alpha^{-1} = \{(b,a)~|~(a,b) \in \alpha \}$
  \end{definition}

  \begin{itemize}
    \item $M_{\alpha^{-1}} = M_{\alpha}^T$
  \end{itemize}
\end{frame}

\begin{frame}
  \frametitle{Converse Relation Theorems}

  \begin{itemize}
    \item $(\alpha^{-1})^{-1} = \alpha$
    \item $(\alpha \cup \beta)^{-1} = \alpha^{-1} \cup \beta^{-1}$
    \item $(\alpha \cap \beta)^{-1} = \alpha^{-1} \cap \beta^{-1}$
    \item $\overline{\alpha}^{-1} = \overline{\alpha^{-1}}$
    \item $(\alpha - \beta)^{-1} = \alpha^{-1} - \beta^{-1}$
    \item $\alpha \subset \beta \Rightarrow \alpha^{-1} \subset \beta^{-1}$
  \end{itemize}
\end{frame}

\begin{frame}
  \frametitle{Converse Relation Theorems}

  \begin{proof}[$\overline{\alpha}^{-1} = \overline{\alpha^{-1}}$]
    \pause
    \begin{eqnarray*}
      &                 & (b,a) \in \overline{\alpha}^{-1}\\\pause
      & \Leftrightarrow & (a,b) \in \overline{\alpha}\\\pause
      & \Leftrightarrow & (a,b) \notin \alpha\\\pause
      & \Leftrightarrow & (b,a) \notin \alpha^{-1}\\\pause
      & \Leftrightarrow & (b,a) \in \overline{\alpha^{-1}}
    \end{eqnarray*}
  \end{proof}
\end{frame}

\begin{frame}
  \frametitle{Converse Relation Theorems}

  \begin{proof}[$(\alpha \cap \beta)^{-1} = \alpha^{-1} \cap \beta^{-1}$]
    \pause
    \begin{eqnarray*}
      &                 & (b,a) \in (\alpha \cap \beta)^{-1}\\\pause
      & \Leftrightarrow & (a,b) \in (\alpha \cap \beta)\\\pause
      & \Leftrightarrow & (a,b) \in \alpha \wedge (a,b) \in \beta\\\pause
      & \Leftrightarrow & (b,a) \in \alpha^{-1}
                   \wedge (b,a) \in \beta^{-1}\\\pause
      & \Leftrightarrow & (b,a) \in \alpha^{-1} \cap \beta^{-1}
    \end{eqnarray*}
  \end{proof}
\end{frame}

\begin{frame}
  \frametitle{Converse Relation Theorems}

  \begin{proof}[$(\alpha - \beta)^{-1} = \alpha^{-1} - \beta^{-1}$]
    \pause
    \begin{eqnarray*}
      (\alpha - \beta)^{-1} & = & (\alpha \cap \overline{\beta})^{-1}\\\pause
                            & = & \alpha^{-1} \cap \overline{\beta}^{-1}\\\pause
                            & = & \alpha^{-1} \cap \overline{\beta^{-1}}\\\pause
                            & = & \alpha^{-1} - \beta^{-1}
    \end{eqnarray*}
  \end{proof}
\end{frame}

\begin{frame}
  \frametitle{Relation Composition Converse}

  \begin{theorem}
    $(\alpha \beta)^{-1} = \beta^{-1} \alpha^{-1}$
  \end{theorem}

  \pause
  \begin{proof}
    \begin{eqnarray*}
      &                 & (c,a) \in (\alpha \beta)^{-1}\\\pause
      & \Leftrightarrow & (a,c) \in \alpha \beta\\\pause
      & \Leftrightarrow & \exists b~[(a,b) \in \alpha
                              \wedge (b,c) \in \beta]\\\pause
      & \Leftrightarrow & \exists b~[(b,a) \in \alpha^{-1}
                              \wedge (c,b) \in \beta^{-1}]\\\pause
      & \Leftrightarrow & (c,a) \in \beta^{-1}\alpha^{-1}
    \end{eqnarray*}
  \end{proof}
\end{frame}

\subsection{Relation Properties}

\begin{frame}
  \frametitle{Relation Properties}

  \begin{itemize}
    \item $\alpha \subseteq A \times A$
    \begin{itemize}
      \item \emph{binary relation on $A$}
    \end{itemize}

    \pause
    \medskip
    \item let $\alpha^n$ mean $\alpha \alpha \cdots \alpha$

    \pause
    \medskip
    \item \alert{identity relation}: $E = \{(x,x)~|~x \in A\}$
  \end{itemize}
\end{frame}

\begin{frame}
  \frametitle{Reflexivity}

  \begin{block}{reflexive}
    $\alpha \subseteq A \times A$\\
    $\forall a~[a \alpha a]$
  \end{block}

  \pause
  \begin{itemize}
    \item $E \subseteq \alpha$

    \pause
    \item nonreflexive:\\
      $\exists a~[\neg (a \alpha a)]$

    \pause
    \item irreflexive:\\
      $\forall a~[\neg (a \alpha a)]$
  \end{itemize}
\end{frame}

\begin{frame}
  \frametitle{Reflexivity Examples}

  \begin{columns}[t]
    \column{.45\textwidth}
    \begin{example}
      $\mathcal{R}_1 \subseteq \{1,2\} \times \{1,2\}$\\
      $\mathcal{R}_1 = \{(1,1), (1,2), (2,2)\}$

      \medskip
      \begin{itemize}
        \item $\mathcal{R}_1$ is reflexive
      \end{itemize}
    \end{example}

    \pause
    \column{.45\textwidth}
    \begin{example}
      $\mathcal{R}_2 \subseteq \{1,2,3\} \times \{1,2,3\}$\\
      $\mathcal{R}_2 = \{(1,1), (1,2), (2,2)\}$

      \medskip
      \begin{itemize}
        \item $\mathcal{R}_2$ is nonreflexive
      \end{itemize}
    \end{example}
  \end{columns}
\end{frame}

\begin{frame}
  \frametitle{Reflexivity Examples}

  \begin{example}
    $\mathcal{R} \subseteq \{1,2,3\} \times \{1,2,3\}$\\
    $\mathcal{R} = \{(1,2), (2,1), (2,3)\}$

    \medskip
    \begin{itemize}
      \item $\mathcal{R}$ is irreflexive
    \end{itemize}
  \end{example}
\end{frame}

\begin{frame}
  \frametitle{Reflexivity Examples}

  \begin{example}
    $\mathcal{R} \subseteq \mathbb{Z} \times \mathbb{Z}$\\
    $\mathcal{R} = \{(a,b)~|~ab \geq 0\}$

    \medskip
    \begin{itemize}
      \item $\mathcal{R}$ is reflexive
    \end{itemize}
  \end{example}
\end{frame}

\begin{frame}
  \frametitle{Symmetry}

  \begin{block}{symmetric}
    $\alpha \subseteq A \times A$\\
    $\forall a,b~[(a=b) \vee (a \alpha b \wedge b \alpha a)
                        \vee (\neg(a \alpha b) \wedge \neg(b \alpha a))]$

    \pause
    $\forall a,b~[(a=b) \vee (a \alpha b \leftrightarrow b \alpha a)]$
  \end{block}

  \pause
  \begin{itemize}
    \item $\alpha^{-1} = \alpha$

    \pause
    \item asymmetric:\\
      $\exists a,b~[(a \neq b) \wedge ((a \alpha b \wedge \neg(b \alpha a))
                               \vee (\neg (a \alpha b) \wedge b \alpha a))]$

    \pause
    \item antisymmetric:
    \vspace{-12pt}
    \begin{eqnarray*}
                      & \forall a,b~
                    [(a=b) \vee (a \alpha b \rightarrow \neg(b \alpha a))]\\\pause
      \Leftrightarrow & \forall a,b~
                    [(a=b) \vee \neg(a \alpha b) \vee \neg(b \alpha a)]\\\pause
      \Leftrightarrow & \forall a,b~
                    [\neg(a \alpha b \wedge b \alpha a) \vee (a=b)]\\\pause
      \Leftrightarrow & \forall a,b~
                    [(a \alpha b \wedge b \alpha a) \rightarrow (a=b)]
    \end{eqnarray*}
  \end{itemize}
\end{frame}

\begin{frame}
  \frametitle{Symmetry Examples}

  \begin{example}
    $\mathcal{R} \subseteq \{1,2,3\} \times \{1,2,3\}$\\
    $\mathcal{R} = \{(1,2), (2,1), (2,3)\}$

    \medskip
    \begin{itemize}
      \item $\mathcal{R}$ is asymmetric
    \end{itemize}
  \end{example}
\end{frame}

\begin{frame}
  \frametitle{Symmetry Examples}

  \begin{example}
    $\mathcal{R} \subseteq \mathbb{Z} \times \mathbb{Z}$\\
    $\mathcal{R} = \{(a,b)~|~ab \geq 0\}$

    \medskip
    \begin{itemize}
      \item $\mathcal{R}$ is symmetric
    \end{itemize}
  \end{example}
\end{frame}

\begin{frame}
  \frametitle{Symmetry Examples}

  \begin{example}
    $\mathcal{R} \subseteq \{1,2,3\} \times \{1,2,3\}$\\
    $\mathcal{R} = \{(1,1), (2,2)\}$

    \begin{itemize}
      \item $\mathcal{R}$ is symmetric and antisymmetric
    \end{itemize}
  \end{example}
\end{frame}

\begin{frame}
  \frametitle{Transitivity}

  \begin{block}{transitive}
    $\alpha \subseteq A \times A$\\
    $\forall a,b,c~[(a \alpha b \wedge b \alpha c) \rightarrow (a \alpha c)]$
  \end{block}

  \pause
  \begin{itemize}
    \item $\alpha^2 \subseteq \alpha$

    \pause
    \item nontransitive:\\
      $\exists a,b,c~[(a \alpha b \wedge b \alpha c) \wedge \neg (a \alpha c)]$

    \pause
    \item antitransitive:\\
      $\forall a,b,c~[(a \alpha b \wedge b \alpha c) \rightarrow \neg (a \alpha c)]$
  \end{itemize}
\end{frame}

\begin{frame}
  \frametitle{Transitivity Examples}

  \begin{example}
    $\mathcal{R} \subseteq \{1,2,3\} \times \{1,2,3\}$\\
    $\mathcal{R} = \{(1,2), (2,1), (2,3)\}$

    \medskip
    \begin{itemize}
      \item $\mathcal{R}$ is antitransitive
    \end{itemize}
  \end{example}
\end{frame}

\begin{frame}
  \frametitle{Transitivity Examples}

  \begin{example}
    $\mathcal{R} \subseteq \mathbb{Z} \times \mathbb{Z}$\\
    $\mathcal{R} = \{(a,b)~|~ab \geq 0\}$

    \medskip
    \begin{itemize}
      \item $\mathcal{R}$ is nontransitive
    \end{itemize}
  \end{example}
\end{frame}

\begin{frame}
  \frametitle{Converse Relation Properties}

  \begin{theorem}
    The reflexivity, symmetry and transitivity properties\\
    are preserved in the converse relation.
  \end{theorem}
\end{frame}

\begin{frame}
  \frametitle{Closures}

  \begin{itemize}
    \item reflexive closure:\\
      $r_{\alpha} = \alpha \cup E$

    \pause
    \medskip
    \item symmetric closure:\\
      $s_{\alpha} = \alpha \cup \alpha^{-1}$

    \pause
    \medskip
    \item transitive closure:\\
      $t_{\alpha} = \bigcup_{i=1,2,3, \ldots}~\alpha^i
        = \alpha \cup \alpha^2 \cup \alpha^3 \cup \cdots$
  \end{itemize}
\end{frame}

% TODO: add examples for closures

\begin{frame}
  \frametitle{Special Relations}

  \begin{block}{predecessor - successor}
    $\mathcal{R} \subseteq \mathbb{Z} \times \mathbb{Z}$\\
    $\mathcal{R} = \{(a,b)~|~a-b=1\}$

    \medskip
    \begin{itemize}
      \item irreflexive
      \item antisymmetric
      \item antitransitive
    \end{itemize}
  \end{block}
\end{frame}

\begin{frame}
  \frametitle{Special Relations}

  \begin{block}{adjacency}
    $\mathcal{R} \subseteq \mathbb{Z} \times \mathbb{Z}$\\
    $\mathcal{R} = \{(a,b)~|~|a-b|=1\}$

    \medskip
    \begin{itemize}
      \item irreflexive
      \item symmetric
      \item antitransitive
    \end{itemize}
  \end{block}
\end{frame}

\begin{frame}
  \frametitle{Special Relations}

  \begin{block}{strict order}
    $\mathcal{R} \subseteq \mathbb{Z} \times \mathbb{Z}$\\
    $\mathcal{R} = \{(a,b)~|~a<b\}$

    \medskip
    \begin{itemize}
      \item irreflexive
      \item antisymmetric
      \item transitive
    \end{itemize}
  \end{block}
\end{frame}

\begin{frame}
  \frametitle{Special Relations}

  \begin{block}{partial order}
    $\mathcal{R} \subseteq \mathbb{Z} \times \mathbb{Z}$\\
    $\mathcal{R} = \{(a,b)~|~ a \leq b\}$

    \medskip
    \begin{itemize}
      \item reflexive
      \item antisymmetric
      \item transitive
    \end{itemize}
  \end{block}
\end{frame}

\begin{frame}
  \frametitle{Special Relations}

  \begin{block}{preorder}
    $\mathcal{R} \subseteq \mathbb{Z} \times \mathbb{Z}$\\
    $\mathcal{R} = \{(a,b)~|~|a| \leq |b|\}$

    \medskip
    \begin{itemize}
      \item reflexive
      \item asymmetric
      \item transitive
    \end{itemize}
  \end{block}
\end{frame}

\begin{frame}
  \frametitle{Special Relations}

  \begin{block}{limited difference}
    $\mathcal{R} \subseteq \mathbb{Z} \times \mathbb{Z}, m \in \mathbb{Z}^+$\\
    $\mathcal{R} = \{(a,b)~|~|a-b| \leq m\}$

    \medskip
    \begin{itemize}
      \item reflexive
      \item symmetric
      \item nontransitive
    \end{itemize}
  \end{block}
\end{frame}

\begin{frame}
  \frametitle{Special Relations}

  \begin{block}{comparability}
    $\mathcal{R} \subseteq \mathbb{U} \times \mathbb{U}$\\
    $\mathcal{R} = \{(a,b)~|~(a \subseteq b) \vee (b \subseteq a)\}$

    \medskip
    \begin{itemize}
      \item reflexive
      \item symmetric
      \item nontransitive
    \end{itemize}
  \end{block}
\end{frame}

\begin{frame}
  \frametitle{Special Relations}

  \begin{block}{sibling}
    \begin{itemize}
      \item irreflexive
      \item symmetric
      \item transitive
    \end{itemize}

    \pause
    \medskip
    \begin{itemize}
      \item how can a relation be symmetric, transitive and nonreflexive?
    \end{itemize}
  \end{block}
\end{frame}

\subsection{Equivalence Relations}

\begin{frame}
  \frametitle{Compatibility Relations}

  \begin{definition}
    \alert{compatibility relation}: $\gamma$
    \begin{itemize}
      \item reflexive
      \item symmetric
    \end{itemize}
  \end{definition}

  \pause
  \begin{itemize}
    \item when drawing, lines instead of arrows
    \item matrix representation as a triangle matrix
  \end{itemize}

  \pause
  \begin{itemize}
    \item $\alpha \alpha^{-1}$ is a compatibility relation
  \end{itemize}
\end{frame}

\begin{frame}
  \frametitle{Compatibility Relation Example}

  \begin{example}
    \begin{columns}
      \column{.45\textwidth}
      \begin{eqnarray*}
        A           & = & \{a_1,a_2,a_3,a_4\}\\
        \mathcal{R} & = & \{(a_1,a_1),(a_2,a_2),\\
                    &   & ~~(a_3,a_3),(a_4,a_4),\\
                    &   & ~~(a_1,a_2),(a_2,a_1),\\
                    &   & ~~(a_2,a_4),(a_4,a_2),\\
                    &   & ~~(a_3,a_4),(a_4,a_3)\}
      \end{eqnarray*}

      \column{.2\textwidth}
      \begin{center}
        \pgfuseimage{compatible1}

        \bigskip
        \pgfuseimage{compatible2}
      \end{center}

      \pause
      \column{.3\textwidth}
      \begin{center}
        \[
          \begin{array}{|cccc|}
            1  &  1  &  0  &  0\\
            1  &  1  &  0  &  1\\
            0  &  0  &  1  &  1\\
            0  &  1  &  1  &  1
          \end{array}
        \]

        \[
          \begin{array}{|ccc|}
            1  &     & \\
            0  &  0  & \\
            0  &  1  &  1
          \end{array}
        \]
      \end{center}
    \end{columns}
  \end{example}
\end{frame}

\begin{frame}
  \frametitle{Compatibility Relation Example}

  \begin{example}[$\alpha \alpha^{-1}$]
    $P$: persons, $L$: languages\\
    $P=\{p_1,p_2,p_3,p_4,p_5,p_6\}$\\
    $L=\{l_1,l_2,l_3,l_4,l_5\}$\\

    \medskip
    $\alpha \subseteq P \times L$

    \pause
    \begin{columns}
      \column{.45\textwidth}
      \[
        M_\alpha =
          \begin{array}{|ccccc|}
            1  &  1  &  0  &  0  &  0\\
            1  &  1  &  0  &  0  &  0\\
            0  &  0  &  1  &  0  &  1\\
            1  &  0  &  1  &  1  &  0\\
            0  &  0  &  0  &  1  &  0\\
            0  &  1  &  1  &  0  &  0
          \end{array}
      \]

      \column{.45\textwidth}
      \[
        M_{\alpha^{-1}} =
          \begin{array}{|cccccc|}
            1  &  1  &  0  &  1  &  0  &  0\\
            1  &  1  &  0  &  0  &  0  &  1\\
            0  &  0  &  1  &  1  &  0  &  1\\
            0  &  0  &  0  &  1  &  1  &  0\\
            0  &  0  &  1  &  0  &  0  &  0
          \end{array}
      \]
    \end{columns}
  \end{example}
\end{frame}

\begin{frame}
  \frametitle{Compatibility Relation Example}

  \begin{example}[$\alpha \alpha^{-1}$]
    $\alpha \alpha^{-1} \subseteq P \times P$

    \begin{columns}
      \column{.45\textwidth}
      \[
        M_{\alpha \alpha^{-1}} =
          \begin{array}{|cccccc|}
            1  &  1  &  0  &  1  &  0  &  1\\
            1  &  1  &  0  &  1  &  0  &  1\\
            0  &  0  &  1  &  1  &  0  &  1\\
            1  &  1  &  1  &  1  &  1  &  1\\
            0  &  0  &  0  &  1  &  1  &  0\\
            1  &  1  &  1  &  1  &  0  &  1
          \end{array}
      \]

      \column{.45\textwidth}
      \begin{center}
        \pgfuseimage{compatible3}
      \end{center}
    \end{columns}
  \end{example}
\end{frame}

\begin{frame}
  \frametitle{Compatibility Block}

  \begin{definition}
    \alert{compatibility block}: $C \subseteq A$\\
      $\forall a,b~[a \in C \wedge b \in C \rightarrow a \gamma b]$
  \end{definition}

  \pause
  \medskip
  \begin{itemize}
    \item \alert{maximal compatibility block}:\\
      not a subset of another compatibility block
    \item an element can be a member of more than one MCB

    \pause
    \medskip
    \item \alert{complete cover}: $C_\gamma$\\
      set of all MCBs
  \end{itemize}
\end{frame}

\begin{frame}
  \frametitle{Compatibility Block Example}

  \begin{example}[$\alpha \alpha^{-1}$]
    \begin{columns}
      \column{.3\textwidth}
      \begin{center}
        \pgfuseimage{compatible3}
      \end{center}

      \pause
      \column{.65\textwidth}
      \begin{itemize}
        \item $C_1=\{p_4,p_6\}$
        \item $C_2=\{p_2,p_4,p_6\}$
        \item $C_3=\{p_1,p_2,p_4,p_6\}$ (MCB)
      \end{itemize}

      \pause
      \medskip
      \begin{eqnarray*}
        C_\gamma & = & \{\{p_1,p_2,p_4,p_6\},\\
                 &   & ~~\{p_3,p_4,p_6\},\\
                 &   & ~~\{p_4,p_5\}\}
      \end{eqnarray*}
    \end{columns}
  \end{example}
\end{frame}

\begin{frame}
  \frametitle{Equivalence Relations}

  \begin{definition}
    \alert{equivalence relation}: $\epsilon$
    \begin{itemize}
      \item reflexive
      \item symmetric
      \item transitive
    \end{itemize}
  \end{definition}

  \pause
  \begin{itemize}
    \item \alert{equivalence classes (partitions)}
    \item every element is a member of exactly one equivalence class

    \medskip
    \item complete cover: $C_\epsilon$
  \end{itemize}
\end{frame}

\begin{frame}
  \frametitle{Equivalence Relation Example}

  \begin{example}
    $\mathcal{R} \subseteq \mathbb{Z} \times \mathbb{Z}$\\
    $\mathcal{R} = \{(a,b)~|~\exists m \in \mathbb{Z}~[a - b = 5m]\}$

    \bigskip
    \begin{itemize}
      \item $\mathcal{R}$ partitions $\mathbb{Z}$ into 5~equivalence classes
    \end{itemize}

% TODO: add drawing
  \end{example}
\end{frame}

\subsection*{References}

\begin{frame}
  \frametitle{References}

  \begin{block}{Required Reading: Grimaldi}
    \begin{itemize}
      \item Chapter 5: Relations and Functions
      \begin{itemize}
        \item 5.1. \alert{Cartesian Products and Relations}
      \end{itemize}
      \item Chapter 7: Relations: The Second Time Around
      \begin{itemize}
        \item 7.1. \alert{Relations Revisited: Properties of Relations}
        \item 7.4. \alert{Equivalence Relations and Partitions}
      \end{itemize}
    \end{itemize}
  \end{block}

  \begin{block}{Supplementary Reading: O'Donnell, Hall, Page}
    \begin{itemize}
      \item Chapter 10: Relations
    \end{itemize}
  \end{block}
\end{frame}

\section{Functions}

\subsection{Introduction}

\begin{frame}
  \frametitle{Functions}

  \begin{definition}
    \alert{function}: $f: X \rightarrow Y$\\
    $\forall x \in X~\forall y_1,y_2 \in Y~
      (x,y_1),(x,y_2) \in f \Rightarrow y_1=y_2$
  \end{definition}

  \medskip
  \begin{itemize}
    \item $X$: \alert{domain}, $Y$: \alert{codomain} (or \emph{range})

    \pause
    \medskip
    \item $y = f(x)$ is the same as $(x,y) \in f$
    \item $y$ is the \emph{image} of $x$ under $f$

    \pause
    \medskip
    \item let $f: X \rightarrow Y$, and $X_1 \subseteq X$\\
      \emph{subset image}: $f(X_1) = \{f(x)~|~x \in X_1\}$
 \end{itemize}
\end{frame}

\begin{frame}
  \frametitle{Subset Image Examples}

  \begin{example}
    $f: \mathbb{R} \rightarrow \mathbb{R}$\\
    $f(x) = x^2$

    \pause
    \bigskip
    $f(\mathbb{Z}) = \{0,1,4,9,16,\dots\}$

    \medskip
    $f(\{-2,1\}) = \{1,4\}$
  \end{example}
\end{frame}

\begin{frame}
  \frametitle{Function Properties}

  \begin{definition}
    $f: X \rightarrow Y$ is \alert{one-to-one} (or \alert{injective}):\\
      $\forall x_1,x_2 \in X~f(x_1)=f(x_2) \Rightarrow x_1=x_2$
  \end{definition}

  \pause
  \begin{definition}
    $f: X \rightarrow Y$ is \alert{onto} (or \alert{surjective}):\\
    $\forall y \in Y~\exists x \in X~f(x)=y$

    \begin{itemize}
      \item $f(X)=Y$
    \end{itemize}
  \end{definition}

  \pause
  \begin{definition}
    $f: X \rightarrow Y$ is \alert{bijective}:\\
    $f$ is one-to-one and onto
  \end{definition}
\end{frame}

\begin{frame}
  \frametitle{One-to-one Function Examples}

  \begin{columns}[t]
    \column{.5\textwidth}
    \begin{example}
      $\begin{array}{l}
        f: \mathbb{R} \rightarrow \mathbb{R}\\
        f(x) = 3x + 7
      \end{array}$

      \pause
      \bigskip
      $\begin{array}{llll}
                    & f(x_1)    & = & f(x_2)\pause\\
        \Rightarrow & 3 x_1 + 7 & = & 3 x_2 + 7\pause\\
        \Rightarrow & 3 x_1     & = & 3 x_2\pause\\
        \Rightarrow & x_1       & = & x_2
      \end{array}$
    \end{example}

    \pause
    \column{.5\textwidth}
    \begin{block}{Counterexample}
      $\begin{array}{l}
        g: \mathbb{Z} \rightarrow \mathbb{Z}\\
        g(x) = x^4 - x
      \end{array}$

      \pause
      \bigskip
      $\begin{array}{lllll}
        g(0) & = & 0^4 - 0 & = & 0\pause\\
        g(1) & = & 1^4 - 1 & = & 0
      \end{array}$
    \end{block}
  \end{columns}
\end{frame}

\begin{frame}
  \frametitle{Onto Function Examples}

  \begin{columns}
    \column{.5\textwidth}
    \begin{example}
      $f: \mathbb{R} \rightarrow \mathbb{R}$\\
      $f(x) = x^3$
    \end{example}

    \pause
    \column{.5\textwidth}
    \begin{block}{Counterexample}
      $f: \mathbb{Z} \rightarrow \mathbb{Z}$\\
      $f(x) = 3x + 1$
    \end{block}
  \end{columns}
\end{frame}

\begin{frame}
  \frametitle{Function Composition}

  \begin{definition}
    let $f: X \rightarrow Y, g: Y \rightarrow Z$

    \medskip
    $g \circ f: X \rightarrow Z$\\
    $(g \circ f)(x) = g(f(x))$
  \end{definition}

  \pause
  \begin{itemize}
    \item function composition is not commutative
    \item function composition is associative:\\
      $f \circ (g \circ h) = (f \circ g) \circ h$
  \end{itemize}
\end{frame}

\begin{frame}
  \frametitle{Function Composition Examples}

  \begin{example}[commutativity]
    $f: \mathbb{R} \rightarrow \mathbb{R}$\\
    $f(x) = x^2$

    \medskip
    $g: \mathbb{R} \rightarrow \mathbb{R}$\\
    $g(x) = x + 5$

    \pause
    \bigskip
    $g \circ f: \mathbb{R} \rightarrow \mathbb{R}$\\
    $(g \circ f)(x) = x^2 + 5$

    \pause
    \medskip
    $f \circ g: \mathbb{R} \rightarrow \mathbb{R}$\\
    $(f \circ g)(x) = (x + 5)^2$
  \end{example}
\end{frame}

\begin{frame}
  \frametitle{Composite Function Theorems}

  \begin{theorem}
    let $f: X \rightarrow Y, g: Y \rightarrow Z$\\
    $f$ is one-to-one $\wedge$ $g$ is one-to-one
      $\Rightarrow$ $g \circ f$ is one-to-one
  \end{theorem}

  \pause
  \begin{proof}
    \[
      \begin{array}{crcl}
                  & (g \circ f) (a_1) & = & (g \circ f) (a_2)\\\pause
      \Rightarrow & g(f(a_1))         & = & g(f(a_2))\\\pause
      \Rightarrow & f(a_1)            & = & f(a_2)\\\pause
      \Rightarrow & a_1               & = & a_2
      \end{array}
    \]
  \end{proof}
\end{frame}

\begin{frame}
  \frametitle{Composite Function Theorems}

  \begin{theorem}
    let $f: X \rightarrow Y, g: Y \rightarrow Z$\\
    $f$ is onto $\wedge$ $g$ is onto $\Rightarrow$ $g \circ f$ is onto
  \end{theorem}

  \pause
  \begin{proof}
    $\forall z \in Z~\exists y \in Y~g(y) = z$\\\pause
    $\forall y \in Y~\exists x \in X~f(x) = y$\\\pause
      $\Rightarrow \forall z \in Z~\exists x \in X~g(f(x)) = z$
  \end{proof}
\end{frame}

\begin{frame}
  \frametitle{Identity Function}

  \begin{definition}
    \alert{identity function}: $1_X$

    \medskip
    $1_X: X \rightarrow X$\\
    $1_X(x) = x$
  \end{definition}
\end{frame}

\begin{frame}
  \frametitle{Inverse Function}

  \begin{definition}
    $f: X \rightarrow Y$ is \alert{invertible}:\\
      $\exists f^{-1}: Y \rightarrow X~[f^{-1} \circ f = 1_X
                                      \wedge f \circ f^{-1} = 1_Y]$

    \begin{itemize}
      \item $f^{-1}$: \alert{inverse} of function f
    \end{itemize}
  \end{definition}
\end{frame}

\begin{frame}
  \frametitle{Inverse Function Examples}

  \begin{example}
    $f: \mathbb{R} \rightarrow \mathbb{R}$\\
    $f(x) = 2x + 5$

    \pause
    \bigskip
    $f^{-1}: \mathbb{R} \rightarrow \mathbb{R}$\\
    $f^{-1}(x) = \frac{x - 5}{2}$

    \pause
    \bigskip
    $(f^{-1} \circ f)(x) = f^{-1}(f(x))$
    \pause
    $ = f^{-1}(2x + 5)$
    \pause
    $ = \frac{(2x + 5) - 5}{2}$
    \pause
    $ = \frac{2x}{2} = x$
    \medskip

    \pause
    $(f \circ f^{-1})(x) = f(f^{-1}(x))$
    \pause
    $ = f(\frac{x - 5}{2})$
    \pause
    $ = 2 \frac{x - 5}{2} + 5$
    \pause
    $ = (x - 5) + 5 = x$
  \end{example}
\end{frame}

\begin{frame}
  \frametitle{Inverse Function}

  \begin{theorem}
    If a function is invertible, its inverse is unique.
  \end{theorem}

  \pause
  \begin{proof}
    let $f: X \rightarrow Y$

    \pause
    \medskip
    let $g,h: Y \rightarrow X$ such that:\\
    $g \circ f = 1_X \wedge f \circ g = 1_Y$\\
    $h \circ f = 1_X \wedge f \circ h = 1_Y$

    \pause
    \medskip
    $h = h \circ 1_Y$
    \pause
    $ = h \circ (f \circ g)$
    \pause
    $ = (h \circ f) \circ g$
    \pause
    $ = 1_X \circ g$
    \pause
    $ = g$
  \end{proof}
\end{frame}

\begin{frame}
  \frametitle{Invertible Function}

  \begin{theorem}
    A function is invertible if and only if it is one-to-one and onto.
  \end{theorem}
\end{frame}

\begin{frame}
  \frametitle{Invertible Function}

  \begin{columns}[t]
    \column{.6\textwidth}
    \begin{proof}[If invertible then one-to-one.]
      $f: A \rightarrow B$
      \begin{eqnarray*}
        &             & f(a_1) = f(a_2)\\\pause
        & \Rightarrow & f^{-1}(f(a_1)) = f^{-1}(f(a_2))\\\pause
        & \Rightarrow & (f^{-1} \circ f) (a_1) = (f^{-1} \circ f) (a_2)\\\pause
        & \Rightarrow & 1_A (a_1) = 1_A (a_2)\\\pause
        & \Rightarrow & a_1 = a_2
      \end{eqnarray*}
    \end{proof}

    \pause
    \column{.35\textwidth}
    \begin{proof}[If invertible then onto.]
      $f: A \rightarrow B$
      \begin{eqnarray*}
        &   & b\\\pause
        & = & 1_B (b)\\\pause
        & = & (f \circ f^{-1}) (b)\\\pause
        & = & f(f^{-1} (b))
      \end{eqnarray*}
    \end{proof}
  \end{columns}
\end{frame}

\begin{frame}
  \frametitle{Invertible Function}

  \begin{proof}[If bijective then invertible.]
    $f: A \rightarrow B$
    \begin{itemize}
      \item $f$ is onto $\Rightarrow \forall b \in B~\exists a \in A~f(a)=b$
      \item let $g: B \rightarrow A$ be defined by $a=g(b)$

      \pause
      \medskip
      \item is it possible that $g(b) = a_1 \neq a_2 = g(b)$ ?

      \pause
      \item this would mean: $f(a_1) = b = f(a_2)$
      \item but $f$ is one-to-one
    \end{itemize}
  \end{proof}
\end{frame}

\subsection{Pigeonhole Principle}

\begin{frame}
  \frametitle{Pigeonhole Principle}

  \begin{definition}
    \alert{Pigeonhole Principle} (Dirichlet drawers):\\
    If $m$ pigeons go into $n$ holes and $m>n$,\\
    then at least one hole contains more than one pigeon.
  \end{definition}

  \pause
  \begin{itemize}
    \item let $f: X \rightarrow Y$\\
      if $|X|>|Y|$ then $f$ cannot be one-to-one

    \item $\exists x_1,x_2 \in X~[x_1 \neq x_2 \wedge f(x_1)=f(x_2)]$
  \end{itemize}
\end{frame}

\begin{frame}
  \frametitle{Pigeonhole Principle Examples}

  \begin{example}
    \begin{itemize}
      \item Among 367 people, at least two have the same birthday.

      \pause
      \item In an exam where the grades integers between 0 and 100,\\
        how many students have to take the exam to make sure that\\
        at least two students will have the same grade?
    \end{itemize}
  \end{example}
\end{frame}

\begin{frame}
  \frametitle{Generalized Pigeonhole Principle}

  \begin{definition}
    \alert{Generalized Pigeonhole Principle}:\\
    If $m$ objects are distributed to $n$ drawers,\\
    then at least one of the drawers contains $\lceil m / n \rceil$ objects.
  \end{definition}

  \pause
  \begin{example}
    Among 100 people, at least 9 ($\lceil 100 / 12 \rceil$) were born\\
    in the same month.
  \end{example}
\end{frame}

\begin{frame}
  \frametitle{Pigeonhole Principle Example}

  \begin{theorem}
    In any subset of cardinality 6 of the set S = \{1,2,3,\dots,9\},\\
    there are two elements which total 10.
  \end{theorem}
\end{frame}

\begin{frame}
  \frametitle{Pigeonhole Principle Example}

  \begin{theorem}
    Let $S$ be a set of positive integers smaller than or equal to 14,\\
    with cardinality 6. The sums of the elements\\
    in all nonempty subsets of $S$ cannot be all different.
  \end{theorem}

  \pause
  \begin{columns}[t]
    \column{.5\textwidth}
    \begin{block}{Proof Trial}
      $A \subseteq S$\\
      $s_A:$ sum of the elements of $A$

      \pause
      \begin{itemize}
        \item holes:\\
          $1 \leq s_A \leq 9 + \dots + 14 = 69$
        \item pigeons: $2^6 - 1 = 63$
      \end{itemize}
    \end{block}

    \pause
    \column{.5\textwidth}
    \begin{proof}
      look at the subsets for which $|A| \leq 5$

      \pause
      \begin{itemize}
        \item holes:\\
          $1 \leq s_A \leq 10 + \dots + 14 = 60$
        \item pigeons: $2^6 - 2 = 62$
      \end{itemize}
    \end{proof}
  \end{columns}
\end{frame}

\begin{frame}
  \frametitle{Pigeonhole Principle Example}

  \begin{theorem}
    There is at least one pair of elements among 101 elements\\
    chosen from set $S = \{1,2,3,\dots,200\}$,\\
    so that one of the elements of the pair divides the other.
  \end{theorem}

  \pause
  \begin{block}{Proof Method}
    \begin{itemize}
      \item we first show that
        $\forall n~\exists ! p~
          [n = 2^r p \wedge r \in \mathbb{N}
            \wedge \exists t \in \mathbb Z~[p = 2t + 1]]$\\

      \item then, by using this theorem we prove the main theorem
    \end{itemize}
  \end{block}
\end{frame}

\begin{frame}
  \frametitle{Pigeonhole Principle Example}

  \begin{theorem}
    $\forall n~\exists ! p~
      [n = 2^r p \wedge r \in \mathbb{N}
        \wedge \exists t \in \mathbb Z~[p = 2t + 1]]$\\
  \end{theorem}

  \pause
  \begin{columns}[t]
    \column{.55\textwidth}
    \begin{proof}[Proof of existence]
      $n = 1$: $r = 0, p = 1$\\

      \pause
      $n \leq k$: assume $n = 2^r p$

      \pause
      $n = k + 1$:
      \vspace{-9pt}
      \[
        \begin{array}{ll}
          n=2:            & r = 1, p = 1\pause\\
          n~prime~(n>2):  & r = 0, p = n\pause\\
          \neg (n~prime): & n = n_1 n_2\pause\\
                          & n = 2^{r_1} p_1 \cdot 2^{r_2} p_2\pause\\
                          & n = 2^{r_1+r_2} \cdot p_1 p_2
        \end{array}
      \]
    \end{proof}

    \pause
    \column{.45\textwidth}
    \begin{proof}[Proof of uniqueness]
      if not unique:

      \pause
      \[
        \begin{array}{lllll}
          n & =           & 2^{r_1} p_1     & = & 2^{r_2} p_2\pause\\
            & \Rightarrow & 2^{r_1-r_2} p_1 & = & p_2\pause\\
            & \Rightarrow & 2 | p_2
        \end{array}
      \]
    \end{proof}
  \end{columns}
\end{frame}

\begin{frame}
  \frametitle{Pigeonhole Principle Example}

  \begin{theorem}
    There is at least one pair of elements among 101 elements\\
    chosen from set $S = \{1,2,3,\dots,200\}$\\
    so that one of the elements of the pair divides the other.
  \end{theorem}

  \pause
  \begin{proof}
    \begin{itemize}
      \item $T=\{t~|~t \in S, \exists i \in \mathbb{Z}~[t=2i+1]\}$, $|T|=100$

      \pause
      \item $f: S \rightarrow T$, $r \in \mathbb{N}$ olsun\\
        $s = 2^r t \rightarrow f(s) = t$
      \begin{itemize}
        \item if 101 elements are chosen from $S$, at least two of them\\
          will have the same image in $T$:
          $f(s_1)=f(s_2) \Rightarrow 2^{m_1} t = 2^{m_2} t$

        \pause
        \[
          \frac {s_1} {s_2} = \frac {2^{m_1} t} {2^{m_2} t} = 2^{m_1 - m_2}
        \]
      \end{itemize}
    \end{itemize}
  \end{proof}
\end{frame}

\subsection{Recursion}

\begin{frame}
  \frametitle{Recursive Functions}

  \begin{definition}
    \alert{recursive function}: a function defined in terms of itself

    \medskip
    $f(n) = h(f(m))$
  \end{definition}

  \begin{itemize}
    \item \emph{inductively defined function}: a recursive function\\
      where the size is reduced at every step

    \medskip
    $f(n) = \left\{
      \begin{array}{ll}
        k         & n = 0\\
        h(f(n-1)) & n > 0
      \end{array}
    \right.$
  \end{itemize}
\end{frame}

\begin{frame}
  \frametitle{Recursion Examples}

  \begin{example}
    $f91(n) = \left\{
      \begin{array}{ll}
        n - 10         & n > 100\\
        f91(f91(n+11)) & n \leq 100
      \end{array}
    \right.$
  \end{example}

  \pause
  \medskip
  \begin{example}[factorial]
    $f(n) = \left\{
      \begin{array}{ll}
        1              & n = 0\\
        n \cdot f(n-1) & n > 0
      \end{array}
    \right.$
  \end{example}
\end{frame}

\begin{frame}
  \frametitle{Euclid Algorithm}

  \begin{example}[greatest common divisor]
    $gcd(a,b) = \left\{
      \begin{array}{ll}
        b              & b | a\\
        gcd(b,a~mod~b) & b \nmid a
      \end{array}
    \right.$

    \pause
    \medskip
    \begin{eqnarray*}
      gcd(333,84) & = & gcd(84, 333~mod~84)\\
                  & = & gcd(84, 81)\\
                  & = & gcd(81, 84~mod~81)\\
                  & = & gcd(81, 3)\\
                  & = & 3
    \end{eqnarray*}
  \end{example}
\end{frame}

\begin{frame}
  \frametitle{Fibonacci Series}

  \begin{block}{Fibonacci series}
    $F_n = fib(n) = \left\{
      \begin{array}{ll}
        1                   & n = 1\\
        1                   & n = 2\\
        fib(n-1) + fib(n-2) & n > 2
      \end{array}
    \right.$
  \end{block}

  \bigskip
  $\begin{array}{ccccccccc}
     F_1 & F_2 & F_3 & F_4 & F_5 & F_6 & F_7 & F_8 & \dots\\
     1   & 1   & 2   & 3   & 5   & 8   & 13  & 21  & \dots
  \end{array}$
\end{frame}

\begin{frame}
  \frametitle{Fibonacci Series}

  \begin{theorem}
    $\sum_{i=1}^{n} {F_i}^2 = F_n \cdot F_{n+1}$
  \end{theorem}

  \pause
  \begin{proof}
    $\begin{array}{llcl}
      n=2:   & \sum_{i=1}^{2} {F_i}^2   & = & {F_1}^2+{F_2}^2=1+1=1 \cdot 2=F_2 \cdot F_3
      \pause
      \medskip\\
      n=k:   & \sum_{i=1}^{k} {F_i}^2   & = & F_k \cdot F_{k+1}
      \pause
      \medskip\\
      n=k+1: & \sum_{i=1}^{k+1} {F_i}^2 & = & \sum_{i=1}^{k} {F_i}^2 + {F_{k+1}}^2
      \pause
      \smallskip\\
             &                          & = & F_k \cdot F_{k+1} + {F_{k+1}}^2
      \pause
      \smallskip\\
             &                          & = & F_{k+1} \cdot (F_k + F_{k+1})
      \pause
      \smallskip\\
             &                          & = & F_{k+1} \cdot F_{k+2}
    \end{array}$
  \end{proof}
\end{frame}

\begin{frame}
  \frametitle{Ackermann Function}

  \begin{block}{Ackermann function}
    $ack(x,y) = \left\{
      \begin{array}{ll}
        y+1                 & x = 0\\
        ack(x-1, 1)         & y = 0\\
        ack(x-1,ack(x,y-1)) & x > 0 \wedge y > 0
      \end{array}
    \right.$
  \end{block}
\end{frame}

\subsection*{References}

\begin{frame}
  \frametitle{References}

  \begin{block}{Required Reading: Grimaldi}
    \begin{itemize}
      \item Chapter 5: Relations and Functions
      \begin{itemize}
        \item 5.2. \alert{Functions: Plain and One-to-One}
        \item 5.3. \alert{Onto Functions: Stirling Numbers of the Second Kind}
        \item 5.5. \alert{The Pigeonhole Principle}
        \item 5.6. \alert{Function Composition and Inverse Functions}
      \end{itemize}
    \end{itemize}
  \end{block}

  \begin{block}{Supplementary Reading: O'Donnell, Hall, Page}
    \begin{itemize}
      \item Chapter 11: Functions
    \end{itemize}
  \end{block}
\end{frame}

\end{document}
