% Copyright (c) 2013
%       Ayşegül Gençata Yayımlı <gencata@itu.edu.tr>
%       H. Turgut Uyar <uyar@itu.edu.tr>
%
% These notes are licensed using the
% "Creative Commons Attribution-NonCommercial-ShareAlike License".
% You are free to copy, distribute and transmit the work, and to adapt the work
% as long as you attribute the authors, do not use it for commercial purposes,
% and any derivative work is under the same or a similar license.
%
% Read the full legal code at:
% http://creativecommons.org/licenses/by-nc-sa/3.0/

\documentclass[dvipsnames]{beamer}

\usepackage{ae}
\usepackage[T1]{fontenc}
\usepackage[utf8]{inputenc}
\usepackage{textcomp}
\setbeamertemplate{navigation symbols}{}

\mode<presentation>
{
  \usetheme{Rochester}
  \setbeamercovered{transparent}
}

\title{Discrete Mathematics}
\subtitle{Principles of Counting}

\author{Ayşegül Gençata Yayımlı \and H. Turgut Uyar}
\date{2013}

\AtBeginSubsection[]
{
  \begin{frame}<beamer>
    \frametitle{Topics}
    \tableofcontents[currentsection,currentsubsection]
  \end{frame}
}

%\beamerdefaultoverlayspecification{<+->}

\pgfdeclareimage[width=2cm]{license}{../../license}

\begin{document}

\begin{frame}
  \titlepage
\end{frame}

\begin{frame}
  \frametitle{Licence}

  \pgfuseimage{license}\hfill
  \copyright 2013 A. Yayımlı, T. Uyar

  \vfill
  \begin{tiny}
    You are free:
    \begin{itemize}
      \item to Share — to copy, distribute and transmit the work
      \item to Remix — to adapt the work
    \end{itemize}

    Under the following conditions:
    \begin{itemize}
      \item Attribution — You must attribute the work in the manner specified by
        the author or licensor (but not in any way that suggests that they
        endorse you or your use of the work).

      \item Noncommercial — You may not use this work for commercial purposes.

      \item Share Alike — If you alter, transform, or build upon this work, you
        may distribute the resulting work only under the same or similar license
        to this one.
    \end{itemize}
  \end{tiny}

  \vfill
  Legal code (the full license):\\
  \url{http://creativecommons.org/licenses/by-nc-sa/3.0/}
\end{frame}

\begin{frame}
  \frametitle{Topics}
  \tableofcontents
\end{frame}

\section{Basic Principles}

\subsection{Introduction}

\begin{frame}
  \frametitle{Basic Principles}

  \begin{itemize}
    \item counting = enumeration

    \pause
    \medskip
    \item two basic principles of counting:
    \begin{itemize}
      \item the rule of sum
      \item the rule of product
    \end{itemize}

    \item decompose more complex problems into smaller ones
    \item piece together partial solutions to arrive at the final answer
  \end{itemize}
\end{frame}

\subsection{The Rule of Sum}

\begin{frame}
  \frametitle{The Rule of Sum}

  \begin{block}{The Rule of Sum}
    \begin{itemize}
      \item a first task can be performed in $m$ (distinct) ways
      \item a second task can be performed in $n$ (distinct) ways
      \item the two tasks cannot be performed simultaneously

      \medskip
      \item performing either task can be accomplished\\
    in any one of $m+n$ ways
    \end{itemize}
  \end{block}
\end{frame}

\begin{frame}
  \frametitle{The Rule of Sum}

  \begin{example}
    \begin{itemize}
      \item a college library has 40 textbooks on sociology
      \item and 50 textbooks on anthropology

      \medskip
      \item a student can select from $40 + 50 = 90$ textbooks\\
        in order to learn more about one or the other subject
    \end{itemize}
  \end{example}
\end{frame}

\begin{frame}
  \frametitle{The Rule of Sum}

  \begin{example}
    \begin{itemize}
      \item a computer science instructor has two colleagues
      \item one colleague has 3 textbooks on ``Analysis of Algorithms''
      \item the other colleague has 5 such textbooks

      \medskip
      \item $n$: maximum number of different books\\
        that the instructor can borrow
      \item since both colleagues may own copies of the same book:\\
        $5 \leq n \leq 8$
   \end{itemize}
  \end{example}
\end{frame}

\subsection{The Rule of Product}

\begin{frame}
  \frametitle{The Rule of Product}

  \begin{block}{The Rule of Product}
    \begin{itemize}
      \item a procedure can be broken down into first and second stages
      \item there are $m$ possible outcomes for the first stage
      \item for each of these outcomes,\\
        there are $n$ possible outcomes for the second stage

      \medskip
      \item the total procedure can be carried out in $m \cdot n$ ways
    \end{itemize}
  \end{block}
\end{frame}

\begin{frame}
  \frametitle{The Rule of Product}

  \begin{example}
    \begin{itemize}
      \item the drama club is holding tryouts for a play
      \item there are 6 men and 8 women auditioning for the leading roles

      \medskip
      \item the director can cast the leading couple in
        $6 \cdot 8 = 48$ ways
    \end{itemize}
  \end{example}
\end{frame}

\begin{frame}
  \frametitle{The Rule of Product}

  \begin{example}
    \begin{itemize}
      \item license plates consist of 2 letters, followed by 4 digits
      \item how many plates?

      \pause
      \medskip
      \item if no letter or digit can be repeated:\\
        $26 \cdot 25 \cdot 10 \cdot 9 \cdot 8 \cdot 7 = 3,276,000$

      \pause
      \medskip
      \item if repetitions are allowed for both letters and digits:\\
        $26 \cdot 26 \cdot 10 \cdot 10 \cdot 10 \cdot 10 = 6,760,000$

      \pause
      \medskip
      \item if repetitions are allowed for both letters and digits,\\
        how many plates consist of only vowels and even digits?\\
        $5 \cdot 5 \cdot 5 \cdot 5 \cdot 5 \cdot 5 = 15,625$
   \end{itemize}
  \end{example}
\end{frame}

\section{Permutations and Combinations}

\subsection{Permutations}

\begin{frame}
  \frametitle{Permutation}

  \begin{definition}
    \alert{permutation}: any linear arrangement of $n$ distinct objects
  \end{definition}

  \begin{itemize}
    \item ``arrangement'' means that the order is important
  \end{itemize}
\end{frame}

\begin{frame}
  \frametitle{Permutations}

  \begin{example}
    \begin{itemize}
      \item a class has 10 students: $A, B, C, \ldots, I, J$
      \item 5 students are to be chosen and seated in a row for a picture:
      \begin{itemize}
        \item $BCEFI, CEFIB, ABCFG, \ldots$
      \end{itemize}
      \item how many such linear arrangements are possible?

      \pause
      \medskip
      \item the filling of a position is a stage of the counting procedure:\\
        $10 \cdot 9 \cdot 8 \cdot 7 \cdot 6 = 30,240$
    \end{itemize}
  \end{example}
\end{frame}

\begin{frame}
  \frametitle{Permutations}

  \begin{example}
    \begin{eqnarray*}
    10 \cdot 9 \cdot 8 \cdot 7 \cdot 6 & = &
      10 \cdot 9 \cdot 8 \cdot 7 \cdot 6 \cdot
      \frac{5 \cdot 4 \cdot 3 \cdot 2 \cdot 1}
      {5 \cdot 4 \cdot 3 \cdot 2 \cdot 1}\\\pause
    & = & \frac{10!}{5!}
    \end{eqnarray*}
  \end{example}
\end{frame}

\begin{frame}
  \frametitle{Number of Permutations}

  \begin{block}{number of permutations}
    \begin{itemize}
      \item $n$ distinct objects
      \item number of permutations of size $r$ (where $1 \leq r \leq n$):\\
      \begin{eqnarray*}
        P(n,r) & = & n \cdot (n-1) \cdot (n-2) \cdots (n-r+1)\\
               & = & \frac{n!}{(n-r)!}
      \end{eqnarray*}

      \pause
      \medskip
      \item if repetitions are allowed: $n^r$
    \end{itemize}
  \end{block}
\end{frame}

\begin{frame}
  \frametitle{Permutations}

  \begin{example}
    \begin{itemize}
      \item what is the number of permutations of the letters in ``BALL''?
      \item the two L's are indistinguishable
    \end{itemize}

    \pause
    \begin{columns}[t]
      \column{.4\textwidth}
      \begin{tabular}{c c c c}
 A & B & L & L\\
 A & L & B & L\\
 A & L & L & B\\
 B & A & L & L\\
 B & L & A & L\\
 B & L & L & A
      \end{tabular}

      \column{.4\textwidth}
      \begin{tabular}{c c c c}
 L & A & B & L\\
 L & A & L & B\\
 L & B & A & L\\
 L & B & L & A\\
 L & L & A & B\\
 L & L & B & A
      \end{tabular}
    \end{columns}

    \pause
    \begin{itemize}
      \item number of permutations: $\frac{4!}{2} = 12$
    \end{itemize}
  \end{example}
\end{frame}

\begin{frame}
  \frametitle{Permutations}

  \begin{example}
    \begin{itemize}
      \item arrangements of all letters in ``DATABASES``

      \pause
      \medskip
      \item for each arrangement in which the A's are not distinguished,\\
        there are $3! = 6$ arrangements with the A's distinguished:\\
        $DA_{1}TA_{2}BA_{3}SES,DA_{1}TA_{3}BA_{2}SES, DA_{2}TA_{1}BA_{3}SES$,\\
        $DA_{2}TA_{3}BA_{1}SES, DA_{3}TA_{1}BA_{2}SES, DA_{3}TA_{2}BA_{1}SES$

      \pause
      \item for each of these,\\
        there are 2 arrangements where the S's are distinguished:\\
        $DA_{1}TA_{2}BA_{3}S_{1}ES_{2},DA_{1}TA_{2}BA_{3}S_{2}ES_{1}$

      \pause
      \medskip
      \item number of arrangements: $\frac{9!}{2! \cdot 3!} = 30,240$
   \end{itemize}
  \end{example}
\end{frame}

\begin{frame}
  \frametitle{Generalization}
  
  \begin{block}{number of arrangements}
    \begin{itemize}
      \item $n$ objects
      \item $n_1$ indistinguishable objects of $type_1$,\\
        $n_2$ indistinguishable objects of $type_2$,\\
        \ldots $n_r$ indistinguishable objects of $type_r$
      \item $n_1 + n_2 + ... + n_r = n$

      \medskip
      \item number of linear arrangements of these $n$ objects:
      \begin{equation*}
        \frac{n!}{n_1! \cdot n_2! \cdots n_r!}
      \end{equation*}
    \end{itemize}
  \end{block}
\end{frame}

\begin{frame}
  \frametitle{Arrangements}

  \begin{example}
    \begin{itemize}
      \item xy-plane from $(2,1)$ to $(7,4)$
      \item staircase path: each step going one unit to the right ($R$)\\
        or one unit upwards ($U$)
      \item for example: $RURRURRU$, $URRRUURR$
      \item how many such paths?

      \pause
      \medskip
      \item each path consists of 5 R's and 3 U's
      \item number of paths: $\frac{8!}{5! \cdot 3!} = 56$
    \end{itemize}
  \end{example}
\end{frame}

\begin{frame}
  \frametitle{Circular Arrangements}

  \begin{example}
    \begin{itemize}
      \item six people are seated around a round table: $A,B,C,D,E,F$
      \item how many different circular arrangements?
      \begin{itemize}
        \item arrangements are considered to be the same\\
          when one can be obtained from the other by rotation
        \item $ABEFCD, DABEFC, CDABEF, FCDABE, EFCDAB, BEFCDA$
      \end{itemize}

      \pause
      \medskip
      \item each circular arrangement (CA) corresponds to\\
        6 linear arrangements (LA)
      \item $6 \cdot \#CA = \#LA = 6!$
      \item number of circular arrangements: $\frac{6!}{6} = 120$
    \end{itemize}
  \end{example}
\end{frame}

\subsection{Combinations}

\begin{frame}
  \frametitle{Combinations}

  \begin{example}
    \begin{itemize}
      \item deck of playing cards with 52 cards
      \item 4 suits: clubs, diamonds, hearts, spades
      \item 13 ranks in each suit: Ace, 2, 3, \ldots, 10, Jack, Queen, King
      \item draw 3 cards in succession, without replacement
      \item how many possible draws?
    \end{itemize}

    \pause
    \begin{equation*}
      52 \cdot 51 \cdot 50 = \frac{52!}{49!} = P(52,3) = 132,600
    \end{equation*}  
  \end{example}
\end{frame}

\begin{frame}
  \frametitle{Combinations}

  \begin{example}
    \begin{itemize}
      \item assume one such draw is:\\
        $AH$ (ace of hearts), $9C$ (9 of clubs), $KD$ (king of diamonds)
      \item if we select all 3 cards at once, the order doesn't matter
      \item then, the 6 permutations of $(AH,9C,KD)$\\
        all correspond to just one selection
    \end{itemize}
    \begin{equation*}
      \frac{52!}{3! \cdot 49!} = 22,100
    \end{equation*}  
  \end{example}
\end{frame}

\begin{frame}
  \frametitle{Number of Combinations}

  \begin{block}{Combinations}
    \begin{itemize}
      \item $n$ distinct objects
      \item each selection, or \alert{combination} of $r$ of these objects,\\
        with no reference to order,\\
        corresponds to $r!$ permutations of size $r$

      \pause
      \medskip
      \item number of combinations of size $r$ (where $0 \leq r \leq n$):
      \begin{equation*}
        C(n,r) = {n \choose r} = \frac{P(n,r)}{r!} = \frac{n!}{r! \cdot (n-r)!}
      \end{equation*}
    \end{itemize}
  \end{block}
\end{frame}

\begin{frame}
  \frametitle{Number of Combinations}

  \begin{itemize}
    \item number of combinations:
    \begin{equation*}
      C(n,r) = \frac{n!}{r! \cdot (n-r)!}
    \end{equation*}

    \item note that:
    \begin{eqnarray*}
      C(n,0) & = 1 = & C(n,n)\\
      C(n,1) & = n = & C(n,n-1)
    \end{eqnarray*}
  \end{itemize}
\end{frame}

\begin{frame}
  \frametitle{Combinations}
  
  \begin{example}
    \begin{itemize}
      \item Lynn and Patti decide to buy a powerball ticket
      \item to win, one must match five numbers selected from 1 to 49
      \item and then must also match the powerball, 1 to 42
      \item how many possible tickets?

      \pause
      \medskip
      \item Lynn selects the five numbers from 1 to 49: $C(49,5)$ ways
      \item Patti selects the powerball from 1 to 42: $C(42,1)$ ways
      \item number of possible tickets:
        ${49 \choose 5}{42 \choose 1} = 80,089,128$
    \end{itemize}
  \end{example}
\end{frame}

\begin{frame}
  \frametitle{Combinations}

  \begin{example}
    \begin{itemize}
      \item for a volleyball team, the gym teacher must select nine girls\\
        from the junior and senior classes
      \item 28 junior and 25 senior candidates
      \item how many different ways?

      \pause
      \medskip
      \item if no restrictions: ${53 \choose 9} = 4,431,613,550$
      \pause
      \item if two juniors and one senior are the best spikers\\
        and must be on the team: ${50 \choose 6} = 15,890,700$
      \pause
      \item if there has to be four juniors and five seniors:
        ${28 \choose 4}{25 \choose 5} = 1,087,836,750$
    \end{itemize}
  \end{example}
\end{frame}

\begin{frame}
  \frametitle{The Binomial Theorem}

  \begin{theorem}
    If $x$ and $y$ are variables and $n$ is a positive integer, then
    \begin{eqnarray*}
      (x+y)^n & = & {n\choose 0} x^0 y^n
                    + {n\choose 1} x^1 y^{n-1}
                    + {n\choose 2} x^2 y^{n-2} + \cdots\\
              &   & + {n\choose n-1} x^{n-1} y^1
                    + {n\choose n} x^n y^0 \\
              & = & \sum^n_{k=0}{{n\choose k} x^k y^{n-k}}
    \end{eqnarray*}
  \end{theorem}

  \pause
  \begin{itemize}
    \item ${n\choose k}$ is a \alert{binomial coefficient}
  \end{itemize}
\end{frame}

\begin{frame}
  \frametitle{The Binomial Theorem}

  \begin{example}
    \begin{itemize}
      \item in the expansion of $(x+y)^7$, the coefficient of $x^5 y^2$:\\
        ${7\choose 5} = {7 \choose 2} = 21$
    \end{itemize}
  \end{example}

  \pause
  \begin{example}
    \begin{itemize}
      \item in the expansion of $(2a-3b)^7$, the coefficient of $a^5 b^2$:
      \item $x=2a, y=-3b$
      \begin{equation*}
      {7\choose 5} x^5 y^2 = {7\choose 5} (2a)^5 (-3b)^2
                           = {7\choose 5} (2)^5 (-3)^2 a^5 b^2 = 6048 a^5 b^2
      \end{equation*}
    \end{itemize}
  \end{example}  
\end{frame}

\begin{frame}
  \frametitle{The Multinomial Theorem}

  \begin{theorem}
    For positive integers $n, t$, the coefficient of
    $x_{1}^{n_1} x_{2}^{n_2} x_{3}^{n_3} \cdots x_{t}^{n_t}$\\
    in the expansion of $(x_1 + x_2 + x_3 + \cdots + x_t)^n$ is
    \begin{equation*}
      \frac{n!}{n_1! \cdot n_2! \cdot n_3! \cdots n_t!}
    \end{equation*}
    where each $n_i$ is an integer with $0 \leq n_i \leq n$,
    for all $1 \leq i \leq t$, and\\
    $n_1 + n_2 + n_3 + ... + n_t = n$
  \end{theorem}
\end{frame}

\begin{frame}
  \frametitle{The Multinomial Theorem}

  \begin{example}
    \begin{itemize}
      \item in the expansion of $(x+y+z)^7$, the coefficient of $x^2 y^2 z^3$:
      \begin{equation*}
        {7 \choose 2,2,3} = \frac{7!}{2! \cdot 2! \cdot 3!} = 210
      \end{equation*}
    \end{itemize}
    \begin{itemize}
      \item the coefficient of $x y z^5$:
      \begin{equation*}
        {7 \choose 1,1,5} = \frac{7!}{1! \cdot 1! \cdot 5!} = 42
      \end{equation*}
    \end{itemize}
  \end{example}
\end{frame}

\subsection{Combinations with Repetition}

\begin{frame}
  \frametitle{Combinations with Repetition}

  \begin{example}
    \begin{itemize}
      \item 7 students visit a restaurant
      \item each of them orders one of the following:\\
        cheeseburger (c), hot dog (h), taco (t), fish sandwich (f)
      \item how many different purchases are possible?
    \end{itemize}
  \end{example}
\end{frame}

\begin{frame}
  \frametitle{Combinations with Repetition}

  \begin{example}
    \begin{table}
      \texttt{
      \begin{tabular}{|l|l|}\hline
c c h h t t f & x x | x x | x x | x\\
c c c c h t f & x x x x | x | x | x\\
c c c c c c f & x x x x x x | | | x\\
h t t f f f f & | x | x x | x x x x\\
t t t t t t t & | | x x x x x x x |\\
f f f f f f f & | | | x x x x x x x\\\hline
      \end{tabular}
      }
    \end{table}

    \begin{itemize}
      \item enumerate all arrangements of 10 symbols\\
        consisting of seven x's and three |'s
      \item number of different purchases:
        $\frac{10!}{7! \cdot 3!} = {10 \choose 7} = 120$
    \end{itemize}
  \end{example}
\end{frame}

\begin{frame}
  \frametitle{Number of Combinations with Repetition}

  \begin{block}{Number of Combinations with Repetition}
    \begin{itemize}
      \item select, with repetition, $r$ of $n$ distinct objects
      \item considering all arrangements of $r$ x's and $n-1$ |'s
      \begin{equation*}
        \frac{(n+r-1)!}{r! \cdot (n-1)!} = {n+r-1 \choose r}
      \end{equation*}
    \end{itemize}
  \end{block}
\end{frame}

\begin{frame}
  \frametitle{Number of Combinations with Repetition}

  \begin{example}
    \begin{itemize}
      \item distribute 7 bananas and 6 oranges among 4 children
      \item each child receives at least one banana
      \item how many ways?

      \pause
      \medskip
      \item step 1: give each child a banana
      \item step 2: distribute 3 bananas to 4 children
      \begin{table}
        \texttt{
        \begin{tabular}{|l|l|}\hline
1 1 1 0 & b | b | b |\\
1 0 2 0 & b | | b b |\\
0 0 1 2 & | | b | b b\\
0 0 0 3 & | | | b b b\\\hline
        \end{tabular}
        }
      \end{table}

      \pause
      \item $C(6,3) = 20$ ways
    \end{itemize}
  \end{example}
\end{frame}

\begin{frame}
  \frametitle{Number of Combinations with Repetition}

  \begin{example}
    \begin{itemize}
      \item step 3: distribute 6 oranges to 4 children
      \begin{table}
        \texttt{
        \begin{tabular}{|l|l|}\hline
          1 2 2 1 & o | o o | o o | o\\
          1 2 0 3 & o | o o | | o o o\\
          0 3 3 0 & | o o o | o o o |\\
          0 0 0 6 & | | | o o o o o o\\\hline
        \end{tabular}
        }
      \end{table}

      \pause
      \item $C(9,6) = 84$ ways

      \pause
      \medskip
      \item step 4: by the rule of product: $20 \cdot 84 = 1,680$ ways
    \end{itemize}
  \end{example}
\end{frame}

\section*{References}

\begin{frame}
  \frametitle{References}
  \begin{block}{Required Reading: Grimaldi}
    \begin{itemize}
      \item Chapter 1: Fundamental Principles of Counting
      \begin{itemize}
        \item 1.1. \alert{The Rules of Sum and Product}
        \item 1.2. \alert{Permutations}
        \item 1.3. \alert{Combinations}
        \item 1.4. \alert{Combinations with Repetition}
      \end{itemize}
    \end{itemize}
  \end{block}
\end{frame}

\end{document}
