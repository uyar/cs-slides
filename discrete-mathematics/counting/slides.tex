% Copyright (c) 2013-2015
%       Ayşegül Gençata Yayımlı <gencata@itu.edu.tr>
%       H. Turgut Uyar <uyar@itu.edu.tr>
%
% This work is licensed under a "Creative Commons
% Attribution-NonCommercial-ShareAlike 4.0 International License".
% For more information, please visit:
% https://creativecommons.org/licenses/by-nc-sa/4.0/

\documentclass[dvipsnames]{beamer}

\usepackage{ae}
\usepackage[T1]{fontenc}
\usepackage[utf8]{inputenc}
\setbeamertemplate{navigation symbols}{}
\setbeamersize{text margin left=2em, text margin right=2em}

\mode<presentation>
{
  \usetheme{Rochester}
  \setbeamercovered{transparent}
}

\title{Discrete Mathematics}
\subtitle{Counting}

\author{Ayşegül Gençata Yayımlı \and H. Turgut Uyar}
\date{2013-2015}

\AtBeginSubsection[]
{
  \begin{frame}<beamer>
    \frametitle{Topics}
    \tableofcontents[currentsection,currentsubsection]
  \end{frame}
}

\pgfdeclareimage[height=1cm]{license}{../license}

\pgfdeclareimage[width=5cm]{staircase}{staircase}

\begin{document}

\begin{frame}
  \titlepage
\end{frame}

\begin{frame}
  \frametitle{License}

  \pgfuseimage{license}\hfill
  \copyright~2013-2015 A. Yayımlı, T. Uyar

  \vfill
  \begin{footnotesize}
    You are free to:
    \begin{itemize}
      \itemsep0em
      \item Share -- copy and redistribute the material in any medium or format
      \item Adapt -- remix, transform, and build upon the material
    \end{itemize}

    Under the following terms:
    \begin{itemize}
      \itemsep0em
      \item Attribution -- You must give appropriate credit, provide a link to
        the license, and indicate if changes were made.

      \item NonCommercial -- You may not use the material for commercial
        purposes.

      \item ShareAlike -- If you remix, transform, or build upon the material,
        you must distribute your contributions under the same license as the
        original.
    \end{itemize}
  \end{footnotesize}

  \begin{small}
    For more information:\\
    \url{https://creativecommons.org/licenses/by-nc-sa/4.0/}

    \smallskip
    Read the full license:\\
    \url{https://creativecommons.org/licenses/by-nc-sa/4.0/legalcode}
  \end{small}
\end{frame}

\begin{frame}
  \frametitle{Topics}
  \tableofcontents
\end{frame}

\section{Basic Principles}

\subsection{Introduction}

\begin{frame}
  \frametitle{Basic Principles}

  \begin{itemize}
    \item decompose large problem into smaller ones
    \item piece together partial solutions to arrive at the whole solution

    \pause
    \bigskip
    \item rule of sum
    \item rule of product
  \end{itemize}
\end{frame}

\subsection{Rule of Sum}

\begin{frame}
  \frametitle{Rule of Sum}

  \begin{itemize}
    \item $TaskA$ can be performed in $m$ (distinct) ways
    \item $TaskB$ can be performed in $n$ (distinct) ways
    \item $TaskA$ and $TaskB$ cannot be performed simultaneously

    \pause
    \bigskip
    \item performing either $TaskA$ or $TaskB$ can be accomplished\\
      in any one of $m+n$ ways
  \end{itemize}
\end{frame}

\begin{frame}
  \frametitle{Rule of Sum Example}

  \begin{itemize}
    \item a college library has 40 textbooks on sociology,
    \item and 50 textbooks on anthropology

    \medskip
    \item to learn about sociology or anthropology\\
      a student can choose from $40 + 50 = 90$ textbooks
  \end{itemize}
\end{frame}

\begin{frame}
  \frametitle{Rule of Sum Example}

  \begin{itemize}
    \item a computer science instructor has two colleagues
    \item one colleague has 3 textbooks on ``Introduction to Programming''
    \item the other colleague has 5 textbooks on the same subject

    \medskip
    \item $n$: maximum number of different books that can be borrowed
    \item both colleagues may own copies of the same book:\\
      $5 \leq n \leq 8$
  \end{itemize}
\end{frame}

\subsection{Rule of Product}

\begin{frame}
  \frametitle{Rule of Product}

  \begin{itemize}
    \item $ProcedureA$ can be broken down into $Stage1$ and $Stage2$
    \item $m$ possible outcomes for $Stage1$
    \item for each of these, $n$ possible outcomes for $Stage2$

    \medskip
    \item $ProcedureA$ can be carried out in $m \cdot n$ ways
  \end{itemize}
\end{frame}

\begin{frame}
  \frametitle{Rule of Product Example}

  \begin{itemize}
    \item drama club is holding tryouts for a play
    \item 6 men and 8 women auditioning for the leading roles

    \medskip
    \item director can cast the leading couple in $6 \cdot 8 = 48$ ways
  \end{itemize}
\end{frame}

\begin{frame}
  \frametitle{Rule of Product Examples}

  \begin{itemize}
    \item license plates consist of 2 letters, followed by 4 digits
    \item how many possible plates?

    \pause
    \medskip
    \item if no letter or digit can be repeated:\\
      $26 \cdot 25 \cdot 10 \cdot 9 \cdot 8 \cdot 7 = 3,276,000$

    \pause
    \medskip
    \item if repetitions are allowed for both letters and digits:\\
      $26 \cdot 26 \cdot 10 \cdot 10 \cdot 10 \cdot 10 = 6,760,000$

    \pause
    \medskip
    \item if repetitions are allowed for both letters and digits,\\
      how many plates consist of only vowels and even digits?\\
      $5 \cdot 5 \cdot 5 \cdot 5 \cdot 5 \cdot 5 = 15,625$
  \end{itemize}
\end{frame}

\section{Permutations}

\subsection{Introduction}

\begin{frame}
  \frametitle{Permutation}

  \begin{itemize}
    \item \alert{permutation}: a linear arrangement of distinct objects
    \item order is important
  \end{itemize}
\end{frame}

\begin{frame}
  \frametitle{Permutation Example}

  \begin{itemize}
    \item a class has 10 students: $A, B, C, \ldots, I, J$
    \item 4 students are to be seated in a row for a picture:\\
      \smallskip
      $BCEF, CEFI, ABCF, \ldots$
    \item how many such linear arrangements?

    \pause
    \medskip
    \item filling of a position: a stage of the counting procedure\\
      $10 \cdot 9 \cdot 8 \cdot 7 = 5,040$
  \end{itemize}
\end{frame}

\begin{frame}
  \frametitle{Permutation Example}

  \begin{eqnarray*}
    10 \cdot 9 \cdot 8 \cdot 7 & = &
      10 \cdot 9 \cdot 8 \cdot 7 \cdot
      \frac{6 \cdot 5 \cdot 4 \cdot 3 \cdot 2 \cdot 1}
      {6 \cdot 5 \cdot 4 \cdot 3 \cdot 2 \cdot 1}\\\pause
    & = & \frac{10!}{6!}
  \end{eqnarray*}
\end{frame}

\subsection{Number of Permutations}

\begin{frame}
  \frametitle{Number of Permutations}

  \begin{itemize}
    \item $n$ distinct objects
    \item number of permutations of size $r$ (where $1 \leq r \leq n$):\\
    \begin{eqnarray*}
      P(n,r) & = & n \cdot (n-1) \cdot (n-2) \cdots (n-r+1)\\
             & = & \frac{n!}{(n-r)!}
    \end{eqnarray*}

    \pause
    \medskip
    \item if repetitions are allowed: $n^r$
  \end{itemize}
\end{frame}

\begin{frame}
  \frametitle{Number of Permutations Example}

  \begin{itemize}
    \item number of permutations of the letters in ``BALL''
    \item two L's are indistinguishable
  \end{itemize}

  \pause
  \begin{columns}[t]
    \column{.4\textwidth}
    \begin{tabular}{c c c c}
A & B & L & L\\
A & L & B & L\\
A & L & L & B\\
B & A & L & L\\
B & L & A & L\\
B & L & L & A
    \end{tabular}

    \column{.4\textwidth}
    \begin{tabular}{c c c c}
L & A & B & L\\
L & A & L & B\\
L & B & A & L\\
L & B & L & A\\
L & L & A & B\\
L & L & B & A
    \end{tabular}
  \end{columns}

  \pause
  \begin{itemize}
    \item number of permutations: $\frac{4!}{2} = 12$
  \end{itemize}
\end{frame}

\begin{frame}
  \frametitle{Number of Permutations Example}

  \begin{itemize}
    \item arrangements of all letters in ``DATABASES``

    \pause
    \medskip
    \item for each arrangement where the A's \textbf{are not} distinguished,\\
      there are $3! = 6$ arrangements where the A's \textbf{are} distinguished:\\
      \smallskip
      $DA_{1}TA_{2}BA_{3}SES,DA_{1}TA_{3}BA_{2}SES, DA_{2}TA_{1}BA_{3}SES$,\\
      $DA_{2}TA_{3}BA_{1}SES, DA_{3}TA_{1}BA_{2}SES, DA_{3}TA_{2}BA_{1}SES$

    \pause
    \item for each of these, 2 arrangements where the S's are distinguished:\\
      \smallskip
      $DA_{1}TA_{2}BA_{3}S_{1}ES_{2},DA_{1}TA_{2}BA_{3}S_{2}ES_{1}$

    \pause
    \medskip
    \item number of arrangements: $\frac{9!}{2! \cdot 3!} = 30,240$
  \end{itemize}
\end{frame}

\begin{frame}
  \frametitle{Number of Arrangements}

  \begin{itemize}
    \item $n$ objects
    \item $n_1$ indistinguishable objects of $type_1$\\
      $n_2$ indistinguishable objects of $type_2$\\
      \ldots\\
      $n_r$ indistinguishable objects of $type_r$
    \item $n_1 + n_2 + ... + n_r = n$

    \pause
    \medskip
    \item number of linear arrangements:
    \begin{equation*}
      \frac{n!}{n_1! \cdot n_2! \cdots n_r!}
    \end{equation*}
  \end{itemize}
\end{frame}

\begin{frame}
  \frametitle{Number of Arrangements Example}

  \begin{columns}[t]
    \column{.4\textwidth}
    \begin{center}
      \pgfuseimage{staircase}
    \end{center}

    \column{.6\textwidth}
    \begin{itemize}
      \item go from $(2,1)$ to $(7,4)$
      \item each step one unit to the right ($R$)\\
        or one unit upwards ($U$)
      \item $RURRURRU$, $URRRUURR$
      \item how many such paths?
    \end{itemize}
  \end{columns}

  \pause
  \medskip
  \begin{itemize}
    \item each path consists of 5 R's and 3 U's
    \item number of paths: $\frac{8!}{5! \cdot 3!} = 56$
  \end{itemize}
\end{frame}

\subsection{Circular Arrangements}

% TODO: define circular arrangement

\begin{frame}
  \frametitle{Number of Circular Arrangements Example}

  \begin{itemize}
    \item six people are seated around a round table: $A,B,C,D,E,F$
    \item arrangements are considered to be the same\\
      when one can be obtained from the other by rotation:\\
      \smallskip
      $ABEFCD, DABEFC, CDABEF, FCDABE, EFCDAB, BEFCDA$
    \item how many different circular arrangements?

    \pause
    \medskip
    \item each circular arrangement corresponds to 6 linear arrangements
    \item number of circular arrangements: $\frac{6!}{6} = 120$
  \end{itemize}
\end{frame}

\section{Combinations}

\subsection{Introduction}

\begin{frame}
  \frametitle{Combination}

  \begin{itemize}
    \item \alert{combination}: choosing from distinct objects
    \item order isn't important
  \end{itemize}
\end{frame}

\begin{frame}
  \frametitle{Combination Example}

  \begin{itemize}
    \item a deck of 52 playing cards
    \item 4 suits: clubs, diamonds, hearts, spades
    \item 13 ranks in each suit: Ace, 2, 3, \ldots, 10, Jack, Queen, King
    \item draw 3 cards in succession, without replacement
    \item how many possible draws?
  \end{itemize}

  \pause
  \begin{equation*}
    52 \cdot 51 \cdot 50 = \frac{52!}{49!} = P(52,3) = 132,600
  \end{equation*}
\end{frame}

\begin{frame}
  \frametitle{Combination Example}

  \begin{itemize}
    \item assume one such draw is:\\
      $AH$ (ace of hearts), $9C$ (9 of clubs), $KD$ (king of diamonds)
    \item if we select all 3 cards at once, order doesn't matter

    \pause
    \item 6 permutations of $(AH,9C,KD)$ correspond to just one selection
  \end{itemize}
  \begin{equation*}
    \frac{52!}{3! \cdot 49!} = 22,100
  \end{equation*}
\end{frame}

\begin{frame}
  \frametitle{Number of Combinations}

  \begin{itemize}
    \item $n$ distinct objects
    \item each combination of $r$ of these objects\\
      corresponds to $r!$ permutations of size $r$

    \pause
    \medskip
    \item number of combinations of size $r$ (where $0 \leq r \leq n$):
    \begin{equation*}
      C(n,r) = {n \choose r} = \frac{P(n,r)}{r!} = \frac{n!}{r! \cdot (n-r)!}
    \end{equation*}
  \end{itemize}
\end{frame}

\begin{frame}
  \frametitle{Number of Combinations}

  \begin{itemize}
    \item number of combinations:
    \begin{equation*}
      C(n,r) = \frac{n!}{r! \cdot (n-r)!}
    \end{equation*}

    \item note that:
    \begin{eqnarray*}
      C(n,0) & = 1 = & C(n,n)\\
      C(n,1) & = n = & C(n,n-1)
    \end{eqnarray*}
  \end{itemize}
\end{frame}

\begin{frame}
  \frametitle{Number of Combinations Example}

  \begin{itemize}
    \item Lynn and Patti decide to buy a powerball ticket
    \item to win, one must match five numbers selected from 1 to 49
    \item and then must also match the powerball, 1 to 42
    \item how many possible tickets?

    \pause
    \medskip
    \item Lynn selects five numbers from 1 to 49: $C(49,5)$ ways
    \item Patti selects the powerball from 1 to 42: $C(42,1)$ ways
    \item number of possible tickets:
      ${49 \choose 5}{42 \choose 1} = 80,089,128$
  \end{itemize}
\end{frame}

\begin{frame}
  \frametitle{Number of Combinations Examples}

  \begin{itemize}
    \item for a volleyball team, the gym teacher must select nine girls\\
      from the junior and senior classes
    \item 28 junior and 25 senior candidates
    \item how many different ways?

    \pause
    \medskip
    \item if no restrictions: ${53 \choose 9} = 4,431,613,550$
    \pause
    \item if two juniors and one senior are the best spikers\\
      and must be on the team: ${50 \choose 6} = 15,890,700$
    \pause
    \item if there has to be four juniors and five seniors:
      ${28 \choose 4}{25 \choose 5} = 1,087,836,750$
  \end{itemize}
\end{frame}

\begin{frame}
  \frametitle{Binomial Theorem}

  \begin{theorem}
    if $x$ and $y$ are variables and $n$ is a positive integer, then:
    \begin{eqnarray*}
      (x+y)^n & = & {n\choose 0} x^0 y^n
                    + {n\choose 1} x^1 y^{n-1}
                    + {n\choose 2} x^2 y^{n-2} + \cdots\\
              &   & + {n\choose n-1} x^{n-1} y^1
                    + {n\choose n} x^n y^0 \\
              & = & \sum^n_{k=0}{{n\choose k} x^k y^{n-k}}
    \end{eqnarray*}
  \end{theorem}

  \begin{itemize}
    \item ${n\choose k}$: \alert{binomial coefficient}
  \end{itemize}
\end{frame}

\begin{frame}
  \frametitle{Binomial Theorem Examples}

  \begin{exampleblock}{}
    \begin{itemize}
      \item in the expansion of $(x+y)^7$, the coefficient of $x^5 y^2$:\\
        ${7\choose 5} = {7 \choose 2} = 21$
    \end{itemize}
  \end{exampleblock}

  \pause
  \begin{exampleblock}{}
    \begin{itemize}
      \item in the expansion of $(2a-3b)^7$, the coefficient of $a^5 b^2$:
      \item $x=2a, y=-3b$
      \begin{equation*}
      {7\choose 5} x^5 y^2 = {7\choose 5} (2a)^5 (-3b)^2
                           = {7\choose 5} (2)^5 (-3)^2 a^5 b^2 = 6048 a^5 b^2
      \end{equation*}
    \end{itemize}
  \end{exampleblock}
\end{frame}

\begin{frame}
  \frametitle{Multinomial Theorem}

  \begin{theorem}
    For positive integers $n, t$, the coefficient of
    $x_{1}^{n_1} x_{2}^{n_2} x_{3}^{n_3} \cdots x_{t}^{n_t}$\\
    in the expansion of $(x_1 + x_2 + x_3 + \cdots + x_t)^n$ is
    \begin{equation*}
      \frac{n!}{n_1! \cdot n_2! \cdot n_3! \cdots n_t!}
    \end{equation*}
    where each $n_i$ is an integer with $0 \leq n_i \leq n$,
    for all $1 \leq i \leq t$, and\\
    $n_1 + n_2 + n_3 + ... + n_t = n$.
  \end{theorem}
\end{frame}

\begin{frame}
  \frametitle{Multinomial Theorem Examples}

  \begin{itemize}
    \item in the expansion of $(x+y+z)^7$, the coefficient of $x^2 y^2 z^3$:
    \begin{equation*}
      {7 \choose 2,2,3} = \frac{7!}{2! \cdot 2! \cdot 3!} = 210
    \end{equation*}
  \end{itemize}
  \begin{itemize}
    \item the coefficient of $x y z^5$:
    \begin{equation*}
      {7 \choose 1,1,5} = \frac{7!}{1! \cdot 1! \cdot 5!} = 42
    \end{equation*}
  \end{itemize}
\end{frame}

\subsection{With Repetition}

\begin{frame}
  \frametitle{Combinations with Repetition Example}

  \begin{itemize}
    \item 7 students visit a restaurant
    \item each of them orders one of the following:\\
      cheeseburger (c), hot dog (h), taco (t), fish sandwich (f)
    \item how many different purchases are possible?
  \end{itemize}
\end{frame}

\begin{frame}
  \frametitle{Combinations with Repetition Example}

  \begin{table}
    \texttt{
    \begin{tabular}{|l|l|}\hline
c c h h t t f & x x | x x | x x | x\\
c c c c h t f & x x x x | x | x | x\\
c c c c c c f & x x x x x x | | | x\\
h t t f f f f & | x | x x | x x x x\\
t t t t t t t & | | x x x x x x x |\\
f f f f f f f & | | | x x x x x x x\\\hline
    \end{tabular}
    }
  \end{table}

  \begin{itemize}
    \item enumerate all arrangements of 10 symbols\\
      consisting of seven x's and three |'s
    \item number of different purchases:
      $\frac{10!}{7! \cdot 3!} = {10 \choose 7} = 120$
  \end{itemize}
\end{frame}

\begin{frame}
  \frametitle{Number of Combinations with Repetition}

  \begin{itemize}
    \item select, with repetition, $r$ of $n$ distinct objects
    \item considering all arrangements of $r$ x's and $n-1$ |'s
    \begin{equation*}
      \frac{(n+r-1)!}{r! \cdot (n-1)!} = {n+r-1 \choose r}
    \end{equation*}
  \end{itemize}
\end{frame}

\begin{frame}
  \frametitle{Number of Combinations with Repetition Example}

  \begin{itemize}
    \item distribute 7 bananas and 6 oranges among 4 children
    \item each child receives at least one banana
    \item how many ways?

    \pause
    \medskip
    \item step 1: give each child a banana
    \item step 2: distribute 3 bananas to 4 children
    \begin{table}
      \texttt{
      \begin{tabular}{|l|l|}\hline
1 1 1 0 & b | b | b |\\
1 0 2 0 & b | | b b |\\
0 0 1 2 & | | b | b b\\
0 0 0 3 & | | | b b b\\\hline
      \end{tabular}
      }
    \end{table}

    \pause
    \item $C(6,3) = 20$ ways
  \end{itemize}
\end{frame}

\begin{frame}
  \frametitle{Number of Combinations with Repetition Example}

  \begin{itemize}
    \item step 3: distribute 6 oranges to 4 children
    \begin{table}
      \texttt{
      \begin{tabular}{|l|l|}\hline
        1 2 2 1 & o | o o | o o | o\\
        1 2 0 3 & o | o o | | o o o\\
        0 3 3 0 & | o o o | o o o |\\
        0 0 0 6 & | | | o o o o o o\\\hline
      \end{tabular}
      }
    \end{table}

    \pause
    \item $C(9,6) = 84$ ways

    \pause
    \medskip
    \item step 4: by the rule of product: $20 \cdot 84 = 1,680$ ways
  \end{itemize}
\end{frame}

\section*{References}

\begin{frame}
  \frametitle{References}
  \begin{block}{Required Reading: Grimaldi}
    \begin{itemize}
      \item Chapter 1: Fundamental Principles of Counting
      \begin{itemize}
        \item 1.1. \alert{The Rules of Sum and Product}
        \item 1.2. \alert{Permutations}
        \item 1.3. \alert{Combinations}
        \item 1.4. \alert{Combinations with Repetition}
      \end{itemize}
    \end{itemize}
  \end{block}
\end{frame}

\end{document}
