% Copyright (c) 2001-2014
%       H. Turgut Uyar <uyar@itu.edu.tr>
%       Ayşegül Gençata Yayımlı <gencata@itu.edu.tr>
%       Emre Harmancı <harmanci@itu.edu.tr>
%
% This work is licensed under a "Creative Commons
% Attribution-NonCommercial-ShareAlike 4.0 International License".
% For more information, please visit:
% https://creativecommons.org/licenses/by-nc-sa/4.0/

\documentclass[dvipsnames]{beamer}

\usepackage{ae}
\usepackage[T1]{fontenc}
\usepackage[utf8]{inputenc}
\setbeamertemplate{navigation symbols}{}
\setbeamersize{text margin left=2em, text margin right=2em}

\mode<presentation>
{
  \usetheme{Rochester}
  \setbeamercovered{transparent}
}

\title{Discrete Mathematics}
\subtitle{Trees}

\author{H. Turgut Uyar \and Ayşegül Gençata Yayımlı \and Emre Harmancı}
\date{2001-2014}

\AtBeginSubsection[]
{
  \begin{frame}<beamer>
    \frametitle{Topics}
    \tableofcontents[currentsection,currentsubsection]
  \end{frame}
}

%\beamerdefaultoverlayspecification{<+->}

\pgfdeclareimage[width=2cm]{license}{../license}

\pgfdeclareimage{tree}{tree}
\pgfdeclareimage{cycle}{cycle}
\pgfdeclareimage{nodecount}{nodecount}
\pgfdeclareimage[height=5cm]{rooted}{rooted}
\pgfdeclareimage[height=4cm]{book}{book}
\pgfdeclareimage{dictionary}{dictionary}
\pgfdeclareimage{expr1a}{expr1a}
\pgfdeclareimage{expr1b}{expr1b}
\pgfdeclareimage{expr2a}{expr2a}
\pgfdeclareimage{expr2b}{expr2b}
\pgfdeclareimage{expr3}{expr3}
\pgfdeclareimage{expr}{expr}
\pgfdeclareimage{scale1}{scale1}
\pgfdeclareimage{scale2}{scale2}
\pgfdeclareimage[height=4cm]{spanning}{spanning}
\pgfdeclareimage[height=4cm]{kruskal1}{kruskal1}
\pgfdeclareimage[height=4cm]{kruskal2}{kruskal2}
\pgfdeclareimage[height=4cm]{kruskal3}{kruskal3}
\pgfdeclareimage[height=4cm]{kruskal4}{kruskal4}
\pgfdeclareimage[height=4cm]{kruskal5}{kruskal5}
\pgfdeclareimage[height=4cm]{kruskal}{kruskal}
\pgfdeclareimage[height=4cm]{prim1}{prim1}
\pgfdeclareimage[height=4cm]{prim2}{prim2}
\pgfdeclareimage[height=4cm]{prim3}{prim3}
\pgfdeclareimage[height=4cm]{prim4}{prim4}
\pgfdeclareimage[height=4cm]{prim5}{prim5}
\pgfdeclareimage[height=4cm]{prim}{prim}

\begin{document}

\begin{frame}
  \titlepage
\end{frame}

\begin{frame}
  \frametitle{License}

  \pgfuseimage{license}\hfill
  \copyright~2001-2014 T. Uyar, A. Yayımlı, E. Harmancı

  \vfill
  \begin{footnotesize}
    You are free to:
    \begin{itemize}
      \itemsep0em
      \item Share -- copy and redistribute the material in any medium or format
      \item Adapt -- remix, transform, and build upon the material
    \end{itemize}

    Under the following terms:
    \begin{itemize}
      \itemsep0em
      \item Attribution -- You must give appropriate credit, provide a link to
        the license, and indicate if changes were made.

      \item NonCommercial -- You may not use the material for commercial
        purposes.

      \item ShareAlike -- If you remix, transform, or build upon the material,
        you must distribute your contributions under the same license as the
        original.
    \end{itemize}
  \end{footnotesize}

  \begin{small}
    For more information:\\
    \url{https://creativecommons.org/licenses/by-nc-sa/4.0/}

    \smallskip
    Read the full license:\\
    \url{https://creativecommons.org/licenses/by-nc-sa/4.0/legalcode}
  \end{small}
\end{frame}

\begin{frame}
  \frametitle{Topics}
  \tableofcontents
\end{frame}

\section{Trees}

\subsection{Introduction}

\begin{frame}
  \frametitle{Tree}

  \begin{definition}
    \alert{tree}: a connected graph that contains no cycle
  \end{definition}

  \begin{itemize}
    \item \emph{forest}: a graph where the connected components are trees
  \end{itemize}
\end{frame}

\begin{frame}
  \frametitle{Tree Examples}

  \begin{example}
    \begin{center}
      \pgfuseimage{tree}
    \end{center}
  \end{example}
\end{frame}

\begin{frame}
  \frametitle{Tree Theorems}

  \begin{theorem}
    In a tree, there is one and only one path\\
    between any two distinct nodes.
  \end{theorem}

  \begin{itemize}
    \item there is at least one path because the tree is connected
    \item if there were more than one path, they would form a cycle
    \medskip
    \begin{center}
      \pgfuseimage{cycle}
    \end{center}
  \end{itemize}
\end{frame}

\begin{frame}
  \frametitle{Tree Theorems}

  \begin{theorem}
    let $T=(V,E)$ be a tree:

    \[|E| = |V| - 1\]
  \end{theorem}

  \begin{itemize}
    \item proof method: induction on the number of edges
  \end{itemize}
\end{frame}

\begin{frame}
  \frametitle{Tree Theorems}

  \begin{block}{Proof: Base step.}
    \begin{itemize}
      \item $|E|=0 \Rightarrow |V|=1$
      \item $|E|=1 \Rightarrow |V|=2$
      \item $|E|=2 \Rightarrow |V|=3$

      \pause
      \medskip
      \item assume that $|E| = |V| - 1$ for $|E| \leq k$
    \end{itemize}
  \end{block}
\end{frame}

\begin{frame}
  \frametitle{Tree Theorems}

  \begin{proof}[Proof: Induction step]
    \begin{itemize}
      \item $|E|=k+1$
    \end{itemize}

    \begin{columns}[t]
      \column{.4\textwidth}
      \begin{center}
        \pgfuseimage{nodecount}
      \end{center}

      \pause
      \column{.55\textwidth}
      \begin{itemize}
        \item let's remove the edge $(y,z)$:\\
          $T_1=(V_1,E_1)$, $T_2=(V_2,E_2)$
      \end{itemize}
      \pause
      \begin{eqnarray*}
        |V| & = & |V_1|+|V_2|\\\pause
            & = & |E_1|+1+|E_2|+1\\\pause
            & = & (|E_1|+|E_2|+1)+1\\\pause
            & = & |E|+1
      \end{eqnarray*}
    \end{columns}
  \end{proof}
\end{frame}

\begin{frame}
  \frametitle{Tree Theorems}

  \begin{theorem}
    In a tree, there are at least two nodes with degree 1.
  \end{theorem}

  \pause
  \begin{proof}
    \begin{itemize}
      \item $2 |E| = \sum_{v \in V} d_v$

      \pause
      \item assume that there is only 1 node with degree 1:\\
        \pause
        $\Rightarrow 2 |E| \geq 2 (|V| - 1) + 1$\\
        \pause
        $\Rightarrow 2 |E| \geq 2 |V| - 1$\\
        \pause
        $\Rightarrow |E| \geq |V| - \frac{1}{2}$
        \pause
        $> |V| - 1$
    \end{itemize}
  \end{proof}
\end{frame}

\begin{frame}
  \frametitle{Tree Theorems}

  \begin{theorem}
    \begin{center}
      $T$ is a tree ($T$ is connected and contains no cycle).\\
      $\Leftrightarrow$\\
      There is one and only one path\\
        between any two distinct nodes in $T$.\\
      $\Leftrightarrow$\\
      $T$ is connected, but if any edge is removed\\
        it will no longer be connected.\\
      $\Leftrightarrow$\\
      $T$ contains no cycle, but if an edge is added\\
        between any pair of nodes one and only one cycle will be formed.
    \end{center}
  \end{theorem}
\end{frame}

\begin{frame}
  \frametitle{Tree Theorems}

  \begin{theorem}
    \begin{center}
      $T$ is a tree ($T$ is connected and contains no cycle).\\
      $\Leftrightarrow$\\
      $T$ is connected and $|E| = |V| - 1$.\\
      $\Leftrightarrow$\\
      $T$ contains no cycle and $|E| = |V| - 1$.
    \end{center}
  \end{theorem}
\end{frame}

\subsection{Rooted Trees}

\begin{frame}
  \frametitle{Rooted Tree}

  \begin{itemize}
    \item a hierarchy is defined between nodes
    \item hierarchy creates a natural direction on edges:\\
      in and out degrees

    \pause
    \medskip
    \item in-degree 0: \alert{root} (only 1 such node)
    \item out-degree 0: \alert{leaf}
    \item not a leaf: \alert{internal} node
  \end{itemize}
\end{frame}

\begin{frame}
  \frametitle{Node Level}

  \begin{definition}
    \alert{level} of a node: distance from the root
  \end{definition}

  \begin{itemize}
    \item \emph{parent}: adjacent node in the next upper level (only 1 such node)
    \item \emph{child}: adjacent node in the next lower level
    \item \emph{sibling}: node which has the same parent
  \end{itemize}
\end{frame}

\begin{frame}
  \frametitle{Rooted Tree Example}

  \begin{example}
    \begin{columns}
      \column{.4\textwidth}
      \begin{center}
        \pgfuseimage{rooted}
      \end{center}

      \column{.58\textwidth}
      \begin{itemize}
        \item root: $r$
        \item leaves: $x ~ y ~ z ~ u ~ v$
        \item internal nodes: $r ~ p ~ n ~ t ~ s ~ q ~ w$
        \item parent of $y$: $w$\\
          children of $w$: $y$ and $z$\\
	\item $y$ and $z$ are siblings
      \end{itemize}
    \end{columns}
  \end{example}
\end{frame}

\begin{frame}
  \frametitle{Rooted Tree Example}

  \begin{example}
    \begin{columns}
      \column{.65\textwidth}
      \begin{center}
        \pgfuseimage{book}
      \end{center}

      \column{.33\textwidth}
      Book
      \begin{itemize}
        \item C1
        \begin{itemize}
          \item S1.1
          \item S1.2
        \end{itemize}
        \item C2
        \item C3
        \begin{itemize}
          \item S3.1
          \item S3.2
          \begin{itemize}
            \item S3.2.1
            \item S3.2.2
          \end{itemize}
          \item S3.3
        \end{itemize}
      \end{itemize}
    \end{columns}
  \end{example}
\end{frame}

\begin{frame}
  \frametitle{Ordered Rooted Tree}

  \begin{itemize}
    \item sibling nodes are ordered from left to right

    \medskip
    \item \alert{universal address system}
    \begin{itemize}
      \item assign the address $0$ to the root
      \item assign the positive integers $1,2,3,\ldots$ to the nodes at level 1,\\
        from left to right
      \item let $v$ be an internal node with address $a$,\\
        assign the addresses $a.1,a.2,a.3,\ldots$ to the children of $v$\\
        from left to right
    \end{itemize}
  \end{itemize}
\end{frame}

\begin{frame}
  \frametitle{Lexicographic Order}

  \begin{definition}
    Let $b$ and $c$ be two addresses.

    $b$ comes before $c$ if one of the following holds:
    \begin{enumerate}
      \item $b=a_1 a_2 \ldots a_m x_1 \ldots$\\
        $c=a_1 a_2 \ldots a_m x_2 \ldots$\\
        $x_1$ comes before $x_2$
      \pause
      \item $b=a_1 a_2 \ldots a_m$\\
        $c=a_1 a_2 \ldots a_m a_{m+1} \ldots$
    \end{enumerate}
  \end{definition}
\end{frame}

\begin{frame}
  \frametitle{Lexicographic Order Example}

  \begin{example}
    \begin{columns}
      \column{.57\textwidth}
      \begin{center}
        \pgfuseimage{dictionary}
      \end{center}

      \column{.4\textwidth}
      \begin{itemize}
        \item 0 - 1 - 1.1 - 1.2\\
          - 1.2.1 - 1.2.2 - 1.2.3\\
          - 1.2.3.1 - 1.2.3.2\\
          - 1.3 - 1.4 - 2\\
          - 2.1 - 2.2 - 2.2.1\\
          - 3 - 3.1 - 3.2
      \end{itemize}
    \end{columns}
  \end{example}
\end{frame}

\subsection{Binary Trees}

\begin{frame}
  \frametitle{Binary Trees}

  \begin{definition}
    $T=(V,E)$ is a \alert{binary tree}:\\
    $\forall v \in V~{d_v}^o \in \{0,1,2\}$
  \end{definition}

  \begin{itemize}
    \item $T=(V,E)$ is a \emph{complete} binary tree:\\
      $\forall v \in V~{d_v}^o \in \{0,2\}$
  \end{itemize}

\end{frame}

\begin{frame}
  \frametitle{Expression Tree}

  \begin{itemize}
    \item a binary operation can be represented as a binary tree
    \begin{itemize}
      \item operator as the root, operands as the children
    \end{itemize}

    \medskip
    \item every mathematical expression can be represented as a tree
    \begin{itemize}
      \item operators at internal nodes, variables and values at the leaves
    \end{itemize}
  \end{itemize}
\end{frame}

\begin{frame}
  \frametitle{Expression Tree Examples}

  \begin{columns}[t]
    \column{.5\textwidth}
    \begin{example}[$7-a$]
      \begin{center}
        \pgfuseimage{expr1a}
      \end{center}
    \end{example}

    \column{.5\textwidth}
    \begin{example}[$a+b$]
      \begin{center}
        \pgfuseimage{expr1b}
      \end{center}
    \end{example}
  \end{columns}
\end{frame}

\begin{frame}
  \frametitle{Expression Tree Examples}

  \begin{columns}[t]
    \column{.5\textwidth}
    \begin{example}[$(7-a)/5$]
      \begin{center}
        \pgfuseimage{expr2a}
      \end{center}
    \end{example}

    \column{.5\textwidth}
    \begin{example}[$(a+b) \uparrow 3$]
      \begin{center}
        \pgfuseimage{expr2b}
      \end{center}
    \end{example}
  \end{columns}
\end{frame}

\begin{frame}
  \frametitle{Expression Tree Examples}

  \begin{example}[$((7-a)/5)*((a+b) \uparrow 3)$]
    \begin{center}
      \pgfuseimage{expr3}
    \end{center}
  \end{example}
\end{frame}

\begin{frame}
  \frametitle{Expression Tree Examples}

  \begin{example}[$t+(u*v)/(w+x-y \uparrow z)$]
    \begin{center}
      \pgfuseimage{expr}
    \end{center}
  \end{example}
\end{frame}

\begin{frame}
  \frametitle{Expression Tree Traversals}

  \begin{enumerate}
    \item \alert{inorder} traversal:\\
      traverse left subtree, visit root, traverse right subtree

    \medskip
    \item \alert{preorder} traversal:\\
      visit root, traverse left subtree, traverse right subtree

    \medskip
    \item \alert{postorder} traversal:\\
      traverse left subtree, traverse right subtree, visit root
    \begin{itemize}
      \item \emph{reverse Polish notation}
    \end{itemize}
  \end{enumerate}
\end{frame}

\begin{frame}
  \frametitle{Inorder Traversal Example}

  \begin{example}
    \begin{columns}
      \column{.4\textwidth}
      \begin{center}
        \pgfuseimage{expr}
      \end{center}

      \column{.6\textwidth}
      $t ~ + ~ u ~ * ~ v ~ / ~ w ~ + ~ x ~ - ~ y ~ \uparrow ~ z$
    \end{columns}
  \end{example}
\end{frame}

\begin{frame}
  \frametitle{Preorder Traversal Example}

  \begin{example}
    \begin{columns}
      \column{.4\textwidth}
      \begin{center}
        \pgfuseimage{expr}
      \end{center}

      \column{.6\textwidth}
      $+ ~ t ~ / ~ * ~ u ~ v ~ + ~ w ~ - ~ x ~ \uparrow ~ y ~ z$
    \end{columns}
  \end{example}
\end{frame}

\begin{frame}
  \frametitle{Postorder Traversal Example}

  \begin{example}
    \begin{columns}
      \column{.4\textwidth}
      \begin{center}
        \pgfuseimage{expr}
      \end{center}

      \column{.6\textwidth}
      $t ~ u ~ v ~ * ~ w ~ x ~ y ~ z ~ \uparrow ~ - ~ + ~ / ~ +$
    \end{columns}
  \end{example}
\end{frame}

\begin{frame}
  \frametitle{Expression Tree Evaluation}

  \begin{itemize}
    \item inorder traversal requires parantheses for precedence
    \item preorder and postorder traversals do not require parantheses
  \end{itemize}
\end{frame}

\begin{frame}
  \frametitle{Postorder Evaluation Example}

  \begin{example}[$t ~ u ~ v ~ * ~ w ~ x ~ y ~ z ~ \uparrow ~ - ~ + ~ / ~ +$]
    $4 ~ 2 ~ 3 ~ * ~ 1 ~ 9 ~ 2 ~ 3 ~ \uparrow ~ - ~ + ~ / ~ +$

    \pause
    \medskip
    \[
      \begin{array}{ccccccc}
  4 & \pause 2 & \pause 3 & \pause * &          &          &                \\
  4 &        6 & \pause 1 & \pause 9 & \pause 2 & \pause 3 & \pause \uparrow\\
  4 &        6 &        1 &        9 &        8 & \pause - &                \\
  4 &        6 &        1 &        1 & \pause + &          &                \\
  4 &        6 &        2 & \pause / &          &          &                \\
  4 &        3 & \pause + &          &          &          &                \\
  7 &          &          &          &          &          &                \\
      \end{array}
    \]
  \end{example}
\end{frame}

\begin{frame}
  \frametitle{Regular Trees}

  \begin{definition}
    $T=(V,E)$ is an \alert{m-ary tree}: $\forall v \in V~{d_v}^o \leq m$
  \end{definition}

  \begin{itemize}
    \item $T=(V,E)$ is a complete m-ary tree:\\
      $\forall v \in V~{d_v}^o \in \{0,m\}$
  \end{itemize}

\end{frame}

\begin{frame}
  \frametitle{Regular Tree Theorem}

  \begin{theorem}
    Let $T=(V,E)$ be a complete $m$-ary tree.

    \begin{itemize}
      \item $n$: number of nodes
      \item $l$: number of leaves
      \item $i$: number of internal nodes
    \end{itemize}

    Then:
    \pause
    \begin{itemize}
      \item $n = m \cdot i + 1$

      \pause
      \item  $l = n - i = \pause m \cdot i + 1 - i
        \pause = (m - 1) \cdot i + 1$

      \pause
      \[
        i = \frac{l - 1}{m - 1}
      \]
    \end{itemize}
  \end{theorem}
\end{frame}

\begin{frame}
  \frametitle{Regular Tree Examples}

  \begin{example}
    \begin{itemize}
      \item how many matches are played in a tennis tournament\\
        with 27 players?

      \pause
      \bigskip
      \item every player is a leaf: $l = 27$
      \item every match is an internal node: $m = 2$
      \item number of matches: $i = \frac{l - 1}{m - 1} = \frac{27 - 1}{2 - 1} = 26$
    \end{itemize}
  \end{example}
\end{frame}

\begin{frame}
  \frametitle{Regular Tree Examples}

  \begin{example}
    \begin{itemize}
      \item how many extension cords with 4 outlets are required\\
        to connect 25 computers to a wall socket?

      \pause
      \bigskip
      \item every computer is a leaf: $l = 25$
      \item every extension cord is an internal node: $m = 4$
      \item number of cords: $i = \frac{l - 1}{m - 1} = \frac{25 - 1}{4 - 1} = 8$
    \end{itemize}
  \end{example}
\end{frame}

\subsection{Decision Trees}

% TODO: write an introductory slide, games

\begin{frame}
  \frametitle{Decision Trees}

  \begin{example}
    \begin{itemize}
      \item one of 8 coins is counterfeit (it's heavier)
      \item find the counterfeit coin using a beam balance
    \end{itemize}
  \end{example}
\end{frame}

\begin{frame}
  \frametitle{Decision Trees}

  \begin{example}[in 3 weighings]
    \begin{center}
      \pgfuseimage{scale1}
    \end{center}
  \end{example}
\end{frame}

\begin{frame}
  \frametitle{Decision Trees}

  \begin{example}[in 2 weighings]
    \begin{center}
      \pgfuseimage{scale2}
    \end{center}
  \end{example}
\end{frame}

% TODO: give the decision tree for tic-tac-toe and/or nim

\section{Tree Problems}

\subsection{Minimum Spanning Tree}

\begin{frame}
  \frametitle{Spanning Tree}

  \begin{itemize}
    \item $T$ is a \alert{spanning tree} of $G$:\\
      $T$ is a subgraph of $G$ such that $T$ is a tree and\\
      contains all the nodes of $G$
  \end{itemize}

  \begin{definition}
    $MST$ is a \alert{minimum spanning tree} of $G$:\\
    $MST$ is a spanning tree of $G$ such that the total weight\\
    of the edges in $MST$ is minimal
  \end{definition}
\end{frame}

\begin{frame}
  \frametitle{Kruskal's Algorithm}

  \begin{block}{Kruskal's algorithm}
    \begin{enumerate}
      \item $i \leftarrow 1$, $e_1 \in E$, $wt(e_1)$ is minimal

      \pause
      \item for $1 \leq i \leq n-2$:\\
        the selected edges are $e_1,e_2,\dots,e_i$\\
        select a new edge $e_{i+1}$ from the remaining edges such that:
      \begin{itemize}
        \item $wt(e_{i+1})$ is minimal
        \item $e_1,e_2,\dots,e_i,e_{i+1}$ contains no cycle
      \end{itemize}

      \pause
      \item $i \leftarrow i+1$
      \begin{itemize}
        \item $i=n-1$ $\Rightarrow$ the subgraph $G$ containing the edges\\
         $e_1,e_2,\dots,e_{n-1}$ is a minimum spanning tree
        \item $i<n-1$ $\Rightarrow$ go to step 2
      \end{itemize}
    \end{enumerate}
  \end{block}
\end{frame}

\begin{frame}
  \frametitle{Kruskal's Algorithm Example}

  \begin{example}[initialization]
    \begin{columns}
      \column{.4\textwidth}
      \begin{center}
        \pgfuseimage{spanning}
      \end{center}

      \pause
      \column{.6\textwidth}
      \begin{itemize}
        \item $i \leftarrow 1$
        \item minimum weight: $1$\\
          $(e,g)$

        \pause
        \item $T = \{ (e,g) \}$
      \end{itemize}
    \end{columns}
  \end{example}
\end{frame}

\begin{frame}
  \frametitle{Kruskal's Algorithm Example}

  \begin{example}[$1 < 6$]
    \begin{columns}
      \column{.4\textwidth}
      \begin{center}
        \pgfuseimage{kruskal1}
      \end{center}

      \pause
      \column{.6\textwidth}
      \begin{itemize}
        \item minimum weight: $2$\\
          $(d,e), (d,f), (f,g)$

        \pause
        \item $T = \{ (e,g), (d,f) \}$
        \item $i \leftarrow 2$
      \end{itemize}
    \end{columns}
  \end{example}
\end{frame}

\begin{frame}
  \frametitle{Kruskal's Algorithm Example}

  \begin{example}[$2 < 6$]
    \begin{columns}
      \column{.4\textwidth}
      \begin{center}
        \pgfuseimage{kruskal2}
      \end{center}

      \pause
      \column{.6\textwidth}
      \begin{itemize}
        \item minimum weight: $2$\\
          $(d,e), (f,g)$

        \pause
        \item $T = \{ (e,g), (d,f), (d,e) \}$
        \item $i \leftarrow 3$
      \end{itemize}
    \end{columns}
  \end{example}
\end{frame}

\begin{frame}
  \frametitle{Kruskal's Algorithm Example}

  \begin{example}[$3 < 6$]
    \begin{columns}
      \column{.4\textwidth}
      \begin{center}
        \pgfuseimage{kruskal3}
      \end{center}

      \pause
      \column{.6\textwidth}
      \begin{itemize}
        \item minimum weight: $2$\\
          $(f,g)$ forms a cycle

        \pause
        \item minimum weight: $3$\\
          $(c,e), (c,g), (d,g)$\\
          $(d,g)$ forms a cycle

        \pause
        \item $T = \{ (e,g), (d,f), (d,e), (c,e) \}$
        \item $i \leftarrow 4$
      \end{itemize}
    \end{columns}
  \end{example}
\end{frame}

\begin{frame}
  \frametitle{Kruskal's Algorithm Example}

  \begin{example}[$4 < 6$]
    \begin{columns}
      \column{.4\textwidth}
      \begin{center}
        \pgfuseimage{kruskal4}
      \end{center}

      \pause
      \column{.6\textwidth}
      \begin{itemize}
        \item $T = \{$\\
          $~~(e,g), (d,f), (d,e),$\\
          $~~(c,e), (b,e)$\\
          $\}$
        \item $i \leftarrow 5$
      \end{itemize}
    \end{columns}
  \end{example}
\end{frame}

\begin{frame}
  \frametitle{Kruskal's Algorithm Example}

  \begin{example}[$5 < 6$]
    \begin{columns}
      \column{.4\textwidth}
      \begin{center}
        \pgfuseimage{kruskal5}
      \end{center}

      \pause
      \column{.6\textwidth}
      \begin{itemize}
        \item $T = \{$\\
          $~~(e,g), (d,f), (d,e),$\\
          $~~(c,e), (b,e), (a,b)$\\
          $\}$
        \item $i \leftarrow 6$
      \end{itemize}
    \end{columns}
  \end{example}
\end{frame}

\begin{frame}
  \frametitle{Kruskal's Algorithm Example}

  \begin{example}[$6 \nless 6$]
    \begin{columns}
      \column{.4\textwidth}
      \begin{center}
        \pgfuseimage{kruskal}
      \end{center}

      \column{.6\textwidth}
      \begin{itemize}
        \item total weight: $17$
      \end{itemize}
    \end{columns}
  \end{example}
\end{frame}

\begin{frame}
  \frametitle{Prim's Algorithm}

  \begin{block}{Prim's algorithm}
    \begin{enumerate}
      \item $i \leftarrow 1$, $v_1 \in V$, $P=\{v_1\}$, $N=V-\{v_1\}$,
        $T=\emptyset$

      \pause
      \item for $1 \leq i \leq n-1$:\\
        $P=\{v_1,v_2,\dots,v_i\}$, $T=\{e_1,e_2,\dots,e_{i-1}\}$, $N=V-P$\\
        select a node $v_{i+1} \in N$ such that for a node $x \in P$\\
        $e=(x,v_{i+1}) \notin T$, $wt(e)$ is minimal\\
        $P \leftarrow P+\{v_{i+1}\}$, $N \leftarrow N-\{v_{i+1}\}$,
        $T \leftarrow T+\{e\}$

      \pause
      \item $i \leftarrow i+1$
      \begin{itemize}
        \item $i=n$ $\Rightarrow$: the subgraph $G$ containing the edges\\
          $e_1,e_2,\dots,e_{n-1}$ is a minimum spanning tree
        \item $i<n$ $\Rightarrow$ go to step 2

      \end{itemize}
    \end{enumerate}
  \end{block}
\end{frame}

\begin{frame}
  \frametitle{Prim's Algorithm Example}

  \begin{example}[initialization]
    \begin{columns}
      \column{.4\textwidth}
      \begin{center}
        \pgfuseimage{spanning}
      \end{center}

      \pause
      \column{.6\textwidth}
      \begin{itemize}
        \item $i \leftarrow 1$
        \item $P = \{ a \}$
        \item $N = \{ b, c, d, e, f, g \}$
        \item $T = \emptyset$
      \end{itemize}
    \end{columns}
  \end{example}
\end{frame}

\begin{frame}
  \frametitle{Prim's Algorithm Example}

  \begin{example}[$1 < 7$]
    \begin{columns}
      \column{.4\textwidth}
      \begin{center}
        \pgfuseimage{spanning}
      \end{center}

      \pause
      \column{.6\textwidth}
      \begin{itemize}
        \item $T = \{ (a,b) \}$
        \item $P = \{ a, b \}$
        \item $N = \{ c, d, e, f, g \}$
        \item $i \leftarrow 2$
      \end{itemize}
    \end{columns}
  \end{example}
\end{frame}

\begin{frame}
  \frametitle{Prim's Algorithm Example}

  \begin{example}[$2 < 7$]
    \begin{columns}
      \column{.4\textwidth}
      \begin{center}
        \pgfuseimage{prim1}
      \end{center}

      \pause
      \column{.6\textwidth}
      \begin{itemize}
        \item $T = \{ (a,b), (b,e) \}$
        \item $P = \{ a, b, e \}$
        \item $N = \{ c, d, f, g \}$
        \item $i \leftarrow 3$
      \end{itemize}
    \end{columns}
  \end{example}
\end{frame}

\begin{frame}
  \frametitle{Prim's Algorithm Example}

  \begin{example}[$3 < 7$]
    \begin{columns}
      \column{.4\textwidth}
      \begin{center}
        \pgfuseimage{prim2}
      \end{center}

      \pause
      \column{.6\textwidth}
      \begin{itemize}
        \item $T = \{ (a,b), (b,e), (e,g) \}$
        \item $P = \{ a, b, e, g \}$
        \item $N = \{ c, d, f \}$
        \item $i \leftarrow 4$
      \end{itemize}
    \end{columns}
  \end{example}
\end{frame}

\begin{frame}
  \frametitle{Prim's Algorithm Example}

  \begin{example}[$4 < 7$]
    \begin{columns}
      \column{.4\textwidth}
      \begin{center}
        \pgfuseimage{prim3}
      \end{center}

      \pause
      \column{.6\textwidth}
      \begin{itemize}
        \item $T = \{ (a,b), (b,e), (e,g), (d,e) \}$
        \item $P = \{ a, b, e, g, d \}$
        \item $N = \{ c, f \}$
        \item $i \leftarrow 5$
      \end{itemize}
    \end{columns}
  \end{example}
\end{frame}

\begin{frame}
  \frametitle{Prim's Algorithm Example}

  \begin{example}[$5 < 7$]
    \begin{columns}
      \column{.4\textwidth}
      \begin{center}
        \pgfuseimage{prim4}
      \end{center}

      \pause
      \column{.6\textwidth}
      \begin{itemize}
        \item $T = \{$\\
          $~~(a,b), (b,e), (e,g),$\\
          $~~(d,e), (f,g)$\\
          $\}$
        \item $P = \{ a, b, e, g, d, f \}$
        \item $N = \{ c \}$
        \item $i \leftarrow 6$
      \end{itemize}
    \end{columns}
  \end{example}
\end{frame}

\begin{frame}
  \frametitle{Prim's Algorithm Example}

  \begin{example}[$6 < 7$]
    \begin{columns}
      \column{.4\textwidth}
      \begin{center}
        \pgfuseimage{prim5}
      \end{center}

      \pause
      \column{.6\textwidth}
      \begin{itemize}
        \item $T = \{$\\
          $~~(a,b), (b,e), (e,g),$\\
          $~~(d,e), (f,g), (c,g)$\\
          $\}$
        \item $P = \{ a, b, e, g, d, f, c \}$
        \item $N = \emptyset$
        \item $i \leftarrow 7$
      \end{itemize}
    \end{columns}
  \end{example}
\end{frame}

\begin{frame}
  \frametitle{Prim's Algorithm Example}

  \begin{example}[$7 \nless 7$]
    \begin{columns}
      \column{.4\textwidth}
      \begin{center}
        \pgfuseimage{prim}
      \end{center}

      \column{.6\textwidth}
      \begin{itemize}
        \item total weight: $17$
      \end{itemize}
    \end{columns}
  \end{example}
\end{frame}

\subsection*{References}

\begin{frame}
  \frametitle{References}

  \begin{block}{Required Reading: Grimaldi}
    \begin{itemize}
      \item Chapter 12: Trees
      \begin{itemize}
        \item 12.1. \alert{Definitions and Examples}
        \item 12.2. \alert{Rooted Trees}
      \end{itemize}

      \item Chapter 13: Optimization and Matching
      \begin{itemize}
        \item 13.2. \alert{Minimal Spanning Trees:\\
                           The Algorithms of Kruskal and Prim}
      \end{itemize}
    \end{itemize}
  \end{block}
\end{frame}

\end{document}
