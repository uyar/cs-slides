% Copyright (c) 2001-2016
%       H. Turgut Uyar <uyar@itu.edu.tr>
%       Ayşegül Gençata Yayımlı <gencata@itu.edu.tr>
%       Emre Harmancı <harmanci@itu.edu.tr>
%
% This work is licensed under a "Creative Commons
% Attribution-NonCommercial-ShareAlike 4.0 International License".
% For more information, please visit:
% https://creativecommons.org/licenses/by-nc-sa/4.0/

\documentclass[dvipsnames]{beamer}

\usepackage{ae}
\usepackage[scaled=0.88]{beramono}
\usepackage[T1]{fontenc}
\usepackage[utf8]{inputenc}
\setbeamersize{text margin left=2em, text margin right=2em}

\mode<presentation>
{
  \usetheme{Rochester}
  \setbeamercovered{transparent}
}

\title{Discrete Mathematics}
\subtitle{Trees}

\author{H. Turgut Uyar \and Ayşegül Gençata Yayımlı \and Emre Harmancı}
\date{2001-2016}

\AtBeginSubsection[]
{
  \begin{frame}<beamer>
    \frametitle{Topics}
    \tableofcontents[currentsection,currentsubsection]
  \end{frame}
}

\pgfdeclareimage[height=1cm]{license}{../license}

\pgfdeclareimage[height=3.5cm]{tree}{tree}
\pgfdeclareimage[height=4cm]{nodecount}{nodecount}
\pgfdeclareimage[height=5cm]{rooted}{rooted}
\pgfdeclareimage[width=6.5cm]{book}{book}
\pgfdeclareimage[width=6.5cm]{dictionary}{dictionary}
\pgfdeclareimage[height=2cm]{expr1a}{expr1a}
\pgfdeclareimage[height=2cm]{expr1b}{expr1b}
\pgfdeclareimage[height=3cm]{expr2a}{expr2a}
\pgfdeclareimage[height=3cm]{expr2b}{expr2b}
\pgfdeclareimage[height=4cm]{expr3}{expr3}
\pgfdeclareimage[height=6cm]{expr}{expr}
\pgfdeclareimage[width=10cm]{scale1}{scale1}
\pgfdeclareimage[width=10cm]{scale2}{scale2}
\pgfdeclareimage[height=4cm]{spanning}{spanning}
\pgfdeclareimage[height=4cm]{kruskal1}{kruskal1}
\pgfdeclareimage[height=4cm]{kruskal2}{kruskal2}
\pgfdeclareimage[height=4cm]{kruskal3}{kruskal3}
\pgfdeclareimage[height=4cm]{kruskal4}{kruskal4}
\pgfdeclareimage[height=4cm]{kruskal5}{kruskal5}
\pgfdeclareimage[height=4cm]{kruskal}{kruskal}
\pgfdeclareimage[height=4cm]{prim1}{prim1}
\pgfdeclareimage[height=4cm]{prim2}{prim2}
\pgfdeclareimage[height=4cm]{prim3}{prim3}
\pgfdeclareimage[height=4cm]{prim4}{prim4}
\pgfdeclareimage[height=4cm]{prim5}{prim5}
\pgfdeclareimage[height=4cm]{prim}{prim}

\begin{document}

\begin{frame}
  \titlepage
\end{frame}

\begin{frame}
  \frametitle{License}

  \pgfuseimage{license}\hfill
  \copyright~2001-2016 T. Uyar, A. Yayımlı, E. Harmancı

  \vfill
  \begin{footnotesize}
    You are free to:
    \begin{itemize}
      \itemsep0em
      \item Share -- copy and redistribute the material in any medium or format
      \item Adapt -- remix, transform, and build upon the material
    \end{itemize}

    Under the following terms:
    \begin{itemize}
      \itemsep0em
      \item Attribution -- You must give appropriate credit, provide a link to
        the license, and indicate if changes were made.

      \item NonCommercial -- You may not use the material for commercial
        purposes.

      \item ShareAlike -- If you remix, transform, or build upon the material,
        you must distribute your contributions under the same license as the
        original.
    \end{itemize}
  \end{footnotesize}

  \begin{small}
    For more information:\\
    \url{https://creativecommons.org/licenses/by-nc-sa/4.0/}

    \smallskip
    Read the full license:\\
    \url{https://creativecommons.org/licenses/by-nc-sa/4.0/legalcode}
  \end{small}
\end{frame}

\begin{frame}
  \frametitle{Topics}
  \tableofcontents
\end{frame}

\section{Trees}

\subsection{Introduction}

\begin{frame}
  \frametitle{Tree}

  \begin{definition}
    \alert{tree}: connected graph with no cycle
  \end{definition}

  \begin{exampleblock}{examples}
    \begin{center}
      \pgfuseimage{tree}
    \end{center}
  \end{exampleblock}
\end{frame}

\begin{frame}
  \frametitle{Tree Theorems}

  \begin{theorem}
    \begin{center}
      $T$ is a tree ($T$ is connected and contains no cycle).\\
      $\Leftrightarrow$\\
      There is one and only one path\\
        between any two distinct nodes in $T$.\\
      $\Leftrightarrow$\\
      $T$ is connected, but if any edge is removed\\
        it will no longer be connected.\\
      $\Leftrightarrow$\\
      $T$ contains no cycle, but if an edge is added\\
        between any pair of nodes one and only one cycle will be formed.
    \end{center}
  \end{theorem}
\end{frame}

\begin{frame}
  \frametitle{Tree Theorems}

  \begin{theorem}
    \[|E| = |V| - 1\]
  \end{theorem}

  \begin{itemize}
    \item proof method: induction on the number of edges
  \end{itemize}
\end{frame}

\begin{frame}
  \frametitle{Tree Theorems}

  \begin{block}{Proof: Base step.}
    \begin{itemize}
      \item $|E|=0 \Rightarrow |V|=1$
      \item $|E|=1 \Rightarrow |V|=2$
      \item $|E|=2 \Rightarrow |V|=3$

      \pause
      \medskip
      \item assume that $|E| = |V| - 1$ for $|E| \leq k$
    \end{itemize}
  \end{block}
\end{frame}

\begin{frame}
  \frametitle{Tree Theorems}

  \begin{proof}[Proof: Induction step]
    \begin{itemize}
      \item $|E|=k+1$
    \end{itemize}

    \begin{columns}[t]
      \column{.4\textwidth}
      \begin{center}
        \pgfuseimage{nodecount}
      \end{center}

      \pause
      \column{.55\textwidth}
      \begin{itemize}
        \item remove edge $(y,z)$:\\
          $T_1=(V_1,E_1)$, $T_2=(V_2,E_2)$
      \end{itemize}
      \pause
      \begin{eqnarray*}
        |V| & = & |V_1|+|V_2|\\\pause
            & = & |E_1|+1+|E_2|+1\\\pause
            & = & (|E_1|+|E_2|+1)+1\\\pause
            & = & |E|+1
      \end{eqnarray*}
    \end{columns}
  \end{proof}
\end{frame}

\begin{frame}
  \frametitle{Tree Theorems}

  \begin{theorem}
    \begin{center}
      $T$ is a tree ($T$ is connected and contains no cycle).\\
      $\Leftrightarrow$\\
      $T$ is connected and $|E| = |V| - 1$.\\
      $\Leftrightarrow$\\
      $T$ contains no cycle and $|E| = |V| - 1$.
    \end{center}
  \end{theorem}
\end{frame}

\begin{frame}
  \frametitle{Tree Theorems}

  \begin{theorem}
    In a tree, there are at least two nodes with degree 1.
  \end{theorem}

  \pause
  \begin{proof}
    \begin{itemize}
      \item $2 |E| = \sum_{v \in V} d_v$

      \pause
      \item assume: only 1 node with degree 1:\\
        \pause
        $\Rightarrow 2 |E| \geq 2 (|V| - 1) + 1$\\
        \pause
        $\Rightarrow 2 |E| \geq 2 |V| - 1$\\
        \pause
        $\Rightarrow |E| \geq |V| - \frac{1}{2}$
        \pause
        $> |V| - 1$
    \end{itemize}
  \end{proof}
\end{frame}

\subsection{Rooted Trees}

\begin{frame}
  \frametitle{Rooted Tree}

  \begin{itemize}
    \item hierarchy between nodes
    \item creates implicit direction on edges: in and out degrees

    \pause
    \medskip
    \item in-degree 0: \alert{root} (only 1 such node)
    \item out-degree 0: \alert{leaf}
    \item not a leaf: \alert{internal} node
  \end{itemize}
\end{frame}

\begin{frame}
  \frametitle{Node Level}

  \begin{itemize}
    \item \alert{level} of node: distance from root

    \medskip
    \item \emph{parent}: adjacent node closer to root (only 1 such node)
    \item \emph{child}: adjacent nodes further from root
    \item \emph{sibling}: nodes with same parent

    \medskip
    \item \alert{depth} of tree: maximum level in tree
  \end{itemize}
\end{frame}

\begin{frame}
  \frametitle{Rooted Tree Example}

  \begin{columns}
    \column{.4\textwidth}
    \begin{center}
      \pgfuseimage{rooted}
    \end{center}

    \column{.58\textwidth}
    \begin{itemize}
      \item root: $r$
      \item leaves: $x ~ y ~ z ~ u ~ v$
      \item internal nodes: $r ~ p ~ n ~ t ~ s ~ q ~ w$
      \item parent of $y$: $w$\\
        children of $w$: $y$ and $z$\\
      \item $y$ and $z$ are siblings
    \end{itemize}
  \end{columns}
\end{frame}

\begin{frame}
  \frametitle{Rooted Tree Example}

  \begin{columns}
    \column{.65\textwidth}
    \begin{center}
      \pgfuseimage{book}
    \end{center}

    \column{.33\textwidth}
    Book
    \begin{itemize}
      \item C1
      \begin{itemize}
        \item S1.1
        \item S1.2
      \end{itemize}
      \item C2
      \item C3
      \begin{itemize}
        \item S3.1
        \item S3.2
        \begin{itemize}
          \item S3.2.1
          \item S3.2.2
        \end{itemize}
        \item S3.3
      \end{itemize}
    \end{itemize}
  \end{columns}
\end{frame}

\begin{frame}
  \frametitle{Ordered Rooted Tree}

  \begin{itemize}
    \item siblings ordered from left to right

    \medskip
    \item \alert{universal address system}
    \smallskip
    \item root: $0$
    \item children of root: $1,2,3,\ldots$
    \item $v$: internal node with address $a$\\
      children of $v$: $a.1,a.2,a.3,\ldots$
  \end{itemize}
\end{frame}

\begin{frame}
  \frametitle{Lexicographic Order}

  \begin{itemize}
    \item address $A$ comes before address $B$ if one of:

    \medskip
    \item $A=x_1 x_2 \ldots x_i x_j \ldots$\\
      $B=x_1 x_2 \ldots x_i x_k \ldots$\\
      $x_j$ comes before $x_k$

    \pause
    \smallskip
    \item $A=x_1 x_2 \ldots x_i$\\
      $B=x_1 x_2 \ldots x_i x_k \ldots$
  \end{itemize}
\end{frame}

\begin{frame}
  \frametitle{Lexicographic Order Example}

  \begin{columns}
    \column{.57\textwidth}
    \begin{center}
      \pgfuseimage{dictionary}
    \end{center}

    \column{.4\textwidth}
    \begin{itemize}
      \item 0 - 1 - 1.1 - 1.2\\
        - 1.2.1 - 1.2.2 - 1.2.3\\
        - 1.2.3.1 - 1.2.3.2\\
        - 1.3 - 1.4 - 2\\
        - 2.1 - 2.2 - 2.2.1\\
        - 3 - 3.1 - 3.2
    \end{itemize}
  \end{columns}
\end{frame}

\subsection{Binary Trees}

\begin{frame}
  \frametitle{Binary Trees}

  \begin{itemize}
    \item $T=(V,E)$ is a \alert{binary tree}:\\
      $\forall v \in V~[{d_v}^o \in \{0,1,2\}]$
  \end{itemize}

  \begin{itemize}
    \item $T=(V,E)$ is a \emph{complete} binary tree:\\
      $\forall v \in V~[{d_v}^o \in \{0,2\}]$
  \end{itemize}

\end{frame}

\begin{frame}
  \frametitle{Expression Tree}

  \begin{itemize}
    \item binary operations can be represented as binary trees
    \item root: operator, children: operands

    \medskip
    \item mathematical expression can be represented as trees
    \item internal nodes: operators, leaves: variables and values
  \end{itemize}
\end{frame}

\begin{frame}
  \frametitle{Expression Tree Examples}

  \vspace{-24pt}
  \begin{columns}[t]
    \column{.5\textwidth}
    \begin{center}
      \[ 7-a \]

      \medskip
      \pgfuseimage{expr1a}
    \end{center}

    \column{.5\textwidth}
    \begin{center}
      \[ a+b \]

      \medskip
      \pgfuseimage{expr1b}
    \end{center}
  \end{columns}
\end{frame}

\begin{frame}
  \frametitle{Expression Tree Examples}

  \vspace{-24pt}
  \begin{columns}[t]
    \column{.5\textwidth}
    \begin{center}
      \[ \frac{7-a}{5} \]

      \medskip
      \pgfuseimage{expr2a}
    \end{center}

    \column{.5\textwidth}
    \begin{center}
      \[ (a+b)^3 \]

      \medskip
      \pgfuseimage{expr2b}
    \end{center}
  \end{columns}
\end{frame}

\begin{frame}
  \frametitle{Expression Tree Examples}

  \vspace{-24pt}
  \begin{center}
    \[ \frac{7-a}{5} \cdot (a+b)^3 \]

    \medskip
    \pgfuseimage{expr3}
  \end{center}
\end{frame}

\begin{frame}
  \frametitle{Expression Tree Examples}

  \begin{columns}
    \column{.5\textwidth}
    \[ t + \frac{u*v}{w+x-y^z} \]

    \column{.5\textwidth}
    \pgfuseimage{expr}
  \end{columns}
\end{frame}

\begin{frame}
  \frametitle{Expression Tree Traversals}

  \begin{enumerate}
    \item \alert{inorder} traversal:\\
      traverse left subtree, visit root, traverse right subtree

    \pause
    \medskip
    \item \alert{preorder} traversal:\\
      visit root, traverse left subtree, traverse right subtree

    \pause
    \medskip
    \item \alert{postorder} traversal (reverse Polish notation):\\
      traverse left subtree, traverse right subtree, visit root
  \end{enumerate}
\end{frame}

\begin{frame}
  \frametitle{Inorder Traversal Example}

  \begin{columns}
    \column{.4\textwidth}
    \begin{center}
      \pgfuseimage{expr}
    \end{center}

    \column{.6\textwidth}
    $t ~ + ~ u ~ * ~ v ~ / ~ w ~ + ~ x ~ - ~ y ~ \uparrow ~ z$
  \end{columns}
\end{frame}

\begin{frame}
  \frametitle{Preorder Traversal Example}

  \begin{columns}
    \column{.4\textwidth}
    \begin{center}
      \pgfuseimage{expr}
    \end{center}

    \column{.6\textwidth}
    $+ ~ t ~ / ~ * ~ u ~ v ~ + ~ w ~ - ~ x ~ \uparrow ~ y ~ z$
  \end{columns}
\end{frame}

\begin{frame}
  \frametitle{Postorder Traversal Example}

  \begin{columns}
    \column{.4\textwidth}
    \begin{center}
      \pgfuseimage{expr}
    \end{center}

    \column{.6\textwidth}
    $t ~ u ~ v ~ * ~ w ~ x ~ y ~ z ~ \uparrow ~ - ~ + ~ / ~ +$
  \end{columns}
\end{frame}

\begin{frame}
  \frametitle{Expression Tree Evaluation}

  \begin{itemize}
    \item inorder traversal requires parantheses for precedence
    \item preorder and postorder traversals do not require parantheses
  \end{itemize}
\end{frame}

\begin{frame}
  \frametitle{Postorder Evaluation Example}

  $t ~ u ~ v ~ * ~ w ~ x ~ y ~ z ~ \uparrow ~ - ~ + ~ / ~ +$

  $4 ~ 2 ~ 3 ~ * ~ 1 ~ 9 ~ 2 ~ 3 ~ \uparrow ~ - ~ + ~ / ~ +$

  \pause
  \medskip
  \[
    \begin{array}{ccccccc}
4 & \pause 2 & \pause 3 & \pause * &          &          &                \\
4 &        6 & \pause 1 & \pause 9 & \pause 2 & \pause 3 & \pause \uparrow\\
4 &        6 &        1 &        9 &        8 & \pause - &                \\
4 &        6 &        1 &        1 & \pause + &          &                \\
4 &        6 &        2 & \pause / &          &          &                \\
4 &        3 & \pause + &          &          &          &                \\
7 &          &          &          &          &          &                \\
    \end{array}
  \]
\end{frame}

\begin{frame}
  \frametitle{Regular Trees}

  \begin{itemize}
    \item $T=(V,E)$ is an \alert{m-ary tree}:\\
      $\forall v \in V~[{d_v}^o \leq m]$
    \item $T=(V,E)$ is a complete m-ary tree:\\
      $\forall v \in V~[{d_v}^o \in \{0,m\}]$
  \end{itemize}
\end{frame}

\begin{frame}
  \frametitle{Regular Tree Theorem}

  \begin{theorem}
    $T=(V,E)$: complete $m$-ary tree

    \begin{itemize}
      \item $n$: number of nodes
      \item $l$: number of leaves
      \item $i$: number of internal nodes
    \end{itemize}

    \pause
    \begin{itemize}
      \item $n = m \cdot i + 1$

      \pause
      \item  $l = n - i = \pause m \cdot i + 1 - i
        \pause = (m - 1) \cdot i + 1$

      \pause
      \[
        i = \frac{l - 1}{m - 1}
      \]
    \end{itemize}
  \end{theorem}
\end{frame}

\begin{frame}
  \frametitle{Regular Tree Examples}

  \begin{itemize}
    \item how many matches are played in a tennis tournament\\
      with 27 players?

    \pause
    \bigskip
    \item every player is a leaf: $l = 27$
    \item every match is an internal node: $m = 2$
    \item number of matches: $i = \frac{l - 1}{m - 1} = \frac{27 - 1}{2 - 1} = 26$
  \end{itemize}
\end{frame}

\begin{frame}
  \frametitle{Regular Tree Examples}

  \begin{itemize}
    \item how many extension cords with 4 outlets are required\\
      to connect 25 computers to a wall socket?

    \pause
    \bigskip
    \item every computer is a leaf: $l = 25$
    \item every extension cord is an internal node: $m = 4$
    \item number of cords: $i = \frac{l - 1}{m - 1} = \frac{25 - 1}{4 - 1} = 8$
  \end{itemize}
\end{frame}

\subsection{Decision Trees}

% TODO: write an introductory slide, games

\begin{frame}
  \frametitle{Decision Trees}

  \begin{itemize}
    \item one of 8 coins is counterfeit (heavier)
    \item find counterfeit coin using a beam balance

    \medskip
    \item depth of tree: number of weighings
  \end{itemize}
\end{frame}

\begin{frame}
  \frametitle{Decision Trees}

  \begin{center}
    \pgfuseimage{scale1}
  \end{center}
\end{frame}

\begin{frame}
  \frametitle{Decision Trees}

  \begin{center}
    \pgfuseimage{scale2}
  \end{center}
\end{frame}

% TODO: give the decision tree for tic-tac-toe and/or nim

\section{Tree Problems}

\subsection{Minimum Spanning Tree}

\begin{frame}
  \frametitle{Spanning Tree}

  \begin{itemize}
    \item $T=(V',E')$ is a \alert{spanning tree} of $G(V,E)$:\\
      $T$ is a subgraph of $G$\\
      $T$ is a tree\\
      $V'=V$

    \medskip
    \item \alert{minimum spanning tree}:\\
      total weight of edges in $E'$ is minimal
  \end{itemize}
\end{frame}

\begin{frame}
  \frametitle{Kruskal's Algorithm}

  \begin{enumerate}
    \item $G'=(V',E'), V'=\emptyset, E'=\emptyset$
    \item select $e=(v_1,v_2) \in E-E'$ such that:\\
      $E' \cup \{e\}$ contains no cycle, and $wt(e)$ is minimal
    \item $E'=E \cup \{e\}, V'=V' \cup \{v_1,v_2\}$
    \item if $|E'|=|V|-1$: result is $G'$
    \item go to step 2
  \end{enumerate}
\end{frame}

\begin{frame}
  \frametitle{Kruskal's Algorithm Example}

  \begin{columns}
    \column{.4\textwidth}
    \begin{center}
      \pgfuseimage{spanning}
    \end{center}

    \pause
    \column{.6\textwidth}
    \begin{itemize}
      \item minimum weight: $1$\\
        $(e,g)$

      \pause
      \item $E' = \{ (e,g) \}$
      \item $|E'|=1$
    \end{itemize}
  \end{columns}
\end{frame}

\begin{frame}
  \frametitle{Kruskal's Algorithm Example}

  \begin{columns}
    \column{.4\textwidth}
    \begin{center}
      \pgfuseimage{kruskal1}
    \end{center}

    \pause
    \column{.6\textwidth}
    \begin{itemize}
      \item minimum weight: $2$\\
        $(d,e), (d,f), (f,g)$

      \pause
      \item $E' = \{ (e,g), (d,f) \}$
      \item $|E'| = 2$
    \end{itemize}
  \end{columns}
\end{frame}

\begin{frame}
  \frametitle{Kruskal's Algorithm Example}

  \begin{columns}
    \column{.4\textwidth}
    \begin{center}
      \pgfuseimage{kruskal2}
    \end{center}

    \pause
    \column{.6\textwidth}
    \begin{itemize}
      \item minimum weight: $2$\\
        $(d,e), (f,g)$

      \pause
      \item $E' = \{ (e,g), (d,f), (d,e) \}$
      \item $|E'| = 3$
    \end{itemize}
  \end{columns}
\end{frame}

\begin{frame}
  \frametitle{Kruskal's Algorithm Example}

  \begin{columns}
    \column{.4\textwidth}
    \begin{center}
      \pgfuseimage{kruskal3}
    \end{center}

    \pause
    \column{.6\textwidth}
    \begin{itemize}
      \item minimum weight: $2$\\
        $(f,g)$ forms a cycle

      \pause
      \item minimum weight: $3$\\
        $(c,e), (c,g), (d,g)$\\
        $(d,g)$ forms a cycle

      \pause
      \item $E' = \{ (e,g), (d,f), (d,e), (c,e) \}$
      \item $|E'| = 4$
    \end{itemize}
  \end{columns}
\end{frame}

\begin{frame}
  \frametitle{Kruskal's Algorithm Example}

  \begin{columns}
    \column{.4\textwidth}
    \begin{center}
      \pgfuseimage{kruskal4}
    \end{center}

    \pause
    \column{.6\textwidth}
    \begin{itemize}
      \item $E' = \{$\\
        $~~(e,g), (d,f), (d,e),$\\
        $~~(c,e), (b,e)$\\
        $\}$
      \item $|E'| = 5$
    \end{itemize}
  \end{columns}
\end{frame}

\begin{frame}
  \frametitle{Kruskal's Algorithm Example}

  \begin{columns}
    \column{.4\textwidth}
    \begin{center}
      \pgfuseimage{kruskal5}
    \end{center}

    \pause
    \column{.6\textwidth}
    \begin{itemize}
      \item $E' = \{$\\
        $~~(e,g), (d,f), (d,e),$\\
        $~~(c,e), (b,e), (a,b)$\\
        $\}$
      \item $|E'| = 6$
    \end{itemize}
  \end{columns}
\end{frame}

\begin{frame}
  \frametitle{Kruskal's Algorithm Example}

  \begin{columns}
    \column{.4\textwidth}
    \begin{center}
      \pgfuseimage{kruskal}
    \end{center}

    \column{.6\textwidth}
    \begin{itemize}
      \item total weight: $17$
    \end{itemize}
  \end{columns}
\end{frame}

\begin{frame}
  \frametitle{Prim's Algorithm}

  \begin{enumerate}
    \item $T'=(V',E'), E'=\emptyset, v_0 \in V, V'=\{v_0\}$
    \item select $v \in V-V'$ such that for a node $x \in V'$\\
      $e=(x,v), E' \cup \{e\}$ contains no cycle, and $wt(e)$ is minimal
    \item $E'=E \cup \{e\}, V'=V' \cup \{x\}$
    \item if $|V'|=|V|$: result is $T'$
    \item go to step 2
  \end{enumerate}
\end{frame}

\begin{frame}
  \frametitle{Prim's Algorithm Example}

  \begin{columns}
    \column{.4\textwidth}
    \begin{center}
      \pgfuseimage{spanning}
    \end{center}

    \pause
    \column{.6\textwidth}
    \begin{itemize}
      \item $E' = \emptyset$
      \item $V' = \{ a \}$
      \item $|V'| = 1$
    \end{itemize}
  \end{columns}
\end{frame}

\begin{frame}
  \frametitle{Prim's Algorithm Example}

  \begin{columns}
    \column{.4\textwidth}
    \begin{center}
      \pgfuseimage{spanning}
    \end{center}

    \pause
    \column{.6\textwidth}
    \begin{itemize}
      \item $E' = \{ (a,b) \}$
      \item $V' = \{ a, b \}$
      \item $|V'| = 2$
    \end{itemize}
  \end{columns}
\end{frame}

\begin{frame}
  \frametitle{Prim's Algorithm Example}

  \begin{columns}
    \column{.4\textwidth}
    \begin{center}
      \pgfuseimage{prim1}
    \end{center}

    \pause
    \column{.6\textwidth}
    \begin{itemize}
      \item $E' = \{ (a,b), (b,e) \}$
      \item $V' = \{ a, b, e \}$
      \item $|V'| = 3$
    \end{itemize}
  \end{columns}
\end{frame}

\begin{frame}
  \frametitle{Prim's Algorithm Example}

  \begin{columns}
    \column{.4\textwidth}
    \begin{center}
      \pgfuseimage{prim2}
    \end{center}

    \pause
    \column{.6\textwidth}
    \begin{itemize}
      \item $E' = \{ (a,b), (b,e), (e,g) \}$
      \item $V' = \{ a, b, e, g \}$
      \item $|V'| = 4$
    \end{itemize}
  \end{columns}
\end{frame}

\begin{frame}
  \frametitle{Prim's Algorithm Example}

  \begin{columns}
    \column{.4\textwidth}
    \begin{center}
      \pgfuseimage{prim3}
    \end{center}

    \pause
    \column{.6\textwidth}
    \begin{itemize}
      \item $E' = \{ (a,b), (b,e), (e,g), (d,e) \}$
      \item $V' = \{ a, b, e, g, d \}$
      \item $|V'| = 5$
    \end{itemize}
  \end{columns}
\end{frame}

\begin{frame}
  \frametitle{Prim's Algorithm Example}

  \begin{columns}
    \column{.4\textwidth}
    \begin{center}
      \pgfuseimage{prim4}
    \end{center}

    \pause
    \column{.6\textwidth}
    \begin{itemize}
      \item $E' = \{$\\
        $~~(a,b), (b,e), (e,g),$\\
        $~~(d,e), (f,g)$\\
        $\}$
      \item $V' = \{ a, b, e, g, d, f \}$
      \item $|V'| = 6$
    \end{itemize}
  \end{columns}
\end{frame}

\begin{frame}
  \frametitle{Prim's Algorithm Example}

  \begin{columns}
    \column{.4\textwidth}
    \begin{center}
      \pgfuseimage{prim5}
    \end{center}

    \pause
    \column{.6\textwidth}
    \begin{itemize}
      \item $E' = \{$\\
        $~~(a,b), (b,e), (e,g),$\\
        $~~(d,e), (f,g), (c,g)$\\
        $\}$
      \item $V' = \{ a, b, e, g, d, f, c \}$
      \item $|V'| = 7$
    \end{itemize}
  \end{columns}
\end{frame}

\begin{frame}
  \frametitle{Prim's Algorithm Example}

  \begin{columns}
    \column{.4\textwidth}
    \begin{center}
      \pgfuseimage{prim}
    \end{center}

    \column{.6\textwidth}
    \begin{itemize}
      \item total weight: $17$
    \end{itemize}
  \end{columns}
\end{frame}

\subsection*{References}

\begin{frame}
  \frametitle{References}

  \begin{block}{Required Reading: Grimaldi}
    \begin{itemize}
      \item Chapter 12: Trees
      \begin{itemize}
        \item 12.1. \alert{Definitions and Examples}
        \item 12.2. \alert{Rooted Trees}
      \end{itemize}

      \item Chapter 13: Optimization and Matching
      \begin{itemize}
        \item 13.2. \alert{Minimal Spanning Trees:\\
                           The Algorithms of Kruskal and Prim}
      \end{itemize}
    \end{itemize}
  \end{block}
\end{frame}

\end{document}
