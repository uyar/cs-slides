% Copyright (c) 2001-2023
%       H. Turgut Uyar <uyar@itu.edu.tr>
%       Ayşegül Gençata Yayımlı <gencata@itu.edu.tr>
%       Emre Harmancı <harmanci@itu.edu.tr>
%
% This work is licensed under a "Creative Commons
% Attribution-NonCommercial-ShareAlike 4.0 International License".
% For more information, please visit:
% https://creativecommons.org/licenses/by-nc-sa/4.0/

\documentclass[dvipsnames]{beamer}

\usepackage{ae}
\usepackage[scaled=0.88]{beramono}
\usepackage[T1]{fontenc}
\usepackage[utf8]{inputenc}
\setbeamersize{text margin left=2em, text margin right=2em}

\mode<presentation>
{
  \usetheme{Rochester}
  \setbeamercovered{transparent}
}

\title{Discrete Mathematics}
\subtitle{Predicates and Proofs}

\author{H. Turgut Uyar \and Ayşegül Gençata Yayımlı \and Emre Harmancı}
\date{2001-2023}

\AtBeginSubsection[]
{
  \begin{frame}<beamer>
    \frametitle{Topics}
    \tableofcontents[currentsection,currentsubsection]
  \end{frame}
}

%\beamerdefaultoverlayspecification{<+->}

\pgfdeclareimage[height=1cm]{license}{../license}

\pgfdeclareimage{induction}{induction}
\pgfdeclareimage{error1}{error1}
\pgfdeclareimage{error2}{error2}

\begin{document}

\begin{frame}
  \titlepage
\end{frame}

\begin{frame}
  \frametitle{License}

  \pgfuseimage{license}\hfill
  \copyright~2001-2023 T. Uyar, A. Yayımlı, E. Harmancı

  \vfill
  \begin{footnotesize}
    You are free to:
    \begin{itemize}
      \itemsep0em
      \item Share -- copy and redistribute the material in any medium or format
      \item Adapt -- remix, transform, and build upon the material
    \end{itemize}

    Under the following terms:
    \begin{itemize}
      \itemsep0em
      \item Attribution -- You must give appropriate credit, provide a link to
        the license, and indicate if changes were made.

      \item NonCommercial -- You may not use the material for commercial
        purposes.

      \item ShareAlike -- If you remix, transform, or build upon the material,
        you must distribute your contributions under the same license as the
        original.
    \end{itemize}
  \end{footnotesize}

  \begin{small}
    For more information:\\
    \url{https://creativecommons.org/licenses/by-nc-sa/4.0/}
  \end{small}
\end{frame}

\begin{frame}
  \frametitle{Topics}
  \tableofcontents
\end{frame}

\section{Predicates}

\subsection{Introduction}

\begin{frame}
  \frametitle{Predicates}

  \begin{definition}
    \alert{predicate}: declarative sentence which
    \begin{itemize}
      \item contains one or more variables, and
      \item is not a proposition, but
      \item becomes a proposition when variables are replaced\\
        by allowable choices
    \end{itemize}
  \end{definition}

  \begin{itemize}
    \item set of allowable choices: \alert{universe of discourse} ($\mathcal{U}$)
  \end{itemize}
\end{frame}

\begin{frame}
  \frametitle{Sets}

  \begin{columns}[t]
    \column{.55\textwidth}
    \begin{itemize}
      \item explicit notation: $\{a_1,a_2,\dots,a_n\}$

      \medskip
      \item $a \in S$: $a$ is an element of $S$
      \item $a \notin S$: $a$ is not an element of $S$
    \end{itemize}

    \pause
    \column{.45\textwidth}
    \begin{itemize}
      \item $\mathbb{Z}$: integers
      \item $\mathbb{N}$: natural numbers
      \item $\mathbb{Z}^+$: positive integers
      \item $\mathbb{Q}$: rational numbers
      \item $\mathbb{R}$: real numbers
      \item $\mathbb{C}$: complex numbers
    \end{itemize}
  \end{columns}
\end{frame}

\begin{frame}
  \frametitle{Predicate Examples}

  $\mathcal{U} = \mathbb{N}$\\
  $p(x)$: $x+2$ is an even integer.

  \smallskip
  $p(5)$: $F$\\
  $p(8)$: $T$

  \pause
  \bigskip
  $\neg p(x)$: $x+2$ is not an even integer.

  \pause
  \bigskip
  $\mathcal{U} = \mathbb{N}$\\
  $q(x,y)$: $x+y$ and $x-2y$ are even integers.

  \smallskip
  $q(11,3)$: $F$, $q(14,4)$: $T$
\end{frame}

\subsection{Quantifiers}

\begin{frame}
  \frametitle{Quantifiers}

  \begin{columns}[t]
    \column{.48\textwidth}
    \begin{definition}
      \alert{existential quantifier}:  $\exists$\\
        predicate is true for some values
    \end{definition}

    \begin{itemize}
      \item read: \emph{there exists}

      \pause
      \item one and only one: $\exists!$
    \end{itemize}

    \pause
    \column{.48\textwidth}
    \begin{definition}
      \alert{universal quantifier}: $\forall$\\
        predicate is true for all values
    \end{definition}

    \begin{itemize}
      \item read: \emph{for all}
    \end{itemize}
  \end{columns}

  \pause
  \bigskip
  \begin{center}
  $\mathcal{U} = \{x_1,x_2,\ldots,x_n\}$

  $\exists x~p(x) \Leftrightarrow p(x_1) \vee p(x_2) \vee \cdots \vee p(x_n)$

  $\forall x~p(x) \Leftrightarrow p(x_1) \wedge p(x_2) \wedge \cdots \wedge p(x_n)$
  \end{center}
\end{frame}

\begin{frame}
  \frametitle{Quantifier Examples}

  \begin{columns}[t]
    \column{.53\textwidth}
    $\mathcal{U} = \mathbb{R}$\\

    \begin{itemize}
      \item $p(x): x \geq 0$
      \item $q(x): x^2 \geq 0$
      \item $r(x): (x-4) (x+1) = 0$
      \item $s(x): x^2 -3 > 0$
    \end{itemize}

    are the following expressions true?

    \column{.4\textwidth}
    \begin{itemize}
      \pause
      \item $\exists x~[p(x) \wedge r(x)]$

      \pause
      \item $\forall x~[p(x) \rightarrow q(x)]$

      \pause
      \item $\forall x~[q(x) \rightarrow s(x)]$

      \pause
      \item $\forall x~[r(x) \vee s(x)]$

      \pause
      \item $\forall x~[r(x) \rightarrow p(x)]$
    \end{itemize}
  \end{columns}
\end{frame}

\begin{frame}
  \frametitle{Negating Quantifiers}

  \begin{itemize}
    \item replace $\forall$ with $\exists$, and $\exists$ with $\forall$
    \item negate the predicate
  \end{itemize}

  \pause
  \begin{eqnarray*}
    \neg \exists x~p(x)      & \Leftrightarrow & \forall x~\neg p(x)\\
    \neg \exists x~\neg p(x) & \Leftrightarrow & \forall x~p(x)\\
    \neg \forall x~p(x)      & \Leftrightarrow & \exists x~\neg p(x)\\
    \neg \forall x~\neg p(x) & \Leftrightarrow & \exists x~p(x)
  \end{eqnarray*}
\end{frame}

\begin{frame}
  \frametitle{Negating Quantifiers}

  \begin{theorem}
    $\neg \exists x~p(x) \Leftrightarrow \forall x~\neg p(x)$
  \end{theorem}

  \pause
  \begin{proof}
    \begin{eqnarray*}
      \neg \exists x~p(x) & \Leftrightarrow & \neg [p(x_1) \vee p(x_2) \vee \cdots
                                              \vee p(x_n)]\\\pause
                          & \Leftrightarrow & \neg p(x_1) \wedge \neg p(x_2) \wedge \cdots
                                              \wedge \neg p(x_n)\\\pause
                          & \Leftrightarrow & \forall x~\neg p(x)
    \end{eqnarray*}
  \end{proof}
\end{frame}

\begin{frame}
  \frametitle{Predicate Theorems}

  \begin{itemize}
    \item $\exists x~[p(x) \vee q(x)]
      \Leftrightarrow \exists x~p(x) \vee \exists x~q(x)$

    \pause
    \medskip
    \item $\forall x~[p(x) \wedge q(x)]
      \Leftrightarrow \forall x~p(x) \wedge \forall x~q(x)$

    \pause
    \bigskip
    \item $\forall x~p(x) \Rightarrow \exists x~p(x)$

    \pause
    \bigskip
    \item $\exists x~[p(x) \wedge q(x)]
      \Rightarrow \exists x~p(x) \wedge \exists x~q(x)$

    \pause
    \medskip
    \item $\forall x~p(x) \vee \forall x~q(x)
      \Rightarrow \forall x~[p(x) \vee q(x)]$
  \end{itemize}
\end{frame}

\subsection{Multiple Quantifiers}

\begin{frame}
  \frametitle{Multiple Quantifiers}

  \begin{itemize}
    \item quantifiers can be combined

    \medskip
    \item $\exists x \exists y~p(x,y)$
    \item $\forall x \exists y~p(x,y)$
    \item $\exists x \forall y~p(x,y)$
    \item $\forall x \forall y~p(x,y)$

    \medskip
    \item order of quantifiers is significant
  \end{itemize}
\end{frame}

\begin{frame}
  \frametitle{Multiple Quantifier Example}

  $\mathcal{U}=\mathbb{Z}$\\
  $p(x,y): x+y=17$

  \pause
  \medskip
  \begin{itemize}
    \item $\forall x \exists y~p(x,y)$:\\
      for every $x$ there exists a $y$ such that $x+y=17$

    \pause
    \item $\exists y \forall x~p(x,y)$:\\
      there exists a $y$ so that for all $x$,  $x+y=17$

    \pause
    \medskip
    \item what changes if $\mathcal{U}=\mathbb{N}$?
  \end{itemize}
\end{frame}

\begin{frame}
  \frametitle{Multiple Quantifiers}

  \begin{center}
    $\mathcal{U}_x = \{1,2\} \wedge \mathcal{U}_y = \{A,B\}$
  \end{center}

  \begin{eqnarray*}
    \exists x \exists y~p(x,y) & \Leftrightarrow & [p(1,A) \vee p(1,B)]
                                      \vee [p(2,A) \vee p(2,B)]\\\pause
    \exists x \forall y~p(x,y) & \Leftrightarrow & [p(1,A) \wedge p(1,B)]
                                      \vee [p(2,A) \wedge p(2,B)]\\\pause
    \forall x \exists y~p(x,y) & \Leftrightarrow & [p(1,A) \vee p(1,B)]
                                    \wedge [p(2,A) \vee p(2,B)]\\\pause
    \forall x \forall y~p(x,y) & \Leftrightarrow & [p(1,A) \wedge p(1,B)]
                                    \wedge [p(2,A) \wedge p(2,B)]
  \end{eqnarray*}
\end{frame}

\section{Proofs}

\subsection{Introduction}

\begin{frame}
  \frametitle{Method of Exhaustion}

  \begin{itemize}
    \item examining all possible cases one by one
  \end{itemize}

  \pause
  \begin{theorem}
    Every even number between $2$ and $26$ can be written\\
    as the sum of at most 3 square numbers.
  \end{theorem}

  \begin{proof}
    \begin{tabular}{lll}
      2 = 1+1   & 10 = 9+1    & 20 = 16+4\\
      4 = 4     & 12 = 4+4+4  & 22 = 9+9+4\\
      6 = 4+1+1 & 14 = 9+4+1  & 24 = 16+4+4\\
      8 = 4+4   & 16 = 16     & 26 = 25+1\\
                & 18 = 9+9    &
    \end{tabular}\\
  \end{proof}
\end{frame}

\begin{frame}
  \frametitle{Universal Specification}

  \begin{block}{Universal Specification (US)}
    $\forall x~p(x) \Rightarrow p(a)$
  \end{block}
\end{frame}

\begin{frame}
  \frametitle{Universal Specification Example}

  \begin{quote}
    All humans are mortal. Socrates is human.\\
    Therefore, Socrates is mortal.
  \end{quote}

  \pause
  \begin{itemize}
    \item $\mathcal{U}$: all humans
    \item $p(x)$: $x$ is mortal.
    \item $\forall x~p(x)$: All humans are mortal.
    \item $a$: Socrates, $a \in \mathcal{U}$: Socrates is human.
    \item therefore, $p(a)$: Socrates is mortal.
  \end{itemize}
\end{frame}

\begin{frame}
  \frametitle{Universal Specification Example}

  \begin{small}
  \begin{columns}
    \column{.4\textwidth}
    \[
    \frac
      {
        \begin{array}{c}
          \forall x~[j(x) \vee s(x) \rightarrow \neg p(x)]\\
          p(m)
        \end{array}
      }
      {
        \therefore \neg s(m)
      }
    \]

    \pause
    \column{.55\textwidth}
    \begin{eqnarray*}
      1. & \forall x~[j(x) \vee s(x) \rightarrow \neg p(x)] & A\\\pause
      2. & p(m)                                             & A\\\pause
      3. & j(m) \vee s(m) \rightarrow \neg p(m)             & US:1\\\pause
      4. & \neg (j(m) \vee s(m))                            & MT:3,2\\\pause
      5. & \neg j(m) \wedge \neg s(m)                       & DM:4\\\pause
      6. & \neg s(m)                                        & AndE:5
    \end{eqnarray*}
  \end{columns}
  \end{small}
\end{frame}

\begin{frame}
  \frametitle{Universal Generalization}

  \begin{block}{Universal Generalization (UG)}
    $p(a)$ for an \alert{arbitrarily chosen} $a$
      $\Rightarrow$ $\forall x~p(x)$
  \end{block}
\end{frame}

\begin{frame}
  \frametitle{Universal Generalization Example}

  \begin{columns}
    \column{.35\textwidth}
    \[
    \frac
      {
        \begin{array}{c}
          \forall x~[p(x) \rightarrow q(x)]\\
          \forall x~[q(x) \rightarrow r(x)]
        \end{array}
      }
      {
        \therefore \forall x~[p(x) \rightarrow r(x)]
      }
    \]

    \pause
    \column{.6\textwidth}
    \begin{eqnarray*}
      1. & \forall x~[p(x) \rightarrow q(x)] & A\\\pause
      2. & p(c) \rightarrow q(c)             & US:1\\\pause
      3. & \forall x~[q(x) \rightarrow r(x)] & A\\\pause
      4. & q(c) \rightarrow r(c)             & US:3\\\pause
      5. & p(c) \rightarrow r(c)             & HS:2,4\\\pause
      6. & \forall x~[p(x) \rightarrow r(x)] & UG:5
    \end{eqnarray*}
  \end{columns}
\end{frame}

\begin{frame}
  \frametitle{Vacuous Proof}

  \begin{block}{vacuous proof}
    \begin{tabular}{ll}
      to prove: & $\forall x~[p(x) \rightarrow q(x)]$\\
      show:     & $\forall x~\neg p(x)$
    \end{tabular}
  \end{block}
\end{frame}

\begin{frame}
  \frametitle{Vacuous Proof Example}

  \begin{theorem}
    $\forall x \in \mathbb{N}~[x < 0 \rightarrow \sqrt{x} < 0]$
  \end{theorem}

  \pause
  \begin{proof}
    $\forall x \in \mathbb{N}~[x \nless 0]$
  \end{proof}
\end{frame}

\begin{frame}
  \frametitle{Trivial Proof}

  \begin{block}{trivial proof}
    \begin{tabular}{ll}
      to prove: & $\forall x~[p(x) \rightarrow q(x)]$\\
      show:     & $\forall x~q(x)$
    \end{tabular}
  \end{block}
\end{frame}

\begin{frame}
  \frametitle{Trivial Proof Example}

  \begin{theorem}
    $\forall x \in \mathbb{R}~[x \geq 0 \rightarrow x^2 \geq 0]$
  \end{theorem}

  \pause
  \begin{proof}
    $\forall x \in \mathbb{R}~[x^2 \geq 0]$
  \end{proof}
\end{frame}

\subsection{Direct Proof}

\begin{frame}
  \frametitle{Direct Proof}

  \begin{block}{direct proof}
    \begin{tabular}{ll}
      to prove: & $\forall x~[p(x) \rightarrow q(x)]$\\
      show:     & $\forall x~[p(x) \vdash q(x)]$
    \end{tabular}
  \end{block}
\end{frame}

\begin{frame}
  \frametitle{Direct Proof Example}

  \begin{theorem}
    $\forall a \in \mathbb{Z}~[3~|~(a-2) \rightarrow 3~|~(a^2-1)]$\\
    $x~|~y$: $y~mod~x = 0$
  \end{theorem}

  \pause
  \begin{proof}
    \begin{itemize}
      \item assume: $3~|~(a-2)$
    \end{itemize}

    \vspace{-2em}
    \begin{eqnarray*}
      & \Rightarrow & \exists k \in \mathbb{Z}~[a-2 = 3k]\\\pause
      & \Rightarrow & a+1 = a-2 + 3 = 3k+3 = 3(k+1)\\\pause
      & \Rightarrow & a^2-1 = (a+1)(a-1) = 3(k+1)(a-1)
    \end{eqnarray*}
  \end{proof}
\end{frame}

%\subsection{Indirect Proof}

\begin{frame}
  \frametitle{Indirect Proof}

  \begin{block}{indirect proof}
    \begin{tabular}{ll}
      to prove: & $\forall x~[p(x) \rightarrow q(x)]$\\
      show:     & $\forall x~[\neg q(x) \vdash \neg p(x)]$
    \end{tabular}
  \end{block}
\end{frame}

\begin{frame}
  \frametitle{Indirect Proof Example}

  \begin{theorem}
    $\forall x,y \in \mathbb{N}~[x \cdot y > 25
      \rightarrow (x > 5) \vee (y > 5)]$
  \end{theorem}

  \pause
  \begin{proof}
    \begin{itemize}
      \item assume: $\neg ((x > 5) \vee (y > 5))$
    \end{itemize}

    \vspace{-2em}
    \begin{eqnarray*}
      & \Rightarrow & (0 \leq x \leq 5) \wedge (0 \leq y \leq 5)\\\pause
      & \Rightarrow & x \cdot y \leq 5 \cdot 5 = 25
    \end{eqnarray*}
  \end{proof}
\end{frame}

\begin{frame}
  \frametitle{Indirect Proof Example}

  \begin{theorem}
    $\forall a,b \in \mathbb{N}$\\
      $\exists k \in \mathbb{N}~[ab=2k] \rightarrow
        (\exists i \in \mathbb{N}~[a=2i]) \vee
        (\exists j \in \mathbb{N}~[b=2j])$
  \end{theorem}

  \pause
  \begin{proof}
    \begin{itemize}
      \item assume: $(\neg \exists i \in \mathbb{N}~[a=2i])
                          \wedge (\neg \exists j \in \mathbb{N}~[b=2j])$
    \end{itemize}

    \vspace{-2em}
    \begin{eqnarray*}
      & \Rightarrow & (\exists x \in \mathbb{N}~[a=2x+1])
                          \wedge (\exists y \in \mathbb{N}~[b=2y+1])\\\pause
      & \Rightarrow & ab=(2x+1)(2y+1)\\\pause
      & \Rightarrow & ab=4xy+2x+2y+1\\\pause
      & \Rightarrow & ab=2(2xy+x+y)+1\\\pause
      & \Rightarrow & \neg (\exists k \in \mathbb{N}~[ab=2k])
    \end{eqnarray*}
  \end{proof}
\end{frame}

\subsection{Proof by Contradiction}

\begin{frame}
  \frametitle{Proof by Contradiction}

  \begin{block}{proof by contradiction}
    \begin{tabular}{ll}
      to prove: & $P$\\
      show:     & $\neg P \vdash Q \wedge \neg Q$
    \end{tabular}
  \end{block}
\end{frame}

\begin{frame}
  \frametitle{Proof by Contradiction Example}

  \begin{theorem}
    There is no largest prime number.
  \end{theorem}

  \pause
  \begin{proof}
    \begin{itemize}
      \item assume: There is a largest prime number.

      \pause
      \item $Q$: The largest prime number is $s$.

      \pause
      \item prime numbers: $2,3,5,7,11,\dots,s$

      \pause
      \item let $z = 2 \cdot 3 \cdot 5 \cdot 7 \cdot 11 \cdots s + 1$\\
      \item $z$ is not divisible by any prime number in the range $[2, s]$
    \end{itemize}

    \pause
    \begin{enumerate}
      \item either $z$ is a prime number (note that $z > s$): $\neg Q$

      \pause
      \item or $z$ is divisible by a prime number $t$ ($t > s$): $\neg Q$
    \end{enumerate}
  \end{proof}
\end{frame}

\begin{frame}
  \frametitle{Proof by Contradiction Example}

  \begin{theorem}
    $\neg \exists a,b \in \mathbb{Z}^+~[\sqrt{2}=\frac{a}{b}]$
  \end{theorem}

  \pause
  \begin{proof}
    \begin{itemize}
      \item assume: $\exists a,b \in \mathbb{Z}^+~[\sqrt{2}=\frac{a}{b}]$
      \item $Q$: $gcd(a,b)=1$
    \end{itemize}

    \pause
    \vspace{-0.7cm}
    \begin{columns}[t]
      \column{.5\textwidth}
      \begin{eqnarray*}
        & \Rightarrow & 2 = \frac{a^2}{b^2}\\\pause
        & \Rightarrow & a^2 = 2b^2\\\pause
        & \Rightarrow & \exists i \in \mathbb{Z}^+~[a^2=2i]\\\pause
        & \Rightarrow & \exists j \in \mathbb{Z}^+~[a=2j]
      \end{eqnarray*}

      \pause
      \column{.5\textwidth}
      \begin{eqnarray*}
        & \Rightarrow & 4j^2 = 2b^2\\\pause
        & \Rightarrow & b^2 = 2j^2\\\pause
        & \Rightarrow & \exists k \in \mathbb{Z}^+~[b^2=2k]\\\pause
        & \Rightarrow & \exists l \in \mathbb{Z}^+~[b=2l]\\\pause
        & \Rightarrow & gcd(a,b) \geq 2: \neg Q
      \end{eqnarray*}
    \end{columns}
  \end{proof}
\end{frame}

\begin{frame}
  \frametitle{Proof by Contradiction Example}

  \begin{theorem}
    $0.\overline{9} = 1$
  \end{theorem}

  \pause
  \begin{proof}
    \begin{itemize}
      \item assume: $0.\overline{9} < 1$
      \item let $x = \frac{0.\overline{9} + 1}{2}$
      \item $Q$: $0.\overline{9} < x < 1$
      \item what digit other than $9$ can $x$ contain?
    \end{itemize}
  \end{proof}
\end{frame}
%
% \subsection{Proof of Equivalence}
%
% \begin{frame}
%   \frametitle{Proof of Equivalence}
%
%   \begin{itemize}
%     \item to prove $P \Leftrightarrow Q$, both $P \Rightarrow Q$ and
%       $Q \Rightarrow P$ must be proven
%
%     \pause
%     \medskip
%     \item a method to prove
%       $P_1 \Leftrightarrow P_2 \Leftrightarrow \cdots \Leftrightarrow P_n$:\\
%       $P_1 \Rightarrow P_2 \Rightarrow \cdots \Rightarrow P_n \Rightarrow P_1$
%   \end{itemize}
% \end{frame}
%
% \begin{frame}
%   \frametitle{Equivalence Proof Example}
%
%   \begin{theorem}
%     $a,b,n,q_1,r_1,q_2,r_2 \in \mathbb{Z}^+$\\
%     $a = q_1 \cdot n + r_1$\\
%     $b = q_2 \cdot n + r_2$\\
%
%     \bigskip
%     $r_1 = r_2 \Leftrightarrow n | (a - b)$
%   \end{theorem}
% \end{frame}
%
% \begin{frame}
%   \frametitle{Equivalence Proof Example}
%
%   \begin{columns}[t]
%     \column{.55\textwidth}
%     \begin{proof}[$r_1 = r_2 \Rightarrow n | (a - b)$]
%       \begin{eqnarray*}
%         a - b & = & (q_1 \cdot n + r_1)\\
%               &   & -(q_2 \cdot n + r_2)\\\pause
%               & = & (q_1 - q_2) \cdot n\\
%               &   & + (r_1 - r_2)\\\pause
%         r_1 = r_2 & \Rightarrow & r_1 - r_2 = 0\\\pause
%                   & \Rightarrow & a - b = (q_1 - q_2) \cdot n
%       \end{eqnarray*}
%     \end{proof}
%
%     \pause
%     \column{.45\textwidth}
%     \begin{proof}[$n | (a - b) \Rightarrow r_1 = r_2$]
%       \begin{eqnarray*}
%         a - b & = & (q_1 \cdot n + r_1)\\
%               &   & -(q_2 \cdot n + r_2)\\\pause
%               & = & (q_1 - q_2) \cdot n\\
%               &   & + (r_1 - r_2)\\\pause
%         n | (a - b) & \Rightarrow & r_1 - r_2 = 0\\\pause
%                     & \Rightarrow & r_1 = r_2
%       \end{eqnarray*}
%     \end{proof}
%   \end{columns}
% \end{frame}

\subsection{Induction}

\begin{frame}
  \frametitle{Induction}

  \begin{definition}
    $S(n)$: a predicate defined on $n \in \mathbb{Z}^+$

    \pause
    \medskip
    $S(n_0) \wedge (\forall k \geq n_0~[S(k) \rightarrow S(k+1)])
      \Rightarrow \forall n \geq n_0~S(n)$
  \end{definition}

  \pause
  \medskip
  \begin{itemize}
    \item $S(n_0)$: \emph{base step}
    \item $\forall k \geq n_0~[S(k) \rightarrow S(k+1)]$: \emph{induction step}
  \end{itemize}
\end{frame}

\begin{frame}
  \frametitle{Induction}

  \begin{center}
    \pgfuseimage{induction}
  \end{center}
\end{frame}

\begin{frame}
  \frametitle{Induction Example}

  \begin{theorem}
    $\forall n \in \mathbb{Z}^+~[1+3+5+\cdots+(2n-1)=n^2]$
  \end{theorem}

  \pause
  \begin{proof}
    \begin{itemize}
      \item $n=1$: $1=1^2$

      \pause
      \item $n=k$: assume $1+3+5+\cdots+(2k-1)=k^2$

      \pause
      \item $n=k+1$:
      \begin{eqnarray*}
        &   & 1+3+5+\cdots+(2k-1)+(2k+1)\\\pause
        & = & k^2+2k+1\\\pause
        & = & (k+1)^2
      \end{eqnarray*}
    \end{itemize}
  \end{proof}
\end{frame}

\begin{frame}
  \frametitle{Induction Example}

  \begin{theorem}
    $\forall n \in \mathbb{Z}^+, n \geq 4~[2^n < n!]$
  \end{theorem}

  \pause
  \begin{proof}
    \begin{itemize}
      \item $n=4$: $2^4=16<24=4!$

      \pause
      \item $n=k$: assume $2^k < k!$

      \pause
      \item $n=k+1$:\\
        $2^{k+1} = 2 \cdot 2^k < 2 \cdot k! < (k+1) \cdot k! = (k+1)!$
    \end{itemize}
  \end{proof}
\end{frame}

\begin{frame}
  \frametitle{Induction Example}

  \begin{theorem}
    $\forall n \in \mathbb{Z}^+, n \geq 14~\exists i,j \in \mathbb{N}~[n=3i+8j]$
  \end{theorem}

  \pause
  \begin{proof}
    \begin{itemize}
      \item $n=14$: $14=3 \cdot 2 + 8 \cdot 1$

      \pause
      \item $n=k$: assume $k=3i+8j$

      \pause
      \item $n=k+1$:
      \begin{itemize}
        \item $k=3i+8j, j>0 \Rightarrow k+1=k-8+3 \cdot 3$\\
          $\Rightarrow k+1=3(i+3)+8(j-1)$
        \item $k=3i+8j, j=0, i \geq 5 \Rightarrow k+1=k-5 \cdot 3+2 \cdot 8$\\
          $\Rightarrow k+1=3(i-5)+8(j+2)$
      \end{itemize}
    \end{itemize}
  \end{proof}
\end{frame}

\begin{frame}
  \frametitle{Strong Induction}

  \begin{definition}
    $S(n_0) \wedge
      (\forall k \geq n_0~[(\forall i \leq k~S(i)) \rightarrow S(k+1)])
      \Rightarrow \forall n \geq n_0~S(n)$
  \end{definition}
\end{frame}

\begin{frame}
  \frametitle{Strong Induction Example}

  \begin{theorem}
    $\forall n \in \mathbb{Z}^+, n \geq 2$\\
      n can be written as the product of prime numbers.
  \end{theorem}

  \pause
  \begin{proof}
    \begin{itemize}
      \item $n=2$: $2=2$

      \pause
      \item assume that the theorem is true for $\forall i \leq k$

      \pause
      \item $n=k+1$:
      \begin{enumerate}
        \item if $n$ is prime: $n=n$

        \pause
        \item if $n$ is not prime: $n = u \cdot v$\\
          $u \leq k \Rightarrow$ $u = u_1 \cdot u_2 \cdots$
            ~~~where $u_1, u_2, \ldots$ are prime\\
          $v \leq k \Rightarrow$ $v = v_1 \cdot v_2 \cdots$
            ~~~where $v_1, v_2, \ldots$ are prime\\
          $n = u_1 \cdot u_2 \cdots v_1 \cdot v_2 \cdots$
      \end{enumerate}
    \end{itemize}
  \end{proof}
\end{frame}

\begin{frame}
  \frametitle{Strong Induction Example}

  \begin{theorem}
    $\forall n \in \mathbb{Z}^+, n \geq 14~\exists i,j \in \mathbb{N}~[n=3i+8j]$
  \end{theorem}

  \pause
  \begin{proof}
    \begin{itemize}
      \item $n=14$: $14=3 \cdot 2 + 8 \cdot 1$\\
        $n=15$: $15=3 \cdot 5 + 8 \cdot 0$\\
        $n=16$: $16=3 \cdot 0 + 8 \cdot 2$

      \pause
      \item $n \leq k$: assume $k=3i+8j$

      \pause
      \item $n=k+1$: $k+1=(k-2)+3$
    \end{itemize}
  \end{proof}
\end{frame}

% \begin{frame}
%   \frametitle{Flawed Induction Example}
%
%   \begin{theorem}
%     $\forall n \in \mathbb{Z}^+~[1+2+3+\cdots+n=\frac{n^2+n+2}{2}]$
%   \end{theorem}
%
%   \pause
%   \begin{block}{invalid base step}
%     \begin{itemize}
%       \item $n=k$: assume $1+2+3+\cdots+k=\frac{k^2+k+2}{2}$
%
%       \pause
%       \item $n=k+1$:
%       \begin{eqnarray*}
%         &   & 1+2+3+\cdots+k+(k+1)\\\pause
%         & = & \frac{k^2+k+2}{2}+k+1
%           =   \frac{k^2+k+2}{2}+\frac{2k+2}{2}\\\pause
%         & = & \frac{k^2+3k+4}{2}
%           =   \frac{(k+1)^2+(k+1)+2}{2}
%       \end{eqnarray*}
%
%       \pause
%       \item $n=1$: $1 \neq \frac{1^2+1+2}{2}=2$
%     \end{itemize}
%   \end{block}
% \end{frame}
%
% \begin{frame}
%   \frametitle{Flawed Induction Example}
%
%   \begin{center}
%     \pgfuseimage{error1}
%   \end{center}
% \end{frame}
%
% \begin{frame}
%   \frametitle{Flawed Induction Example}
%
%   \begin{theorem}
%     All horses are of the same color.\\
%     \pause
%     \bigskip
%     $A(n)$: All horses in sets of $n$ horses are of the same color.
%
%     \medskip
%     $\forall n \in \mathbb{N^+}~A(n)$
%   \end{theorem}
% \end{frame}
%
% \begin{frame}
%   \frametitle{Flawed Induction Example}
%
%   \begin{block}{invalid induction step}
%     \begin{itemize}
%       \item $n=1$: $A(1)$\\
%         All horses in sets of $1$~horse are of the same color.
%
%       \pause
%       \medskip
%       \item $n=k$: assume $A(k)$ is true\\
%         All horses in sets of $k$~horses are of the same color.
%       \pause
%       \medskip
%       \item $A(k+1)=\{a_1,a_2,\dots,a_k\} \cup \{a_2,a_3,\dots,a_{k+1}\}$
%       \begin{itemize}
%         \item All horses in set $\{a_1,a_2,\dots,a_k\}$
%           are of the same color ($a_2$).
%         \item All horses in set $\{a_2,a_3,\dots,a_{k+1}\}$
%           are of the same color ($a_2$).
%       \end{itemize}
%     \end{itemize}
%   \end{block}
% \end{frame}
%
% \begin{frame}
%   \frametitle{Flawed Induction Examples}
%
%   \begin{center}
%     \pgfuseimage{error2}
%   \end{center}
% \end{frame}

\subsection*{References}

\begin{frame}
  \frametitle{References}

  \begin{block}{Required reading: Grimaldi}
    \begin{itemize}
      \item Chapter 2: Fundamentals of Logic
      \begin{itemize}
        \item 2.4. \alert{The Use of Quantifiers}
        \item 2.5. \alert{Quantifiers, Definitions, and the Proofs of Theorems}
      \end{itemize}
      \item Chapter 4: Properties of Integers: Mathematical Induction
      \begin{itemize}
        \item 4.1. \alert{The Well-Ordering Principle: Mathematical Induction}
      \end{itemize}
    \end{itemize}
  \end{block}
\end{frame}

\end{document}
