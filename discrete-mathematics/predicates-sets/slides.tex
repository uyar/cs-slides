% Copyright (c) 2001-2011
%       H. Turgut Uyar <uyar@itu.edu.tr>
%       Ayşegül Gençata Yayımlı <gencata@itu.edu.tr>
%       Emre Harmancı <harmanci@itu.edu.tr>
%
% These notes are licensed using the
% "Creative Commons Attribution-NonCommercial-ShareAlike License".
% You are free to copy, distribute and transmit the work, and to adapt the work
% as long as you attribute the authors, do not use it for commercial purposes,
% and any derivative work is under the same or a similar license.
%
% Read the full legal code at:
% http://creativecommons.org/licenses/by-nc-sa/3.0/

\documentclass[dvipsnames]{beamer}

\usepackage{ae}
\usepackage[T1]{fontenc}
\usepackage[utf8]{inputenc}
\setbeamertemplate{navigation symbols}{}

\mode<presentation>
{
  \usetheme{Rochester}
  \setbeamercovered{transparent}
}

\title{Discrete Mathematics}
\subtitle{Predicates and Sets}

\author{H. Turgut Uyar \and Ayşegül Gençata Yayımlı \and Emre Harmancı}
\date{2001-2011}

\AtBeginSubsection[]
{
  \begin{frame}<beamer>
    \frametitle{Topics}
    \tableofcontents[currentsection,currentsubsection]
  \end{frame}
}

%\beamerdefaultoverlayspecification{<+->}

\pgfdeclareimage[width=2cm]{license}{../../license}

\begin{document}

\begin{frame}
  \titlepage
\end{frame}

\begin{frame}
  \frametitle{License}

  \pgfuseimage{license}\hfill
  \copyright 2001-2011 T. Uyar, A. Yayımlı, E. Harmancı

  \vfill
  \begin{tiny}
    You are free:
    \begin{itemize}
      \item to Share — to copy, distribute and transmit the work
      \item to Remix — to adapt the work
    \end{itemize}

    Under the following conditions:
    \begin{itemize}
      \item Attribution — You must attribute the work in the manner specified by
        the author or licensor (but not in any way that suggests that they
        endorse you or your use of the work).

      \item Noncommercial — You may not use this work for commercial purposes.

      \item Share Alike — If you alter, transform, or build upon this work, you
        may distribute the resulting work only under the same or similar license
        to this one.
    \end{itemize}
  \end{tiny}

  \vfill
  Legal code (the full license):\\
  \url{http://creativecommons.org/licenses/by-nc-sa/3.0/}
\end{frame}

\begin{frame}
  \frametitle{Topics}
  \tableofcontents
\end{frame}

\section{Predicates}

\subsection{Introduction}

\begin{frame}
  \frametitle{Predicate}

  \begin{definition}
    \alert{predicate} (or \alert{open statement}): a statement which
    \begin{itemize}
      \item contains one or more variables
      \item is not a statement, but
      \item becomes a statement when the variables in it are replaced by certain
        allowable choices
    \end{itemize}
  \end{definition}
\end{frame}

\begin{frame}
  \frametitle{Universe of Discourse}

  \begin{definition}
    \alert{universe of discourse}: $\mathcal{U}$\\
      set of allowable choices
  \end{definition}

  \pause
  \begin{itemize}
    \item example universes:
    \begin{itemize}
      \item $\mathbb{Z}$: integers
      \item $\mathbb{N}$: naturel numbers
      \item $\mathbb{Z}^+$: positive integers
      \item $\mathbb{Q}$: rational numbers
      \item $\mathbb{R}$: real numbers
      \item $\mathbb{C}$: complex numbers
    \end{itemize}
  \end{itemize}
\end{frame}

\begin{frame}
  \frametitle{Predicate Examples}

  \begin{example}
    $\mathcal{U} = \mathbb{N}$\\
    $p(x)$: $x+2$ is an even integer

    \bigskip
    $p(5)$: $F$\\
    $p(8)$: $T$

    \pause
    \bigskip
    $\neg p(x)$: $x+2$ is not an even integer
  \end{example}

  \pause
  \begin{example}
    $\mathcal{U} = \mathbb{N}$\\
    $q(x,y)$: $x+y$ and $x-2y$ are even integers

    \bigskip
    $q(11,3)$: $F$, $q(14,4)$: $T$
  \end{example}
\end{frame}

\subsection{Quantifiers}

\begin{frame}
  \frametitle{Quantifiers}

  \begin{columns}[t]
    \column{.48\textwidth}
    \begin{definition}
      \alert{existential quantifier}:\\
        predicate is true for some values

      \begin{itemize}
        \item symbol: $\exists$
        \item read: \emph{there exists}

        \pause
        \medskip
        \item symbol: $\exists!$
        \item read: \emph{there exists only one}
      \end{itemize}
    \end{definition}

    \pause
    \column{.48\textwidth}
    \begin{definition}
      \alert{universal quantifier}:\\
        predicate is true for all values

      \begin{itemize}
        \item symbol: $\forall$
        \item read: \emph{for all}
      \end{itemize}
    \end{definition}
  \end{columns}
\end{frame}

\begin{frame}
  \frametitle{Quantifiers}

  \begin{block}{existential quantifier}
    $\mathcal{U} = \{x_1,x_2,\cdots,x_n\}$\\
    $\exists x~p(x) \equiv p(x_1) \vee p(x_2) \vee \cdots \vee p(x_n)$

    \begin{itemize}
      \item \emph{$p(x)$ is true for some $x$}
    \end{itemize}
  \end{block}

  \pause
  \begin{block}{universal quantifier}
    $\mathcal{U} = \{x_1,x_2,\cdots,x_n\}$\\
    $\forall x~p(x) \equiv p(x_1) \wedge p(x_2) \wedge \cdots \wedge p(x_n)$

    \begin{itemize}
      \item \emph{$p(x)$ is true for all $x$}
    \end{itemize}
  \end{block}
\end{frame}

\begin{frame}
  \frametitle{Quantifier Examples}

  \begin{example}
    \begin{columns}[t]
      \column{.5\textwidth}
      $\mathcal{U} = \mathbb{R}$\\

      given the definitions below

      \begin{itemize}
        \item $p(x): x \geq 0$
        \item $q(x): x^2 \geq 0$
        \item $r(x): (x-4) (x+1) = 0$
        \item $s(x): x^2 -3 > 0$
      \end{itemize}

      what are the values of the statements on the right?

      \column{.4\textwidth}
      \begin{itemize}
        \pause
        \item $\exists x~[p(x) \wedge r(x)]$

        \pause
        \item $\forall x~[p(x) \rightarrow q(x)]$

        \pause
        \item $\forall x~[q(x) \rightarrow s(x)]$

        \pause
        \item $\forall x~[r(x) \vee s(x)]$

        \pause
        \item $\forall x~[r(x) \rightarrow p(x)]$
      \end{itemize}
    \end{columns}
  \end{example}
\end{frame}

\begin{frame}
  \frametitle{Negating Quantifiers}

  \begin{itemize}
    \item replace $\forall$ with $\exists$, and $\exists$ with $\forall$
    \item negate the predicate
  \end{itemize}

  \pause
  \begin{eqnarray*}
    \neg \exists x~p(x)      & \Leftrightarrow & \forall x~\neg p(x)\\
    \neg \exists x~\neg p(x) & \Leftrightarrow & \forall x~p(x)\\
    \neg \forall x~p(x)      & \Leftrightarrow & \exists x~\neg p(x)\\
    \neg \forall x~\neg p(x) & \Leftrightarrow & \exists x~p(x)
  \end{eqnarray*}
\end{frame}

\begin{frame}
  \frametitle{Negating Quantifiers}

  \begin{theorem}
    $\neg \exists x~p(x) \Leftrightarrow \forall x~\neg p(x)$
  \end{theorem}

  \pause
  \begin{proof}
    \begin{eqnarray*}
      \neg \exists x~p(x) & \equiv          & \neg [p(x_1) \vee p(x_2) \vee \cdots
                                              \vee p(x_n)]\\\pause
                          & \Leftrightarrow & \neg p(x_1) \wedge \neg p(x_2) \wedge \cdots
                                              \wedge \neg p(x_n)\\\pause
                          & \equiv          & \forall x~\neg p(x)
    \end{eqnarray*}
  \end{proof}
\end{frame}

\begin{frame}
  \frametitle{Predicate Equivalences}

  \begin{theorem}
    $\exists x~[p(x) \vee q(x)]
      \Leftrightarrow \exists x~p(x) \vee \exists x~q(x)$
  \end{theorem}

  \pause
  \begin{theorem}
    $\forall x~[p(x) \wedge q(x)]
      \Leftrightarrow \forall x~p(x) \wedge \forall x~q(x)$
  \end{theorem}
\end{frame}

\begin{frame}
  \frametitle{Predicate Implications}

  \begin{theorem}
    $\forall x~p(x) \Rightarrow \exists x~p(x)$
  \end{theorem}

  \pause
  \begin{theorem}
    $\exists x~[p(x) \wedge q(x)]
      \Rightarrow \exists x~p(x) \wedge \exists x~q(x)$
  \end{theorem}

  \pause
  \begin{theorem}
    $\forall x~p(x) \vee \forall x~q(x)
      \Rightarrow \forall x~[p(x) \vee q(x)]$
  \end{theorem}
\end{frame}

\subsection{Multiple Quantifiers}

\begin{frame}
  \frametitle{Multiple Quantifiers}

  \begin{itemize}
    \item $\exists x \exists y~p(x,y)$
    \item $\forall x \exists y~p(x,y)$
    \item $\exists x \forall y~p(x,y)$
    \item $\forall x \forall y~p(x,y)$
  \end{itemize}
\end{frame}

\begin{frame}
  \frametitle{Multiple Quantifier Examples}

  \begin{example}
    $\mathcal{U}=\mathbb{Z}$\\
    $p(x,y): x+y=17$

    \begin{itemize}
      \pause
      \item $\forall x \exists y~p(x,y)$:\\
        for every $x$ there exists a $y$ such that $x+y=17$

      \pause
      \item $\exists y \forall x~p(x,y)$:\\
        there exists a $y$ so that for all $x$,  $x+y=17$

      \pause
      \bigskip
      \item what if $\mathcal{U}=\mathbb{N}$?
    \end{itemize}
  \end{example}
\end{frame}

\begin{frame}
  \frametitle{Multiple Quantifiers}

  \begin{example}
    $\mathcal{U}_x = \{1,2\} \wedge \mathcal{U}_y = \{A,B\}$

    \pause
    \begin{eqnarray*}
      \exists x \exists y~p(x,y) & \equiv & [p(1,A) \vee p(1,B)]
                                       \vee [p(2,A) \vee p(2,B)]\\\pause
      \exists x \forall y~p(x,y) & \equiv & [p(1,A) \wedge p(1,B)]
                                       \vee [p(2,A) \wedge p(2,B)]\\\pause
      \forall x \exists y~p(x,y) & \equiv & [p(1,A) \vee p(1,B)]
                                     \wedge [p(2,A) \vee p(2,B)]\\\pause
      \forall x \forall y~p(x,y) & \equiv & [p(1,A) \wedge p(1,B)]
                                     \wedge [p(2,A) \wedge p(2,B)]
    \end{eqnarray*}
  \end{example}
\end{frame}

\subsection*{References}

\begin{frame}
  \frametitle{References}

  \begin{block}{Required Text: Grimaldi}
    \begin{itemize}
      \item Chapter 2: Fundamentals of Logic
      \begin{itemize}
        \item 2.4. \alert{The Use of Quantifiers}
      \end{itemize}
    \end{itemize}
  \end{block}

  \begin{block}{Supplementary Text: O'Donnell, Hall, Page}
    \begin{itemize}
      \item Chapter 7: Predicate Logic
    \end{itemize}
  \end{block}
\end{frame}

\section{Sets}

\subsection{Introduction}

\begin{frame}
  \frametitle{Set}

  \begin{definition}
    \alert{set}: a collection of elements that are
    \begin{itemize}
      \item distinct
      \item unsorted
      \item not repeating
    \end{itemize}
  \end{definition}
\end{frame}

\begin{frame}
  \frametitle{Set Representation}

  \begin{itemize}
    \item \emph{explicit representation}\\
      elements are listed within braces: $\{a_1,a_2,\dots,a_n\}$

    \pause
    \medskip
    \item \emph{implicit representation}\\
      elements that validate a predicate: $\{x | x \in G, p(x)\}$

    \pause
    \medskip
    \item $\emptyset$: empty set

    \pause
    \medskip
    \item let $S$ be a set, and $a$ be an element:
    \begin{itemize}
      \item $a \in S$: $a$ is an element of set $S$
      \item $a \notin S$: $a$ is not an element of set $S$
    \end{itemize}
  \end{itemize}
\end{frame}

\begin{frame}
  \frametitle{Explicit Representation Examples}

  \begin{example}
    $\{3,8,2,11,5\}$\\
    $11 \in \{3,8,2,11,5\}$
  \end{example}
\end{frame}

\begin{frame}
  \frametitle{Implicit Representation Examples}

  \begin{example}
    $\{ x | x \in \mathbb{Z}^+, 20 < x^3 < 100 \} \equiv \{3,4\}$

    $\{ 2x-1 | x \in \mathbb{Z}^+, 20 < x^3 < 100 \} \equiv \{5,7\}$
  \end{example}

  \pause
  \begin{example}
    $A = \{ x | x \in \mathbb{R}, 1 \leq x \leq 5 \}$
  \end{example}

  \pause
  \begin{example}
    $E = \{ n | n \in \mathbb{N}, \exists k \in \mathbb{N}~[n=2k] \}$\\
    $A = \{ x | x \in E, 1 \leq x \leq 5 \}$
  \end{example}
\end{frame}

\begin{frame}
  \frametitle{Set Dilemma}

  \begin{itemize}
    \item There is a barber who lives in a small town.\\
      The barber shaves all those men who do not shave themselves.\\
      He does not shave those men who do shave themselves.

    \begin{quote}
      Does the barber shave himself?
    \end{quote}

    \pause
    \item \emph{no}: the barber shaves all those men who do not shave themselves
      $\rightarrow$ yes, he does

    \pause
    \item \emph{yes}: he does not shave those men who do shave themselves
      $\rightarrow$ no, he doesn't
  \end{itemize}
\end{frame}

\begin{frame}
  \frametitle{Set Dilemma}

  \begin{itemize}
    \item $S$ is a set of sets

    \pause
    \item set of sets that are not the element of themselves:\\
      $S = \{ A | A \notin A \}$

    \pause
    \begin{quote}
      is $S$ an element of itself?
    \end{quote}

    \pause
    \item \emph{yes}: the statement is not valid $\rightarrow$ no

    \pause
    \item \emph{no}: the statement is valid $\rightarrow$ yes
  \end{itemize}
\end{frame}

\begin{frame}
  \frametitle{Finite Set}

  \begin{definition}
    \alert{countable set}:\\
      a set whose elements are enumerable

    \begin{itemize}
      \item the set $\mathbb{R}$ in uncountable
    \end{itemize}
  \end{definition}

  \pause
  \begin{definition}
    \alert{finite set}:\\
      a set which is countable and has finite number of elements

    \begin{itemize}
      \item the set $\mathbb{N}$ is countable but infinite
      \item number of elements: \alert{cardinality}, shown as: $|S|$
    \end{itemize}
  \end{definition}
\end{frame}

\subsection{Subset}

\begin{frame}
  \frametitle{Subset}

  \begin{definition}
    $A \subseteq B \Leftrightarrow \forall x~[x \in A \rightarrow x \in B]$
  \end{definition}

  \pause
  \begin{itemize}
    \item \alert{set equality}:\\
      $A = B \Leftrightarrow (A \subseteq B) \wedge (B \subseteq A)$

    \pause
    \item \alert{proper subset}:\\
      $A \subset B \Leftrightarrow (A \subseteq B) \wedge (A \neq B)$

    \pause
    \item $\forall S~[\emptyset \subseteq S]$
  \end{itemize}
\end{frame}

\begin{frame}
  \frametitle{Subset}

  \begin{block}{not a subset}
    \begin{eqnarray*}
      A \nsubseteq B & \Leftrightarrow
                     & \neg \forall x~[x \in A \rightarrow x \in B]\\\pause
                     & \Leftrightarrow
                     & \exists x~\neg [x \in A \rightarrow x \in B]\\\pause
                     & \Leftrightarrow
                     & \exists x~\neg [\neg (x \in A) \vee (x \in B)]\\\pause
                     & \Leftrightarrow
                     & \exists x~[(x \in A) \wedge \neg (x \in B)]\\\pause
                     & \Leftrightarrow
                     & \exists x~[(x \in A) \wedge (x \notin B)]
    \end{eqnarray*}
  \end{block}
\end{frame}

\begin{frame}
  \frametitle{Power Set}

  \begin{definition}
    \alert{power set}:\\
      set of all subsets of a set, including empty set and the set itself

    \begin{itemize}
      \item shown as: $\mathcal{P}(S)$
    \end{itemize}
  \end{definition}

  \pause
  \begin{itemize}
    \item if a set has $n$ elements, its power set has $2^n$ elements
  \end{itemize}
\end{frame}

\begin{frame}
  \frametitle{Example of Power Set}

  \begin{example}
    \begin{eqnarray*}
      \mathcal{P}(\{1,2,3\}) & = \{ &\\
                             &      & \emptyset\\
                             &      & \{1\},\{2\},\{3\}\\
                             &      & \{1,2\},\{1,3\},\{2,3\}\\
                             &      & \{1,2,3\}\\
                             &   \} &
    \end{eqnarray*}
  \end{example}
\end{frame}

\subsection{Set Operations}

\begin{frame}
  \frametitle{Set Operations}

  \begin{block}{complement}
    $\overline{A} = \{ x | x \notin A \} $
  \end{block}

  \pause
  \begin{block}{intersection}
    $A \cap B = \{ x | (x \in A) \wedge (x \in B) \}$

    \pause
    \begin{itemize}
      \item if $A \cap B = \emptyset$ then $A$ and $B$ are \alert{disjoint}
    \end{itemize}
  \end{block}

  \pause
  \begin{block}{union}
    $A \cup B = \{ x | (x \in A) \vee (x \in B) \}$
  \end{block}
\end{frame}

\begin{frame}
  \frametitle{Set Operations}

  \begin{block}{difference}
    $A - B = \{ x | (x \in A) \wedge (x \notin B) \}$

    \pause
    \begin{itemize}
      \item $A-B = A \cap \overline{B}$

      \pause
      \item \emph{symmetric difference}:\\
        $A \bigtriangleup B = \{ x | (x \in A \cup B)
                              \wedge (x \notin A \cap B) \}$
    \end{itemize}
  \end{block}
\end{frame}

\begin{frame}
  \frametitle{Cartesian Product}

  \begin{definition}
    \alert{cartesian product}:\\
      $A \times B = \{ (a,b) | a \in A, b \in B \}$

      \pause
      \medskip
      $A \times B \times C \dots \times N =
        \{ (a,b,\dots,n) | a \in A, b \in B, \dots, n \in N \}$
  \end{definition}
\end{frame}

\begin{frame}
  \frametitle{Cartesian Product Example}

  \begin{example}
    $A = \{a_1.a_2,a_3,a_4\}$

    $B = \{b_1,b_2,b_3\}$

    \medskip
    \begin{eqnarray*}
      A \times B & = \{ & \\
                 &      & (a_1,b_1),(a_1,b_2),(a_1,b_3),\\
                 &      & (a_2,b_1),(a_2,b_2),(a_2,b_3),\\
                 &      & (a_3,b_1),(a_3,b_2),(a_3,b_3),\\
                 &      & (a_4,b_1),(a_4,b_2),(a_4,b_3)\\
                 &  \}  &
    \end{eqnarray*}
  \end{example}
\end{frame}

\begin{frame}
  \frametitle{Equivalences}

  \begin{tabular}{ll}
    \alert{double complement} &\\
      $\overline{\overline{A}} = A$\\\\
    \pause
    \alert{commutative law} &\\
      $A \cap B = B \cap A$ &
      $A \cup B = B \cup A$\\\\
    \pause
    \alert{associative law} &\\
      $(A \cap B) \cap C = A \cap (B \cap C)$ &
      $(A \cup B) \cup C = A \cup (B \cup C)$\\\\
    \pause
    \alert{idempotent law} &\\
      $A \cap A = A$ &
      $A \cup A = A$\\\\
    \pause
    \alert{inverse law} &\\
      $A \cap \overline{A} = \emptyset$ &
      $A \cup \overline{A} = U$\\\\
  \end{tabular}
\end{frame}

\begin{frame}
  \frametitle{Equivalences}

  \begin{tabular}{ll}
    \alert{identity law} &\\
      $A \cap U = A$ &
      $A \cup \emptyset = A$\\\\
    \pause
    \alert{dominance law} &\\
      $A \cap \emptyset = \emptyset$ &
      $A \cup U = U$\\\\
    \pause
    \alert{distributive law} &\\
      $A \cap (B \cup C) = (A \cap B) \cup (A \cap C)$ &
      $A \cup (B \cap C) = (A \cup B) \cap (A \cup C)$\\\\
    \pause
    \alert{absorption law} &\\
      $A \cap (A \cup B) = A$ &
      $A \cup (A \cap B) = A$\\\\
    \pause
    \alert{DeMorgan's laws} &\\
      $\overline{A \cap B} = \overline{A} \cup \overline{B}$ &
      $\overline{A \cup B} = \overline{A} \cap \overline{B}$\\\\
  \end{tabular}
\end{frame}

\begin{frame}
  \frametitle{DeMorgan's Laws}

  \begin{proof}
    \begin{eqnarray*}
      \overline{A \cap B} & = & \{x | x \notin (A \cap B)\}\\\pause
                          & = & \{x | \neg (x \in (A \cap B))\}\\\pause
                          & = & \{x | \neg ((x \in A) \wedge (x \in B))\}\\\pause
                          & = & \{x | \neg (x \in A) \vee \neg (x \in B)\}\\\pause
                          & = & \{x | (x \notin A) \vee (x \notin B)\}\\\pause
                          & = & \{x | (x \in \overline{A}) \vee (x \in \overline{B})\}\\\pause
                          & = & \{x | x \in \overline{A} \cup \overline{B}\}\\\pause
                          & = & \overline{A} \cup \overline{B}
    \end{eqnarray*}
  \end{proof}
\end{frame}

\begin{frame}
  \frametitle{Example of Equivalence}

  \begin{theorem}
    $A \cap (B-C) = (A \cap B) - (A \cap C)$
  \end{theorem}
\end{frame}

\begin{frame}
  \frametitle{Equivalence Example}

  \begin{proof}
    \begin{eqnarray*}
      (A \cap B) - (A \cap C)
          & = & (A \cap B) \cap \overline{(A \cap C)}\\\pause
          & = & (A \cap B) \cap (\overline{A} \cup \overline{C})\\\pause
          & = & ((A \cap B) \cap \overline{A}) \cup ((A \cap B) \cap \overline{C}))\\\pause
          & = & \emptyset \cup ((A \cap B) \cap \overline{C}))\\\pause
          & = & (A \cap B) \cap \overline{C}\\\pause
          & = & A \cap (B \cap \overline{C})\\\pause
          & = & A \cap (B - C)
    \end{eqnarray*}
  \end{proof}
\end{frame}

\subsection{Inclusion-Exclusion}

\begin{frame}
  \frametitle{Principle of Inclusion-Exclusion}

  \begin{itemize}
    \item $|A \cup B| = |A| + |B| - |A \cap B|$

    \pause
    \item $|A \cup B \cup C| = |A| + |B| + |C|
      - (|A \cap B| + |A \cap C| + |B \cap C|)
      + |A \cap B \cap C|$
  \end{itemize}

  \pause
  \begin{theorem}
    \begin{eqnarray*}
      |A_1 \cup A_2 \cup \cdots \cup A_n| & = & \sum_i{|A_i|}
          - \sum_{i,j}{|A_i \cap A_j|}\\
      & & + \sum_{i,j,k}{|A_i \cap A_j \cap A_k|}\\
      & & \cdots + -1^{n-1} {|A_i \cap A_j \cap \cdots \cap A_n|}
    \end{eqnarray*}
  \end{theorem}
\end{frame}

\begin{frame}
  \frametitle{Inclusion-Exclusion Example}

  \begin{example}[sieve of Eratosthenes]
    \begin{itemize}
      \item a method to identify prime numbers
    \end{itemize}

    \pause
    \begin{tiny}
    \begin{tabular}{ccccccccccccccccccccccc}
  2 &  3 &  4 &  5 &  6 &  7 &  8 &  9 & 10 & 11 & 12 & 13 & 14 & 15 & 16 & 17\\
 18 & 19 & 20 & 21 & 22 & 23 & 24 & 25 & 26 & 27 & 28 & 29 & 30\\
\\ \pause
  2 &  3 &    &  5 &    &  7 &    &  9 &    & 11 &    & 13 &    & 15 &    & 17\\
    & 19 &    & 21 &    & 23 &    & 25 &    & 27 &    & 29 & \\
\\  \pause
  2 &  3 &    &  5 &    &  7 &    &    &    & 11 &    & 13 &    &    &    & 17\\
    & 19 &    &    &    & 23 &    & 25 &    &    &    & 29 & \\
\\  \pause
  2 &  3 &    &  5 &    &  7 &    &    &    & 11 &    & 13 &    &    &    & 17\\
    & 19 &    &    &    & 23 &    &    &    &    &    & 29 & \\
    \end{tabular}
    \end{tiny}
  \end{example}
\end{frame}

\begin{frame}
  \frametitle{Inclusion-Exclusion Example}

  \begin{example}[sieve of Eratosthenes]
    \begin{itemize}
      \item number of primes between 1 and 100
      \medskip

      \pause
      \item numbers that are not divisible by 2, 3, 5 and 7
      \begin{itemize}
        \item $A_2$: set of numbers divisible by 2
        \item $A_3$: set of numbers divisible by 3
        \item $A_5$: set of numbers divisible by 5
        \item $A_7$: set of numbers divisible by 7
      \end{itemize}

      \pause
      \item $|A_2 \cup A_3 \cup A_5 \cup A_7|$
    \end{itemize}
  \end{example}
\end{frame}

\begin{frame}
  \frametitle{Inclusion-Exclusion Example}

  \begin{example}[sieve of Eratosthenes]
    \begin{columns}[t]
      \column{.5\textwidth}
      \begin{itemize}
        \item $|A_2| = \left\lfloor 100/2 \right\rfloor = 50$
        \item $|A_3| = \left\lfloor 100/3 \right\rfloor = 33$
        \item $|A_5| = \left\lfloor 100/5 \right\rfloor = 20$
        \item $|A_7| = \left\lfloor 100/7 \right\rfloor = 14$
      \end{itemize}

      \pause
      \column{.5\textwidth}
      \begin{itemize}
        \item $|A_2 \cap A_3| = \left\lfloor 100/6  \right\rfloor = 16$
        \item $|A_2 \cap A_5| = \left\lfloor 100/10 \right\rfloor = 10$
        \item $|A_2 \cap A_7| = \left\lfloor 100/14 \right\rfloor = 7$
        \item $|A_3 \cap A_5| = \left\lfloor 100/15 \right\rfloor = 6$
        \item $|A_3 \cap A_7| = \left\lfloor 100/21 \right\rfloor = 4$
        \item $|A_5 \cap A_7| = \left\lfloor 100/35 \right\rfloor = 2$
      \end{itemize}
    \end{columns}
  \end{example}
\end{frame}

\begin{frame}
  \frametitle{Inclusion-Exclusion Example}

  \begin{example}[sieve of Eratosthenes]
    \begin{itemize}
      \item $|A_2 \cap A_3 \cap A_5| = \left\lfloor 100/30  \right\rfloor = 3$
      \item $|A_2 \cap A_3 \cap A_7| = \left\lfloor 100/42  \right\rfloor = 2$
      \item $|A_2 \cap A_5 \cap A_7| = \left\lfloor 100/70  \right\rfloor = 1$
      \item $|A_3 \cap A_5 \cap A_7| = \left\lfloor 100/105 \right\rfloor = 0$
    \end{itemize}

    \pause
    \begin{itemize}
      \item $|A_2 \cap A_3 \cap A_5 \cap A_7| = \left\lfloor 100/210 \right\rfloor = 0$
    \end{itemize}
  \end{example}
\end{frame}

\begin{frame}
  \frametitle{Inclusion-Exclusion Example}

  \begin{example}[sieve of Eratosthenes]
    \begin{eqnarray*}
      |A_2 \cup A_3 \cup A_5 \cup A_7| & = & (50 + 33 + 20 +14)\\
                                       & - & (16 + 10 + 7 + 6 + 4 + 2)\\
                                       & + & (3 + 2 + 1 + 0)\\
                                       & - & (0)\\
                                       & = & 78
    \end{eqnarray*}

    \pause
    \begin{itemize}
      \item number of primes: $(100 - 78) + 4 - 1 = 25$
    \end{itemize}
  \end{example}
\end{frame}

\subsection*{References}

\begin{frame}
  \frametitle{References}

  \begin{block}{Required Text: Grimaldi}
    \begin{itemize}
      \item Chapter 3: Set Theory
      \begin{itemize}
        \item 3.1. \alert{Sets and Subsets}
        \item 3.2. \alert{Set Operations and the Laws of Set Theory}
      \end{itemize}

      \item Chapter 8: The Principle of Inclusion and Exclusion
      \begin{itemize}
        \item 8.1. \alert{The Principle of Inclusion and Exclusion}
      \end{itemize}
    \end{itemize}
  \end{block}

  \begin{block}{Supplementary Text: O'Donnell, Hall, Page}
    \begin{itemize}
      \item Chapter 8: Set Theory
    \end{itemize}
  \end{block}
\end{frame}

\end{document}
