% Copyright (c) 2005-2012
%       H. Turgut Uyar <uyar@itu.edu.tr>
%       Şule Gündüz Öğüdücü <sgunduz@itu.edu.tr>
%
% These notes are licensed using the
% "Creative Commons Attribution-NonCommercial-ShareAlike License".
% You are free to copy, distribute and transmit the work, and to adapt the work
% as long as you attribute the authors, do not use it for commercial purposes,
% and any derivative work is under the same or a similar license.
%
% Read the full legal code at:
% http://creativecommons.org/licenses/by-nc-sa/3.0/

\documentclass[dvipsnames]{beamer}

\usepackage{ae}
\usepackage[T1]{fontenc}
\usepackage[utf8]{inputenc}
\usepackage[turkish]{babel}
\setbeamertemplate{navigation symbols}{}

\usepackage{listings}
\lstset{basicstyle=\ttfamily, keywordstyle=\color{blue}, showstringspaces=false}
\lstset{language=Java}

\mode<presentation> {
  \usetheme{Warsaw}
  \usecolortheme[named=ForestGreen]{structure}
  \setbeamercovered{transparent}
}

\title{Veri Tabanı Sistemleri}
\subtitle{Nesne Veri Tabanları}

\author{H. Turgut Uyar \and Şule Öğüdücü}
\date{2005-2012}

\AtBeginSubsection[]{
  \begin{frame}<beamer>
    \frametitle{Konular}
    \tableofcontents[currentsection,currentsubsection]
  \end{frame}
}

\theoremstyle{definition}
\newtheorem{tanim}[theorem]{Tanım}

\theoremstyle{example}
\newtheorem{ornek}[theorem]{Örnek}

\theoremstyle{plain}

\pgfdeclareimage[width=2cm]{license}{../../license}

\pgfdeclareimage{nesne}{nesne}
\pgfdeclareimage{oid}{oid}

\begin{document}

\begin{frame}
  \titlepage
\end{frame}

\begin{frame}
  \frametitle{License}

  \pgfuseimage{license}\hfill
  \copyright 2005-2012 T. Uyar, Ş. Öğüdücü

  \vfill
  \begin{tiny}
    You are free:
    \begin{itemize}
      \item to Share -- to copy, distribute and transmit the work
      \item to Remix -- to adapt the work
    \end{itemize}

    Under the following conditions:
    \begin{itemize}
      \item Attribution -- You must attribute the work in the manner specified by
        the author or licensor (but not in any way that suggests that they
        endorse you or your use of the work).

      \item Noncommercial -- You may not use this work for commercial purposes.

      \item Share Alike -- If you alter, transform, or build upon this work, you
        may distribute the resulting work only under the same or similar license
        to this one.
    \end{itemize}
  \end{tiny}

  \vfill
  Legal code (the full license):\\
  \url{http://creativecommons.org/licenses/by-nc-sa/3.0/}
\end{frame}

\begin{frame}
  \frametitle{Konular}
  \tableofcontents
\end{frame}

\section{Nesne Veri Tabanları}

\subsection{Giriş}

\begin{frame}
  \frametitle{Nesne Modeli}

  \begin{itemize}
    \item veri modeli ile yazılım modeli arasında uyumsuzluk
    \begin{itemize}
      \item veride: bağıntı, çoklu, dış anahtar, \ldots
      \item yazılımda: nesne, yöntem, \ldots
    \end{itemize}
  \end{itemize}
\end{frame}

\begin{frame}[fragile]
  \frametitle{Model Farkı Örneği}

  \begin{ornek}[filme oyuncu ekleme - SQL tanımları]
    \begin{lstlisting}[language=SQL]
CREATE TABLE MOVIE (ID INTEGER PRIMARY KEY,
    TITLE VARCHAR(80) NOT NULL)

CREATE TABLE PERSON (ID INTEGER PRIMARY KEY,
    NAME VARCHAR(40) NOT NULL)

CREATE TABLE CASTING(
    MOVIEID INTEGER REFERENCES MOVIE,
    ACTORID INTEGER REFERENCES PERSON,
    PRIMARY KEY (MOVIEID, ACTORID)
)
    \end{lstlisting}
  \end{ornek}
\end{frame}

\begin{frame}[fragile]
  \frametitle{Model Farkı Örneği}

  \begin{ornek}[filme oyuncu ekleme - SQL işlemleri]
    \begin{lstlisting}[language=SQL]
INSERT INTO MOVIE (ID, TITLE)
  VALUES (110, 'Sleepy Hollow')

INSERT INTO PERSON (ID, NAME)
  VALUES (26, 'Johnny Depp')

INSERT INTO CASTING (MOVIEID, ACTORID)
  VALUES (110, 26)
    \end{lstlisting}
  \end{ornek}
\end{frame}

\begin{frame}[fragile]
  \frametitle{Model Farkı Örneği}

  \begin{ornek}[filme oyuncu ekleme - Java tanımları]
    \begin{lstlisting}
public class Movie {
    ...
    private List<Person> cast;

    ...
    public void addActor(Person p) {
        this.cast.add(p);
    }
}
    \end{lstlisting}
  \end{ornek}
\end{frame}

\begin{frame}[fragile]
  \frametitle{Düzey Farkı Örneği}

  \begin{ornek}[filme oyuncu ekleme - Java işlemleri]
    \begin{lstlisting}
Movie m = new Movie("Sleepy Hollow", ...);
Person p = new Person("Johnny Depp", ...);
m.addActor(p);
    \end{lstlisting}
  \end{ornek}
\end{frame}

\subsection{Nesne Kimliği}

\begin{frame}
  \frametitle{Nesne Kimliği}

  \begin{itemize}
    \item her nesnenin bir nesne kimliği var
    \begin{itemize}
      \item özellikler değişse bile kimlik aynı kalır
    \end{itemize}

    \pause
    \item birincil anahtardan farklı
    \begin{itemize}
      \item birincil anahtar görünür (kullanıcı tanımlar)
      \item birincil anahtardaki değerler değişebilir
    \end{itemize}

    \pause
    \item programlama dillerindeki referanslara karşı düşer
    \begin{itemize}
      \item nesne kimliği ile başka nesnelere başvurulabilir:\\
        \emph{içerme hiyerarşisi}
    \end{itemize}
  \end{itemize}
\end{frame}

\begin{frame}
  \frametitle{İçerme Hiyerarşisi Örneği}

  \begin{ornek}
    \begin{center}
      \pgfuseimage{nesne}
    \end{center}
  \end{ornek}
\end{frame}

\begin{frame}
  \frametitle{İçerme Hiyerarşisi Örneği}

  \begin{ornek}
    \begin{center}
      \pgfuseimage{oid}
    \end{center}
  \end{ornek}
\end{frame}

\begin{frame}
  \frametitle{Nesne Veri Tabanları}

  \begin{itemize}
    \item uygulamanın kalıcı nesneleri veri tabanında\\
      bağıntılar değil nesneler halinde saklanır

    \medskip
    \item yazma: nesne $\rightarrow$ iç format (\alert{serileştirme})
    \item okuma: iç format $\rightarrow$ nesne
  \end{itemize}
\end{frame}

\subsection{Örnek: db4o}

\begin{frame}
  \frametitle{db4o}

  \begin{itemize}
    \item gömülü çalışabilen bir nesne veri tabanı sistemi

    \pause
    \medskip
    \item koşulla sorgulama
    \item örneğe göre sorgulama
    \begin{itemize}
      \item sorgulanacak sınıftan bir nesne yaratılır
      \item istenen nitelikler ayarlanır, diğerleri boş bırakılır
      \item bu nesneye benzer nesneler aratılır
    \end{itemize}

    \pause
    \medskip
    \item güncellenecek ya da silinecek nesnelerin\\
      veri tabanından çekilmiş olması gerek (nesne kimliği)
  \end{itemize}
\end{frame}

\begin{frame}
  \frametitle{db4o Arayüzü}

  \begin{itemize}
    \item veri tabanına bağlantı (gömülü kip):\\
      \lstinline!Db4oEmbedded.openFile(filePath)!
        $\rightarrow$ \lstinline!ObjectContainer!

    \pause
    \medskip
    \item ekleme ve güncelleme:\\
      \lstinline!ObjectContainer.store(object)!
    \item silme:\\
      \lstinline!ObjectContainer.delete(object)!
  \end{itemize}
\end{frame}

\begin{frame}
  \frametitle{db4o Arayüzü}

  \begin{itemize}
    \item bir sınıftan bütün nesneler:\\
      \lstinline!ObjectContainer.query(Class.class)!
       $\rightarrow$ \lstinline!List<Class>!

    \pause
    \medskip
    \item örneğe göre sorgulama:\\
      \lstinline!ObjectContainer.queryByExample(Class prototype)!\\
       $\rightarrow$ \lstinline!ObjectSet<Class>!
  \end{itemize}
\end{frame}

\begin{frame}
  \frametitle{db4o Arayüzü}

  \begin{itemize}
    \item sorgulama koşulu: \lstinline!Predicate<Class>!
    \item bu sınıfın \lstinline!match! yöntemi gerçeklenir:\\
      \lstinline!public boolean match(Class object)!

    \pause
    \medskip
    \item sorgulama:\\
      \lstinline!ObjectContainer.query(Predicate<Class> predicate)!\\
      $\rightarrow$ \lstinline!List<Class>!
  \end{itemize}
\end{frame}

\begin{frame}[fragile]
  \frametitle{db4o Örnekleri}

  \begin{ornek}[veri tabanına bağlantı]
    \begin{lstlisting}
ObjectContainer db = Db4oEmbedded.openFile(
    "imdb.db4o"
);
    \end{lstlisting}
  \end{ornek}
\end{frame}

\begin{frame}[fragile]
  \frametitle{db4o Örnekleri}

  \begin{ornek}[sorgulama: bütün filmler]
    \begin{lstlisting}
List<Movie> movies = db.query(Movie.class);
for (Movie movie : movies) {
    ...
}
    \end{lstlisting}
  \end{ornek}
\end{frame}

\begin{frame}[fragile]
  \frametitle{db4o Örnekleri}

  \begin{ornek}[örneğe göre sorgulama: 1977'de çekilmiş filmler]
    \begin{lstlisting}
Movie prototype = new Movie(null);
prototype.setYear(1977);
ObjectSet<Movie> movies =
    db.queryByExample(prototype);
while (movies.hasNext()) {
    Movie m = movies.next();
    ...
}
    \end{lstlisting}
  \end{ornek}
\end{frame}

\begin{frame}[fragile]
  \frametitle{db4o Örnekleri}

  \begin{ornek}[koşulla sorgulama: 1977'den sonra çekilmiş filmler]
    \begin{lstlisting}
List<Movie> movies = db.query(
    new Predicate<Movie>() {
        public boolean match(Movie movie) {
            return movie.getYear() > 1977;
        }
});
    \end{lstlisting}
  \end{ornek}
\end{frame}

\begin{frame}[fragile]
  \frametitle{db4o Örnekleri}

  \begin{ornek}[ekleme]
    \begin{lstlisting}
Movie m = new Movie("Casablanca");
m.setYear(1942);
db.store(m);
db.commit();
    \end{lstlisting}
  \end{ornek}
\end{frame}

\begin{frame}[fragile]
  \frametitle{db4o Örnekleri}

  \begin{ornek}[güncelleme]
    \begin{lstlisting}
Movie prototype = new Movie("Casablanca");
ObjectSet<Movie> result =
    db.queryByExample(prototype);
Movie found = result.next();
found.setYear(1943);
db.store(found);
db.commit();
    \end{lstlisting}
  \end{ornek}
\end{frame}

\begin{frame}[fragile]
  \frametitle{db4o Örnekleri}

  \begin{ornek}[silme]
    \begin{lstlisting}
Movie prototype = new Movie("Casablanca");
ObjectSet<Movie> result =
    db.queryByExample(prototype);
Movie found = result.next();
db.delete(found);
db.commit();
    \end{lstlisting}
  \end{ornek}
\end{frame}

\subsection*{Kaynaklar}

\begin{frame}
  \frametitle{Kaynaklar}

  \begin{block}{Okunacak: Date}
    \begin{itemize}
      \item Chapter 25: \alert{Object Databases}
    \end{itemize}
  \end{block}
\end{frame}

\section{Nesne-Bağıntı Veri Tabanları}

\subsection{Giriş}

\begin{frame}
  \frametitle{Nesne-Bağıntı Eşleştirmesi}

  \begin{itemize}
    \item yazılım nesneye dayalı
    \item veri tabanı bağıntılar şeklinde
    \item yazılım bileşenleri veri tabanı bileşenleriyle eşleştirilir
  \end{itemize}
\end{frame}

\subsection{Örnek: Persist}

\begin{frame}
  \frametitle{Örnek: Persist}

  \begin{itemize}
    \item bir JDBC bağlantısını sarmalıyor
    \item nesne veri tabanı arayüzünü JDBC komutlarına çeviriyor
  \end{itemize}
\end{frame}

\begin{frame}
  \frametitle{Persist Arayüzü}

  \begin{itemize}
    \item veri tabanına bağlantı: \lstinline!Connection connection!\\
      \lstinline!Persist(connection)! $\rightarrow$ \lstinline!Persist!

    \pause
    \medskip
    \item ekleme:\\
      \lstinline!Persist.insert(object)!
    \item güncelleme:\\
      \lstinline!Persist.update(object)!
    \item silme:\\
      \lstinline!Persist.delete(object)!
  \end{itemize}
\end{frame}

\begin{frame}
  \frametitle{Persist Arayüzü}

  \begin{itemize}
    \item sorgulama: bir sınıftan bütün nesneler\\
      \lstinline!Persist.readList(Class.class)!
       $\rightarrow$ \lstinline!List<Class>!
    \item SQL ile sorgulama: hazır komut yazımına benzer şekilde\\
      \lstinline!Persist.readList(Class.class, String query, params)!\\
       $\rightarrow$ \lstinline!List<Class>!
  \end{itemize}
\end{frame}

\begin{frame}[fragile]
  \frametitle{Persist Örnekleri}

  \begin{ornek}[veri tabanına bağlantı]
    \begin{lstlisting}
Connection connection =
    DriverManager.getConnection(jdbcURL);
Persist db = new Persist(connection);
    \end{lstlisting}
  \end{ornek}
\end{frame}

\begin{frame}[fragile]
  \frametitle{Persist Örnekleri}

  \begin{ornek}[sorgulama: bütün filmler]
    \begin{lstlisting}
List<Movie> movies = db.readList(Movie.class);
for (Movie movie : movies) {
    ...
}
    \end{lstlisting}
  \end{ornek}
\end{frame}

\begin{frame}[fragile]
  \frametitle{Persist Örnekleri}

  \begin{ornek}[SQL ile sorgulama: 1977'de çekilmiş filmler]
    \begin{lstlisting}
List<Movie> movies = db.readList(Movie.class,
    "SELECT * FROM MOVIE WHERE (YEAR = ?)",
    1977);
    \end{lstlisting}
  \end{ornek}
\end{frame}

\begin{frame}[fragile]
  \frametitle{Persist Örnekleri}

  \begin{ornek}[ekleme]
    \begin{lstlisting}
Movie m = new Movie("Casablanca");
m.setYear(1942);
db.insert(m);
    \end{lstlisting}
  \end{ornek}
\end{frame}

\begin{frame}[fragile]
  \frametitle{Persist Örnekleri}

  \begin{ornek}[güncelleme]
    \begin{lstlisting}
List<Movie> movies = db.readList(Movie.class,
    "SELECT * FROM MOVIE WHERE (TITLE = ?)",
    "Casablanca");
Movie found = movies.get(0);
found.setYear(1943);
db.update(found);
    \end{lstlisting}
  \end{ornek}
\end{frame}

\begin{frame}[fragile]
  \frametitle{Persist Örnekleri}

  \begin{ornek}[silme]
    \begin{lstlisting}
List<Movie> movies = db.readList(Movie.class,
    "SELECT * FROM MOVIE WHERE (TITLE = ?)",
    "Casablanca");
Movie found = movies.get(0);
db.delete(found);
    \end{lstlisting}
  \end{ornek}
\end{frame}

\subsection*{Kaynaklar}

\begin{frame}
  \frametitle{Kaynaklar}

  \begin{block}{Okunacak: Date}
    \begin{itemize}
      \item Chapter 26: \alert{Object/Relational Databases}
    \end{itemize}
  \end{block}
\end{frame}

% TODO: JPA

\section{XML Veri Tabanları}

\subsection{Giriş}

\begin{frame}
  \frametitle{XML}

  \begin{itemize}
    \item XML kendisi bir dil değil
    \begin{itemize}
      \item dil tanımlama kuralları
    \end{itemize}

    \pause
    \item XML ile tanımlanmış diller
    \begin{itemize}
      \item XHTML, DocBook, SVG, MathML, ...
    \end{itemize}

    \pause
    \item XML ile bağlantılı diller
    \begin{itemize}
      \item XPath, XQuery, XSL Transforms, SOAP, XLink, ...
    \end{itemize}
  \end{itemize}
\end{frame}

\begin{frame}
  \frametitle{XML Yapısı}

  \begin{itemize}
    \item ağaç yapısı

    \item düğümler: \emph{eleman}
    \begin{itemize}
      \item kök düğümü: \emph{belge elemanı}
      \item yapraklar: \emph{karakter verileri}, boş elemanlar
    \end{itemize}

    \pause
    \medskip
    \item açılış-kapanış takıları
    \item nitelikler
  \end{itemize}
\end{frame}

\begin{frame}[fragile]
  \frametitle{XML Örneği}

  \begin{ornek}[HTML]
    \begin{lstlisting}[language=XML]
<html>
<head><title>Foo Bar</title></head>
<body>
  <h1>Welcome to Foo Bar!</h1>
  <p>You can get more information from the
    <a href="http://www.foobar.net/">
      foobar page</a>.</p>
  <img src="logo.jpg" alt="Foo Bar logo" />
</body>
</html>
    \end{lstlisting}
  \end{ornek}
\end{frame}

\begin{frame}[fragile]
  \frametitle{XML Örneği}

  \begin{ornek}[DocBook]
    \begin{lstlisting}[language=XML]
<book lang="tr">
  <title>Foobar Report</title>
  <bookinfo>...</bookinfo>
  <chapter>...</chapter>
  <chapter>...</chapter>
  ...
</book>
    \end{lstlisting}
  \end{ornek}
\end{frame}

\begin{frame}[fragile]
  \frametitle{XML Örneği}

  \begin{ornek}[DocBook]
    \begin{lstlisting}[language=XML]
  <bookinfo>
    <author>
      <firstname>John</firstname>
      <surname>Foobar</surname>
    </author>
    <date>2007</date>
  </bookinfo>
    \end{lstlisting}
  \end{ornek}
\end{frame}

\begin{frame}[fragile]
  \frametitle{XML Örneği}

  \begin{ornek}[DocBook]
    \begin{lstlisting}[language=XML]
  <chapter>
    <title>Introduction</title>
    <section>
      <title>Description</title>
      <para>Foobar is ...</para>
    </section>
    ...
  </chapter>
    \end{lstlisting}
  \end{ornek}
\end{frame}

\begin{frame}[fragile]
  \frametitle{XML Örneği}

  \begin{ornek}[filmler]
    \begin{lstlisting}[language=XML]
<movies>
  <movie color="Color">
    <title>Usual Suspects</title>
    ...
  </movie>
  <movie color="Color">
    <title>Being John Malkovich</title>
    ...
  </movie>
  ...
</movies>
    \end{lstlisting}
  \end{ornek}
\end{frame}

\begin{frame}[fragile]
  \frametitle{XML Örneği}

  \begin{ornek}[filmler]
    \begin{lstlisting}[language=XML]
  <movie color="Color">
    <title>Usual Suspects</title>
    <year>1995</year>
    <score>8.7</score>
    <votes>35027</votes>
    <director>Bryan Singer</director>
    <cast>
      <actor>Gabriel Byrne</actor>
      <actor>Benicio Del Toro</actor>
    </cast>
  </movie>
    \end{lstlisting}
  \end{ornek}
\end{frame}

\subsection{XQuery}

\begin{frame}
  \frametitle{XQuery}

  \begin{itemize}
    \item XPath: XML belgelerinden düğüm ve veri çekme
    \item XQuery: XPath + güncelleme işlemleri
  \end{itemize}
\end{frame}

\begin{frame}
  \frametitle{XPath}

  \begin{itemize}
    \item aranılan düğümlerin yolu: yol adımları zinciri
    \begin{itemize}
      \item kök düğümünden başlanarak (mutlak)
      \item bulunulan düğümden başlanarak (bağıl)

      \medskip
      \item yol adımları \lstinline!/! işaretleriyle ayrılır
    \end{itemize}

    \pause
    \begin{ornek}
      \begin{itemize}
       \item \lstinline!/movies/movie!
       \item \lstinline!cast/actor! veya \lstinline!./cast/actor!
       \item \lstinline!../../year!
      \end{itemize}
    \end{ornek}
  \end{itemize}
\end{frame}

\begin{frame}
  \frametitle{Yol Adımları}

  \begin{itemize}
    \item yol adımı yapısı:\\
      \lstinline!axis::node_selector[predicate]!

    \pause
    \medskip
    \item eksen: nerede aranacak?
    \item seçici: ne aranacak?
    \item yüklem: hangi koşula uyanlar aranacak?
  \end{itemize}
\end{frame}

\begin{frame}
  \frametitle{Eksenler}

  \begin{itemize}
    \item \lstinline!child!:
      bütün çocuklarda, bir düzey (varsayılan eksen)
    \item \lstinline!descendant!:
      bütün çocuklarda, rekürsif olarak\\
      (kısa gösterilim: \lstinline!//!)
    \item \lstinline!parent!:
      anne düğümde, bir düzey
    \item \lstinline!ancestor!:
      bütün anne düğümlerde, köke kadar
    \item \lstinline!attribute!: niteliklerde (kısa gösterilim: \lstinline!@!)
    \item \lstinline!following-sibling!:
      sonra gelen kardeşlerde
    \item \lstinline!preceding-sibling!:
      önce gelen kardeşlerde
    \item ...
  \end{itemize}
\end{frame}

\begin{frame}
  \frametitle{Düğüm Seçiciler}

  \begin{itemize}
    \item düğüm takısı
    \item düğüm niteliği
    \item düğüm metni: \lstinline!text()!
    \item düğümün bütün çocukları: \lstinline!*!
  \end{itemize}
\end{frame}

\begin{frame}
  \frametitle{XPath Örnekleri}

  \begin{ornek}
    \begin{itemize}
      \item bütün yönetmenlerin isimleri:\\
        \lstinline!/movies/movie/director/text()!\\
        \lstinline!//director/text()!

      \pause
      \item bu filmdeki bütün oyuncular:\\
        \lstinline!./cast/actor!\\
        \lstinline!.//actor!

      \pause
      \item bütün filmlerin renkleri:\\
        \lstinline!//movie/@color!

      \pause
      \item bundan sonraki filmlerin puanları:\\
        \lstinline!./following-sibling::movie/score!
    \end{itemize}
  \end{ornek}
\end{frame}

\begin{frame}
  \frametitle{XPath Yüklemleri}

  \begin{itemize}
    \item düğüm sırası sınaması: \lstinline![position]!

    \pause
    \medskip
    \item çocuk varlığı sınaması: \lstinline![child_tag]!
    \item çocuk değeri sınaması: \lstinline![child_tag="value"]!

    \pause
    \medskip
    \item nitelik varlığı sınaması: \lstinline![@attribute]!
    \item nitelik değeri sınaması: \lstinline![@attribute="value"]!
  \end{itemize}
\end{frame}

\begin{frame}
  \frametitle{XPath Örnekleri}

  \begin{ornek}
    \begin{itemize}
      \item birinci filmin başlığı:\\
        \lstinline!/movies/movie[1]/title/text()!

      \pause
      \item 1997 yılında çekilmiş filmler:\\
        \lstinline!//movie[year="1997"]!

      \pause
      \item siyah-beyaz filmler:\\
        \lstinline!//movie[@color="BW"]!
    \end{itemize}
  \end{ornek}
\end{frame}

\subsection{Örnek: DBXML}

\begin{frame}
  \frametitle{Örnek: Oracle Berkeley DBXML}

  \begin{itemize}
    \item gömülü XML veri tabanı
    \item XML belgeleri tutuyor
    \item XQuery ile veri üzerinde işlem yapılabiliyor
    \item kendi istemcisi ile kullanılabiliyor
    \item çeşitli programlama dilleri için arayüzleri var
  \end{itemize}
\end{frame}

\begin{frame}
  \frametitle{DBXML Arayüzü}

  \begin{itemize}
    \item veri tabanı yaratma:\\
    \begin{itemize}
      \item \lstinline!XmlManager! nesnesi yaratılır
      \item \lstinline!XmlManager.createContainer(name)!
        $\rightarrow$ \lstinline!XmlContainer!
      \item belge elemanı (kök) yerleştirilir:\\
        \lstinline!XmlContainer.putDocument(namespace, xml_string,!\\
        ~~~~~~~~~~~~~~~~~~~~~~~~~~~~~~~~~~~~~~~\lstinline!configuration)!
    \end{itemize}

    \pause
    \item var olan veri tabanına bağlanma:\\
    \begin{itemize}
      \item \lstinline!XmlManager! nesnesi yaratılır
      \item \lstinline?XmlManager.existsContainer(name) != 0? ise
      \item \lstinline!XmlManager.openContainer(name)!
        $\rightarrow$ \lstinline!XmlContainer!
    \end{itemize}
  \end{itemize}
\end{frame}

\begin{frame}
  \frametitle{DBXML Arayüzü}

  \begin{itemize}
    \item \lstinline!XmlManager.createQueryContext()!
      $\rightarrow$ \lstinline!XmlQueryContext!
    \item \lstinline!XmlQueryContext.setNamespace(namespace, URL)!

    \pause
    \item sorgu katarı: \lstinline!collection(name)/xpath_expression!
    \item sorgunun çalıştırılması:\\
      \lstinline!XmlManager.query(query, context)!
      $\rightarrow$ \lstinline!XmlResults!

    \pause
    \medskip
    \item \lstinline!XmlResults! yineleyicisinin her bir elemanı bir
      \lstinline!XmlValue!
    \item \lstinline!getFirstChild()!, \lstinline!getLastChild()!,
      \lstinline!getNextSibling()!, ...
    \item karakter verisi: \lstinline!getNodeValue()!
      $\rightarrow$ \lstinline!String!
    \item nitelikler:
      \lstinline!XmlValue.getAttributes()! $\rightarrow$ \lstinline!XmlResults!
  \end{itemize}
\end{frame}

\begin{frame}[fragile]
  \frametitle{DBXML Örnekleri}

  \begin{ornek}[veri tabanına bağlantı]
    \begin{lstlisting}
db = new XmlManager();
XmlContainer container = null;
if (db.existsContainer("imdb.dbxml") != 0)
  container = db.openContainer("imdb.dbxml");
else {
  container = db.createContainer("imdb.dbxml");
  container.putDocument("movies",
      "<movies />",
      (XmlDocumentConfig) null);
}
    \end{lstlisting}
  \end{ornek}
\end{frame}

\begin{frame}[fragile]
  \frametitle{DBXML Örnekleri}

  \begin{ornek}[film nesnesini XML katarına çevirme]
    \begin{lstlisting}
public static String toXml(Movie movie) {
  StringBuffer buffer = new StringBuffer();
  buffer.append("<movie>");
  buffer.append("<title>"
    + movie.getTitle() + "</title>");
  buffer.append("<year>"
    + movie.getYear().toString() + "</year>");
  buffer.append("</movie>");
  return buffer.toString();
}
    \end{lstlisting}
  \end{ornek}
\end{frame}

\begin{frame}[fragile]
  \frametitle{DBXML Örnekleri}

  \begin{ornek}[XML düğümünü film nesnesine çevirme]
    \begin{lstlisting}
private static Movie fromNode(XmlValue node)
        throws XmlException {
  XmlValue tn = node.getFirstChild();
  String title =
      tn.getFirstChild().getNodeValue();
  XmlValue yn = tn.getNextSibling();
  String yearValue =
      yn.getFirstChild().getNodeValue();
  Integer year = Integer.parseInt(yearValue);
  Movie movie = new Movie(title);
  movie.setYear(year);
  return movie;
}
    \end{lstlisting}
  \end{ornek}
\end{frame}

\begin{frame}[fragile]
  \frametitle{DBXML Örnekleri}

  \begin{ornek}[sorgulama: bütün filmler]
    \begin{lstlisting}
XmlQueryContext context = ...;
context.setNamespace(...);
String query =
  "collection(\"imdb.dbxml\")/movies/movie";
XmlResults results = db.query(query, context);
if (results.hasNext()) {
  XmlValue node = results.next();
  Movie movie = fromNode(node);
  ...
}
results.delete();
    \end{lstlisting}
  \end{ornek}
\end{frame}

\begin{frame}[fragile]
  \frametitle{DBXML Örnekleri}

  \begin{ornek}[ekleme]
    \begin{lstlisting}
Movie m = new Movie("Casablanca");
m.setYear(1942);

XmlQueryContext context = ...;
context.setNamespace(...);
String query = "insert nodes " + toXml(m)
  + " into collection(\"imdb.dbxml\")/movies";
XmlResults results = db.query(query, context);
results.delete();
    \end{lstlisting}
  \end{ornek}
\end{frame}

\subsection*{Kaynaklar}

\begin{frame}
  \frametitle{Kaynaklar}

  \begin{block}{Okunacak: Date}
    \begin{itemize}
      \item Chapter 27: \alert{The World Wide Web and XML}
    \end{itemize}
  \end{block}
\end{frame}

\end{document}
