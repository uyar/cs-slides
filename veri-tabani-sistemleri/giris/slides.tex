% Copyright (c) 2002-2016
%       H. Turgut Uyar <uyar@itu.edu.tr>
%       Şule Gündüz Öğüdücü <sgunduz@itu.edu.tr>
%
% This work is licensed under a "Creative Commons
% Attribution-NonCommercial-ShareAlike 4.0 International License".
% For more information, please visit:
% https://creativecommons.org/licenses/by-nc-sa/4.0/

\documentclass[dvipsnames]{beamer}

\usepackage{ae}
\usepackage[scaled=0.88]{beramono}
\usepackage[T1]{fontenc}
\usepackage[utf8]{inputenc}
\usepackage[turkish]{babel}
\setbeamersize{text margin left=2em, text margin right=2em}

\mode<presentation>
{
  \usetheme{Warsaw}
  \usecolortheme[named=ForestGreen]{structure}
  \setbeamercovered{transparent}
}

\title{Veri Tabanı Sistemleri}
\subtitle{Giriş}

\author{H. Turgut Uyar \and Şule Öğüdücü}
\date{2002-2016}

\AtBeginSubsection[]{
  \begin{frame}<beamer>
    \frametitle{Konular}
    \tableofcontents[currentsection,currentsubsection]
  \end{frame}
}

%\theoremstyle{definition}
%\newtheorem{tanim}[theorem]{Tanım}

%\theoremstyle{example}
%\newtheorem{ornek}[theorem]{Örnek}

\theoremstyle{plain}

\pgfdeclareimage[width=2cm]{license}{../license}

\pgfdeclareimage{kayit}{kayit}
\pgfdeclareimage{vtys}{vtys}
\pgfdeclareimage{sparc}{sparc}
\pgfdeclareimage{mimari1}{mimari1}
\pgfdeclareimage{mimari2}{mimari2}
\pgfdeclareimage{mimari3}{mimari3}

\begin{document}

\begin{frame}
  \titlepage
\end{frame}

\begin{frame}
  \frametitle{License}

 \pgfuseimage{license}\hfill
  \copyright~2002-2016 T. Uyar, Ş. Öğüdücü

  \vfill
  \begin{footnotesize}
    You are free to:
    \begin{itemize}
      \itemsep0em
      \item Share -- copy and redistribute the material in any medium or format
      \item Adapt -- remix, transform, and build upon the material
    \end{itemize}

    Under the following terms:
    \begin{itemize}
      \itemsep0em
      \item Attribution -- You must give appropriate credit, provide a link to
        the license, and indicate if changes were made.

      \item NonCommercial -- You may not use the material for commercial
        purposes.

      \item ShareAlike -- If you remix, transform, or build upon the material,
        you must distribute your contributions under the same license as the
        original.
    \end{itemize}
  \end{footnotesize}

  \begin{small}
    For more information:\\
    \url{https://creativecommons.org/licenses/by-nc-sa/4.0/}

    \smallskip
    Read the full license:\\
    \url{https://creativecommons.org/licenses/by-nc-sa/4.0/legalcode}
  \end{small}
\end{frame}

\begin{frame}
  \frametitle{Konular}
  \tableofcontents
\end{frame}

\section{Giriş}

\subsection{Problem}

\begin{frame}
  \frametitle{Veri İşleme}

  \begin{itemize}
    \item büyük miktarda verinin etkin biçimde tutulması ve işlenmesi

    \medskip  
    \item yeni veri ekleme
    \item olan verilerde değişiklik
    \item veri silme
    \item sorgulama: planlı - plansız
     
    \pause
    \smallskip    
    \item \alert{CRUD}: create - read - update - delete
  \end{itemize}
\end{frame}


\begin{frame}
  \frametitle{Veri Türleri}

  \begin{itemize}
    \item \alert{kalıcı veriler}:\\
      tutulacak bilginin doğası gereği bulunması zorunlu olan veriler

    \bigskip
    \item geçici veriler
    \smallskip   
      \item çıkış verileri: kalıcı verilerden türetilebilen veriler\\
        (sorgu sonuçları, raporlar v.b.)
      \item giriş verileri: sisteme yeni giren, henüz işlenmemiş veriler
      \begin{itemize}
        \item kalıcı verilere eklenebilir
        \item kalıcı verilerde değişikliklere yol açabilir
        \item hiç kullanılmayabilir
       \end{itemize}
  \end{itemize}
\end{frame}


\begin{frame}
  \frametitle{Örnek: Öğrenci Verileri}

 
    \begin{columns}[t]
      \begin{column}<1->{5.1cm}
      \begin{itemize}
        \item Öğrenci İşleri:\\
          öğrencinin adı, numarası,\\
          bölümü, aldığı dersler,\\
          stajları, \ldots
      \end{itemize}
      \end{column}

      \begin{column}<2->{5.1cm}
      \begin{itemize}
        \item ortak veriler:\\
          öğrencinin adı, numarası,\\
          bölümü, \ldots
      \end{itemize}
      \end{column}
    \end{columns}

    \begin{columns}[t]
      \begin{column}<1->{5.1cm}
      \begin{itemize}
        \item Kitaplık:\\
          öğrencinin adı, numarası,\\
          bölümü, aldığı kitaplar, \ldots
      \end{itemize}
      \end{column}

      \begin{column}<2->{5.1cm}
      \begin{itemize}
        \item uygulamaya özel veriler:\\
          öğrencinin aldığı dersler,\\
          stajlar, kitaplar, \ldots
      \end{itemize}
      \end{column}
    \end{columns}
  
\end{frame}


\subsection{Kayıt Dosyaları}

\begin{frame}
  \frametitle{Kayıt Dosyaları}

  \begin{columns}[b]
    \column{.4\textwidth}
    \begin{center}
      \pgfuseimage{kayit}
    \end{center}

    \column{.6\textwidth}
    \begin{itemize}
      \item her uygulamanın kendi verileri var
      \item her uygulama verilerini\\
	kendi yönettiği dosyalarda tutuyor
    \end{itemize}
  \end{columns}
\end{frame}

%-----------------------------------------

\begin{frame}
  \frametitle{Tekrarlılık}

  \begin{itemize}
    \item aynı veri birden fazla yerde tutuluyor
    \item disk alanı israfı
  \end{itemize}

 \medskip
  \begin{exampleblock}{örnek}
    \begin{itemize}
      \item öğrenci adı, numarası ve bölümü Öğrenci İşleri'nde ayrı,\\
	Kitaplık'ta ayrı tutuluyor
    \end{itemize}
  \end{exampleblock}
\end{frame}

\begin{frame}
  \frametitle{Tutarsızlık}

  \begin{itemize}
    \item birden fazla yerde tutulan veriler farklılık gösterebilir
  \end{itemize}

  \medskip
 \begin{exampleblock}{örnek}
    \begin{itemize}
      \item aynı öğrencinin adı Öğrenci İşleri'nde "Lily Wachowski",\\
        Kitaplık'ta "Andy Wachowski" görünebilir
    \end{itemize}
  \end{exampleblock}
\end{frame}

\begin{frame}
  \frametitle{Bütünlük Bozulması}

  \begin{itemize}
    \item bilginin doğruluğunu sağlamak zordur
  \end{itemize}

  \medskip
  \begin{exampleblock}{örnek}
    \begin{itemize}
      \item "Kontrol ve Bilgisayar Mühendisliği" bölümü kapatılır\\
        ama öğrencilerinin bölüm verisi eskisi gibi kalır
    \end{itemize}
  \end{exampleblock}
\end{frame}

\begin{frame}
  \frametitle{Yeni Uygulamalarda Zorluklar}

  \begin{itemize}
    \item her yeni uygulama için benzer işlerin yeniden yapılması gerekir
  \end{itemize}

  \medskip
  \begin{exampleblock}{örnek}
    \begin{itemize}
      \item Burs İşleri için de uygulama yazılacak
    \end{itemize}
  \end{exampleblock}
\end{frame}

\begin{frame}
  \frametitle{Politika Boşlukları}

  \begin{itemize}
    \item kurum uygulamalarında standart eksikliği
    \item yaklaşım, yöntem, programlama dili farklılıkları
    \item uygulamalar arasında veri alışverişi

    \pause
    \medskip
    \item her birim yalnızca kendi gereksinimlerine göre karar verir
  \end{itemize}
\end{frame}

\begin{frame}
  \frametitle{Güvenlik}

  \begin{itemize}
    \item ayrıntılı güvenlik izinleri tanımlamak zor
    \item güvenlik yalnızca işletim sistemine bağlı
  \end{itemize}
\end{frame}

\begin{frame}
  \frametitle{Veriye Bağımlılık}

  \begin{itemize}
    \item \alert{veriye bağımlılık}: uygulama kodunun\\
       veri düzeni ve \\
       erişim yöntemine bağımlı olması

    \medskip
   \item uygulamada değişiklik yapmak çok zor
    \end{itemize}
\end{frame}

\begin{frame}
  \frametitle{Veriye Bağımlılık Örneği}

  \begin{itemize}
      \item öğrenci numarası Öğrenci İşleri'nde katar, Kitaplık'ta sayı

      \medskip
      \item Öğrenci İşleri kayıtlarında öğrenci numarası için\\
        B-ağacı dizin tutuluyor
      \item arama yapılırken B-ağacı algoritmaları kullanılıyor
      \smallskip 
      
      \item hash dizin kullanılmak istenirse ne olacak?     
      
    \end{itemize}
  \end{frame}

\section{Veri Tabanı Yönetim Sistemleri}

\subsection{Giriş}

\begin{frame}
  \frametitle{Veri Tabanı Yönetim Sistemleri}

  \begin{columns}[b]
    \column{.45\textwidth}
    \begin{center}
      \pgfuseimage{vtys}
    \end{center}

    \column{.55\textwidth}
    \begin{itemize}
      \item veriler ortak bir sistemde\\
        tutuluyor
      \item uygulamalar verilere ortak\\
        bir arayüz üzerinden erişiyor
    \end{itemize}
  \end{columns}
\end{frame}

\begin{frame}
  \frametitle{ANSI/SPARC Mimarisi}

  \begin{center}
    \pgfuseimage{sparc}
  \end{center}
\end{frame}

\begin{frame}
  \frametitle{Dış Düzey}

  \begin{itemize}
    \item son kullanıcı açısından dış düzey:
    \item verinin kendine gereken altkümesi
    \item kullandığı uygulama programının arayüzü
    

    \pause
    \bigskip
    \item uygulama programcısı açısından dış düzey:
    \item kullandığı programlama dili
    \item bu dile veri tabanı işlemleri için yapılan ekler:\\
       \alert{veri altdili}
  
  \end{itemize}
\end{frame}

\begin{frame}
  \frametitle{Kavramsal Düzey}

  \begin{itemize}
    \item kavramsal düzey: verinin bütünü
    \item veriden bağımsızlığın sağlandığı düzey

    \bigskip
    \item \alert{katalog}: verinin içeriğini betimleyen tanımlar
      \item veri tabanları
      \item veri tipleri, bütünlük kısıtlamaları
      \item kullanıcılar, yetkiler, güvenlik kısıtlamaları
  \end{itemize}
\end{frame}

\begin{frame}
  \frametitle{İç Düzey}

  \begin{itemize}
    \item iç düzey: gerçekleme ayrıntıları

    \medskip
    \item verinin nasıl temsil edildiği:
    \medskip
      \item dosyalar, kayıtlar
    
    \smallskip
    \item veriye nasıl erişileceği
    \item işaretçiler, dizinler, B-ağaçları
  \end{itemize}
\end{frame}

\begin{frame}
  \frametitle{Dönüşümler}

  \begin{itemize}
    \item veri bağımsızlığı için düzeyler arasında dönüşümler
  \end{itemize}

  \medskip
  \begin{exampleblock}{örnek: kavramsal - dış}
    \begin{itemize}
      \item öğrenci numarasını\\
        Öğrenci İşleri uygulamasına katar,\\
        Kitaplık uygulamasına sayı olarak sun
    \end{itemize}
  \end{exampleblock}

  \pause
  \begin{exampleblock}{örnek: kavramsal - iç}
    \begin{itemize}
      \item öğrenci numarası için dizin oluştur
    \end{itemize}
  \end{exampleblock}
\end{frame}

%\begin{frame}
%  \frametitle{Roller}
%
%  \begin{itemize}
%    \item \emph{son kullanıcılar}:\\
%      veri üzerinde işlem yapanlar
%    \begin{itemize}
%      \item teknik konularda bilgileri olmadığı varsayılır
%    \end{itemize}
%
%    \pause
%    \bigskip
%    \item \emph{uygulama programcıları}:\\
%      son kullanıcıların kullandıkları programları yazanlar
%  \end{itemize}
%\end{frame}

\begin{frame}
  \frametitle{Yönetici Rolleri}

  \begin{itemize}
    \item veri yöneticisi: kararları verir
    \item hangi veriler tutulacak?
    \item hangi veriye kim erişebilir?
    

    \pause
    \bigskip
    \item veri tabanı yöneticisi: kararları uygular
    \item kavramsal - dış/iç düzey dönüşümlerini tanımlar
    \item sistem başarımını ayarlar
    \item sistemin sürekliliğini sağlar
  \end{itemize}
\end{frame}

\begin{frame}
  \frametitle{VTYS İşlevleri}

  \begin{itemize}
    \item veri tanımlama dili
    \item veri işleme dili

    \pause
    \medskip
    \item veri işleme isteklerinin güvenlik açısından değerlendirilmesi 
    \item veri işleme isteklerinin bütünlük açısından değerlendirilmesi
    \item eşzamanlı isteklerin uygun biçimde yürütülmeleri
    \item başarım
  \end{itemize}
\end{frame}

\subsection{İstemci / Sunucu}

\begin{frame}
  \frametitle{İstemci / Sunucu Yapısı}

  \begin{itemize}
    \item \alert{sunucu}: VTYS işlevlerini yerine getirir

    \bigskip
    \item \alert{istemci}: kullanıcı ile sunucu arasında etkileşimi sağlar
    \item hazır paketler (sorgu dili işleyiciler, rapor üreteçleri, \ldots)
    \item uygulama programcılarının yazdıkları
  \end{itemize}
\end{frame}

\begin{frame}
  \frametitle{Yapı}

  \begin{columns}
    \column{.4\textwidth}
    \begin{center}
      \pgfuseimage{mimari1}
    \end{center}

    \column{.6\textwidth}
    \begin{itemize}
      \item istemci ile sunucu aynı makinada\\
        ya da farklı makinada olabilirler
    \end{itemize}
  \end{columns}
\end{frame}

\begin{frame}
  \frametitle{Çok İstemci / Tek Sunucu}

  \begin{columns}
    \column{.45\textwidth}
    \begin{center}
      \pgfuseimage{mimari2}
    \end{center}

    \column{.55\textwidth}
    \begin{itemize}
      \item birden çok istemci bir sunucuya\\
        bağlanarak çalışabilir
    \end{itemize}

    \bigskip
      \begin{itemize}
        \item sunucu darboğaza girebilir
        \item Yansıtma
      \end{itemize}
  \end{columns}
\end{frame}

\subsection{SQL}

\begin{frame}
  \frametitle{SQL}

  \begin{itemize}
    \item \alert{Structured Query Language}
    \item veri tanımlama dili
    \item veri işleme dili
    \item genel amaçlı programlama dilleriyle etkileşim

    \bigskip
    \item 1970'lerde IBM başlatıyor
    \item standartlar: 1992, 1999, 2003
  \end{itemize}
\end{frame}

\begin{frame}
  \frametitle{SQL Ürünleri}

  \begin{itemize}
    \item Oracle, IBM DB2, MS-SQL, \ldots
    \item açık: PostgreSQL, MySQL, \ldots 
    \item gömülü: SQLite, \ldots
  \end{itemize}
\end{frame}

\section*{Kaynaklar}

\begin{frame}
  \frametitle{Kaynaklar}

  \begin{block}{Okunacak: Date}
    \begin{itemize}
      \item Chapter 1: An Overview of Database Management
      \begin{itemize}
        \item 1.4. \alert{Why Database?}
        \item 1.5. \alert{Data Independence}
      \end{itemize}

      \item Chapter 2: \alert{Database System Architecture}
    \end{itemize}
  \end{block}
\end{frame}

\end{document}
