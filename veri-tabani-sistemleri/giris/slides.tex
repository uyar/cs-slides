% Copyright (c) 2002-2012
%       H. Turgut Uyar <uyar@itu.edu.tr>
%       Şule Gündüz Öğüdücü <sgunduz@itu.edu.tr>
%
% These notes are licensed using the
% "Creative Commons Attribution-NonCommercial-ShareAlike License".
% You are free to copy, distribute and transmit the work, and to adapt the work
% as long as you attribute the authors, do not use it for commercial purposes,
% and any derivative work is under the same or a similar license.
%
% Read the full legal code at:
% http://creativecommons.org/licenses/by-nc-sa/3.0/

\documentclass[dvipsnames]{beamer}

\usepackage{ae}
\usepackage[T1]{fontenc}
\usepackage[utf8]{inputenc}
\usepackage[turkish]{babel}
\setbeamertemplate{navigation symbols}{}

\mode<presentation>
{
  \usetheme{Warsaw}
  \usecolortheme[named=ForestGreen]{structure}
  \setbeamercovered{transparent}
}

\title{Veri Tabanı Sistemleri}
\subtitle{Giriş}

\author{H. Turgut Uyar \and Şule Öğüdücü}
\date{2002-2012}

\AtBeginSubsection[]{
  \begin{frame}<beamer>
    \frametitle{Konular}
    \tableofcontents[currentsection,currentsubsection]
  \end{frame}
}

\theoremstyle{definition}
\newtheorem{tanim}[theorem]{Tanım}

\theoremstyle{example}
\newtheorem{ornek}[theorem]{Örnek}

\theoremstyle{plain}

\pgfdeclareimage[width=2cm]{license}{../../license}

\pgfdeclareimage{kayit}{kayit}
\pgfdeclareimage{vtys}{vtys}
\pgfdeclareimage{sparc}{sparc}
\pgfdeclareimage{mimari1}{mimari1}
\pgfdeclareimage{mimari2}{mimari2}
\pgfdeclareimage{mimari3}{mimari3}

\begin{document}

\begin{frame}
  \titlepage
\end{frame}

\begin{frame}
  \frametitle{License}

  \pgfuseimage{license}\hfill
  \copyright 2002-2012 T. Uyar, Ş. Öğüdücü

  \vfill
  \begin{tiny}
    You are free:
    \begin{itemize}
      \item to Share -- to copy, distribute and transmit the work
      \item to Remix -- to adapt the work
    \end{itemize}

    Under the following conditions:
    \begin{itemize}
      \item Attribution -- You must attribute the work in the manner specified by
        the author or licensor (but not in any way that suggests that they
        endorse you or your use of the work).

      \item Noncommercial -- You may not use this work for commercial purposes.

      \item Share Alike -- If you alter, transform, or build upon this work, you
        may distribute the resulting work only under the same or similar license
        to this one.
    \end{itemize}
  \end{tiny}

  \vfill
  Legal code (the full license):\\
  \url{http://creativecommons.org/licenses/by-nc-sa/3.0/}
\end{frame}

\begin{frame}
  \frametitle{Konular}
  \tableofcontents
\end{frame}

\section{Veri İşleme}

\subsection{Giriş}

\begin{frame}
  \frametitle{Veri İşleme}

  \begin{itemize}
    \item büyük miktarda verinin etkin biçimde tutulması ve işlenmesi

    \pause
    \medskip
    \item temel işlemler:
    \begin{itemize}
      \item yeni veri ekleme
      \item olan verilerde değişiklik
      \item veri silme
      \item sorgulama: planlı - plansız
    \end{itemize}
    \item \alert{CRUD}: create - read - update - delete
  \end{itemize}
\end{frame}

\begin{frame}
  \frametitle{Veri Türleri}

  \begin{itemize}
    \item \emph{kalıcı veriler}:\\
      tutulacak bilginin doğası gereği bulunması zorunlu olan veriler

    \pause
    \bigskip
    \item \emph{geçici veriler}
    \begin{itemize}
      \item \emph{çıkış verileri}: kalıcı verilerden türetilebilen veriler\\
        (sorgu sonuçları, raporlar v.b.)

      \pause
      \medskip
      \item \emph{giriş verileri}: sisteme yeni giren, henüz işlenmemiş veriler
      \begin{itemize}
        \item kalıcı verilere eklenebilir
        \item kalıcı verilerde değişikliklere yol açabilir
        \item hiç kullanılmayabilir
      \end{itemize}
    \end{itemize}
  \end{itemize}
\end{frame}

\begin{frame}
  \frametitle{Roller}

  \begin{itemize}
    \item \emph{son kullanıcılar}:\\
      veri üzerinde işlem yapanlar
    \begin{itemize}
      \item teknik konularda bilgileri olmadığı varsayılır
    \end{itemize}

    \pause
    \bigskip
    \item \emph{uygulama programcıları}:\\
      son kullanıcıların kullandıkları programları yazanlar
  \end{itemize}
\end{frame}

\begin{frame}
  \frametitle{Uygulama Örneği}

  \begin{ornek}[öğrenci verileri]
    \begin{columns}[t]
      \begin{column}<1->{5.1cm}
      \begin{itemize}
        \item Öğrenci İşleri:\\
          öğrencinin adı, numarası,\\
          bölümü, aldığı dersler,\\
          stajları v.b.
      \end{itemize}
      \end{column}

      \begin{column}<2->{5.1cm}
      \begin{itemize}
        \item ortak veriler:\\
          öğrencinin adı, numarası,\\
          bölümü v.b.
      \end{itemize}
      \end{column}
    \end{columns}

    \begin{columns}[t]
      \begin{column}<1->{5.1cm}
      \begin{itemize}
        \item Kitaplık:\\
          öğrencinin adı, numarası,\\
          bölümü, aldığı kitaplar v.b.
      \end{itemize}
      \end{column}

      \begin{column}<2->{5.1cm}
      \begin{itemize}
        \item uygulamaya özel veriler:\\
          öğrencinin aldığı dersler,\\
          stajlar, kitaplar v.b.
      \end{itemize}
      \end{column}
    \end{columns}
  \end{ornek}
\end{frame}

\subsection{Kayıt Dosyaları}

\begin{frame}
  \frametitle{Kayıt Dosyaları}

  \begin{columns}[b]
    \column{.4\textwidth}
    \begin{center}
      \pgfuseimage{kayit}
    \end{center}

    \column{.6\textwidth}
    \begin{itemize}
      \item her uygulamanın kendi verileri var
      \item her uygulama verilerini\\
	kendi yönettiği dosyalarda tutuyor
    \end{itemize}
  \end{columns}
\end{frame}

\begin{frame}
  \frametitle{Tekrarlılık}

  \begin{itemize}
    \item aynı veri birden fazla yerde tutuluyor
    \begin{itemize}
      \item disk alanı israfı
    \end{itemize}
  \end{itemize}

  \pause
  \begin{ornek}
    \begin{itemize}
      \item öğrenci adı, numarası ve bölümü Öğrenci İşleri'nde ayrı,\\
	Kitaplık'ta ayrı tutuluyor
    \end{itemize}
  \end{ornek}
\end{frame}

\begin{frame}
  \frametitle{Tutarsızlık}

  \begin{itemize}
    \item birden fazla yerde tutulan veriler farklılık gösterebilir
  \end{itemize}

  \pause
  \begin{ornek}
    \begin{itemize}
      \item aynı öğrencinin adı Öğrenci İşleri'nde "Victoria Adams",\\
	Kitaplık'ta "Victoria Beckham" görünebilir
    \end{itemize}
  \end{ornek}
\end{frame}

\begin{frame}
  \frametitle{Bütünlük Bozulması}

  \begin{itemize}
    \item bilginin doğruluğunu sağlamak zordur
  \end{itemize}

  \pause
  \begin{ornek}
    \begin{itemize}
      \item "Kontrol ve Bilgisayar Mühendisliği" bölümü kapatılır\\
	ama öğrencilerinin bölüm verisi eskisi gibi kalır
    \end{itemize}
  \end{ornek}
\end{frame}

\begin{frame}
  \frametitle{Yeni Uygulamalarda Zorluklar}

  \begin{itemize}
    \item her yeni uygulama için benzer işlerin yeniden yapılması gerekir
  \end{itemize}

  \pause
  \begin{ornek}
    \begin{itemize}
      \item Burs İşleri için de uygulama yazılacak
    \end{itemize}
  \end{ornek}
\end{frame}

\begin{frame}
  \frametitle{Politika Boşlukları}

  \begin{itemize}
    \item kurum uygulamalarında standart eksikliği
    \begin{itemize}
      \item yaklaşım, yöntem, programlama dili farklılıkları
      \item uygulamalar arasında veri alışverişi
    \end{itemize}

    \pause
    \item her birim yalnızca kendi gereksinimlerine göre karar verir
  \end{itemize}
\end{frame}

\begin{frame}
  \frametitle{Güvenlik}

  \begin{itemize}
    \item ayrıntılı güvenlik izinleri tanımlamak zor
    \item güvenlik yalnızca işletim sistemine bağlı
  \end{itemize}
\end{frame}

\begin{frame}
  \frametitle{Veriye Bağımlılık}

  \begin{tanim}
    \alert{veriye bağımlılık}:\\
      uygulama kodunun veri düzeni ve erişim yöntemine bağımlı olması

    \begin{itemize}
      \item uygulamada değişiklik yapmak çok zor
    \end{itemize}
  \end{tanim}
\end{frame}

\begin{frame}
  \frametitle{Veriye Bağımlılık}

  \begin{ornek}
    \begin{itemize}
      \item öğrenci numarası Öğrenci İşleri'nde katar, Kitaplık'ta sayı

      \pause
      \medskip
      \item Öğrenci İşleri kayıtlarında öğrenci numarası için\\
	B-ağacı dizin tutuluyor
      \begin{itemize}
	\item arama yapılırken B-ağacı algoritmaları kullanılıyor
      \end{itemize}
    \end{itemize}
  \end{ornek}
\end{frame}

\section{Veri Tabanı Yönetim Sistemleri}

\subsection{Giriş}

\begin{frame}
  \frametitle{Veri Tabanı Yönetim Sistemleri}

  \begin{columns}[b]
    \column{.45\textwidth}
    \begin{center}
      \pgfuseimage{vtys}
    \end{center}

    \column{.55\textwidth}
    \begin{itemize}
      \item veriler ortak bir sistemde tutuluyor
      \item uygulamalar verilere ortak bir arayüz üzerinden erişiyor
    \end{itemize}
  \end{columns}
\end{frame}

\begin{frame}
  \frametitle{ANSI/SPARC Mimarisi}

  \begin{center}
    \pgfuseimage{sparc}
  \end{center}
\end{frame}

\begin{frame}
  \frametitle{Dış Düzey}

  \begin{itemize}
    \item son kullanıcı açısından dış düzey:
    \begin{itemize}
      \item verinin kendine gereken altkümesi
      \item kullandığı uygulama programının arayüzü
    \end{itemize}

    \pause
    \bigskip
    \item uygulama programcısı açısından dış düzey:
    \begin{itemize}
      \item kullandığı programlama dili
      \item bu dile veri tabanı işlemleri için yapılan ekler:\\
        \alert{veri altdili}
    \end{itemize}
  \end{itemize}
\end{frame}

\begin{frame}
  \frametitle{Kavramsal Düzey}

  \begin{itemize}
    \item kavramsal düzey: verinin bütünü
    \item veriden bağımsızlığın sağlandığı düzey

    \pause
    \bigskip
    \item \alert{katalog}:\\
      verinin içeriğini betimleyen tanımlar
    \begin{itemize}
      \item veri tabanları
      \item veri tipleri, bütünlük kısıtlamaları
      \item kullanıcılar, yetkiler, güvenlik kısıtlamaları
    \end{itemize}
  \end{itemize}
\end{frame}

\begin{frame}
  \frametitle{İç Düzey}

  \begin{itemize}
    \item iç düzey: gerçekleme ayrıntıları

    \pause
    \item verinin nasıl temsil edildiği:
    \begin{itemize}
      \item dosyalar, kayıtlar
    \end{itemize}

    \pause
    \item veriye nasıl erişileceği
    \begin{itemize}
      \item işaretçiler, dizinler, B-ağaçları
    \end{itemize}
  \end{itemize}
\end{frame}

\begin{frame}
  \frametitle{Dönüşümler}

  \begin{itemize}
    \item veri bağımsızlığı için düzeyler arasında dönüşümler
  \end{itemize}

  \pause
  \begin{ornek}[kavramsal - dış]
    \begin{itemize}
      \item öğrenci numarasını\\
        Öğrenci İşleri uygulamasına katar,\\
        Kitaplık uygulamasına sayı olarak sun
    \end{itemize}
  \end{ornek}

  \pause
  \begin{ornek}[kavramsal - iç]
    \begin{itemize}
      \item öğrenci numarası için dizin oluştur
    \end{itemize}
  \end{ornek}
\end{frame}

\begin{frame}
  \frametitle{Yönetici Rolleri}

  \begin{itemize}
    \item \emph{veri yöneticisi}: kararları verir
    \begin{itemize}
      \item hangi veriler tutulacak?
      \item hangi veriye kim erişebilir?
    \end{itemize}

    \pause
    \bigskip
    \item \emph{veri tabanı yöneticisi}: kararları uygular
    \begin{itemize}
      \item kavramsal - dış/iç düzey dönüşümlerini tanımlar
      \item sistem başarımını ayarlar
      \item sistemin sürekliliğini sağlar
    \end{itemize}
  \end{itemize}
\end{frame}

\begin{frame}
  \frametitle{VTYS İşlevleri}

  \begin{itemize}
    \item veri tanımlama dili

    \pause
    \item veri işleme dili

    \pause
    \item veri işleme isteklerinin bütünlük ve güvenlik\\
      açısından değerlendirilmesi

    \pause
    \item eşzamanlı isteklerin uygun biçimde yürütülmeleri

    \pause
    \item başarım
  \end{itemize}
\end{frame}

\subsection{İstemci/Sunucu}

\begin{frame}
  \frametitle{İstemci/Sunucu Yapısı}

  \begin{itemize}
    \item \alert{sunucu}:\\
      VTYS işlevlerini yerine getirir

    \pause
    \bigskip
    \item \alert{istemci}:\\
      kullanıcı ile sunucu arasında etkileşimi sağlar
    \begin{itemize}
      \item hazır paketler (sorgu dili işleyiciler, rapor üreteçleri vs.)
      \item uygulama programcılarının yazdıkları
    \end{itemize}
  \end{itemize}
\end{frame}

\begin{frame}
  \frametitle{Yapı}

  \begin{columns}
    \column{.4\textwidth}
    \begin{center}
      \pgfuseimage{mimari1}
    \end{center}

    \column{.6\textwidth}
    \begin{itemize}
      \item istemci ile sunucu aynı makinada\\
	ya da farklı makinada olabilirler
    \end{itemize}
  \end{columns}
\end{frame}

\begin{frame}
  \frametitle{Çok İstemci/Tek Sunucu}

  \begin{columns}
    \column{.45\textwidth}
    \begin{center}
      \pgfuseimage{mimari2}
    \end{center}

    \column{.55\textwidth}
    \begin{itemize}
      \item birden çok istemci bir sunucuya\\
	bağlanarak çalışabilir
    \end{itemize}

    \pause
    \bigskip
    \begin{ornek}[Banka]
      \begin{itemize}
        \item merkezde bir sunucu
        \item şubelerde istemciler
      \end{itemize}
    \end{ornek}
  \end{columns}
\end{frame}

\begin{frame}
  \frametitle{Çok İstemci/Çok Sunucu}

  \begin{columns}
    \column{.45\textwidth}
    \begin{center}
      \pgfuseimage{mimari3}
    \end{center}

    \column{.55\textwidth}
    \begin{itemize}
      \item sunucular da dağıtık olabilir
    \end{itemize}

    \pause
    \bigskip
    \begin{ornek}[Banka]
      \begin{itemize}
        \item her şube kendi hesaplarının\\
	  sunucusu (ve istemcisi)
        \item her şube diğer şube hesaplarının\\
	  istemcisi
      \end{itemize}
    \end{ornek}
  \end{columns}
\end{frame}

\subsection{SQL}

\begin{frame}
  \frametitle{SQL}

  \begin{itemize}
    \item \emph{Structured Query Language}
    \begin{itemize}
      \item veri tanımlama dili
      \item veri işleme dili
      \item genel amaçlı programlama dilleriyle etkileşim
    \end{itemize}

    \pause
    \bigskip
    \item tarihçe
    \begin{itemize}
      \item 1970'lerde IBM başlatıyor
      \item standartlar: 1992, 1999, 2003
    \end{itemize}
  \end{itemize}
\end{frame}

\begin{frame}
  \frametitle{SQL Ürünleri}

  \begin{itemize}
    \item Oracle
    \item IBM DB2, Progress, MS-SQL, Sybase
    \item açık: PostgreSQL, MySQL, Firebird
    \item gömülü: SQLite, MS Access
  \end{itemize}
\end{frame}

\section*{Kaynaklar}

\begin{frame}
  \frametitle{Kaynaklar}

  \begin{block}{Okunacak: Date}
    \begin{itemize}
      \item Chapter 1: An Overview of Database Management
      \begin{itemize}
        \item 1.4. \alert{Why Database?}
        \item 1.5. \alert{Data Independence}
      \end{itemize}

      \item Chapter 2: \alert{Database System Architecture}
    \end{itemize}
  \end{block}
\end{frame}

\end{document}
