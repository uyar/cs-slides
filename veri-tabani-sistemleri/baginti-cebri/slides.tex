% Copyright (c) 2002-2016
%       H. Turgut Uyar <uyar@itu.edu.tr>
%       Şule Gündüz Öğüdücü <sgunduz@itu.edu.tr>
%
% This work is licensed under a "Creative Commons
% Attribution-NonCommercial-ShareAlike 4.0 International License".
% For more information, please visit:
% https://creativecommons.org/licenses/by-nc-sa/4.0/

\documentclass[dvipsnames]{beamer}

\usepackage{ae}
\usepackage[scaled=0.88]{beramono}
\usepackage[T1]{fontenc}
\usepackage[utf8]{inputenc}
\usepackage[turkish]{babel}
\setbeamersize{text margin left=2em, text margin right=2em}
\usepackage[labelformat=empty, aboveskip=1pt, belowskip=1pt]{caption}

\usepackage{listings}
\lstdefinelanguage{TutorialD}[]{}{
  morekeywords={AND, AS, BASE, BOOL, CAST_AS_, CAST_AS_RATIONAL, CHAR,
                CONSTRAINT, DELETE, DIVIDEBY, DROP, INSERT, INTEGER, INTERSECT,
                JOIN, KEY, MINUS, OR, PER, POSSREP, RATIONAL, RELATION, RENAME,
                THE_, TUPLE, TYPE, UNION, UPDATE, VAR, WHERE}
}
\lstdefinelanguage{FullSQL}[]{SQL}{
  morekeywords={BINARY, BOOLEAN, CYCLE, FINAL, INCREMENT, IS, LARGE, MAXVALUE,
                MINVALUE, NO_ACTION, OBJECT, REFERENCES, RENAME, SEQUENCE,
                START, TO, TYPE, VACUUM}
}
\lstset{basicstyle=\ttfamily, keywordstyle=\color{ForestGreen},
        showstringspaces=false}

\mode<presentation>
{
  \usetheme{Warsaw}
  \usecolortheme[named=ForestGreen]{structure}
  \setbeamercovered{transparent}
}

\title{Veri Tabanı Sistemleri}
\subtitle{Bağıntı Cebri}

\author{H. Turgut Uyar \and Şule Öğüdücü}
\date{2002-2016}

\AtBeginSubsection[]{
  \begin{frame}<beamer>
    \frametitle{Konular}
    \tableofcontents[currentsection,currentsubsection]
  \end{frame}
}

\theoremstyle{plain}

\pgfdeclareimage[width=2cm]{license}{../license}

\begin{document}

\begin{frame}
  \titlepage
\end{frame}

\begin{frame}
  \frametitle{License}

  \pgfuseimage{license}\hfill
  \copyright~2002-2016 T. Uyar, Ş. Öğüdücü

  \vfill
  \begin{footnotesize}
    You are free to:
    \begin{itemize}
      \itemsep0em
      \item Share -- copy and redistribute the material in any medium or format
      \item Adapt -- remix, transform, and build upon the material
    \end{itemize}

    Under the following terms:
    \begin{itemize}
      \itemsep0em
      \item Attribution -- You must give appropriate credit, provide a link to
        the license, and indicate if changes were made.

      \item NonCommercial -- You may not use the material for commercial
        purposes.

      \item ShareAlike -- If you remix, transform, or build upon the material,
        you must distribute your contributions under the same license as the
        original.
    \end{itemize}
  \end{footnotesize}

  \begin{small}
    For more information:\\
    \url{https://creativecommons.org/licenses/by-nc-sa/4.0/}

    \smallskip
    Read the full license:\\
    \url{https://creativecommons.org/licenses/by-nc-sa/4.0/legalcode}
  \end{small}
\end{frame}

\begin{frame}
  \frametitle{Konular}
  \tableofcontents
\end{frame}

\lstset{language=TutorialD}

\section{Bağıntı Cebri}

\subsection{Giriş}

\begin{frame}
  \frametitle{Kapalılık}

  \begin{itemize}
    \item \alert{kapalılık}: bütün işlemlerin girdileri de çıktıları da bağıntı

    \medskip  
    \item bir işlemin çıktısı diğer bir işlemin girdisi olabilir
    \item içiçe işlemler yapılabilir
  \end{itemize}
\end{frame}

\begin{frame}
  \frametitle{Örnek Bağıntılar}

  \begin{tiny}
  \begin{table}
    \caption{MOVIE}
    \begin{tabular}{|r|l|r|r|r|r|}\hline
\underline{MOVIE\#} & TITLE           & YEAR & SCORE & VOTES & DIRECTOR\#\\[2pt]\hline\hline
   6 & The Usual Suspects             & 1995 &   8.7 & 35027 &        639\\\hline
  70 & Being John Malkovich           & 1999 &   8.3 & 13809 &       1485\\\hline
 107 & Batman \& Robin                & 1997 &   3.5 & 10577 &        105\\\hline
 110 & Sleepy Hollow                  & 1999 &   7.5 & 10514 &        148\\\hline
 112 & Three Kings                    & 1999 &   7.7 & 10319 &       1070\\\hline
 151 & Gattaca                        & 1997 &   7.4 &  8388 &       2020\\\hline
 213 & Blade                          & 1998 &   6.7 &  6885 &       2861\\\hline
 228 & Ed Wood                        & 1994 &   7.8 &  6587 &        148\\\hline
 251 & End of Days                    & 1999 &   5.5 &  6095 &        103\\\hline
 281 & Dangerous Liaisons             & 1988 &   7.7 &  5651 &        292\\\hline
 373 & Fear and Loathing in Las Vegas & 1998 &   6.5 &  4658 &         59\\\hline
 432 & Stigmata                       & 1999 &   6.1 &  4141 &       2557\\\hline
 433 & eXistenZ                       & 1999 &   6.9 &  4130 &         97\\\hline
 573 & Dead Man                       & 1995 &   7.4 &  3333 &        175\\\hline
1468 & Europa                         & 1991 &   7.6 &  1042 &        615\\\hline
1512 & Suspiria                       & 1977 &   7.1 &  1004 &       2259\\\hline
1539 & Cry-Baby                       & 1990 &   5.9 &   972 &        364\\\hline
    \end{tabular}
  \end{table}
  \end{tiny}
\end{frame}

\begin{frame}
  \frametitle{Örnek Bağıntılar}

  \begin{columns}[b]
    \column{.5\textwidth}
    \begin{tiny}
    \begin{table}
      \caption{PERSON}
      \begin{tabular}{|r|l|}\hline
\underline{PERSON\#} & NAME\\[2pt]\hline\hline
   9 & Arnold Schwarzenegger\\\hline
  26 & Johnny Depp          \\\hline
  59 & Terry Gilliam        \\\hline
  97 & David Cronenberg     \\\hline
 103 & Peter Hyams          \\\hline
 105 & Joel Schumacher      \\\hline
 138 & George Clooney       \\\hline
 148 & Tim Burton           \\\hline
 175 & Jim Jarmusch         \\\hline
 187 & Christina Ricci      \\\hline
 243 & Uma Thurman          \\\hline
 282 & Cameron Diaz         \\\hline
 292 & Stephen Frears       \\\hline
 302 & Benicio Del Toro     \\\hline
 308 & Gabriel Byrne        \\\hline
 350 & Jennifer Jason Leigh \\\hline
      \end{tabular}
    \end{table}
    \end{tiny}

    \column{.5\textwidth}
    \begin{tiny}
    \begin{table}
      \begin{tabular}{|r|l|}\hline
 364 & John Waters          \\\hline
 406 & Patricia Arquette    \\\hline
 503 & John Malkovich       \\\hline
 615 & Lars von Trier       \\\hline
 639 & Bryan Singer         \\\hline
 745 & Udo Kier             \\\hline
 793 & Jude Law             \\\hline
1070 & David O. Russell     \\\hline
1485 & Spike Jonze          \\\hline
1641 & Iggy Pop             \\\hline
2020 & Andrew Niccol        \\\hline
2259 & Dario Argento        \\\hline
2557 & Rupert Wainwright    \\\hline
2861 & Stephen Norrington   \\\hline
3578 & Traci Lords          \\\hline
      \end{tabular}
    \end{table}
    \end{tiny}
  \end{columns}
\end{frame}

\begin{frame}
  \frametitle{Örnek Bağıntılar}

   \begin{columns}[b]
    \column{.5\textwidth}
    \begin{tiny}
    \begin{table}
      \caption{CASTING}
      \begin{tabular}{|r|r|r|}\hline
\underline{MOVIE\#} & \underline{ACTOR\#} & ORD\\[2pt]\hline\hline
      6 &     308 &   2\\\hline
      6 &     302 &   3\\\hline
     70 &     282 &   2\\\hline
     70 &     503 &  14\\\hline
    107 &       9 &   1\\\hline
    107 &     138 &   2\\\hline
    107 &     243 &   4\\\hline
    110 &      26 &   1\\\hline
    110 &     187 &   2\\\hline
    112 &     138 &   1\\\hline
    112 &    1485 &   4\\\hline
    151 &     243 &   2\\\hline
    151 &     793 &   3\\\hline
    213 &     745 &   6\\\hline
    213 &    3578 &   8\\\hline
    228 &      26 &   1\\\hline
    228 &     406 &   4\\\hline
    251 &       9 &   1\\\hline
    251 &     308 &   2\\\hline
      \end{tabular}
    \end{table}
    \end{tiny}

    \column{.5\textwidth}
    \begin{tiny}
    \begin{table}
      \begin{tabular}{|r|r|r|}\hline
    251 &     745 &  10\\\hline
    281 &     243 &   7\\\hline
    281 &     503 &   2\\\hline
    373 &      26 &   1\\\hline
    373 &     187 &   6\\\hline
    373 &     282 &   8\\\hline
    373 &     302 &   2\\\hline
    432 &     308 &   2\\\hline
    432 &     406 &   1\\\hline
    433 &     350 &   1\\\hline
    433 &     793 &   2\\\hline
    573 &      26 &   1\\\hline
    573 &     308 &  12\\\hline
    573 &    1641 &   6\\\hline
   1468 &     745 &   3\\\hline
   1512 &     745 &   9\\\hline
   1539 &      26 &   1\\\hline
   1539 &    1641 &   5\\\hline
   1539 &    3578 &   7\\\hline
      \end{tabular}
    \end{table}
    \end{tiny}
  \end{columns}
\end{frame}

\subsection{Seçme}

\begin{frame}[fragile]
  \frametitle{Seçme}

  \begin{itemize}
    \item \alert{seçme}: bir koşula uyan çokluları seçme
    \begin{lstlisting}
relation WHERE condition
    \end{lstlisting}
    
  \medskip
    \item çıktı başlığı = girdi başlığı
  \end{itemize}
\end{frame}

\begin{frame}[fragile]
  \frametitle{Seçme Örnekleri - 1}

    \begin{itemize}
      \item 10000'den fazla oy almış filmler (\texttt{S1})

      \begin{lstlisting}
MOVIE WHERE (VOTES > 10000)
      \end{lstlisting}
    \end{itemize}

    \vspace{-10pt}
    \begin{tiny}
    \begin{table}
      \caption{S1}
      \begin{tabular}{|r|l|r|r|r|r|}\hline
\underline{MOVIE\#} & TITLE & YEAR & SCORE & VOTES & DIRECTOR\#\\[2pt]\hline\hline
  6 & Usual Suspects        & 1995 &   8.7 & 35027 &        639\\\hline
 70 & Being John Malkovich  & 1999 &   8.3 & 13809 &       1485\\\hline
107 & Batman \& Robin       & 1997 &   3.5 & 10577 &        105\\\hline
110 & Sleepy Hollow         & 1999 &   7.5 & 10514 &        148\\\hline
112 & Three Kings           & 1999 &   7.7 & 10319 &       1070\\\hline
      \end{tabular}
    \end{table}
    \end{tiny}
\end{frame}

\begin{frame}[fragile]
  \frametitle{Seçme Örnekleri - 2}

    \begin{itemize}
      \item 1992'den önce çekilmiş, puanı 7.5'in üzerinde olan filmler
        (\texttt{S2})   

      \begin{lstlisting}
MOVIE WHERE ((YEAR < YEAR(1992))
         AND (SCORE > SCORE(7.5)))
      \end{lstlisting}
    \end{itemize}

    \vspace{-10pt}
    \begin{tiny}
    \begin{table}
      \caption{S2}
      \begin{tabular}{|r|l|r|r|r|r|}\hline
\underline{MOVIE\#} & TITLE & YEAR & SCORE & VOTES & DIRECTOR\#\\[2pt]\hline\hline
   281 & Dangerous Liaisons & 1988 &   7.7 &  5651 &        292\\\hline
  1468 & Europa             & 1991 &   7.6 &  1042 &        615\\\hline
      \end{tabular}
    \end{table}
    \end{tiny}
\end{frame}

\begin{frame}[fragile]
  \frametitle{İzdüşüm}

  \begin{itemize}
    \item \alert{izdüşüm}: bir nitelik kümesini seçme
    \begin{lstlisting}
relation { attribute_name [, ...] }
    \end{lstlisting}

    \medskip
     \item çıktı başlığı = nitelik listesi
  \end{itemize}
\end{frame}

\begin{frame}[fragile]
  \frametitle{İzdüşüm Örnekleri - 1}

    \begin{itemize}
      \item bütün filmlerin başlıkları (\texttt{P1})
    \begin{lstlisting}
MOVIE { TITLE }
    \end{lstlisting}
    \end {itemize}

    \vspace{-10pt}
    \begin{columns}[b]
      \column{.5\textwidth}
      \begin{tiny}
      \begin{table}
        \caption{P1}
        \begin{tabular}{|l|}\hline
\underline{TITLE}             \\[2pt]\hline\hline
Usual Suspects                \\\hline
Being John Malkovich          \\\hline
Batman \& Robin               \\\hline
Sleepy Hollow                 \\\hline
Three Kings                   \\\hline
Gattaca                       \\\hline
Blade                         \\\hline
Ed Wood                       \\\hline
End of Days                   \\\hline
        \end{tabular}
      \end{table}
      \end{tiny}

      \column{.5\textwidth}
      \begin{tiny}
      \begin{table}
        \begin{tabular}{|l|}\hline
Dangerous Liaisons            \\\hline
Fear and Loathing in Las Vegas\\\hline
Stigmata                      \\\hline
eXistenZ                      \\\hline
Dead Man                      \\\hline
Europa                        \\\hline
Suspiria                      \\\hline
Cry-Baby                      \\\hline
        \end{tabular}
      \end{table}
      \end{tiny}
    \end{columns}
\end{frame}

\begin{frame}[fragile]
  \frametitle{İzdüşüm Örnekleri - 2}

    \begin{itemize}
      \item bütün filmlerin başlıkları ve yılları (\texttt{P2})
    \begin{lstlisting}
MOVIE { TITLE, YEAR }
    \end{lstlisting}
    \end{itemize}

    \vspace{-10pt}
    \begin{columns}[b]
      \column{.5\textwidth}
      \begin{tiny}
      \begin{table}
        \caption{P2}
        \begin{tabular}{|l|r|}\hline
\underline{TITLE}              & \underline{YEAR}\\[2pt]\hline\hline
Batman \& Robin                & 1997\\\hline
Being John Malkovich           & 1999\\\hline
Blade                          & 1998\\\hline
Cry-Baby                       & 1990\\\hline
Dangerous Liaisons             & 1988\\\hline
Dead Man                       & 1995\\\hline
Ed Wood                        & 1994\\\hline
End of Days                    & 1999\\\hline
Europa                         & 1991\\\hline
        \end{tabular}
      \end{table}
      \end{tiny}

      \column{.5\textwidth}
      \begin{tiny}
      \begin{table}
        \begin{tabular}{|l|r|}\hline
Fear and Loathing in Las Vegas & 1998\\\hline
Gattaca                        & 1997\\\hline
Sleepy Hollow                  & 1999\\\hline
Stigmata                       & 1999\\\hline
Suspiria                       & 1977\\\hline
Three Kings                    & 1999\\\hline
Usual Suspects                 & 1995\\\hline
eXistenZ                       & 1999\\\hline
        \end{tabular}
      \end{table}
      \end{tiny}
    \end{columns}
\end{frame}

\begin{frame}[fragile]
  \frametitle{İzdüşüm Örnekleri - 3}

    \begin{itemize}
      \item bütün filmlerin yılları (\texttt{P3})
    \begin{lstlisting}
MOVIE { YEAR }
    \end{lstlisting}
    \end{itemize}

    \vspace{-10pt}
    \begin{tiny}
    \begin{table}
      \caption{P3}
      \begin{tabular}{|r|}\hline
\underline{YEAR}\\[2pt]\hline\hline
            1995\\\hline
            1999\\\hline
            1997\\\hline
            1998\\\hline
            1994\\\hline
            1988\\\hline
            1991\\\hline
            1977\\\hline
            1990\\\hline
      \end{tabular}
    \end{table}
    \end{tiny}
\end{frame}

\begin{frame}
  \frametitle{İzdüşüm Örnekleri - 4}

    \begin{itemize}
      \item 5000'den fazla oy almış\\
      ve puanı 7.0'ın üzerinde olan filmlerin başlıkları (\texttt{P4})
    \end{itemize}

    \pause
    \begin{enumerate}
      \item 5000'den fazla oy almış\\
      ve puanı 7.0'ın üzerinde olan filmler (\texttt{P4A})
      \item \texttt{P4A}'daki başlıklar (\texttt{P4})
    \end{enumerate}
\end{frame}

\begin{frame}[fragile]
  \frametitle{İzdüşüm Örnekleri - 4}

    \begin{itemize}
      \item 5000'den fazla oy almış\\
      ve puanı 7.0'ın üzerinde olan filmler (\texttt{P4A})

    \begin{lstlisting}
MOVIE WHERE ((VOTES > 5000)
         AND (SCORE > SCORE(7.0)))
    \end{lstlisting}
    \end{itemize}

    \vspace{-10pt}
    \begin{tiny}
    \begin{table}
      \caption{P4A}
      \begin{tabular}{|r|l|r|r|r|r|}\hline
\underline{MOVIE\#} & TITLE & YEAR & SCORE & VOTES & DIRECTOR\#\\[2pt]\hline\hline
   6 & Usual Suspects       & 1995 &   8.7 & 35027 &        639\\\hline
  70 & Being John Malkovich & 1999 &   8.3 & 13809 &       1485\\\hline
 110 & Sleepy Hollow        & 1999 &   7.5 & 10514 &        148\\\hline
 112 & Three Kings          & 1999 &   7.7 & 10319 &       1070\\\hline
 151 & Gattaca              & 1997 &   7.4 &  8388 &       2020\\\hline
 228 & Ed Wood              & 1994 &   7.8 &  6587 &        148\\\hline
 281 & Dangerous Liaisons   & 1988 &   7.7 &  5651 &        292\\\hline
      \end{tabular}
    \end{table}
    \end{tiny}
\end{frame}

\begin{frame}[fragile]
  \frametitle{İzdüşüm Örnekleri - 4}

    \begin{itemize}
      \item \texttt{P4A}'daki başlıklar (\texttt{P4})

    \begin{lstlisting}
P4A { TITLE }
    \end{lstlisting}
    \end{itemize}

    \vspace{-10pt}
    \begin{tiny}
    \begin{table}
      \caption{P4}
      \begin{tabular}{|l|}\hline
\underline{TITLE}   \\[2pt]\hline\hline
Being John Malkovich\\\hline
Dangerous Liaisons  \\\hline
Ed Wood             \\\hline
Gattaca             \\\hline
Sleepy Hollow       \\\hline
Three Kings         \\\hline
Usual Suspects      \\\hline
      \end{tabular}
    \end{table}
    \end{tiny}
\end{frame}

\begin{frame}[fragile]
  \frametitle{İzdüşüm Örnekleri - 4}

    \begin{itemize}
      \item 5000'den fazla oy almış\\
      ve puanı 7.0'ın üzerinde olan filmlerin başlıkları (\texttt{P4})
   

    \begin{lstlisting}
( MOVIE
    WHERE ((VOTES > 5000)
       AND (SCORE > SCORE(7.0))) )
  { TITLE }
    \end{lstlisting}
    \end{itemize}
\end{frame}

\subsection{Katma}

\begin{frame}[fragile]
  \frametitle{Katma}

  \begin{itemize}
    \item \alert{katma}: iki bağıntının çoklularını, bir ya da birden fazla niteliğin\\
      ortak değerleri üzerinden eşleştirme
      \begin{lstlisting}
relation1 JOIN relation2
    \end{lstlisting}
    
    \pause
    \medskip
    \item \alert{doğal katma}: aynı isimli niteliklerin ortak değerleri\\
      üzerinden eşleştirme    
  \end{itemize}
\end{frame}

\begin{frame}[fragile]
  \frametitle{Katma}

  \begin{itemize}
    \item iki bağıntının Kartezyen çarpımından,\\
      verilen nitelikler için aynı değeri taşıyan çokluları seçme
    \item eşleşen nitelikler çıktıda tekrarlanmaz
    \medskip
    \item çıktı başlığı = relation1 başlığı $\cup$ relation2 başlığı
  \end{itemize}
\end{frame}

\begin{frame}
  \frametitle{Katma Örnekleri - 1}

    \begin{itemize}
      \item bütün filmlerin başlıkları ve yönetmenlerinin isimleri (\texttt{J1})
    \end{itemize}

    \pause
    \begin{enumerate}
      \item bütün filmler ve yönetmenleri (\texttt{J1A})

      \item \texttt{J1A}'daki film başlıkları ve yönetmen isimleri
         (\texttt{J1})
    \end{enumerate}
\end{frame}

\begin{frame}[fragile]
  \frametitle{Katma Örnekleri - 1}

    \begin{itemize}
      \item bütün filmler ve yönetmenleri (\texttt{J1A})

    \begin{lstlisting}
MOVIE JOIN
  (PERSON RENAME (PERSON# AS DIRECTOR#))
    \end{lstlisting}
    \end{itemize}

    \vspace{-10pt}
    \begin{tiny}
    \begin{table}
      \caption{J1A}
      \begin{tabular}{|r|l|c|r|l|}\hline
\underline{MOVIE\#} & TITLE & ... & DIRECTOR\# & NAME            \\[2pt]\hline\hline
   6 & Usual Suspects       & ... &     639  & Bryan Singer    \\\hline
  70 & Being John Malkovich & ... &      1485  & Spike Jonze     \\\hline
 107 & Batman \& Robin      & ... &       105  & Joel Schumacher \\\hline
 ... & ...                  & ... &       ...  & ...             \\\hline
1468 & Europa               & ... &       615  & Lars von Trier  \\\hline
1512 & Suspiria             & ... &      2259  & Dario Argento   \\\hline
1539 & Cry-Baby             & ... &       364  & John Waters     \\\hline
      \end{tabular}
    \end{table}
    \end{tiny}
\end{frame}

\begin{frame}[fragile]
  \frametitle{Katma Örnekleri - 1}

    \begin{itemize}
      \item \texttt{J1A}'daki film başlıkları ve yönetmen isimleri
         (\texttt{J1})

    \begin{lstlisting}
J1A { TITLE, NAME }
    \end{lstlisting}
    \end{itemize}

    \vspace{-10pt}
    \begin{tiny}
    \begin{table}
      \caption{J1}
      \begin{tabular}{|l|l|}\hline
\underline{TITLE}    & \underline{NAME}\\[2pt]\hline\hline
Batman \& Robin      & Joel Schumacher \\\hline
Being John Malkovich & Spike Jonze     \\\hline
Blade                & Stephen Norrington\\\hline
...                  & ...             \\\hline
Three Kings          & Spike Jonze     \\\hline
Usual Suspects       & Bryan Singer    \\\hline
eXistenZ             & David Cronenberg\\\hline
      \end{tabular}
    \end{table}
    \end{tiny}
\end{frame}

\begin{frame}
  \frametitle{Katma Örnekleri - 2}

    \begin{itemize}
      \item bütün filmlerin başlıkları, oyuncularının isimleri ve sıraları
        (\texttt{J2})
    \end{itemize}

    \pause
    \begin{enumerate}
      \item bütün filmler ve oyunculuk verileri (\texttt{J2A})
      \item \texttt{J2A}'daki bütün verilerin kişilerle eşlenmesi (\texttt{J2B})
      \item \texttt{J2B}'deki film başlıkları, oyuncu isimleri ve sıraları
	(\texttt{J2})
    \end{enumerate}
\end{frame}

\begin{frame}[fragile]
  \frametitle{Katma Örnekleri - 2}

    \begin{itemize}
      \item bütün filmler ve oyunculuk verileri (\texttt{J2A})

    \begin{lstlisting}
MOVIE JOIN CASTING
    \end{lstlisting}
    \end{itemize}

    \vspace{-10pt}
    \begin{tiny}
    \begin{table}
      \caption{J2A}
      \begin{tabular}{|r|l|c|r|r|}\hline
\underline{MOVIE\#} & TITLE & ... & \underline{ACTOR\#} & ORD\\[2pt]\hline\hline
   6 & Usual Suspects       & ... &                 302 &   3\\\hline
   6 & Usual Suspects       & ... &                 308 &   2\\\hline
  70 & Being John Malkovich & ... &                 282 &   2\\\hline
  70 & Being John Malkovich & ... &                 503 &  14\\\hline
 ... & ...                  & ... &                 ... & ...\\\hline
1539 & Cry-Baby             & ... &                  26 &   1\\\hline
1539 & Cry-Baby             & ... &                1641 &   5\\\hline
1539 & Cry-Baby             & ... &                3578 &   7\\\hline
      \end{tabular}
    \end{table}
    \end{tiny}
\end{frame}

\begin{frame}[fragile]
  \frametitle{Katma Örnekleri - 2}

    \begin{itemize}
      \item \texttt{J2A}'daki bütün verilerin kişilerle eşlenmesi (\texttt{J2B})

    \begin{lstlisting}
J2A JOIN
  (PERSON RENAME { PERSON# AS ACTOR# })
    \end{lstlisting}
     \end{itemize}

    \vspace{-10pt}
    \begin{tiny}
    \begin{table}
      \caption{J2B}
      \begin{tabular}{|r|l|c|r|r|l|}\hline
\underline{MOVIE\#} & TITLE & ... & \underline{ACTOR\#} & ORD & NAME\\[2pt]\hline\hline
   6 & Usual Suspects       & ... &     302 &   3 & Benicio Del Toro\\\hline
   6 & Usual Suspects       & ... &     308 &   2 & Gabriel Byrne   \\\hline
  70 & Being John Malkovich & ... &     282 &   2 & Cameron Diaz    \\\hline
  70 & Being John Malkovich & ... &     503 &  14 & John Malkovich  \\\hline
 ... & ...                  & ... &     ... & ... & ...             \\\hline
1539 & Cry-Baby             & ... &      26 &   1 & Johnny Depp     \\\hline
1539 & Cry-Baby             & ... &    1641 &   5 & Iggy Pop        \\\hline
1539 & Cry-Baby             & ... &    3578 &   7 & Traci Lords     \\\hline
      \end{tabular}
    \end{table}
    \end{tiny}
\end{frame}

\begin{frame}[fragile]
  \frametitle{Katma Örnekleri - 2}

    \begin{itemize}
      \item \texttt{J2B}'deki film başlıkları, oyuncu isimleri ve sıraları
        (\texttt{J2})

    \begin{lstlisting}
J2B { TITLE, NAME, ORD }
    \end{lstlisting}
    \end{itemize}

    \vspace{-10pt}
    \begin{tiny}
    \begin{table}
      \caption{J2}
      \begin{tabular}{|l|l|r|}\hline
\underline{TITLE}    & \underline{NAME} & \underline{ORD}\\[2pt]\hline\hline
Usual Suspects       & Benicio Del Toro &   3\\\hline
Usual Suspects       & Gabriel Byrne    &   2\\\hline
Being John Malkovich & Cameron Diaz     &   2\\\hline
Being John Malkovich & John Malkovich   &  14\\\hline
...                  & ...              & ...\\\hline
Cry-Baby             & Johnny Depp      &   1\\\hline
Cry-Baby             & Iggy Pop         &   5\\\hline
Cry-Baby             & Traci Lords      &   7\\\hline
      \end{tabular}
    \end{table}
    \end{tiny}
\end{frame}

\begin{frame}
  \frametitle{Katma Örnekleri - 3}

    \begin{itemize}
      \item Johnny Depp'in filmlerindeki oyuncuların isimleri (\texttt{J3})
    \end{itemize}

    \pause
    \begin{enumerate}
      \item Johnny Depp'in filmlerinin kimlikleri (\texttt{J3A})
      \item  \texttt{J3A}'daki filmlerde oynamış oyuncuların kimlikleri
        (\texttt{J3B})
      \item \texttt{J3B}'deki oyuncuların isimleri (\texttt{J3})
    \end{enumerate}
\end{frame}

\begin{frame}[fragile]
  \frametitle{Katma Örnekleri - 3}

    \begin{itemize}
      \item Johnny Depp'in filmlerinin kimlikleri (\texttt{J3A})


    \begin{lstlisting}
(((PERSON RENAME (PERSON# AS ACTOR#))
    JOIN CASTING)
  WHERE (NAME = 'Johnny Depp')) { MOVIE# }
  \end{lstlisting}
   \end{itemize}

    \vspace{-10pt}
    \begin{tiny}
    \begin{table}
      \caption{J3A}
      \begin{tabular}{|r|l|r|r|}\hline
\underline{MOVIE\#}\\[2pt]\hline\hline
                110\\\hline
                228\\\hline
                373\\\hline
                573\\\hline
               1539\\\hline
      \end{tabular}
    \end{table}
    \end{tiny}
\end{frame}

\begin{frame}[fragile]
  \frametitle{Katma Örnekleri - 3}

    \begin{itemize}
      \item  \texttt{J3A}'daki filmlerde oynamış oyuncuların kimlikleri
        (\texttt{J3B})

    \begin{lstlisting}
(J3A JOIN CASTING) { ACTOR# }
    \end{lstlisting}
    \end{itemize}

    \vspace{-10pt}
    \begin{tiny}
    \begin{table}
      \caption{J3B}
      \begin{tabular}{|r|}\hline
\underline{ACTOR\#}\\[2pt]\hline\hline
                 26\\\hline
                187\\\hline
                282\\\hline
                302\\\hline
                308\\\hline
                406\\\hline
               1641\\\hline
               3578\\\hline
      \end{tabular}
    \end{table}
    \end{tiny}
\end{frame}

\begin{frame}[fragile]
  \frametitle{Katma Örnekleri - 3}

    \begin{itemize}
      \item \texttt{J3B}'deki oyuncuların isimleri (\texttt{J3})

    \begin{lstlisting}
((J3B RENAME (ACTOR# AS PERSON#))
    JOIN PERSON) { NAME }
    \end{lstlisting}
    \end{itemize}

    \vspace{-10pt}
    \begin{tiny}
    \begin{table}
      \caption{J3}
      \begin{tabular}{|l|}\hline
\underline{NAME} \\[2pt]\hline\hline
Johnny Depp      \\\hline
Christina Ricci  \\\hline
Cameron Diaz     \\\hline
Benicio Del Toro \\\hline
Gabriel Byrne    \\\hline
Patricia Arquette\\\hline
Iggy Pop         \\\hline
Traci Lords      \\\hline
      \end{tabular}
    \end{table}
    \end{tiny}
\end{frame}

\begin{frame}[fragile]
  \frametitle{Bölme}
   
   \begin{itemize}
    \item \alert{bölme}: birinci bağıntıdaki çoklular arasından\\
      ikinci bağıntıdaki bütün çoklularla\\
      bir ara bağıntıda eşleşenleri seçme

    \begin{lstlisting}
relation1 DIVIDEBY relation2
  PER (relation3)
    \end{lstlisting}
  \end{itemize}
\end{frame}

\begin{frame}
  \frametitle{Bölme Örneği}

    \begin{itemize}
      \item Johnny Depp ile Christina Ricci'nin birlikte oynadıkları\\
        filmlerin başlıkları (\texttt{V1})
    \end{itemize}

    \pause
    \begin{enumerate}
      \item Johnny Depp ve Christina Ricci'nin kimlikleri (\texttt{V1A})
      \item \texttt{V1A}'daki oyuncuların birlikte oynadıkları filmlerin
        kimlikleri\\
        (\texttt{V1B})
      \item \texttt{V1B}'deki filmlerin başlıkları (\texttt{V1})
    \end{enumerate}
\end{frame}

\begin{frame}[fragile]
  \frametitle{Bölme Örneği}
 
    \begin{itemize}
      \item Johnny Depp ve Christina Ricci'nin kimlikleri (\texttt{V1A})

    \begin{lstlisting}
(PERSON
    WHERE ((NAME = "Johnny Depp")
        OR (NAME = "Christina Ricci")))
  { PERSON# }
    \end{lstlisting}
   \end{itemize}

    \vspace{-10pt}
    \begin{tiny}
    \begin{table}
      \caption{V1A}
      \begin{tabular}{|r|}\hline
\underline{PERSON\#}\\[2pt]\hline\hline
                  26\\\hline
                 187\\\hline
      \end{tabular}
    \end{table}
    \end{tiny}
\end{frame}

\begin{frame}[fragile]
  \frametitle{Bölme Örneği}

    \begin{itemize}
      \item \texttt{V1A}'daki oyuncuların birlikte oynadıkları filmlerin
        kimlikleri\\
        (\texttt{V1B})

    \begin{lstlisting}
(MOVIE { MOVIE# })
    DIVIDEBY (V1A RENAME (PERSON# AS ACTOR#))
      PER (CASTING { MOVIE#, ACTOR# })
    \end{lstlisting}
   \end{itemize}

    \vspace{-10pt}
    \begin{tiny}
    \begin{table}
      \caption{V1B}
      \begin{tabular}{|r|}\hline
\underline{MOVIE\#}\\[2pt]\hline\hline
                110\\\hline
                373\\\hline
      \end{tabular}
    \end{table}
    \end{tiny}
\end{frame}

\begin{frame}[fragile]
  \frametitle{Bölme Örneği}

    \begin{itemize}
      \item \texttt{V1B}'deki filmlerin başlıkları (\texttt{V1})

    \begin{lstlisting}
(V1B JOIN MOVIE) { TITLE }
    \end{lstlisting}
    \end{itemize}

    \vspace{-10pt}
    \begin{tiny}
    \begin{table}
      \caption{V1}
      \begin{tabular}{|l|}\hline
\underline{TITLE}             \\[2pt]\hline\hline
Fear and Loathing in Las Vegas\\\hline
Sleepy Hollow                 \\\hline
      \end{tabular}
    \end{table}
    \end{tiny}
\end{frame}

\begin{frame}[fragile]
  \frametitle{Bölme Örneği}

  \begin{itemize}
    \item çarpma - bölme ilişkisi:\\
      \lstinline!V1B JOIN V1A! $\subseteq$
      \lstinline!CASTING { MOVIE#, ACTOR# }!
  \end{itemize}

  \begin{tiny}
  \begin{table}
    \begin{tabular}{|r|r|}\hline
\underline{MOVIE\#} & \underline{ACTOR\#}\\[2pt]\hline\hline
                110 &                  26\\\hline
                110 &                 187\\\hline
                373 &                  26\\\hline
                373 &                 187\\\hline
    \end{tabular}
  \end{table}
  \end{tiny}
\end{frame}
 
\subsection{Küme İşlemleri}

\begin{frame}[fragile]
  \frametitle{Kesişim}

  \begin{itemize}
    \item \alert{kesişim}: iki bağıntıda da bulunan çokluları seçme

    \begin{lstlisting}
relation1 INTERSECT relation2
    \end{lstlisting}

  \medskip
    \item çıktı başlığı = relation1 başlığı = relation2 başlığı
  \end{itemize}
\end{frame}

\begin{frame}
  \frametitle{Kesişim Örneği}

    \begin{itemize}
      \item oyunculuk yapmış bütün yönetmenlerin isimleri (\texttt{I1})
    \end{itemize}

    \pause
    \begin{enumerate}
      \item oyunculuk yapmış bütün yönetmenlerin kimlikleri (\texttt{I1A})
      \item \texttt{I1A}'daki bütün kişilerin isimleri (\texttt{I1})
    \end{enumerate}
\end{frame}

\begin{frame}[fragile]
  \frametitle{Kesişim Örneği}

    \begin{itemize}
      \item oyunculuk yapmış bütün yönetmenlerin kimlikleri (\texttt{I1A})

    \begin{lstlisting}
(MOVIE { DIRECTOR# }
      RENAME (DIRECTOR# AS PERSON#))
  INTERSECT
(CASTING { ACTOR# }
      RENAME (ACTOR# AS PERSON#))
    \end{lstlisting}
    \end{itemize}

    \vspace{-10pt}
    \begin{tiny}
    \begin{table}
      \caption{I1A}
      \begin{tabular}{|r|}\hline
\underline{PERSON\#}\\[2pt]\hline\hline
                1485\\\hline
      \end{tabular}
    \end{table}
    \end{tiny}
\end{frame}

\begin{frame}[fragile]
  \frametitle{Kesişim Örneği}

    \begin{itemize}
      \item \texttt{I1A}'daki bütün kişilerin isimleri (\texttt{I1})

    \begin{lstlisting}
(I1A JOIN PERSON) { NAME }
    \end{lstlisting}
    \end{itemize}

    \vspace{-10pt}
    \begin{tiny}
    \begin{table}
      \caption{I1}
      \begin{tabular}{|l|}\hline
\underline{NAME}\\[2pt]\hline\hline
Spike Jonze     \\\hline
      \end{tabular}
    \end{table}
    \end{tiny}
\end{frame}

\begin{frame}[fragile]
  \frametitle{Birleşim}

  \begin{itemize}
    \item \alert{birleşim}: iki bağıntıdan en az birinde bulunan çokluları seçme

    \begin{lstlisting}
relation1 UNION relation2
    \end{lstlisting}

    \medskip
  
    \item çıktı başlığı = relation1 başlığı = relation2 başlığı
  \end{itemize}
\end{frame}

\begin{frame}
  \frametitle{Birleşim Örneği}

    \begin{itemize}
      \item 1997'den sonra çekilen filmlerin yönetmenlerinin\\
        ve oyuncularının isimleri (\texttt{U1})
    \end{itemize}

    \pause
    \begin{enumerate}
      \item 1997'den sonra çekilen filmlerin kimlikleri\\
        ve yönetmen kimlikleri (\texttt{U1A})

      \item \texttt{U1A}'daki filmlerin bütün oyuncularının kimlikleri (\texttt{U1B})

      \item \texttt{U1A} ile \texttt{U1B}'den en az birinde bulunan yönetmen\\
        ve oyuncuların kimlikleri (\texttt{U1C})

      \item \texttt{U1C}'deki bütün kişilerin isimleri (\texttt{U1})
    \end{enumerate}
\end{frame}

\begin{frame}[fragile]
  \frametitle{Birleşim Örneği}

    \begin{itemize}
      \item 1997'den sonra çekilen filmlerin kimlikleri\\
        ve yönetmen kimlikleri (\texttt{U1A})

    \begin{lstlisting}
(MOVIE WHERE (YEAR > YEAR(1997)))
  { MOVIE#, DIRECTOR# }
    \end{lstlisting}
    \end{itemize}

    \vspace{-10pt}
    \begin{tiny}
    \begin{table}
      \caption{U1A}
      \begin{tabular}{|r|r|}\hline
\underline{MOVIE\#} & DIRECTOR\#\\[2pt]\hline\hline
 70 &       1485\\\hline
110 &        148\\\hline
112 &       1070\\\hline
213 &       2861\\\hline
251 &        103\\\hline
373 &         59\\\hline
432 &       2557\\\hline
433 &         97\\\hline
      \end{tabular}
    \end{table}
    \end{tiny}
\end{frame}

\begin{frame}[fragile]
  \frametitle{Birleşim Örneği}

    \begin{itemize}
      \item \texttt{U1A}'daki filmlerin bütün oyuncularının kimlikleri (\texttt{U1B})

    \begin{lstlisting}
(U1A JOIN CASTING) { ACTOR# }
    \end{lstlisting}
    \end{itemize}

    \vspace{-10pt}
    \begin{columns}[b]
      \column{.5\textwidth}
      \begin{tiny}
      \begin{table}
        \caption{U1B}
        \begin{tabular}{|r|}\hline
\underline{ACTOR\#}\\[2pt]\hline\hline
                  9\\\hline
                 26\\\hline
                138\\\hline
                187\\\hline
                282\\\hline
                302\\\hline
                308\\\hline
        \end{tabular}
      \end{table}
      \end{tiny}

      \column{.5\textwidth}
      \begin{tiny}
      \begin{table}
        \begin{tabular}{|r|}\hline
                350\\\hline
                406\\\hline
                503\\\hline
                745\\\hline
                793\\\hline
               1485\\\hline
               3578\\\hline
        \end{tabular}
      \end{table}
      \end{tiny}
    \end{columns}
\end{frame}

\begin{frame}[fragile]
  \frametitle{Birleşim Örneği}

    \begin{itemize}
      \item \texttt{U1A} ile \texttt{U1B}'den en az birinde bulunan yönetmen\\
        ve oyuncuların kimlikleri (\texttt{U1C})
    

    \begin{lstlisting}
(U1A { DIRECTOR# }
      RENAME (DIRECTOR# AS PERSON#))
UNION (U1B RENAME (ACTOR# AS PERSON#))
    \end{lstlisting}
    \end{itemize}

    \vspace{-10pt}
    \begin{columns}[b]
    \column{.3\textwidth}
    \begin{tiny}
    \begin{table}
      \caption{U1C}
      \begin{tabular}{|r|}\hline
\underline{PERSON\#}\\[2pt]\hline\hline
                  9\\\hline
                 26\\\hline
                 59\\\hline
                 97\\\hline
                103\\\hline
                138\\\hline
                148\\\hline
      \end{tabular}
    \end{table}
    \end{tiny}

    \column{.2\textwidth}
    \begin{tiny}
    \begin{table}
      \begin{tabular}{|r|}\hline
                187\\\hline
                282\\\hline
                302\\\hline
                308\\\hline
                350\\\hline
                406\\\hline
                503\\\hline
      \end{tabular}
    \end{table}
    \end{tiny}

    \column{.2\textwidth}
    \begin{tiny}
    \begin{table}
      \begin{tabular}{|r|}\hline
                745\\\hline
                793\\\hline
               1070\\\hline
               1485\\\hline
               2557\\\hline
               2861\\\hline
               3578\\\hline
      \end{tabular}
    \end{table}
    \end{tiny}
  \end{columns}
\end{frame}

\begin{frame}[fragile]
  \frametitle{Birleşim Örneği}

    \begin{itemize}
      \item \texttt{U1C}'deki bütün kişilerin isimleri (\texttt{U1})

    \begin{lstlisting}
(U1C JOIN PERSON) { NAME }
    \end{lstlisting}
    \end{itemize}

    \vspace{-10pt}
   \begin{columns}[b]
    \column{.3\textwidth}
    \begin{tiny}
    \begin{table}
      \caption{U1}
      \begin{tabular}{|l|}\hline
\underline{NAME}     \\[2pt]\hline\hline
Arnold Schwarzenegger\\\hline
Benicio Del Toro     \\\hline
Cameron Diaz         \\\hline
Christina Ricci      \\\hline
David Cronenberg     \\\hline
David O. Russell     \\\hline
Gabriel Byrne        \\\hline
      \end{tabular}
    \end{table}
    \end{tiny}

    \column{.2\textwidth}
    \begin{tiny}
    \begin{table}
      \begin{tabular}{|l|}\hline
George Clooney       \\\hline
Jennifer Jason Leigh \\\hline
John Malkovich       \\\hline
Johnny Depp          \\\hline
Jude Law             \\\hline
Patricia Arquette    \\\hline
Peter Hyams          \\\hline
      \end{tabular}
    \end{table}
    \end{tiny}

    \column{.2\textwidth}
    \begin{tiny}
    \begin{table}
      \begin{tabular}{|l|}\hline
Rupert Wainwright    \\\hline
Spike Jonze          \\\hline
Stephen Norrington   \\\hline
Terry Gilliam        \\\hline
Tim Burton           \\\hline
Traci Lords          \\\hline
Udo Kier             \\\hline
      \end{tabular}
    \end{table}
    \end{tiny}
  \end{columns}
\end{frame}

\begin{frame}[fragile]
  \frametitle{Fark}

  \begin{itemize}
    \item \alert{fark}: birinci bağıntıda bulunan ama ikincide bulunmayan\\
      çokluları seçme

    \begin{lstlisting}
relation1 MINUS relation2
    \end{lstlisting}

  \medskip
  
    \item çıktı başlığı = relation1 başlığı = relation2 başlığı
  \end{itemize}
\end{frame}

\begin{frame}
  \frametitle{Fark Örneği}

    \begin{itemize}
      \item Johnny Depp'in filmlerinde oynamamış oyuncuların isimleri\\
        (\texttt{D1})
    \end{itemize}

    \pause
    \begin{enumerate}
      \item Johnny Depp'in filmlerinde oynamış oyuncuların kimlikleri\\
        (\texttt{J3B})
      \item \texttt{J3B}'de olmayan bütün oyuncuların isimleri (\texttt{D1})
    \end{enumerate}
\end{frame}

\begin{frame}[fragile]
  \frametitle{Fark Örneği}

    \begin{itemize}
      \item \texttt{J3B}'de olmayan bütün oyuncuların isimleri (\texttt{D1})

    \begin{lstlisting}
(((CASTING { ACTOR# } MINUS J3B)
      RENAME (ACTOR# AS PERSON#))
  JOIN PERSON) {NAME}
    \end{lstlisting}
    \end{itemize}

    \vspace{-10pt}
\begin{tiny}
  \begin{table}
    \caption{D1}
    \begin{tabular}{|l|}\hline
\underline{NAME}     \\[2pt]\hline\hline
Arnold Schwarzenegger\\\hline
George Clooney       \\\hline
Jennifer Jason Leigh \\\hline
John Malkovich       \\\hline
Jude Law             \\\hline
Spike Jonze          \\\hline
Udo Kier             \\\hline
Uma Thurman          \\\hline
    \end{tabular}
  \end{table}
  \end{tiny}
\end{frame}

\subsection*{Kaynaklar}

\begin{frame}
  \frametitle{Kaynaklar}

  \begin{block}{Okunacak: Date}
    \begin{itemize}
      \item Chapter 7: Relational Algebra
      \begin{itemize}
        \item 7.1. \alert{Introduction}
        \item 7.2. \alert{Closure Revisited}
        \item 7.4. \alert{The Original Algebra: Semantics}
      \end{itemize}
    \end{itemize}
  \end{block}
\end{frame}

\lstset{language=FullSQL}

\section{SQL}

\subsection{Giriş}

\begin{frame}[fragile]
  \frametitle{Sütun Seçme}
  
  \begin{itemize}
    \item tablodan sütunları seçme:
    \begin{lstlisting}
SELECT [ ALL | DISTINCT ] column_name [, ...]
  FROM table_name
    \end{lstlisting}

    \pause
    \medskip
    
    \item tekrarlı satırlara izin var
    \begin{itemize}
        \item \lstinline!ALL!: tekrarlı satırlar korunsun (varsayılan)
        \item \lstinline!DISTINCT!: tekrarlı satırlar bir taneye indirilsin
    \end{itemize}

    \item \lstinline!*!: bütün sütunlar
  \end{itemize}
\end{frame}

\begin{frame}[fragile]
  \frametitle{Sorgulama Örnekleri}
  
  \begin{itemize}
  \item bütün filmlerin bütün verileri
    \begin{lstlisting}
SELECT * FROM MOVIE
    \end{lstlisting}

  \pause
  \item bütün filmlerin başlıkları ve yılları
    \begin{lstlisting}
SELECT TITLE, YR FROM MOVIE
    \end{lstlisting}

  \pause
  \item hangi yıllarda film çekildiği
    \begin{lstlisting}
SELECT DISTINCT YR FROM MOVIE
    \end{lstlisting}
  \end{itemize}
\end{frame}

\begin{frame}[fragile]
  \frametitle{Sonuçların Sıralanması}

  \begin{itemize}
    \item sonuç tablosundaki satırların sıralanması
    \begin{lstlisting}
SELECT [ ALL | DISTINCT ] column_name [, ...]
  FROM table_name
  [ ORDER BY { column_name [ ASC | DESC ] }
      [, ...] ]
    \end{lstlisting}


    \medskip
    \item \lstinline!ASC!: artan sırada (varsayılan)
    \item \lstinline!DESC!: azalan sırada
  \end{itemize}
\end{frame}

\begin{frame}[fragile]
  \frametitle{Sorgulama Örnekleri}

  \begin{itemize}
  \item hangi yıllarda film çekildiği, yıla göre artan sırada
    \begin{lstlisting}
SELECT DISTINCT YR FROM MOVIE
  ORDER BY YR
    \end{lstlisting}

  \pause
  \item hangi yıllarda film çekildiği, yıla göre azalan sırada
    \begin{lstlisting}
SELECT DISTINCT YR FROM MOVIE
  ORDER BY YR DESC
    \end{lstlisting}
  \end{itemize}
\end{frame}

\begin{frame}[fragile]
  \frametitle{Sorgulama Örnekleri}

  \begin{itemize}
    \item bütün filmlerin bütün verileri, yıla göre azalan\\
      başlıklara göre artan sırada
    \begin{lstlisting}
SELECT * FROM MOVIE
  ORDER BY YR DESC, TITLE ASC
    \end{lstlisting}
  \end{itemize}
\end{frame}

\begin{frame}[fragile]
  \frametitle{Deyimler}

   \begin{itemize}
     \item sütunlar üzerinde deyimlerin değerlendirilmesi
    \begin{lstlisting}
SELECT [ ALL | DISTINCT ]
  { expression [ AS column_name ] } [, ...]
  FROM table_name
  [ ORDER BY { column_name [ ASC | DESC ] }
      [, ...] ]
    \end{lstlisting}

  \medskip
    \item oluşan sütuna yeni isim verilebilir: \lstinline!AS!
    \item sıralamada sütunun ismi ya da numarası kullanılabilir
  \end{itemize}
\end{frame}

\begin{frame}[fragile]
  \frametitle{Sorgulama Örnekleri}

  \begin{itemize}
    \item bütün filmlerin başlıkları ve toplam puanları
    \begin{lstlisting}
SELECT TITLE, SCORE * VOTES
  FROM MOVIE
    \end{lstlisting}
  \end{itemize}
\end{frame}

\begin{frame}[fragile]
  \frametitle{Sorgulama Örnekleri}

  \begin{itemize}
    \item bütün filmlerin başlıkları ve toplam puanları,\\
                toplam puana göre azalan sırada]
    \begin{lstlisting}
SELECT TITLE, SCORE * VOTES AS POINTS
  FROM MOVIE
  ORDER BY POINTS DESC
    \end{lstlisting}

    \pause
    \begin{lstlisting}
SELECT TITLE, SCORE * VOTES
  FROM MOVIE
  ORDER BY 2 DESC
    \end{lstlisting}
  \end{itemize}
\end{frame}

\begin{frame}[fragile]
  \frametitle{Satır Seçme}

  \begin{itemize}
    \item tablodan satır seçme
    \begin{lstlisting}
SELECT [ ALL | DISTINCT ]
  { expression [ AS column_name ] } [, ...]
  FROM table_name
  [ WHERE condition ]
  [ ORDER BY { column_name [ ASC | DESC ] }
      [, ...] ]
    \end{lstlisting}
  \end{itemize}
\end{frame}

\begin{frame}[fragile]
  \frametitle{Sorgulama Örnekleri}

  \begin{itemize}
    \item "Citizen Kane" başlıklı filmlerin yılları
    \begin{lstlisting}
SELECT YR FROM MOVIE
  WHERE (TITLE = 'Citizen Kane')
    \end{lstlisting}


  \pause
  \item puanı 3'den küçük ve 10'dan fazla oy almış filmlerin\\
                başlıkları
    \begin{lstlisting}
SELECT TITLE FROM MOVIE
  WHERE ((SCORE < 3) AND (VOTES > 10))
    \end{lstlisting}
  \end{itemize}
\end{frame}

\begin{frame}[fragile]
  \frametitle{Koşul Deyimleri}

  \begin{itemize}
    \item sütunun boş olup olmadığı:\\
    \begin{lstlisting}
column_name IS { NULL | NOT NULL }
    \end{lstlisting}

    \pause
    \medskip
    \item küme üyeliği:\\
    \begin{lstlisting}
column_name IN (value_set)
    \end{lstlisting}

    \pause
    \medskip
    \item katar karşılaştırması
    \begin{lstlisting}
column_name LIKE pattern
    \end{lstlisting}
    \item desende \lstinline!%! işareti herhangi bir simge grubu yerine geçer
  \end{itemize}
\end{frame}

\begin{frame}[fragile]
  \frametitle{Sorgulama Örnekleri}

  \begin{itemize}
    \item yılı belli olmayan filmlerin başlıkları
    \begin{lstlisting}
SELECT TITLE FROM MOVIE
  WHERE (YR IS NULL)
    \end{lstlisting}

  \pause
  \item 1967, 1954 ve 1988 yıllarında çekilmiş filmlerin başlıkları ve yılları
    \begin{lstlisting}
SELECT TITLE, YR FROM MOVIE
  WHERE (YR IN (1967, 1954, 1988))
    \end{lstlisting}
  \end{itemize}
\end{frame}

\begin{frame}[fragile]
  \frametitle{Sorgulama Örnekleri}

  \begin{itemize}
    \item "Police Academy" filmlerinin başlıkları ve puanları
    \begin{lstlisting}
SELECT TITLE, SCORE FROM MOVIE
  WHERE (TITLE LIKE 'Police Academy%')
    \end{lstlisting}
  \end{itemize}
\end{frame}

\begin{frame}[fragile]
  \frametitle{Gruplama}

  \begin{itemize}
  \item seçilen satırları gruplama
    \begin{lstlisting}
SELECT [ ALL | DISTINCT ]
  { expression [ AS column_name ] } [, ...]
  FROM table_name
  [ WHERE condition ]
  [ GROUP BY column_name [, ...] ]
  [ HAVING condition ]
  [ ORDER BY { column_name [ ASC | DESC ] }
      [, ...] ]
    \end{lstlisting}

  \medskip
    \item seçilen satırlar gruplanabilir
    \item gruplar içinden seçim yapılabilir: \lstinline!HAVING!
  \end{itemize}
\end{frame}

\begin{frame}
  \frametitle{İşleniş Sırası}

  \begin{itemize}
    \item \lstinline!WHERE! koşulunu sağlayan satırlar seçilir

    \pause
    \item \lstinline!GROUP BY! ile belirtilen sütunlara göre gruplanır
    \begin{itemize}
      \item gruplama yoksa sonuç tek grup kabul edilir
    \end{itemize}

    \pause
    \item \lstinline!HAVING! koşulunu sağlayan gruplar seçilir

    \pause
    \item sütun listesinde verilen deyimler hesaplanır

    \pause
    \item \lstinline!ORDER BY! ile belirtilen sütun listesine göre sıralanır
  \end{itemize}
\end{frame}

\begin{frame}[fragile]
  \frametitle{Grup Değerleri}

  \begin{itemize}
    \item her grup için tek bir değer oluşmalı
    \item gruplayan sütunun değeri
    \item biriktirme fonksiyonu sonucu

    \pause
    \medskip
    \item biriktirme fonksiyonları: \lstinline!COUNT SUM AVG MAX MIN!
    \item parametre olarak sütun adı verilir
    \item boş değerler hesaba katılmaz
  \end{itemize}
\end{frame}

\begin{frame}[fragile]
  \frametitle{Sorgulama Örnekleri}

  \begin{itemize}
    \item puanı 8.5'den büyük filmlerin hangi yıllarda,\\
                kaçar tane çekildiği
    \begin{lstlisting}
SELECT YR, COUNT(*) FROM MOVIE
  WHERE (SCORE > 8.5)
  GROUP BY YR
    \end{lstlisting}
    
    \pause
    \item her yılın en beğenilen filminin puanı,\\
      yıllara göre artan sırada
    \begin{lstlisting}
SELECT YR, MAX(SCORE) FROM MOVIE
  GROUP BY YR
  ORDER BY YR
    \end{lstlisting}   
  \end{itemize}
\end{frame}

\begin{frame}[fragile]
  \frametitle{Sorgulama Örnekleri}

  \begin{itemize}
    \item kullanılan toplam oy sayısı
    \begin{lstlisting}
SELECT SUM(VOTES) FROM MOVIE
    \end{lstlisting}
  \end{itemize}
\end{frame}

\begin{frame}[fragile]
  \frametitle{Sorgulama Örnekleri}

  \begin{itemize}
    \item 40'dan fazla kişinin oy kullandığı\\
                en az 25 filmin olduğu yıllardaki\\
                filmlerin puanlarının ortalamaları,\\
                yıllara göre artan sırada

\medskip
\lstinline!SELECT! \uncover<5->{\lstinline!YR, AVG(SCORE)!\\}
~~~~\lstinline!FROM MOVIE!\\
\pause
~~~~\lstinline!WHERE (VOTES > 40)!\\
\pause
~~~~\lstinline!GROUP BY YR!\\
\pause
~~~~\lstinline!HAVING (COUNT(ID) >= 25)!\\
\pause\pause
~~~~\lstinline!ORDER BY YR!
  \end{itemize}
\end{frame}

\subsection{Katma}

\begin{frame}
  \frametitle{Katma}

  \begin{itemize}
    \item katma işlemi \lstinline!WHERE! koşulları yardımıyla yapılabilir
    \item tablo listesinde katılacak tablolar belirtilir
    \item eş isimli sütunlar için noktalı gösterilim kullanılır

    \pause
    \medskip
    \item işleniş sırası (kavramsal):
    \begin{itemize}
      \item tabloların Kartezyen çarpımı alınır
      \item \lstinline!WHERE! koşulunu sağlayan satırlar seçilir
      \item \ldots
    \end{itemize}
  \end{itemize}
\end{frame}

\begin{frame}[fragile]
  \frametitle{Sorgulama Örnekleri}

  \begin{itemize}
    \item "Star Wars" başlıklı filmlerin yönetmenlerinin isimleri
    \begin{lstlisting}
SELECT NAME
  FROM MOVIE, PERSON
  WHERE ((DIRECTORID = PERSON.ID)
     AND (TITLE = 'Star Wars'))
    \end{lstlisting}
  \end{itemize}
\end{frame}

\begin{frame}[fragile]
  \frametitle{Sorgulama Örnekleri}

  \begin{itemize}
    \item "Alien" başlıklı filmlerde oynayan oyuncuların isimleri
     \begin{lstlisting}
SELECT NAME
  FROM MOVIE, PERSON, CASTING
  WHERE ((TITLE = 'Alien')
     AND (MOVIEID = MOVIE.ID)
     AND (ACTORID = PERSON.ID))
      \end{lstlisting}
    \end{itemize}
\end{frame}

\begin{frame}[fragile]
  \frametitle{Sorgulama Örnekleri}

  \begin{itemize}
    \item "Harrison Ford" isimli oyuncuların oynadığı\\
                filmlerin başlıkları
    \begin{lstlisting}
SELECT TITLE
  FROM MOVIE, PERSON, CASTING
  WHERE ((NAME = 'Harrison Ford')
     AND (MOVIEID = MOVIE.ID)
     AND (ACTORID = PERSON.ID))
    \end{lstlisting}
  \end{itemize}
\end{frame}

\begin{frame}[fragile]
  \frametitle{Sorgulama Örnekleri}

  \begin{itemize}
    \item 1962 yılında çekilmiş filmlerin başlıkları\\
                ve başrol oyuncularının isimleri
    \begin{lstlisting}
SELECT TITLE, NAME
  FROM MOVIE, PERSON, CASTING
  WHERE ((YR = 1962)
     AND (MOVIEID = MOVIE.ID)
     AND (ACTORID = PERSON.ID)
     AND (ORD = 1))
    \end{lstlisting}
  \end{itemize}
\end{frame}

\begin{frame}
  \frametitle{Sorgulama Örnekleri}

  \begin{itemize}
    \item John Travolta'nın hangi yıl kaç filmde oynadığı
    
    \medskip
\lstinline!SELECT! \uncover<4->{\lstinline!YR, COUNT(MOVIEID)!\\}
~~~~\lstinline!FROM MOVIE, PERSON, CASTING!\\
\pause
~~~~\lstinline!WHERE ((NAME = 'John Travolta')!\\
~~~~~~~~~\lstinline!AND (MOVIEID = MOVIE.ID)!\\
~~~~~~~~~\lstinline!AND (ACTORID = PERSON.ID))!\\
\pause
~~~~\lstinline!GROUP BY YR!
  \end{itemize}
\end{frame}

\begin{frame}
  \frametitle{Sorgulama Örnekleri}
\begin{itemize}
    \item 1978 yılında çekilmiş filmlerin başlıkları ve oyuncu sayıları,\\
      oyuncu sayısına göre azalan sırada

    \medskip
\lstinline!SELECT! \uncover<4->{\lstinline!TITLE, COUNT(ACTORID)!}\\
~~~~\lstinline!FROM MOVIE, CASTING!\\
\pause
~~~~\lstinline!WHERE ((YR = 1978)!\\
~~~~~~~~~\lstinline!AND (MOVIE.ID = CASTING.MOVIEID))!\\
\pause
~~~~\lstinline!GROUP BY MOVIEID, TITLE!\\
\pause\pause
~~~~\lstinline!ORDER BY 2 DESC!
  \end{itemize}
\end{frame}

\begin{frame}[fragile]
  \frametitle{Tablo Deyimleri}

  \begin{itemize}
    \item katma işlemi bir tablo deyimi olarak yazılabilir:
    \begin{lstlisting}
SELECT ...
  FROM table_expression [ AS table_name ]
  WHERE selection_condition
  ...
    \end{lstlisting}
    \medskip
    \item çarpma
    \item koşul belirterek
    \item eş isimli sütunlar üzerinden
    \item doğal katma
    \item dış katma
  \end{itemize}
\end{frame}

\begin{frame}[fragile]
  \frametitle{Katma Deyimleri}

  \begin{itemize}
    \item çarpma
   \begin{lstlisting}
SELECT ...
  FROM table1_name CROSS JOIN table2_name
  ...
    \end{lstlisting}

  \pause
  \medskip
  \item koşul belirterek katma
    \begin{lstlisting}
SELECT ...
  FROM table1_name JOIN table2_name
         ON condition
  ...
    \end{lstlisting}
  \end{itemize}
\end{frame}

\begin{frame}[fragile]
  \frametitle{Sorgulama Örnekleri}

  \begin{itemize}
  \item "Star Wars" başlıklı filmlerin yönetmenlerinin isimleri
   \begin{lstlisting}
SELECT NAME
  FROM MOVIE, PERSON
  WHERE ((DIRECTORID = PERSON.ID)
     AND (TITLE = 'Star Wars'))
    \end{lstlisting}

    \pause
    \begin{lstlisting}
SELECT NAME
  FROM MOVIE JOIN PERSON
          ON (DIRECTORID = PERSON.ID)
  WHERE (TITLE = 'Star Wars')
    \end{lstlisting}
  \end{itemize}
\end{frame}

\begin{frame}[fragile]
  \frametitle{Katma Deyimleri}

  \begin{itemize}
    \item eş isimli sütunlar üzerinden
    \begin{lstlisting}
SELECT ...
  FROM table1_name JOIN table2_name
         USING (column_name [, ...])
  ...
    \end{lstlisting}
    \item tekrarlı sütunlar bir kere alınır
    
    \pause
    \medskip
    \item doğal katma
    \begin{lstlisting}
SELECT ...
  FROM table1_name NATURAL JOIN table2_name
  ...
    \end{lstlisting}
  \end{itemize}
\end{frame}

\begin{frame}[fragile]
  \frametitle{Dış Katma}

  \begin{itemize}
    \item iç katma: eşleşmeyen satırlar dahil edilmez
    
    \medskip
    \item \alert{dış katma}: eşleşmeyen satırlar dahil edilir
    \item diğer tablodan gelen sütunlar boş
    \begin{lstlisting}
SELECT ...
  FROM table1_name [ LEFT | RIGHT | FULL ]
         [ OUTER ] JOIN table2_name
  ...
    \end{lstlisting}
  \end{itemize}
\end{frame}

\begin{frame}[fragile]
  \frametitle{Dış Katma Örnekleri}

  \begin{itemize}
    \item soldan dış katma
    \begin{columns}[t]
      \column{.5\textwidth}
      \begin{tiny}
      \begin{table}
        \caption{T1}
        \begin{tabular}{|r|l|}\hline
NUM & NAME\\\hline\hline
  1 & a   \\\hline
  2 & b   \\\hline
  3 & c   \\\hline
        \end{tabular}
      \end{table}
      \end{tiny}

      \column{.5\textwidth}
      \begin{tiny}
      \begin{table}
        \caption{T2}
        \begin{tabular}{|r|l|}\hline
NUM & VALUE\\\hline\hline
  1 & xxx  \\\hline
  3 & yyy  \\\hline
  5 & zzz  \\\hline
        \end{tabular}
      \end{table}
      \end{tiny}
    \end{columns}

    \pause
    \begin{center}
      \begin{tiny}
      \begin{table}
        \caption{SELECT * FROM T1 LEFT JOIN T2}
        \begin{tabular}{|r|l|r|l|}\hline
NUM & NAME & NUM & VALUE\\\hline\hline
  1 & a    &   1 & xxx  \\\hline
  2 & b    &     &      \\\hline
  3 & c    &   3 & yyy  \\\hline
        \end{tabular}
      \end{table}
      \end{tiny}
    \end{center}
  \end{itemize}
\end{frame}

\begin{frame}[fragile]
  \frametitle{Dış Katma Örnekleri}

  \begin{itemize}
    \item sağdan dış katma
    \begin{columns}[t]
      \column{.5\textwidth}
      \begin{tiny}
      \begin{table}
        \caption{T1}
        \begin{tabular}{|r|l|}\hline
NUM & NAME\\\hline\hline
  1 & a   \\\hline
  2 & b   \\\hline
  3 & c   \\\hline
        \end{tabular}
      \end{table}
      \end{tiny}

      \column{.5\textwidth}
      \begin{tiny}
      \begin{table}
        \caption{T2}
        \begin{tabular}{|r|l|}\hline
NUM & VALUE\\\hline\hline
  1 & xxx  \\\hline
  3 & yyy  \\\hline
  5 & zzz  \\\hline
        \end{tabular}
      \end{table}
      \end{tiny}
    \end{columns}

    \pause
    \begin{center}
      \begin{tiny}
      \begin{table}
        \caption{SELECT * FROM T1 RIGHT JOIN T2}
        \begin{tabular}{|r|l|r|l|}\hline
NUM & NAME & NUM & VALUE\\\hline\hline
  1 & a    &   1 & xxx  \\\hline
  3 & c    &   3 & yyy  \\\hline
    &      &   5 & zzz  \\\hline
        \end{tabular}
      \end{table}
      \end{tiny}
    \end{center}
  \end{itemize}
\end{frame}

\begin{frame}[fragile]
  \frametitle{Dış Katma Örnekleri}

  \begin{itemize}
    \item çift taraflı dış katma
    \begin{columns}[t]
      \column{.5\textwidth}
      \begin{tiny}
      \begin{table}
        \caption{T1}
        \begin{tabular}{|r|l|}\hline
NUM & NAME\\\hline\hline
  1 & a   \\\hline
  2 & b   \\\hline
  3 & c   \\\hline
        \end{tabular}
      \end{table}
      \end{tiny}

      \column{.5\textwidth}
      \begin{tiny}
      \begin{table}
        \caption{T2}
        \begin{tabular}{|r|l|}\hline
NUM & VALUE\\\hline\hline
  1 & xxx  \\\hline
  3 & yyy  \\\hline
  5 & zzz  \\\hline
        \end{tabular}
      \end{table}
      \end{tiny}
    \end{columns}

    \pause
    \begin{center}
      \begin{tiny}
      \begin{table}
        \caption{SELECT * FROM T1 FULL JOIN T2}
        \begin{tabular}{|r|l|r|l|}\hline
NUM & NAME & NUM & VALUE\\\hline\hline
  1 & a    &   1 & xxx  \\\hline
  2 & b    &     &      \\\hline
  3 & c    &   3 & yyy  \\\hline
    &      &   5 & zzz  \\\hline
        \end{tabular}
      \end{table}
      \end{tiny}
    \end{center}
  \end{itemize}
\end{frame}

\begin{frame}[fragile]
  \frametitle{Sorgulama Örnekleri}

  \begin{itemize}
    \item hiçbir oyuncusu bilinmeyen filmlerin başlıkları
    \begin{lstlisting}
SELECT TITLE
  FROM MOVIE LEFT JOIN CASTING
         ON (MOVIEID = MOVIE.ID)
  WHERE (ACTORID IS NULL)
    \end{lstlisting}
  \end{itemize}
\end{frame}

\begin{frame}[fragile]
  \frametitle{Kendisiyle Katma}

  \begin{itemize}
    \item katılmak istenen sütunlar aynı tablodaysa
    \item deyimde tabloya yeni isim vererek
  \end{itemize}
\end{frame}

\begin{frame}[fragile]
  \frametitle{Sorgulama Örnekleri}

  \begin{itemize}
    \item aynı sayıda oy almış filmlerin başlıkları
    
    \medskip
    \lstinline!SELECT M1.TITLE, M2.TITLE!\\
    \lstinline!  FROM MOVIE AS M1, MOVIE AS M2!\\
    \lstinline!  WHERE (M1.VOTES = M2.VOTES)!\\
    \pause
    \lstinline!     AND (M1.ID < M2.ID)!
  \end{itemize}
\end{frame}

\subsection{Altsorgular}

\begin{frame}[fragile]
  \frametitle{Altsorgular}

  \begin{itemize}
    \item altsorgu sonuçlarının koşul deyiminde kullanılması
    \begin{lstlisting}
SELECT ...
  WHERE expression operator
      [ ALL | ANY ] (subquery)
  ...
    \end{lstlisting}

    \item altsorgu sonucunun satır ve sütun sayıları uygun olmalı
    \item \lstinline!ALL!: altsorgudan gelen bütün değerler için
    \item \lstinline!ANY!: altsorgudan gelen en az bir değer için
  \end{itemize}
\end{frame}

\begin{frame}[fragile]
  \frametitle{Sorgulama Örnekleri}

  \begin{itemize}
    \item "Star Wars" filminden daha yüksek puanlı filmlerin\\
                başlıkları ve puanları, puana göre azalan sırada
    \medskip
\uncover<3->{
\lstinline!SELECT TITLE, SCORE FROM MOVIE!\\
~~~~\lstinline!WHERE (!
}
\uncover<2->{
\lstinline!SCORE >!\\
~~~~~~~~\lstinline!(!
}
\lstinline!SELECT SCORE FROM MOVIE!\\
~~~~~~~~~~~~~~\lstinline!WHERE (TITLE = 'Star Wars')!
\uncover<2->{
\lstinline!)!\\
}
\uncover<3->{
\lstinline!) ORDER BY SCORE DESC!
}
  \end{itemize}
\end{frame}

\begin{frame}[fragile]
  \frametitle{Sorgulama Örnekleri}

  \begin{itemize}
    \item bütün "Police Academy" filmlerinin puanlarından\\
                daha düşük puana sahip filmlerin başlıkları
                
    \medskip
\uncover<3->{
\lstinline!SELECT TITLE FROM MOVIE!\\
~~~~\lstinline!WHERE (!
}
\uncover<2->{
\lstinline!SCORE < ALL!\\
~~~~~~~~\lstinline!(!
}
\lstinline!SELECT SCORE FROM MOVIE!\\
~~~~~~~~~~~~~~\lstinline!WHERE (TITLE LIKE 'Police Academy%')!
\uncover<2->{
\lstinline!)!\\
}
\uncover<3->{
\lstinline!)!
}
  \end{itemize}
\end{frame}

\begin{frame}[fragile]
  \frametitle{Sorgulama Örnekleri}

  \begin{itemize}
    \item 1930 yılından önce çekilmiş herhangi bir filmin aldığından\\
                daha az oy almış filmlerin başlıkları
    
    \medskip
\uncover<3->{
\lstinline!SELECT TITLE FROM MOVIE!\\
~~~~\lstinline!WHERE ((YR >= 1930) AND (!
}
\uncover<2->{
\lstinline!VOTES < ANY (!\\
~~~~~~~~\lstinline!(!
}
\lstinline!SELECT VOTES FROM MOVIE!\\
~~~~~~~~~~~~~~\lstinline!WHERE (YR < 1930)!
\uncover<2->{
\lstinline!)!\\
}
\uncover<3->{
\lstinline!))!
}
  \end{itemize}
\end{frame}

\begin{frame}[fragile]
  \frametitle{Sorgulama Örnekleri}

  \begin{itemize}
    \item Johnny Depp ile oynamış oyuncuların isimleri
     \medskip
\uncover<2->{
\lstinline!SELECT NAME FROM PERSON, CASTING!\\
~~~~\lstinline!WHERE ((ACTORID = PERSON.ID)!\\
~~~~~~~~~\lstinline!AND (MOVIEID IN (!\\
}
~~~~~~~~~~~~~~~~~\lstinline!SELECT MOVIEID!\\
~~~~~~~~~~~~~~~~~~~~~\lstinline!FROM PERSON, CASTING!\\
~~~~~~~~~~~~~~~~~~~~~\lstinline!WHERE ((ACTORID = PERSON.ID)!\\
~~~~~~~~~~~~~~~~~~~~~~~~~~\lstinline!AND (NAME = 'Johnny Depp'))!\\
\uncover<2->{
\lstinline!) ))!
}
  \end{itemize}
\end{frame}

\begin{frame}[fragile]
  \frametitle{Sorgulama Örnekleri}

  \begin{itemize}
    \item en az 10 başrol oynamış oyuncuların isimleri
    \medskip
\uncover<4->{
\lstinline!SELECT NAME FROM PERSON!\\
~~~~\lstinline!WHERE (ID IN (!\\
}
~~~~~~~~~~~~\lstinline!SELECT ACTORID FROM CASTING!\\
~~~~~~~~~~~~~~~~\lstinline!WHERE (ORD = 1)!\\
\pause
~~~~~~~~~~~~~~~~\lstinline!GROUP BY ACTORID!\\
\pause
~~~~~~~~~~~~~~~~\lstinline!HAVING (COUNT(MOVIEID) >= 10)!\\
\uncover<4->{
\lstinline!) )!
}
  \end{itemize}
\end{frame}

\subsection{Küme İşlemleri}

\begin{frame}
  \frametitle{Küme İşlemleri}

  \begin{itemize}
    \item iki altsorgu sonucu üzerinde işlem

    \medskip
    \item kesişim: \lstinline!INTERSECT!
    \item birleşim: \lstinline!UNION!
    \item fark: \lstinline!EXCEPT!

    \medskip
    \item sonuç tablolarında tekrarlı satırlar bulunmaz
  \end{itemize}
\end{frame}

\begin{frame}[fragile]
  \frametitle{Sorgulama Örnekleri}

  \begin{itemize}
    \item hem yönetmenlik hem de oyunculuk yapmış olanların sayısı
    \medskip
\uncover<3->{
\lstinline!SELECT COUNT(*) FROM (!\\
}
\uncover<2->{
~~~~\lstinline!(!
}
\lstinline!SELECT DISTINCT DIRECTORID FROM MOVIE!
\uncover<2->{
\lstinline!)!\\
~~~~\lstinline!INTERSECT!\\
~~~~\lstinline!(!
}
\lstinline!SELECT DISTINCT ACTORID FROM CASTING!
\uncover<2->{
\lstinline!)!\\
}
\uncover<3->{
\lstinline!) AS DIRECTOR_ACTOR!
}
  \end{itemize}
\end{frame}


\begin{frame}[fragile]
  \frametitle{Sorgulama Örnekleri}

  \begin{itemize}
    \item 1930'dan önce çekilmiş filmlerde çalışan kişilerin sayısı
    
    \medskip
\uncover<3->{
\lstinline!SELECT COUNT(*) FROM (!\\
}
\uncover<2->{
~~~~\lstinline!(!
}
\lstinline!SELECT DISTINCT DIRECTORID FROM MOVIE!\\
~~~~~~~~~~~\lstinline!WHERE (YR < 1930)!
\uncover<2->{
\lstinline!)!\\
~~~~\lstinline!UNION!\\
~~~~\lstinline!(!
}
\lstinline!SELECT DISTINCT ACTORID FROM CASTING!\\
~~~~~~~~~~~\lstinline!WHERE (MOVIEID IN!\\
~~~~~~~~~~~~~~~~~~~~~~~~\lstinline!( SELECT ID FROM MOVIE!\\
~~~~~~~~~~~~~~~~~~~~~~~~~~~~~~~\lstinline!WHERE (YR < 1930) ))!
\uncover<2->{
\lstinline!)!\\
}
\uncover<3->{
\lstinline!) AS OLD_MOVIE_PERSON_IDS!
}
  \end{itemize}
\end{frame}

\begin{frame}[fragile]
  \frametitle{Sorgulama Örnekleri}

  \begin{itemize}
    \item oyunculuk yapmamış yönetmenlerin sayısı
    \medskip
\uncover<3->{
\lstinline!SELECT COUNT(*) FROM (!\\
}
\uncover<2->{
~~~~\lstinline!(!
}
\lstinline!SELECT DISTINCT DIRECTORID FROM MOVIE!
\uncover<2->{
\lstinline!)!\\
~~~~\lstinline!EXCEPT!\\
~~~~\lstinline!(!
}
\lstinline!SELECT DISTINCT ACTORID FROM CASTING!
\uncover<2->{
\lstinline!)!\\
}
\uncover<3->{
\lstinline!) AS DIRECTOR_ONLY!
}
  \end{itemize}
\end{frame}

\subsection*{Kaynaklar}

\begin{frame}
  \frametitle{Kaynaklar}
  
  \begin{block}{Okunacak: Date}
    \begin{itemize}
      \item Chapter 8: Relational Calculus
      \begin{itemize}
        \item 8.6. \alert{SQL Facilities}
      \end{itemize}

      \item Appendix B: \alert{SQL Expressions}
      \item Chapter 19: Missing Information
    \end{itemize}
  \end{block}
  
  \begin{block}{Yardımcı Kaynak}
    \begin{itemize}
      \item A Gentle Introduction to SQL:\\
        \url{http://sqlzoo.net/}
    \end{itemize}
  \end{block}
\end{frame}

\end{document}
