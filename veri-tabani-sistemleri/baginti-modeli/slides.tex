% Copyright (c) 2002-2016
%       H. Turgut Uyar <uyar@itu.edu.tr>
%       Şule Gündüz Öğüdücü <sgunduz@itu.edu.tr>
%
% This work is licensed under a "Creative Commons
% Attribution-NonCommercial-ShareAlike 4.0 International License".
% For more information, please visit:
% https://creativecommons.org/licenses/by-nc-sa/4.0/

\documentclass[dvipsnames]{beamer}

\usepackage{ae}
\usepackage[scaled=0.88]{beramono}
\usepackage[T1]{fontenc}
\usepackage[utf8]{inputenc}
\usepackage[turkish]{babel}
\setbeamersize{text margin left=2em, text margin right=2em}
\usepackage[labelformat=empty, aboveskip=1pt, belowskip=1pt]{caption}

\usepackage{listings}
\lstdefinelanguage{TutorialD}[]{}{
  morekeywords={AND, AS, BASE, BOOL, CAST_AS_, CAST_AS_RATIONAL, CHAR,
                CONSTRAINT, DELETE, DIVIDEBY, DROP, INSERT, INTEGER, INTERSECT,
                JOIN, KEY, MINUS, OR, PER, POSSREP, RATIONAL, RELATION, RENAME,
                THE_, TUPLE, TYPE, UNION, UPDATE, VAR, WHERE}
}
\lstdefinelanguage{FullSQL}[]{SQL}{
  morekeywords={BINARY, BOOLEAN, CYCLE, FINAL, INCREMENT, IS, LARGE, MAXVALUE,
                MINVALUE, NO_ACTION, OBJECT, REFERENCES, RENAME, SEQUENCE,
                START, TO, TYPE, VACUUM}
}
\lstset{basicstyle=\ttfamily, keywordstyle=\color{ForestGreen},
        showstringspaces=false}

\mode<presentation>
{
  \usetheme{Warsaw}
  \usecolortheme[named=ForestGreen]{structure}
  \setbeamercovered{transparent}
}


\title{Veri Tabanı Sistemleri}
\subtitle{Bağıntı Modeli}

\author{H. Turgut Uyar \and Şule Öğüdücü}
\date{2002-2016}

\AtBeginSubsection[]{
  \begin{frame}<beamer>
    \frametitle{Konular}
    \tableofcontents[currentsection,currentsubsection]
  \end{frame}
}

\theoremstyle{plain}

\pgfdeclareimage[width=2cm]{license}{../license}


\begin{document}

\begin{frame}
  \titlepage
\end{frame}

\begin{frame}
 \frametitle{License}

  \pgfuseimage{license}\hfill
  \copyright~2002-2016 T. Uyar, Ş. Öğüdücü

  \vfill
  \begin{footnotesize}
    You are free to:
    \begin{itemize}
      \itemsep0em
      \item Share -- copy and redistribute the material in any medium or format
      \item Adapt -- remix, transform, and build upon the material
    \end{itemize}

    Under the following terms:
    \begin{itemize}
      \itemsep0em
      \item Attribution -- You must give appropriate credit, provide a link to
        the license, and indicate if changes were made.

      \item NonCommercial -- You may not use the material for commercial
        purposes.

      \item ShareAlike -- If you remix, transform, or build upon the material,
        you must distribute your contributions under the same license as the
        original.
    \end{itemize}
  \end{footnotesize}

  \begin{small}
    For more information:\\
    \url{https://creativecommons.org/licenses/by-nc-sa/4.0/}

    \smallskip
    Read the full license:\\
    \url{https://creativecommons.org/licenses/by-nc-sa/4.0/legalcode}
  \end{small}
\end{frame}

\begin{frame}
  \frametitle{Konular}
  \tableofcontents
\end{frame}

\lstset{language=TutorialD}

\section{Bağıntı Modeli}

\subsection{Giriş}

\begin{frame}
  \frametitle{Bağıntı Modeli}

  \begin{itemize}
    \item Dr. E. F. Codd, 1970
    \item veri \alert{bağıntılar} şeklinde modellenir:\\
      $\alpha \subseteq A \times B \times C \times ...$

    \pause
    \medskip
    \item bağıntılar \alert{bağıntı değişkenlerine} atanır
    \item bağıntının her elemanı bir \alert{çoklu}
    \item elemanların her verisi bir \alert{nitelik}

    \pause
    \medskip
    \item bağıntılar tablolarla temsil edilir
    \item bağıntı $\rightarrow$ tablo, çoklu $\rightarrow$ satır,
        nitelik $\rightarrow$ sütun
 
  \end{itemize}
\end{frame}

\begin{frame}
  \frametitle{Bağıntı Örneği}

  \begin{footnotesize}
  \begin{table}
    \caption{MOVIE}
    \begin{tabular}{|l|r|l|r|r|}\hline
TITLE                & YEAR & DIRECTOR      & SCORE & VOTES\\\hline\hline
The Usual Suspects   & 1995 & Bryan Singer  &   8.7 &  3502\\\hline
Suspiria             & 1977 & Dario Argento &   7.1 &  1004\\\hline
Being John Malkovich & 1999 & Spike Jonze   &   8.3 & 13809\\\hline
...                  &  ... & ...           &   ... &   ...\\\hline
    \end{tabular}
  \end{table}
  \end{footnotesize}

    \begin{itemize}
      \item bağıntı değişkeninin adı \texttt{MOVIE}
      \item \texttt{(Usual Suspects, 1995, Bryan Singer, 8.7, 3502)}\\
        film bağıntısının bir çoklusu
      \item \texttt{YEAR}, film bağıntısının bir niteliği
    \end{itemize}

\end{frame}

\begin{frame}
  \frametitle{Bağıntı Yapısı}

     \begin{itemize}
      \item bağıntı başlığı: bağıntının nitelikleri kümesi
      \item veri tanımlama dili komutlarından etkilenir

  \medskip
      \item bağıntı gövdesi: bağıntıdaki çoklular kümesi
      \item veri işleme dili komutlarından etkilenir
    \end{itemize}
\end{frame}

\begin{frame}
  \frametitle{Bağıntı Yüklemi}
  \begin{itemize}
    \item \alert{bağıntı yüklemi}: bağıntının "anlamını" ifade eden cümle
   \end{itemize}

\medskip
  \begin{exampleblock}{örnek}
    \begin{itemize}
      \item "\texttt{TITLE} başlıklı film \texttt{YEAR} yılında çekilmiştir.\\
	 \texttt{DIRECTOR} tarafından yönetilmiştir.\\
	 Verilen \texttt{VOTES} oyun ortalaması \texttt{SCORE}'dur."
    \end{itemize}
  \end{exampleblock}
   
\end{frame}

\begin{frame}
  \frametitle{Çoklu Değerleri}

   \begin{itemize}
      \item her çoklu yükleme göre \emph{Doğru} ya da \emph{Yanlış} değerini alır
    \end{itemize}
    

    \begin{exampleblock}{örnek: MOVIE bağıntısı}
    \begin{itemize}
      \item \texttt{(Suspiria, 1977, Dario Argento, 1004, 7.1)}\\
        çoklusu doğrudur
      \item \texttt{(Suspiria, 1978, Dario Argento, 1004, 7.1)}\\
        çoklusu yanlıştır
    \end{itemize}
  \end{exampleblock}
\end{frame}
%---------------------------------

\begin{frame}
  \frametitle{Çoklu Sırası}

  \begin{itemize}
    \item çokluların sırası önemsizdir
  \end{itemize}

  \medskip
  \begin{exampleblock}{örnek}
    \begin{itemize}
      \item şu iki bağıntı eşdeğerlidir:
    \end{itemize}

    \begin{columns}
      \column{.5\textwidth}
      \begin{footnotesize}
      \begin{table}
        %\caption{MOVIE}
        \begin{tabular}{|l|l|}\hline
TITLE                & ...\\\hline\hline
The Usual Suspects   & ...\\\hline
Suspiria             & ...\\\hline
Being John Malkovich & ...\\\hline
        \end{tabular}
      \end{table}
      \end{footnotesize}

      \column{.5\textwidth}
      \begin{footnotesize}
      \begin{table}
        %\caption{MOVIE}
        \begin{tabular}{|l|l|}\hline
TITLE                & ...\\\hline\hline
Suspiria             & ...\\\hline
Being John Malkovich & ...\\\hline
The Usual Suspects   & ...\\\hline
        \end{tabular}
      \end{table}
      \end{footnotesize}
    \end{columns}
  \end{exampleblock}
\end{frame}

\begin{frame}
  \frametitle{Nitelik Sırası}

  \begin{itemize}
    \item niteliklerin sırası önemsizdir
  \end{itemize}

  \medskip
  \begin{exampleblock}{örnek}
    \begin{itemize}
      \item şu iki bağıntı eşdeğerlidir:
    \end{itemize}

     \begin{columns}
      \column{.45\textwidth}
      \begin{footnotesize}
      \begin{table}
        %\caption{MOVIE}
        \begin{tabular}{|l|r|l|}\hline
TITLE              & YEAR & ...\\\hline\hline
The Usual Suspects & 1995 & ...\\\hline
Suspiria           & 1977 & ...\\\hline
        \end{tabular}
      \end{table}
      \end{footnotesize}

      \column{.45\textwidth}
      \begin{footnotesize}
      \begin{table}
        %\caption{MOVIE}
        \begin{tabular}{|r|l|l|}\hline
YEAR & TITLE              & ...\\\hline\hline
1995 & The Usual Suspects & ...\\\hline
1977 & Suspiria           & ...\\\hline
        \end{tabular}
      \end{table}
      \end{footnotesize}
    \end{columns}
  \end{exampleblock}
\end{frame}

\begin{frame}
  \frametitle{Eş Çoklular}

  \begin{itemize}
    \item bir bağıntıda birbirinin eşi çoklular bulunamaz
     \item her çoklu diğerlerinden ayırt edilebilmelidir
  \end{itemize}

  \medskip
 \begin{exampleblock}{örnek}
    \begin{footnotesize}
    \begin{table}
      %\caption{MOVIE}
      \begin{tabular}{|l|r|l|r|r|}\hline
TITLE                & YEAR & DIRECTOR      & SCORE & VOTES\\\hline\hline
The Usual Suspects   & 1995 & Bryan Singer  &   8.7 &  3502\\\hline
Suspiria             & 1977 & Dario Argento &   7.1 &  1004\\\hline
Being John Malkovich & 1999 & Spike Jonze   &   8.3 & 13809\\\hline
...                  &  ... & ...           &   ... &   ...\\\hline
Suspiria             & 1977 & Dario Argento &   7.1 &  1004\\\hline
...                  &  ... & ...           &   ... &   ...\\\hline
      \end{tabular}
    \end{table}
    \end{footnotesize}

    \begin{picture}(0, 0)
      \color[rgb]{1, 0.2, 0.1}
      \thicklines
      \put(5, 30){\line(0, 0){35}}
      \put(5, 30){\vector(1, 0){15}}
      \put(5, 65){\vector(1, 0){15}}
    \end{picture}
  \end{exampleblock}
\end{frame}

\begin{frame}
  \frametitle{Tanım Kümeleri}

  \begin{itemize}
    \item aynı niteliğe ilişkin değerler aynı tanım kümesinden seçilmeli
    \item karşılaştırma işlemi ancak aynı tanım kümesinden seçilmiş\\
        değerler arasında anlamlıdır
 
    \medskip
    \item pratikte veri tipleri kullanılır
  \end{itemize}
\end{frame}

\begin{frame}
  \frametitle{Tanım Kümesi Örneği}

    \begin{itemize}
      \item \texttt{TITLE} başlıklar kümesinden, \texttt{YEAR} yıllar
        kümesinden,\\
        \texttt{DIRECTOR} yönetmenler kümesinden, \ldots

      \item veri tipi kullanılırsa:\\
        \texttt{TITLE} katar, \texttt{YEAR} tamsayı, \texttt{DIRECTOR} katar,
         \ldots
         
         \medskip
       \item \texttt{DIRECTOR} niteliğine \texttt{"Suspiria"} değerini
          vermek\\
          veri tipi açısından doğru, yüklem açısından yanlış

        \item \texttt{YEAR} ve \texttt{VOTES} değerleri birer tamsayı\\
          ancak bunları karşılaştırmak anlamlı değil    
    \end{itemize}
  \end{frame}

\begin{frame}
  \frametitle{Nitelik Değerleri}

  \begin{itemize}
    \item niteliklere verilen değerler tek boyutlu olmalıdır
    \item dizi, liste, kayıt gibi değerlere izin verilmez
  \end{itemize}
  
  \begin{exampleblock}{örnek: birden fazla yönetmen}
    \begin{footnotesize}
    \begin{table}
%       \caption{MOVIE}
      \begin{tabular}{|l|l|l|l|}\hline
TITLE      & ... & DIRECTORS                      & ...\\[2pt]\hline\hline
...        & ... & ...                            & ...\\\hline
The Matrix & ... & Andy Wachowski, Lana Wachowski & ...\\\hline
...        & ... & ...                            & ...\\\hline
      \end{tabular}
    \end{table}
    \end{footnotesize}

    \pause
    \begin{picture}(0, 0)
      \color[rgb]{1, 0.2, 0.1}
      \thicklines
      \only<2->{
        \put(115, 30){\line(1, 0){145}}
      }
    \end{picture}
  \end{exampleblock}
\end{frame}

\begin{frame}
  \frametitle{Boş Değer}

  \begin{columns}[t]
    \column{.5\textwidth}
    \begin{itemize}
      \item çoklu için o niteliğin\\
	değeri bilinmiyor
    \end{itemize}
    
    \medskip
    \begin{exampleblock}{örnek}
      \begin{itemize}
        \item filmin yönetmeni\\
	  bilinmiyor
      \end{itemize}
    \end{exampleblock}

    \pause
    \column{.5\textwidth}
    \begin{itemize}
      \item çoklu o nitelik için\\
	bir değer taşımıyor
    \end{itemize}
    
     \medskip
    \begin{exampleblock}{örnek}
      \begin{itemize}
        \item film için oy kullanılmamış,\\
          o yüzden \texttt{SCORE} yok
      \end{itemize}
    \end{exampleblock}
  \end{columns}
\end{frame}


\begin{frame}
  \frametitle{Varsayılan Değer}

  \begin{itemize}
    \item boş değer yerine varsayılan bir değer kullanılabilir
    \item niteliğin alabileceği geçerli değerlerden biri olmamalı
  \end{itemize}
  
   \medskip
  \begin{exampleblock}{örnek}
    \begin{itemize}
      \item \texttt{SCORE} niteliği 1.0 ile 10.0 arasında değer alıyorsa\\
	varsayılan değeri \texttt{0.0} seçilebilir
    \end{itemize}
  \end{exampleblock}
\end{frame}

\subsection{Anahtarlar}

\begin{frame}
  \frametitle{Anahtarlar}

  \begin{itemize}
    \item $B$ bağıntının bütün nitelikleri kümesi olsun\\
      ve $A \subseteq B$ olsun
      
    \smallskip  

    \item $A$'nın bir anahtar adayı olabilmesi için\\
      şu koşullar sağlanmalı:

    \item \alert{eşsizlik}: herhangi iki çoklu $A$'da yer alan\\
        bütün nitelikler için aynı değeri taşımazlar

    \item \alert{indirgenemezlik}: $A$'nın hiçbir altkümesi\\
        eşsizlik özelliğini sağlamaz

    \pause
    \bigskip
    \item her bağıntının en az bir anahtar adayı vardır
  \end{itemize}
\end{frame}

\begin{frame}
  \frametitle{Anahtar Adayı Örneği}

    \begin{itemize}
      \item \texttt{\{TITLE\}} ?

      \pause
      \item \texttt{\{TITLE, YEAR\}} ?

      \pause
      \item \texttt{\{TITLE, DIRECTOR\}} ?

      \pause
      \item \texttt{\{TITLE, YEAR, DIRECTOR\}} ?
    \end{itemize}
\end{frame}

\begin{frame}
  \frametitle{Anahtar Eşdeğeri}

  \begin{itemize}
    \item bir \alert{doğal anahtar} bulunamıyorsa\\
      bir \alert{anahtar eşdeğeri} tanımlanabilir

    \pause
    \medskip
    \item kimlik niteliği
    \item değerinin ne olduğunun önemi yok
    \item sistem tarafından üretilebilir
  \end{itemize}
\end{frame}

\begin{frame}
  \frametitle{Anahtar Eşdeğeri Örneği}

 \begin{footnotesize}
    \begin{table}
      %\caption{MOVIE}
      \begin{tabular}{|r|l|r|l|r|r|}\hline
MOVIE\# & TITLE                & YEAR & DIRECTOR      & SCORE & VOTES\\\hline\hline
    ... & ...                  &  ... & ...           &   ... &   ...\\\hline
      6 & The Usual Suspects   & 1995 & Bryan Singer  &   ... &   ...\\\hline
   1512 & Suspiria             & 1977 & Dario Argento &   ... &   ...\\\hline
     70 & Being John Malkovich & 1999 & Spike Jonze   &   ... &   ...\\\hline
    ... & ...                  &  ... & ...           &   ... &   ...\\\hline
      \end{tabular}
    \end{table}
    \end{footnotesize}
    
    \begin{itemize}
      \item \texttt{\{MOVIE\#\}} anahtar adayıdır
      \item \texttt{\{MOVIE\#, TITLE\}} anahtar adayı değildir
    \end{itemize}
\end{frame}

\begin{frame}
  \frametitle{Birincil Anahtar}

  \begin{itemize}
    \item bir bağıntının birden fazla anahtar adayı varsa biri \alert{birincil anahtar} seçilir
    \item diğerleri anahtar seçeneği olur
    \item birincil anahtarı oluşturan nitelikler altı çizili gösterilir

    \pause
    \medskip
    \item birincil anahtarın parçası olan hiçbir niteliğin değeri\\
      hiçbir çokluda boş olamaz
    \item her bağıntının bir birincil anahtarı bulunması zorunludur
    
  \end{itemize}
\end{frame}

\begin{frame}
  \frametitle{Birincil Anahtar Örneği}
\begin{footnotesize}
  \begin{table}
    %\caption{MOVIE}
    \begin{tabular}{|r|l|r|l|r|r|}\hline
\underline{MOVIE\#} & TITLE & YEAR & DIRECTOR      & SCORE & VOTES\\[2pt]\hline\hline
 ... & ...                  &  ... & ...           &   ... &   ...\\\hline
   6 & The Usual Suspects   & 1995 & Bryan Singer  &   ... &   ...\\\hline
1512 & Suspiria             & 1977 & Dario Argento &   ... &   ...\\\hline
  70 & Being John Malkovich & 1999 & Spike Jonze   &   ... &   ...\\\hline
 ... & ...                  &  ... & ...           &   ... &   ...\\\hline
    \end{tabular}
  \end{table}
  \end{footnotesize}
\end{frame}

\subsection{Başvuru Bütünlüğü}

\begin{frame}
  \frametitle{Tek Boyutluluk Örneği}

  \begin{itemize}
    \item oyuncular nasıl tutulacak?

    \begin{footnotesize}
    \begin{table}
      \caption{MOVIE}
      \begin{tabular}{|r|l|c|l|}\hline
\underline{MOVIE\#} & TITLE    & ... & ACTORS                      \\[2pt]\hline\hline
      6 & The Usual Suspects   & ... & Gabriel Byrne               \\\hline
    ... & ...                  & ... & ...                         \\\hline
     70 & Being John Malkovich & ... & Cameron Diaz, John Malkovich\\\hline
    ... & ...                  & ... & ...                         \\\hline
      \end{tabular}
    \end{table}
    \end{footnotesize}

    \pause
    \begin{picture}(0, 0)
      \color[rgb]{1, 0.2, 0.1}
      \thicklines
      \only<2->{
        \put(170, 25){\line(1, 0){125}}
      }
    \end{picture}
  \end{itemize}
\end{frame}

\begin{frame}
  \frametitle{Tek Boyutluluk Örneği}

  \begin{itemize}
    \item Tek boyutluluk için çokluların tekrar edilmesi gerekir
  \end{itemize}

  \begin{footnotesize}
  \begin{table}
    \caption{MOVIE}
    \begin{tabular}{|r|l|c|l|}\hline
MOVIE\# & TITLE                & ... & ACTOR         \\\hline\hline
      6 & The Usual Suspects   & ... & Gabriel Byrne \\\hline
    ... & ...                  & ... & ...           \\\hline
     70 & Being John Malkovich & ... & Cameron Diaz  \\\hline
     70 & Being John Malkovich & ... & John Malkovich\\\hline
    ... & ...                  & ... & ...           \\\hline
    \end{tabular}
  \end{table}
  \end{footnotesize}
\end{frame}

\begin{frame}
  \frametitle{Tek Boyutluluk Örneği}

  \begin{footnotesize}
  \begin{table}
    \caption{MOVIE}
    \begin{tabular}{|r|l|c|}\hline
\underline{MOVIE\#} & TITLE                & ...\\[2pt]\hline\hline
                  6 & The Usual Suspects   & ...\\\hline
               1512 & Suspiria             & ...\\\hline
                 70 & Being John Malkovich & ...\\\hline
                ... & ...                  & ...\\\hline
    \end{tabular}
  \end{table}
  \end{footnotesize}

  \vspace{-12pt}
  \begin{columns}[t]
    \column{.5\textwidth}
    \begin{footnotesize}
    \begin{table}
      \caption{ACTOR}
      \begin{tabular}{|r|l|}\hline
\underline{ACTOR\#} & NAME          \\[2pt]\hline\hline
                308 & Gabriel Byrne \\\hline
                282 & Cameron Diaz  \\\hline
                503 & John Malkovich\\\hline
                ... & ...           \\\hline
      \end{tabular}
    \end{table}
    \end{footnotesize}

    \column{.5\textwidth}
    \begin{footnotesize}
    \begin{table}
      \caption{CASTING}
      \begin{tabular}{|r|r|r|}\hline
\underline{MOVIE\#} & \underline{ACTOR\#} & ORD\\[2pt]\hline\hline
                  6 &                 308 &   2\\\hline
                 70 &                 282 &   2\\\hline
                 70 &                 503 &  14\\\hline
                ... &                 ... & ...\\\hline
      \end{tabular}
    \end{table}
    \end{footnotesize}
  \end{columns}
\end{frame}

\begin{frame}
  \frametitle{Tek Boyutluluk Örneği}

  \vspace{-12pt}
  \begin{footnotesize}
  \begin{table}
    \caption{MOVIE}
    \begin{tabular}{|r|l|c|r|}\hline
\underline{MOVIE\#} & TITLE                & ... & DIRECTOR\#\\[2pt]\hline\hline
                  6 & The Usual Suspects   & ... &        639\\\hline
               1512 & Suspiria             & ... &       2259\\\hline
                 70 & Being John Malkovich & ... &       1485\\\hline
                ... & ...                  & ... &        ...\\\hline
    \end{tabular}
  \end{table}
  \end{footnotesize}

  \vspace{-24pt}
  \begin{columns}[t]
    \column{.5\textwidth}
    \begin{footnotesize}
    \begin{table}
      \caption{PERSON}
      \begin{tabular}{|r|l|}\hline
\underline{PERSON\#} & NAME          \\[2pt]\hline\hline
                 308 & Gabriel Byrne \\\hline
                1485 & Spike Jonze   \\\hline
                 639 & Bryan Singer  \\\hline
                 282 & Cameron Diaz  \\\hline
                2259 & Dario Argento \\\hline
                 503 & John Malkovich\\\hline
                 ... & ...           \\\hline
      \end{tabular}
    \end{table}
    \end{footnotesize}

    \column{.5\textwidth}
    \begin{footnotesize}
    \begin{table}
      \caption{CASTING}
      \begin{tabular}{|r|r|r|}\hline
\underline{MOVIE\#} & \underline{ACTOR\#} & ORD\\[2pt]\hline\hline
                  6 &                 308 &   2\\\hline
                 70 &                 282 &   2\\\hline
                 70 &                 503 &  14\\\hline
                ... &                 ... & ...\\\hline
      \end{tabular}
    \end{table}
    \end{footnotesize}
  \end{columns}
\end{frame}

\begin{frame}
  \frametitle{Dış Anahtarlar}

  \begin{itemize}
   \item \alert{dış anahtar}: bir bağıntının bir niteliğinin\\
      başka bir bağıntının anahtar adayı olması
  \end{itemize}
\end{frame}

\begin{frame}
  \frametitle{Dış Anahtar Örnekleri}

   \begin{columns}[t]
    \column{.63\textwidth}
    \begin{tiny}
    \begin{table}
      \caption{MOVIE}
      \begin{tabular}{|r|l|c|r|}\hline
\underline{MOVIE\#} & TITLE & ... & DIRECTOR\#\\[2pt]\hline\hline
   6 & The Usual Suspects   & ... &        639\\\hline
1512 & Suspiria             & ... &       2259\\\hline
  70 & Being John Malkovich & ... &       1485\\\hline
 ... & ...                  & ... &        ...\\\hline
      \end{tabular}
    \end{table}
    \end{tiny}

    \column{.37\textwidth}
    \begin{tiny}
    \begin{table}
      \caption{PERSON}
      \begin{tabular}{|r|l|}\hline
\underline{PERSON\#} & NAME\\[2pt]\hline\hline
 308 & Gabriel Byrne \\\hline
1485 & Spike Jonze   \\\hline
 639 & Bryan Singer  \\\hline
 282 & Cameron Diaz  \\\hline
2259 & Dario Argento \\\hline
 503 & John Malkovich\\\hline
 ... & ...           \\\hline
      \end{tabular}
    \end{table}
    \end{tiny}
  \end{columns}

  \begin{picture}(0, 0)
    \color[rgb]{0.1, 0.6, 0.1}
    \thicklines
    \only<2->{
      \put(169, 54){\oval(20, 8)}       % movie.director#=2259
      \put(250, 32){\oval(20, 8)}       % person.person#=2259
      \put(179, 54){\vector(3, -1){60}} % movie.director#=2259 -> person.person#=2259
    }
    \only<3->{
      \put(157, 73){\oval(44, 9)}       % movie.director#
      \put(242, 73){\oval(36, 9)}       % person.person#
      \put(179, 73){\vector(1, 0){44}}  % movie.director# -> person.person#
    }
  \end{picture}
\end{frame}

\begin{frame}
  \frametitle{Dış Anahtar Örnekleri}

  \begin{columns}
    \column{.63\textwidth}
    \begin{tiny}
    \begin{table}
      \caption{MOVIE}
      \begin{tabular}{|r|l|c|r|}\hline
\underline{MOVIE\#} & TITLE     & ... & DIRECTOR\#\\[2pt]\hline\hline
         6 & The Usual Suspects & ... &        639\\\hline
      1512 & Suspiria           & ... &       2259\\\hline
       ... & ...                & ... &        ...\\\hline
      \end{tabular}
    \end{table}
    \end{tiny}

    \column{.37\textwidth}
    \begin{tiny}
    \begin{table}
      \caption{PERSON}
      \begin{tabular}{|r|l|}\hline
\underline{PERSON\#} & NAME\\[2pt]\hline\hline
        308 & Gabriel Byrne\\\hline
       1485 & Spike Jonze  \\\hline
        ... & ...          \\\hline
      \end{tabular}
    \end{table}
    \end{tiny}
  \end{columns}

  \begin{tiny}
  \begin{table}
    \caption{CASTING}
    \begin{tabular}{|r|r|r|}\hline
\underline{MOVIE\#} & \underline{ACTOR\#} & ORD\\[2pt]\hline\hline
                  6 &                 308 &   2\\\hline
                 70 &                 282 &   2\\\hline
                ... &                 ... & ...\\\hline
    \end{tabular}
  \end{table}
  \end{tiny}

  \begin{picture}(0, 0)
    \color[rgb]{0.1, 0.6, 0.1}
    \thicklines
    \only<2->{
      \put(153, 112){\oval(45, 9)}        % movie.director#
      \put(243, 112){\oval(36, 9)}        % person.person#
      \put(175, 112){\vector(1, 0){49}}   % movie.director# -> person.person#
    }
    \only<3->{
      \put(128,  43){\oval(33, 9)}        % casting.movie#
      \put( 28, 112){\oval(33, 9)}        % movie.movie#
      \put(121,  47){\vector(-4, 3){78}}  % casting.movie# -> movie.movie#
    }
    \only<4->{
      \put(167,  43){\oval(32, 9)}        % casting.actor#
      \put(182,  47){\vector(3, 4){45}}   % casting.actor# -> person.person#
    }
  \end{picture}
\end{frame}

\begin{frame}
  \frametitle{Başvuru Bütünlüğü}

  \begin{itemize}
    \item \alert{başvuru bütünlüğü}:\\
      dış anahtar niteliğinin aldığı bütün değerler\\
      başvurulan bağıntının ilgili niteliğinde bulunmalı

  \pause
  \bigskip
  \item bir işlem isteği başvuru bütünlüğünü bozuyorsa:
  \smallskip
  \item izin verme
  \item işlemi etkilenen çoklulara yansıt
  \item boş değer ata
  \item varsayılan değer ata  
  \end{itemize}
\end{frame}

\begin{frame}
  \frametitle{Başvuru Bütünlüğü Örnekleri}

   \begin{columns}[t]
    \column{.55\textwidth}
    \begin{footnotesize}
    \begin{table}
      \caption{MOVIE}
      \begin{tabular}{|r|l|c|r|}\hline
\underline{MOVIE\#} & TITLE & ... & DIRECTOR\#\\[2pt]\hline\hline
                ... & ...      & ... &        ...\\\hline
               1512 & Suspiria & ... &       2259\\\hline
                ... & ...      & ... &        ...\\\hline
      \end{tabular}
    \end{table}
    \end{footnotesize}

    \column{.4\textwidth}
    \begin{footnotesize}
    \begin{table}
      \caption{PERSON}
      \begin{tabular}{|r|l|}\hline
\underline{PERSON\#} & NAME\\[2pt]\hline\hline
        ... & ...          \\\hline
       2259 & Dario Argento\\\hline
        ... & ...          \\\hline
      \end{tabular}
    \end{table}
    \end{footnotesize}
  \end{columns}
  
    \begin{itemize}
      \item \texttt{(2259, Dario Argento)} çoklusunu sil
      \item \texttt{(2259, Dario Argento)} çoklusunu\\
        \texttt{(2871, Dario Argento)} olarak değiştir
    \end{itemize}
\end{frame}

%\subsection{Veri Tipleri}

\subsection{TutorialD}

\begin{frame}
  \frametitle{Tutorial D Veri Tipleri}

  \begin{itemize}
    \item \texttt{INTEGER}
    \item \texttt{RATIONAL}
    \item \texttt{BOOL}
    \item \texttt{CHAR}
  \end{itemize}
\end{frame}

\begin{frame}[fragile]
  \frametitle{Tip Tanımlama}

  \begin{itemize}
    \item yeni bir tip tanımlama:
    \begin{lstlisting}
TYPE type_name POSSREP
  { field_name field_type
    [, ...]
    [ CONSTRAINT condition ] };
    \end{lstlisting}

    \item tip silme:
    \begin{lstlisting}
DROP TYPE type_name;
    \end{lstlisting}
  \end{itemize}
\end{frame}

\begin{frame}[fragile]
  \frametitle{Tip Tanımlama Örnekleri}

  \begin{lstlisting}
TYPE PERSON# POSSREP
  { VALUE INTEGER };

TYPE MOVIE# POSSREP
  { VALUE INTEGER };

TYPE YEAR POSSREP
  { VALUE INTEGER };

TYPE SCORE POSSREP
  { VALUE RATIONAL
    CONSTRAINT (VALUE >= 1.0)
           AND (VALUE <= 10.0) };
  \end{lstlisting}
\end{frame}

\begin{frame}[fragile]
  \frametitle{Tip İşlemleri}
  
  \begin{itemize}
    \item türetilen tip için değer üretme:
    \begin{lstlisting}
type_name(base_value [, ...])
    \end{lstlisting}
  \end{itemize}

  \medskip
  \begin{exampleblock}{example}
    \begin{itemize}
      \item bir \texttt{SCORE} değeri üretme:
      \begin{lstlisting}
SCORE(8.7)
      \end{lstlisting}
    \end{itemize}
  \end{exampleblock}
\end{frame}

\begin{frame}[fragile]
  \frametitle{Tip İşlemleri}

  \begin{itemize}
    \item bir alanın değerini alma: \lstinline!THE_! işleçleri
    \begin{lstlisting}
THE_field_name(variable_name)
    \end{lstlisting}
  \end{itemize}

  \medskip
  \begin{exampleblock}{örnek}
    \begin{itemize}
      \item bir \texttt{SCORE} değişkeninin \texttt{VALUE} alanını alma:
      \begin{lstlisting}
THE_VALUE(SCORE)
      \end{lstlisting}
    \end{itemize}
  \end{exampleblock}
\end{frame}

\begin{frame}[fragile]
  \frametitle{Tip İşlemleri}

  \begin{itemize}
    \item tip dönüşümü: \lstinline!CAST_AS_! işleçleri
    \begin{lstlisting}
CAST_AS_target_type(value)
    \end{lstlisting}
  \end{itemize}

  \medskip
  \begin{exampleblock}{örnek}
    \begin{itemize}
      \item bir tamsayı \texttt{VOTES} değerini bir \texttt{RATIONAL} değere
        dönüştürme:
      \begin{lstlisting}
CAST_AS_RATIONAL(VOTES)
      \end{lstlisting}
    \end{itemize}
  \end{exampleblock}
\end{frame}
  %------------

%  \begin{itemize}
%    \item türetilen tip için değer üretme:
%    \begin{lstlisting}
%type_name(base_value [, ...])!
%    \end{lstlisting}
%
%    \pause
%    \item bir alanın değerini alma: \lstinline!THE_! işleçleri
%    \begin{lstlisting}
%THE_field_name(variable_name)
%    \end{lstlisting}
%
%    \pause
%    \item tip dönüşümü: \lstinline!CAST_AS_! işleçleri
%    \begin{lstlisting}
%CAST_AS_target_type(value)
%    \end{lstlisting}
%
%    \pause
%    \item bir niteliğin adını değiştirme:
%    \begin{lstlisting}
%RENAME (attribute_name AS new_name)
%    \end{lstlisting}
%  \end{itemize}
%\end{frame}

%\begin{frame}[fragile]
%  \frametitle{Tip İşlemleri Örnekleri}
%
%  \begin{ornek}
%    \begin{itemize}
%      \item bir \texttt{SCORE} değeri üretme:
%      \begin{lstlisting}
%SCORE(8.7)
%      \end{lstlisting}
%
%      \pause
%      \item bir \texttt{SCORE} değişkeninin \texttt{VALUE} alanını alma:
%      \begin{lstlisting}
%THE_VALUE(SCORE)
%      \end{lstlisting}
%
%      \pause
%      \item bir tamsayı \texttt{VOTES} değerini bir \texttt{RATIONAL} değere
%        dönüştürme:
%      \begin{lstlisting}
%CAST_AS_RATIONAL(VOTES)
%      \end{lstlisting}
%
%      \pause
%      \item \texttt{MOVIE\#} niteliğinin adını değiştirme:
%      \begin{lstlisting}
%RENAME (MOVIE# AS MOVIENO)
%      \end{lstlisting}
%    \end{itemize}
%  \end{ornek}
%\end{frame}
%
%\begin{frame}[fragile]
%  \frametitle{Değer Kısıtlamaları}
%
%  \begin{block}{Komut}
%    \begin{lstlisting}
%TYPE type_name POSSREP {
%  field_name field_type
%  [, ...]
%  CONSTRAINT condition
%};
%    \end{lstlisting}
%  \end{block}
%\end{frame}
%
%\begin{frame}[fragile]
%  \frametitle{Değer Kısıtlaması Örneği}
%
%  \begin{example}
%    \begin{itemize}
%      \item \texttt{SCORE} değerleri \texttt{1.0} ile \texttt{10.0} arasında
%        olmalı
%    \end{itemize}
%
%    \begin{lstlisting}
%TYPE SCORE POSSREP {
%  VALUE RATIONAL
%  CONSTRAINT
%    (VALUE >= 1.0) AND (VALUE <= 10.0)
%};
%    \end{lstlisting}
%  \end{example}
%\end{frame}

%\subsection{Bağıntı Yönetimi}

\begin{frame}[fragile]
  \frametitle{Bağıntı Tanımlama}

  \begin{itemize}
    \item yeni bir bağıntı tanımlama:
    \begin{lstlisting}
RELATION
  { attribute_name attribute_type
    [, ...] }
  KEY { attribute_name [, ...] }
    \end{lstlisting}
  \end{itemize}
\end{frame}

\begin{frame}[fragile]
  \frametitle{Bağıntı Tanımlama Örneği}

 \begin{lstlisting}
RELATION
  { MOVIE# MOVIE#,
    TITLE CHAR,
    YEAR YEAR,
    DIRECTOR# PERSON#,
    SCORE SCORE,
    VOTES INTEGER }
  KEY { MOVIE# }
  \end{lstlisting}
\end{frame}

\begin{frame}[fragile]
  \frametitle{Bağıntı Değişkenleri}

  \begin{itemize}
    \item yeni bir bağıntı değişkeni tanımlama
    \begin{lstlisting}
VAR relvar_name BASE RELATION
  { ... }
  KEY { ... };
    \end{lstlisting}

    \item bağıntı değişkeni silme:
    \begin{lstlisting}
DROP VAR relvar_name;
    \end{lstlisting}
  \end{itemize}
\end{frame}

%\begin{frame}[fragile]
%  \frametitle{Taban Bağıntı Değişkeni Yaratma}
%
%  \begin{block}{Komut}
%    \begin{lstlisting}
%VAR relvar_name BASE RELATION
%  { ... }
%  KEY { ... };
%    \end{lstlisting}
%  \end{block}
%
%  \pause
%  \begin{block}{Değişken Silme}
%    \begin{lstlisting}
%DROP VAR relvar_name;
%    \end{lstlisting}
%  \end{block}
%\end{frame}

\begin{frame}[fragile]
  \frametitle{Bağıntı Değişkeni Örnekleri}

  \begin{lstlisting}
VAR MOVIE BASE RELATION
  { MOVIE# MOVIE#,
    TITLE CHAR,
    YEAR YEAR,
    DIRECTOR# PERSON#,
    SCORE SCORE,
    VOTES INTEGER }
  KEY { MOVIE# };
  \end{lstlisting}
\end{frame}

\begin{frame}[fragile]
  \frametitle{Bağıntı Değişkeni Örnekleri}

  \begin{lstlisting}
VAR PERSON BASE RELATION
  { PERSON# PERSON#,
    NAME CHAR }
  KEY { PERSON# };
  \end{lstlisting}

  \begin{lstlisting}
VAR CASTING BASE RELATION
  { MOVIE# MOVIE#,
    ACTOR# PERSON#,
    ORD INTEGER }
  KEY { MOVIE#, ACTOR# };
  \end{lstlisting}
\end{frame}

\begin{frame}[fragile]
  \frametitle{Çoklu Üretme}

  \begin{itemize}
    \item çoklu üretme:
    \begin{lstlisting}
TUPLE
  { attribute_name attribute_value
    [, ...] }
    \end{lstlisting}
  \end{itemize}
\end{frame}

\begin{frame}[fragile]
  \frametitle{Çoklu Üretme Örnekleri}

  \begin{lstlisting}
TUPLE
  { MOVIE# MOVIE#(6),
    TITLE "The Usual Suspects",
    YEAR YEAR(1995),
    DIRECTOR# PERSON#(639),
    SCORE SCORE(8.7),
    VOTES 35027 }
  \end{lstlisting}

  \begin{lstlisting}
TUPLE
  { PERSON# PERSON#(639),
    NAME "Bryan Singer" }
  \end{lstlisting}
\end{frame}

%\begin{frame}[fragile]
%  \frametitle{Çoklu ve Bağıntı Üretme}
%
%  \begin{block}{Çoklu Üretme}
%    \begin{lstlisting}
%TUPLE {
%  attribute_name attribute_value
%  [, ...]
%}
%    \end{lstlisting}
%  \end{block}
%
%  \pause
%  \begin{block}{Bağıntı Üretme}
%    \begin{lstlisting}
%RELATION {
%  TUPLE { ... }
%  [, ...]
%}
%    \end{lstlisting}
%  \end{block}
%\end{frame}

\begin{frame}[fragile]
  \frametitle{Bağıntı Üretme}

  \begin{itemize}
    \item bağıntı üretme:
    \begin{lstlisting}
RELATION
  { TUPLE
      { ... }
    [, ...] }
    \end{lstlisting}

    \item bağıntıların bağıntı değişkenlerine atanması:
    \begin{lstlisting}
relvar_name := RELATION { ... };
    \end{lstlisting}
  \end{itemize}
\end{frame}

\begin{frame}[fragile]
  \frametitle{Bağıntı Değişkenine Atama Örneği}

  \begin{lstlisting}
MOVIE := RELATION
  { TUPLE
      { MOVIE# MOVIE#(6), TITLE "The Usual Suspects",
        YEAR YEAR(1995), DIRECTOR# PERSON#(639),
        SCORE SCORE(8.7), VOTES 35027 },
    TUPLE
      { MOVIE# MOVIE#(70), TITLE "Being John Malkovich",
        YEAR YEAR(1999), DIRECTOR# PERSON#(1485),
        SCORE SCORE(8.3), VOTES 13809 } };
  \end{lstlisting}
\end{frame}

%\begin{frame}[fragile]
%  \frametitle{Bağıntı Değişkenine Atama}
%
%  \begin{itemize}
%    \item bağıntıların bağıntı değişkenlerine atanması:
%    \begin{lstlisting}
%relvar_name := RELATION { ... };
%    \end{lstlisting}
%  \end{itemize}
%\end{frame}
%
%\begin{frame}[fragile]
%  \frametitle{Bağıntı Değişkenine Atama Örneği}
%
%  \begin{ornek}
%    \begin{lstlisting}
%MOVIE := RELATION {
%  TUPLE { MOVIE# MOVIE#(6),
%    TITLE "Usual Suspects",
%    YEAR YEAR(1995), DIRECTOR "Bryan Singer",
%    SCORE SCORE(8.7), VOTES 35027 },
%  TUPLE { MOVIE# MOVIE#(70),
%    TITLE "Being John Malkovich",
%    YEAR YEAR(1999), DIRECTOR "Spike Jonze",
%    SCORE SCORE(8.3), VOTES 13809 }
%};
%    \end{lstlisting}
%  \end{ornek}
%\end{frame}

%\subsection{Veriyi Değiştirme}

\begin{frame}[fragile]
  \frametitle{Çoklu Ekleme}

  \begin{itemize}
    \item çokluları ekleme:
    \begin{lstlisting}
INSERT relvar_name RELATION
  { TUPLE { ... }
    [, ...] };
    \end{lstlisting}
  \end{itemize}
\end{frame}

\begin{frame}[fragile]
  \frametitle{Çoklu Ekleme Örneği}

  \begin{lstlisting}
INSERT MOVIE RELATION
  { TUPLE
    { MOVIE# MOVIE#(1),
      TITLE "Star Wars",
      YEAR YEAR(1977),
      DIRECTOR# PERSON#(360),
      SCORE SCORE(8.8),
      VOTES 53567 } };
  \end{lstlisting}
\end{frame}

\begin{frame}[fragile]
  \frametitle{Çoklu Silme}
  
  \begin{itemize}
    \item çokluları silme:
    \begin{lstlisting}
DELETE relvar_name
  [ WHERE condition ];
    \end{lstlisting}

    \item koşul belirtilmezse bütün çoklular silinir
  \end{itemize}
\end{frame}

\begin{frame}[fragile]
  \frametitle{Çoklu Silme Örneği}

  \begin{itemize}
    \item puanı 3.0'dan düşük, oy sayısı 4'den fazla olan filmleri sil
    \begin{lstlisting}
DELETE MOVIE
  WHERE ((SCORE < SCORE(3.0))
     AND (VOTES > 4));
    \end{lstlisting}
  \end{itemize}
\end{frame}

\begin{frame}[fragile]
  \frametitle{Çoklu Güncelleme}
  
  \begin{itemize}
    \item çokluları güncelleme:
    \begin{lstlisting}
UPDATE relvar_name
  [ WHERE condition ]
  ( attribute_name := attribute_value
    [, ...] );
    \end{lstlisting}

    \item koşul belirtilmezse bütün çoklular güncellenir
  \end{itemize}
\end{frame}

\begin{frame}[fragile]
  \frametitle{Çoklu Güncelleme Örneği}
  
  \begin{itemize}
    \item "Suspiria" filmi için yeni verilen bir oyu (9) işle
    \begin{lstlisting}
UPDATE MOVIE
  WHERE (TITLE = "Suspiria") (
    SCORE := SCORE(
      (THE_VALUE(SCORE)
         * CAST_AS_RATIONAL(VOTES)
         + CAST_AS_RATIONAL(9))
       / CAST_AS_RATIONAL(VOTES + 1)
    ),
    VOTES := VOTES + 1
  );
    \end{lstlisting}
  \end{itemize}
\end{frame}

\begin{frame}[fragile]
  \frametitle{Nitelik Adı Değiştirme}

  \begin{itemize}
    \item bir niteliğin adını değiştirme:
    \begin{lstlisting}
RENAME { attribute_name AS new_name }
    \end{lstlisting}
  \end{itemize}

  \medskip
  \begin{exampleblock}{örnek}
    \begin{itemize}
      \item \texttt{DIRECTOR\#} niteliğinin adını değiştirme:
      \begin{lstlisting}
RENAME { DIRECTOR# AS PERSON# }
      \end{lstlisting}
    \end{itemize}
  \end{exampleblock}
\end{frame}



\begin{frame}[fragile]
  \frametitle{Dış Anahtar Tanımlama}
  
  \begin{itemize}
    \item dış anahtar tanımlama:
    \begin{lstlisting}
CONSTRAINT constraint_name
  referencing_relvar_name
      { attribute_name }
    <= referenced_relvar_name
        { attribute_name };
    \end{lstlisting}

    \item  iki bağıntıdaki nitelik isimlerinin aynı olması gerek
     \item değilse nitelik isimleri değiştirilmeli
  \end{itemize}
\end{frame}

\begin{frame}[fragile]
  \frametitle{Dış Anahtar Örnekleri}
  
  \begin{lstlisting}
CONSTRAINT MOVIE_FKEY_DIRECTOR
    MOVIE { DIRECTOR# }
        RENAME { DIRECTOR# AS PERSON# }
      <= PERSON { PERSON# };
  \end{lstlisting}
\end{frame}

\begin{frame}[fragile]
  \frametitle{Dış Anahtar Tanımlama Örnekleri}

  \begin{lstlisting}
CONSTRAINT CASTING_FKEY_MOVIE
  CASTING { MOVIE# } <= MOVIE { MOVIE# };
  \end{lstlisting}

  \begin{lstlisting}
CONSTRAINT CASTING_FKEY_ACTOR
  CASTING { ACTOR# }
      RENAME { ACTOR# AS PERSON# }
    <= PERSON { PERSON# };
  \end{lstlisting}
\end{frame}



%\begin{frame}[fragile]
%  \frametitle{Örnek Veri Tabanının Yaratılması}
%
%  \begin{ornek}[tipler]
%    \begin{lstlisting}
%TYPE MOVIE# POSSREP { VALUE INTEGER };
%TYPE YEAR POSSREP { VALUE INTEGER };
%TYPE SCORE POSSREP { VALUE RATIONAL
%  CONSTRAINT (VALUE >= 1.0)
%         AND (VALUE <= 10.0) };
%TYPE PERSON# POSSREP { VALUE INTEGER };
%    \end{lstlisting}
%  \end{ornek}
%\end{frame}
%
%\begin{frame}[fragile]
%  \frametitle{Örnek Veri Tabanının Yaratılması}
%
%  \begin{ornek}[\texttt{MOVIE} bağıntı değişkeni]
%    \begin{lstlisting}
%VAR MOVIE BASE RELATION
%  { MOVIE# MOVIE#, TITLE CHAR, YEAR YEAR,
%    SCORE SCORE, VOTES INTEGER,
%    DIRECTOR# PERSON# }
%  KEY { MOVIE# };
%    \end{lstlisting}
%  \end{ornek}
%\end{frame}
%
%\begin{frame}[fragile]
%  \frametitle{Örnek Veri Tabanının Yaratılması}
%
%  \begin{ornek}[\texttt{PERSON} bağıntı değişkeni]
%    \begin{lstlisting}
%VAR PERSON BASE RELATION
%    { PERSON# PERSON#, NAME CHAR }
%    KEY { PERSON# };
%    \end{lstlisting}
%  \end{ornek}
%\end{frame}
%
%\begin{frame}[fragile]
%  \frametitle{Örnek Veri Tabanının Yaratılması}
%
%  \begin{ornek}[\texttt{CASTING} bağıntı değişkeni]
%    \begin{lstlisting}
%VAR CASTING BASE RELATION
%    { MOVIE# MOVIE#, ACTOR# PERSON#,
%      ORD INTEGER }
%    KEY { MOVIE#, ACTOR# };
%    \end{lstlisting}
%  \end{ornek}
%\end{frame}
%
%\begin{frame}[fragile]
%  \frametitle{Örnek Veri Tabanının Yaratılması}
%
%  \begin{ornek}[\texttt{MOVIE} bağıntı değişkeninin dış anahtarları]
%    \begin{lstlisting}
%CONSTRAINT MOVIE_FKEY_DIRECTOR
%  MOVIE { DIRECTOR# }
%      RENAME (DIRECTOR# AS PERSON#)
%    <= PERSON { PERSON# };
%    \end{lstlisting}
%  \end{ornek}
%\end{frame}
%
%\begin{frame}[fragile]
%  \frametitle{Örnek Veri Tabanının Yaratılması}
%
%  \begin{ornek}[\texttt{CASTING} bağıntı değişkeninin dış anahtarları]
%    \begin{lstlisting}
%CONSTRAINT CASTING_FKEY_MOVIE
%  CASTING { MOVIE# } <= MOVIE { MOVIE# };
%
%CONSTRAINT CASTING_FKEY_ACTOR
%  CASTING { ACTOR# }
%      RENAME (ACTOR# AS PERSON#)
%    <= PERSON { PERSON# };
%    \end{lstlisting}
%  \end{ornek}
%\end{frame}
%
\lstset{language=FullSQL}

\section{SQL}

\subsection{Veri Tipleri}

\begin{frame}
  \frametitle{Veri Tipleri}

  \begin{itemize}
    \item \texttt{INTEGER}

    \medskip
    \item \texttt{NUMERIC (precision, scale)}
    \begin{itemize}
      \item \texttt{precision}: toplam hane sayısı
      \item \texttt{scale}: noktadan sonraki hane sayısı
      \item eşanlamlısı: \texttt{DECIMAL (precision, scale)}
    \end{itemize}

    \medskip
    \item \texttt{FLOAT}

    \medskip
    \item \texttt{BOOLEAN}
  \end{itemize}
\end{frame}

\begin{frame}
  \frametitle{Katar Veri Tipleri}

  \begin{itemize}
    \item \texttt{CHARACTER [VARYING] (n)}
    \item \texttt{CHARACTER (n)}: katar \texttt{n} simgeden kısaysa\\
        sona boşluk eklenir

    \medskip
    \item \texttt{CHARACTER (n)} yerine \texttt{CHAR (n)}
    \item \texttt{CHARACTER VARYING (n)} yerine \texttt{VARCHAR (n)}
  \end{itemize}
\end{frame}

\begin{frame}
  \frametitle{Tarih - Zaman Veri Tipleri}

  \begin{itemize}
    \item \texttt{DATE}
    \begin{itemize}
      \item değer örneği: \texttt{2005-09-26}
    \end{itemize}

    \medskip
    \item \texttt{TIME}
    \begin{itemize}
      \item değer örneği: \texttt{11:59:22.078717}
    \end{itemize}

    \medskip
    \item \texttt{TIMESTAMP}
    \begin{itemize}
      \item değer örneği: \texttt{2005-09-26 11:59:22.078717}
    \end{itemize}

    \medskip
    \item \texttt{INTERVAL}
    \begin{itemize}
      \item değer örneği: \texttt{3 days}
    \end{itemize}
  \end{itemize}
\end{frame}

\begin{frame}
  \frametitle{Büyük Nesne Veri Tipleri}

  \begin{itemize}
    \item rasgele uzunluklu nesneler
    
    \medskip
    \item ikili: \texttt{BINARY LARGE OBJECT (n)}
    \item \texttt{BLOB}
    
    \pause
    \medskip
    \item metin: \texttt{CHARACTER LARGE OBJECT (n)}
    \item \texttt{CLOB}

    \pause
    \medskip
    \item sorgulamada kullanılamaz  
  \end{itemize}
\end{frame}

\begin{frame}[fragile]
  \frametitle{Tanım Kümesi Yaratma}

  \begin{itemize}
  \item tanım kümesi yaratma:
  \begin{lstlisting}
CREATE DOMAIN domain_name [ AS ] base_type
  [ DEFAULT default_value ]
  [ { CHECK ( condition ) } [, ...] ]
  \end{lstlisting}
    
  \pause
  \item tanım kümesi silme:
  \begin{lstlisting}
DROP DOMAIN domain_name [, ...]
    \end{lstlisting} 
  \end{itemize}    
\end{frame}

\begin{frame}[fragile]
  \frametitle{Tanım Kümesi Örneği}
  
  \begin{itemize}
    \item geçerli \texttt{SCORE} değerleri için bir tanım kümesi:
    \begin{lstlisting}
CREATE DOMAIN SCORES AS FLOAT
  CHECK ((VALUE >= 1.0) AND (VALUE <= 10.0))
    \end{lstlisting}
  \end{itemize}
\end{frame}

\subsection{Veri Tanımlama}

\begin{frame}[fragile]
  \frametitle{Tablo Yaratma}
  
  \begin{itemize}
    \item tablo yaratma:
    \begin{lstlisting}
CREATE TABLE table_name (
  { column_name data_type }
  [, ... ]
)
    \end{lstlisting}

    \item tablo silme:
    \begin{lstlisting}
DROP TABLE table_name [, ... ]
    \end{lstlisting}
  \end{itemize}
\end{frame}

\begin{frame}[fragile]
  \frametitle{Tablo Yaratma Örneği}

  \begin{columns}[b]
    \column{.5\textwidth}
    \begin{lstlisting}
CREATE TABLE MOVIE (
  ID INTEGER,
  TITLE VARCHAR(80),
  YR NUMERIC(4),
  DIRECTORID INTEGER,
  SCORE FLOAT,
  VOTES INTEGER
)
    \end{lstlisting}

    \pause
    \column{.5\textwidth}
    \begin{itemize}
      \item tanım kümesi kullanarak:
    \end{itemize}

    \begin{lstlisting}
CREATE TABLE MOVIE (
  ID INTEGER,
  TITLE VARCHAR(80),
  YR NUMERIC(4),
  DIRECTORID INTEGER,
  SCORE SCORES,
  VOTES INTEGER
)
    \end{lstlisting}
  \end{columns}
\end{frame}

\begin{frame}[fragile]
  \frametitle{Boş ve Varsayılan Değerler}
  
  \begin{itemize}
    \item \texttt{NULL} değeri alabilen sütun ve varsayılan değer tanımlama:
    \begin{lstlisting}
CREATE TABLE table_name (
  { column_name data_type
              [ NULL | NOT NULL ]
              [ DEFAULT default_value ] }
  [, ... ]
)
    \end{lstlisting}

    \item \texttt{NULL}: niteliğin boş bırakılmasına izin var (varsayılan)
    \item \texttt{NOT NULL}: niteliğin boş bırakılmasına izin yok
  \end{itemize}
\end{frame}

\begin{frame}[fragile]
  \frametitle{Tablo Yaratma Örneği}

  \begin{lstlisting}
CREATE TABLE MOVIE (
  ID INTEGER,
  TITLE VARCHAR(80) NOT NULL,
  YR NUMERIC(4),
  DIRECTORID INTEGER,
  SCORE FLOAT,
  VOTES INTEGER DEFAULT 0
)
  \end{lstlisting}
\end{frame}

\begin{frame}[fragile]
  \frametitle{Değer Kısıtlamaları}

  \begin{itemize}
    \item değerler üzerinde kısıt tanımlama:
    \begin{lstlisting}
CREATE TABLE table_name (
  { column_name data_type
              [ NULL | NOT NULL ]
              [ DEFAULT default_value ] }
  [ { CHECK ( condition ) }
    [, ...] ]
)
    \end{lstlisting}
  \end{itemize}
\end{frame}

\begin{frame}[fragile]
  \frametitle{Değer Kısıtlaması Örneği}

  \begin{itemize}
    \item \texttt{SCORE} değerleri \texttt{1.0} ile \texttt{10.0} arasında
    \begin{lstlisting}
CREATE TABLE MOVIE (
  ID INTEGER,
  ...,
  SCORE FLOAT,
  VOTES INTEGER DEFAULT 0,
  CHECK ((SCORE >= 1.0) AND (SCORE <= 10.0))
)
    \end{lstlisting}
  \end{itemize}
\end{frame}

\begin{frame}[fragile]
  \frametitle{Birincil Anahtarlar}

  \begin{itemize}
    \item birincil anahtar tanımlama:
    \begin{lstlisting}
CREATE TABLE table_name (
  { column_name data_type
              [ NULL | NOT NULL ]
              [ DEFAULT default_value ] }
  [, ... ]
  [ PRIMARY KEY ( column_name [, ...] ) ]
)
    \end{lstlisting}
  \end{itemize}
\end{frame}

\begin{frame}[fragile]
  \frametitle{Birincil Anahtar Tanımlama Örneği}

  \begin{lstlisting}
CREATE TABLE MOVIE (
  ID INTEGER,
  TITLE VARCHAR(80) NOT NULL,
  YR NUMERIC(4),
  DIRECTORID INTEGER,
  SCORE FLOAT,
  VOTES INTEGER DEFAULT 0,
  PRIMARY KEY (ID)
)
  \end{lstlisting}
\end{frame}

\begin{frame}[fragile]
  \frametitle{Birincil Anahtarlar}

  \begin{itemize}
    \item birincil anahtar tek bir sütundan oluşuyorsa,\\
        doğrudan sütun tanımında belirtilebilir:
    \begin{lstlisting}
column_name data_type PRIMARY KEY
    \end{lstlisting}
  \end{itemize}

  \begin{exampleblock}{örnek}
    \begin{lstlisting}
CREATE TABLE MOVIE (
  ID INTEGER PRIMARY KEY,
  ...
  VOTES INTEGER DEFAULT 0
)
    \end{lstlisting}
  \end{exampleblock}
\end{frame}

\begin{frame}[fragile]
  \frametitle{Kendiliğinden Artırılan Değerler}

  \begin{itemize}
    \item kendiliğinden artırılan değer tanımlamada standart yok
  
    \bigskip
      \item PostgreSQL: \texttt{SERIAL} data type\\
      \lstinline!ID SERIAL PRIMARY KEY!

    \smallskip
    \item MySQL: \texttt{AUTO\_INCREMENT} property\\
      \lstinline!ID INTEGER PRIMARY KEY AUTO_INCREMENT!

    \smallskip
    \item SQLite: \texttt{AUTOINCREMENT} property\\
      \lstinline!ID INTEGER PRIMARY KEY AUTOINCREMENT!
  \end{itemize}
\end{frame}

\begin{frame}[fragile]
  \frametitle{Eşsizlik}

  \begin{itemize}
    \item eşsiz sütun tanımlama:
    \begin{lstlisting}
CREATE TABLE table_name (
  ...
  [ { UNIQUE ( column_name [, ...] ) }
    [, ...] ]
  ...
)
    \end{lstlisting}

    \item boş değerler dikkate alınmaz
  \end{itemize}
\end{frame}
\begin{frame}[fragile]
  \frametitle{Eşssizlik Tanımı Örneği}

  \begin{itemize}
    \item (başlıklar) ve (yönetmen, yıl) eşsiz:
    \begin{lstlisting}
CREATE TABLE MOVIE (
  ID SERIAL PRIMARY KEY,
  TITLE VARCHAR(80) NOT NULL,
  YR NUMERIC(4),
  DIRECTORID INTEGER,
  SCORE FLOAT,
  VOTES INTEGER DEFAULT 0,
  UNIQUE (TITLE),
  UNIQUE (DIRECTORID, YR)
)
    \end{lstlisting}
  \end{itemize}
\end{frame}

\begin{frame}[fragile]
  \frametitle{Eşsizlik}

  \begin{itemize}
    \item eşsizlik kısıtlaması tek bir sütundan oluşuyorsa,\\
       doğrudan sütun tanımında belirtilebilir:
    \begin{lstlisting}
column_name data_type UNIQUE
    \end{lstlisting}
  \end{itemize}

  \begin{exampleblock}{örnek: kişi isimleri eşsiz}
    \begin{lstlisting}
CREATE TABLE PERSON (
  ID SERIAL PRIMARY KEY,
  NAME VARCHAR(40) UNIQUE NOT NULL
)
    \end{lstlisting}
  \end{exampleblock}
\end{frame}

\begin{frame}[fragile]
  \frametitle{Dizinler}

  \begin{itemize}
    \item dizin yaratma:
    \begin{lstlisting}
CREATE [ UNIQUE ] INDEX index_name
  ON table_name (column_name [, ...])
    \end{lstlisting}

    \smallskip
    \item aramaları hızlandırır
    \item ekleme ve güncellemeleri yavaşlatır
  \end{itemize}

  \begin{exampleblock}{örnek: yıl sütunda dizin yaratma}
    \begin{lstlisting}
CREATE INDEX MOVIE_YEAR ON MOVIE (YR)
    \end{lstlisting}
  \end{exampleblock}
\end{frame}

\begin{frame}[fragile]
  \frametitle{Tablo Adı Değiştirme}

  \begin{itemize}
    \item bir tablonun adını değiştirme:
    \begin{lstlisting}
ALTER TABLE table_name
  RENAME TO new_name
    \end{lstlisting}
  \end{itemize}

  \medskip
  \begin{exampleblock}{örnek}
    \begin{lstlisting}
ALTER TABLE MOVIE
  RENAME TO FILM
    \end{lstlisting}
  \end{exampleblock}
\end{frame}

\begin{frame}[fragile]
  \frametitle{Sütun Ekleme}

  \begin{itemize}
    \item varolan bir tabloya sütun ekleme:
    \begin{lstlisting}
ALTER TABLE table_name
  ADD [ COLUMN ] column_name data_type
                 [ NULL | NOT NULL ]
                 [ DEFAULT default_value ]
    \end{lstlisting}
  \end{itemize}

  \medskip
  \begin{exampleblock}{örnek}
    \begin{lstlisting}
ALTER TABLE MOVIE
  ADD COLUMN RUNTIME INTEGER
    \end{lstlisting}
  \end{exampleblock}
\end{frame}

\begin{frame}[fragile]
  \frametitle{Sütun Silme}

  \begin{itemize}
    \item bir tablodan sütun silme:
    \begin{lstlisting}
ALTER TABLE table_name
  DROP [ COLUMN ] column_name
    \end{lstlisting}
  \end{itemize}

  \medskip
  \begin{exampleblock}{örnek}
    \begin{lstlisting}
ALTER TABLE MOVIE
  DROP COLUMN RUNTIME
    \end{lstlisting}
  \end{exampleblock}
\end{frame}

\begin{frame}[fragile]
  \frametitle{Sütun Adı Değiştirme}

  \begin{itemize}
    \item sütunun adını değiştirme:
    \begin{lstlisting}
ALTER TABLE table_name
  RENAME [ COLUMN ] column_name TO new_name
    \end{lstlisting}
  \end{itemize}

  \medskip
  \begin{exampleblock}{örnek}
    \begin{lstlisting}
ALTER TABLE MOVIE
  RENAME COLUMN TITLE TO NAME
    \end{lstlisting}
  \end{exampleblock}
\end{frame}

\begin{frame}[fragile]
  \frametitle{Sütun Varsayılan Değeri}

  \begin{itemize}
    \item sütun varsayılan değeri değiştirme:
    \begin{lstlisting}
ALTER TABLE table_name
  ALTER [ COLUMN ] column_name
  SET DEFAULT default_value
    \end{lstlisting}

    \item sütun varsayılan değeri silme:
    \begin{lstlisting}
ALTER TABLE table_name
  ALTER [ COLUMN ] column_name
  DROP DEFAULT
    \end{lstlisting}
  \end{itemize}  
\end{frame}

\begin{frame}[fragile]
  \frametitle{Kısıtlama Ekleme}
  
  \begin{itemize}
    \item bir tabloya yeni bir kısıt ekleme:
    \begin{lstlisting}
ALTER TABLE table_name
  ADD [ CONSTRAINT constraint_name ]
    constraint_definition
    \end{lstlisting}

   \item tablodan bir kısıt silme:
    \begin{lstlisting}
ALTER TABLE table_name
  DROP [ CONSTRAINT ] constraint_name
    \end{lstlisting}

    \pause
    \item kısıt eklendiğinde var olan çoklular ne olacak?
  \end{itemize}
\end{frame}

\begin{frame}[fragile]
  \frametitle{Kısıtlama Ekleme Örneği}
  
  \begin{itemize}
    \item \texttt{YR} değerleri 1888'den küçük olamasın
    \begin{lstlisting}
ALTER TABLE MOVIE
  ADD CONSTRAINT MINIMUM_YEAR
    CHECK (YR >= 1888)
    \end{lstlisting}
  \end{itemize}

  \begin{itemize}
    \item minimum yıl kısıtlamasını kaldır
    \begin{lstlisting}
ALTER TABLE MOVIE
  DROP CONSTRAINT MINIMUM_YEAR
    \end{lstlisting}
  \end{itemize}
\end{frame}

\subsection{Veriyi Değiştirme}

\begin{frame}[fragile]
  \frametitle{Satır Ekleme}
  
  \begin{itemize}
    \item bir tabloya bir satır eklemee:
    \begin{lstlisting}
INSERT INTO table_name
  [ ( column_name [, ...] ) ]
  VALUES ( column_value [, ...] )
    \end{lstlisting}

    \pause
    \item değer sırası sütun adı sırasına uymalıdır
    \item sütun adları belirtilmezse sütun değerleri\\
      tablo yaratılırken verilen sırayla yazılmalıdır

    \pause
    \item belirtilmeyen sütunlara varsayılan değerleri atanır
    \item otomatik üretilecek sütunları belirtmemek gerekir
  \end{itemize}
\end{frame}

\begin{frame}[fragile]
  \frametitle{Satır Ekleme Örnekleri}

  \begin{lstlisting}
INSERT INTO MOVIE VALUES (
  6,
  'The Usual Suspects',
  1995,
  639,
  8.7,
  35027
)
  \end{lstlisting}
\end{frame}

\begin{frame}[fragile]
  \frametitle{Satır Ekleme Örnekleri}

  \begin{lstlisting}
INSERT INTO MOVIE (YR, TITLE) VALUES (
  1995,
  'The Usual Suspects'
)
  \end{lstlisting}

  \begin{itemize}
    \item \texttt{ID} değeri otomatik üretilir
  \end{itemize}  
\end{frame}

\begin{frame}[fragile]
  \frametitle{Satır Silme}

  \begin{itemize}
    \item satır silme:
    \begin{lstlisting}
DELETE FROM table_name
  [ WHERE condition ]
    \end{lstlisting}

    \item koşul belirtilmezse bütün satırlar silinir
  \end{itemize}
\end{frame}

\begin{frame}[fragile]
  \frametitle{Satır Silme Örneği}

  \begin{itemize}
    \item puanı 3.0'dan düşük, oy sayısı 4'den fazla olan filmleri sil:
    \begin{lstlisting}
DELETE FROM MOVIE
  WHERE ((SCORE < 3.0) AND (VOTES > 4))
    \end{lstlisting}
  \end{itemize}  
\end{frame}

\begin{frame}[fragile]
  \frametitle{Satır Güncelleme}

  \begin{itemize}
    \item satır güncelleme:
    \begin{lstlisting}
UPDATE table_name
  SET { column_name = column_value } [, ...]
  [ WHERE condition ]
    \end{lstlisting}

    \item koşul belirtilmezse bütün satırlar güncellenir
    \item sütun sıralamasının önemi yoktur
  \end{itemize}  
\end{frame}

\begin{frame}[fragile]
  \frametitle{Satır Güncelleme Örneği}

  \begin{itemize}
    \item "Suspiria" filmi için yeni verilen bir oyu (9) işle
    \begin{lstlisting}
UPDATE MOVIE
  SET SCORE = (SCORE * VOTES + 9)
                  / (VOTES + 1),
      VOTES = VOTES + 1
  WHERE (TITLE = 'Suspiria')
    \end{lstlisting}
  \end{itemize}  
\end{frame}

\subsection{Başvuru Bütünlüğü}

\begin{frame}[fragile]
  \frametitle{Dış Anahtarlar}

  \begin{itemize}
    \item dış anahtar tanımlama:
    \begin{lstlisting}
CREATE TABLE table_name (
  ...
  [ { FOREIGN KEY ( column_name [, ...] )
        REFERENCES table_name
          [ ( column_name [, ...] ) ] }
    [, ...] ]
  ...
)
    \end{lstlisting}
  \end{itemize}  
\end{frame}

\begin{frame}[fragile]
  \frametitle{Dış Anahtar Tanımlama Örneği}

  \begin{lstlisting}
CREATE TABLE MOVIE (
  ID SERIAL PRIMARY KEY,
  TITLE VARCHAR(80) NOT NULL,
  YR NUMERIC(4),
  DIRECTORID INTEGER,
  SCORE FLOAT,
  VOTES INTEGER DEFAULT 0,
  FOREIGN KEY DIRECTORID REFERENCES PERSON (ID)
)
  \end{lstlisting}
\end{frame}

\begin{frame}[fragile]
  \frametitle{Dış Anahtarlar}

  \begin{itemize}
    \item dış anahtar tek bir sütundan oluşuyorsa,\\
       sütun tanımında belirtilebilir:
    \begin{lstlisting}
column_name data_type
  REFERENCES table_name [ ( column_name ) ]
    \end{lstlisting}
  \end{itemize}

  \begin{exampleblock}{örnek}
    \begin{lstlisting}
CREATE TABLE MOVIE (
  ID SERIAL PRIMARY KEY,
  ...
  DIRECTORID INTEGER REFERENCES PERSON (ID),
  ...
)
    \end{lstlisting}
  \end{exampleblock}
\end{frame}

\begin{frame}[fragile]
  \frametitle{Dış Anahtarlar}

  \begin{itemize}
    \item dış anahtar, başvurulan tabloda birincil anahtarla eşleşiyorsa\\
       \texttt{REFERENCES} kısmında belirtilmesi zorunlu değildir
  \end{itemize}

  \begin{exampleblock}{example}
    \begin{lstlisting}
CREATE TABLE MOVIE (
  ID SERIAL PRIMARY KEY,
  ...
  DIRECTORID INTEGER REFERENCES PERSON,
  ...
)
    \end{lstlisting}
  \end{exampleblock}
\end{frame}


\begin{frame}
  \frametitle{Başvuru Bütünlüğü Seçenekleri}
  
  \begin{itemize}
    \item yapılan işlem bütünlük kısıtlamasını bozarsa ne olacak?

    \medskip
    \item işleme izin verme: \texttt{RESTRICT}, \texttt{NO\_ACTION}
    \item işlemi etkilenen çoklulara yansıt: \texttt{CASCADE}
    \item boş değer ata: \texttt{SET NULL}
    \item varsayılan değer ata: \texttt{SET DEFAULT}
  \end{itemize}
\end{frame}

\begin{frame}[fragile]
  \frametitle{Dış Anahtarlar}

  \begin{itemize}
    \item bütünlük kısıtlamaları seçenekleri:
    \begin{lstlisting}
CREATE TABLE table_name (
  ...
  [ { FOREIGN KEY ( column_name [, ...] )
        REFERENCES table_name
          [ ( column_name [, ...] ) ]
        [ ON DELETE option ]
        [ ON UPDATE option ] } [, ...] ]
  ...
)
    \end{lstlisting}
  \end{itemize}
\end{frame}

\begin{frame}[fragile]
  \frametitle{Dış Anahtar Örneği}

  \begin{lstlisting}
CREATE TABLE MOVIE (
  ID SERIAL PRIMARY KEY,
  ...
  DIRECTORID INTEGER,
  ...,
  FOREIGN KEY DIRECTORID
    REFERENCES PERSON (ID)
    ON DELETE RESTRICT
    ON UPDATE CASCADE
)
  \end{lstlisting}
\end{frame}

\begin{frame}
  \frametitle{Bütünlük Kısıtlamaları Örnekleri}

  \begin{columns}
    \column{.63\textwidth}
    \begin{tiny}
    \begin{table}
      \caption{MOVIE}
      \begin{tabular}{|r|l|c|r|}\hline
\underline{ID} & TITLE             & ... & DIRECTORID\\[2pt]\hline\hline
          6 & The Usual Suspects   & ... &        639\\\hline
         70 & Being John Malkovich & ... &       1485\\\hline
        107 & Batman \& Robin      & ... &        105\\\hline
      \end{tabular}
    \end{table}
    \end{tiny}

    \column{.37\textwidth}
    \begin{tiny}
    \begin{table}
      \caption{PERSON}
      \begin{tabular}{|r|l|}\hline
\underline{ID} & NAME\\[2pt]\hline\hline
           308 & Gabriel Byrne\\\hline
          1485 & Spike Jonze  \\\hline
      \end{tabular}
    \end{table}
    \end{tiny}
  \end{columns}
  
  \begin{itemize}
    \item \lstinline!MOVIE.DIRECTORID: ON DELETE RESTRICT!
    \smallskip
    \item PERSON tablosundan "Spike Jonze" sil: izin verilmez
    \item PERSON tablosundan "Gabriel Byrne" sil: izin verilir
  \end{itemize}
\end{frame}

\begin{frame}
  \frametitle{Bütünlük Kısıtlamaları Örnekleri}

  \begin{columns}[t]
    \column{.45\textwidth}
    \begin{tiny}
    \begin{table}
      \caption{MOVIE}
      \begin{tabular}{|r|l|r|}\hline
\underline{ID} & TITLE             & DIRECTORID\\[2pt]\hline\hline
          6 & The Usual Suspects   &        639\\\hline
         70 & Being John Malkovich &       1485\\\hline
        107 & Batman \& Robin      &        105\\\hline
        112 & Three Kings          &       1070\\\hline
      \end{tabular}
    \end{table}
    \end{tiny}

    \column{.25\textwidth}
    \begin{tiny}
    \begin{table}
      \caption{PERSON}
      \begin{tabular}{|r|l|}\hline
\underline{ID} & NAME\\[2pt]\hline\hline
           308 & Gabriel Byrne\\\hline
          1485 & Spike Jonze  \\\hline
      \end{tabular}
    \end{table}
    \end{tiny}

    \column{.3\textwidth}
    \begin{tiny}
    \begin{table}
      \caption{CASTING}
      \begin{tabular}{|r|r|r|}\hline
\underline{MOVIEID} & \underline{ACTORID} & ORD\\[2pt]\hline\hline
                  6 &                 308 &   2\\\hline
                 70 &                 282 &   2\\\hline
                112 &                1485 &   4\\\hline
      \end{tabular}
    \end{table}
    \end{tiny}
  \end{columns}

  \begin{itemize}
    \item \lstinline!MOVIE.DIRECTORID: ON DELETE CASCADE!
    \item \lstinline!CASTING.MOVIEID: ON DELETE CASCADE!
    \item \lstinline!CASTING.ACTORID: ON DELETE CASCADE!
    \smallskip
    \item PERSON tablosundan "Spike Jonze" sil: hangi satırlar silinir?
  \end{itemize}
\end{frame}

\begin{frame}
  \frametitle{Bütünlük Kısıtlamaları Örnekleri}

  \begin{columns}[t]
    \column{.45\textwidth}
    \begin{tiny}
    \begin{table}
      \caption{MOVIE}
      \begin{tabular}{|r|l|r|}\hline
\underline{ID} & TITLE             & DIRECTORID\\[2pt]\hline\hline
          6 & The Usual Suspects   &        639\\\hline
         70 & Being John Malkovich &       1485\\\hline
        107 & Batman \& Robin      &        105\\\hline
        112 & Three Kings          &       1070\\\hline
      \end{tabular}
    \end{table}
    \end{tiny}

    \column{.25\textwidth}
    \begin{tiny}
    \begin{table}
      \caption{PERSON}
      \begin{tabular}{|r|l|}\hline
\underline{ID} & NAME\\[2pt]\hline\hline
           308 & Gabriel Byrne\\\hline
          1485 & Spike Jonze  \\\hline
      \end{tabular}
    \end{table}
    \end{tiny}

    \column{.3\textwidth}
    \begin{tiny}
    \begin{table}
      \caption{CASTING}
      \begin{tabular}{|r|r|r|}\hline
\underline{MOVIEID} & \underline{ACTORID} & ORD\\[2pt]\hline\hline
                  6 &                 308 &   2\\\hline
                 70 &                 282 &   2\\\hline
                112 &                1485 &   4\\\hline
      \end{tabular}
    \end{table}
    \end{tiny}
  \end{columns}

  \begin{itemize}
    \item \lstinline!MOVIE.DIRECTORID: ON DELETE RESTRICT!
    \item \lstinline!CASTING.MOVIEID: ON DELETE CASCADE!
    \item \lstinline!CASTING.ACTORID: ON DELETE CASCADE!
    \smallskip
    \item PERSON tablosundan "Spike Jonze" sil: hangi satırlar silinir?
  \end{itemize}
\end{frame}

\begin{frame}[fragile]
  \frametitle{Örnek Veri Tabanı}

  \begin{lstlisting}
CREATE TABLE MOVIE (
  ID SERIAL PRIMARY KEY,
  TITLE VARCHAR(80) NOT NULL,
  YR NUMERIC(4),
  DIRECTORID INTEGER REFERENCES PERSON (ID)
  SCORE FLOAT,
  VOTES INTEGER DEFAULT 0
)
  \end{lstlisting}
\end{frame}

\begin{frame}[fragile]
  \frametitle{Örnek Veri Tabanı}

  \begin{lstlisting}
CREATE TABLE PERSON (
  ID SERIAL PRIMARY KEY,
  NAME VARCHAR(40) UNIQUE NOT NULL
)
  \end{lstlisting}
\end{frame}

\begin{frame}[fragile]
  \frametitle{Örnek Veri Tabanı}

  \begin{lstlisting}
CREATE TABLE CASTING (
  MOVIEID INTEGER REFERENCES MOVIE (ID),
  ACTORID INTEGER REFERENCES PERSON (ID),
  ORD INTEGER,
  PRIMARY KEY (MOVIEID, ACTORID)
)
  \end{lstlisting}
\end{frame}

\section*{References}

\begin{frame}
  \frametitle{Kaynaklar}

  \begin{block}{Okunacak: Date}
    \begin{itemize}
      \item Chapter 3: An Introduction to Relational Databases
      \begin{itemize}
        \item 3.2. \alert{An Informal Look at the Relational Model}
        \item 3.3. \alert{Relations and Relvars}
      \end{itemize}

      \item Chapter 6: \alert{Relations}

      \item Chapter 9: Integrity
      \begin{itemize}
        \item 9.10. \alert{Keys}
        \item 9.12. \alert{SQL Facilities}
      \end{itemize}
    \end{itemize}
  \end{block}
\end{frame}
\end{document}
