% Copyright (c) 2002-2012
%       H. Turgut Uyar <uyar@itu.edu.tr>
%       Şule Gündüz Öğüdücü <sgunduz@itu.edu.tr>
%
% These notes are licensed using the
% "Creative Commons Attribution-NonCommercial-ShareAlike License".
% You are free to copy, distribute and transmit the work, and to adapt the work
% as long as you attribute the authors, do not use it for commercial purposes,
% and any derivative work is under the same or a similar license.
%
% Read the full legal code at:
% http://creativecommons.org/licenses/by-nc-sa/3.0/

\documentclass[dvipsnames]{beamer}

\usepackage{ae}
\usepackage[T1]{fontenc}
\usepackage[utf8]{inputenc}
\usepackage[turkish]{babel}
\setbeamertemplate{navigation symbols}{}
\usepackage[labelformat=empty, aboveskip=1pt, belowskip=1pt]{caption}

\mode<presentation>
{
  \usetheme{Warsaw}
  \usecolortheme[named=ForestGreen]{structure}
  \setbeamercovered{transparent}
}

\title{Veri Tabanı Sistemleri}
\subtitle{Veri Tabanı Tasarımı}

\author{H. Turgut Uyar \and Şule Öğüdücü}
\date{2002-2012}

\AtBeginSubsection[]{
  \begin{frame}<beamer>
    \frametitle{Konular}
    \tableofcontents[currentsection,currentsubsection]
  \end{frame}
}

\theoremstyle{definition}
\newtheorem{tanim}[theorem]{Tanım}

\theoremstyle{example}
\newtheorem{ornek}[theorem]{Örnek}

\theoremstyle{plain}
\newtheorem{teorem}[theorem]{Teorem}

\pgfdeclareimage[width=2cm]{license}{../../license}

\pgfdeclareimage{norm1}{norm1}
\pgfdeclareimage{norm2}{norm2}
\pgfdeclareimage{norm3}{norm3}
\pgfdeclareimage{imdb1}{imdb1}
\pgfdeclareimage{imdb2}{imdb2}

\begin{document}

\begin{frame}
  \titlepage
\end{frame}

\begin{frame}
  \frametitle{License}

  \pgfuseimage{license}\hfill
  \copyright 2002-2012 T. Uyar, Ş. Öğüdücü

  \vfill
  \begin{tiny}
    You are free:
    \begin{itemize}
      \item to Share -- to copy, distribute and transmit the work
      \item to Remix -- to adapt the work
    \end{itemize}

    Under the following conditions:
    \begin{itemize}
      \item Attribution -- You must attribute the work in the manner specified by
        the author or licensor (but not in any way that suggests that they
        endorse you or your use of the work).

      \item Noncommercial -- You may not use this work for commercial purposes.

      \item Share Alike -- If you alter, transform, or build upon this work, you
        may distribute the resulting work only under the same or similar license
        to this one.
    \end{itemize}
  \end{tiny}

  \vfill
  Legal code (the full license):\\
  \url{http://creativecommons.org/licenses/by-nc-sa/3.0/}
\end{frame}

\begin{frame}
  \frametitle{Konular}
  \tableofcontents
\end{frame}

\section{Normalizasyon}

\subsection{Giriş}

\begin{frame}
  \frametitle{İşlevsel Bağımlılık}

  \begin{tanim}
    \begin{itemize}
      \item $Z$: $R$ bağıntısının bütün nitelikleri kümesi
      \item $A,B \subseteq Z$

      \pause
      \item \alert{$A$, $B$'yi işlevsel olarak belirliyor}: $A \rightarrow B$\\
        her $A$ değerine karşılık tek bir $B$ değeri olabilir
    \end{itemize}
  \end{tanim}

  \pause
  \begin{itemize}
    \item her işlevsel bağımlılık bir bütünlük kısıtlaması
  \end{itemize}
\end{frame}

\begin{frame}
  \frametitle{Örnek Bağıntı}

  \begin{ornek}
    \begin{tiny}
    \begin{table}
      \caption{R}
      \begin{tabular}{|r|l|c|c|r|l|r|}\hline
\underline{MOVIEID} & TITLE    & COU & LANG & \underline{ACTORID} & NAME & ORD\\[2pt]\hline\hline
      6 & Usual Suspects       & UK  &  EN  &     308 & Gabriel Byrne    &   2\\\hline
    228 & Ed Wood              & US  &  EN  &      26 & Johnny Depp      &   1\\\hline
     70 & Being John Malkovich & US  &  EN  &     282 & Cameron Diaz     &   2\\\hline
   1512 & Suspiria             & IT  &  IT  &     745 & Udo Kier         &   9\\\hline
     70 & Being John Malkovich & US  &  EN  &     503 & John Malkovich   &  14\\\hline
      \end{tabular}
    \end{table}
    \end{tiny}

    \pause
    \begin{itemize}
      \item varsayım: film hangi ülkede çekildiyse o ülkenin dilinde
    \end{itemize}
  \end{ornek}
\end{frame}

\begin{frame}
  \frametitle{İşlevsel Bağımlılık Örnekleri}

  \begin{ornek}
    \begin{itemize}
      \item MOVIEID $\rightarrow$ COUNTRY

      \pause
      \item ACTORID $\rightarrow$ NAME

      \pause
      \item MOVIEID $\rightarrow$ \{TITLE, COUNTRY, LANGUAGE\}

      \pause
      \item \{MOVIEID, ACTORID\} $\rightarrow$ COUNTRY

      \pause
      \item \{MOVIEID, ACTORID\} $\rightarrow$ MOVIEID

      \pause
      \item \{MOVIEID, ACTORID\} $\rightarrow$ ORD

      \pause
      \item \{MOVIEID, ACTORID\} $\rightarrow$ \{COUNTRY, ORD\}

      \pause
      \item COUNTRY $\rightarrow$ LANGUAGE
    \end{itemize}
  \end{ornek}
\end{frame}

\begin{frame}
  \frametitle{İndirgenemez Küme}

  \begin{itemize}
    \item $S$: bağıntının bütün işlevsel bağımlılıkları kümesi

    \item $T \subseteq S$, öyle ki
    \begin{itemize}
      \item $T$ olabildiğince az eleman içerir
      \item $S$'deki her işlevsel bağımlılık $T$'dekilerden türetilebilir
    \end{itemize}

    \pause
    \medskip
    \item işlevsel bağımlılıkların sağ yanlarında tek nitelik yer alsın
  \end{itemize}
\end{frame}

\begin{frame}
  \frametitle{İndirgenemez Küme Örneği}

  \begin{ornek}
    \begin{itemize}
      \item MOVIEID $\rightarrow$ TITLE
      \item MOVIEID $\rightarrow$ COUNTRY
      \item COUNTRY $\rightarrow$ LANGUAGE
      \item ACTORID $\rightarrow$ NAME
      \item \{MOVIEID, ACTORID\} $\rightarrow$ ORD
    \end{itemize}
  \end{ornek}
\end{frame}

\begin{frame}
  \frametitle{Bağımlılık Çizeneği}

  \begin{ornek}
    \begin{center}
      \pgfuseimage{norm1}
    \end{center}
  \end{ornek}
\end{frame}

\subsection{Normal Formlar}

\begin{frame}
  \frametitle{Normal Formlar}

  \begin{itemize}
    \item 1NF, 2NF, 3NF, BCNF, 4NF, 5NF

    \item her form bir önceki formun kapsamını daraltır
    \begin{itemize}
      \item bütün 2NF bağıntılar aynı zamanda 1NF
      \item bütün 3NF bağıntılar aynı zamanda 2NF, ...
    \end{itemize}

    \pause
    \medskip
    \item 1NF: nitelik değerleri bölünmezdir
  \end{itemize}
\end{frame}

\begin{frame}
  \frametitle{Normalizasyon}

  \begin{tanim}
    \alert{normalizasyon}:\\
      bir formdan daha dar kapsamlı bir sonraki forma geçiş

    \begin{itemize}
      \item formlar arası geçişler kayıpsız olmalı
    \end{itemize}
  \end{tanim}

  \pause
  \begin{teorem}[Heath Kuramı]
    \begin{itemize}
      \item $Z$: $R$ bağıntısının bütün nitelikleri kümesi
      \item $A,B,C \subseteq Z$

      \pause
      \item $A \rightarrow B$ ise $R$ bağıntısı $\{A,B\}$ ile $\{A,C\}$
        bağıntılarının\\
        birleştirilmesiyle elde edilebilir
    \end{itemize}
  \end{teorem}
\end{frame}

\begin{frame}
  \frametitle{Kayıpsız Geçiş Örneği}

  \begin{ornek}
    \begin{columns}[c]
      \column{.57\textwidth}
      \begin{tiny}
      \begin{table}
        \caption{R1}
        \begin{tabular}{|r|l|c|c|}\hline
MOVIEID & TITLE                & COU & LANG\\\hline\hline
      6 & Usual Suspects       & UK  &  EN \\\hline
    228 & Ed Wood              & US  &  EN \\\hline
     70 & Being John Malkovich & US  &  EN \\\hline
   1512 & Suspiria             & IT  &  IT \\\hline
        \end{tabular}
      \end{table}
      \end{tiny}

      \vspace{-0.7cm}
      \begin{tiny}
      \begin{table}
        \caption{R2}
        \begin{tabular}{|r|r|l|r|}\hline
MOVIEID & ACTORID & NAME           & ORD\\\hline\hline
      6 &     308 & Gabriel Byrne  &   2\\\hline
    228 &      26 & Johnny Depp    &   1\\\hline
     70 &     282 & Cameron Diaz   &   2\\\hline
   1512 &     745 & Udo Kier       &   9\\\hline
     70 &     503 & John Malkovich &  14\\\hline
        \end{tabular}
      \end{table}
      \end{tiny}

      \pause
      \column{.4\textwidth}
      \begin{itemize}
        \item R $=$ R1 JOIN R2
      \end{itemize}
    \end{columns}
  \end{ornek}
\end{frame}

\begin{frame}
  \frametitle{Kayıplı Geçiş Örneği}

  \begin{ornek}
    \begin{columns}[c]
      \column{.57\textwidth}
      \begin{tiny}
      \begin{table}
        \caption{R1}
        \begin{tabular}{|r|l|c|c|}\hline
MOVIEID & TITLE                & COU & LANG\\\hline\hline
      6 & Usual Suspects       & UK  &  EN \\\hline
    228 & Ed Wood              & US  &  EN \\\hline
     70 & Being John Malkovich & US  &  EN \\\hline
   1512 & Suspiria             & IT  &  IT \\\hline
        \end{tabular}
      \end{table}
      \end{tiny}

      \vspace{-0.7cm}
      \begin{tiny}
      \begin{table}
        \caption{R2}
        \begin{tabular}{|c|r|l|r|}\hline
COU & ACTORID & NAME           & ORD\\\hline\hline
UK  &     308 & Gabriel Byrne  &   2\\\hline
US  &      26 & Johnny Depp    &   1\\\hline
US  &     282 & Cameron Diaz   &   2\\\hline
IT  &     745 & Udo Kier       &   9\\\hline
US  &     503 & John Malkovich &  14\\\hline
        \end{tabular}
      \end{table}
      \end{tiny}

      \pause
      \column{.4\textwidth}
      \begin{itemize}
        \item R $\neq$ R1 JOIN R2

        \pause
        \item \tiny{\{MOVIEID, ACTORID\} $\rightarrow$ ORD}
      \end{itemize}
    \end{columns}
  \end{ornek}
\end{frame}

\begin{frame}
  \frametitle{Aykırılıklar}

  \begin{itemize}
    \item \emph{ekleme}
    \begin{itemize}
      \item bilinen bir verinin kısıtlamalar nedeniyle tutulamaması
    \end{itemize}

    \pause
    \item \emph{silme}
    \begin{itemize}
        \item bir veri silinmek istendiğinde başka bir verinin de yitirilmesi
    \end{itemize}

    \pause
    \item \emph{güncelleme}
    \begin{itemize}
      \item bir veriyi güncellemek için birden fazla çokluda değişiklik\\
        gerekmesi
    \end{itemize}
  \end{itemize}
\end{frame}

\begin{frame}
  \frametitle{Aykırılık Örnekleri}

  \begin{ornek}
    \begin{itemize}
      \item "Gattaca" filminin ülkesinin US olduğu biliniyor\\
	ama filmde oynayan bir oyuncu olmadıkça eklenemiyor

      \pause
      \item Gabriel Byrne'in "Usual Suspects" filminde oynadığı silinirse\\
        filmin ülkesinin UK olduğu da siliniyor

      \pause
      \item "Being John Malkovich" filminin ülkesinin güncellenmesi\\
        iki çokluda değişiklik gerektiriyor
    \end{itemize}
  \end{ornek}
\end{frame}

\begin{frame}
  \frametitle{2. Normal Form}

  \begin{tanim}
    \alert{2NF}: anahtar olmayan her nitelik birincil anahtara bağımlı
  \end{tanim}

  \begin{block}{1NF'den 2NF'ye geçiş}
    \begin{itemize}
      \item 1NF'ye uyan bir $R$ bağıntısında:
      \begin{itemize}
        \item $R(A,B,C,D)$, birincil anahtar: $\{A,B\}$
        \item $A \rightarrow D$
      \end{itemize}

      \pause
      \item 2NF olması için:
      \begin{itemize}
        \item $R1(A,D)$, birincil anahtar: $A$
        \item $R2(A,B,C)$, birincil anahtar: $\{A,B\}$\\
          $A$, $R1$'e başvuran dış anahtar
      \end{itemize}
    \end{itemize}
  \end{block}
\end{frame}

\begin{frame}
  \frametitle{1NF-2NF Geçişi Örneği}

  \begin{ornek}
    \begin{itemize}
      \item anahtar olmayan niteliklerden ORD dışındakiler\\
	birincil anahtara bağımlı değil

      \pause
      \begin{itemize}
        \item $A$: MOVIEID
        \item $B$: ACTORID
        \item $C$: \{NAME, ORD\}
        \item $D$: \{TITLE, COUNTRY, LANGUAGE\}
      \end{itemize}
    \end{itemize}
  \end{ornek}
\end{frame}

\begin{frame}
  \frametitle{1NF-2NF Geçişi Örneği}

  \begin{ornek}
    \begin{itemize}
      \item R1(MOVIEID, TITLE, COUNTRY, LANGUAGE)\\
        birincil anahtar: MOVIEID

      \pause
      \item R2(MOVIEID, ACTORID, NAME, ORD)\\
        birincil anahtar: \{MOVIEID, ACTORID\}\\
        MOVIEID, R1'e başvuran dış anahtar
    \end{itemize}
 \end{ornek}
\end{frame}

\begin{frame}
  \frametitle{1NF-2NF Geçişi Örneği}

  \begin{ornek}
    \begin{itemize}
      \item R2 hala 2NF değil: ACTORID $\rightarrow$ NAME

      \pause
      \begin{itemize}
        \item $A$: ACTORID
        \item $B$: MOVIEID
        \item $C$: ORD
        \item $D$: NAME
      \end{itemize}
    \end{itemize}

    \pause
    \begin{itemize}
      \item R3(ACTORID, NAME)\\
        birincil anahtar: ACTORID

      \pause
      \item R4(MOVIEID, ACTORID, ORD)\\
        birincil anahtar: \{MOVIEID, ACTORID\}\\
        ACTORID, R3'e başvuran dış anahtar
    \end{itemize}
  \end{ornek}
\end{frame}

\begin{frame}
  \frametitle{2NF Bağıntı Örnekleri}

  \begin{ornek}
    \begin{center}
      \begin{tiny}
      \begin{table}
        \caption{R1}
        \begin{tabular}{|r|l|c|c|}\hline
\underline{MOVIEID} & TITLE    & COU & LANG\\[2pt]\hline\hline
      6 & Usual Suspects       & UK  &  EN \\\hline
    228 & Ed Wood              & US  &  EN \\\hline
     70 & Being John Malkovich & US  &  EN \\\hline
   1512 & Suspiria             & IT  &  IT \\\hline
        \end{tabular}
      \end{table}
      \end{tiny}
    \end{center}

    \vspace{-0.7cm}
    \begin{columns}[t]
      \column{.5\textwidth}
      \begin{tiny}
      \begin{table}
        \caption{R3}
        \begin{tabular}{|r|l|}\hline
\underline{ACTORID} & NAME\\[2pt]\hline\hline
      308 & Gabriel Byrne \\\hline
       26 & Johnny Depp   \\\hline
      282 & Cameron Diaz  \\\hline
      745 & Udo Kier      \\\hline
      503 & John Malkovich\\\hline
        \end{tabular}
      \end{table}
      \end{tiny}

      \column{.5\textwidth}
      \begin{tiny}
      \begin{table}
        \caption{R4}
        \begin{tabular}{|r|r|r|}\hline
\underline{MOVIEID} & \underline{ACTORID} & ORD\\[2pt]\hline\hline
   6 & 308 &  2\\\hline
 228 &  26 &  1\\\hline
  70 & 282 &  2\\\hline
1512 & 745 &  9\\\hline
  70 & 503 & 14\\\hline
        \end{tabular}
      \end{table}
      \end{tiny}
    \end{columns}
  \end{ornek}
\end{frame}

\begin{frame}
  \frametitle{Bağımlılık Çizeneği Örneği}

  \begin{ornek}
    \begin{center}
      \pgfuseimage{norm2}
    \end{center}
  \end{ornek}
\end{frame}

\begin{frame}
  \frametitle{2NF Düzelen Aykırılıklar}

  \begin{ornek}
    \begin{itemize}
      \item "Gattaca" filminin ülkesinin US olduğu biliniyorsa\\
	bu bilgi R1 bağıntısına eklenebilir

      \pause
      \item Gabriel Byrne'in "Usual Suspects" filminde oynadığı silinse de\\
        filmin ülkesinin UK olduğu bilgisi R1 bağıntısında kalır

      \pause
      \item "Being John Malkovich" filminin ülkesini güncellemek için\\
        R1 bağıntısında tek çokluda değişiklik yapmak yeterli
    \end{itemize}
  \end{ornek}
\end{frame}

\begin{frame}
  \frametitle{2NF Düzelmeyen Aykırılıklar}

  \begin{ornek}
    \begin{itemize}
      \item Brezilya'da çekilen filmlerin Portekizce olduğu biliniyor\\
	ama Brezilya'da çekilen bir film olmadıkça eklenemiyor

      \pause
      \item "Suspiria" filmi silinirse İtalya'da çekilen filmlerin\\
        İtalyanca olduğu da siliniyor

      \pause
      \item Amerika'da çekilen filmlerin dilinin güncellenmesi\\
        iki çokluda değişiklik gerektiriyor
    \end{itemize}
  \end{ornek}
\end{frame}

\subsection{3. Normal Form}

\begin{frame}
  \frametitle{3. Normal Form}

  \begin{tanim}
    \alert{3NF}: anahtar olmayan nitelikler birincil anahtar dışında\\
      niteliklere bağımlı değil
  \end{tanim}

  \pause
  \begin{block}{2NF'den 3NF'ye geçiş}
    \begin{itemize}
      \item 2NF'ye uyan bir $R$ bağıntısında:
      \begin{itemize}
        \item $R(A,B,C,D)$, birincil anahtar: $A$
        \item $C \rightarrow D$
      \end{itemize}

      \pause
      \item 3NF olması için:
      \begin{itemize}
        \item $R1(C,D)$, birincil anahtar: $C$
        \item $R2(A,B,C)$, birincil anahtar: $A$\\
          $C$, $R1$'e başvuran dış anahtar
      \end{itemize}
    \end{itemize}
 \end{block}
\end{frame}

\begin{frame}
  \frametitle{2NF-3NF Geçişi Örneği}

  \begin{ornek}
    \begin{itemize}
      \item R1: COUNTRY $\rightarrow$ LANGUAGE

      \pause
      \begin{itemize}
        \item $A$: MOVIEID
        \item $B$: TITLE
        \item $C$: COUNTRY
        \item $D$: LANGUAGE
      \end{itemize}
    \end{itemize}

    \pause
    \begin{itemize}
      \item R5(COUNTRY, LANGUAGE)\\
        birincil anahtar: COUNTRY

      \pause
      \item R6(MOVIEID, TITLE, COUNTRY)\\
        birincil anahtar: MOVIEID\\
        COUNTRY, R5'e başvuran dış anahtar
    \end{itemize}
  \end{ornek}
\end{frame}

\begin{frame}
  \frametitle{3NF Bağıntı Örnekleri}

  \begin{ornek}
    \begin{columns}[t]
      \column{.5\textwidth}
      \begin{tiny}
      \begin{table}
        \caption{R6}
        \begin{tabular}{|r|l|c|}\hline
\underline{MOVIEID} & TITLE    & COU\\[2pt]\hline\hline
      6 & Usual Suspects       & UK \\\hline
    228 & Ed Wood              & US \\\hline
     70 & Being John Malkovich & US \\\hline
   1512 & Suspiria             & IT \\\hline
        \end{tabular}
      \end{table}
      \end{tiny}

      \column{.5\textwidth}
      \begin{tiny}
      \begin{table}
        \caption{R5}
        \begin{tabular}{|c|c|}\hline
\underline{COU} & LANG\\[2pt]\hline\hline
UK & EN\\\hline
US & EN\\\hline
IT & IT\\\hline
        \end{tabular}
      \end{table}
      \end{tiny}
    \end{columns}

    \vspace{-0.7cm}
    \begin{columns}[t]
      \column{.5\textwidth}
      \begin{tiny}
      \begin{table}
        \caption{R3}
        \begin{tabular}{|r|l|}\hline
\underline{ACTORID} & NAME\\[2pt]\hline\hline
      308 & Gabriel Byrne \\\hline
       26 & Johnny Depp   \\\hline
      282 & Cameron Diaz  \\\hline
      745 & Udo Kier      \\\hline
      503 & John Malkovich\\\hline
        \end{tabular}
      \end{table}
      \end{tiny}

      \column{.5\textwidth}
      \begin{tiny}
      \begin{table}
        \caption{R4}
        \begin{tabular}{|r|r|r|}\hline
\underline{MOVIEID} & \underline{ACTORID} & ORD\\[2pt]\hline\hline
   6 & 308 &  2\\\hline
 228 &  26 &  1\\\hline
  70 & 282 &  2\\\hline
1512 & 745 &  9\\\hline
  70 & 503 & 14\\\hline
        \end{tabular}
      \end{table}
      \end{tiny}
    \end{columns}
  \end{ornek}
\end{frame}

\begin{frame}
  \frametitle{Bağımlılık Çizeneği Örneği}

  \begin{ornek}
    \begin{center}
      \pgfuseimage{norm3}
    \end{center}
  \end{ornek}
 \end{frame}

\begin{frame}
  \frametitle{3NF Düzelen Aykırılıklar}

  \begin{ornek}
    \begin{itemize}
      \item Brezilya'da çekilen filmlerin Portekizce olduğu biliniyorsa\\
	R5 bağıntısına eklenebilir

      \pause
      \item "Suspiria" filmi silinse de İtalya'da çekilen filmlerin\\
	İtalyanca olduğu R5 bağıntısında kalır

      \pause
      \item Amerika'da çekilen filmlerin dilini güncellemek için\\
	R5 bağıntısında tek çokluda değişiklik yapmak yeterli
    \end{itemize}
  \end{ornek}
\end{frame}

\begin{frame}
  \frametitle{Boyce-Codd Normal Formu}

  \begin{tanim}
    \alert{BCNF}: bütün işlevsel bağımlılıklar anahtar adaylarına
  \end{tanim}

  \pause
  \begin{itemize}
    \item anahtarı oluşturan nitelikler arasındaki bağımlılıklar\\
      dikkate alınmalı
  \end{itemize}
\end{frame}

\begin{frame}
  \frametitle{BCNF Örneği}

  \begin{ornek}[filmlerin başlık nitelikleri eşsiz]
    \begin{itemize}
      \item anahtar adayları:
      \begin{itemize}
        \item \{MOVIEID, ACTORID\}
        \item \{TITLE, ACTORID\}
      \end{itemize}

      \pause
      \item aykırı işlevsel bağımlılıklar:
      \begin{itemize}
        \item MOVIEID $\rightarrow$ TITLE
        \item TITLE $\rightarrow$ MOVIEID
      \end{itemize}
    \end{itemize}
  \end{ornek}
\end{frame}

\subsection*{Kaynaklar}

\begin{frame}
  \frametitle{Kaynaklar}

  \begin{block}{Okunacak: Date}
    \begin{itemize}
      \item Chapter 11: \alert{Functional Dependencies}
      \item Chapter 12: \alert{Further Normalization I: 1NF, 2NF, 3NF, BCNF}
    \end{itemize}
  \end{block}
\end{frame}

\section{Varlık/İlişki Modeli}

\subsection{Giriş}

\begin{frame}
  \frametitle{Varlık/İlişki Modeli}

  \begin{itemize}
    \item modelleme yaklaşımı
    \begin{itemize}
      \item Chen 1976
    \end{itemize}

    \pause
    \item bileşenleri
    \begin{itemize}
      \item varlıklar
      \item özellikler
      \item ilişkiler
    \end{itemize}
  \end{itemize}
\end{frame}

\begin{frame}
  \frametitle{Varlıklar}

  \begin{tanim}
    \alert{varlık}: aynı özellikleri taşıyan "şeyler" kümesi

    \pause
    \begin{itemize}
      \item küme elemanları varlık tipinin birer \emph{örneği}
    \end{itemize}

    \pause
    \begin{itemize}
      \item \emph{güçlü}: tek başına var olabilir
      \item \emph{zayıf}: varlığı başka bir varlığa bağlı
    \end{itemize}
  \end{tanim}
\end{frame}

\begin{frame}
  \frametitle{Varlık Örnekleri}

  \begin{ornek}
    \begin{itemize}
      \item varlık: film, kişi

      \pause
      \item kişi örneği: Johnny Depp

      \pause
      \medskip
      \item güçlü varlık: kişi
      \item zayıf varlık: film
    \end{itemize}
  \end{ornek}
\end{frame}

\begin{frame}
  \frametitle{Özellikler}

  \begin{tanim}
    \alert{özellik}: varlıkları ya da ilişkileri betimleyen veriler

    \pause
    \begin{itemize}
      \item basit - bileşke
      \item anahtar
      \item tekli - çoklu değerli
      \item boş
      \item taban - türetilmiş
    \end{itemize}
  \end{tanim}
\end{frame}

\begin{frame}
  \frametitle{Özellik Örnekleri}

  \begin{ornek}
    \begin{itemize}
      \item özellik: başlık, ülke, dil

      \pause
      \medskip
      \item basit: önad, soyad
      \item bileşke: tam ad

      \pause
      \medskip
      \item taban: doğum tarihi
      \item türetilmiş: yaş
    \end{itemize}
  \end{ornek}
\end{frame}

\begin{frame}
  \frametitle{İlişkiler}

  \begin{tanim}
    \alert{ilişki}: varlıklar arasındaki bağlantılar

    \pause
    \begin{itemize}
      \item \emph{katılımcı}: ilişkideki varlıklar
      \item \emph{derece}: katılımcı sayısı
      \item \emph{total} - \emph{kısmi}: bütün örnekler ilişkiye
        katılıyor - katılmıyor
    \end{itemize}
  \end{tanim}
\end{frame}

\begin{frame}
  \frametitle{İlişki Türleri}

  \begin{itemize}
    \item \emph{bire bir}
    \item \emph{bire çok}
    \item \emph{çoka çok}
  \end{itemize}
\end{frame}

\begin{frame}
  \frametitle{İlişki Örnekleri}

  \begin{ornek}[bire bir]
    \begin{itemize}
      \item ülkeler ile şehirler arasındaki başkentlik ilişkisi
    \end{itemize}
  \end{ornek}

  \pause
  \begin{ornek}[bire çok]
    \begin{itemize}
      \item çalışanlar ile projeler arasındaki yöneticilik ilişkisi
    \end{itemize}
  \end{ornek}

  \pause
  \begin{ornek}[çoka çok]
    \begin{itemize}
      \item öğrenciler ile dersler arasındaki kayıt ilişkisi
    \end{itemize}
  \end{ornek}
\end{frame}

\subsection{V/İ Çizenekleri}

\begin{frame}
  \frametitle{Varlık/İlişki Çizenekleri}

  \begin{itemize}
    \item varlık: dikdörtgen
    \begin{itemize}
      \item zayıf: çift çizgi
    \end{itemize}

    \pause
    \item özellik: elips
    \begin{itemize}
      \item türetilmiş: kesikli çizgi
      \item çoklu değerli: çift çizgi
      \item bileşke: alt-elipsler
    \end{itemize}

    \pause
    \item ilişki: eşkenar dörtgen
    \begin{itemize}
      \item zayıf-güçlü arasında: çift çizgi
      \item total: bağlantı çift çizgi
      \item ilişkinin türüne göre 1 ya da n
    \end{itemize}
 \end{itemize}
\end{frame}

\begin{frame}
  \frametitle{Varlık/İlişki Çizeneği Örneği}

  \begin{ornek}
    \begin{center}
      \pgfuseimage{imdb1}
    \end{center}
  \end{ornek}
\end{frame}

\begin{frame}
  \frametitle{Varlık/İlişki Çizeneği Örneği}

  \begin{ornek}
    \begin{center}
      \pgfuseimage{imdb2}
    \end{center}
  \end{ornek}
\end{frame}

\begin{frame}
  \frametitle{Tasarıma Geçiş}

  \begin{itemize}
    \item her varlık bir bağıntı

    \pause
    \item her özellik bir nitelik

    \pause
    \item her çoka çok ilişki bir bağıntı
    \begin{itemize}
      \item katılımcı varlıklara dış anahtarlar
    \end{itemize}

    \pause
    \item her bire çok ilişki için ilişkinin "çok" tarafından\\
      "bir" tarafına dış anahtar
 \end{itemize}
\end{frame}

\subsection*{Kaynaklar}

\begin{frame}
  \frametitle{Kaynaklar}

  \begin{block}{Okunacak: Date}
    \begin{itemize}
      \item Chapter 14: \alert{Semantic Modeling}
    \end{itemize}
  \end{block}
\end{frame}

\end{document}
