% Copyright (c) 2002-2012
%       H. Turgut Uyar <uyar@itu.edu.tr>
%       Şule Gündüz Öğüdücü <sgunduz@itu.edu.tr>
%
% These notes are licensed using the
% "Creative Commons Attribution-NonCommercial-ShareAlike License".
% You are free to copy, distribute and transmit the work, and to adapt the work
% as long as you attribute the authors, do not use it for commercial purposes,
% and any derivative work is under the same or a similar license.
%
% Read the full legal code at:
% http://creativecommons.org/licenses/by-nc-sa/3.0/

\documentclass[dvipsnames]{beamer}

\usepackage{ae}
\usepackage[scaled=0.88]{beramono}
\usepackage[T1]{fontenc}
\usepackage[utf8]{inputenc}
%\usepackage[turkish]{babel}
%\setbeamertemplate{navigation symbols}{}
\setbeamersize{text margin left=2em, text margin right=2em}
\usepackage[labelformat=empty, aboveskip=1pt, belowskip=1pt]{caption}

\usepackage{listings}

\lstdefinelanguage{FullSQL}[]{SQL}{
  morekeywords={BINARY, BOOLEAN, CYCLE, FINAL, INCREMENT, IS, LARGE, MAXVALUE,
                MINVALUE, NO_ACTION, OBJECT, REFERENCES, RENAME, SEQUENCE,
                START, TO, TYPE, VACUUM}
}
\lstdefinelanguage{ExtendedSQL}[]{FullSQL}{
  morekeywords={AFTER, BEFORE, DO, EACH, FOR, FUNCTION, INSTEAD, LANGUAGE,
                OPTION, PROCEDURE, RETURNS, ROW, RULE, SNAPSHOT, STATEMENT,
                WITH}
}
\lstset{basicstyle=\ttfamily, keywordstyle=\color{ForestGreen},
        showstringspaces=false}

\mode<presentation>
{
  \usetheme{Warsaw}
  \usecolortheme[named=ForestGreen]{structure}
  \setbeamercovered{transparent}
}

\title{Veri Tabanı Sistemleri}
\subtitle{Veri Tabanı Tasarımı}

\author{H. Turgut Uyar \and Şule Öğüdücü}
\date{2002-2016}

\AtBeginSubsection[]{
  \begin{frame}<beamer>
    \frametitle{Konular}
    \tableofcontents[currentsection,currentsubsection]
  \end{frame}
}

\theoremstyle{theorem}
\newtheorem{teorem}[theorem]{Teorem}

\pgfdeclareimage[width=2cm]{license}{../license}

\pgfdeclareimage{norm1}{norm1}
\pgfdeclareimage{norm2}{norm2}
\pgfdeclareimage{norm3}{norm3}
\pgfdeclareimage{imdb1}{imdb1}
\pgfdeclareimage{imdb2}{imdb2}

\begin{document}

\begin{frame}
  \titlepage
\end{frame}

\begin{frame}
  \frametitle{License}

  \pgfuseimage{license}\hfill
  \copyright~2002-2016 T. Uyar, Ş. Öğüdücü

  \vfill
  \begin{footnotesize}
    You are free to:
    \begin{itemize}
      \itemsep0em
      \item Share -- copy and redistribute the material in any medium or format
      \item Adapt -- remix, transform, and build upon the material
    \end{itemize}

    Under the following terms:
    \begin{itemize}
      \itemsep0em
      \item Attribution -- You must give appropriate credit, provide a link to
        the license, and indicate if changes were made.

      \item NonCommercial -- You may not use the material for commercial
        purposes.

      \item ShareAlike -- If you remix, transform, or build upon the material,
        you must distribute your contributions under the same license as the
        original.
    \end{itemize}
  \end{footnotesize}

  \begin{small}
    For more information:\\
    \url{https://creativecommons.org/licenses/by-nc-sa/4.0/}

    \smallskip
    Read the full license:\\
    \url{https://creativecommons.org/licenses/by-nc-sa/4.0/legalcode}
  \end{small}
\end{frame}

\begin{frame}
  \frametitle{Topics}
  \tableofcontents
\end{frame}

\section{Normalizasyon}

\subsection{Giriş}

\begin{frame}
  \frametitle{İşlevsel Bağımlılık}

    \begin{itemize}
      \item $Z$: $R$ bağıntısının bütün nitelikleri kümesi
      \item $A,B \subseteq Z$
      \item \alert{$A$, $B$'yi işlevsel olarak belirliyor}: $A \rightarrow B$\\
        her $A$ değerine karşılık tek bir $B$ değeri olabilir

      \medskip
      \item her işlevsel bağımlılık bir bütünlük kısıtlaması
  \end{itemize}
\end{frame}

\begin{frame}
\label{example_db_1}
  \frametitle{Örnek Bağıntı}
  
  \begin{tiny}
  \begin{table}
    \caption{R}
    \begin{tabular}{|r|l|c|c|r|l|r|}\hline
\underline{MOVIEID} & TITLE    & COU & LANG & \underline{ACTORID} & NAME & ORD\\[2pt]\hline\hline
      6 & The Usual Suspects   & UK  &  EN  &     308 & Gabriel Byrne    &   2\\\hline
    228 & Ed Wood              & US  &  EN  &      26 & Johnny Depp      &   1\\\hline
     70 & Being John Malkovich & US  &  EN  &     282 & Cameron Diaz     &   2\\\hline
   1512 & Suspiria             & IT  &  IT  &     745 & Udo Kier         &   9\\\hline
     70 & Being John Malkovich & US  &  EN  &     503 & John Malkovich   &  14\\\hline
    \end{tabular}
  \end{table}
  \end{tiny}

  \pause
  \begin{itemize}
    \item varsayım: film hangi ülkede çekildiyse o ülkenin dilinde
  \end{itemize}
\end{frame}


\begin{frame}
  \frametitle{İşlevsel Bağımlılık Örnekleri}
  
  \begin{itemize}
    \item MOVIEID $\rightarrow$ TITLE
    \item MOVIEID $\rightarrow$ \{TITLE, COUNTRY, LANGUAGE\}
    \item ACTORID $\rightarrow$ NAME
    \item \{MOVIEID, ACTORID\} $\rightarrow$ ORD

    \pause
    \medskip
    \item gereksiz: MOVIEID $\rightarrow$ MOVIEID
    \item tekrarlı: \{MOVIEID, ACTORID\} $\rightarrow$ COUNTRY
  \end{itemize}
\end{frame}

\begin{frame}
  \frametitle{İndirgenemez Küme}

  \begin{itemize}
    \item $S$: bağıntının bütün işlevsel bağımlılıkları kümesi
    \item $T \subseteq S$

    \medskip
    \item $T$ işlevsel bağımlılıkların indirgenemez bir kümesidir:

    \smallskip
    \item $T$ olabildiğince az eleman içerir
    \item $S$'deki her işlevsel bağımlılık $T$'dekilerden türetilebilir

    \pause
    \medskip
    \item işlevsel bağımlılıkların sağ yanlarında tek nitelik yer alsın
  \end{itemize}
\end{frame}


\begin{frame}
  \label{example_db_1}
  \frametitle{İndirgenemez Küme Örneği}

    \begin{itemize}
       \item MOVIEID $\rightarrow$ TITLE
       \item MOVIEID $\rightarrow$ COUNTRY
       \item COUNTRY $\rightarrow$ LANGUAGE
       \item ACTORID $\rightarrow$ NAME
       \item \{MOVIEID, ACTORID\} $\rightarrow$ ORD
  \end{itemize}
\end{frame}

\begin{frame}
  \frametitle{Bağımlılık Çizeneği}

  \begin{center}
    \pgfuseimage{norm1}
  \end{center}
\end{frame}

\subsection{Normal Formlar}

\begin{frame}
  \frametitle{Normal Formlar}

  \begin{itemize}
    \item normal form: bağıntıların sağlamak zorunda oldukları koşullar 
    
    \medskip
    \item 1NF, 2NF, 3NF, BCNF, 4NF, 5NF
    \item her form bir önceki formun kapsamını daraltır
    \item veri tabanının normal formu = en geniş kapsamlı bağıntının normal formu 
    
    \pause
    \medskip
    \item 1NF: nitelik değerleri bölünmezdir
    
    \pause
    \medskip
    \item tanımlar için basitleştirici varsayım:\\
      \alert{BİR TABLONUN BİRİNCİL ANAHTARI DA OLAN\\
          SADECE BİR ANAHTAR ADAYI VARDIR.}
  \end{itemize}
\end{frame}

\begin{frame}
  \frametitle{Normalizasyon}

  \begin{itemize}
    \item \alert{normalizasyon}:\\
      bir formdan daha dar kapsamlı bir sonraki forma geçiş

      \item formlar arası geçişler kayıpsız olmalı
    \end{itemize}

  \pause
  \begin{teorem}[Heath Kuramı]
    \begin{itemize}
      \item $Z$: $R$ bağıntısının bütün nitelikleri kümesi
      \item $A,B,C \subseteq Z$

      \medskip
      \item $A \rightarrow B$ ise $R$ bağıntısı $\{A,B\}$ ile $\{A,C\}$
        bağıntılarının\\
        birleştirilmesiyle elde edilebilir
    \end{itemize}
  \end{teorem}
\end{frame}

\begin{frame}
  \frametitle{Kayıpsız Geçiş Örneği}

\begin{columns}[c]
    \column{.65\textwidth}
    \vspace{-12pt}
    \begin{footnotesize}
    \begin{table}
      \caption{R1}
      \vspace{-6pt}
      \begin{tabular}{|r|l|c|c|}\hline
MOVIEID & TITLE                & COU & LANG\\\hline\hline
      6 & The Usual Suspects   & UK  &  EN \\\hline
    228 & Ed Wood              & US  &  EN \\\hline
     70 & Being John Malkovich & US  &  EN \\\hline
   1512 & Suspiria             & IT  &  IT \\\hline
      \end{tabular}
    \end{table}
    \end{footnotesize}

    \vspace{-12pt}
    \begin{footnotesize}
    \begin{table}
      \caption{R2}
      \begin{tabular}{|r|r|l|r|}\hline
MOVIEID & ACTORID & NAME           & ORD\\\hline\hline
      6 &     308 & Gabriel Byrne  &   2\\\hline
    228 &      26 & Johnny Depp    &   1\\\hline
     70 &     282 & Cameron Diaz   &   2\\\hline
   1512 &     745 & Udo Kier       &   9\\\hline
     70 &     503 & John Malkovich &  14\\\hline
      \end{tabular}
    \end{table}
    \end{footnotesize}

    \column{.35\textwidth}
    \begin{itemize}
      \item $R = R1 ~join~ R2$
    \end{itemize}
  \end{columns}
\end{frame}

\begin{frame}
  \frametitle{Kayıplı Geçiş Örneği}

   \begin{columns}[c]
    \column{.62\textwidth}
    \vspace{-12pt}
    \begin{footnotesize}
    \begin{table}
      \caption{R1}
      \vspace{-6pt}
      \begin{tabular}{|r|l|c|c|}\hline
MOVIEID & TITLE                & COU & LANG\\\hline\hline
      6 & The Usual Suspects   & UK  &  EN \\\hline
    228 & Ed Wood              & US  &  EN \\\hline
     70 & Being John Malkovich & US  &  EN \\\hline
   1512 & Suspiria             & IT  &  IT \\\hline
      \end{tabular}
    \end{table}
    \end{footnotesize}

    \vspace{-12pt}
    \begin{footnotesize}
    \begin{table}
      \caption{R2}
      \begin{tabular}{|c|r|l|r|}\hline
COU & ACTORID & NAME           & ORD\\\hline\hline
UK  &     308 & Gabriel Byrne  &   2\\\hline
US  &      26 & Johnny Depp    &   1\\\hline
US  &     282 & Cameron Diaz   &   2\\\hline
IT  &     745 & Udo Kier       &   9\\\hline
US  &     503 & John Malkovich &  14\\\hline
      \end{tabular}
    \end{table}
    \end{footnotesize}

    \column{.38\textwidth}
    \begin{itemize}
      \item $R \neq R1 ~join~ R2$

      \pause
      \item \footnotesize{\{MOVIEID, ACTORID\} \\ $\rightarrow$ ORD}
    \end{itemize}
  \end{columns}
\end{frame}

\begin{frame}
  \frametitle{Aykırılıklar}

  \begin{itemize}
    \item \emph{ekleme}: bilinen bir verinin kısıtlamalar nedeniyle tutulamaması

    \medskip
    \item \emph{silme}: bir veri silinmek istendiğinde başka bir verinin de yitirilmesi

    \medskip
    \item \emph{güncelleme}: bir veriyi güncellemek için birden fazla çokluda\\ 
       değişiklik gerekmesi
  \end{itemize}
\end{frame}

\begin{frame}
  \frametitle{Aykırılık Örnekleri}
     
   \hyperlink{exampl}{\beamergotobutton{example database}} 

    \begin{itemize}
      \item "Gattaca" filminin ülkesinin US olduğu biliniyor\\
	ama filmde oynayan bir oyuncu olmadıkça eklenemiyor

      \pause
      \medskip
      \item Gabriel Byrne'in "Usual Suspects" filminde oynadığı silinirse\\
        filmin ülkesinin UK olduğu da siliniyor

      \pause
      \medskip
      \item "Being John Malkovich" filminin ülkesinin güncellenmesi\\
        iki çokluda değişiklik gerektiriyor
    \end{itemize}
\end{frame}

\begin{frame}
  \frametitle{2. Normal Form}

  \begin{itemize}
    \item \alert{2NF}: anahtar olmayan her nitelik birincil anahtara bağımlı
    
    \pause
    \medskip
    
    \item 1NF'ye uyan bir $R$ bağıntısında:
    \item $R(A,B,C,D)$, birincil anahtar: $\{A,B\}$ ve
    \item $A \rightarrow D$

    \medskip  
    \item 2NF olması için:
    \item $R1(A,D)$, birincil anahtar: $A$ ve \\
      $R2(A,B,C)$, birincil anahtar: $\{A,B\}$\\
      $A$, $R1$'e başvuran dış anahtar
    \end{itemize}
\end{frame}

\begin{frame}
  \frametitle{1NF-2NF Geçişi Örneği}

    \begin{itemize}
      \item anahtar olmayan niteliklerden sadece ORD birincil anahtara bağımlı
  
      \item $A$: \{MOVIEID\}\\
        $B$: \{ACTORID\}\\
        $C$: \{NAME, ORD\}\\
        $D$: \{TITLE, COUNTRY, LANGUAGE\}
         
       \pause
       \medskip
       
       \item R1(MOVIEID, TITLE, COUNTRY, LANGUAGE)\\
          birincil anahtar: MOVIEID
        
        \item R2(MOVIEID, ACTORID, NAME, ORD)\\
          birincil anahtar: \{MOVIEID, ACTORID\}\\
          MOVIEID, R1'e başvuran dış anahtar
    \end{itemize}
\end{frame}


\begin{frame}
  \frametitle{1NF-2NF Geçişi Örneği}

    \begin{itemize}
      \item R2 hala 2NF değil: ACTORID $\rightarrow$ NAME
      \item $A$: \{ACTORID\}\\
        $B$: \{MOVIEID\}\\
        $C$: \{ORD\}\\
        $D$: \{NAME\}

    \pause
    \medskip

      \item R3(ACTORID, NAME)\\
        birincil anahtar: ACTORID


      \item R4(MOVIEID, ACTORID, ORD)\\
        birincil anahtar: \{MOVIEID, ACTORID\}\\
        ACTORID, R3'e başvuran dış anahtar
    \end{itemize}
\end{frame}

\begin{frame}
\label{example_db_2}
  \frametitle{2NF Bağıntı Örnekleri}

\vspace{-12pt}
  \begin{center}
    \begin{footnotesize}
    \begin{table}
      \caption{R1}
      \begin{tabular}{|r|l|c|c|}\hline
\underline{MOVIEID} & TITLE    & COU & LANG\\[2pt]\hline\hline
      6 & The Usual Suspects   & UK  &  EN \\\hline
    228 & Ed Wood              & US  &  EN \\\hline
     70 & Being John Malkovich & US  &  EN \\\hline
   1512 & Suspiria             & IT  &  IT \\\hline
      \end{tabular}
    \end{table}
    \end{footnotesize}
  \end{center}

  \vspace{-12pt}
  \begin{columns}[t]
    \column{.5\textwidth}
    \begin{footnotesize}
    \begin{table}
      \caption{R3}
      \begin{tabular}{|r|l|}\hline
\underline{ACTORID} & NAME\\[2pt]\hline\hline
     308 & Gabriel Byrne \\\hline
      26 & Johnny Depp   \\\hline
     282 & Cameron Diaz  \\\hline
     745 & Udo Kier      \\\hline
     503 & John Malkovich\\\hline
      \end{tabular}
    \end{table}
    \end{footnotesize}

    \column{.5\textwidth}
    \begin{footnotesize}
    \begin{table}
      \caption{R4}
      \begin{tabular}{|r|r|r|}\hline
\underline{MOVIEID} & \underline{ACTORID} & ORD\\[2pt]\hline\hline
   6 & 308 &  2\\\hline
 228 &  26 &  1\\\hline
  70 & 282 &  2\\\hline
1512 & 745 &  9\\\hline
  70 & 503 & 14\\\hline
      \end{tabular}
    \end{table}
    \end{footnotesize}
  \end{columns}
\end{frame}

\begin{frame}
  \frametitle{Bağımlılık Çizeneği Örneği}

  \begin{center}
    \pgfuseimage{norm2}
  \end{center}
\end{frame}

\begin{frame}
  \frametitle{2NF Düzelen Aykırılıklar}
  
  \hyperlink{example_db_2}{\beamergotobutton{example database}}

    \begin{itemize}
      \item "Gattaca" filminin ülkesinin US olduğu biliniyorsa\\
	bu bilgi R1 bağıntısına eklenebilir

      \pause
      \medskip
      \item Gabriel Byrne'in "Usual Suspects" filminde oynadığı silinse de\\
        filmin ülkesinin UK olduğu bilgisi R1 bağıntısında kalır

      \pause
      \medskip
      \item "Being John Malkovich" filminin ülkesini güncellemek için\\
        R1 bağıntısında tek çokluda değişiklik yapmak yeterli
    \end{itemize}
\end{frame}

\begin{frame}
  \frametitle{2NF Düzelmeyen Aykırılıklar}
  
  \hyperlink{example_db_2}{\beamergotobutton{example database}}

    \begin{itemize}
      \item Brezilya'da çekilen filmlerin Portekizce olduğu biliniyor\\
	ama Brezilya'da çekilen bir film olmadıkça eklenemiyor

      \pause
      \medskip
      \item "Suspiria" filmi silinirse İtalya'da çekilen filmlerin\\
        İtalyanca olduğu da siliniyor

      \pause
      \medskip
      \item Amerika'da çekilen filmlerin dilinin güncellenmesi\\
        iki çokluda değişiklik gerektiriyor
    \end{itemize}

\end{frame}

\subsection{3. Normal Form}

\begin{frame}
  \frametitle{3. Normal Form}

  \begin{itemize}
    \item \alert{3NF}: anahtar olmayan nitelikler birincil anahtar dışında\\
      niteliklere bağımlı değil


  \pause
  \medskip
      \item 2NF'ye uyan bir $R$ bağıntısında:
      \item $R(A,B,C,D)$, birincil anahtar: $A$ ve \\
        $C \rightarrow D$

      \medskip
      \item 2NF'den 3NF'ye geçiş için:
        \item $R1(C,D)$, birincil anahtar: $C$\\
          $R2(A,B,C)$, birincil anahtar: $A$\\
          $C$, $R1$'e başvuran dış anahtar
    \end{itemize}

\end{frame}

\begin{frame}
  \frametitle{2NF-3NF Geçişi Örneği}

    \begin{itemize}
      \item R1: COUNTRY $\rightarrow$ LANGUAGE
      \item $A$: \{MOVIEID\}\\
      $B$: \{TITLE\}\\
      $C$: \{COUNTRY\}\\
      $D$: \{LANGUAGE\}

    \pause
    \medskip
      \item R5(COUNTRY, LANGUAGE)\\
        birincil anahtar: COUNTRY

      \item R6(MOVIEID, TITLE, COUNTRY)\\
        birincil anahtar: MOVIEID\\
        COUNTRY, R5'e başvuran dış anahtar
    \end{itemize}
\end{frame}

\begin{frame}
\label{example_db_3}
  \frametitle{3NF Bağıntı Örnekleri}

  \vspace{-24pt}
  \begin{columns}[t]
    \column{.6\textwidth}
    \begin{footnotesize}
    \begin{table}
      \caption{R6}
      \begin{tabular}{|r|l|c|}\hline
\underline{MOVIEID} & TITLE    & COU\\[2pt]\hline\hline
      6 & The Usual Suspects   & UK \\\hline
    228 & Ed Wood              & US \\\hline
     70 & Being John Malkovich & US \\\hline
   1512 & Suspiria             & IT \\\hline
      \end{tabular}
    \end{table}
    \end{footnotesize}

    \column{.4\textwidth}
    \begin{footnotesize}
    \begin{table}
      \caption{R5}
      \begin{tabular}{|c|c|}\hline
\underline{COU} & LANG\\[2pt]\hline\hline
UK & EN\\\hline
US & EN\\\hline
IT & IT\\\hline
      \end{tabular}
    \end{table}
    \end{footnotesize}
  \end{columns}

  \vspace{-24pt}
  \begin{columns}[t]
    \column{.5\textwidth}
    \begin{footnotesize}
    \begin{table}
      \caption{R3}
      \begin{tabular}{|r|l|}\hline
\underline{ACTORID} & NAME\\[2pt]\hline\hline
      308 & Gabriel Byrne \\\hline
       26 & Johnny Depp   \\\hline
      282 & Cameron Diaz  \\\hline
      745 & Udo Kier      \\\hline
      503 & John Malkovich\\\hline
      \end{tabular}
    \end{table}
    \end{footnotesize}

    \column{.5\textwidth}
    \begin{footnotesize}
    \begin{table}
      \caption{R4}
      \begin{tabular}{|r|r|r|}\hline
\underline{MOVIEID} & \underline{ACTORID} & ORD\\[2pt]\hline\hline
   6 & 308 &  2\\\hline
 228 &  26 &  1\\\hline
  70 & 282 &  2\\\hline
1512 & 745 &  9\\\hline
  70 & 503 & 14\\\hline
      \end{tabular}
    \end{table}
    \end{footnotesize}
  \end{columns}
\end{frame}

\begin{frame}
  \frametitle{Bağımlılık Çizeneği Örneği}

  \begin{center}
    \pgfuseimage{norm3}
  \end{center}
\end{frame}

\begin{frame}
  \frametitle{3NF Düzelen Aykırılıklar}
  
  \hyperlink{example_db_3}{\beamergotobutton{örnek veri tabanı}}

    \begin{itemize}
      \item Brezilya'da çekilen filmlerin Portekizce olduğu biliniyorsa\\
	R5 bağıntısına eklenebilir

      \pause
      \medskip
      \item "Suspiria" filmi silinse de İtalya'da çekilen filmlerin\\
	İtalyanca olduğu R5 bağıntısında kalır

      \pause
      \medskip
      \item Amerika'da çekilen filmlerin dilini güncellemek için\\
	R5 bağıntısında tek çokluda değişiklik yapmak yeterli
    \end{itemize}
\end{frame}

\begin{frame}
  \frametitle{Boyce-Codd Normal Formu}

  \begin{itemize}
    \item \alert{BCNF}: bütün işlevsel bağımlılıklar anahtar adaylarına
    \item anahtarı oluşturan nitelikler arasındaki bağımlılıklar\\
      dikkate alınmalı
  \end{itemize}
\end{frame}

\begin{frame}
  \frametitle{BCNF Örneği}
  \hyperlink{example_db_1}{\beamergotobutton{örnek veri tabanı}}

     \begin{itemize}  
       \item filmlerin başlık nitelikleri eşsiz
       \item anahtar adayları:
          \{MOVIEID, ACTORID\}
          \{TITLE, ACTORID\}
 

      \pause
      \medskip
      \item aykırı işlevsel bağımlılıklar:
 
         MOVIEID $\rightarrow$ TITLE
         TITLE $\rightarrow$ MOVIEID
    \end{itemize}
\end{frame}

\subsection{Görüntüler}

\lstset{language=ExtendedSQL}

\begin{frame}[fragile]
  \frametitle{Görüntüler}

  \begin{itemize}
    \item türetilmiş tabloyu taban tablo gibi göstermek: \alert{görüntü}
    \item veri tabanı yapısındaki değişikliklerden\\
      kullanıcıların ve uygulama programlarının etkilenmemesi

    \pause
    \bigskip
    \item görüntü yaratma:
    \begin{lstlisting}
CREATE VIEW view_name AS
  SELECT ...
    \end{lstlisting}

    \medskip
    \item görüntü üzerindeki her işlemde\\
      \lstinline!SELECT! komutu yeniden çalıştırılır
  \end{itemize}
\end{frame}

\begin{frame}[fragile]
  \frametitle{Görüntü Örneği}

  \begin{itemize}
    \item filmlerin kimlik bilgileri, başlıkları ve yılları

    \medskip
    \begin{lstlisting}
CREATE VIEW NEW_MOVIE AS
  SELECT ID, TITLE, YR FROM MOVIE
    WHERE (YR > 1995)

SELECT * FROM NEW_MOVIE
    \end{lstlisting}
  \end{itemize}
\end{frame}

\begin{frame}[fragile]
  \frametitle{Görüntüde Güncelleme}

  \begin{itemize}
    \item güncellemeler taban tablolarda yapılmalı
    \item kural belirtilmeli

    \pause
    \bigskip
    \item kural yaratma:
    \begin{lstlisting}
CREATE RULE rule_name AS
  ON event TO view_name
  [ WHERE condition ]
  DO [ INSTEAD ] sql_statement
    \end{lstlisting}
  \end{itemize}
\end{frame}

\begin{frame}[fragile]
  \frametitle{Görüntü Kuralı Örneği}

  \begin{itemize}
    \item film başlığının değişmesi 
    \begin{lstlisting}
UPDATE NEW_MOVIE SET TITLE = ...
  WHERE (ID = ...)
    \end{lstlisting}

    \pause
    \medskip
    \item taban tablonun güncellenmesi için kural
    \begin{lstlisting}
CREATE RULE UPDATE_TITLE AS
  ON UPDATE TO NEW_MOVIE
  DO INSTEAD
    UPDATE MOVIE SET TITLE = new.TITLE
      WHERE (ID = old.ID)
    \end{lstlisting}
  \end{itemize}
\end{frame}

\subsection*{Kaynaklar}

\begin{frame}
  \frametitle{Kaynaklar}

  \begin{block}{Okunacak: Date}
    \begin{itemize}
      \item Chapter 11: \alert{Functional Dependencies}
      \item Chapter 12: \alert{Further Normalization I: 1NF, 2NF, 3NF, BCNF}
       \item Chapter 10: \alert{Views}
    \end{itemize}
  \end{block}
\end{frame}

\section{Varlık/İlişki Modeli}

\subsection{Giriş}

\begin{frame}
  \frametitle{Varlık/İlişki Modeli}

  \begin{itemize}
    \item modelleme yaklaşımı (Chen 1976)

    \medskip
      \item varlıklar
      \item özellikler
      \item ilişkiler
  \end{itemize}
\end{frame}

\begin{frame}
  \frametitle{Varlıklar}

  \begin{itemize}
    \item \alert{varlık}: aynı özellikleri taşıyan "şeyler" kümesi
    \item küme elemanları varlık tipinin birer \emph{örneği}

    \medskip
    \item \emph{güçlü}: tek başına var olabilir
    \item \emph{zayıf}: varlığı başka bir varlığa bağlı
    \end{itemize}
\end{frame}

\begin{frame}
  \frametitle{Varlık Örnekleri}

    \begin{itemize}
      \item varlık: film, kişi
      \item kişi örneği: Johnny Depp

      \pause
      \medskip
      \item güçlü varlık: kişi
      \item zayıf varlık: film
    \end{itemize}
\end{frame}

\begin{frame}
  \frametitle{Özellikler}

  \begin{itemize}
    \item \alert{özellik}: varlıkları ya da ilişkileri betimleyen veriler

    \medskip
    \item basit / bileşke
    \item anahtar
    \item tekli / çoklu değerli
    \item boş
    \item taban / türetilmiş
  \end{itemize}

\end{frame}

\begin{frame}
  \frametitle{Özellik Örnekleri}

    \begin{itemize}
      \item özellik: başlık, ülke, dil

      \pause
      \medskip
      \item basit: önad, soyad
      \item bileşke: tam ad

      \pause
      \medskip
      \item taban: doğum tarihi
      \item türetilmiş: yaş
    \end{itemize}
\end{frame}

\begin{frame}
  \frametitle{İlişkiler}

  \begin{itemize}
    \item \alert{ilişki}: varlıklar arasındaki bağlantılar

    \medskip

      \item \emph{katılımcı}: ilişkideki varlıklar
      \item \emph{derece}: katılımcı sayısı
      \item \emph{total} / \emph{kısmi}: bütün örnekler ilişkiye
        katılıyor / katılmıyor
    \end{itemize}
\end{frame}

\begin{frame}
  \frametitle{İlişki Türleri}
  
  \begin{itemize}
    \item \alert{bire bir}
    \item örneğin ülkeler ile şehirler arasındaki başkentlik ilişkisi

    \pause
    \medskip
    \item \alert{bire çok}
    \item örneğin çalışanlar ile projeler arasındaki yöneticilik ilişkisi

    \pause
    \medskip
    \item \alert{çoka çok}
    \item örneğin öğrenciler ile dersler arasında kayıt ilişkisi
  \end{itemize}

\end{frame}



\subsection{V/İ Çizenekleri}

\begin{frame}
  \frametitle{Varlık/İlişki Çizenekleri}
  
  \begin{itemize}
    \item varlık: dikdörtgen
    \item zayıf: çift çizgi

    \pause
    \medskip
    \item özellik: elips
    \item bileşke: alt-elipsler

    \pause
    \medskip
    \item ilişki: eşkenar dörtgen
    \item zayıf-güçlü arasında: çift çizgi
    \item total: bağlantı çift çizgi
    \item ilişkinin türüne göre 1 ya da n
 \end{itemize}
\end{frame}

\begin{frame}
  \frametitle{Varlık/İlişki Çizeneği Örneği}

  \begin{center}
    \pgfuseimage{imdb1}
  \end{center}
\end{frame}

\begin{frame}
  \frametitle{Varlık/İlişki Çizeneği Örneği}

  \begin{center}
    \pgfuseimage{imdb2}
  \end{center}
\end{frame}

\begin{frame}
  \frametitle{Tasarıma Geçiş}

  \begin{itemize}
    \item her varlık bir bağıntı

    \pause
    \item her özellik bir nitelik

    \pause
    \item her çoka çok ilişki bir bağıntı
    \begin{itemize}
      \item katılımcı varlıklara dış anahtarlar
    \end{itemize}

    \pause
    \item her bire çok ilişki için ilişkinin "çok" tarafından\\
      "bir" tarafına dış anahtar
 \end{itemize}
\end{frame}

\subsection*{Kaynaklar}

\begin{frame}
  \frametitle{Kaynaklar}

  \begin{block}{Okunacak: Date}
    \begin{itemize}
      \item Chapter 14: \alert{Semantic Modeling}
    \end{itemize}
  \end{block}
\end{frame}

\end{document}
