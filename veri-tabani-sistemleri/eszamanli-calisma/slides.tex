% Copyright (c) 2002-2016
%       H. Turgut Uyar <uyar@itu.edu.tr>
%       Şule Gündüz Öğüdücü <sgunduz@itu.edu.tr>
%
% This work is licensed under a "Creative Commons
% Attribution-NonCommercial-ShareAlike 4.0 International License".
% For more information, please visit:
% https://creativecommons.org/licenses/by-nc-sa/4.0/

\documentclass[dvipsnames]{beamer}

\usepackage{ae}
\usepackage[scaled=0.88]{beramono}
\usepackage[T1]{fontenc}
\usepackage[utf8]{inputenc}
\usepackage[turkish]{babel}
\setbeamersize{text margin left=2em, text margin right=2em}
\usepackage[labelformat=empty, aboveskip=1pt, belowskip=1pt]{caption}

\usepackage{listings}
\lstdefinelanguage{FullSQL}[]{SQL}{
  morekeywords={BINARY, BOOLEAN, CYCLE, FINAL, INCREMENT, IS, LARGE, MAXVALUE,
                MINVALUE, NO_ACTION, OBJECT, REFERENCES, RENAME, SEQUENCE,
                START, TO, TYPE, VACUUM}
}
\lstdefinelanguage{ExtendedSQL}[]{FullSQL}{
  morekeywords={ACCESS, BEGIN, COMMITTED, ERROR, EXCLUSIVE, FOR, GOTO, LOCK,
                MODE, ON, REPEATABLE, ROW, SERIALIZABLE, SHARE, UNCOMMITTED,
                WORK}
}
\lstset{basicstyle=\ttfamily, keywordstyle=\color{blue}, showstringspaces=false}
\lstset{language=ExtendedSQL}

\mode<presentation>
{
  \usetheme{Warsaw}
  \usecolortheme[named=ForestGreen]{structure}
  \setbeamercovered{transparent}
}

\title{Veri Tabanı Sistemleri}
\subtitle{Eşzamanlı Çalışma}

\author{H. Turgut Uyar \and Şule Öğüdücü}
\date{2002-2016}

\AtBeginSubsection[]{
  \begin{frame}<beamer>
    \frametitle{Konular}
    \tableofcontents[currentsection,currentsubsection]
  \end{frame}
}

\theoremstyle{plain}

\pgfdeclareimage[width=2cm]{license}{../license}

\pgfdeclareimage[height=4.5cm]{duzeltme}{duzeltme}
\pgfdeclareimage{bekleme}{bekleme}
\pgfdeclareimage[height=4.5cm]{oncelik}{oncelik}

\begin{document}

\begin{frame}
  \titlepage
\end{frame}

\begin{frame}
 \frametitle{License}

  \pgfuseimage{license}\hfill
  \copyright~2002-2016 T. Uyar, Ş. Öğüdücü

  \vfill
  \begin{footnotesize}
    You are free to:
    \begin{itemize}
      \itemsep0em
      \item Share -- copy and redistribute the material in any medium or format
      \item Adapt -- remix, transform, and build upon the material
    \end{itemize}

    Under the following terms:
    \begin{itemize}
      \itemsep0em
      \item Attribution -- You must give appropriate credit, provide a link to
        the license, and indicate if changes were made.

      \item NonCommercial -- You may not use the material for commercial
        purposes.

      \item ShareAlike -- If you remix, transform, or build upon the material,
        you must distribute your contributions under the same license as the
        original.
    \end{itemize}
  \end{footnotesize}

  \begin{small}
    For more information:\\
    \url{https://creativecommons.org/licenses/by-nc-sa/4.0/}

    \smallskip
    Read the full license:\\
    \url{https://creativecommons.org/licenses/by-nc-sa/4.0/legalcode}
  \end{small}
\end{frame}

\section{Hareket Yönetimi}

\subsection{Giriş}

\begin{frame}
  \frametitle{Hareket Yönetimi}

  \begin{itemize}
    \item birden fazla işlemin topluca yapılması gerekebilir
    \item bir işlemin yapılıp diğerlerinin yapılmaması tutarsızlık yaratabilir
    \item \alert{hareket}: bir işin mantıksal bir birimi

    \pause
    \medskip
    \item birden fazla işlemin topluca yapılması garanti edilemez
    \item en azından değişikliklerden önceki duruma dönülebilmeli
  \end{itemize}
\end{frame}

\begin{frame}[fragile]
  \frametitle{Hareket Örneği}

  \begin{itemize}
    \item bir banka hesabından diğerine para aktarma

    \medskip
    \begin{lstlisting}
UPDATE ACCOUNTS SET BALANCE = BALANCE - 100
  WHERE ACCOUNTID = 123

UPDATE ACCOUNTS SET BALANCE = BALANCE + 100
  WHERE ACCOUNTID = 456
    \end{lstlisting}
  \end{itemize}
\end{frame}

\begin{frame}
  \frametitle{Hareket Özellikleri}

  \begin{itemize}
    \item bölünmezlik: ya tam yapılır, ya hiç yapılmaz

    \pause
    \medskip
    \item tutarlılık: bir tutarlı durumdan diğer bir tutarlı duruma geçiş

    \pause
    \medskip
    \item yalıtım: sona ermemiş bir hareketin işlemlerinin diğer hareketleri\\
        etkileyip etkilemediği

    \pause
    \medskip
    \item kalıcılık: bir hareket sonlandırıldıktan sonra sistem çökse de\\
        verilerin zarar görmemesi
  \end{itemize}
\end{frame}

\begin{frame}[fragile]
  \frametitle{SQL Hareketleri}

  \begin{itemize}
    \item başlatma
    \begin{lstlisting}
BEGIN [ WORK | TRANSACTION ]
    \end{lstlisting}

    \item sonlandırma
    \begin{lstlisting}
COMMIT [ WORK | TRANSACTION ]
    \end{lstlisting}

    \item vazgeçme
    \begin{lstlisting}
ROLLBACK [ WORK | TRANSACTION ]
    \end{lstlisting}
  \end{itemize}
\end{frame}

\begin{frame}[fragile]
  \frametitle{Hareket Örneği}

  \begin{lstlisting}
  BEGIN TRANSACTION
  ON ERROR GOTO UNDO
  UPDATE ACCOUNTS SET BALANCE = BALANCE - 100
    WHERE (ACCOUNTID = 123)
  UPDATE ACCOUNTS SET BALANCE = BALANCE + 100
    WHERE (ACCOUNTID = 456)
  COMMIT
  ...

UNDO:
  ROLLBACK
  \end{lstlisting}
\end{frame}

\subsection{Sistemin Düzeltilmesi}

\begin{frame}
  \frametitle{Sistemin Düzeltilmesi}

  \begin{itemize}
    \item bir hareket sürerken sistemin çöktüğünü düşünün
    \item bellek tamponlarındaki veriler diske yazılmamış durumda

    \pause
    \medskip
    \item kalıcılık nasıl sağlanacak?
    \item veri, sistemde başka yerde yazılı verilerden türetilebilmeli
  \end{itemize}
\end{frame}

\begin{frame}
  \frametitle{Günlük}

  \begin{itemize}
    \item \alert{günlük} her işlemden etkilenen her çoklunun\\
      işlemden önceki ve sonraki değerlerini tutar

    \pause
    \medskip
    \item \alert{günlüğe önceden yazma kuralı}:\\
      hareket sonlanmadan önce günlük fiziksel ortama yazılmalı

    \pause
    \medskip
    \item günlük kayıtlarına erişim işlemin doğası gereği ardışıl
  \end{itemize}
\end{frame}

\begin{frame}
  \frametitle{Denetim Noktaları}

  \begin{itemize}
    \item belli aralıklarla günlükte \alert{denetim noktaları} oluşturulur

    \medskip
    \item bellek tamponlarındaki veriler fiziksel ortama yazılır
    \item denetim noktası günlüğe not edilir
    \item o an sürmekte olan hareketler not edilir
  \end{itemize}
\end{frame}

\begin{frame}
  \frametitle{Düzeltme Listeleri}

  \begin{itemize}
    \item aksaklıktan sonra hangi hareketler geri alınacak,\\
      hangileri sonlandırılacak?
    \item iki liste oluştur: \emph{geri alınacaklar} (G),
        \emph{yinelenecekler} (Y)

    \pause
    \medskip
    \item $t_C$: günlükte kayıtlı son denetim noktası
    \item $t_C$ anında etkin olan hareketleri G'ye ekle

    \medskip
    \item $t_C$'den başlayarak kayıtları ileri doğru tara
    \item başlayan bir hareketle karşılaşırsan G'ye ekle
    \item biten bir hareketle karşılaşırsan Y'ye geçir
  \end{itemize}
\end{frame}

\begin{frame}
  \frametitle{Düzeltme Örneği}

    \begin{columns}[t]
      \column{.4\textwidth}
      \begin{center}
        \pgfuseimage{duzeltme}
      \end{center}

      \pause
      \column{.5\textwidth}
      \begin{itemize}
        \item $t_C$:\\
          $G=[T_2,T_3]$
          $Y=[]$

        \pause
        \item $T_4$ başladı:\\
          $G=[T_2,T_3,T_4]$
          $Y=[]$

        \pause
        \item $T_2$ bitti:\\
          $G=[T_3,T_4]$
          $Y=[T_2]$

        \pause
        \item $T_5$ başladı:\\
          $G=[T_3,T_4,T_5]$
          $Y=[T_2]$

        \pause
        \item $T_4$ bitti:\\
          $G=[T_3,T_5]$
          $Y=[T_2,T_4]$
      \end{itemize}
    \end{columns}
\end{frame}

\begin{frame}
  \frametitle{Düzeltme Süreci}

  \begin{itemize}
    \item kayıtları günlük sonundan geriye doğru tara
    \item G'deki hareketlerin yaptıkları değişiklikleri geri al

    \medskip
    \item kayıtları ileriye doğru tara
    \item Y'deki hareketlerin yaptıkları değişiklikleri yinele
  \end{itemize}
\end{frame}

\subsection{İki Aşamalı Sonlandırma}

\begin{frame}
  \frametitle{İki Aşamalı Sonlandırma}

  \begin{itemize}
    \item farklı kaynak yöneticileri var
    \item geri alma - sonlandırma sistemleri ayrı

    \medskip
    \item etkilenecek veriler farklı kaynak yöneticilerinde
    \item ya hepsinde birden sonlandırılacak\\
        ya da hepsinde birden geri alınacak

    \medskip
    \item eşgüdüm sağlayıcı
  \end{itemize}
\end{frame}

\begin{frame}
  \frametitle{Protokol}

  \begin{itemize}
    \item eşgüdüm sağlayıcı $\rightarrow$ katılımcılar: \\
      hareketle ilgili bütün verilerin kayıtlarını kalıcı ortama yaz

    \pause
    \item eşgüdüm sağlayıcı $\rightarrow$ katılımcılar: \\
      hareketi başlat ve sonucu bildir
    
    \pause
    \medskip
    \item bütün katılımcılar başarılı: başarı
    \item diğer durumda: başarısız    
    
    \medskip    
    \item başarı ise: eşgüdüm sağlayıcı $\rightarrow$ katılımcılar: sonlandır
    \item  başarısız ise: eşgüdüm sağlayıcı $\rightarrow$ katılımcılar: vazgeç
  \end{itemize}
\end{frame}

\subsection*{Kaynaklar}

\begin{frame}
  \frametitle{Kaynaklar}

  \begin{block}{Okunacak: Date}
    \begin{itemize}
      \item Chapter 15: \alert{Recovery}
    \end{itemize}
  \end{block}
\end{frame}

\section{Eşzamanlı Çalışma}

\subsection{Giriş}

\begin{frame}
  \frametitle{Eşzamanlı Çalışma}

  \begin{itemize}
    \item eşzamanlı çalışan hareketler nedeniyle çıkabilecek sorunlar:

    \bigskip
    \item yitirilen güncelleme
    \item kesinleşmemiş veriye bağımlılık
    \item tutarsız çözümleme
  \end{itemize}
\end{frame}

\begin{frame}[fragile]
  \frametitle{Yitirilen Güncelleme}

  \begin{table}
    \begin{tabular}{ll}
Hareket A     & Hareket B    \\\hline
...           & ...          \\\pause
RETRIEVE p    & ...          \\\pause
...           & ...          \\
...           & RETRIEVE p   \\\pause
...           & ...          \\
UPDATE p      & ...          \\\pause
...           & ...          \\
...           & UPDATE p     \\
...           & ...
    \end{tabular}
  \end{table}
\end{frame}

\begin{frame}[fragile]
  \frametitle{Kesinleşmemiş Veriye Bağımlılık}


    \begin{table}
      \begin{tabular}{ll}
Hareket A  & Hareket B\\\hline
...        & ...      \\\pause
...        & UPDATE p \\\pause
...        & ...      \\
RETRIEVE p & ...      \\\pause
...        & ...      \\
...        & ROLLBACK \\
...        &
      \end{tabular}
    \end{table}
\end{frame}

\begin{frame}[fragile]
  \frametitle{Tutarsız Çözümleme}

  \begin{itemize}
    \item hesap toplamı: acc1=40, acc2=50, acc3=30
    \smallskip
    \begin{table}
      \begin{tabular}{ll}
Hareket A             & Hareket B                        \\\hline
...                   & ...                              \\\pause
RETRIEVE acc1 ($40$)  & ...                              \\\pause
RETRIEVE acc2 ($90$)  & ...                              \\\pause
...                   & ...                              \\
...                   & UPDATE acc3 ($30 \rightarrow 20$)\\
...                   & UPDATE acc1 ($40 \rightarrow 50$)\\
...                   & COMMIT                           \\\pause
...                   & ...                              \\
RETRIEVE acc3 ($110$) &                                  \\
...                   &
      \end{tabular}
    \end{table}
  \end{itemize}
\end{frame}

\begin{frame}
  \frametitle{Çelişmeler}

  \begin{itemize}
    \item A okuyor, B okuyor: sorun yok
    \item A okuyor, B yazıyor: yinelenemez okuma (tutarsız çözümleme)
    \item A yazıyor, B okuyor: kirli okuma (kesinleşmemiş veriye bağımlılık)
    \item A yazıyor, B yazıyor: kirli yazma (yitirilen güncelleme)
  \end{itemize}
\end{frame}


\subsection{Kilitleme}

\begin{frame}
  \frametitle{Kilitleme}

  \begin{itemize}
    \item hareketler üzerinde işlem yapacakları çokluları kilitlesinler

    \medskip
    \item \alert{okuma} kilidi (S)
    \item \alert{yazma} kilidi (X)
  \end{itemize}
\end{frame}

%\begin{frame}
%  \frametitle{Kilitleme}
%
%  \begin{itemize}
%    \item hareketler üzerinde işlem yapacakları çokluları kilitlesinler
%    \begin{itemize}
%      \item okuma kilidi (S)
%      \item yazma kilidi (X)
%    \end{itemize}
%
%    \item işleri bitince kilitleri bıraksınlar
%  \end{itemize}
%\end{frame}
%
\begin{frame}
  \frametitle{Kilit İstekleri}

    \begin{table}
      \caption{kilit tipi uyumluluk matrisi}
      \begin{tabular}{|c||c|c|c|}\hline
  & X & S & -\\\hline\hline
X & H & H & E\\\hline
S & H & E & E\\\hline
      \end{tabular}
    \end{table}
    
     \medskip
  \begin{itemize}
    \item okuma kildi varsa: sadece okuma kilidi istekleri kabul edilir
    \item yazma kilidi varsa: bütün kilit istekleri reddedilir
  \end{itemize}
\end{frame}

\begin{frame}
  \frametitle{Kilitleme Protokolü}

  \begin{itemize}
    \item hareket, yapmak istediği işleme göre kilit isteğinde bulunur
    \item okuma kilidi varsa yazma kilidine çevrilmesi

    \medskip
    \item istek yerine getirilemiyorsa beklemeye başlar
    \item diğer hareket kilidi bırakınca devam eder
    
    \medskip
    \item \alert{sonsuz bekleme}
  \end{itemize}
\end{frame}

\begin{frame}[fragile]
  \frametitle{Yitirilen Güncelleme}

    \begin{table}
      \begin{tabular}{ll}
Hareket A       & Hareket B      \\\hline
...             & ...            \\\pause
RETRIEVE p (S+) & ...            \\\pause
...             & ...            \\
...             & RETRIEVE p (S+)\\\pause
...             & ...            \\
UPDATE p (X-)   & ...            \\
bekle           & ...            \\\pause
bekle           & UPDATE p (X-)  \\
bekle           & bekle
      \end{tabular}
    \end{table}
\end{frame}

\begin{frame}[fragile]
  \frametitle{Kesinleşmemiş Veriye Bağımlılık}

    \begin{table}
      \begin{tabular}{ll}
Hareket A       & Hareket B    \\\hline
...             & ...          \\\pause
...             & UPDATE p (X+)\\\pause
...             & ...          \\
RETRIEVE p (S-) & ...          \\
bekle           & ...          \\\pause
bekle           & ROLLBACK     \\
RETRIEVE p (S+) &              \\
...             &
      \end{tabular}
    \end{table}
\end{frame}

\begin{frame}[fragile]
  \frametitle{Tutarsız Çözümleme}

    \begin{table}
      \begin{tabular}{ll}
Hareket A            & Hareket B       \\\hline
...                  & ...             \\\pause
RETRIEVE acc1 (S+)   & ...             \\\pause
RETRIEVE acc2 (S+)   & ...             \\\pause
...                  & ...             \\
...                  & UPDATE acc3 (X+)\\\pause
...                  & UPDATE acc1 (X-)\\
...                  & bekle           \\\pause
RETRIEVE acc3 (S-)   & bekle           \\
bekle                & bekle
      \end{tabular}
    \end{table}
\end{frame}

\begin{frame}
  \frametitle{Ölümcül Kilitlenme}

    \begin{itemize}
    \item \alert{ölümcül kilitlenme}: hareketlerin birbirlerinin kilitleri\\ 
       bırakmalarını beklemesi
    \item neredeyse her zaman iki hareket arasında
  
    \medskip
    \item farketmek ve çözmek
    \item önlemek
    \end{itemize}
\end{frame}

\begin{frame}
  \frametitle{Ölümcül Kilitlenmenin Çözülmesi}

  \begin{columns}[t]
    \column{.5\textwidth}
    \begin{exampleblock}{örnek}
      \begin{center}
        \pgfuseimage{bekleme}
      \end{center}
    \end{exampleblock}

    \column{.5\textwidth}
    \begin{itemize}
      \item bekleme grafı
      \item bir \alert{kurban} seç ve öldür
    \end{itemize}
  \end{columns}
\end{frame}


\begin{frame}
  \frametitle{Ölümcül Kilitlenmenin Önlenmesi}

  \begin{itemize}
    \item her hareketin başlama zamanı mührü var

    \pause
    \item A hareketinin kilit isteği\\
      B hareketinin tuttuğu bir kilitle çelişiyorsa:
      
    \medskip  
      \item \alert{bekle-öl}: A, B'den yaşlıysa bekler, gençse ölür\\
        A geri alınıp yeniden başlatılır

      \item \alert{yarala-bekle}: A, B'den gençse bekler, yaşlıysa B'yi
        yaralar\\
        B geri alınıp yeniden başlatılır

      \pause
      \medskip
      \item yeniden başlatılan hareketin zaman mührü değiştirilmez
  \end{itemize}
\end{frame}

\begin{frame}[fragile]
  \frametitle{Kilit Komutları}

  \begin{itemize}
    \item okuma kilidi
    \begin{lstlisting}
SELECT query FOR SHARE
    \end{lstlisting}

    \item yazma kilidi
    \begin{lstlisting}
SELECT query FOR UPDATE
    \end{lstlisting}
  \end{itemize}
\end{frame}

\subsection{Yalıtım Düzeyleri}

\begin{frame}
  \frametitle{Yalıtım Düzeyleri}

  \begin{itemize}
    \item yalıtım azaltılırsa eşzamanlılık artırılabilir
    \item değişik yalıtım düzeyleri:

    \bigskip
    \item serileştirilebilir
    \item yinelenebilir okuma
    \item sonlandırılanları okuyabilme
    \item sonlandırılmayanları okuyabilme
  \end{itemize}
\end{frame}

\begin{frame}
  \frametitle{Serileştirilebilirlik}

  \begin{itemize}
    \item \emph{seri çalıştırma}:\\
      hareketlerin biri bitmeden diğeri başlamıyor

    \pause
    \item \alert{serileştirilebilir}: eşzamanlı çalışmanın sonucu\\
      her zaman seri çalıştırmalardan birinin sonucu ile aynı
  \end{itemize}

  \pause
  \begin{exampleblock}{örnek}
    \begin{itemize}
      \item $x = 10$
      \item A hareketi: $x = x + 1$
      \item B hareketi: $x = 2 * x$

      \pause
      \medskip
      \item önce A, sonra B: $x = 22$
      \item önce B, sonra A: $x = 21$
    \end{itemize}
  \end{exampleblock}  
\end{frame}

\begin{frame}
  \frametitle{İki Aşamalı Kilitleme}

  \begin{itemize}
    \item \alert{iki aşamalı kilitleme}:\\
      herhangi bir kilit bırakıldıktan sonra\\
      yeni kilit isteğinde bulunulmaz
      \item genişleme aşaması: alınan kilit sayısı artıyor
      \item daralma aşaması: alınan kilit sayısı azalıyor

      \medskip
      \item \alert{iki aşamalı sıkı kilitleme}:\\
        bütün kilitler hareketin sonunda bırakılır

      \pause
      \medskip
      \item \emph{Bütün hareketler iki aşamalı kilitleme protokolüne uyarsa\\
      bütün eşzamanlı çalıştırmalar serileştirilebilir.}
  \end{itemize}
\end{frame}

\begin{frame}[fragile]
  \frametitle{Sonlandırılanları Okuyabilme}

  \begin{itemize}
    \item read commited: yalnızca yazma kilitleri hareket sonuna\\ 
      kadar tutulur
  \end{itemize}

    \begin{table}
      \begin{tabular}{ll}
Hareket A       & Hareket B    \\\hline
...             & ...          \\\pause
RETRIEVE p (S+) & ...          \\\pause
...             & ...          \\
kilidi bırak    & ...          \\\pause
...             & ...          \\
...             & UPDATE p (X+)\\
...             & COMMIT       \\\pause
RETRIEVE p (S+) &
      \end{tabular}
    \end{table}
\end{frame}

\begin{frame}
  \frametitle{Hayaletler}

  \begin{itemize}
    \item \alert{hayalet}: sorgu yeniden çalıştırıldığında yeni çoklular ortaya çıkıyor
  \end{itemize}

  \pause
  \begin{exampleblock}{örnek}
    \begin{itemize}
      \item A hareketi bir müşterinin hesaplarının ortalamasını hesaplıyor:\\
        \smallskip
        $\frac{100 + 100 + 100}{3}=100$

      \pause
      \item B hareketi aynı müşteriye ($200$) birimlik yeni bir hesap yaratıyor
      \item A hareketi hesabı yeniden yapıyor:\\
        \smallskip
        $\frac{100 + 100 + 100 + 200}{4}=125$
    \end{itemize}
  \end{exampleblock}
\end{frame}

\begin{frame}[fragile]
  \frametitle{Yalıtım Düzeyleri}

  \begin{itemize}
    \item bir yalıtım düzeyi belirleme
    \begin{lstlisting}
SET TRANSACTION ISOLATION LEVEL
  [ SERIALIZABLE | REPEATABLE READ |
    READ COMMITTED | READ UNCOMMITTED ]
    \end{lstlisting}
  \end{itemize}
\end{frame}

\begin{frame}[fragile]
  \frametitle{Yalıtım Düzeyi Sorunları}

  \begin{table}
    \begin{tabular}{|l||c|c|c|}\hline
yalıtım düzeyi   & kirli & yinelemeyen & hayalet\\
                 & okuma & okuma       &        \\\hline\hline
READ UNCOMMITTED & E     & E           & E      \\\hline
READ COMMITTED   & H     & E           & E      \\\hline
REPEATABLE READ  & H     & H           & E      \\\hline
SERIALIZABLE     & H     & H           & H      \\\hline
    \end{tabular}
  \end{table}
\end{frame}

\subsection{Niyet Kilitleri}

\begin{frame}
  \frametitle{Kilitleme Birimi}

  \begin{itemize}
    \item kilitleme çoklu değil bağıntı değişkeni biriminde yapılabilir
    \item hatta veri tabanı biriminde

    \medskip
    \item birim: kilitlenen birim    
    \item birim genişledikçe eşzamanlılık azalır

    \pause
    \medskip
    \item çoklular üzerinde alınmış kilitlerin bulunması zor\\
      $\rightarrow$ önce bağıntı değişkeni düzeyinde \alert{niyet kilitleri}
      alınsın
  \end{itemize}
\end{frame}

\begin{frame}
  \frametitle{Niyet Kilitleri}

  \begin{itemize}
    \item Parçayı Okuma (IS):\\
      hareket bazı çokluları okumaya niyetleniyor

    \item Parçaya Yazma (IX):\\
      IS + hareket bazı çoklulara yazmaya niyetleniyor

    \item Bütünü Okuma (S):\\
      bağıntıda eşzamanlı okuyucular olabilir ama yazıcılar olmamalı

    \item Bütünü Okuma + Parçaya Yazma (SIX):\\
      S + IX

    \item Bütüne Yazma (X):\\
      bağıntıda hiçbir eşzamanlı çalışma olmamalı
  \end{itemize}
\end{frame}

\begin{frame}
  \frametitle{Kilit İstekleri}

    \begin{table}
    \caption{kilit uyumluluk matrisi}
      \begin{tabular}{|c||c|c|c|c|c|c|}\hline
    & X & SIX & IX & S & IS & -\\\hline\hline
  X & H &  H  & H  & H & H  & E\\\hline
SIX & H &  H  & H  & H & E  & E\\\hline
 IX & H &  H  & E  & H & E  & E\\\hline
  S & H &  H  & H  & E & E  & E\\\hline
 IS & H &  E  & E  & E & E  & E\\\hline
      \end{tabular}
    \end{table}
\end{frame}

\begin{frame}
  \frametitle{Kilit Öncelikleri}

  \begin{columns}[t]
    \column{.4\textwidth}
    \begin{center}
      \pgfuseimage{oncelik}
    \end{center}

    \column{.6\textwidth}
    \begin{itemize}
      \item çoklu üzerinde okuma kilidi için\\
	bağıntı üzerinde en az IS
      \item çoklu üzerinde yazma kilidi için\\
	bağıntı üzerinde en az IX
    \end{itemize}
  \end{columns}
\end{frame}

\begin{frame}[fragile]
  \frametitle{Kilitleme Komutları}

    \begin{itemize}
      \item bir tabloyu kilitleme    
    \begin{lstlisting}
LOCK [ TABLE ] table_name
     [ IN lock_mode MODE ]
    \end{lstlisting}

      \item kilit kipleri:
      \begin{itemize}
        \item \lstinline!ACCESS SHARE!
        \item \lstinline!ROW SHARE!
        \item \lstinline!ROW EXCLUSIVE!
        \item \lstinline!SHARE UPDATE EXCLUSIVE!
        \item \lstinline!SHARE!
        \item \lstinline!SHARE ROW EXCLUSIVE!
        \item \lstinline!EXCLUSIVE!
        \item \lstinline!ACCESS EXCLUSIVE!
      \end{itemize}
    \end{itemize}
\end{frame}

\subsection*{Kaynaklar}

\begin{frame}
  \frametitle{Kaynaklar}

  \begin{block}{Okunacak: Date}
    \begin{itemize}
      \item Chapter 16: \alert{Concurrency}
    \end{itemize}
  \end{block}
\end{frame}

\end{document}
