% Copyright (c) 2002-2012
%       H. Turgut Uyar <uyar@itu.edu.tr>
%       Şule Gündüz Öğüdücü <sgunduz@itu.edu.tr>
%
% These notes are licensed using the
% "Creative Commons Attribution-NonCommercial-ShareAlike License".
% You are free to copy, distribute and transmit the work, and to adapt the work
% as long as you attribute the authors, do not use it for commercial purposes,
% and any derivative work is under the same or a similar license.
%
% Read the full legal code at:
% http://creativecommons.org/licenses/by-nc-sa/3.0/

\documentclass[dvipsnames]{beamer}

\usepackage{ae}
\usepackage[T1]{fontenc}
\usepackage[utf8]{inputenc}
\usepackage[turkish]{babel}
\setbeamertemplate{navigation symbols}{}
\usepackage[labelformat=empty, aboveskip=1pt, belowskip=1pt]{caption}

\usepackage{listings}
\lstdefinelanguage{FullSQL}[]{SQL}{
  morekeywords={BINARY, BOOLEAN, CYCLE, FINAL, INCREMENT, IS, LARGE, MAXVALUE,
                MINVALUE, NO_ACTION, OBJECT, REFERENCES, RENAME, SEQUENCE,
                START, TO, TYPE, VACUUM}
}
\lstdefinelanguage{ExtendedSQL}[]{FullSQL}{
  morekeywords={ACCESS, BEGIN, COMMITTED, ERROR, EXCLUSIVE, FOR, GOTO, LOCK,
                MODE, ON, REPEATABLE, ROW, SERIALIZABLE, SHARE, UNCOMMITTED,
                WORK}
}
\lstset{basicstyle=\ttfamily, keywordstyle=\color{blue}, showstringspaces=false}
\lstset{language=ExtendedSQL}

\mode<presentation>
{
  \usetheme{Warsaw}
  \usecolortheme[named=ForestGreen]{structure}
  \setbeamercovered{transparent}
}

\title{Veri Tabanı Sistemleri}
\subtitle{Eşzamanlı Çalışma}

\author{H. Turgut Uyar \and Şule Öğüdücü}
\date{2002-2012}

\AtBeginSubsection[]{
  \begin{frame}<beamer>
    \frametitle{Konular}
    \tableofcontents[currentsection,currentsubsection]
  \end{frame}
}

\theoremstyle{definition}
\newtheorem{tanim}[theorem]{Tanım}

\theoremstyle{example}
\newtheorem{ornek}[theorem]{Örnek}

\theoremstyle{plain}

\pgfdeclareimage[width=2cm]{license}{../../license}

\pgfdeclareimage[height=4.5cm]{duzeltme}{duzeltme}
\pgfdeclareimage{bekleme}{bekleme}
\pgfdeclareimage[height=4.5cm]{oncelik}{oncelik}

\begin{document}

\begin{frame}
  \titlepage
\end{frame}

\begin{frame}
  \frametitle{License}

  \pgfuseimage{license}\hfill
  \copyright 2002-2012 T. Uyar, Ş. Öğüdücü

  \vfill
  \begin{tiny}
    You are free:
    \begin{itemize}
      \item to Share -- to copy, distribute and transmit the work
      \item to Remix -- to adapt the work
    \end{itemize}

    Under the following conditions:
    \begin{itemize}
      \item Attribution -- You must attribute the work in the manner specified by
        the author or licensor (but not in any way that suggests that they
        endorse you or your use of the work).

      \item Noncommercial -- You may not use this work for commercial purposes.

      \item Share Alike -- If you alter, transform, or build upon this work, you
        may distribute the resulting work only under the same or similar license
        to this one.
    \end{itemize}
  \end{tiny}

  \vfill
  Legal code (the full license):\\
  \url{http://creativecommons.org/licenses/by-nc-sa/3.0/}
\end{frame}

\begin{frame}
  \frametitle{Konular}
  \tableofcontents
\end{frame}

\section{Hareket Yönetimi}

\subsection{Giriş}

\begin{frame}
  \frametitle{Hareket Yönetimi}

  \begin{itemize}
    \item birden fazla işlemin topluca yapılması gerekebilir
    \begin{itemize}
      \item bir işlemin yapılıp diğerlerinin yapılmaması tutarsızlık yaratabilir
    \end{itemize}

    \pause
    \item birden fazla işlemin topluca yapılması garanti edilemez
    \begin{itemize}
      \item en azından değişikliklerden önceki duruma dönülebilmeli
    \end{itemize}
  \end{itemize}

  \pause
  \begin{tanim}
    \alert{hareket}: bir işin mantıksal bir birimi
  \end{tanim}
\end{frame}

\begin{frame}[fragile]
  \frametitle{Hareket Örneği}

  \begin{ornek}[bir banka hesabından diğerine para aktarma]
    \begin{lstlisting}
UPDATE ACCOUNTS SET BALANCE = BALANCE - 100
  WHERE ACCOUNTID = 123

UPDATE ACCOUNTS SET BALANCE = BALANCE + 100
  WHERE ACCOUNTID = 456
    \end{lstlisting}
  \end{ornek}
\end{frame}

\begin{frame}
  \frametitle{Hareket Özellikleri}

  \begin{itemize}
    \item bölünmezlik
    \begin{itemize}
      \item ya tam yapılır, ya hiç yapılmaz
    \end{itemize}

    \pause
    \item tutarlılık
    \begin{itemize}
      \item bir tutarlı durumdan diğer bir tutarlı duruma geçiş
    \end{itemize}

    \pause
    \item yalıtım
    \begin{itemize}
      \item sona ermemiş bir hareketin işlemlerinin diğer hareketleri\\
        etkileyip etkilemediği
    \end{itemize}

    \pause
    \item kalıcılık
    \begin{itemize}
      \item bir hareket sonlandırıldıktan sonra sistem çökse de\\
        verilerin zarar görmemesi
    \end{itemize}
  \end{itemize}
\end{frame}

\begin{frame}[fragile]
  \frametitle{Hareket İşlemleri}

  \begin{block}{başlatma}
    \begin{lstlisting}
BEGIN [ WORK | TRANSACTION ]
    \end{lstlisting}
  \end{block}

  \pause
  \begin{block}{sonlandırma}
    \begin{lstlisting}
COMMIT [ WORK | TRANSACTION ]
    \end{lstlisting}
  \end{block}

  \pause
  \begin{block}{vazgeçme}
    \begin{lstlisting}
ROLLBACK [ WORK | TRANSACTION ]
    \end{lstlisting}
  \end{block}
\end{frame}

\begin{frame}[fragile]
  \frametitle{Hareket Örneği}

  \begin{ornek}
    \begin{lstlisting}
  BEGIN TRANSACTION
  ON ERROR GOTO UNDO
  UPDATE ACCOUNTS SET BALANCE = BALANCE - 100
    WHERE (ACCOUNTID = 123)
  UPDATE ACCOUNTS SET BALANCE = BALANCE + 100
    WHERE (ACCOUNTID = 456)
  COMMIT
  ...

UNDO:
  ROLLBACK
    \end{lstlisting}
  \end{ornek}
\end{frame}

\subsection{Sistemin Düzeltilmesi}

\begin{frame}
  \frametitle{Sistemin Düzeltilmesi}

  \begin{itemize}
    \item bir hareket sürerken sistemin çöktüğünü düşünün
    \begin{itemize}
      \item bellek tamponlarındaki veriler diske yazılmamış durumda
    \end{itemize}

    \pause
    \item kalıcılık nasıl sağlanacak?
    \begin{itemize}
      \item veri, sistemde başka yerde yazılı verilerden türetilebilmeli
      \item iç düzeyde
    \end{itemize}
  \end{itemize}
\end{frame}

\begin{frame}
  \frametitle{Günlük}

  \begin{itemize}
    \item \alert{günlük} her işlemden etkilenen her çoklunun\\
      işlemden önceki ve sonraki değerlerini tutar

    \pause
    \medskip
    \item \alert{günlüğe önceden yazma kuralı}:\\
      hareket sonlanmadan önce günlük fiziksel ortama yazılmalı

    \pause
    \medskip
    \item günlük kayıtlarına erişim işlemin doğası gereği ardışıl
  \end{itemize}
\end{frame}

\begin{frame}
  \frametitle{Denetim Noktaları}

  \begin{itemize}
    \item belli aralıklarla günlükte \alert{denetim noktaları} oluşturulur

    \pause
    \medskip
    \item bellek tamponlarındaki veriler fiziksel ortama yazılır
    \item denetim noktası günlüğe not edilir
    \item o an sürmekte olan hareketler not edilir
  \end{itemize}
\end{frame}

\begin{frame}
  \frametitle{Düzeltme Listeleri}

  \begin{itemize}
    \item aksaklıktan sonra hangi hareketler geri alınacak,\\
      hangileri sonlandırılacak?
    \begin{itemize}
      \item iki liste oluştur: \emph{geri alınacaklar} (G),
        \emph{yinelenecekler} (Y)
    \end{itemize}

    \pause
    \item $t_C$: günlükte kayıtlı son denetim noktası
    \begin{itemize}
      \item $t_C$ anında etkin olan hareketleri G'ye ekle
    \end{itemize}

    \pause
    \item $t_C$'den başlayarak kayıtları ileri doğru tara
    \begin{itemize}
      \item başlayan bir hareketle karşılaşırsan G'ye ekle
      \item biten bir hareketle karşılaşırsan Y'ye geçir
    \end{itemize}
  \end{itemize}
\end{frame}

\begin{frame}
  \frametitle{Düzeltme Örneği}

  \begin{ornek}
    \begin{columns}[t]
      \column{.4\textwidth}
      \begin{center}
        \pgfuseimage{duzeltme}
      \end{center}

      \pause
      \column{.5\textwidth}
      \begin{itemize}
        \item $t_C$:\\
          $G=\{T_2$,$T_3\}$
          $Y=\emptyset$

        \pause
        \item $T_4$ başladı:\\
          $G=\{T_2,T_3,T_4\}$
          $Y=\emptyset$

        \pause
        \item $T_2$ bitti:\\
          $G=\{T_3,T_4\}$
          $Y=\{T_2\}$

        \pause
        \item $T_5$ başladı:\\
          $G=\{T_3,T_4,T_5\}$
          $Y=\{T_2\}$

        \pause
        \item $T_4$ bitti:\\
          $G=\{T_3,T_5\}$
          $Y=\{T_2,T_4\}$
      \end{itemize}
    \end{columns}
  \end{ornek}
\end{frame}

\begin{frame}
  \frametitle{Düzeltme Süreci}

  \begin{itemize}
    \item kayıtları günlük sonundan geriye doğru tara
    \begin{itemize}
      \item G'deki hareketlerin yaptıkları değişiklikleri geri al
    \end{itemize}

    \pause
    \item kayıtları ileriye doğru tara
    \begin{itemize}
      \item Y'deki hareketlerin yaptıkları değişiklikleri yinele
    \end{itemize}
  \end{itemize}
\end{frame}

\subsection{İki Aşamalı Sonlandırma}

\begin{frame}
  \frametitle{İki Aşamalı Sonlandırma}

  \begin{itemize}
    \item farklı kaynak yöneticileri var
    \begin{itemize}
      \item geri alma - sonlandırma sistemleri ayrı
    \end{itemize}

    \pause
    \item etkilenecek veriler farklı kaynak yöneticilerinde
    \begin{itemize}
      \item ya hepsinde birden sonlandırılacak\\
        ya da hepsinde birden geri alınacak
    \end{itemize}

    \pause
    \item \alert{eşgüdüm sağlayıcı}
  \end{itemize}
\end{frame}

\begin{frame}
  \frametitle{Protokol}

  \begin{itemize}
    \item eşgüdüm sağlayıcı, bütün katılımcılara hareketle ilgili\\
      bütün verilerin kayıtlarını kalıcı ortama yazmalarını söyler

    \pause
    \item eşgüdüm sağlayıcı, bütün katılımcılardan\\
      hareketi başlatmalarını ve sonucu kendisine bildirmelerini ister
    \begin{itemize}
      \item bütün katılımcılardan "başarılı" yanıtı alırsa\\
        hareketin sonlandırılmasına karar verir
      \item bir tane bile "başarısız" yanıtı gelirse\\
        hareketin geri alınmasına karar verir
    \end{itemize}

    \pause
    \item eşgüdüm sağlayıcı, bütün katılımcılara kararı bildirir
  \end{itemize}
\end{frame}

\subsection*{Kaynaklar}

\begin{frame}
  \frametitle{Kaynaklar}

  \begin{block}{Okunacak: Date}
    \begin{itemize}
      \item Chapter 15: \alert{Recovery}
    \end{itemize}
  \end{block}
\end{frame}

\section{Eşzamanlı Çalışma}

\subsection{Giriş}

\begin{frame}
  \frametitle{Eşzamanlı Çalışma}

  \begin{itemize}
    \item eşzamanlı çalışan hareketler nedeniyle çıkabilecek sorunlar:

    \bigskip
    \item yitirilen güncelleme
    \item kesinleşmemiş veriye bağımlılık
    \item tutarsız çözümleme
  \end{itemize}
\end{frame}

\begin{frame}[fragile]
  \frametitle{Yitirilen Güncelleme}

  \begin{ornek}
    \begin{table}
      \begin{tabular}{ll}
Hareket A  & Hareket B \\\hline
...        & ...       \\\pause
RETRIEVE p & ...       \\\pause
...        & ...       \\
...        & RETRIEVE p\\\pause
...        & ...       \\
UPDATE p   & ...       \\\pause
...        & ...       \\
...        & UPDATE p  \\
...        & ...
      \end{tabular}
    \end{table}
  \end{ornek}
\end{frame}

\begin{frame}[fragile]
  \frametitle{Kesinleşmemiş Veriye Bağımlılık}

  \begin{ornek}
    \begin{table}
      \begin{tabular}{ll}
Hareket A  & Hareket B\\\hline
...        & ...      \\\pause
...        & UPDATE p \\\pause
...        & ...      \\
RETRIEVE p & ...      \\\pause
...        & ...      \\
...        & ROLLBACK \\
...        &
      \end{tabular}
    \end{table}
  \end{ornek}
\end{frame}

\begin{frame}[fragile]
  \frametitle{Tutarsız Çözümleme}

  \begin{ornek}[hesap toplamı: acc1=40, acc2=50, acc3=30]
    \begin{table}
      \begin{tabular}{ll}
Hareket A             & Hareket B                        \\\hline
...                   & ...                              \\\pause
RETRIEVE acc1 ($40$)  & ...                              \\\pause
RETRIEVE acc2 ($90$)  & ...                              \\\pause
...                   & ...                              \\
...                   & UPDATE acc3 ($30 \rightarrow 20$)\\
...                   & UPDATE acc1 ($40 \rightarrow 50$)\\
...                   & COMMIT                           \\\pause
...                   & ...                              \\
RETRIEVE acc3 ($110$) &                                  \\
...                   &
      \end{tabular}
    \end{table}
  \end{ornek}
\end{frame}

\begin{frame}
  \frametitle{Çakışmalar}

  \begin{itemize}
    \item A okuyor, B okuyor
    \begin{itemize}
      \item sorun yok
    \end{itemize}

    \pause
    \item A okuyor, B yazıyor
    \begin{itemize}
      \item yinelenemez okuma (tutarsız çözümleme)
    \end{itemize}

    \pause
    \item A yazıyor, B okuyor
    \begin{itemize}
      \item kirli okuma (kesinleşmemiş veriye bağımlılık)
    \end{itemize}

    \pause
    \item A yazıyor, B yazıyor
    \begin{itemize}
      \item kirli yazma (yitirilen güncelleme)
    \end{itemize}
  \end{itemize}
\end{frame}

\subsection{Kilitleme}

\begin{frame}
  \frametitle{Kilitleme}

  \begin{itemize}
    \item hareketler üzerinde işlem yapacakları çokluları kilitlesinler
    \begin{itemize}
      \item okuma kilidi (S)
      \item yazma kilidi (X)
    \end{itemize}

    \item işleri bitince kilitleri bıraksınlar
  \end{itemize}
\end{frame}

\begin{frame}
  \frametitle{Kilit İstekleri}

  \begin{block}{kilit tipi uyumluluk matrisi}
    \begin{table}
      \begin{tabular}{|c||c|c|c|}\hline
  & X & S & -\\\hline\hline
X & H & H & E\\\hline
S & H & E & E\\\hline
      \end{tabular}
    \end{table}
  \end{block}

  \begin{itemize}
    \item yazma kilidi varsa başka hareketlerin her türlü isteği reddedilir

    \pause
    \item okuma kilidi varsa:
    \begin{itemize}
      \item başka hareketlerin yazma kilidi istekleri reddedilir
      \item başka hareketlerin okuma kilidi istekleri kabul edilir
    \end{itemize}
  \end{itemize}
\end{frame}

\begin{frame}
  \frametitle{Kilitleme Protokolü}

  \begin{itemize}
    \item hareket, yapmak istediği işleme göre kilit isteğinde bulunur
    \begin{itemize}
      \item okuma kilidi varsa yazma kilidine çevrilmesi
    \end{itemize}

    \pause
    \item istek yerine getirilemiyorsa beklemeye başlar
    \begin{itemize}
      \item diğer hareket kilidi bırakınca devam eder
      \item \alert{sonsuz bekleme}
    \end{itemize}
  \end{itemize}
\end{frame}

\begin{frame}[fragile]
  \frametitle{Yitirilen Güncelleme}

  \begin{ornek}
    \begin{table}
      \begin{tabular}{ll}
Hareket A       & Hareket B      \\\hline
...             & ...            \\\pause
RETRIEVE p (S+) & ...            \\\pause
...             & ...            \\
...             & RETRIEVE p (S+)\\\pause
...             & ...            \\
UPDATE p (X-)   & ...            \\
bekle           & ...            \\\pause
bekle           & UPDATE p (X-)  \\
bekle           & bekle
      \end{tabular}
    \end{table}
  \end{ornek}
\end{frame}

\begin{frame}[fragile]
  \frametitle{Kesinleşmemiş Veriye Bağımlılık}

  \begin{ornek}
    \begin{table}
      \begin{tabular}{ll}
Hareket A       & Hareket B    \\\hline
...             & ...          \\\pause
...             & UPDATE p (X+)\\\pause
...             & ...          \\
RETRIEVE p (S-) & ...          \\
bekle           & ...          \\\pause
bekle           & ROLLBACK     \\
RETRIEVE p (S+) &              \\
...             &
      \end{tabular}
    \end{table}
  \end{ornek}
\end{frame}

\begin{frame}[fragile]
  \frametitle{Tutarsız Çözümleme}

  \begin{ornek}[hesap toplamı: acc1=40, acc2=50, acc3=30]
    \begin{table}
      \begin{tabular}{ll}
Hareket A            & Hareket B       \\\hline
...                  & ...             \\\pause
RETRIEVE acc1 (S+)   & ...             \\\pause
RETRIEVE acc2 (S+)   & ...             \\\pause
...                  & ...             \\
...                  & UPDATE acc3 (X+)\\\pause
...                  & UPDATE acc1 (X-)\\
...                  & bekle           \\\pause
RETRIEVE acc3 (S-)   & bekle           \\
bekle                & bekle
      \end{tabular}
    \end{table}
  \end{ornek}
\end{frame}

\begin{frame}
  \frametitle{Ölümcül Kilitlenme}

  \begin{tanim}
    \alert{ölümcül kilitlenme}:\\
      hareketlerin birbirlerinin kilitleri bırakmalarını beklemesi
  \end{tanim}

  \pause
  \begin{itemize}
    \item neredeyse her zaman iki hareket arasında
    \item yapılabilecekler:
    \begin{itemize}
      \item farketmek ve çözmek
      \item önlemek
    \end{itemize}
  \end{itemize}
\end{frame}

\begin{frame}
  \frametitle{Ölümcül Kilitlenmenin Çözülmesi}

  \begin{columns}[t]
    \column{.5\textwidth}
    \begin{ornek}
      \begin{center}
        \pgfuseimage{bekleme}
      \end{center}
    \end{ornek}

    \column{.5\textwidth}
    \begin{itemize}
      \item bekleme grafı

      \pause
      \item bir \alert{kurban} seç ve öldür
    \end{itemize}
  \end{columns}
\end{frame}

\begin{frame}
  \frametitle{Ölümcül Kilitlenmenin Önlenmesi}

  \begin{itemize}
    \item her hareketin başlama zamanı mührü var

    \pause
    \item A hareketinin kilit isteği\\
      B hareketinin tuttuğu bir kilitle çelişiyorsa:
    \begin{itemize}
      \item \alert{bekle-öl}: A, B'den yaşlıysa bekler, gençse ölür\\
        A geri alınıp yeniden başlatılır

      \item \alert{yarala-bekle}: A, B'den gençse bekler, yaşlıysa B'yi
        yaralar\\
        B geri alınıp yeniden başlatılır
    \end{itemize}

    \pause
    \item yeniden başlatılan hareketin zaman mührü değiştirilmez
  \end{itemize}
\end{frame}

\begin{frame}[fragile]
  \frametitle{Kilit Komutları}

  \begin{block}{okuma kilidi}
    \begin{lstlisting}
SELECT query FOR SHARE
    \end{lstlisting}
  \end{block}

  \pause
  \begin{block}{yazma kilidi}
    \begin{lstlisting}
SELECT query FOR UPDATE
    \end{lstlisting}
  \end{block}
\end{frame}

\subsection{Yalıtım Düzeyleri}

\begin{frame}
  \frametitle{Yalıtım Düzeyleri}

  \begin{itemize}
    \item yalıtım azaltılırsa eşzamanlılık artırılabilir
    \item değişik yalıtım düzeyleri:

    \bigskip
    \item serileştirilebilir
    \item yinelenebilir okuma
    \item sonlandırılanları okuyabilme
    \item sonlandırılmayanları okuyabilme
  \end{itemize}
\end{frame}

\begin{frame}
  \frametitle{Serileştirilebilirlik}

  \begin{itemize}
    \item \emph{seri çalıştırma}:\\
      hareketlerin biri bitmeden diğeri başlamıyor

    \pause
    \item \alert{serileştirilebilir}: eşzamanlı çalışmanın sonucu\\
      her zaman seri çalıştırmalardan birinin sonucu ile aynı
  \end{itemize}

  \pause
  \begin{ornek}
    \begin{itemize}
      \item $x = 10$
      \item A hareketi: $x = x + 1$
      \item B hareketi: $x = 2 * x$

      \pause
      \medskip
      \item önce A, sonra B: $x = 22$
      \item önce B, sonra A: $x = 21$
    \end{itemize}
  \end{ornek}
\end{frame}

\begin{frame}
  \frametitle{İki Aşamalı Kilitleme}

  \begin{itemize}
    \item \alert{iki aşamalı kilitleme}:\\
      herhangi bir kilit bırakıldıktan sonra\\
      yeni kilit isteğinde bulunulmaz
    \begin{itemize}
      \item genişleme aşaması: alınan kilit sayısı artıyor
      \item daralma aşaması: alınan kilit sayısı azalıyor
    \end{itemize}

    \pause
    \item \alert{iki aşamalı sıkı kilitleme}:\\
      bütün kilitler hareketin sonunda bırakılır

    \pause
    \medskip
    \item \emph{Bütün hareketler iki aşamalı kilitleme protokolüne uyarsa\\
      bütün eşzamanlı çalıştırmalar serileştirilebilir.}
  \end{itemize}
\end{frame}

\begin{frame}[fragile]
  \frametitle{Sonlandırılanları Okuyabilme}

  \begin{itemize}
    \item yalnızca yazma kilitleri hareket sonuna kadar tutulur
  \end{itemize}

  \begin{ornek}
    \begin{table}
      \begin{tabular}{ll}
Hareket A       & Hareket B    \\\hline
...             & ...          \\\pause
RETRIEVE p (S+) & ...          \\\pause
...             & ...          \\
kilidi bırak    & ...          \\\pause
...             & ...          \\
...             & UPDATE p (X+)\\
...             & COMMIT       \\\pause
RETRIEVE p (S+) &
      \end{tabular}
    \end{table}
  \end{ornek}
\end{frame}

\begin{frame}
  \frametitle{Hayaletler}

  \begin{tanim}
    \alert{hayalet}: sorgu yeniden çalıştırıldığında yeni çoklular ortaya
      çıkıyor
  \end{tanim}

  \pause
  \begin{ornek}
    \begin{itemize}
      \item A hareketi bir müşterinin hesaplarının ortalamasını hesaplıyor:\\
        $\frac{100 + 100 + 100}{3}=100$

      \pause
      \item B hareketi aynı müşteriye $200$ birimlik yeni bir hesap yaratıyor
      \item A hareketi hesabı yeniden yapıyor:\\
        $\frac{100 + 100 + 100 + 200}{4}=125$
    \end{itemize}
  \end{ornek}
\end{frame}

\begin{frame}[fragile]
  \frametitle{Yalıtım Düzeyi Belirleme}

  \begin{block}{Komut}
    \begin{lstlisting}
SET TRANSACTION ISOLATION LEVEL
  [ SERIALIZABLE | REPEATABLE READ |
    READ COMMITTED | READ UNCOMMITTED ]
    \end{lstlisting}
  \end{block}
\end{frame}

\begin{frame}[fragile]
  \frametitle{Yalıtım Düzeyi Sorunları}

  \begin{table}
    \begin{tabular}{|l||c|c|c|}\hline
yalıtım düzeyi   & kirli & yinelemeyen & hayalet\\
                 & okuma & okuma       &        \\\hline\hline
READ UNCOMMITTED & E     & E           & E      \\\hline
READ COMMITTED   & H     & E           & E      \\\hline
REPEATABLE READ  & H     & H           & E      \\\hline
SERIALIZABLE     & H     & H           & H      \\\hline
    \end{tabular}
  \end{table}
\end{frame}

\subsection{Niyet Kilitleri}

\begin{frame}
  \frametitle{Kilitleme Birimi}

  \begin{itemize}
    \item kilitleme çoklu değil bağıntı değişkeni biriminde yapılabilir
    \begin{itemize}
      \item hatta veri tabanı biriminde
    \end{itemize}

    \item birim genişledikçe eşzamanlılık azalır

    \pause
    \item çoklular üzerinde alınmış kilitlerin bulunması zor\\
      $\rightarrow$ önce bağıntı değişkeni düzeyinde \alert{niyet kilitleri}
      alınsın
  \end{itemize}
\end{frame}

\begin{frame}
  \frametitle{Niyet Kilitleri}

  \begin{itemize}
    \item Parçayı Okuma (IS):\\
      hareket bazı çokluları okumaya niyetleniyor

    \pause
    \item Parçaya Yazma (IX):\\
      IS + hareket bazı çoklulara yazmaya niyetleniyor

    \pause
    \item Bütünü Okuma (S):\\
      bağıntıda eşzamanlı okuyucular olabilir ama yazıcılar olmamalı

    \pause
    \item Bütünü Okuma + Parçaya Yazma (SIX):\\
      S + IX

    \pause
    \item Bütüne Yazma (X):\\
      bağıntıda hiçbir eşzamanlı çalışma olmamalı
  \end{itemize}
\end{frame}

\begin{frame}
  \frametitle{Kilit İstekleri}

  \begin{block}{kilit uyumluluk matrisi}
    \begin{table}
      \begin{tabular}{|c||c|c|c|c|c|c|}\hline
    & X & SIX & IX & S & IS & -\\\hline\hline
  X & H &  H  & H  & H & H  & E\\\hline
SIX & H &  H  & H  & H & E  & E\\\hline
 IX & H &  H  & E  & H & E  & E\\\hline
  S & H &  H  & H  & E & E  & E\\\hline
 IS & H &  E  & E  & E & E  & E\\\hline
      \end{tabular}
    \end{table}
  \end{block}
\end{frame}

\begin{frame}
  \frametitle{Kilit Öncelikleri}

  \begin{columns}[t]
    \column{.4\textwidth}
    \begin{center}
      \pgfuseimage{oncelik}
    \end{center}

    \pause
    \column{.6\textwidth}
    \begin{itemize}
      \item çoklu üzerinde okuma kilidi için\\
	bağıntı üzerinde en az IS
      \item çoklu üzerinde yazma kilidi için\\
	bağıntı üzerinde en az IX
    \end{itemize}
  \end{columns}
\end{frame}

\begin{frame}[fragile]
  \frametitle{Kilitleme Komutları}

  \begin{block}{Komut}
    \begin{lstlisting}
LOCK [ TABLE ] table_name
     [ IN lock_mode MODE ]
    \end{lstlisting}

    \pause
    \begin{itemize}
      \item kilit kipleri:
      \begin{itemize}
        \item \lstinline!ACCESS SHARE!
        \item \lstinline!ROW SHARE!
        \item \lstinline!ROW EXCLUSIVE!
        \item \lstinline!SHARE UPDATE EXCLUSIVE!
        \item \lstinline!SHARE!
        \item \lstinline!SHARE ROW EXCLUSIVE!
        \item \lstinline!EXCLUSIVE!
        \item \lstinline!ACCESS EXCLUSIVE!
      \end{itemize}
    \end{itemize}

  \end{block}
\end{frame}

\subsection*{Kaynaklar}

\begin{frame}
  \frametitle{Kaynaklar}

  \begin{block}{Okunacak: Date}
    \begin{itemize}
      \item Chapter 16: \alert{Concurrency}
    \end{itemize}
  \end{block}
\end{frame}

\end{document}
