\documentclass[dvipsnames]{beamer}

\usepackage{ae}
\usepackage[scaled=0.88]{beramono}
\usepackage[T1]{fontenc}
\usepackage[utf8]{inputenc}
\setbeamertemplate{navigation symbols}{}
\setbeamersize{text margin left=2em, text margin right=2em}

\usepackage{listings}
\lstset{basicstyle=\ttfamily, keywordstyle=\color{blue},
        showstringspaces=false, frame=ltrb}
\lstset{language=C}

\mode<presentation>
{
  \usetheme{default}
  \setbeamercovered{transparent}
}

\title{Systems Programming}
\subtitle{Devices}

\author{H. Turgut Uyar \and Şima Uyar}
\date{2001-2014}

\AtBeginSubsection[]{
  \begin{frame}<beamer>
    \frametitle{Topics}
    \tableofcontents[currentsection,currentsubsection]
  \end{frame}
}

\pgfdeclareimage[height=7.3cm]{pport}{pport}
\pgfdeclareimage[height=7.3cm]{cycle}{cycle}
\pgfdeclareimage[width=6.5cm]{simplescull1}{simplescull1}
\pgfdeclareimage[width=5.5cm]{simplescull2}{simplescull2}
\pgfdeclareimage[width=5cm]{simplescull3}{simplescull1}
\pgfdeclareimage[width=5cm]{simplescull}{simplescull}

\begin{document}

\begin{frame}
  \titlepage
\end{frame}

\begin{frame}
  \frametitle{Topics}
  \tableofcontents
\end{frame}

\section{I/O Subsystem}

\subsection{Introduction}

\begin{frame}
  \frametitle{I/O Devices}

  \begin{itemize}
    \item O/S controls all I/O devices

    \medskip
    \item issues commands to devices
    \item catches interrupts
    \item handles errors

    \medskip
    \item provides interface
  \end{itemize}
\end{frame}

\begin{frame}
  \frametitle{Device Controllers}

  \begin{itemize}
    \item devices consist of:
    \begin{itemize}
      \item mechanical components
      \item electronic components: \emph{device controller}
    \end{itemize}

    \medskip
    \item O/S deals with controller
    \begin{itemize}
      \item connected through a standard interface
      \item SCSI, USB, Firewire, ...
    \end{itemize}
  \end{itemize}
\end{frame}

\begin{frame}
  \frametitle{Controller Registers}

  \begin{itemize}
    \item CPU communicates with the controller through registers

    \medskip
    \item data register: for sending/receiving data
    \item control register: for sending commands to device
    \item status register: for getting/setting the state of device
  \end{itemize}
\end{frame}

\begin{frame}
  \frametitle{I/O Architecture}

  \begin{itemize}
    \item \alert{ports}: special address space for I/O
    \begin{itemize}
      \item separate lines for I/O ports
      \item special instructions for I/O
    \end{itemize}
  \end{itemize}

  \begin{itemize}
    \item \alert{memory-mapped}: registers part of regular address space
    \begin{itemize}
      \item directly-mapped: part of address space reserved for I/O
      \item software-mapped: I/O space part of virtual memory
    \end{itemize}
  \end{itemize}
\end{frame}

\begin{frame}
  \frametitle{PC Parallel Interface}

  \begin{itemize}
    \item parallel interface base addresses on a PC: 0x378, 0x278

    \medskip
    \item ports:
    \begin{itemize}
      \item +0: bidirectional data register
      \item +1: status register (read-only)\\
        online, out-of-paper, busy
      \item +2: control register (write-only)\\
        enable/disable interrupts
    \end{itemize}
  \end{itemize}
\end{frame}

\begin{frame}
  \frametitle{Parallel Interface}

  \begin{center}
    \pgfuseimage{pport}
  \end{center}
\end{frame}

\subsection{Device Types}

\begin{frame}
  \frametitle{Device Types}

  \begin{itemize}
    \item character devices
    \item block devices
    \item network interfaces
    \item clocks and timers
  \end{itemize}
\end{frame}

\begin{frame}
  \frametitle{Character Devices}

  \begin{itemize}
    \item a character device acts like a stream of characters

    \medskip
    \item arbitrary-sized data transfer
    \item not addressable: no seek operation
  \end{itemize}

  \begin{exampleblock}{examples}
    \begin{itemize}
      \item console, mouse
      \item sound card
      \item serial port, parallel port
    \end{itemize}
  \end{exampleblock}
\end{frame}

\begin{frame}
  \frametitle{Block Devices}

  \begin{itemize}
    \item a block device can host a filesystem

    \medskip
    \item data transfer in fixed-size blocks
    \item each block has its own address
    \item read/write each block independently
  \end{itemize}

  \begin{exampleblock}{example}
    \begin{itemize}
      \item disks
    \end{itemize}
  \end{exampleblock}
\end{frame}

\begin{frame}
  \frametitle{Device Type}

  \begin{itemize}
    \item the device type is more the characteristic of the driver\\
      rather than the device itself
  \end{itemize}

  \begin{exampleblock}{example: disk}
    \begin{itemize}
      \item usually a block device
      \item it can also be used as a character device: tar
    \end{itemize}
  \end{exampleblock}
\end{frame}

\subsection{I/O Software}

\begin{frame}
  \frametitle{I/O Software}

  \begin{itemize}
    \item blocking vs interrupt-driven
    \begin{itemize}
      \item better for CPU to work interrupt-driven fashion
      \item better for user-space programs to work in blocking fashion
      \item easier to develop programs that work in blocking fashion
      \item O/S makes interrupt-driven operations look blocking
    \end{itemize}

    \pause
    \medskip
    \item standardized interface
    \item uniform naming
  \end{itemize}
\end{frame}

\begin{frame}
  \frametitle{Unix Device Naming}

  \begin{itemize}
    \item in Unix, every device has a \alert{device node}
    \item under the \texttt{/dev} folder

    \pause
    \medskip
    \item \texttt{/dev/sda}: first SCSI disk
    \item \texttt{/dev/sdb}: second SCSI disk
    \item \texttt{/dev/sdb1}: first partition of the second SCSI disk
    \item \texttt{/dev/sdb2}: second partition of the second SCSI disk

    \pause
    \medskip
    \item \texttt{/dev/parport0}: first parallel port
  \end{itemize}
\end{frame}

\begin{frame}
  \frametitle{Unix Device Naming}

  \begin{itemize}
    \item device nodes have major and minor numbers
    \item major number identifies the driver
    \item minor number identifies the physical device

    \pause
    \medskip
    \item all \texttt{/dev/sd*} devices have the same major number
    \item they all have different minor numbers

    \pause
    \medskip
    \item (recently) major number alone doesn't identify driver
    \item major number + region of minor numbers
  \end{itemize}
\end{frame}

\begin{frame}
  \frametitle{I/O Services}

  \begin{itemize}
    \item copy semantics: transfer the snapshot of data\\
      at the time of the I/O request

    \pause
    \item scheduling: issue order may not be the best execution order

    \pause
    \item buffering: adapt between different data transfer sizes

    \pause
    \item caching

    \pause
    \item spooling: deal with dedicated devices (e.g. printers)
    \begin{itemize}
      \item a daemon for controlling the device
      \item a spooling directory
    \end{itemize}

    \pause
    \item error handling
  \end{itemize}
\end{frame}

\begin{frame}
  \frametitle{Software Layers}

  \begin{itemize}
    \item top-down:

    \medskip
    \item user-space applications
    \item device-independent software
    \item device drivers
    \item interrupt handlers
  \end{itemize}
\end{frame}

\begin{frame}
  \frametitle{Device-Independent Software}

  \begin{itemize}
    \item functions common to all devices
    \item uniform interface to user-level software
    \item device naming
    \item device protection
    \item provide device-independent block sizes
    \item buffering
    \item allocating and releasing dedicated devices
    \item error reporting
  \end{itemize}
\end{frame}

\begin{frame}
  \frametitle{Device Drivers}

  \begin{itemize}
    \item device-dependent code
    \item a driver for each device type

    \medskip
    \item accept request from device-independent software
    \item decide on sequence of controller operations
  \end{itemize}
\end{frame}

\begin{frame}
  \frametitle{Interrupt Handlers}

  \begin{itemize}
    \item interrupts hidden from rest of system
    \begin{itemize}
      \item requesting process is blocked until I/O is completed
    \end{itemize}

    \item when I/O is completed, interrupt occurs
    \begin{itemize}
      \item process is made to unblock
    \end{itemize}
  \end{itemize}
\end{frame}

\begin{frame}
  \frametitle{I/O Life Cycle}

  \begin{center}
    \pgfuseimage{cycle}
  \end{center}
\end{frame}

\subsection{Accessing Devices}

\begin{frame}
  \frametitle{Accessing Devices}

  \begin{itemize}
    \item directly: using ports or memory

    \medskip
    \item through device drivers: using the device driver interface
  \end{itemize}
\end{frame}

\begin{frame}[fragile]
  \frametitle{Direct Access}

  \begin{itemize}
    \item input: \lstinline|inb|, \lstinline|inw|, \lstinline|inl|
    \item output: \lstinline|outb|, \lstinline|outw|, \lstinline|outl|

    \pause
    \medskip
    \item get permission from O/S: \lstinline{ioperm} system call
    \begin{lstlisting}
int ioperm(unsigned long from,
           unsigned long num,
           int turn_on);
    \end{lstlisting}
  \end{itemize}
\end{frame}

\begin{frame}[fragile]
  \frametitle{Direct Access Example}

  \begin{exampleblock}{output to parallel interface}
    \begin{lstlisting}
ioperm(0x378, 1, 255);
outb(0xff, 0x378);
    \end{lstlisting}
  \end{exampleblock}
\end{frame}

\subsection*{Reading Material}

\begin{frame}
  \frametitle{Reading Material}

  \begin{itemize}
    \item Silberschatz, 8/e
    \begin{itemize}
      \item Chapter 13: \alert{I/O Systems}
    \end{itemize}
  \end{itemize}
\end{frame}

\section{Device Drivers}

\subsection{Interface}

\begin{frame}
  \frametitle{Unix Device Driver Interface}

  \begin{itemize}
    \item in Unix, the device driver interface\\
      is similar to the file interface

    \medskip
    \item \lstinline|open|, \lstinline|close|
    \item \lstinline|read|, \lstinline|write|
  \end{itemize}
\end{frame}

\begin{frame}
  \frametitle{Device Specific Operations}

  \begin{itemize}
    \item some operations are neither \lstinline|read| nor \lstinline|write|
    \item \lstinline|ioctl|: issue command specific to device
  \end{itemize}

  \begin{exampleblock}{device-specific operation examples}
    \begin{itemize}
      \item eject CDROM
      \item make the speaker beep
      \item set communication parameters for modem
    \end{itemize}
  \end{exampleblock}
\end{frame}

\begin{frame}[fragile]
  \frametitle{System Calls}

  \begin{itemize}
    \item \lstinline|open|:
    \begin{lstlisting}
int open(const char *pathname,
         int flags,
         mode_t mode);
    \end{lstlisting}

    \item flags:
    \begin{itemize}
      \item \lstinline|O_RDONLY O_WRONLY O_RDWR|
      \item \lstinline|O_CREAT O_APPEND|
    \end{itemize}

    \item mode: permissions

    \item returns: file descriptor
  \end{itemize}
\end{frame}

\begin{frame}[fragile]
  \frametitle{System Calls}

  \begin{itemize}
    \item \lstinline|close|:
    \begin{lstlisting}
int close(int fd);
    \end{lstlisting}
    \item returns: success / failure status
  \end{itemize}
\end{frame}

\begin{frame}[fragile]
  \frametitle{System Calls}

  \begin{itemize}
    \item \lstinline|read|:
    \begin{lstlisting}
ssize_t read(int fd,
             void *buf,
             size_t count);
    \end{lstlisting}
    \item returns: number of bytes read $(x)$
    \begin{itemize}
      \item $x=count$: successful completion
      \item $x=0$: end-of-file
      \item $x<0$: error
      \item $0<x<count$: partial transfer, retry remaining part
    \end{itemize}
  \end{itemize}
\end{frame}

\begin{frame}[fragile]
  \frametitle{System Calls}

  \begin{itemize}
    \item \lstinline|write|:
    \begin{lstlisting}
ssize_t write(int fd,
              const void *buf,
              size_t count);
    \end{lstlisting}
    \item returns: number of bytes written $(x)$
    \begin{itemize}
      \item $x=count$: successful completion
      \item $x=0$: end-of-file
      \item $x<0$: error
      \item $0<x<count$: partial transfer, retry remaining part
    \end{itemize}
  \end{itemize}
\end{frame}

\begin{frame}[fragile]
  \frametitle{System Calls}

  \begin{itemize}
    \item \lstinline|ioctl|:
    \begin{lstlisting}
int ioctl(int fd,
          int request,
          ...);
    \end{lstlisting}
    \item parameter and return values depend on request
  \end{itemize}
\end{frame}

\begin{frame}[fragile]
  \frametitle{Device Access Example}

  \begin{exampleblock}{output to parallel port}
    \begin{lstlisting}
fd = open("/dev/parport0", O_WRONLY);
if (fd == -1)
{
    perror("cannot access device");
    exit(EXIT_FAILURE);
}
write(fd, buffer, len);
close(fd);
    \end{lstlisting}
  \end{exampleblock}
\end{frame}
%
% \begin{frame}[fragile]
%   \frametitle{Device Specific Command Example}
%
%   \begin{example}[Ejecting the CDROM]
%     \begin{lstlisting}
% int fd, status;
%
% fd = open("/dev/cdrom", O_RDONLY);
%
% status = ioctl(fd, CDROMEJECT);
% if (status == -1) {
%     perror("cannot eject CD-ROM");
%     exit(EXIT_FAILURE);
% }
%
% close(fd);
%     \end{lstlisting}
%   \end{example}
% \end{frame}

\begin{frame}[fragile]
  \frametitle{Device Specific Command Example}

  \begin{exampleblock}{make the speaker beep}
    \begin{lstlisting}
fd = open("/dev/console", O_RDWR);
status = ioctl(fd, KDMKTONE, 0x100011AA);
if (status == -1)
{
    perror("cannot generate beep");
    exit(EXIT_FAILURE);
}
close(fd);
    \end{lstlisting}
  \end{exampleblock}
\end{frame}

\subsection{Implementation}

\begin{frame}
  \frametitle{Implementing Device Drivers}

  \begin{itemize}
    \item implement system calls for device
    \item convert system calls to device specific I/O instructions
  \end{itemize}
\end{frame}

\begin{frame}
  \frametitle{Device Driver Example}

  \begin{exampleblock}{simplified scull}
    \begin{itemize}
      \item use memory as device
      \begin{itemize}
        \item \texttt{/dev/scull0}
        \item \texttt{/dev/scull1}
      \end{itemize}

      \item each device can hold data up to a limit
      \begin{itemize}
        \item data persists during module's lifetime
      \end{itemize}
    \end{itemize}
  \end{exampleblock}
\end{frame}

\begin{frame}
  \frametitle{Memory Layout}

  \begin{columns}
    \column{.55\textwidth}
    \begin{center}
      \pgfuseimage{simplescull1}
    \end{center}

    \column{.4\textwidth}
    \begin{itemize}
      \item each device has\\
        a quantum set
      \item each quantum contains\\
        the actual data
      \item memory is allocated\\
        as data is written
    \end{itemize}
  \end{columns}
\end{frame}

\begin{frame}[fragile]
  \frametitle{Global Definitions}

  \begin{exampleblock}{\texttt{scull.h}}
    \begin{lstlisting}
#define SCULL_MAJOR 0
#define SCULL_NR_DEVS 4
#define SCULL_QUANTUM 4000
#define SCULL_QSET 1000
    \end{lstlisting}
  \end{exampleblock}
\end{frame}

\begin{frame}[fragile]
  \frametitle{Module Parameters}

  \begin{exampleblock}{}
    \begin{lstlisting}
int scull_major = SCULL_MAJOR;
int scull_minor = 0;
int scull_nr_devs = SCULL_NR_DEVS;
int scull_quantum = SCULL_QUANTUM;
int scull_qset = SCULL_QSET;

module_param(scull_major, int, S_IRUGO);
module_param(scull_minor, int, S_IRUGO);
module_param(scull_nr_devs, int, S_IRUGO);
module_param(scull_quantum, int, S_IRUGO);
module_param(scull_qset, int, S_IRUGO);
    \end{lstlisting}
  \end{exampleblock}
\end{frame}

\begin{frame}[fragile]
  \frametitle{Data Structures}

  \begin{center}
    \pgfuseimage{simplescull2}
  \end{center}

  \begin{lstlisting}
struct scull_dev {
    char **data;
    int quantum;
    int qset;
    unsigned long size;
    struct semaphore sem;
    struct cdev cdev;
};

struct scull_dev *scull_devices;
  \end{lstlisting}
\end{frame}

\begin{frame}
  \frametitle{Module Initialization}

  \begin{itemize}
    \item allocate I/O region
    \begin{itemize}
      \item base address
      \item number of ports
    \end{itemize}

    \medskip
    \item register driver with the kernel
    \begin{itemize}
      \item major and minor numbers
      \item capabilities: file operations
    \end{itemize}
  \end{itemize}
\end{frame}

\begin{frame}[fragile]
  \frametitle{Module Initialization}

  \begin{exampleblock}{driver registration: major and minor numbers}
    \begin{lstlisting}
if (scull_major)
{
    devno = MKDEV(scull_major, scull_minor);
    result = register_chrdev_region(devno,
                scull_nr_devs, "scull");
}
else    /* dynamic */
{
    result = alloc_chrdev_region(&devno,
        scull_minor, scull_nr_devs, "scull");
    scull_major = MAJOR(devno);
}
    \end{lstlisting}
  \end{exampleblock}
\end{frame}

\begin{frame}[fragile]
  \frametitle{Module Initialization}

  \begin{exampleblock}{data structure allocation}
    \begin{lstlisting}
scull_devices = kmalloc(scull_nr_devs *
       sizeof(struct scull_dev), GFP_KERNEL);
if (!scull_devices)
{
    result = -ENOMEM;
    goto fail;
}
memset(scull_devices, 0,
       scull_nr_devs * sizeof(struct scull_dev));
    \end{lstlisting}
  \end{exampleblock}
\end{frame}

\begin{frame}[fragile]
  \frametitle{File Operations}

  \begin{itemize}
    \item map system calls to functions:\\
      \lstinline|struct file_operations|
  \end{itemize}

  \begin{exampleblock}{}
    \begin{lstlisting}
struct file_operations scull_fops = {
    .open    = scull_open,
    .release = scull_release,
    .read    = scull_read,
    .write   = scull_write,
    .llseek  = scull_llseek,
    .ioctl   = scull_ioctl,
};
    \end{lstlisting}
  \end{exampleblock}
\end{frame}

\begin{frame}[fragile]
  \frametitle{Module Initialization}

  \begin{exampleblock}{driver activation}
    \begin{lstlisting}
for (i = 0; i < scull_nr_devs; i++)
{
    dev = &scull_devices[i];
    dev->quantum = scull_quantum;
    dev->qset = scull_qset;
    init_MUTEX(&dev->sem);

    devno = MKDEV(scull_major, scull_minor + i);
    cdev_init(&dev->cdev, &scull_fops);
    dev->cdev.owner = THIS_MODULE;
    dev->cdev.ops = &scull_fops;
    cdev_add(&dev->cdev, devno, 1);
}
    \end{lstlisting}
  \end{exampleblock}
\end{frame}

\begin{frame}[fragile]
  \frametitle{Module Cleanup}

  \begin{exampleblock}{}
    \begin{lstlisting}
dev_t devno = MKDEV(scull_major, scull_minor);

if (scull_devices)
{
    for (i = 0; i < scull_nr_devs; i++)
    {
        scull_trim(scull_devices + i);
        cdev_del(&scull_devices[i].cdev);
    }
    kfree(scull_devices);
}

unregister_chrdev_region(devno, scull_nr_devs);
    \end{lstlisting}
  \end{exampleblock}
\end{frame}

\begin{frame}[fragile]
  \frametitle{Module Cleanup}

  \begin{columns}
    \column{.43\textwidth}
    \pgfuseimage{simplescull3}

    \column{.57\textwidth}
    \begin{exampleblock}{data structure deallocation}
      \begin{lstlisting}
if (dev->data)
{
    for (i = 0; i < dev->qset;
         i++)
    {
        if (dev->data[i])
            kfree(dev->data[i]);
    }
    kfree(dev->data);
}
dev->data = NULL;
dev->quantum = scull_quantum;
dev->qset = scull_qset;
dev->size = 0;
      \end{lstlisting}
    \end{exampleblock}
  \end{columns}
\end{frame}

\begin{frame}[fragile]
  \frametitle{Kernel Data Structures}

  \begin{itemize}
    \item a structure for each device node:\\
      \lstinline|struct inode|

    \medskip
    \item a structure for each open file:\\
      \lstinline|struct file|
    \begin{itemize}
      \item \lstinline|f_mode|: readable, writable, both
      \item \lstinline|f_pos|: current reading/writing position
      \item \lstinline|f_flags|
      \item \lstinline|f_op|: operations associated with the file
      \item \lstinline|private_data|: pointer to allocated data
    \end{itemize}
  \end{itemize}
\end{frame}

\begin{frame}
  \frametitle{Open}

  \begin{itemize}
    \item identify actual device
    \item check for device-specific errors
    \item initialize device
    \item allocate and initialize data structures
  \end{itemize}
\end{frame}

\begin{frame}[fragile]
  \frametitle{Kernel System Call Interface}

  \begin{itemize}
    \item \lstinline|open| system call:
    \begin{lstlisting}
int open(const char *pathname,
         int flags,
         mode_t mode);
    \end{lstlisting}

    \medskip
    \item kernel function to implement:
    \begin{lstlisting}
int foo_open(struct inode *inode,
             struct file *filp);
    \end{lstlisting}
  \end{itemize}
\end{frame}

\begin{frame}[fragile]
  \frametitle{Open}

  \begin{columns}
    \column{.43\textwidth}
    \pgfuseimage{simplescull}

    \column{.57\textwidth}
    \begin{exampleblock}{}
      \begin{lstlisting}
struct scull_dev *dev;

dev = container_of(
        inode->i_cdev,
        struct scull_dev,
        cdev
);
filp->private_data = dev;
      \end{lstlisting}
    \end{exampleblock}
  \end{columns}
\end{frame}
%
% \begin{frame}[fragile]
%   \frametitle{Opening the Device}
%
%   \begin{example}
%     \begin{lstlisting}
% /* trim device if open was write-only */
% if ((filp->f_flags & O_ACCMODE) == O_WRONLY) {
%     if (down_interruptible(&dev->sem))
%         return -ERESTARTSYS;
%     scull_trim(dev);
%     up(&dev->sem);
% }
%     \end{lstlisting}
%   \end{example}
% \end{frame}

\begin{frame}[fragile]
  \frametitle{Kernel System Call Interface}

  \begin{itemize}
    \item \lstinline|close| system call:
    \begin{lstlisting}
int close(int fd);
    \end{lstlisting}

    \medskip
    \item kernel function to implement:
    \begin{lstlisting}
int foo_release(struct inode *inode,
                struct file *filp);
    \end{lstlisting}
  \end{itemize}
\end{frame}

\begin{frame}[fragile]
  \frametitle{Kernel System Call Interface}

  \begin{itemize}
    \item \lstinline|write| system call:
    \begin{lstlisting}
ssize_t write(int fd,
              const void *buf,
              size_t count);
    \end{lstlisting}

    \medskip
    \item kernel function to implement:
    \begin{lstlisting}
ssize_t foo_write(struct file *filp,
                  const char __user *buf,
                  size_t count,
                  loff_t *f_pos);
    \end{lstlisting}
  \end{itemize}
\end{frame}

\begin{frame}[fragile]
  \frametitle{Write}

  \begin{exampleblock}{}
    \begin{lstlisting}
  struct scull_dev *dev = filp->private_data;
  ssize_t retval = -ENOMEM;

  if (down_interruptible(&dev->sem))
      return -ERESTARTSYS;

  /* determine position */
  /* allocate quantum if necessary */
  /* copy from user space */
  /* update size */

out:
  up(&dev->sem);
  return retval;
    \end{lstlisting}
  \end{exampleblock}
\end{frame}

\begin{frame}[fragile]
  \frametitle{Write}

  \begin{exampleblock}{determine position}
    \begin{lstlisting}
int quantum = dev->quantum, qset = dev->qset;
int s_pos, q_pos;

if (*f_pos >= quantum * qset)
{
    retval = 0;
    goto out;
}

s_pos = (long) *f_pos / quantum;
q_pos = (long) *f_pos % quantum;
    \end{lstlisting}
  \end{exampleblock}
\end{frame}

\begin{frame}[fragile]
  \frametitle{Write}

  \begin{exampleblock}{allocate quantum if necessary}
    \begin{lstlisting}
if (!dev->data)
{
    dev->data = kmalloc(qset * sizeof(char *),
                        GFP_KERNEL);
    if (!dev->data)
        goto out;
    memset(dev->data, 0, qset * sizeof(char *));
}
if (!dev->data[s_pos])
{
    dev->data[s_pos] = kmalloc(quantum, GFP_KERNEL);
    if (!dev->data[s_pos])
        goto out;
}
    \end{lstlisting}
  \end{exampleblock}
\end{frame}

\begin{frame}[fragile]
  \frametitle{Write}

  \begin{exampleblock}{copy from user space}
    \begin{lstlisting}
/* adjust write amount */
if (count > quantum - q_pos)
    count = quantum - q_pos;

if (copy_from_user(dev->data[s_pos] + q_pos,
                   buf, count))
{
    retval = -EFAULT;
    goto out;
}
    \end{lstlisting}
  \end{exampleblock}
\end{frame}

\begin{frame}[fragile]
  \frametitle{Write}

  \begin{exampleblock}{update size}
    \begin{lstlisting}
*f_pos += count;
retval = count;

if (dev->size < *f_pos)
    dev->size = *f_pos;
    \end{lstlisting}
  \end{exampleblock}
\end{frame}

\begin{frame}[fragile]
  \frametitle{Kernel System Call Interface}

  \begin{itemize}
    \item \lstinline|read| system call:
    \begin{lstlisting}
ssize_t read(int fd,
             void *buf,
             size_t count);
    \end{lstlisting}

    \medskip
    \item kernel function to implement:
    \begin{lstlisting}
ssize_t foo_read(struct file *filp,
                 char __user *buf,
                 size_t count,
                 loff_t *f_pos);
    \end{lstlisting}
  \end{itemize}
\end{frame}

\begin{frame}[fragile]
  \frametitle{Read}

  \begin{exampleblock}{}
    \begin{lstlisting}
  struct scull_dev *dev = filp->private_data;
  ssize_t retval = 0;

  if (down_interruptible(&dev->sem))
      return -ERESTARTSYS;

  /* determine position */
  /* copy to user space */

out:
  up(&dev->sem);
  return retval;
    \end{lstlisting}
  \end{exampleblock}
\end{frame}

\begin{frame}[fragile]
  \frametitle{Read}

  \begin{exampleblock}{determine position}
    \begin{lstlisting}
int quantum = dev->quantum;
int s_pos, q_pos;

if (*f_pos >= dev->size)
    goto out;
if (*f_pos + count > dev->size)
    count = dev->size - *f_pos;

s_pos = (long) *f_pos / quantum;
q_pos = (long) *f_pos % quantum;

if (dev->data == NULL || ! dev->data[s_pos])
    goto out;
    \end{lstlisting}
  \end{exampleblock}
\end{frame}

\begin{frame}[fragile]
  \frametitle{Reading from the Device}

  \begin{exampleblock}{copy to user space}
    \begin{lstlisting}
/* adjust read amount */
if (count > quantum - q_pos)
    count = quantum - q_pos;

if (copy_to_user(buf, dev->data[s_pos] + q_pos, count))
{
    retval = -EFAULT;
    goto out;
}
*f_pos += count;
retval = count;
    \end{lstlisting}
  \end{exampleblock}
\end{frame}

\begin{frame}[fragile]
  \frametitle{Kernel System Call Interface}

  \begin{itemize}
    \item \lstinline|lseek| system call:
    \begin{lstlisting}
off_t lseek(int fd,
            off_t offset,
            int whence);
    \end{lstlisting}

    \medskip
    \item kernel function to implement:
    \begin{lstlisting}
loff_t foo_llseek(struct file *filp,
                  loff_t off,
                  int whence);
    \end{lstlisting}
  \end{itemize}
\end{frame}

\begin{frame}[fragile]
  \frametitle{Seek}

  \begin{exampleblock}{calculate new position}
    \begin{lstlisting}
switch(whence)
{
    case 0: /* SEEK_SET */
        newpos = off;
        break;
    case 1: /* SEEK_CUR */
        newpos = filp->f_pos + off;
        break;
    case 2: /* SEEK_END */
        newpos = dev->size + off;
        break;
    default: /* can't happen */
        return -EINVAL;
}
    \end{lstlisting}
  \end{exampleblock}
\end{frame}

\begin{frame}[fragile]
  \frametitle{Seek}

  \begin{exampleblock}{set new position}
    \begin{lstlisting}
if (newpos < 0)
    return -EINVAL;
filp->f_pos = newpos;
return newpos;
    \end{lstlisting}
  \end{exampleblock}
\end{frame}

\begin{frame}[fragile]
  \frametitle{Kernel System Call Interface}

  \begin{itemize}
    \item \lstinline|ioctl| system call:
    \begin{lstlisting}
int ioctl(int fd,
          int request,
          ...);
    \end{lstlisting}

    \medskip
    \item kernel function to implement:
    \begin{lstlisting}
int foo_ioctl(struct inode *inode,
              struct file *filp,
              unsigned int cmd,
              unsigned long arg)
    \end{lstlisting}
  \end{itemize}
\end{frame}

\begin{frame}
  \frametitle{Device-Specific Commands}

  \begin{exampleblock}{}
    \begin{itemize}
      \item \lstinline|SCULL_IOCRESET|:\\
        assign default values to quantum set size and quantum size

      \pause
      \medskip
      \item \lstinline|SCULL_IOCSQUANTUM|: set quantum size from pointer
      \item \lstinline|SCULL_IOCTQUANTUM|: (tell) set quantum size from value

      \pause
      \item \lstinline|SCULL_IOCGQUANTUM|: get quantum size to pointer
      \item \lstinline|SCULL_IOCQQUANTUM|: (query) return quantum size

      \pause
      \item \lstinline|SCULL_IOCXQUANTUM|: (exchange) set + get
      \item \lstinline|SCULL_IOCHQUANTUM|: (shift) tell + query

      \pause
      \medskip
      \item similar operations for quantum set size
    \end{itemize}
  \end{exampleblock}
\end{frame}

\begin{frame}[fragile]
  \frametitle{Device Operations}

  \begin{exampleblock}{}
    \begin{lstlisting}
switch(cmd)
{
    case SCULL_IOCRESET:
        scull_quantum = SCULL_QUANTUM;
        scull_qset = SCULL_QSET;
        break;

    /* other cases */
}
    \end{lstlisting}
  \end{exampleblock}
\end{frame}

\begin{frame}[fragile]
  \frametitle{Device Operations}

  \begin{exampleblock}{setting quantum size}
    \begin{lstlisting}
case SCULL_IOCSQUANTUM:
    if (! capable (CAP_SYS_ADMIN))
        return -EPERM;
    retval = __get_user(scull_quantum,
                        (int __user *) arg);
    break;

case SCULL_IOCTQUANTUM:
    if (! capable (CAP_SYS_ADMIN))
        return -EPERM;
    scull_quantum = arg;
    break;
    \end{lstlisting}
  \end{exampleblock}
\end{frame}

\begin{frame}[fragile]
  \frametitle{Device Operations}

  \begin{exampleblock}{getting quantum size}
    \begin{lstlisting}
case SCULL_IOCGQUANTUM:
    retval = __put_user(scull_quantum,
                        (int __user *) arg);
    break;

case SCULL_IOCQQUANTUM:
    return scull_quantum;
    \end{lstlisting}
  \end{exampleblock}
\end{frame}

\subsection{Device Access}

\begin{frame}
  \frametitle{Device Driver Example}

  \begin{exampleblock}{short: read/write I/O ports}
    \begin{itemize}
      \item each device node accesses a different port:
      \begin{itemize}
        \item \texttt{/dev/short0}: port at base
        \item \texttt{/dev/short1}: port at base+1
      \end{itemize}

      \medskip
      \item module parameters:
      \begin{itemize}
        \item major number (default dynamic)
        \item base address (default 0x378)
      \end{itemize}
    \end{itemize}
  \end{exampleblock}
\end{frame}

\begin{frame}[fragile]
  \frametitle{Region Allocation}

  \begin{exampleblock}{module initializion}
    \begin{lstlisting}
if (!request_region(short_base, SHORT_NR_PORTS,
                    "short"))
{
    return -ENODEV;
}
    \end{lstlisting}
  \end{exampleblock}

  \begin{exampleblock}{module cleanup}
    \begin{lstlisting}
release_region(short_base, SHORT_NR_PORTS);
    \end{lstlisting}
  \end{exampleblock}
\end{frame}
%
% \begin{frame}[fragile]
%   \frametitle{Memory-Mapped I/O}
%
%   \begin{example}
%     \begin{lstlisting}
% /* allocating region on init */
% if (!request_mem_region(short_base,
%                         SHORT_NR_PORTS,
%                         "short")) {
%     return -ENODEV;
% }
% short_base = (unsigned long) ioremap(short_base,
%               SHORT_NR_PORTS);
%
% /* releasing region on cleanup */
% iounmap((void __iomem *) short_base);
% release_mem_region(short_base, SHORT_NR_PORTS);
%     \end{lstlisting}
%   \end{example}
% \end{frame}

\begin{frame}[fragile]
  \frametitle{Read}

  \begin{exampleblock}{}
    \begin{lstlisting}
int retval = count;
int minor = iminor(filp->f_dentry->d_inode);
unsigned long port = short_base + (minor & 0x0f);
unsigned char *kbuf, *ptr;

kbuf = kmalloc(count, GFP_KERNEL);
if (!kbuf)
    return -ENOMEM;

/* do the I/O */

kfree(kbuf);
return retval;
    \end{lstlisting}
  \end{exampleblock}
\end{frame}

\begin{frame}[fragile]
  \frametitle{Read}

  \begin{exampleblock}{do the I/O}
    \begin{lstlisting}
ptr = kbuf;
while (count--)
{
    *(ptr++) = inb(port);
    rmb();
}
if ((retval > 0) && copy_to_user(buf, kbuf, retval))
    retval = -EFAULT;
    \end{lstlisting}
  \end{exampleblock}
\end{frame}

\begin{frame}[fragile]
  \frametitle{Write}

  \begin{exampleblock}{}
    \begin{lstlisting}
if (copy_from_user(kbuf, buf, count))
    return -EFAULT;
ptr = kbuf;
while (count--)
{
    outb(*(ptr++), port);
    wmb();
}
    \end{lstlisting}
  \end{exampleblock}
\end{frame}

\subsection*{Reading Material}

\begin{frame}
  \frametitle{Reading Material}

  \begin{itemize}
    \item Corbet-Rubini-Hartman, 3/e
    \begin{itemize}
      \item Chapter 3: \alert{Char Drivers}
      \item Chapter 9: \alert{Communicating with Hardware}
    \end{itemize}
  \end{itemize}
\end{frame}

\end{document}
