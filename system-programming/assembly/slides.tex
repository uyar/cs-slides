\documentclass[dvipsnames]{beamer}

\usepackage{ae}
\usepackage[T1]{fontenc}
\usepackage[utf8]{inputenc}
\usepackage{pythontex}
\setbeamertemplate{navigation symbols}{}
\setbeamersize{text margin left=2em, text margin right=2em}

\mode<presentation>
{
  \usetheme{Frankfurt}
  \setbeamercovered{transparent}
}

\title{System Programming}
\subtitle{Assembly}

\author{H. Turgut Uyar \and Şima Uyar}
\date{2001-2014}

\AtBeginSubsection[]{
  \begin{frame}<beamer>
    \frametitle{Topics}
    \tableofcontents[currentsection,currentsubsection]
  \end{frame}
}

\pgfdeclareimage{stack1}{stack1}
\pgfdeclareimage{stack2}{stack2}
% \pgfdeclareimage[height=6.5cm]{linker}{linker}
% \pgfdeclareimage[height=6.2cm]{linkerexample}{linkerexample}

\begin{document}

\begin{frame}
  \titlepage
\end{frame}

\begin{frame}
  \frametitle{Topics}
  \tableofcontents
\end{frame}

\section{Intel Assembly}

\subsection{x86 Processors}

\begin{frame}[fragile]
  \frametitle{x86 Processors}

  \begin{itemize}
    \item Intel family of processors: x86 (32 bit), x64 (64 bit)

    \medskip
    \item very similar from the programming standpoint
    \item 8086: 16 bit processor, real mode
    \item 80386: 32 bit processor, protected mode (virtual memory)
  \end{itemize}
\end{frame}

\begin{frame}
  \frametitle{Segments}

  \begin{itemize}
    \item programs are divided into \alert{segments}

    \medskip
    \item \alert{code} segment: instructions
    \item \alert{data} segment: initialized data
    \item \alert{bss} segment: uninitialized data
    \item \alert{stack} segment
  \end{itemize}
\end{frame}

\begin{frame}
  \frametitle{8086 Registers}

  \begin{itemize}
    \item 4 general purpose data registers
    \item 2 index registers
    \item 2 pointer registers
    \item 4 segment registers
    \item 2 control registers
  \end{itemize}
\end{frame}

\begin{frame}
  \frametitle{Data Registers}

  \begin{itemize}
    \item AX: accumulator register

    \item BX: base register
    \begin{itemize}
      \item used to address data in memory
    \end{itemize}

    \item CX: counter register
    \begin{itemize}
      \item used as repetition counter in loop operations
    \end{itemize}

    \item DX: data register
    \begin{itemize}
      \item used in multiplication and division operations
    \end{itemize}

    \medskip
    \item high and low halves can be accessed as 8-bit registers:\\
      AH-AL, BH-BL, CH-CL, DH-DL
  \end{itemize}
\end{frame}

\begin{frame}
  \frametitle{Index and Pointer Registers}

  \begin{itemize}
    \item index registers:
    \begin{itemize}
      \item DI: data index
      \item SI: stack index
      \item they can be used like general purpose registers
    \end{itemize}

    \medskip
    \item pointer registers:
    \begin{itemize}
      \item BP: base pointer
      \item SP: stack pointer
    \end{itemize}
  \end{itemize}
\end{frame}

\begin{frame}
  \frametitle{Segment Registers}

  \begin{itemize}
    \item CS: code segment register
    \item DS: data segment register
    \item SS: stack segment register
    \item ES: extra segment register
  \end{itemize}
\end{frame}

\begin{frame}
  \frametitle{Control Registers}

  \begin{itemize}
    \item IP: instruction pointer
    \begin{itemize}
      \item CS + IP: address of next instruction
    \end{itemize}

    \medskip
    \item FLAGS: status conditions
    \begin{itemize}
      \item ZF (zero), OF (overflow), SF (sign), CF (carry), PF (parity)
    \end{itemize}
  \end{itemize}
\end{frame}

\begin{frame}
  \frametitle{80386}

  \begin{itemize}
    \item in 80386, registers are extended to 32 bits:\\
      EAX EBX ECX EDX ESI EDI EBP ESP\\
      EIP

    \medskip
    \item AX, BX, \ldots, BP, SP are still valid (lower 16~bits)
    \item AH, AL, \ldots, DH, DL are still valid
  \end{itemize}
\end{frame}

\subsection{Instructions}

\begin{frame}
  \frametitle{Operand Types}

  \begin{itemize}
    \item register

    \medskip
    \item memory: offset from beginning of segment

    \medskip
    \item immediate: listed in the instruction itself

    \medskip
    \item implied: not explicitly specified
  \end{itemize}
\end{frame}

\begin{frame}
  \frametitle{Basic Instructions}

  \begin{table}
    \begin{tabular}{ll}
      \pygment{nasm}|mov dest, src|
          & move src to dest\\
      \hline
      \pygment{nasm}|add dest, src|
          & add src to dest\\
      \hline
      \pygment{nasm}|adc dest, src|
          & add src to dest with carry\\
      \hline
      \pygment{nasm}|sub dest, src|
          & subtract src from dest\\
      \hline
      \pygment{nasm}|sbb dest, src|
          & subtract src from dest with borrow\\
      \hline
      \pygment{nasm}|inc dest|
          & increment dest\\
      \hline
      \pygment{nasm}|dec dest|
          & decrement dest\\
      \hline
      \pygment{nasm}|mul src|
          & multiply eax with src, result in edx:eax\\
      \hline
      \pygment{nasm}|div src|
          & divide edx:eax by src, result in eax and edx\\
    \end{tabular}
  \end{table}
\end{frame}

\begin{frame}
  \frametitle{Bitwise Instructions}

  \begin{table}
    \begin{tabular}{ll}
      \pygment{nasm}|not dest|
          & bitwise not (one's complement)\\
      \hline
      \pygment{nasm}|and dest, src|
          & bitwise and\\
      \hline
      \pygment{nasm}|or dest, src|
          & bitwise or\\
      \hline
      \pygment{nasm}|xor dest, src|
          & bitwise xor\\
      \hline
      \pygment{nasm}|neg dest|
          & negate (two's complement)\\
      \hline
      \pygment{nasm}|shl dest, amount|
          & logical shift left\\
      \hline
      \pygment{nasm}|shr dest, amount|
          & logical shift right\\
      \hline
      \pygment{nasm}|asl dest, amount|
          & arithmetic shift left\\
      \hline
      \pygment{nasm}|asr dest, amount|
          & arithmetic shift right\\
      \hline
      \pygment{nasm}|rol dest, amount|
          & rotate left\\
      \hline
      \pygment{nasm}|ror dest, amount|
          & rotate right\\
      \hline
      \pygment{nasm}|rcl dest, amount|
          & rotate left with carry\\
      \hline
      \pygment{nasm}|rcr dest, amount|
          & rotate right with carry\\
    \end{tabular}
  \end{table}
\end{frame}

\begin{frame}
  \frametitle{Branching Instructions}

  \begin{table}
    \begin{tabular}{ll}
      \pygment{nasm}|jmp| & unconditional\\
      \hline
      \pygment{nasm}|jz|  & if ZF is set\\
      \hline
      \pygment{nasm}|jnz| & if ZF is unset\\
      \hline
      \pygment{nasm}|jo|  & if OF is set\\
      \hline
      \pygment{nasm}|jno| & if OF is unset\\
      \hline
      \pygment{nasm}|js|  & if SF is set\\
      \hline
      \pygment{nasm}|jns| & if SF is unset\\
      \hline
      \pygment{nasm}|jc|  & if CF is set\\
      \hline
      \pygment{nasm}|jnc| & if CF is unset\\
      \hline
      \pygment{nasm}|jp|  & if PF is set\\
      \hline
      \pygment{nasm}|jnp| & if PF is unset\\
    \end{tabular}
  \end{table}
\end{frame}

\begin{frame}{fragile}
  \frametitle{Branching Instructions}

  \begin{itemize}
   \item \pygment{nasm}|cmp vleft, vright|: compare vleft and vright
  \end{itemize}

  \begin{table}
    \begin{tabular}{lll}
      condition             & signed               & unsigned\\
      \hline\\
      vleft $=$ vright      & \pygment{nasm}|je|   & \pygment{nasm}|je|\\
      \hline
      vleft $\neq$ vright   & \pygment{nasm}|jne|  & \pygment{nasm}|jne|\\
      \hline
      vleft $<$ vright      & \pygment{nasm}|jl|   & \pygment{nasm}|jb|\\
      \hline
      vleft $\nless$ vright & \pygment{nasm}|jnl|  & \pygment{nasm}|jnb|\\
      \hline
      vleft $\leq$ vright   & \pygment{nasm}|jle|  & \pygment{nasm}|jbe|\\
      \hline
      vleft $\nleq$ vright  & \pygment{nasm}|jnle| & \pygment{nasm}|jnbe|\\
      \hline
      vleft $>$ vright      & \pygment{nasm}|jg|   & \pygment{nasm}|ja|\\
      \hline
      vleft $\ngtr$ vright  & \pygment{nasm}|jng|  & \pygment{nasm}|jna|\\
      \hline
      vleft $\geq$ vright   & \pygment{nasm}|jge|  & \pygment{nasm}|jae|\\
      \hline
      vleft $\ngeq$ vright  & \pygment{nasm}|jnge| & \pygment{nasm}|jnae|\\
    \end{tabular}
  \end{table}
\end{frame}

\subsection{Directives}

\begin{frame}
  \frametitle{Directives}

  \begin{itemize}
    \item assembler needs extra info: \emph{directives}
    \item not part of the instruction set

    \bigskip
    \item labels: mark points in code and data
    \begin{itemize}
      \item entry labels have to marked \pygment{nasm}|global|
    \end{itemize}

    \item segments

    \item data definition

    \item named constants: \pygment{nasm}|equ|
    \begin{itemize}
      \item no memory allocated
    \end{itemize}
  \end{itemize}
\end{frame}

\begin{frame}[fragile]
  \frametitle{Code Template}

  \begin{pygments}{nasm}
segment .data
; initialized data definitions

segment .bss
; uninitialized data definitions

segment .text
global _start

_start:
    ; entry point
  \end{pygments}
\end{frame}

\begin{frame}{fragile}
  \frametitle{Data Definition}

  \begin{table}
    \begin{tabular}{lll}
      type  & initialized        & uninitialized\\
      \hline\\
      byte  & \pygment{nasm}|db| & \pygment{nasm}|resb|\\
      \hline
      word  & \pygment{nasm}|dw| & \pygment{nasm}|resw|\\
      \hline
      dword & \pygment{nasm}|dd| & \pygment{nasm}|resd|\\
      \hline
      qword & \pygment{nasm}|dq| & \pygment{nasm}|resq|\\
      \hline
      tword & \pygment{nasm}|dt| & \pygment{nasm}|rest|\\
    \end{tabular}
  \end{table}
\end{frame}

\begin{frame}[fragile]
  \frametitle{Data Definition Examples}

  \begin{pygments}{nasm}
L1 db    0
L2 dw    1000
L3 dd    1A92h
L4 db    0, 1, 2, 3
L5 db    "w", "o", "r", "d", 0
L6 db    'word', 0
L7 times 100 db 0
L8 resb  1
L9 resw  100
  \end{pygments}
\end{frame}

\begin{frame}[fragile]
  \frametitle{Dereferencing}

  \begin{columns}[t]
    \column{.5\textwidth}
    \begin{itemize}
      \item plain label:\\
        address of memory

      \medskip
      \begin{example}
        \begin{pygments}{nasm}
mov eax, L1
        \end{pygments}
      \end{example}
    \end{itemize}

    \column{.5\textwidth}
    \begin{itemize}
      \item label in brackets:\\
        contents of memory

      \medskip
      \begin{example}
        \begin{pygments}{nasm}
mov eax, [L1]
        \end{pygments}
      \end{example}
    \end{itemize}
  \end{columns}
\end{frame}

% \begin{frame}[fragile]
%   \frametitle{Addressing Issues}
%
%   \begin{itemize}
%     \item not allowed to have both operands in memory
%     \item operands must be of the same size
%   \end{itemize}
%
%   \pause
%   \begin{example}
%     \begin{itemize}
%       \item the following instructions are incorrect:
%     \end{itemize}
%
%     \begin{pygments}{nasm}
% mov [L8], [L1]
%
% mov ax, bl
%     \end{pygments}
%   \end{example}
% \end{frame}
%
% \begin{frame}[fragile]
%   \frametitle{Addressing Issues}
%
%   \begin{itemize}
%     \item NASM does not keep track of the type of data
%   \end{itemize}
%
%   \pause
%   \begin{example}
%     \begin{itemize}
%       \item the following instruction is incorrect:
%     \end{itemize}
%
%     \begin{pygments}{nasm}
% mov [L2],1
%     \end{pygments}
%
%     \pause
%     \begin{itemize}
%       \item use this instead:
%     \end{itemize}
%
%     \begin{pygments}{nasm}
% mov word [L2],1
%     \end{pygments}
%   \end{example}
% \end{frame}

\subsection{System Calls}

\begin{frame}
  \frametitle{System Calls}

  \begin{itemize}
    \item system calls are implemented using software interrupt \texttt{80h}
  \end{itemize}

  \bigskip
  \begin{block}{system call setup}
    \texttt{eax} $\leftarrow$ system call number\\
    \texttt{ebx} $\leftarrow$ first argument\\
    \texttt{ecx} $\leftarrow$ second argument\\
    \texttt{edx} $\leftarrow$ third argument\\
    \pygment{nasm}|int 80h|
  \end{block}
\end{frame}

\begin{frame}
  \frametitle{System Call Examples}

  \begin{columns}[b]
    \column{.5\textwidth}
    \begin{itemize}
      \item \pygment{c}{exit} system call number: 1
      \item arg. 1: return status
      \begin{itemize}
        \item 0: success, 1: failure
      \end{itemize}
    \end{itemize}

    \pause
    \bigskip
    \begin{itemize}
      \item read system call number: 3
      \item arg. 1: input descriptor
      \begin{itemize}
        \item 0: stdin, 1: stdout,\\
          2: stderr
      \end{itemize}
      \item arg. 2: start of input buffer
      \item arg. 3: length of input
    \end{itemize}

    \pause
    \column{.5\textwidth}
    \begin{itemize}
      \item write system call number: 4
      \item arg. 1: output descriptor
      \begin{itemize}
        \item 0: stdin, 1: stdout,\\
          2: stderr
      \end{itemize}
      \item arg. 2: start of output buffer
      \item arg. 3: length of output
    \end{itemize}
  \end{columns}
\end{frame}

\begin{frame}[fragile]
  \frametitle{Example: Hello, world!}

    \begin{columns}[t]
      \column{.55\textwidth}
      \begin{pygments}{nasm}
segment .data
msg db  "Hello, world!", 10
len equ 14
      \end{pygments}

      \column{.45\textwidth}
      \begin{pygments}{nasm}
segment .text
global _start

_start:
    mov  eax, 4
    mov  ebx, 1
    mov  ecx, msg
    mov  edx, len
    int  80h

    mov  eax, 1
    mov  ebx, 0
    int  80h
      \end{pygments}
    \end{columns}
\end{frame}

\subsection*{References}

\begin{frame}
  \frametitle{References}

  \begin{block}{Required Reading: Carter}
    \begin{itemize}
      \item Chapter 1: Introduction
      \begin{itemize}
        \item 1.2. \alert{Computer Organization}
        \item 1.3. \alert{Assembly Language}
      \end{itemize}
    \end{itemize}
  \end{block}
\end{frame}

\section{Assembly and C}

\subsection{Subroutines}

\begin{frame}
  \frametitle{Stack}

  \begin{itemize}
    \item the stack is accessed in 4-byte units
  \end{itemize}

  \begin{columns}[t]
    \column{.5\textwidth}
    \begin{block}{push}
      \pygment{nasm}|push operand|
      \begin{itemize}
        \item subtract 4 from esp
        \item store operand to address [esp]
      \end{itemize}
    \end{block}

    \column{.5\textwidth}
    \begin{block}{pop}
      \pygment{nasm}|pop register|
      \begin{itemize}
        \item store operand at address [esp] to register
        \item add 4 to esp
      \end{itemize}
    \end{block}
  \end{columns}
\end{frame}

\begin{frame}[fragile]
  \frametitle{Stack Example}

  \begin{columns}
    \column{.35\textwidth}
    \pgfuseimage{stack1}

    \begin{picture}(0,0)(-55,-38)
      \only<1,7>{\put(25,0){\vector(-1,0){25}}\put(27,-3){esp$_0$}}
      \only<2->{\put(-18,8) 1}
      \only<2,6>{\put(25,11){\vector(-1,0){25}}\put(27,8){esp$_1$}}
      \only<3->{\put(-18,19) 2}
      \only<3,5>{\put(25,22){\vector(-1,0){25}}\put(27,19){esp$_2$}}
      \only<4->{\put(-18,30) 3}
      \only<4>{\put(25,33){\vector(-1,0){25}}\put(27,30){esp$_3$}}
    \end{picture}

    \pause
    \column{.55\textwidth}
    \begin{pygments}{nasm}
push dword 1
    \end{pygments}
    \pause
    \begin{pygments}{nasm}
push dword 2
    \end{pygments}
    \pause
    \begin{pygments}{nasm}
push dword 3
    \end{pygments}
    \pause
    \begin{pygments}{nasm}
pop  eax
    \end{pygments}
    \pause
    \begin{pygments}{nasm}
pop  ebx
    \end{pygments}
    \pause
    \begin{pygments}{nasm}
pop  ecx
    \end{pygments}
  \end{columns}
\end{frame}

\begin{frame}
  \frametitle{Subroutine Call}

  \begin{columns}[t]
    \column{.5\textwidth}
    \begin{block}{call}
      \pygment{nasm}|call target|
      \begin{itemize}
        \item push address of next instruction
        \item jump to target
      \end{itemize}
    \end{block}

    \column{.5\textwidth}
    \begin{block}{ret}
      \pygment{nasm}|ret|
      \begin{itemize}
        \item pop return address
        \item jump to return address
      \end{itemize}
    \end{block}
  \end{columns}
\end{frame}

\begin{frame}
  \frametitle{Stack Parameters}

  \begin{itemize}
    \item called subroutine does not pop parameters
    \item accesses parameters on the stack
  \end{itemize}

  \begin{block}{stack layout}
    \begin{center}
      \pgfuseimage{stack2}

      \begin{picture}(0,0)(22,-38)
        \put(0,31){ret. addr.}
        \put(82,31){esp}
        \put(80,34){\vector(-1,0){25}}
        \put(1,20){parameter}
        \put(82,20){esp+4}
        \put(80,23){\vector(-1,0){25}}
        \put(1,9){parameter}
        \put(82,9){esp+8}
        \put(80,12){\vector(-1,0){25}}
      \end{picture}
    \end{center}
  \end{block}
\end{frame}

\begin{frame}
  \frametitle{Accessing Parameters}

  \begin{itemize}
    \item offsets from esp may change
  \end{itemize}

  \begin{exampleblock}{example: after a push}
    \begin{center}
      \pgfuseimage{stack2}

      \begin{picture}(0,0)(22,-27)
        \put(10,42){data}
        \put(82,42){esp}
        \put(80,45){\vector(-1,0){25}}
        \put(0,31){ret. addr.}
        \put(82,31){esp+4}
        \put(80,34){\vector(-1,0){25}}
        \put(1,20){parameter}
        \put(82,20){esp+8}
        \put(80,23){\vector(-1,0){25}}
        \put(1,9){parameter}
        \put(82,9){esp+12}
        \put(80,12){\vector(-1,0){25}}
      \end{picture}
    \end{center}
  \end{exampleblock}
\end{frame}

\begin{frame}[fragile]
  \frametitle{Accessing Parameters}

  \begin{itemize}
    \item use ebp
  \end{itemize}

  \begin{columns}[t]
    \column{.5\textwidth}
    \begin{block}{subroutine template}
      \begin{pygments}{nasm}
push ebp
mov  ebp, esp

...

pop  ebp
ret
      \end{pygments}
    \end{block}

    \column{.5\textwidth}
    \begin{block}{stack layout}
      \pgfuseimage{stack2}

      \begin{picture}(0,0)(-10,-27)
        \put(10,42){ebp}
        \put(82,42){esp, ebp}
        \put(80,45){\vector(-1,0){25}}
        \put(0,31){ret. addr.}
        \put(82,31){ebp+4}
        \put(80,34){\vector(-1,0){25}}
        \put(1,20){parameter}
        \put(82,20){ebp+8}
        \put(80,23){\vector(-1,0){25}}
        \put(1,9){parameter}
        \put(82,9){ebp+12}
        \put(80,12){\vector(-1,0){25}}
      \end{picture}
    \end{block}
  \end{columns}
\end{frame}

\begin{frame}[fragile]
  \frametitle{Example: Factorial}

  \begin{columns}[t]
    \column{.45\textwidth}
    \begin{pygments}{nasm}
segment .bss
f   resd 1

segment .text

fact:
    push ebp
    mov  ebp, esp

    mov  dword [f], 1
    mov  ecx, [ebp+8]
    \end{pygments}

    \column{.45\textwidth}
    \begin{pygments}{nasm}
back:
    mov  eax, [f]
    mul  ecx
    mov  [f], eax
    dec  ecx
    cmp  ecx, 1
    jne  back

    pop  ebp
    ret
    \end{pygments}
  \end{columns}
\end{frame}

\begin{frame}[fragile]
  \frametitle{Example: Calling Factorial}

  \begin{columns}[t]
    \column{.45\textwidth}
    \begin{pygments}{nasm}
segment .data
k   dd   5

segment .bss
f   resd 1

segment .text
global _start

fact:
    ...
    \end{pygments}

    \column{.45\textwidth}
    \begin{pygments}{nasm}
_start:
    push ebp
    mov  ebp, esp

    push dword [k]
    call fact
    add  esp, 4

    pop  ebp
    ret
    \end{pygments}
  \end{columns}
\end{frame}

\subsection{Calling Conventions}

\begin{frame}
  \frametitle{Calling Conventions}

  \begin{itemize}
    \item how will parameters be passed?
      \item if using stack:
      \begin{itemize}
        \item in what order will the parameters be pushed?
        \item who will remove parameters from the stack?
      \end{itemize}

    \item how will the result be returned?

    \item which registers should remain unchanged?
  \end{itemize}
\end{frame}

\begin{frame}
  \frametitle{C Calling Conventions}

  \begin{itemize}
    \item parameters are passed via the stack
    \begin{itemize}
      \item caller pushes parameters in reverse order
      \item caller removes parameters from the stack
    \end{itemize}

    \item result is returned over eax

    \item ebx, esi, edi, ebp, cs, ds, ss, es should remain unchanged
  \end{itemize}
\end{frame}

\subsection{C from Assembly}

\begin{frame}
  \frametitle{Calling C from Assembly}

  \begin{itemize}
    \item to call a C function from Assembly:

    \medskip
    \item declare function as \pygment{nasm}|extern|
    \item push arguments in reverse order
    \item call function
    \item adjust esp
  \end{itemize}
\end{frame}

\begin{frame}[fragile]
  \frametitle{Example: printf}

  \begin{columns}[t]
    \column{.45\textwidth}
    \begin{pygments}{nasm}
segment .data
k    dd   5
intf db   "%d", 10, 0

segment .bss
f    resd 1

segment .text
global main
extern printf

fact:
    ...
    \end{pygments}

    \column{.45\textwidth}
    \begin{pygments}{nasm}
main:
    ...

    push dword [k]
    call fact
    add  esp, 4

    push dword [f]
    push intf
    call printf
    add  esp, 8

    ...
    \end{pygments}
  \end{columns}
\end{frame}

\begin{frame}
  \frametitle{C Variables}

  \begin{itemize}
    \item global: in fixed memory locations
    \item static: same as global, only scope is different
    \item automatic: on stack
    \item register: in a register (if possible)
    \item volatile: do not optimize
  \end{itemize}
\end{frame}

\begin{frame}[fragile]
  \frametitle{Automatic Variables}

  \begin{itemize}
    \item allocation is done by subtracting from esp
  \end{itemize}

  \begin{columns}[b]
    \column{.5\textwidth}

    \begin{block}{subroutine template}
      \begin{pygments}{nasm}
push ebp
mov  ebp, esp
sub  esp, N_BYTES

...

mov  esp, ebp
pop  ebp
ret
      \end{pygments}
    \end{block}

    \column{.5\textwidth}
    \begin{block}{stack layout}
      \pgfuseimage{stack2}

      \begin{picture}(0,0)(-10,-16)
        \put(5,64){var. 2}
        \put(82,64){esp, ebp-8}
        \put(80,67){\vector(-1,0){25}}
        \put(5,53){var. 1}
        \put(82,53){ebp-4}
        \put(80,56){\vector(-1,0){25}}
        \put(10,42){ebp}
        \put(82,42){ebp}
        \put(80,45){\vector(-1,0){25}}
        \put(0,31){ret. addr.}
        \put(82,31){ebp+4}
        \put(80,34){\vector(-1,0){25}}
        \put(1,20){param. 1}
        \put(82,20){ebp+8}
        \put(80,23){\vector(-1,0){25}}
        \put(1,9){param. 2}
        \put(82,9){ebp+12}
        \put(80,12){\vector(-1,0){25}}
      \end{picture}
    \end{block}
  \end{columns}
\end{frame}

\begin{frame}[fragile]
  \frametitle{Example: Factorial (C)}

  \begin{pygments}{c}
int y;

void fact(int k)
{
  register int i;

  y = 1;
  for (i = k; i > 1; i--)
      y = y * i;
}
  \end{pygments}
\end{frame}

\begin{frame}[fragile]
  \frametitle{Example: Factorial (C)}

  \begin{pygments}{c}
int fact(int k)
{
    int y;
    register int i;

    y = 1;
    for (i = k; i > 1; i--)
        y = y * i;
    return y;
}
  \end{pygments}
\end{frame}

\begin{frame}[fragile]
  \frametitle{Example: Factorial}

  \begin{columns}[t]
    \column{.48\textwidth}
    \begin{pygments}{nasm}
segment .text
global fact

fact:
    push ebp
    mov  ebp, esp
    sub  esp, 4

    mov  dword [ebp-4], 1
    mov  ecx, [ebp+8]
    \end{pygments}

    \column{.46\textwidth}
    \begin{pygments}{nasm}
back:
    mov  eax, [ebp-4]
    mul  ecx
    mov  [ebp-4], eax
    dec  ecx
    cmp  ecx, 1
    jne  back

    mov  eax, [ebp-4]
    mov  esp, ebp
    pop  ebp
    ret
    \end{pygments}
  \end{columns}
\end{frame}

% \begin{frame}[fragile]
%   \frametitle{Function Example}
%
%   \begin{example}[Recursive Factorial (C)]
%     \begin{pygments}{nasm}[language=C]
% int fact(int k)
% {
%     if (k == 1)
%         return 1;
%     else
%         return k * fact(k - 1);
% }
%     \end{pygments}
%   \end{example}
% \end{frame}
%
% \begin{frame}[fragile]
%   \frametitle{Function Example}
%
%   \begin{example}[Recursive Factorial]
%     \begin{columns}[t]
%       \column{.47\textwidth}
%       \begin{pygments}{nasm}
% fact:
%     push ebp
%     mov  ebp, esp
%
%     mov  eax, 1
%     mov  ecx, [ebp+8]
%     cmp  ecx, 1
%     je   end_rec
%
%     push ecx
%       \end{pygments}
%
%       \column{.47\textwidth}
%       \begin{pygments}{nasm}
%     dec  ecx
%     push ecx
%     call fact
%     add  esp, 4
%
%     pop  ecx
%     mul  ecx
%
% end_rec:
%     pop  ebp
%     ret
%       \end{pygments}
%     \end{columns}
%   \end{example}
% \end{frame}

\subsection{Assembly from C}

\begin{frame}
  \frametitle{Calling Assembly from C}

  \begin{itemize}
    \item to call an Assembly function from C:

    \medskip
    \item in Assembly file: declare function as \pygment{nasm}|global|
    \item in C file: declare the prototype
  \end{itemize}
\end{frame}

\begin{frame}[fragile]
  \frametitle{Example: Factorial}

  \begin{pygments}{c}
int fact(int k);

int main(void)
{
    int x, y;

    ...
    y = fact(x);
    ...
}
  \end{pygments}
\end{frame}
%
% \lstdefinelanguage{ATT}[]{Intel}{
%   morekeywords={movb,movl,movw}
% }
% \lstdefinelanguage{Gas}[]{ATT}{
%   morekeywords={ascii,data,globl,long,text}
% }
% \lstset{language=Gas}
%
% \subsection{Inline Assembly}
%
% \begin{frame}[fragile]
%   \frametitle{AT\&T Syntax}
%
%   \begin{columns}
%     \column{.6\textwidth}
%     \begin{itemize}
%       \item order of operands:\\
%         \texttt{opcode src,dest}
%
%       \pause
%       \item operand size specified in opcode:\\
%         \pygment{nasm}|movb movw movl|
%
%       \pause
%       \item register names start with \%
%
%       \pause
%       \item immediate values start with \$
%     \end{itemize}
%
%     \pause
%     \column{.4\textwidth}
%     \begin{example}
%       \begin{pygments}{nasm}
% ; Intel: mov eax,1
% movl $1,%eax
%       \end{pygments}
%     \end{example}
%   \end{columns}
% \end{frame}
%
% \begin{frame}
%   \frametitle{Dereferencing}
%
%   \begin{itemize}
%     \item addresses start with \$:\\
%       \pygment{nasm}|movl $number,\%eax --- mov eax,number|
%
%     \pause
%     \item dereferencing by plain labels:\\
%       \pygment{nasm}|movl number,\%eax ---  mov eax,[number]|
%
%     \pause
%     \item dereferencing register addresses by enclosing in ():\\
%       \pygment{nasm}|movl (\%ebx),\%eax ---  mov eax,[ebx]|
%   \end{itemize}
% \end{frame}
%
% \begin{frame}[fragile]
%   \frametitle{AT\&T Syntax Example}
%
%   \begin{example}[Hello world]
%     \begin{columns}[t]
%       \column{.5\textwidth}
%       \begin{pygments}{nasm}[language=Gas]
% .data
% msg:
%     .ascii "Hello, world!\n"
% len:
%     .long  . - msg
%
% .text
% .globl main
%       \end{pygments}
%
%       \column{.4\textwidth}
%       \begin{pygments}{nasm}[language=Gas]
% main:
%     movl $4,%eax
%     movl $1,%ebx
%     movl $msg,%ecx
%     movl len,%edx
%     int  0x80
%
%     movl $1,%eax
%     movl $0,%ebx
%     int  0x80
%       \end{pygments}
%     \end{columns}
%   \end{example}
% \end{frame}
%
% \begin{frame}[fragile]
%   \frametitle{Inline Assembly}
%
%   \begin{itemize}
%     \item assembly code directly in C source
%     \item \alert{NOT A FUNCTION CALL}
%   \end{itemize}
%
%   \pause
%   \begin{block}{template}
%     \begin{pygments}{nasm}[language=C]
% /* some C code */
%
% asm( code
%      : output values
%      : input values
%      : clobbered registers );
%
% /* some C code */
%     \end{pygments}
%   \end{block}
% \end{frame}
%
% \begin{frame}[fragile]
%   \frametitle{Value Transfer}
%
%   \begin{itemize}
%     \item over registers
%     \begin{itemize}
%       \item \texttt{a}: eax, \texttt{b}: ebx, \texttt{c}: ecx, \texttt{d}: edx
%       \item \texttt{D}: edi, \texttt{S}: esi
%     \end{itemize}
%
%     \pause
%     \item over any register: \texttt{r}
%
%     \pause
%     \item over any register if possible, over memory otherwise: \texttt{g}
%
%     \pause
%     \item accessing in code by index
%     \begin{itemize}
%       \item starting from \texttt{\%0}
%     \end{itemize}
%   \end{itemize}
% \end{frame}
%
% \begin{frame}[fragile]
%   \frametitle{Output Value Example}
%
%   \begin{example}
%     \begin{pygments}{nasm}[language=C]
% ...
% int variable1;
% ...
% asm( code
%      : "=a" (variable1)
%      : input values
%      : clobbered registers );
%
% ...
%     \end{pygments}
%
%     \pause
%     \begin{itemize}
%       \item on exit, transfer value in eax to variable1
%     \end{itemize}
%   \end{example}
% \end{frame}
%
% \begin{frame}[fragile]
%   \frametitle{Input Value Example}
%
%   \begin{example}
%     \begin{pygments}{nasm}[language=C]
% ...
% int variable1, variable2;
% ...
% asm( code
%      : "=a" (variable1)
%      : "g" (variable2)
%      : clobbered registers );
%
% ...
%     \end{pygments}
%
%     \pause
%     \begin{itemize}
%       \item in code, access value of variable2 by its index: \%1
%     \end{itemize}
%   \end{example}
% \end{frame}
%
% \begin{frame}[fragile]
%   \frametitle{Inline Assembly Example}
%
%   \begin{example}[Inline factorial]
%     \begin{pygments}{nasm}[language=C]
% #include <stdio.h>
%
% int main(void)
% {
%     int x, y;
%
%     printf("x: ");
%     scanf("%d", &x);
%
%     asm( ... ); /* y = x! */
%
%     printf("%d\n", y);
%     return 0;
% }
%     \end{pygments}
%   \end{example}
% \end{frame}
%
% \begin{frame}[fragile]
%   \frametitle{Inline Assembly Example}
%
%   \begin{example}[Inline factorial]
%     \begin{pygments}{nasm}[language=C]
% asm("movl  %1,%%ecx\n\t"
%     "movl  $1,%%edi\n"
%     "back:\n\t"
%     "movl  %%edi,%%eax\n\t"
%     "mull  %%ecx\n\t"
%     "movl  %%eax,%%edi\n\t"
%     "decl  %%ecx\n\t"
%     "cmpl  $1,%%ecx\n\t"
%     "jne   back\n\t"
%     : "=D" (y)
%     : "g" (x)
%     : "%eax", "%ecx", "%edx");
%     \end{pygments}
%   \end{example}
% \end{frame}

\subsection*{References}

\begin{frame}
  \frametitle{References}

  \begin{block}{Required Reading: Carter}
    \begin{itemize}
      \item Chapter 4: \alert{Subprograms}
    \end{itemize}
  \end{block}
%
%     \medskip
%     \item \alert{Brennan's Guide to Inline Assembly}
%   \end{itemize}
\end{frame}
%
% \section{Linkers and Loaders}
%
% \subsection{Introduction}
%
% \begin{frame}
%   \frametitle{Basic Functions}
%
%   \begin{itemize}
%     \item binding abstract names to concrete names
%     \begin{itemize}
%       \item easier to write code using abstract names
%     \end{itemize}
%
%     \pause
%     \item related but conceptually different actions:
%     \begin{itemize}
%       \item symbol resolution
%       \item relocation
%       \item program loading
%     \end{itemize}
%   \end{itemize}
% \end{frame}
%
% \begin{frame}
%   \frametitle{Symbol Resolution}
%
%   \begin{itemize}
%     \item references between subprograms are made using \emph{symbols}
%
%     \pause
%     \item linker
%     \begin{itemize}
%       \item notes the location assigned to the called subprogram
%       \item patches the caller's object code
%     \end{itemize}
%   \end{itemize}
%
%   \pause
%   \begin{example}[\texttt{main} calls \texttt{sqrt}]
%     \begin{itemize}
%       \item linker finds location assigned to \texttt{sqrt} in the math library
%       \item patches the object code of \texttt{main} so the call refers to that
%         location
%     \end{itemize}
%   \end{example}
% \end{frame}
%
% \begin{frame}
%   \frametitle{Relocation}
%
%   \begin{itemize}
%     \item compiler generated object code starts at address 0
%     \begin{itemize}
%       \item subprograms have to be loaded at non-overlapping addresses
%     \end{itemize}
%
%     \pause
%     \item linker creates output starting at address 0
%     \begin{itemize}
%       \item subprograms relocated within the big program
%     \end{itemize}
%
%     \pause
%     \item loader picks the actual load address
%     \begin{itemize}
%       \item linked program relocated as a whole
%     \end{itemize}
%   \end{itemize}
% \end{frame}
%
% \begin{frame}
%   \frametitle{Program Loading}
%
%   \begin{itemize}
%     \item loader copies program from secondary storage to memory
%     \begin{itemize}
%       \item copy data from disk to memory
%       \item allocate storage
%       \item set protection bits
%       \item arrange for virtual memory
%     \end{itemize}
%   \end{itemize}
% \end{frame}
%
% \subsection{Address Binding}
%
% \begin{frame}
%   \frametitle{Address Binding}
%
%   \begin{itemize}
%     \item early computers were programmed in machine language
%     \begin{itemize}
%       \item write code on paper
%       \item assemble by hand
%     \end{itemize}
%
%     \pause
%     \medskip
%     \item symbols were bound to addresses:
%     \begin{itemize}
%       \item by the programmer
%       \item at the time of translation
%     \end{itemize}
%   \end{itemize}
% \end{frame}
%
% \begin{frame}
%   \frametitle{Address Binding}
%
%   \begin{itemize}
%     \item if an instruction had to be inserted or deleted:
%     \begin{itemize}
%       \item inspect the whole program
%       \item change affected addresses
%     \end{itemize}
%
%     \pause
%     \medskip
%     \item names bound to addresses too early
%   \end{itemize}
% \end{frame}
%
% \begin{frame}
%   \frametitle{Assemblers}
%
%   \begin{itemize}
%     \item programmers use symbolic names
%     \begin{itemize}
%       \item assemblers bind names to addresses
%     \end{itemize}
%
%     \pause
%     \item if program changes $\rightarrow$ reassemble
%
%     \pause
%     \medskip
%     \item the work of assigning addresses is pushed from the programmer to the
%       assembler
%   \end{itemize}
% \end{frame}
%
% \begin{frame}
%   \frametitle{Operating Systems}
%
%   \begin{itemize}
%     \item before operating systems:
%     \begin{itemize}
%       \item every process can access the entire memory
%       \item assemble and link for fixed memory addresses
%     \end{itemize}
%
%     \pause
%     \medskip
%     \item after operating systems:
%     \begin{itemize}
%       \item processes share memory
%       \item actual addresses aren't known until program is loaded
%       \item final address binding deferred past link time to load time
%     \end{itemize}
%   \end{itemize}
% \end{frame}
%
% \begin{frame}
%   \frametitle{Linker-Loader Separation}
%
%   \begin{itemize}
%     \item linker does part of address binding
%     \begin{itemize}
%       \item assigns relative addresses within each program
%     \end{itemize}
%
%     \pause
%     \item loader does a final relocation
%     \begin{itemize}
%       \item assigns actual addresses
%     \end{itemize}
%   \end{itemize}
% \end{frame}
%
% \begin{frame}
%   \frametitle{Multitasking}
%
%   \begin{itemize}
%     \item multiple programs run at the same time
%
%     \item frequently multiple copies of the same program
%     \begin{itemize}
%       \item some parts of the program are the same among all instances
%       \item other parts are unique to each instance
%     \end{itemize}
%
%     \pause
%     \item separate changing parts from unchanging parts
%     \begin{itemize}
%       \item use single copy of unchanging parts
%     \end{itemize}
%   \end{itemize}
% \end{frame}
%
% \begin{frame}
%   \frametitle{Multitasking}
%
%   \begin{itemize}
%     \item compilers were modified to generate object code in multiple sections
%     \begin{itemize}
%       \item one section for read-only code
%       \item another for writable data
%     \end{itemize}
%
%     \pause
%     \medskip
%     \item linkers had to combine sections of each type
%     \begin{itemize}
%       \item combine code sections to produce a code section
%       \item combine data sections to produce a data section
%     \end{itemize}
%   \end{itemize}
% \end{frame}
%
% \begin{frame}
%   \frametitle{Libraries}
%
%   \begin{itemize}
%     \item even different programs share common code
%     \begin{itemize}
%       \item library functions
%     \end{itemize}
%
%     \pause
%     \medskip
%     \item modern systems provide \alert{shared libraries}
%     \begin{itemize}
%       \item all programs that use a library can share a single copy
%       \item better performance, less resources
%     \end{itemize}
%   \end{itemize}
% \end{frame}
%
% \begin{frame}
%   \frametitle{Static Shared Libraries}
%
%   \begin{itemize}
%     \item addresses are bound when the library is built
%     \begin{itemize}
%       \item linker binds references to these addresses
%     \end{itemize}
%
%     \pause
%     \medskip
%     \item very inflexible
%     \begin{itemize}
%       \item if any part of library changes $\rightarrow$ relink all programs
%     \end{itemize}
%   \end{itemize}
% \end{frame}
%
% \begin{frame}
%   \frametitle{Dynamic Shared Libraries}
%
%   \begin{itemize}
%     \item library symbols are bound when program starts running
%     \begin{itemize}
%       \item linker binds references to these addresses
%     \end{itemize}
%
%     \pause
%     \item can be delayed even farther:
%     \begin{itemize}
%       \item at the time of the first call
%     \end{itemize}
%
%     \pause
%     \item programs can bind to libraries at runtime
%     \begin{itemize}
%       \item load libraries at runtime
%     \end{itemize}
%   \end{itemize}
% \end{frame}
%
% \subsection{Two-Pass Linking}
%
% \begin{frame}
%   \frametitle{Two-Pass Linking}
%
%   \begin{itemize}
%     \item input: a set of object files and libraries
%     \begin{itemize}
%       \item each input file contains segments
%     \end{itemize}
%
%     \pause
%     \medskip
%     \item output: executable or object code
%     \begin{itemize}
%       \item load map, debugger symbols, ...
%     \end{itemize}
%   \end{itemize}
% \end{frame}
%
% \begin{frame}
%   \frametitle{Two-Pass Linking}
%
%   \begin{center}
%     \pgfuseimage{linker}
%   \end{center}
% \end{frame}
%
% \begin{frame}
%   \frametitle{Symbol Table}
%
%   \begin{itemize}
%     \item each input file contains a symbol table
%
%     \pause
%     \item exported symbols
%     \begin{itemize}
%       \item defined within the file for use in other files
%       \item names of subprograms within the file that can be called from
%         elsewhere
%     \end{itemize}
%
%     \pause
%     \item imported symbols
%     \begin{itemize}
%       \item used in the file but defined elsewhere
%       \item names of subprograms called but not present in the file
%     \end{itemize}
%   \end{itemize}
% \end{frame}
%
% \begin{frame}
%   \frametitle{First Pass}
%
%   \begin{itemize}
%     \item scan input files:
%     \begin{itemize}
%       \item find sizes of segments
%       \item collect references and definitions of all symbols
%     \end{itemize}
%
%     \pause
%     \item create:
%     \begin{itemize}
%       \item \emph{segment table}: all segments defined in input files
%       \item \emph{symbol table}: all imported and exported symbols
%     \end{itemize}
%   \end{itemize}
% \end{frame}
%
% \begin{frame}
%   \frametitle{Second Pass}
%
%   \begin{itemize}
%     \item assign numeric locations to symbols
%
%     \pause
%     \item determine size and location of segments in output
%
%     \pause
%     \item substitute numeric addresses for symbol references
%     \begin{itemize}
%       \item adjust memory addresses in code and data to reflect relocated
%         addresses
%     \end{itemize}
%   \end{itemize}
% \end{frame}
%
% \begin{frame}
%   \frametitle{Linking Libraries}
%
%   \begin{itemize}
%     \item library: collection of object code
%
%     \pause
%     \item when resolving symbols:
%     \begin{itemize}
%       \item process all regular input files
%       \item if any imported symbols are still missing:\\
%         link in any library that exports the symbol
%     \end{itemize}
%   \end{itemize}
% \end{frame}
%
% \begin{frame}
%   \frametitle{Linking Libraries}
%
%   \begin{example}
%     \begin{center}
%       \pgfuseimage{linkerexample}
%     \end{center}
%   \end{example}
% \end{frame}
%
% \begin{frame}
%   \frametitle{Linking Shared Libraries}
%
%   \begin{itemize}
%     \item linker identifies the shared libraries that resolve the undefined
%       names
%
%     \item rather than linking, it notes the libraries
%
%     \item shared library is bound when program is loaded
%   \end{itemize}
% \end{frame}
%
% \subsection*{References}
%
% \begin{frame}
%   \frametitle{References}
%
%   \begin{block}{Primary Text: Levine}
%     \begin{itemize}
%       \item Chapter 1: \alert{Linking and Loading}
%       \item Chapter 3: \alert{Object Files}
%     \end{itemize}
%   \end{block}
% \end{frame}

\end{document}
