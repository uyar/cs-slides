\documentclass[dvipsnames]{beamer}

\usepackage{ae}
\usepackage[T1]{fontenc}
\usepackage[utf8]{inputenc}
\usepackage{pythontex}
\usepackage{pythontex}
\setbeamertemplate{navigation symbols}{}
\setbeamersize{text margin left=2em, text margin right=2em}

\mode<presentation>
{
  \usetheme{Frankfurt}
  \setbeamercovered{transparent}
}

\title{Systems Programming}
\subtitle{Kernel Development}

\author{H. Turgut Uyar \and Şima Uyar}
\date{2001-2014}

\AtBeginSubsection[]{
  \begin{frame}<beamer>
    \frametitle{Topics}
    \tableofcontents[currentsection,currentsubsection]
  \end{frame}
}

\pgfdeclareimage[height=6.8cm]{ldr2_0101}{ldr2_0101}
\pgfdeclareimage[height=6.8cm]{ldr2_0201}{ldr2_0201}
\pgfdeclareimage[width=8cm]{ldr2_0202}{ldr2_0202}
\pgfdeclareimage[height=6cm]{task}{task}

\begin{document}

\begin{frame}
  \titlepage
\end{frame}

\begin{frame}
  \frametitle{Topics}
  \tableofcontents
\end{frame}

\section{Kernel}

\subsection{Architecture}

\begin{frame}
  \frametitle{Kernel}

  \begin{itemize}
    \item provides programs with a consistent view of the hardware

    \bigskip
    \item protects against unauthorized access to resources
    \item kernel runs in supervisor mode (kernel space),\\
      applications run in user mode (user space)

    \pause
    \medskip
    \item switching to kernel space:
    \begin{itemize}
      \item system calls: synchronous, in the process context
      \item interrupts: asynchronous
    \end{itemize}
  \end{itemize}
\end{frame}

\begin{frame}
  \frametitle{Kernel Subsystems}

  \begin{center}
    \pgfuseimage{ldr2_0101}
  \end{center}
\end{frame}

\begin{frame}
  \frametitle{Kernel Subsystems}

  \begin{itemize}
    \item process management
    \begin{itemize}
      \item creating and destroying processes
      \item communication between processes
      \item scheduling
    \end{itemize}

    \item memory management
    \begin{itemize}
      \item virtual address space for each process
    \end{itemize}

    \item filesystems
    \begin{itemize}
      \item structured filesystem on top of unstructured hardware
    \end{itemize}

    \item device control

    \item networking
    \begin{itemize}
      \item delivering data packets across program and network interfaces
      \item routing and address resolution
    \end{itemize}
  \end{itemize}
\end{frame}

\begin{frame}
  \frametitle{Kernel Architecture}

  \begin{itemize}
    \item \alert{monolithic}: all functionality in one big chunk of code

    \bigskip
    \item \alert{microkernel}: organized as layers
    \begin{itemize}
      \item most functionality in user space
      \item too much communication overhead
    \end{itemize}
  \end{itemize}
\end{frame}

\subsection{Kernel Development}

\begin{frame}
  \frametitle{Kernel Development}

  \begin{itemize}
    \item recompile the kernel
    \item reboot the computer
    \item test the new kernel
    \item reboot to the original kernel

    \bigskip
    \item very slow development cycle!
    \item no external libraries
  \end{itemize}
\end{frame}

\begin{frame}
  \frametitle{Example: Adding a System Call}

  \begin{itemize}
    \item add an entry to the system call table:\\
      system call number, name, function to invoke, ...
    \item add prototype to the system calls header file
    \item implement system call
  \end{itemize}
\end{frame}

\begin{frame}[fragile]
  \frametitle{Example: Adding a System Call}

  \begin{itemize}
    \item new system call: add two integers

    \pause
    \medskip
    \item add an entry to the system call table
    \begin{exampleblock}{\texttt{arch/x86/kernel/syscall\_table\_32.S}}
      \begin{pygments}{c}
.long sys_mycall
      \end{pygments}
    \end{exampleblock}

    \pause
    \medskip
    \item append entry for system call
    \begin{exampleblock}{\texttt{arch/x86/include/asm/unistd\_32.h}}
      \begin{pygments}{c}
#define __NR_mycall 333
      \end{pygments}
    \end{exampleblock}
  \end{itemize}
\end{frame}

\begin{frame}[fragile]
  \frametitle{Example: Adding a System Call}

  \begin{itemize}
    \item add prototype to the system calls header file
    \begin{exampleblock}{\texttt{include/linux/syscalls.h}}
      \begin{pygments}{c}
asmlinkage int sys_mycall(int i, int j);
      \end{pygments}
    \end{exampleblock}

    \pause
    \medskip
    \item implement system call
    \begin{exampleblock}{mycall.c}
      \begin{pygments}{c}
asmlinkage int sys_mycall(int i, int j)
{
    return i + j;
}
      \end{pygments}
    \end{exampleblock}
  \end{itemize}
\end{frame}

\begin{frame}[fragile]
  \frametitle{Example: Test Program}

  \begin{pygments}{c}
#define __NR_mycall 333

int main(int argc, char **argv)
{
    int x1 = 10, x2 = 20, y;

    y = syscall(__NR_mycall, x1, x2);
    printf("%d\n", y);
    return 0;
}
  \end{pygments}
\end{frame}

\begin{frame}
  \frametitle{Data Transfer}

  \begin{itemize}
    \item special functions for transferring data\\
      between kernel space and user space

    \bigskip
    \item kernel $\rightarrow$ user:\\
      \pygment{c}|copy_to_user(user_buf, kernel_buf, length)|
    \smallskip
    \item user $\rightarrow$ kernel:\\
      \pygment{c}|copy_from_user(kernel_buf, user_buf, length)|
  \end{itemize}
\end{frame}

\begin{frame}[fragile]
  \frametitle{Example: Data Transfer}

  \begin{itemize}
    \item new system call: get the time passed since 1970

    \medskip
    \item kernel structure for representing time
    \begin{pygments}{c}
struct timeval {
    long tv_sec;    /* seconds */
    long tv_usec;   /* microseconds */
};
    \end{pygments}

    \medskip
    \item global variable that keeps the current time
    \begin{pygments}{c}
struct timeval xtime;
    \end{pygments}
  \end{itemize}
\end{frame}

\begin{frame}[fragile]
  \frametitle{Example: Data Transfer}

  \begin{pygments}{c}
asmlinkage int sys_ptime(struct timeval *tm)
{
    copy_to_user(tm, &xtime, sizeof(struct timeval));
    return 0;
}
  \end{pygments}
\end{frame}

\begin{frame}[fragile]
  \frametitle{Example: Test Program}

  \begin{pygments}{c}
#define __NR_ptime 334

int main(int argc, char **argv)
{
    struct timeval utime;
    int res;

    res = syscall(__NR_ptime, &utime);
    printf("%d\n", (int) utime.tv_sec);
    sleep(2);
    res = syscall(__NR_ptime, &utime);
    printf("%d\n", (int) utime.tv_sec);
    return 0;
}
  \end{pygments}
\end{frame}

\subsection{Kernel Modules}

\begin{frame}
  \frametitle{Modular Kernel}

  \begin{itemize}
    \item monolithic architecture
    \item modules added or removed at runtime

    \medskip
    \item no need to reboot: faster development cycle
  \end{itemize}
\end{frame}

\begin{frame}
  \frametitle{Module Registry}

  \begin{center}
    \pgfuseimage{ldr2_0201}
  \end{center}
\end{frame}

\begin{frame}[fragile]
  \frametitle{Example: Hello, world!}

  \begin{pygments}{c}
#include <linux/init.h>
#include <linux/module.h>

MODULE_LICENSE("Dual BSD/GPL");

static int hello_init(void) { ... }

static void hello_exit() { ... }

module_init(hello_init);
module_exit(hello_exit);
  \end{pygments}
\end{frame}

\begin{frame}[fragile]
  \frametitle{Example: Hello, world!}

  \begin{pygments}{c}
static int hello_init(void)
{
    printk(KERN_ALERT "Hello, world!\n");
    return 0;
}

static void hello_exit()
{
    printk(KERN_ALERT "Goodbye, cruel world!\n");
}
  \end{pygments}
\end{frame}

\begin{frame}
  \frametitle{Kernel Symbol Table}

  \begin{itemize}
    \item kernel symbol table contains addresses of global symbols

    \bigskip
    \item when loading a module:
    \smallskip
    \item unresolved symbols are linked to the kernel symbol table
    \item exported symbols become part of the kernel symbol table
  \end{itemize}
\end{frame}

\begin{frame}
  \frametitle{Module Stacking}

  \begin{itemize}
    \item modules can use symbols exported by other modules
  \end{itemize}

  \begin{center}
    \pgfuseimage{ldr2_0202}
  \end{center}
\end{frame}

\subsection*{Reading Material}

\begin{frame}
  \frametitle{Reading Material}

  \begin{itemize}
    \item Corbet-Rubini-Hartman, 3/e
    \begin{itemize}
      \item Chapter 2: \alert{Building and Running Modules}
    \end{itemize}
  \end{itemize}
\end{frame}

\section{Process Management}

\subsection{Data Structures}

\begin{frame}
  \frametitle{Process Descriptor}

  \begin{itemize}
    \item a process descriptor for each process:\\
      \pygment{c}{struct task_struct}

    \medskip
    \item process state
    \item process identification (pid, uid, euid, ...)
   \end{itemize}
\end{frame}

\begin{frame}
  \frametitle{Process Descriptor}

  \begin{center}
    \pgfuseimage{task}
  \end{center}
\end{frame}

\begin{frame}
  \frametitle{Process List}

  \begin{itemize}
    \item doubly linked list of all process descriptors
    \item \pygment{c}|tasks| field of \pygment{c}|task_struct|
    \item \pygment{c}|current| macro gives the process descriptor\\
      of the running process: e.g. \pygment{c}|current->pid|
  \end{itemize}
\end{frame}

\begin{frame}
  \frametitle{Process 0}

  \begin{itemize}
    \item a.k.a. the \textit{idle process} or the \textit{swapper}
    \item the first entry in the process list
    \item created during the initialization stage of the kernel
    \item the only process created without using the \textit{fork} system call
    \item the ancestor of all processes
  \end{itemize}
\end{frame}

\begin{frame}
  \frametitle{Process 0}

  \begin{itemize}
    \item uses a statically allocated data structure
    \begin{itemize}
      \item process descriptor stored in the \pygment{c}|init_task| variable
      \item initialized by the \pygment{c}|INIT_TASK| macro
    \end{itemize}

    \item executes the \pygment{c}|start_kernel()| function
    \begin{itemize}
      \item initializes all data structures needed by kernel
      \item enables interrupts
      \item creates process 1 (commonly known as the \textit{init process})
    \end{itemize}

    \item executes the \pygment{c}|cpu_idle()| function
  \end{itemize}
\end{frame}

\begin{frame}
  \frametitle{Creating Processes}

  \begin{itemize}
    \item \textit{fork} is implemented as the \pygment{c}|clone| system call
    \item \pygment{c}|do_fork()| function handles the \pygment{c}|clone|
      system call:

    \medskip
    \item allocates a pid for the child process
    \item uses \pygment{c}|copy_process()| to set up the process descriptor\\
      and other kernel data structures for new process
    \begin{itemize}
      \item uses \pygment{c}|dup_task_struct()| to allocate a new process
        descriptor\\
        and to copy parent process' process descriptor info
    \end{itemize}
    \item adjusts some parameters of parent and child processes
    \item returns pid of child process
  \end{itemize}
\end{frame}

\begin{frame}
  \frametitle{Destroying Processes}

  \begin{itemize}
    \item through the \pygment{c}|_exit()| system call
    \item uses the \pygment{c}|do_exit()| function
  \end{itemize}
\end{frame}

\subsection{Synchronization}

\begin{frame}
  \frametitle{Synchronization}

  \begin{itemize}
    \item critical sections and race conditions also exist for kernel code
    \item synchronization is needed

    \bigskip
    \item several kernel level synchronization primitives
    \item primitive must be chosen based on requirements of operation
  \end{itemize}
\end{frame}

\begin{frame}
  \frametitle{Synchronization Primitives}

  \begin{itemize}
    \item atomic read-modify-write operations
    \item memory barriers (to avoid instruction reordering)
    \item spin locks (locks with busy waiting)
    \item kernel semaphores (lock with blocking wait)
    \item interrupt disabling (local CPU)
  \end{itemize}
\end{frame}

\begin{frame}[fragile]
  \frametitle{Atomic Operations}

  \begin{itemize}
    \item instructions that execute atomically
    \item no interrupts

    \medskip
    \item to implement counters
    \item to atomically perform an operation and test results:\\
      e.g. \pygment{c}{atomic_dec_and_test}

    \medskip
    \begin{pygments}{c}
typedef struct {
    volatile int counter;
} atomic_t;
    \end{pygments}
  \end{itemize}
\end{frame}

\begin{frame}
  \frametitle{Memory Barriers}

  \begin{itemize}
    \item kernel may reorder assembly instructions for optimization
    \item reordering must be avoided when synchronization is needed
    \item barrier ensures that the instructions before the primitive\\
      are completed before the instructions after the primitive

    \medskip
    \item read memory barrier: \pygment{c}|rmb()|
    \item write memory barrier: \pygment{c}|wmb()|
    \item memory barrier: \pygment{c}|barrier()|
      \textendash ~ same as \pygment{c}|wmb()|
  \end{itemize}
\end{frame}

\begin{frame}
  \frametitle{Spin Locks}

  \begin{itemize}
    \item for locking access to shared data (critical sections)
    \item for multiprocessor environments
    \item uses busy waiting
    \begin{itemize}
      \item kernel resources usually locked for very short periods
      \item more time consuming to release and reacquire cpu
    \end{itemize}

    \medskip
    \item represented by a \pygment{c}|spinlock_t| structure
    \item macros used for working with spin locks
    \item read and write spin locks to increase concurrency:\\
      \pygment{c}|rwlock_t| structure
  \end{itemize}
\end{frame}

\begin{frame}
  \frametitle{Semaphones}

  \begin{itemize}
    \item sleeping locks
    \item suited for locks that are held for a long time
    \item not optimal for locks that are held for short periods

    \medskip
    \item kernel preemption not disabled,\\
      i.e. no adverse effects on scheduling latency
    \item allows arbitrary number of simultaneous lock holders:\\
      counting semaphores

    \medskip
    \item two atomic operations: \pygment{c}{P()} - \pygment{c}{V()}\\
      \pygment{c}{down()} - \pygment{c}{up()}
  \end{itemize}
\end{frame}

\subsection{Scheduling}

\begin{frame}
  \frametitle{Scheduling}

  \begin{itemize}
    \item divide the finite resource of processor time\\
      between the runnable processes on the system

    \medskip
    \item conflicting goals:
    \begin{itemize}
      \item fast process response time (low latency)
      \item maximal system utilization (high throughput)
    \end{itemize}

    \medskip
    \item processor bound processes - I/O bound processes
    \begin{itemize}
      \item Linux favors I/O bound processes, i.e. optimizes for low latency
    \end{itemize}
  \end{itemize}
\end{frame}

\begin{frame}
  \frametitle{O(1) Scheduler}

  \begin{itemize}
    \item constant-time algorithm for timeslice calculation\\
      and per processor runqueues
    \item scalable
    \item ideal for large server workloads
    \item problems for interactive processes
  \end{itemize}
\end{frame}

\begin{frame}
  \frametitle{CFS Scheduler}

  \begin{itemize}
    \item Completely Fair Scheduler
    \item aims at improving scheduling for interactive processes
  \end{itemize}
\end{frame}

\begin{frame}
  \frametitle{Linux Scheduler}

  \begin{itemize}
    \item different algorithms to schedule different types of processes
    \item scheduler classes with priorities
    \item iterate over each scheduler class in order of priority
    \item CFS for normal processes
    \item two policies for real time processes:
    \begin{itemize}
      \item \ttfamily{SCHED\_FIFO}
      \item \ttfamily{SCHED\_RR}
    \end{itemize}
  \end{itemize}
\end{frame}

\begin{frame}
  \frametitle{CFS}

  \begin{itemize}
    \item assign processes a \emph{proportion} of processor
    \item nice value (priority) acts as weight to determine\\
      proportion of processor time
    \item preemptive (based on proportions of processor time consumed)
  \end{itemize}
\end{frame}

\begin{frame}
  \frametitle{CFS}

  \begin{itemize}
    \item \emph{timeslice} proportional to process' weight\\
      over sum of weights of all runnable processes
    \item targeted latency
    \item minimum granularity
  \end{itemize}
\end{frame}

\begin{frame}
  \frametitle{CFS Implementation}

   \begin{itemize}
    \item for process accounting:\\
      \pygment{c}|struct sched_entity|
    \item member of \pygment{c}|struct task_struct|

    \medskip
    \item \emph{virtual runtime} (\pygment{c}|vruntime|):\\
      actual runtime (in ns) of a process\\
      normalized by the number of runnable processes
    \item in a perfectly multitasking system all processes should have\\
      the same virtual runtime
    \item updated periodically by the system timer\\
      and also whenever a process becomes runnable or is blocked
   \end{itemize}
\end{frame}

\begin{frame}
  \frametitle{CFS Implementation}

  \begin{itemize}
    \item the runnable process with the smallest \pygment{c}|vruntime|\\
      is selected to run
    \item red-black tree to manage list of runnable processes:\\
      search in $O(log~n)$
    \begin{itemize}
      \item leftmost node has lowest \pygment{c}|vruntime|
      \item leftmost node is cached
    \end{itemize}
  \end{itemize}
\end{frame}

\subsection*{Reading Material}

\begin{frame}
  \frametitle{Reading Material}
  \begin{itemize}
    \item Linux Kernel Development, 3rd Edition
    \begin{itemize}
      \item Author: Robert Love
      \item Publisher: Addison-Wesley Professional
      \item Year: 2010
      \item Chapters: \alert{3}, \alert{4}, 5, 9 and 10
      \item accessible on Safari e-books through the ITU Library
    \end{itemize}
  \end{itemize}
\end{frame}

% \section{The /proc Filesystem}
%
% \begin{frame}
%   \frametitle{/proc Filesystem}
%   \begin{itemize}
%      \item pseudo filesystem
%      \item for accessing information about the runtime state of the kernel
%      \item some are read-only, e.g. /proc/version
%      \item man 5 proc
%      \item \textit{ps} and \textit{top} work in user space and use /proc
%      \item \textit{sysctl} is used to modify kernel parameters at runtime
%       \begin{itemize}
%               \item parameters defined in /proc/sys
%               \item man 8 sysctl
%       \end{itemize}
%   \end{itemize}
% \end{frame}
%
% \begin{frame}
%   \frametitle{Contents}
%       \begin{itemize}
%          \item /proc/PID per process
%             \begin{itemize}
%               \item command line
%               \item current working directory
%               \item a directory per open file descriptor
%               \item executable file
%               \item memory map
%               \item status and memory usage info
%               \item ...
%             \end{itemize}
%          \item /proc/devices
%          \item /proc/cpuinfo
%          \item /proc/meminfo
%          \item /proc/modules
%          \item /proc/filesystems
%          \item /proc/partitions
%          \item /proc/version
%          \item /proc/sys/kernel
%          \item ...
%       \end{itemize}
% \end{frame}

\end{document}
