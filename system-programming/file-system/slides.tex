\documentclass[dvipsnames]{beamer}

\usepackage{ae}
\usepackage[scaled=0.88]{beramono}
\usepackage[T1]{fontenc}
\usepackage[utf8]{inputenc}
\setbeamertemplate{navigation symbols}{}
\setbeamersize{text margin left=2em, text margin right=2em}

\usepackage{listings}
\lstset{basicstyle=\ttfamily,
        keywordstyle=\color{blue},
        showstringspaces=false}
\lstdefinestyle{syntax}{frame=single}
\lstset{language=C}

\mode<presentation>
{
  \usetheme{default}
  \setbeamercovered{transparent}
}

\title{Systems Programming}
\subtitle{File System}

\author{H. Turgut Uyar \and Şima Uyar}
\date{2009-2014}

\AtBeginSubsection[]{
  \begin{frame}<beamer>
    \frametitle{Topics}
    \tableofcontents[currentsection,currentsubsection]
  \end{frame}
}

\pgfdeclareimage[height=6cm]{fuse_structure}{fuse_structure}

\begin{document}

\begin{frame}
  \titlepage
\end{frame}

\begin{frame}
  \frametitle{Topics}
  \tableofcontents
\end{frame}

\section{File System Development}

\subsection{User-Space Development}

\begin{frame}
  \frametitle{System Programming Levels}

  \begin{itemize}
    \item compiling the kernel:\\
      pros: best performance, every possible functionality\\
      cons: risky, time-consuming

    \medskip
    \item kernel modules:\\
      pros: very good performance, less risky, fast development\\
      cons: can not do everything

    \pause
    \medskip
    \item user-space:\\
      pros: even less risky, fast development, can use external libraries\\
      cons: poorer performance, can do even less
  \end{itemize}
\end{frame}

\subsection{FUSE}

\begin{frame}
  \frametitle{FUSE}

  \begin{itemize}
    \item Filesystem in Userspace
    \item develop a file system in user space on top of a kernel module

    \pause
    \bigskip
    \item non-native filesystems:\\
      NTFS, ZFS, ...
    \item changing data storage:\\
      SQL, ...
    \item providing transparent functionality:\\
      compression, encryption, ...
  \end{itemize}
\end{frame}

\begin{frame}
  \frametitle{FUSE Structure}

  \begin{center}
    \pgfuseimage{fuse_structure}
  \end{center}
\end{frame}

\begin{frame}
  \frametitle{FUSE Development}

  \begin{itemize}
    \item similar to device driver development:\\
      implement system calls

    \bigskip
    \item file related system calls:\\
      \lstinline|open, release, read, write, getattr, unlink, ...|
    \item directory related system calls:\\
      \lstinline|readdir, mkdir, rmdir, ...|
  \end{itemize}
\end{frame}

\begin{frame}
  \frametitle{Mounting}

  \begin{itemize}
    \item associating a file hierarchy with a top-level directory:\\
      \alert{mounting}
    \item requests are relative to top-level directory
  \end{itemize}

  \begin{exampleblock}{example}
    \begin{itemize}
      \item file system mounted on \texttt{/mnt/fuse}
      \medskip
      \item \texttt{ls /mnt/fuse} $\rightarrow$ \lstinline|readdir "/"|
      \item \texttt{mkdir /mnt/fuse/foo} $\rightarrow$ \lstinline|mkdir "/foo"|
    \end{itemize}
  \end{exampleblock}
\end{frame}

\section{Examples}

\subsection{Hello, world!}

\begin{frame}[fragile]
  \frametitle{Example}

  \begin{exampleblock}{Hello, world!}
    \begin{itemize}
      \item virtual filesystem with only one directory and one file
      \item name of the file: \texttt{hello.txt}
      \item contents of the file: \lstinline|Hello, world!|
    \end{itemize}

    \begin{lstlisting}[frame=single]
static const char *hello_path = "/hello.txt";
static const char *hello_str = "Hello, world!\n";
    \end{lstlisting}
  \end{exampleblock}
\end{frame}

\begin{frame}[fragile]
  \frametitle{Example}

  \begin{exampleblock}{map system calls to functions:
    \lstinline|fuse_operations|}
    \begin{lstlisting}
static struct fuse_operations hello_oper = {
    .getattr = hello_getattr,
    .readdir = hello_readdir,
    .open    = hello_open,
    .read    = hello_read,
};
    \end{lstlisting}
  \end{exampleblock}
\end{frame}

\begin{frame}[fragile]
  \frametitle{Directory Listing}

  \begin{itemize}
    \item directory listing: \lstinline|readdir|
    \begin{lstlisting}[style=syntax]
static int hello_readdir(
    const char *path,
    void *buf,
    fuse_fill_dir_t filler,
    off_t offset,
    struct fuse_file_info *fi
);
    \end{lstlisting}
  \end{itemize}
\end{frame}

\begin{frame}[fragile]
  \frametitle{Example}

  \begin{exampleblock}{\lstinline|readdir|}
    \begin{lstlisting}
if (strcmp(path, "/") != 0)
    return -ENOENT;

filler(buf, ".", NULL, 0);
filler(buf, "..", NULL, 0);
filler(buf, hello_path + 1, NULL, 0);
    \end{lstlisting}
  \end{exampleblock}
\end{frame}

\begin{frame}[fragile]
  \frametitle{File Attributes}

  \begin{itemize}
    \item reading file attributes: \lstinline|getattr|
    \begin{lstlisting}[style=syntax]
static int hello_getattr(
    const char *path,
    struct stat *stbuf
);
    \end{lstlisting}
  \end{itemize}
\end{frame}

\begin{frame}[fragile]
  \frametitle{Example}

  \begin{exampleblock}{\lstinline|getattr|}
    \begin{lstlisting}
memset(stbuf, 0, sizeof(struct stat));
if (strcmp(path, "/") == 0)
{
    stbuf->st_mode = S_IFDIR | 0755;
    stbuf->st_nlink = 2;
}
else if (strcmp(path, hello_path) == 0)
{
    stbuf->st_mode = S_IFREG | 0444;
    stbuf->st_nlink = 1;
    stbuf->st_size = strlen(hello_str);
}
else
    res = -ENOENT;
    \end{lstlisting}
  \end{exampleblock}
\end{frame}

\begin{frame}[fragile]
  \frametitle{Reading Files}

  \begin{itemize}
    \item reading from a file: \lstinline|read|
    \begin{lstlisting}[style=syntax]
static int hello_read(
    const char *path,
    char *buf,
    size_t size,
    off_t offset,
    struct fuse_file_info *finfo
);
    \end{lstlisting}
  \end{itemize}
\end{frame}

\begin{frame}[fragile]
  \frametitle{Example}

  \begin{exampleblock}{\lstinline|read|}
    \begin{lstlisting}
if (strcmp(path, hello_path) != 0)
    return -ENOENT;

len = strlen(hello_str);
if (offset < len)
{
    if (offset + size > len)
        size = len - offset;
    memcpy(buf, hello_str + offset, size);
}
else
    size = 0;

return size;
    \end{lstlisting}
  \end{exampleblock}
\end{frame}

\subsection{Read-Only Filesystem}

\begin{frame}
  \frametitle{ROFS}

  \begin{exampleblock}{read-only file system}
    \begin{itemize}
      \item access an underlying directory in read-only mode
      \item all read accesses are delegated to the underlying directory
      \item all write accesses are denied

      \medskip
      \item for example:
      \item \lstinline|rw_path|: \texttt{/home/itucs/Documents}
      \item mounted on \texttt{/mnt/Documents}
    \end{itemize}
  \end{exampleblock}
\end{frame}

\begin{frame}[fragile]
  \frametitle{Example}

  \begin{exampleblock}{operations}
    \begin{lstlisting}
struct fuse_operations rofs_oper = {
    .getattr = rofs_getattr,
    .readdir = rofs_readdir,
    .mkdir   = rofs_mkdir,
    .unlink  = rofs_unlink,
    .rmdir   = rofs_rmdir,
    .rename  = rofs_rename,
    .open    = rofs_open,
    .read    = rofs_read,
    .write   = rofs_write,
    .release = rofs_release,
    ...
};
    \end{lstlisting}
  \end{exampleblock}
\end{frame}

\begin{frame}[fragile]
  \frametitle{Example}

  \begin{exampleblock}{path translation}
    \begin{lstlisting}
char *rPath = malloc(sizeof(char) *
    (strlen(path) + strlen(rw_path) + 1));

strcpy(rPath, rw_path);
if (rPath[strlen(rPath)-1] == '/')
    rPath[strlen(rPath)-1] = '\0';
strcat(rPath, path);

return rPath;
    \end{lstlisting}
  \end{exampleblock}
\end{frame}

\begin{frame}[fragile]
  \frametitle{Example}

  \begin{exampleblock}{directory listing}
    \begin{lstlisting}
upath = translate_path(path);
dp = opendir(upath);  /* DIR *dp; */
free(upath);
if (dp == NULL)
{
    res = -errno;
    return res;
}

/* fill in the directory info */

closedir(dp);
    \end{lstlisting}
  \end{exampleblock}
\end{frame}

\begin{frame}[fragile]
  \frametitle{Example}

  \begin{exampleblock}{directory info}
    \begin{lstlisting}
/* struct dirent *de; */
while ((de = readdir(dp)) != NULL)
{
    struct stat st;
    memset(&st, 0, sizeof(st));
    st.st_ino = de->d_ino;
    st.st_mode = de->d_type << 12;
    if (filler(buf, de->d_name, &st, 0))
        break;
}
    \end{lstlisting}
  \end{exampleblock}
\end{frame}

\begin{frame}[fragile]
  \frametitle{Example}

  \begin{exampleblock}{file attributes}
    \begin{lstlisting}
upath = translate_path(path);
res = lstat(upath, st_data);
free(upath);
if (res == -1)
    return -errno;
    \end{lstlisting}
  \end{exampleblock}
\end{frame}

\begin{frame}[fragile]
  \frametitle{Example}

  \begin{exampleblock}{reading from a file}
    \begin{lstlisting}
upath = translate_path(path);
fd = open(upath, O_RDONLY);
free(upath);
if (fd == -1)
{
    res = -errno;
    return res;
}
res = pread(fd, buf, size, offset);
if (res == -1)
    res = -errno;
close(fd);
    \end{lstlisting}
  \end{exampleblock}
\end{frame}

\begin{frame}[fragile]
  \frametitle{Example}

  \begin{exampleblock}{modification operations}
    \begin{lstlisting}
static int rofs_mkdir(
    const char *path,
    mode_t mode
);

static int rofs_unlink(const char *path);

/* body */
return -EROFS;
    \end{lstlisting}
  \end{exampleblock}
\end{frame}

\end{document}
