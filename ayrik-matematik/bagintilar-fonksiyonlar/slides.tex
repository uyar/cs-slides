% Copyright (c) 2001-2010
%       H. Turgut Uyar <uyar@itu.edu.tr>
%       Ayşegül Gençata Yayımlı <gencata@itu.edu.tr>
%       Emre Harmancı <harmanci@itu.edu.tr>
%
% Bu notlar "Creative Commons Attribution-NonCommercial-ShareAlike License" ile
% lisanslanmıştır. Yazarının açıkça belirtilmesi koşuluyla ve ticari olmayan
% amaçlarla kullanılabilir ve dağıtılabilir. Bu notlardan yola çıkılarak
% oluşturulacak çalışmaların da aynı lisansa bağlı olmaları gerekir.
%
% Lisans ile ilgili ayrıntılı bilgi almak için şu sayfaya başvurabilirsiniz:
% http://creativecommons.org/licenses/by-nc-sa/3.0/

\documentclass[dvipsnames]{beamer}

\usepackage{ae}
\usepackage[T1]{fontenc}
\usepackage[utf8]{inputenc}
\usepackage[turkish]{babel}
\setbeamertemplate{navigation symbols}{}

\mode<presentation>
{
  \usetheme{Rochester}
  \setbeamercovered{transparent}
}

\title{Ayrık Matematik}
\subtitle{Bağıntılar ve Fonksiyonlar}

\author{H. Turgut Uyar \and Ayşegül Gençata Yayımlı \and Emre Harmancı}
\date{2001-2010}

\AtBeginSubsection[]
{
  \begin{frame}<beamer>
    \frametitle{Konular}
    \tableofcontents[currentsection,currentsubsection]
  \end{frame}
}

%\beamerdefaultoverlayspecification{<+->}

\theoremstyle{definition}
\newtheorem{tanim}[theorem]{Tanım}

\theoremstyle{example}
\newtheorem{ornek}[theorem]{Örnek}

\theoremstyle{plain}
\newtheorem{teorem}[theorem]{Teorem}

\pgfdeclareimage[width=2cm]{license}{../../license}

\pgfdeclareimage{baginti}{baginti}
\pgfdeclareimage{bileske1}{bileske1}
\pgfdeclareimage{bileske2}{bileske2}
\pgfdeclareimage{uyusma1}{uyusma1}
\pgfdeclareimage{uyusma2}{uyusma2}
\pgfdeclareimage{uyusma3}{uyusma3}

\begin{document}

\begin{frame}
  \titlepage
\end{frame}

\begin{frame}
  \frametitle{Lisans}

  \pgfuseimage{license}\hfill
  \copyright 2001-2010 T. Uyar, A. Yayımlı, E. Harmancı

  \vfill
  \begin{tiny}
    You are free:
    \begin{itemize}
      \item to Share — to copy, distribute and transmit the work
      \item to Remix — to adapt the work
    \end{itemize}

    Under the following conditions:
    \begin{itemize}
      \item Attribution — You must attribute the work in the manner specified by
        the author or licensor (but not in any way that suggests that they
        endorse you or your use of the work).

      \item Noncommercial — You may not use this work for commercial purposes.

      \item Share Alike — If you alter, transform, or build upon this work, you
        may distribute the resulting work only under the same or similar license
        to this one.
    \end{itemize}
  \end{tiny}

  \vfill
  Legal code (the full license):\\
  \url{http://creativecommons.org/licenses/by-nc-sa/3.0/}
\end{frame}

\begin{frame}
  \frametitle{Konular}
  \tableofcontents
\end{frame}

\section{Bağıntılar}

\subsection{Giriş}

\begin{frame}
  \frametitle{Bağıntı}

  \begin{tanim}
    \alert{bağıntı}: $\alpha \subseteq A \times B \times C \dots \times N$
  \end{tanim}

  \pause
  \begin{itemize}
    \item bağıntının her bir elemanı bir \alert{çoklu}

    \pause
    \medskip
    \item iki küme üzerindeyse: \emph{ikili bağıntı}\\
      $\alpha \subseteq A \times B$

    \pause
    \medskip
    \item gösterilim:
    \begin{itemize}
      \item çizerek
      \item matrisle
    \end{itemize}
  \end{itemize}
\end{frame}

\begin{frame}
  \frametitle{Bağıntı Örneği}

  \begin{ornek}
    $A=\{a_1,a_2,a_3,a_4\}, B=\{b_1,b_2,b_3\}$\\
    $\alpha = \{(a_1,b_1),(a_1,b_3),(a_2,b_2),(a_2,b_3),
                (a_3,b_1),(a_3,b_3),(a_4,b_1)\}$

    \pause
    \medskip
    \begin{columns}
      \column{.3\textwidth}
      \begin{center}
        \pgfuseimage{baginti}
      \end{center}

      \column{.3\textwidth}
      \[ \begin{array}{c|ccc|}
              & b_1 & b_2 & b_3\\\hline
          a_1 &  1  &  0  &  1\\
          a_2 &  0  &  1  &  1\\
          a_3 &  1  &  0  &  1\\
          a_4 &  1  &  0  &  0\\\hline
        \end{array} \]

      \column{.3\textwidth}
      \[ M_\alpha =
        \begin{array}{|ccc|}
          1 & 0 & 1\\
          0 & 1 & 1\\
          1 & 0 & 1\\
          1 & 0 & 0
        \end{array}
      \]
    \end{columns}
  \end{ornek}
\end{frame}

\begin{frame}
  \frametitle{Bağıntı Bileşkesi}

  \begin{tanim}
    \alert{bağıntı bileşkesi}:\\
      $\alpha \subseteq A \times B \wedge \beta \subseteq B \times C$\\
      $\Rightarrow \alpha \beta = \{(a,c) | a \in A, c \in C,
                \exists b \in B [a \alpha b \wedge b \beta c)]\}$
  \end{tanim}

  \pause
  \begin{itemize}
    \item $M_{\alpha \beta} = M_{\alpha} \times M_{\beta}$
  \end{itemize}
\end{frame}

\begin{frame}
  \frametitle{Bağıntı Bileşkesi Örneği}

  \begin{ornek}
    \begin{columns}
      \column{.45\textwidth}
      \begin{center}
        \pgfuseimage{bileske1}
      \end{center}

      \column{.45\textwidth}
      \begin{center}
        \pgfuseimage{bileske2}
      \end{center}
    \end{columns}
  \end{ornek}
\end{frame}

\begin{frame}
  \frametitle{Bileşke Matrisi Örneği}

  \begin{ornek}
    \begin{columns}
      \column{.28\textwidth}
      \[ M_\alpha =
         \begin{array}{|ccc|}
           1 & 0 & 0\\
           0 & 0 & 1\\
           0 & 1 & 1\\
           0 & 1 & 0\\
           1 & 0 & 1
         \end{array}
      \]

      \column{.28\textwidth}
      \[ M_\beta =
         \begin{array}{|cccc|}
           1 & 1 & 0 & 0\\
           0 & 0 & 1 & 1\\
           0 & 1 & 1 & 0
         \end{array}
      \]

      \column{.37\textwidth}
      \[ M_{\alpha \beta} =
         \begin{array}{|cccc|}
           1 & 1 & 0 & 0\\
           0 & 1 & 1 & 0\\
           0 & 1 & 1 & 1\\
           0 & 0 & 1 & 1\\
           1 & 1 & 1 & 0
        \end{array}
      \]
    \end{columns}
  \end{ornek}
\end{frame}

\begin{frame}
  \frametitle{Birleşme Özelliği}

  \begin{teorem}
      $(\alpha \beta) \gamma = \alpha (\beta \gamma) = \alpha \beta \gamma$
  \end{teorem}
\end{frame}

\begin{frame}
  \frametitle{Birleşme Özelliği}

  \begin{proof}[Tanıt]
    \begin{eqnarray*}
      &                 & (a,d) \in (\alpha \beta)\gamma\\\pause
      & \Leftrightarrow & \exists c [(a,c) \in \alpha \beta
                              \wedge (c,d) \in \gamma]\\\pause
      & \Leftrightarrow & \exists c [\exists b [(a,b) \in \alpha
                                         \wedge (b,c) \in \beta]
                                         \wedge (c,d) \in \gamma]\\\pause
      & \Leftrightarrow & \exists b [(a,b) \in \alpha
                              \wedge \exists c [(b,c) \in \beta
                              \wedge (c,d) \in \gamma]]\\\pause
      & \Leftrightarrow & \exists b [(a,b) \in \alpha
                              \wedge (b,d) \in \beta \gamma]\\\pause
      & \Leftrightarrow & (a,d) \in \alpha (\beta \gamma)
    \end{eqnarray*}
  \end{proof}
\end{frame}

\begin{frame}
  \frametitle{Bileşke Özellikleri}

  \begin{itemize}
    \item $\alpha , \delta \subseteq A \times B \wedge
           \beta , \gamma \subseteq B \times C$

    \begin{itemize}
      \item $\alpha (\beta \cup \gamma) = \alpha \beta \cup \alpha \gamma$

      \pause
      \item $\alpha (\beta \cap \gamma)
        \subseteq \alpha \beta \cap \alpha \gamma$

      \pause
      \item $(\alpha \cup \delta) \beta = \alpha \beta \cup \delta \beta$

      \pause
      \item $(\alpha \cap \delta) \beta
        \subseteq \alpha \beta \cap \delta \beta$

      \pause
      \item $(\alpha \subseteq \delta \wedge \beta \subseteq \gamma)
        \Rightarrow \alpha \beta \subseteq \delta \gamma$
    \end{itemize}
  \end{itemize}
\end{frame}

\begin{frame}
  \frametitle{Bileşke Özellikleri}

  \begin{proof}[$\alpha (\beta \cup \gamma) = \alpha \beta \cup \alpha \gamma$]
    \begin{eqnarray*}
      &                 & (x,y) \in \alpha (\beta \cup \gamma)\\\pause
      & \Leftrightarrow & \exists z [(x,z) \in \alpha
                              \wedge (z,y) \in (\beta \cup \gamma)]\\\pause
      & \Leftrightarrow & \exists z [(x,z) \in \alpha
                             \wedge ((z,y) \in \beta
                                \vee (z,y) \in \gamma)]\\\pause
      & \Leftrightarrow & \exists z [((x,z) \in \alpha \wedge (z,y) \in \beta)\\
      &                 &     ~~\vee ((x,z) \in \alpha \wedge (z,y) \in \gamma)]\\\pause
      & \Leftrightarrow & (x,y) \in \alpha \beta \vee (x,y) \in \alpha \gamma\\\pause
      & \Leftrightarrow & (x,y) \in \alpha \beta \cup \alpha \gamma
    \end{eqnarray*}
  \end{proof}
\end{frame}

\begin{frame}
  \frametitle{Evrik Bağıntı}

  \begin{tanim}
    $\alpha^{-1}: \{(y,x) | (x,y) \in \alpha \}$
  \end{tanim}

  \pause
  \begin{itemize}
    \item $M_{\alpha^{-1}} = M_{\alpha}^T$
  \end{itemize}
\end{frame}

\begin{frame}
  \frametitle{Evrik Bağıntının Özellikleri}

  \begin{itemize}
    \item $(\alpha^{-1})^{-1} = \alpha$

    \pause
    \item $(\alpha \cup \beta)^{-1} = \alpha^{-1} \cup \beta^{-1}$

    \pause
    \item $(\alpha \cap \beta)^{-1} = \alpha^{-1} \cap \beta^{-1}$

    \pause
    \item $\overline{\alpha}^{-1} = \overline{\alpha^{-1}}$

    \pause
    \item $(\alpha - \beta)^{-1} = \alpha^{-1} - \beta^{-1}$

    \pause
    \item $\alpha \subset \beta \Rightarrow \alpha^{-1} \subset \beta^{-1}$
  \end{itemize}
\end{frame}

\begin{frame}
  \frametitle{Evrik Bağıntı Teoremleri}

  \begin{proof}[$\overline{\alpha}^{-1} = \overline{\alpha^{-1}}$]
    \begin{eqnarray*}
      &                 & (x,y) \in \overline{\alpha}^{-1}\\\pause
      & \Leftrightarrow & (y,x) \in \overline{\alpha}\\\pause
      & \Leftrightarrow & (y,x) \notin \alpha\\\pause
      & \Leftrightarrow & (x,y) \notin \alpha^{-1}\\\pause
      & \Leftrightarrow & (x,y) \in \overline{\alpha^{-1}}
    \end{eqnarray*}
  \end{proof}
\end{frame}

\begin{frame}
  \frametitle{Evrik Bağıntı Teoremleri}

  \begin{proof}[$(\alpha \cap \beta)^{-1} = \alpha^{-1} \cap \beta^{-1}$]
    \begin{eqnarray*}
      &                 & (x,y) \in (\alpha \cap \beta)^{-1}\\\pause
      & \Leftrightarrow & (y,x) \in (\alpha \cap \beta)\\\pause
      & \Leftrightarrow & (y,x) \in \alpha \wedge (y,x) \in \beta\\\pause
      & \Leftrightarrow & (x,y) \in \alpha^{-1}
                   \wedge (x,y) \in \beta^{-1}\\\pause
      & \Leftrightarrow & (x,y) \in \alpha^{-1} \cap \beta^{-1}
    \end{eqnarray*}
  \end{proof}
\end{frame}

\begin{frame}
  \frametitle{Evrik Bağıntı Teoremleri}

  \begin{proof}[$(\alpha - \beta)^{-1} = \alpha^{-1} - \beta^{-1}$]
    \begin{eqnarray*}
      (\alpha - \beta)^{-1} & = & (\alpha \cap \overline{\beta})^{-1}\\\pause
                            & = & \alpha^{-1} \cap \overline{\beta}^{-1}\\\pause
                            & = & \alpha^{-1} \cap \overline{\beta^{-1}}\\\pause
                            & = & \alpha^{-1} - \beta^{-1}
    \end{eqnarray*}
  \end{proof}
\end{frame}

\begin{frame}
  \frametitle{Bileşke Evriği}

  \begin{teorem}
    $(\alpha \beta)^{-1} = \beta^{-1} \alpha^{-1}$
  \end{teorem}

  \pause
  \begin{proof}[Tanıt]
    \begin{eqnarray*}
      &                 & (c,a) \in (\alpha \beta)^{-1}\\\pause
      & \Leftrightarrow & (a,c) \in \alpha \beta\\\pause
      & \Leftrightarrow & \exists b [(a,b) \in \alpha
                              \wedge (b,c) \in \beta]\\\pause
      & \Leftrightarrow & \exists b [(c,b) \in \beta^{-1}
                              \wedge (b,a) \in \alpha^{-1}]\\\pause
      & \Leftrightarrow & (c,a) \in \beta^{-1}\alpha^{-1}
    \end{eqnarray*}
  \end{proof}
\end{frame}

\begin{frame}
  \frametitle{Bileşke Evriğinin Matrisi}

  \begin{itemize}
    \item $M_{(\alpha \beta)^{-1}} = M_{\beta^{-1}} \times M_{\alpha^{-1}}$
    \item $M_{\alpha \beta}^{T} = M_{\beta}^{T} \times M_{\alpha}^{T}$
  \end{itemize}
\end{frame}

\begin{frame}
  \frametitle{Bileşke Evriğinin Matrisi Örnekleri}

  \begin{ornek}
    \begin{columns}
      \column{.27\textwidth}
      \[ M_\alpha =
        \begin{array}{|ccc|}
          1 & 0 & 0\\
          0 & 0 & 1\\
          0 & 1 & 1\\
          0 & 1 & 0\\
          1 & 0 & 1
        \end{array}
      \]

      \column{.27\textwidth}
      \[ M_\beta =
        \begin{array}{|cccc|}
          1 & 1 & 0 & 0\\
          0 & 0 & 1 & 1\\
          0 & 1 & 1 & 0
        \end{array}
      \]
    \end{columns}

    \[ M_{\alpha \beta^{-1}} =
      \begin{array}{|ccccc|}
        1 & 0 & 0 & 0 & 1\\
        1 & 1 & 1 & 0 & 1\\
        0 & 1 & 1 & 1 & 1\\
        0 & 0 & 1 & 1 & 0
      \end{array}
    \]
  \end{ornek}
\end{frame}

\subsection{Küme İçi Bağıntılar}

\begin{frame}
  \frametitle{Kümeiçi Bağıntı}

  \begin{itemize}
    \item $\alpha \subseteq A \times A$

    \pause
    \medskip
    \item \emph{birim bağıntı} $E = \{(x,x) | x \in A\}$

    \pause
    \medskip
    \item bileşke: $\alpha \alpha = \alpha^2$
    \begin{itemize}
      \item $\alpha \alpha \dots \alpha = \alpha^n$
    \end{itemize}
  \end{itemize}
\end{frame}

\begin{frame}
  \frametitle{Kümeiçi Bağıntı Özellikleri}

  \begin{itemize}
    \item yansıma
    \item bakışlılık
    \item geçişlilik
  \end{itemize}
\end{frame}

\begin{frame}
  \frametitle{Yansıma}

  \begin{block}{yansımalı}
    $\alpha \subseteq A \times A$\\
    $\forall a~[a \alpha a]$
  \end{block}

  \pause
  \begin{itemize}
%     \item $E \subseteq \alpha$
%     \item bağıntı matrisinde ana köşegen bütünüyle 1
    \item yansımasız:\\
      $\exists a~[\neg (a \alpha a)]$

    \pause
    \item ters yansımalı:\\
      $\forall a~[\neg (a \alpha a)]$
  \end{itemize}
\end{frame}
%
% \begin{frame}
%   \frametitle{Yansımasızlık}
%
%   \begin{itemize}
%     \item $\exists a \in A~[\neg (a \alpha a)]$
%
%     \pause
%     \medskip
%     \item $\neg (E \subseteq \alpha)$
%     \item bağıntı matrisinde ana köşegende en az bir tane 0
%   \end{itemize}
% \end{frame}
%
% \begin{frame}
%   \frametitle{Ters Yansıma}
%
%   \begin{itemize}
%     \item $\forall a \in A~[\neg (a \alpha a)]$
%
%     \pause
%     \medskip
%     \item $E \subseteq \overline{\alpha}$
%     \item bağıntı matrisinde ana köşegen bütünüyle 0
%   \end{itemize}
% \end{frame}

\begin{frame}
  \frametitle{Yansıma Örnekleri}

  \begin{columns}[t]
    \column{.45\textwidth}
    \begin{ornek}
      $\mathcal{R}_1 \subseteq \{1,2\} \times \{1,2\}$\\
      $\mathcal{R}_1 = \{(1,1), (2,2)\}$

      \medskip
      \begin{itemize}
        \item $\mathcal{R}_1$ yansımalı
      \end{itemize}
    \end{ornek}

    \pause
    \column{.45\textwidth}
    \begin{ornek}
      $\mathcal{R}_2 \subseteq \{1,2,3\} \times \{1,2,3\}$\\
      $\mathcal{R}_2 = \{(1,1), (2,2)\}$

      \medskip
      \begin{itemize}
        \item $\mathcal{R}_2$ yansımasız
      \end{itemize}
    \end{ornek}
  \end{columns}
\end{frame}

\begin{frame}
  \frametitle{Yansıma Örnekleri}

  \begin{ornek}
    $\mathcal{R} \subseteq \{1,2,3\} \times \{1,2,3\}$\\
    $\mathcal{R} = \{(1,2), (2,1), (2,3)\}$

    \medskip
    \begin{itemize}
      \item $\mathcal{R}$ ters yansımalı
    \end{itemize}
  \end{ornek}
\end{frame}

\begin{frame}
  \frametitle{Yansıma Örnekleri}

  \begin{ornek}
    $\mathcal{R} \subseteq \mathbb{Z} \times \mathbb{Z}$\\
    $(a,b) \in \mathcal{R} \equiv ab \geq 0$

    \medskip
    \begin{itemize}
      \item $\mathcal{R}$ yansımalı
    \end{itemize}
  \end{ornek}
\end{frame}

\begin{frame}
  \frametitle{Bakışlılık}

  \begin{block}{bakışlı}
    $\alpha \subseteq A \times A$\\
    $\forall a,b [(a=b) \vee (a \alpha b \wedge b \alpha a)
                        \vee (\neg(a \alpha b) \wedge \neg(b \alpha a))]$

    \pause
    \medskip
    $\forall a,b [(a=b) \vee (a \alpha b \leftrightarrow b \alpha a)]$
  \end{block}

  \pause
  \begin{itemize}
%     \item $\alpha = \alpha^{-1}$
%     \item bağıntı matrisi ana köşegene göre simetrik:
%       $M_{\alpha} = M_{\alpha}^T$

    \item bakışsız:\\
      $\exists a,b [(a \neq b) \wedge (a \alpha b \wedge \neg(b \alpha a))
                               \vee (\neg (a \alpha b) \wedge b \alpha a))]$

    \pause
    \item ters bakışlı:
    \begin{eqnarray*}
                      & \forall a,b~
                    [(a=b) \vee \neg(a \alpha b) \vee \neg(b \alpha a)]\\\pause
      \Leftrightarrow & \forall a,b~
                    [\neg(a \alpha b \wedge b \alpha a) \vee (a=b)]\\\pause
      \Leftrightarrow & \forall a,b~
                    [(a \alpha b \wedge b \alpha a) \rightarrow (a=b)]
    \end{eqnarray*}
  \end{itemize}
\end{frame}
%
% \begin{frame}
%   \frametitle{Bakışsızlık}
%
%   \begin{itemize}
%     \item $\exists a,b \in A$\\
%       $[a \neq b \wedge ((a \alpha b \wedge \neg(b \alpha a))
%                     \vee (\neg(a \alpha b) \wedge b \alpha a))]$
%
%     \pause
%     \medskip
%     \item $\exists a,b \in A$\\
%       $[a \neq b \wedge ((a \alpha b \vee b \alpha a)
%                  \wedge (\neg(a \alpha b) \vee \neg(b \alpha a)))]$
%   \end{itemize}
% \end{frame}
%
% \begin{frame}
%   \frametitle{Ters Bakışlılık}
%
%   \begin{eqnarray*}
%                     & \forall a,b \in A~
%                   [a = b \vee \neg(a \alpha b) \vee \neg(b \alpha a)]\\\pause
%     \Leftrightarrow & \forall a,b \in A~
%                   [\neg(a \alpha b \wedge b \alpha a) \vee (a=b)]\\\pause
%     \Leftrightarrow & \forall a,b \in A~
%                   [(a \alpha b \wedge b \alpha a) \Rightarrow (a=b)]
%   \end{eqnarray*}
%
%   \pause
%   \medskip
%   \begin{itemize}
%     \item  $\alpha \alpha^{-1} \subseteq E$
%   \end{itemize}
% \end{frame}

\begin{frame}
  \frametitle{Bakışlılık Örnekleri}

  \begin{ornek}
    $\mathcal{R} \subseteq \{1,2,3\} \times \{1,2,3\}$\\
    $\mathcal{R} = \{(1,2), (2,1), (2,3)\}$

    \pause
    \medskip
    \begin{itemize}
      \item $\mathcal{R}$ bakışsız
    \end{itemize}
  \end{ornek}
\end{frame}

\begin{frame}
  \frametitle{Bakışlılık Örnekleri}

  \begin{ornek}
    $\mathcal{R} \subseteq \mathbb{Z} \times \mathbb{Z}$\\
    $(a,b) \in \mathcal{R} \equiv ab \geq 0$

    \medskip
    \begin{itemize}
      \item $\mathcal{R}$ bakışlı
    \end{itemize}
  \end{ornek}
\end{frame}

\begin{frame}
  \frametitle{Bakışlılık Örnekleri}

  \begin{ornek}
    $\mathcal{R} \subseteq \{1,2,3\} \times \{1,2,3\}$\\
    $\mathcal{R} = \{(1,1), (2,2)\}$

    \begin{itemize}
      \item $\mathcal{R}$ bakışlı ve ters bakışlı
    \end{itemize}
  \end{ornek}
\end{frame}

\begin{frame}
  \frametitle{Geçişlilik}

  \begin{block}{geçişli}
    $\alpha \subseteq A \times A$\\
    $\forall a,b,c~[(a \alpha b \wedge b \alpha c) \rightarrow (a \alpha c)]$
  \end{block}

  \pause
  \begin{itemize}
%     \item $\alpha^2 \subseteq \alpha$
    \item geçişsiz:\\
      $\exists a,b,c~[(a \alpha b \wedge b \alpha c) \wedge \neg (a \alpha c)]$

    \pause
    \item ters geçişli:\\
      $\forall a,b,c~[(a \alpha b \wedge b \alpha c) \rightarrow \neg (a \alpha c)]$
  \end{itemize}
\end{frame}
%
% \begin{frame}
%   \frametitle{Geçişsizlik}
%
%   \begin{itemize}
%      \item $\exists a,b,c \in A$\\
%       $a \alpha b \wedge b \alpha c \wedge \neg(a \alpha c)$
%   \end{itemize}
% \end{frame}
%
% \begin{frame}
%   \frametitle{Ters Geçişlilik}
%
%   \begin{itemize}
%      \item $\forall a,b,c \in A$\\
%       $(a \alpha b \wedge b \alpha c) \Rightarrow \neg(a \alpha c)$
%   \end{itemize}
% \end{frame}

\begin{frame}
  \frametitle{Geçişlilik Örnekleri}

  \begin{ornek}
    $\mathcal{R} \subseteq \{1,2,3\} \times \{1,2,3\}$\\
    $\mathcal{R} = \{(1,2), (2,1), (2,3)\}$

    \medskip
    \begin{itemize}
      \item $\mathcal{R}$ ters geçişli
    \end{itemize}
  \end{ornek}
\end{frame}

\begin{frame}
  \frametitle{Geçişlilik Örnekleri}

  \begin{ornek}
    $\mathcal{R} \subseteq \mathbb{Z} \times \mathbb{Z}$\\
    $(a,b) \in \mathcal{R} \equiv ab \geq 0$

    \medskip
    \begin{itemize}
      \item $\mathcal{R}$ geçişsiz
    \end{itemize}
  \end{ornek}
\end{frame}

\begin{frame}
  \frametitle{Evrik Bağıntı}

  \begin{teorem}
    Yansıma, bakışlılık ve geçişlilik özellikleri evrik bağıntıda korunur.
  \end{teorem}
\end{frame}

\begin{frame}
  \frametitle{Örtüler}

  \begin{itemize}
    \item yansımalı örtü:\\
      $r_{\alpha} = \alpha \cup E$

    \pause
    \medskip
    \item bakışlı örtü:\\
      $s_{\alpha} = \alpha \cup \alpha^{-1}$

    \pause
    \medskip
    \item geçişli örtü:\\
      $t_{\alpha} = \bigcup_{i=1 \dots n}~\alpha^i
        = \alpha \cup \alpha^2 \cup \alpha^3 \cup \cdots \cup \alpha^n$
  \end{itemize}
\end{frame}

\begin{frame}
  \frametitle{Özel Bağıntılar}

  \begin{block}{önce gelen - sonra gelen}
    $\mathcal{R} \subseteq \mathbb{Z} \times \mathbb{Z}$\\
    $(a,b) \in \mathcal{R} \equiv a-b=1$

    \medskip
    \begin{itemize}
      \item $\mathcal{R}$ ters yansımalı
      \item $\mathcal{R}$ ters bakışlı
      \item $\mathcal{R}$ ters geçişli
    \end{itemize}
  \end{block}
\end{frame}

\begin{frame}
  \frametitle{Özel Bağıntılar}

  \begin{block}{bitişiklik}
    $\mathcal{R} \subseteq \mathbb{Z} \times \mathbb{Z}$\\
    $(a,b) \in \mathcal{R} \equiv |a-b|=1$

    \medskip
    \begin{itemize}
      \item $\mathcal{R}$ ters yansımalı
      \item $\mathcal{R}$ bakışlı
      \item $\mathcal{R}$ ters geçişli
    \end{itemize}
  \end{block}
\end{frame}

\begin{frame}
  \frametitle{Özel Bağıntılar}

  \begin{block}{dar sıra}
    $\mathcal{R} \subseteq \mathbb{Z} \times \mathbb{Z}$\\
    $(a,b) \in \mathcal{R} \equiv a<b$

    \medskip
    \begin{itemize}
      \item $\mathcal{R}$ ters yansımalı
      \item $\mathcal{R}$ ters bakışlı
      \item $\mathcal{R}$ geçişli
    \end{itemize}
  \end{block}
\end{frame}

\begin{frame}
  \frametitle{Özel Bağıntılar}

  \begin{block}{kısmi sıra}
    $\mathcal{R} \subseteq \mathbb{Z} \times \mathbb{Z}$\\
    $(a,b) \in \mathcal{R} \equiv a \leq b$

    \medskip
    \begin{itemize}
      \item $\mathcal{R}$ yansımalı
      \item $\mathcal{R}$ ters bakışlı
      \item $\mathcal{R}$ geçişli
    \end{itemize}
  \end{block}
\end{frame}

\begin{frame}
  \frametitle{Özel Bağıntılar}

  \begin{block}{önsıra}
    $\mathcal{R} \subseteq \mathbb{Z} \times \mathbb{Z}$\\
    $(a,b) \in \mathcal{R} \equiv |a| \leq |b|$

    \medskip
    \begin{itemize}
      \item $\mathcal{R}$ yansımalı
      \item $\mathcal{R}$ bakışsız
      \item $\mathcal{R}$ geçişli
    \end{itemize}
  \end{block}
\end{frame}

\begin{frame}
  \frametitle{Özel Bağıntılar}

  \begin{block}{sınırlı fark}
    $\mathcal{R} \subseteq \mathbb{Z} \times \mathbb{Z}$\\
    $(a,b) \in \mathcal{R} \equiv |a-b| \leq m$

    \medskip
    \begin{itemize}
      \item $\mathcal{R}$ yansımalı
      \item $\mathcal{R}$ bakışlı
      \item $\mathcal{R}$ geçişsiz
    \end{itemize}
  \end{block}
\end{frame}

\begin{frame}
  \frametitle{Özel Bağıntılar}

  \begin{block}{karşılaştırılabilirlik}
    $\mathcal{R} \subseteq \mathbb{U} \times \mathbb{U}$\\
    $(a,b) \in \mathcal{R} \equiv (a \subseteq b) \vee (b \subseteq a)$

    \medskip
    \begin{itemize}
      \item $\mathcal{R}$ yansımalı
      \item $\mathcal{R}$ bakışlı
      \item $\mathcal{R}$ geçişsiz
    \end{itemize}
  \end{block}
\end{frame}

\begin{frame}
  \frametitle{Özel Bağıntılar}

  \begin{block}{kardeşlik}
    \begin{itemize}
      \item ters yansımalı
      \item bakışlı
      \item geçişli
    \end{itemize}

    \pause
    \medskip
    \begin{itemize}
      \item bir bağıntı bakışlı, geçişli ve ters yansımalı olabilir mi?
    \end{itemize}
  \end{block}
\end{frame}

\subsection{Eşdeğerlilik}

\begin{frame}
  \frametitle{Uyuşma}

  \begin{tanim}
    \alert{uyuşma bağıntısı}: $\gamma$
    \begin{itemize}
      \item yansımalı
      \item bakışlı
    \end{itemize}
  \end{tanim}

  \pause
  \begin{itemize}
    \item çizerek gösterilim yönsüz
    \item matris gösterilimi merdiven şeklinde
  \end{itemize}

  \pause
  \begin{itemize}
    \item $\alpha \alpha^{-1}$ bir uyuşma bağıntısıdır
  \end{itemize}
\end{frame}

\begin{frame}
  \frametitle{Uyuşma Örnekleri}

  \begin{ornek}
    \begin{columns}
      \column{.45\textwidth}
      \begin{eqnarray*}
        A           & = & \{a_1,a_2,a_3,a_4\}\\
        \mathcal{R} & = & \{\\
                    &   & (a_1,a_1),(a_2,a_2),\\
                    &   & (a_3,a_3),(a_4,a_4),\\
                    &   & (a_1,a_2),(a_2,a_1),\\
                    &   & (a_2,a_4),(a_4,a_2),\\
                    &   & (a_3,a_4),(a_4,a_3)\\
                    &   & \}
      \end{eqnarray*}

      \column{.2\textwidth}
      \begin{center}
        \pgfuseimage{uyusma1}

        \bigskip
        \pgfuseimage{uyusma2}
      \end{center}

      \pause
      \column{.3\textwidth}
      \begin{center}
        \[ \begin{array}{|cccc|}
            1  &  1  &  0  &  0\\
            1  &  1  &  0  &  1\\
            0  &  0  &  1  &  1\\
            0  &  1  &  1  &  1
          \end{array} \]

        \[ \begin{array}{|ccc|}
            1  &     & \\
            0  &  0  & \\
            0  &  1  &  1
          \end{array} \]
      \end{center}
    \end{columns}
  \end{ornek}
\end{frame}

\begin{frame}
  \frametitle{Uyuşma Örnekleri}

  \begin{ornek}[$\alpha \alpha^{-1}$]
    $A$: kişiler, $B$: diller

    \medskip
    $A=\{a_1,a_2,a_3,a_4,a_5,a_6\}$\\
    $B=\{b_1,b_2,b_3,b_4,b_5\}$\\
    $\alpha \subseteq A \times B$

    \pause
    \begin{columns}
      \column{.45\textwidth}
      \[ M_\alpha = \begin{array}{|ccccc|}
          1  &  1  &  0  &  0  &  0\\
          1  &  1  &  0  &  0  &  0\\
          0  &  0  &  1  &  0  &  1\\
          1  &  0  &  1  &  1  &  0\\
          0  &  0  &  0  &  1  &  0\\
          0  &  1  &  1  &  0  &  0
        \end{array} \]

      \column{.45\textwidth}
      \[ M_{\alpha^{-1}} = \begin{array}{|cccccc|}
          1  &  1  &  0  &  1  &  0  &  0\\
          1  &  1  &  0  &  0  &  0  &  1\\
          0  &  0  &  1  &  1  &  0  &  1\\
          0  &  0  &  0  &  1  &  1  &  0\\
          0  &  0  &  1  &  0  &  0  &  0
        \end{array} \]
    \end{columns}
  \end{ornek}
\end{frame}

\begin{frame}
  \frametitle{Uyuşma Örnekleri}

  \begin{ornek}[$\alpha \alpha^{-1}$]
    $\alpha \alpha^{-1} \subseteq A \times A$

    \begin{columns}
      \column{.45\textwidth}
      \[ M_{\alpha \alpha^{-1}}= \begin{array}{|cccccc|}
          1  &  1  &  0  &  1  &  0  &  1\\
          1  &  1  &  0  &  1  &  0  &  1\\
          0  &  0  &  1  &  1  &  0  &  1\\
          1  &  1  &  1  &  1  &  1  &  1\\
          0  &  0  &  0  &  1  &  1  &  0\\
          1  &  1  &  1  &  1  &  0  &  1
        \end{array} \]

      \column{.45\textwidth}
      \begin{center}
        \pgfuseimage{uyusma3}
      \end{center}
    \end{columns}
  \end{ornek}
\end{frame}

\begin{frame}
  \frametitle{Uyuşanlar Sınıfı}

  \begin{tanim}
    \alert{uyuşanlar sınıfı}: $C \subseteq A$\\
      $\forall a,b~[a \in C \wedge b \in C \rightarrow a \gamma b]$
  \end{tanim}

  \pause
  \medskip
  \begin{itemize}
    \item \emph{en üst uyuşanlar sınıfı}:\\
      başka bir uyuşanlar sınıfının altkümesi değil

    \pause
    \item bir eleman birden fazla EÜS'ye girebilir

    \pause
    \medskip
    \item \alert{eksiksiz örtü}: $C_\gamma$\\
      tüm EÜS'lerin oluşturduğu küme
  \end{itemize}
\end{frame}

\begin{frame}
  \frametitle{Uyuşanlar Sınıfı Örnekleri}

  \begin{ornek}[$\alpha \alpha^{-1}$]
    \begin{columns}
      \column{.3\textwidth}
      \begin{center}
        \pgfuseimage{uyusma3}
      \end{center}

      \pause
      \column{.65\textwidth}
      \begin{itemize}
        \item $C_1=\{a_4,a_6\}$
        \item $C_2=\{a_2,a_4,a_6\}$
        \item $C_3=\{a_1,a_2,a_4,a_6\}$ (EÜS)
      \end{itemize}

      \pause
      \medskip
      \begin{eqnarray*}
        C_\gamma (A) & = & \{\\
                      &   & \{a_1,a_2,a_4,a_6\},\\
                      &   & \{a_3,a_4,a_6\},\\
                      &   & \{a_4,a_5\}\\
                      &   & \}
      \end{eqnarray*}
    \end{columns}
  \end{ornek}
\end{frame}

\begin{frame}
  \frametitle{Eşdeğerlilik}

  \begin{tanim}
    \alert{eşdeğerlilik bağıntısı}: $\epsilon$
    \begin{itemize}
      \item yansımalı
      \item bakışlı
      \item geçişli
    \end{itemize}
  \end{tanim}

  \pause
  \begin{itemize}
    \item eşdeğerlilik sınıfları
    \item her eleman tek bir eşdeğerlilik sınıfına girer

    \pause
    \medskip
    \item eksiksiz örtü: $C_\epsilon$
  \end{itemize}
\end{frame}

\begin{frame}
  \frametitle{Bölmeleme}

  \begin{itemize}
    \item her eşdeğerlilik bağıntısı tanımlandığı kümeyi\\
      ayrık eşdeğerlilik sınıflarına \emph{bölmeler}

    \pause
    \medskip
    \item her \emph{bölmeleme} bir eşdeğerlilik bağıntısına karşı düşer
  \end{itemize}
\end{frame}

\begin{frame}
  \frametitle{Eşdeğerlilik Örnekleri}

  \begin{ornek}
    $\mathcal{R} \subseteq \mathbb{Z} \times \mathbb{Z}$\\
    $(a,b) \in \mathcal{R} \equiv 5~|~|a - b|$

    \pause
    \bigskip
    $x~mod~5$ işlemi $\mathbb{Z}$ kümesini yukarıdaki bağıntıya göre\\
    5~eşdeğerlilik sınıfına bölmeler
  \end{ornek}
\end{frame}

\subsection*{Kaynaklar}

\begin{frame}
  \frametitle{Kaynaklar}

  \begin{block}{Grimaldi}
    \begin{itemize}
      \item Chapter 5: Relations and Functions
      \begin{itemize}
        \item 5.1. \alert{Cartesian Products and Relations}
      \end{itemize}
      \item Chapter 7: Relations: The Second Time Around
      \begin{itemize}
        \item 7.1. \alert{Relations Revisited: Properties of Relations}
        \item 7.4. \alert{Equivalence Relations and Partitions}
      \end{itemize}
    \end{itemize}
  \end{block}

  \begin{block}{Yardımcı Kitap: O'Donnell, Hall, Page}
    \begin{itemize}
      \item Chapter 10: Relations
    \end{itemize}
  \end{block}
\end{frame}

\section{Fonksiyonlar}

\subsection{Giriş}

\begin{frame}
  \frametitle{Fonksiyon}

  \begin{tanim}
    \alert{fonksiyon}: $f: X \rightarrow Y$\\
    $\forall x \in X~\forall y_1,y_2 \in Y~
      (x,y_1),(x,y_2) \in f \Rightarrow y_1=y_2$
  \end{tanim}

  \pause
  \medskip
  \begin{itemize}
    \item $X$: \emph{tanım kümesi}, $Y$: \emph{değer kümesi}

    \pause
    \medskip
    \item $(x,y) \in f \equiv y = f(x)$
    \item $y$, $x$'in $f$ altındaki \emph{görüntüsü}
  \end{itemize}
\end{frame}

\begin{frame}
  \frametitle{Altküme Görüntüsü}

  \begin{tanim}
    \alert{altküme görüntüsü}:\\
    $f: X \rightarrow Y \wedge X_1 \subseteq X$\\
    $f(X_1) = \{ y | y \in Y, x \in X_1 \wedge y = f(x) \}$
  \end{tanim}
\end{frame}

\begin{frame}
  \frametitle{Altküme Görüntüsü Örnekleri}

  \begin{ornek}
    $\begin{array}{l}
      f: \mathbb{R} \rightarrow \mathbb{R}\\
      f(x) = x^2
    \end{array}$

    \pause
    \medskip
    \begin{itemize}
      \item $f(\mathbb{Z}) = \{0,1,4,9,16,\dots\}$

      \pause
      \medskip
      \item $A = \{-2,1\}$\\
        $f(A) = \{1,4\}$
    \end{itemize}
  \end{ornek}
\end{frame}

\begin{frame}
  \frametitle{Birebir Fonksiyon}

  \begin{tanim}
    $f: X \rightarrow Y$ fonksiyonu \alert{birebir}:\\
      $\forall x_1,x_2 \in X~f(x_1)=f(x_2) \Rightarrow x_1=x_2$
  \end{tanim}
\end{frame}

\begin{frame}
  \frametitle{Birebir Fonksiyon Örnekleri}

  \begin{columns}[t]
    \column{.5\textwidth}
    \begin{ornek}
      $\begin{array}{l}
        f: \mathbb{R} \rightarrow \mathbb{R}\\
        f(x) = 3x + 7
      \end{array}$

      \pause
      \bigskip
      $\begin{array}{llll}
                    & f(x_1)    & = & f(x_2)\pause\\
        \Rightarrow & 3 x_1 + 7 & = & 3 x_2 + 7\pause\\
        \Rightarrow & 3 x_1     & = & 3 x_2\pause\\
        \Rightarrow & x_1       & = & x_2
      \end{array}$
    \end{ornek}

    \pause
    \column{.5\textwidth}
    \begin{block}{Karşı Örnek}
      $\begin{array}{l}
        g: \mathbb{Z} \rightarrow \mathbb{Z}\\
        g(x) = x^4 - x
      \end{array}$

      \pause
      \bigskip
      $\begin{array}{lllll}
        g(0) & = & 0^4 - 0 & = & 0\pause\\
        g(1) & = & 1^4 - 1 & = & 0
      \end{array}$
    \end{block}
  \end{columns}
\end{frame}

\begin{frame}
  \frametitle{Örten Fonksiyon}

  \begin{tanim}
    $f: X \rightarrow Y$ fonksiyonu \alert{örten}:\\
    $\forall y \in Y~\exists x \in X~f(x)=y$
  \end{tanim}

  \pause
  \begin{itemize}
    \item $f(X)=Y$
  \end{itemize}
\end{frame}

\begin{frame}
  \frametitle{Örten Fonksiyon Örnekleri}

  \begin{columns}
    \column{.5\textwidth}
    \begin{ornek}
      $f: \mathbb{R} \rightarrow \mathbb{R}$\\
      $f(x) = x^3$
    \end{ornek}

    \pause
    \column{.5\textwidth}
    \begin{block}{Karşı Örnek}
      $f: \mathbb{Z} \rightarrow \mathbb{Z}$\\
      $f(x) = 3x + 1$
    \end{block}
  \end{columns}
\end{frame}

\begin{frame}
  \frametitle{Bijektif Fonksiyon}

  \begin{tanim}
    $f: X \rightarrow Y$ fonksiyonu \alert{bijektif}:\\
    $f$ fonksiyonu birebir ve örten
  \end{tanim}
\end{frame}

\begin{frame}
  \frametitle{Altküme Görüntüsü Özellikleri}

  \begin{itemize}
    \item $f: A \rightarrow B \wedge A_1,A_2 \subseteq A$:

    \pause
    \medskip
    \begin{itemize}
      \item $f(A_1 \cup A_2) = f(A_1) \cup f(A_2)$
      \item $f(A_1 \cap A_2) \subseteq f(A_1) \cap f(A_2)$

      \pause
      \medskip
      \item $f$ birebir ise:\\
        $f(A_1 \cap A_2) = f(A_1) \cap f(A_2)$
    \end{itemize}
  \end{itemize}
\end{frame}

\begin{frame}
  \frametitle{Fonksiyon Bileşkesi}

  \begin{tanim}
    $f: X \rightarrow Y, g: Y \rightarrow Z$

    \medskip
    $g \circ f: X \rightarrow Z$\\
    $(g \circ f)(x) = g(f(x))$
  \end{tanim}

  \begin{itemize}
    \item değişme özelliği göstermez
    \item birleşme özelliği gösterir:\\
      $f \circ (g \circ h) = (f \circ g) \circ h$
  \end{itemize}
\end{frame}

\begin{frame}
  \frametitle{Fonksiyon Bileşkesi Örnekleri}

  \begin{ornek}[değişme özelliği]
    $f: \mathbb{R} \rightarrow \mathbb{R}$\\
    $f(x) = x^2$

    \medskip
    $g: \mathbb{R} \rightarrow \mathbb{R}$\\
    $g(x) = x + 5$

    \pause
    \bigskip
    $g \circ f: \mathbb{R} \rightarrow \mathbb{R}$\\
    $(g \circ f)(x) = x^2 + 5$

    \pause
    \medskip
    $f \circ g: \mathbb{R} \rightarrow \mathbb{R}$\\
    $(f \circ g)(x) = (x + 5)^2$
  \end{ornek}
\end{frame}

\begin{frame}
  \frametitle{Fonksiyon Bileşkesi Teoremleri}

  \begin{teorem}
    $f: X \rightarrow Y, g: Y \rightarrow Z$:\\
    $f$ birebir $\wedge$ $g$ birebir $\Rightarrow$ $g \circ f$ birebir
  \end{teorem}

  \pause
  \begin{proof}[Tanıt]
    \[\begin{array}{crcl}
                & (g \circ f) (a_1) & = & (g \circ f) (a_2)\\\pause
    \Rightarrow & g(f(a_1))         & = & g(f(a_2))\\\pause
    \Rightarrow & f(a_1)            & = & f(a_2)\\\pause
    \Rightarrow & a_1               & = & a_2
    \end{array}\]
  \end{proof}
\end{frame}

\begin{frame}
  \frametitle{Fonksiyon Bileşkesi Teoremleri}

  \begin{teorem}
    $f: X \rightarrow Y, g: Y \rightarrow Z$:\\
    $f$ örten $\wedge$ $g$ örten $\Rightarrow$ $g \circ f$ örten
  \end{teorem}

  \pause
  \begin{proof}[Tanıt]
    $\forall z \in Z~\exists y \in Y~g(y) = z$\\\pause
    $\forall y \in Y~\exists x \in X~f(x) = y$\\\pause
      $\Rightarrow \forall z \in Z~\exists x \in X~g(f(x)) = z$
  \end{proof}
\end{frame}

\begin{frame}
  \frametitle{Birim Fonksiyon}

  \begin{tanim}
    \alert{birim fonksiyon}: $1_X$

    \medskip
    $1_X: X \rightarrow X$\\
    $1_X(x) = x$
  \end{tanim}
\end{frame}

\begin{frame}
  \frametitle{Evrik Fonksiyon}

  \begin{tanim}
    $f: X \rightarrow Y$ fonksiyonu \alert{evrilebilir}:\\
      $\exists f^{-1}: Y \rightarrow X~f^{-1} \circ f = 1_X \wedge f \circ f^{-1} = 1_Y$

    \begin{itemize}
      \item $f^{-1}$: f fonksiyonunun \alert{evriği}
    \end{itemize}
  \end{tanim}
\end{frame}

\begin{frame}
  \frametitle{Evrilebilir Fonksiyon Örnekleri}

  \begin{ornek}
    $f: \mathbb{R} \rightarrow \mathbb{R}$\\
    $f(x) = 2x + 5$

    \pause
    \bigskip
    $f^{-1}: \mathbb{R} \rightarrow \mathbb{R}$\\
    $f^{-1}(x) = \frac{x - 5}{2}$

    \pause
    \bigskip
    $(f^{-1} \circ f)(x) = f^{-1}(f(x))$
    \pause
    $ = f^{-1}(2x + 5)$
    \pause
    $ = \frac{(2x + 5) - 5}{2}$
    \pause
    $ = \frac{2x}{2} = x$
    \medskip

    \pause
    $(f \circ f^{-1})(x) = f(f^{-1}(x))$
    \pause
    $ = f(\frac{x - 5}{2})$
    \pause
    $ = 2 \frac{x - 5}{2} + 5$
    \pause
    $ = (x - 5) + 5 = x$
  \end{ornek}
\end{frame}

\begin{frame}
  \frametitle{Fonksiyon Evriği}

  \begin{teorem}
    Bir fonksiyon evrilebilirse evriği tektir.
  \end{teorem}

  \pause
  \begin{proof}[Tanıt]
    $f: X \rightarrow Y$

    \pause
    \medskip
    $g,h: Y \rightarrow X$\\
    $g \circ f = 1_X \wedge f \circ g = 1_Y$\\
    $h \circ f = 1_X \wedge f \circ h = 1_Y$

    \pause
    \medskip
    $h = h \circ 1_Y$
    \pause
    $ = h \circ (f \circ g)$
    \pause
    $ = (h \circ f) \circ g$
    \pause
    $ = 1_X \circ g$
    \pause
    $ = g$
  \end{proof}
\end{frame}

\begin{frame}
  \frametitle{Evrilebilir Fonksiyon}

  \begin{teorem}
    Bir fonksiyon yalnız ve ancak birebir ve örten ise evrilebilir.
  \end{teorem}
\end{frame}

\begin{frame}
  \frametitle{Evrilebilir Fonksiyon}

  \begin{columns}[t]
    \column{.6\textwidth}
    \begin{proof}[Evrilebilir ise birebirdir]
      $f: A \rightarrow B$
      \begin{eqnarray*}
        &             & f(a_1) = f(a_2)\\\pause
        & \Rightarrow & f^{-1}(f(a_1)) = f^{-1}(f(a_2))\\\pause
        & \Rightarrow & (f^{-1} \circ f) (a_1) = (f^{-1} \circ f) (a_2)\\\pause
        & \Rightarrow & 1_A (a_1) = 1_A (a_2)\\\pause
        & \Rightarrow & a_1 = a_2
      \end{eqnarray*}
    \end{proof}

    \pause
    \column{.35\textwidth}
    \begin{proof}[Evrilebilir ise örtendir]
      $f: A \rightarrow B$
      \begin{eqnarray*}
        &   & b\\\pause
        & = & 1_B (b)\\\pause
        & = & (f \circ f^{-1}) (b)\\\pause
        & = & f(f^{-1} (b))
      \end{eqnarray*}
    \end{proof}
  \end{columns}
\end{frame}

\begin{frame}
  \frametitle{Evrilebilir Fonksiyon}

  \begin{proof}[Birebir ve örten ise evrilebilir]
    $f: A \rightarrow B$
    \begin{itemize}
      \item $f$ örten $\Rightarrow \forall b \in B~\exists a \in A~f(a)=b$
      \item $g: B \rightarrow A$ fonksiyonu $a=g(b)$ ile belirlensin

      \pause
      \medskip
      \item $g(b) = a_1 \neq a_2 = g(b)$ olabilir mi?

      \pause
      \item $f(a_1) = b = f(a_2)$ olması gerekir

      \pause
      \item olamaz: $f$ birebir
    \end{itemize}
  \end{proof}
\end{frame}

\begin{frame}
  \frametitle{Permutasyonlar}

  \begin{itemize}
    \item permutasyon: küme içi bijektif bir fonksiyon

    \medskip
    $\left(
      \begin{array}{cccc}
       a_1   &  a_2   & \dots &  a_n\\
      p(a_1) & p(a_2) & \dots & p(a_n)
      \end{array}
    \right)$

    \pause
    \medskip
    \item $n$ elemanlı bir kümede $n!$ permutasyon tanımlanabilir
  \end{itemize}
\end{frame}

\begin{frame}
  \frametitle{Permutasyon Örnekleri}

  \begin{ornek}
    $A = \{1,2,3\}$

    \medskip
    $\begin{array}{cc}
      p_1 = \left(
        \begin{array}{ccc}
          1 & 2 & 3\\
          1 & 2 & 3
        \end{array}
      \right) &
      p_2 = \left(
        \begin{array}{ccc}
          1 & 2 & 3\\
          1 & 3 & 2
        \end{array}
      \right)\medskip\\
      p_3 = \left(
        \begin{array}{ccc}
          1 & 2 & 3\\
          2 & 1 & 3
        \end{array}
      \right) &
      p_4 = \left(
        \begin{array}{ccc}
          1 & 2 & 3\\
          2 & 3 & 1
        \end{array}
      \right)\medskip\\
      p_5 = \left(
        \begin{array}{ccc}
          1 & 2 & 3\\
          3 & 1 & 2
        \end{array}
      \right) &
      p_6 = \left(
        \begin{array}{ccc}
          1 & 2 & 3\\
          3 & 2 & 1
        \end{array}
      \right)
    \end{array}$
  \end{ornek}
\end{frame}

\begin{frame}
  \frametitle{Çevrimli Permutasyon}

  \begin{itemize}
    \item \emph{çevrimli permutasyon}:
    \begin{itemize}
      \item elemanların bir altkümesi bir çevrim oluşturuyor
      \item diğerleri yer değiştirmiyor
    \end{itemize}

    \pause
    \medskip
    $(a_i, a_j, a_k) = \left(
      \begin{array}{ccccccccc}
      \dots & a_i & \dots & a_n & \dots & a_j & \dots & a_k & \dots\\
      \dots & a_j & \dots & a_n & \dots & a_k & \dots & a_i & \dots
      \end{array}
    \right)$

    \pause
    \bigskip
    \item \emph{transpozisyon}: 2~uzunluklu çevrimli permutasyon
  \end{itemize}
\end{frame}

\begin{frame}
  \frametitle{Çevrimli Permutasyon Örnekleri}

  \begin{ornek}
    $A = \{1,2,3,4,5\}$

    \medskip
    $(1,3,5) = \left(
      \begin{array}{ccccc}
        1 & 2 & 3 & 4 & 5\\
        3 & 2 & 5 & 4 & 1
      \end{array}
    \right)$
  \end{ornek}
\end{frame}

\begin{frame}
  \frametitle{Permutasyon Bileşkesi}

  \begin{itemize}
    \item permutasyon bileşkesi değişme özelliği göstermez
  \end{itemize}

  \pause
  \begin{ornek}
    $A = \{1,2,3,4,5\}$

    \medskip
    \small{$\begin{array}{lll}
      (4,1,3,5) \diamond (5,2,3) & = & \left(
        \begin{array}{ccccc}
          1 & 2 & 3 & 4 & 5\\
          3 & 2 & 5 & 1 & 4
        \end{array}
      \right) \diamond \left(
        \begin{array}{ccccc}
          1 & 2 & 3 & 4 & 5\\
          1 & 3 & 5 & 4 & 2
        \end{array}
      \right)\smallskip\pause\\
      & = & \left(
        \begin{array}{ccccc}
          1 & 2 & 3 & 4 & 5\\
          5 & 3 & 2 & 1 & 4
        \end{array}
      \right)\medskip\pause\\
      (5,2,3) \diamond (4,1,3,5) & = & \left(
        \begin{array}{ccccc}
          1 & 2 & 3 & 4 & 5\\
          1 & 3 & 5 & 4 & 2
        \end{array}
      \right) \diamond \left(
        \begin{array}{ccccc}
          1 & 2 & 3 & 4 & 5\\
          3 & 2 & 5 & 1 & 4
      \end{array}
    \right)\smallskip\pause\\
    & = & \left(
      \begin{array}{ccccc}
        1 & 2 & 3 & 4 & 5\\
        3 & 5 & 4 & 1 & 2
      \end{array}
    \right)
    \end{array}$}
  \end{ornek}
\end{frame}

\begin{frame}
  \frametitle{Çevrimli Permutasyon Bileşkesi}

  \begin{itemize}
    \item çevrimli olmayan her permutasyon ayrık çevrimlerin bileşkesi olarak
      yazılabilir
  \end{itemize}

  \pause
  \begin{ornek}
    $A = \{1,2,3,4,5,6,7,8\}$

    \medskip
    $\left(
      \begin{array}{cccccccc}
        1 & 2 & 3 & 4 & 5 & 6 & 7 & 8\\
        3 & 4 & 6 & 5 & 2 & 1 & 8 & 7
      \end{array}
    \right) = (1,3,6) \diamond (2,4,5) \diamond (7,8)$
  \end{ornek}
\end{frame}

\begin{frame}
  \frametitle{Transpozisyon Bileşkesi}

  \begin{itemize}
    \item çevrimli her permutasyon transpozisyon bileşkesi olarak yazılabilir
  \end{itemize}

  \pause
  \begin{ornek}
    $A = \{1,2,3,4,5\}$

    \medskip
    $(1,2,3,4,5) = (1,2) \diamond (1,3) \diamond (1,4) \diamond (1,5)$
  \end{ornek}
\end{frame}

\subsection{Güvercin Deliği İlkesi}

\begin{frame}
  \frametitle{Güvercin Deliği İlkesi}

  \begin{tanim}
    \alert{güvercin deliği ilkesi} (Dirichlet kutuları):\\
    $m$ adet güvercin $n$ adet deliğe yerleşirse ve $m>n$ ise\\
    en az bir delikte birden fazla güvercin vardır
  \end{tanim}

  \pause
  \begin{itemize}
    \item $f: X \rightarrow Y \wedge |X|>|Y|$ ise $f$ birebir bir fonksiyon
      olamaz

    \item $\exists x_1,x_2 \in X~x_1 \neq x_2 \wedge f(x_1)=f(x_2)$
  \end{itemize}
\end{frame}

\begin{frame}
  \frametitle{Güvercin Deliği İlkesi Örnekleri}

  \begin{ornek}
    \begin{itemize}
      \item 367~kişinin bulunduğu bir yerde en az iki kişinin doğum günü aynıdır

      \pause
      \item 0 ile 100 arasında notlar alınan bir sınavda en az iki öğrencinin
        aynı notu alması için sınava kaç öğrenci girmiş olmalıdır?
    \end{itemize}
  \end{ornek}
\end{frame}

\begin{frame}
  \frametitle{Genelleştirilmiş Güvercin Deliği İlkesi}

  \begin{tanim}
    \alert{genelleştirilmiş güvercin deliği ilkesi}:\\
    $m$ adet nesne $n$ adet kutuya dağıtılırsa\\
    en az bir kutuda en az $\lceil m / n \rceil$ adet nesne olur
  \end{tanim}

  \pause
  \begin{ornek}
    100 kişinin bulunduğu bir yerde en az $\lceil 100 / 12 \rceil = 9$ kişi
    aynı ayda doğmuştur
  \end{ornek}
\end{frame}

\begin{frame}
  \frametitle{Güvercin Deliği İlkesi Örneği}

  \begin{teorem}
    S = \{1,2,3,\dots,9\} kümesinin 6~elemanlı herhangi bir altkümesinde
    toplamı 10~olan iki sayı vardır.
  \end{teorem}
\end{frame}

\begin{frame}
  \frametitle{Güvercin Deliği İlkesi Örneği}

  \begin{teorem}
    $S$ kümesi en büyüğü 14~olabilen 6~elemanlı bir pozitif tamsayılar kümesi
    olsun. $S$'nin boş olmayan altkümelerinin elemanlarının toplamlarının hepsi
    birbirinden farklı olamaz.
  \end{teorem}

  \pause
  \begin{columns}[t]
    \column{.5\textwidth}
    \begin{block}{Tanıt Denemesi}
      $A \subseteq S$\\
      $s_A: A$'nın elemanlarının toplamı

      \pause
      \begin{itemize}
        \item delik:\\
          $1 \leq s_A \leq 9 + \dots + 14 = 69$
        \item güvercin: $2^6 - 1 = 63$
      \end{itemize}
    \end{block}

    \pause
    \column{.5\textwidth}
    \begin{proof}[Tanıt]
      $|A| \leq 5$ olan altkümelere bakalım.

      \pause
      \begin{itemize}
        \item delik:\\
          $1 \leq s_A \leq 10 + \dots + 14 = 60$
        \item güvercin: $2^6 - 2 = 62$
      \end{itemize}
    \end{proof}
  \end{columns}
\end{frame}

\begin{frame}
  \frametitle{Güvercin Deliği İlkesi Örneği}

  \begin{teorem}
    $S = \{1,2,3,\dots,200\}$ kümesinden seçilecek 101~elemanın içinde\\
    en az bir çift vardır ki çiftin bir elemanı diğerini böler.
  \end{teorem}

  \pause
  \begin{block}{Tanıt Yöntemi}
    \begin{itemize}
      \item $\forall n~\exists ! p~
        (n = 2^r p \wedge r \geq 0 \wedge \exists t \in \mathbb Z~p = 2t + 1)$\\
      olduğu gösterilecek

      \item bu teorem kullanılarak asıl teorem tanıtlanacak
    \end{itemize}
  \end{block}
\end{frame}

\begin{frame}
  \frametitle{Güvercin Deliği İlkesi Örneği}

  \begin{teorem}
    $\forall n~\exists ! p~
      (n = 2^r p \wedge r \geq 0 \wedge \exists t \in \mathbb Z~p = 2t + 1)$\\
  \end{teorem}

  \pause
  \begin{columns}[t]
    \column{.55\textwidth}
    \begin{proof}[Varlık Tanıtı]
      $n = 1$: $r = 0, p = 1$\\
      $n = 2$: $r = 1, p = 1$

      \pause
      $n \leq k$: $n = 2^r p$

      \pause
      $n = k + 1:
      \begin{array}{ll}
         n~asal:        & r = 0, p = n\pause\\
         \neg (n~asal): & n = n_1 n_2\pause\\
                        & n = 2^{r_1} p_1 \cdot 2^{r_2} p_2\pause\\
                        & n = 2^{r_1+r_2} \cdot p_1 p_2
      \end{array}$
    \end{proof}

    \pause
    \column{.45\textwidth}
    \begin{proof}[Teklik Tanıtı]
      tek değilse:

      \pause
      $\begin{array}{lllll}
        n & =           & 2^{r_1} p_1     & = & 2^{r_2} p_2\pause\\
          & \Rightarrow & 2^{r_1-r_2} p_1 & = & p_2\pause\\
          & \Rightarrow & 2 | p_2
      \end{array}$
      \alert{çelişki}
    \end{proof}
  \end{columns}
\end{frame}

\begin{frame}
  \frametitle{Güvercin Deliği İlkesi Örneği}

  \begin{teorem}
    $S = \{1,2,3,\dots,200\}$ kümesinden seçilecek 101~elemanın içinde\\
    en az bir çift vardır ki çiftin bir elemanı diğerini böler.
  \end{teorem}

  \pause
  \begin{proof}[Tanıt]
    \begin{itemize}
      \item $T \subseteq S$, $T$ kümesi $S$ kümesinin bütün tek elemanlarından
        oluşan altkümesi olsun: $|T|=100$

      \pause
      \item $f: S \rightarrow T, (s, t) \in f \equiv s = 2^r t \wedge r \geq 0$
      \begin{itemize}
        \item $S$'den 101~eleman seçilirse en az ikisinin\\
          $T$'deki görüntüsü aynı olur:
          $f(s_1)=f(s_2) \Rightarrow 2^{m_1} t = 2^{m_2} t$

        \pause
        \[
          \frac {s_1} {s_2} = \frac {2^{m_1} t} {2^{m_2} t} = 2^{m_1 - m_2}
        \]
      \end{itemize}
    \end{itemize}
  \end{proof}
\end{frame}

\subsection{Rekürsiyon}

\begin{frame}
  \frametitle{Rekürsif Fonksiyonlar}

  \begin{tanim}
    \alert{rekürsif fonksiyon}:\\
      kendisi cinsinden tanımlanan fonksiyon

    \medskip
    $f(n) = h(f(m))$
  \end{tanim}

  \begin{itemize}
    \item \emph{tümevarımla tanımlanan fonksiyon}:\\
      her rekürsiyonda boyut azalıyor

    \medskip
    $f(n) = \left\{
      \begin{array}{ll}
        k         & n = 0\\
        h(f(n-1)) & n > 0
      \end{array}
    \right.$
  \end{itemize}
\end{frame}

\begin{frame}
  \frametitle{Rekürsif Fonksiyon Örnekleri}

  \begin{ornek}
    $f91(n) = \left\{
      \begin{array}{ll}
        n - 10         & n > 100\\
        f91(f91(n+11)) & n \leq 100
      \end{array}
    \right.$
  \end{ornek}
\end{frame}

\begin{frame}
  \frametitle{Tümevarımla Tanımlanan Fonksiyon Örnekleri}

  \begin{ornek}[faktöryel]
    $f(n) = \left\{
      \begin{array}{ll}
        1              & n = 0\\
        n \cdot f(n-1) & n > 0
      \end{array}
    \right.$
  \end{ornek}

  \pause
  \begin{ornek}[fonksiyon kuvveti]
    $f^n = \left\{
      \begin{array}{ll}
        f               & n = 1\\
        f \circ f^{n-1} & n > 1
      \end{array}
    \right.$
  \end{ornek}
\end{frame}

\begin{frame}
  \frametitle{Euclid Algoritması}

  \begin{ornek}[ortak bölenlerin en büyüğü]
    \begin{eqnarray*}
      333 & = & 3 \cdot 84 + 81\\\pause
       84 & = & 1 \cdot 81 + 3\\\pause
       81 & = & 27 \cdot 3 + 0
    \end{eqnarray*}

    \pause
    $obeb(333,84) = 3$

    \pause
    \medskip
    $obeb(a,b) = \left\{
      \begin{array}{ll}
        b               & b | a\\
        obeb(b,a~mod~b) & b \nmid a
      \end{array}
    \right.$
  \end{ornek}
\end{frame}

\begin{frame}
  \frametitle{Fibonacci Dizisi}

  \begin{block}{Fibonacci dizisi}
    $F_n = fib(n) = \left\{
      \begin{array}{ll}
        1                   & n = 1\\
        1                   & n = 2\\
        fib(n-1) + fib(n-2) & n > 2
      \end{array}
    \right.$
  \end{block}

  \pause
  \bigskip
  $\begin{array}{ccccccccc}
     F_1 & F_2 & F_3 & F_4 & F_5 & F_6 & F_7 & F_8 & \dots\\
     1   & 1   & 2   & 3   & 5   & 8   & 13  & 21  & \dots
  \end{array}$
\end{frame}

\begin{frame}
  \frametitle{Fibonacci Dizisi}

  \begin{teorem}
    $\sum_{i=1}^{n} {F_i}^2 = F_n \cdot F_{n+1}$
  \end{teorem}

  \pause
  \begin{proof}[Tanıt]
    $\begin{array}{llcl}
      n=2:   & \sum_{i=1}^{2} {F_i}^2   & = & {F_1}^2+{F_2}^2=1+1=1 \cdot 2=F_2 \cdot F_3
      \pause
      \medskip\\
      n=k:   & \sum_{i=1}^{k} {F_i}^2   & = & F_k \cdot F_{k+1}
      \pause
      \medskip\\
      n=k+1: & \sum_{i=1}^{k+1} {F_i}^2 & = & \sum_{i=1}^{k} {F_i}^2 + {F_{k+1}}^2
      \pause
      \smallskip\\
             &                          & = & F_k \cdot F_{k+1} + {F_{k+1}}^2
      \pause
      \smallskip\\
             &                          & = & F_{k+1} \cdot (F_k + F_{k+1})
      \pause
      \smallskip\\
             &                          & = & F_{k+1} \cdot F_{k+2}
    \end{array}$
  \end{proof}
\end{frame}

\begin{frame}
  \frametitle{Ackermann Fonksiyonu}

  \begin{block}{Ackermann fonksiyonu}
    $ack(x,y) = \left\{
      \begin{array}{ll}
        y+1                 & x = 0\\
        ack(x-1, 1)         & y = 0\\
        ack(x-1,ack(x,y-1)) & x > 0 \wedge y > 0
      \end{array}
    \right.$
  \end{block}
\end{frame}

\subsection*{Kaynaklar}

\begin{frame}
  \frametitle{Kaynaklar}

  \begin{block}{Grimaldi}
    \begin{itemize}
      \item Chapter 5: Relations and Functions
      \begin{itemize}
        \item 5.2. \alert{Functions: Plain and One-to-One}
        \item 5.3. \alert{Onto Functions: Stirling Numbers of the Second Kind}
        \item 5.5. \alert{The Pigeonhole Principle}
        \item 5.6. \alert{Function Composition and Inverse Functions}
      \end{itemize}
    \end{itemize}
  \end{block}

  \begin{block}{Yardımcı Kitap: O'Donnell, Hall, Page}
    \begin{itemize}
      \item Chapter 11: Functions
    \end{itemize}
  \end{block}
\end{frame}

\end{document}
