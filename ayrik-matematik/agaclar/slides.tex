% Copyright (c) 2001-2010
%       H. Turgut Uyar <uyar@itu.edu.tr>
%       Ayşegül Gençata Yayımlı <gencata@itu.edu.tr>
%       Emre Harmancı <harmanci@itu.edu.tr>
%
% Bu notlar "Creative Commons Attribution-NonCommercial-ShareAlike License" ile
% lisanslanmıştır. Yazarının açıkça belirtilmesi koşuluyla ve ticari olmayan
% amaçlarla kullanılabilir ve dağıtılabilir. Bu notlardan yola çıkılarak
% oluşturulacak çalışmaların da aynı lisansa bağlı olmaları gerekir.
%
% Lisans ile ilgili ayrıntılı bilgi almak için şu sayfaya başvurabilirsiniz:
% http://creativecommons.org/licenses/by-nc-sa/3.0/

\documentclass[dvipsnames]{beamer}

\usepackage{ae}
\usepackage[T1]{fontenc}
\usepackage[utf8]{inputenc}
\usepackage[turkish]{babel}
\setbeamertemplate{navigation symbols}{}

\mode<presentation>
{
  \usetheme{Rochester}
  \setbeamercovered{transparent}
}

\title{Ayrık Matematik}
\subtitle{Ağaçlar}

\author{H. Turgut Uyar \and Ayşegül Gençata Yayımlı \and Emre Harmancı}
\date{2001-2010}

\AtBeginSubsection[]
{
  \begin{frame}<beamer>
    \frametitle{Konular}
    \tableofcontents[currentsection,currentsubsection]
  \end{frame}
}

%\beamerdefaultoverlayspecification{<+->}

\theoremstyle{definition}
\newtheorem{tanim}[theorem]{Tanım}

\theoremstyle{example}
\newtheorem{ornek}[theorem]{Örnek}

\theoremstyle{plain}
\newtheorem{teorem}[theorem]{Teorem}

\pgfdeclareimage[width=2cm]{license}{../../license}

\pgfdeclareimage{agac}{agac}
\pgfdeclareimage{cevre}{cevre}
\pgfdeclareimage{dugumsayisi}{dugumsayisi}
\pgfdeclareimage[height=5cm]{koklu}{koklu}
\pgfdeclareimage[height=4cm]{kitap}{kitap}
\pgfdeclareimage{sozluk}{sozluk}
\pgfdeclareimage{islem1a}{islem1a}
\pgfdeclareimage{islem1b}{islem1b}
\pgfdeclareimage{islem2a}{islem2a}
\pgfdeclareimage{islem2b}{islem2b}
\pgfdeclareimage{islem3}{islem3}
\pgfdeclareimage{islem}{islem}
\pgfdeclareimage[width=3cm]{sirali}{sirali}
\pgfdeclareimage{siralioneksonek}{siralioneksonek}
\pgfdeclareimage{siralifarkli}{siralifarkli}
\pgfdeclareimage{siraliicek}{siraliicek}
\pgfdeclareimage{terazi1}{terazi1}
\pgfdeclareimage{terazi2}{terazi2}
\pgfdeclareimage[height=4cm]{kapsayan}{kapsayan}
\pgfdeclareimage[height=4cm]{kruskal1}{kruskal1}
\pgfdeclareimage[height=4cm]{kruskal2}{kruskal2}
\pgfdeclareimage[height=4cm]{kruskal3}{kruskal3}
\pgfdeclareimage[height=4cm]{kruskal4}{kruskal4}
\pgfdeclareimage[height=4cm]{kruskal5}{kruskal5}
\pgfdeclareimage[height=4cm]{kruskal}{kruskal}
\pgfdeclareimage[height=4cm]{prim1}{prim1}
\pgfdeclareimage[height=4cm]{prim2}{prim2}
\pgfdeclareimage[height=4cm]{prim3}{prim3}
\pgfdeclareimage[height=4cm]{prim4}{prim4}
\pgfdeclareimage[height=4cm]{prim5}{prim5}
\pgfdeclareimage[height=4cm]{prim}{prim}

\begin{document}

\begin{frame}
  \titlepage
\end{frame}

\begin{frame}
  \frametitle{Lisans}

  \pgfuseimage{license}\hfill
  \copyright 2001-2010 T. Uyar, A. Yayımlı, E. Harmancı

  \vfill
  \begin{tiny}
    You are free:
    \begin{itemize}
      \item to Share — to copy, distribute and transmit the work
      \item to Remix — to adapt the work
    \end{itemize}

    Under the following conditions:
    \begin{itemize}
      \item Attribution — You must attribute the work in the manner specified by
        the author or licensor (but not in any way that suggests that they
        endorse you or your use of the work).

      \item Noncommercial — You may not use this work for commercial purposes.

      \item Share Alike — If you alter, transform, or build upon this work, you
        may distribute the resulting work only under the same or similar license
        to this one.
    \end{itemize}
  \end{tiny}

  \vfill
  Legal code (the full license):\\
  \url{http://creativecommons.org/licenses/by-nc-sa/3.0/}
\end{frame}

\begin{frame}
  \frametitle{Konular}
  \tableofcontents
\end{frame}

\section{Ağaçlar}

\subsection{Giriş}

\begin{frame}
  \frametitle{Ağaç}

  \begin{tanim}
    \alert{ağaç}: $T=(V,E)$\\
    çevre içermeyen bağlı çizge
  \end{tanim}

  \pause
  \begin{itemize}
    \item bağlı bileşenleri ağaçlar olan çizge: \emph{orman}
  \end{itemize}
\end{frame}

\begin{frame}
  \frametitle{Ağaç Örnekleri}

  \begin{ornek}
    \begin{center}
      \pgfuseimage{agac}
    \end{center}
  \end{ornek}
\end{frame}

\begin{frame}
  \frametitle{Ağaç Teoremleri}

  \begin{teorem}
    Bir ağaçta herhangi iki ayrık düğüm arasında bir ve yalnız bir yol vardır.
  \end{teorem}

  \begin{itemize}
    \item bağlı olduğu için bir yol var
    \item birden fazla yol olsaydı:
    \begin{center}
      \pgfuseimage{cevre}
    \end{center}
  \end{itemize}
\end{frame}

\begin{frame}
  \frametitle{Ağaç Teoremleri}

  \begin{teorem}
    $T = (V, E)$ ağacında: $|V| = |E| + 1$
  \end{teorem}

  \begin{itemize}
    \item tanıt yöntemi: ayrıt sayısı üzerinden tümevarım
  \end{itemize}
\end{frame}

\begin{frame}
  \frametitle{Ağaç Teoremleri}

  \begin{block}{Tanıt: Taban adımı.}
    \begin{itemize}
      \item $|E|=0 \Rightarrow |V|=1$
      \item $|E|=1 \Rightarrow |V|=2$
      \item $|E|=2 \Rightarrow |V|=3$

      \pause
      \medskip
      \item $|E| \leq k$ için doğru olduğu varsayılsın
    \end{itemize}
  \end{block}
\end{frame}

\begin{frame}
  \frametitle{Ağaç Teoremleri}

  \begin{proof}[Tanıt: Tümevarım adımı]
    \begin{itemize}
      \item $|E|=k+1$
    \end{itemize}

    \begin{columns}[t]
      \column{.4\textwidth}
      \begin{center}
        \pgfuseimage{dugumsayisi}
      \end{center}

      \pause
      \column{.55\textwidth}
      \begin{itemize}
        \item $(y,z)$ çıkarılsın:\\
          $T_1=(V_1,E_1)$, $T_2=(V_2,E_2)$
      \end{itemize}
      \pause
      \begin{eqnarray*}
        |V| & = & |V_1|+|V_2|\\\pause
            & = & |E_1|+1+|E_2|+1\\\pause
            & = & (|E_1|+|E_2|+1)+1\\\pause
            & = & |E|+1
      \end{eqnarray*}
    \end{columns}
  \end{proof}
\end{frame}

\begin{frame}
  \frametitle{Ağaç Teoremleri}

  \begin{teorem}
    Bir ağaçta kertesi 1 olan en az iki düğüm vardır.
  \end{teorem}

  \pause
  \begin{proof}[Tanıt]
    \begin{itemize}
      \item $2 |E| = \sum_{v \in V} d_v$

      \pause
      \item kertesi 1 olan tek bir düğüm olduğunu varsayalım:\\
        \pause
        $\Rightarrow 2 |E| \geq 2 (|V| - 1) + 1$\\
        \pause
        $\Rightarrow 2 |E| \geq 2 |V| - 1$\\
        \pause
        $\Rightarrow |E| \geq |V| - \frac{1}{2}$
        \pause
        $> |V| - 1$ \alert{çelişki}
    \end{itemize}
  \end{proof}
\end{frame}

\begin{frame}
  \frametitle{Ağaç Teoremleri}

  \begin{teorem}
    $T = (V,E) \wedge |V| \geq 2$ ise aşağıdaki önermeler eşdeğerlidir:

    \begin{enumerate}
      \item $T$ bir ağaçtır (bağlıdır ve çevre içermez)
      \item her düğüm çifti arasında bir ve yalnız bir yol vardır
      \item $T$ bağlıdır ama herhangi bir ayrıt çıkarılırsa bu özelliğini
        yitirir
      \item $T$ çevre içermez ama herhangi iki düğüm arasına bir ayrıt eklenirse
        bu özelliğini yitirir
    \end{enumerate}
  \end{teorem}

  \pause
  \begin{itemize}
    \item tanıt yöntemi:
      $1 \Rightarrow 2 \Rightarrow 3 \Rightarrow 4 \Rightarrow 1$\\
      \hyperlink{theoremset2}{\beamergotobutton{tanıtı atla}}
  \end{itemize}
\end{frame}

\begin{frame}
  \frametitle{Ağaç Teoremleri}

  \begin{proof}[Tanıt: $1 \Rightarrow 2$]
    $T$ bağlıdır ve çevre içermez\\
    $\Rightarrow$ her düğüm çifti arasında bir ve yalnız bir yol vardır

    \pause
    \begin{itemize}
      \item varsayıma göre her düğüm çifti arasında bir yol vardır
      \item birden fazla yol olsaydı çevre olurdu
    \end{itemize}
  \end{proof}
\end{frame}

\begin{frame}
  \frametitle{Ağaç Teoremleri}

  \begin{proof}[Tanıt: $2 \Rightarrow 3$]
    her düğüm çifti arasında bir ve yalnız bir yol vardır\\
    $\Rightarrow$ $T$ bağlıdır ama herhangi bir ayrıt çıkarılırsa bu özelliğini
    yitirir

    \pause
    \begin{itemize}
      \item varsayıma göre yani her düğüm çifti arasında bir yol vardır yani
        $T$ bağlıdır

      \pause
      \item $e = (u,v)$ ayrıtını çizgeden çıkaralım:\\
        $e$ ayrıtı $u$ ile $v$ düğümleri arasındaki tek yoldur, çıkarılırsa $T$
        bağlı olmaz
    \end{itemize}
  \end{proof}
\end{frame}

\begin{frame}
  \frametitle{Ağaç Teoremleri}

  \begin{block}{Tanıt: $3 \Rightarrow 4$}
    $T$ bağlıdır ama herhangi bir ayrıt çıkarılırsa bu özelliğini yitirir\\
    $\Rightarrow$ $T$ çevre içermez ama herhangi iki düğüm arasına bir ayrıt
    eklenirse bu özelliğini yitirir

    \pause
    \begin{enumerate}
      \item $T$'nin çevre içermediğinin tanıtı
      \item ayrıt eklemeyle çevre oluştuğunun tanıtı
      \item ayrıt eklemeyle oluşan çevrenin tek olduğunun tanıtı
    \end{enumerate}
  \end{block}
\end{frame}

\begin{frame}
  \frametitle{Ağaç Teoremleri}

  \begin{block}{Tanıt: $3 \Rightarrow 4$}
    $T$ bağlıdır ama herhangi bir ayrıt çıkarılırsa bu özelliğini yitirir\\
    $\Rightarrow$ $T$ çevre içermez ama herhangi iki düğüm arasına bir ayrıt
      eklenirse bu özelliğini yitirir
  \end{block}

  \pause
  \begin{proof}[Çevre içermediğinin tanıtı]
    \begin{itemize}
      \item $T$'de bir $C$ çevresi olsun ve $e = (u,v) \in C$ olsun

      \pause
      \item varsayıma göre $T$ bağlı ama $T-e$ değil\\
        $\Rightarrow$ $T-e$ çizgesinde $u$ ile $v$ ayrı bileşenlerde

      \pause
      \item oysa $u$ ile $v$ arasında $C-e$ yolu var
    \end{itemize}
  \end{proof}
\end{frame}

\begin{frame}
  \frametitle{Ağaç Teoremleri}

  \begin{block}{Tanıt: $3 \Rightarrow 4$}
    $T$ bağlıdır ama herhangi bir ayrıt çıkarılırsa bu özelliğini yitirir\\
    $\Rightarrow$ $T$ çevre içermez ama herhangi iki düğüm arasına bir ayrıt
    eklenirse bu özelliğini yitirir
  \end{block}

  \begin{proof}[Ayrıt eklemeyle çevre oluştuğunun tanıtı]
    \begin{itemize}
      \pause
      \item $T$'ye $e = (u,v)$ ayrıtı eklensin

      \pause
      \item varsayıma göre $u$ ile $v$ arasında bir yol var\\
        $\Rightarrow$ $e$ ayrıtı ikinci bir yol oluşturur\\
        $\Rightarrow$ çevre oluşur
    \end{itemize}
  \end{proof}
\end{frame}

\begin{frame}
  \frametitle{Ağaç Teoremleri}

  \begin{block}{Tanıt: $3 \Rightarrow 4$}
    $T$ bağlıdır ama herhangi bir ayrıt çıkarılırsa bu özelliğini yitirir\\
    $\Rightarrow$ $T$ çevre içermez ama herhangi iki düğüm arasına bir ayrıt
    eklenirse bu özelliğini yitirir
  \end{block}

  \pause
  \begin{proof}[Oluşan çevrenin tek olduğunun tanıtı]
    \begin{itemize}
      \item eklenen ayrıt: $e = (u,v)$

      \pause
      \item oluşan çevre $C = P \cdot e$ çevresi olsun

      \pause
      \item ikinci bir çevre daha oluştuğunu varsayalım:
        $C' = P' \cdot e$\\
        \pause
        $\Rightarrow$ $P$ ve $P'$ yolları çevre oluşturur
    \end{itemize}
  \end{proof}
\end{frame}

\begin{frame}
  \frametitle{Ağaç Teoremleri}

  \begin{proof}[Tanıt: $4 \Rightarrow 1$]
    $T$ çevre içermez ama herhangi iki düğüm arasına bir ayrıt eklenirse bu
    özelliğini yitirir\\
    $\Rightarrow$ $T$ bağlıdır ve çevre içermez

    \pause
    \begin{itemize}
      \item varsayıma göre $T$ çevre içermez

      \pause
      \item herhangi bir $e = (u,v)$ ayrıtı eklendiğinde çevre oluşuyor\\
        $\Rightarrow$ $u$ ile $v$ arasında yol var\\
        $\Rightarrow$ $T$ bağlı
    \end{itemize}
  \end{proof}
\end{frame}

\begin{frame}[label=theoremset2]
  \frametitle{Ağaç Teoremleri}

  \begin{teorem}
    $T = (V,E) \wedge |V| \geq 2$ ise aşağıdaki önermeler eşdeğerlidir:

    \begin{enumerate}
      \item $T$ bir ağaçtır (bağlıdır ve çevre içermez)
      \item $T$ bağlıdır $\wedge$ $|E| = |V| - 1$
      \item $T$ çevre içermez $\wedge$ $|E| = |V| - 1$
    \end{enumerate}
  \end{teorem}
\end{frame}

\subsection{Köklü Ağaçlar}

\begin{frame}
  \frametitle{Köklü Ağaç}

  \begin{itemize}
    \item düğümler arasında hiyerarşi

    \pause
    \item ayrıtlarda doğal yön $\Rightarrow$ giriş ve çıkış kerteleri
    \begin{itemize}
      \item giriş kertesi 0~olan (hiyerarşinin tepesindeki) düğüm: \alert{kök}
      \item çıkış kertesi 0~olan düğümler: \alert{yaprak}
      \item kök ve yapraklar dışında kalan düğümler: \alert{içdüğüm}
    \end{itemize}
  \end{itemize}
\end{frame}

\begin{frame}
  \frametitle{Düğüm Düzeyleri}

  \begin{tanim}
    \alert{düzey}:\\
    köke olan uzaklık

    \pause
    \begin{itemize}
      \item \alert{anne}: kökle arasındaki yolda kendisinden bir önceki düzeyde
        bulunan düğüm

      \item \alert{çocuk}: bir sonraki düzeydeki komşu düğümler
    \end{itemize}
  \end{tanim}
\end{frame}

\begin{frame}
  \frametitle{Köklü Ağaç Örneği}

  \begin{ornek}
    \begin{columns}
      \column{.4\textwidth}
      \begin{center}
        \pgfuseimage{koklu}
      \end{center}

      \pause
      \column{.58\textwidth}
      \begin{itemize}
        \item kök: $r$
        \item yapraklar: $x ~ y ~ z ~ u ~ v$
        \item içdüğümler: $p ~ n ~ t ~ s ~ q ~ w$
        \item $y$ düğümünün annesi: $w$\\
          $w$ düğümünün çocukları: $y$ ve $z$\\
      \end{itemize}
    \end{columns}
  \end{ornek}
\end{frame}

\begin{frame}
  \frametitle{Köklü Ağaç Örneği}

  \begin{ornek}[kitap düzeni]
    \begin{columns}
      \column{.65\textwidth}
      \begin{center}
        \pgfuseimage{kitap}
      \end{center}

      \pause
      \column{.33\textwidth}
      Kitap
      \begin{itemize}
        \item B1
        \begin{itemize}
          \item B1.1
          \item B1.2
        \end{itemize}
        \item B2
        \item B3
        \begin{itemize}
          \item B3.1
          \item B3.2
          \begin{itemize}
            \item B3.2.1
            \item B3.2.2
          \end{itemize}
          \item B3.3
        \end{itemize}
      \end{itemize}
    \end{columns}
  \end{ornek}
\end{frame}

\begin{frame}
  \frametitle{Sıralı Köklü Ağaç}

  \begin{itemize}
    \item düğümler soldan sağa doğru sıralı

    \pause
    \medskip
    \item \alert{evrensel adresleme sistemi}
    \begin{itemize}
      \item köke $0$ adresini ver
      \item 1. düzeydeki düğümlere soldan sağa doğru sırayla $1,2,3,\dots$
        adreslerini ver
      \item $v$ düğümünün adresi $a$ ise, $v$ düğümünün çocuklarına soldan sağa
        doğru sırayla $a.1,a.2,a.3,\dots$ adreslerini ver
    \end{itemize}
  \end{itemize}
\end{frame}

\begin{frame}
  \frametitle{Sözlük Sırası}

  \begin{itemize}
    \item $b$ ve $c$ iki adres olsun
  \end{itemize}

  \begin{tanim}
    \alert{$b < c$} olması için:
    \begin{enumerate}
      \item $b=a_1.a_2. \dots .a_m$\\
        $c=a_1.a_2. \dots .a_m.a_{m+1} \dots a_n$
      \pause
      \item $b=a_1.a_2. \dots .a_m.x_1 \dots y$\\
        $c=a_1.a_2. \dots .a_m.x_2 \dots z$\\
        $x_1 < x_2$
    \end{enumerate}
  \end{tanim}
\end{frame}

\begin{frame}
  \frametitle{Sözlük Sırası Örneği}

  \begin{ornek}
    \begin{columns}
      \column{.57\textwidth}
      \begin{center}
        \pgfuseimage{sozluk}
      \end{center}

      \pause
      \column{.4\textwidth}
      \begin{itemize}
        \item 0 - 1 - 1.1 - 1.2\\
          - 1.2.1 - 1.2.2 - 1.2.3\\
          - 1.2.3.1 - 1.2.3.2\\
          - 1.3 - 1.4 - 2\\
          - 2.1 - 2.2 - 2.2.1\\
          - 3 - 3.1 - 3.2
      \end{itemize}
    \end{columns}
  \end{ornek}
\end{frame}

\subsection{İkili Ağaçlar}

\begin{frame}
  \frametitle{İkili Ağaçlar}

  \begin{tanim}
    \alert{ikili ağaç}:\\
    $\forall v \in V~{d_v}^o \in \{0,1,2\}$
  \end{tanim}

  \pause
  \begin{tanim}
    \alert{tam ikili ağaç}:\\
    $\forall v \in V~{d_v}^o \in \{0,2\}$
  \end{tanim}
\end{frame}

\begin{frame}
  \frametitle{İşlem Ağacı}

  \begin{itemize}
    \item ikili işlem tam ikili ağaçla temsil edilebilir
    \begin{itemize}
      \item kökte işleç, çocuklarda işlenenler
    \end{itemize}

    \pause
    \medskip
    \item her işlem ikili ağaçla temsil edilebilir
    \begin{itemize}
      \item içdüğümlerde işleçler, yapraklara değişkenler ve değerler
      \item \emph{tam ikili ağaç olmayabilir}
    \end{itemize}
  \end{itemize}
\end{frame}

\begin{frame}
  \frametitle{İşlem Ağacı Örnekleri}

  \begin{columns}[t]
    \column{.5\textwidth}
    \begin{ornek}[$7-a$]
      \begin{center}
        \pgfuseimage{islem1a}
      \end{center}
    \end{ornek}

    \column{.5\textwidth}
    \begin{ornek}[$a+b$]
      \begin{center}
        \pgfuseimage{islem1b}
      \end{center}
    \end{ornek}
  \end{columns}
\end{frame}

\begin{frame}
  \frametitle{İşlem Ağacı Örnekleri}

  \begin{columns}[t]
    \column{.5\textwidth}
    \begin{ornek}[$(7-a)/5$]
      \begin{center}
        \pgfuseimage{islem2a}
      \end{center}
    \end{ornek}

    \column{.5\textwidth}
    \begin{ornek}[$(a+b) \uparrow 3$]
      \begin{center}
        \pgfuseimage{islem2b}
      \end{center}
    \end{ornek}
  \end{columns}
\end{frame}

\begin{frame}
  \frametitle{İşlem Ağacı Örnekleri}

  \begin{ornek}[$((7-a)/5)*((a+b) \uparrow 3)$]
    \begin{center}
      \pgfuseimage{islem3}
    \end{center}
  \end{ornek}
\end{frame}

\begin{frame}
  \frametitle{İşlem Ağacı Örnekleri}

  \begin{ornek}[$t+(u*v)/(w+x-y \uparrow z)$]
    \begin{center}
      \pgfuseimage{islem}
    \end{center}
  \end{ornek}
\end{frame}

\begin{frame}
  \frametitle{İşlem Ağacında Geçişler}

  \begin{enumerate}
    \item \alert{içek geçişi}: sol altağacı tara, köke uğra, sağ altağacı tara

    \pause
    \medskip
    \item \alert{önek geçişi}: köke uğra, sol altağacı tara, sağ altağacı tara

    \pause
    \medskip
    \item \alert{sonek geçişi}: sol altağacı tara, sağ altağacı tara, köke uğra
    \begin{itemize}
      \item \emph{ters Polonyalı gösterilimi}
    \end{itemize}
  \end{enumerate}
\end{frame}

\begin{frame}
  \frametitle{Önek Geçişi Örneği}

  \begin{ornek}
    \begin{columns}
      \column{.4\textwidth}
      \begin{center}
        \pgfuseimage{islem}
      \end{center}

      \pause
      \column{.6\textwidth}
      $+ ~ t ~ / ~ * ~ u ~ v ~ + ~ w ~ - ~ x ~ \uparrow ~ y ~ z$
    \end{columns}
  \end{ornek}
\end{frame}

\begin{frame}
  \frametitle{İçek Geçişi Örneği}

  \begin{ornek}
    \begin{columns}
      \column{.4\textwidth}
      \begin{center}
        \pgfuseimage{islem}
      \end{center}

      \pause
      \column{.6\textwidth}
      $t ~ + ~ u ~ * ~ v ~ / ~ w ~ + ~ x ~ - ~ y ~ \uparrow ~ z$
    \end{columns}
  \end{ornek}
\end{frame}

\begin{frame}
  \frametitle{Sonek Geçişi Örneği}

  \begin{ornek}
    \begin{columns}
      \column{.4\textwidth}
      \begin{center}
        \pgfuseimage{islem}
      \end{center}

      \pause
      \column{.6\textwidth}
      $t ~ u ~ v ~ * ~ w ~ x ~ y ~ z ~ \uparrow ~ - ~ + ~ / ~ +$
    \end{columns}
  \end{ornek}
\end{frame}

\begin{frame}
  \frametitle{İşlem Ağacının Değerlendirilmesi}

  \begin{itemize}
    \item işlem ağacında öncelikler:
    \begin{itemize}
      \item içek geçişi parantez gerektirir
      \item önek ve sonek parantez gerektirmez
    \end{itemize}
  \end{itemize}
\end{frame}

\begin{frame}
  \frametitle{İşlem Ağacı Değerlendirme Örneği}

  \begin{ornek}[$+~t~/~*~u~v~+~w~-~x~\uparrow~y~z$]
    \[
      \begin{array}{ccccccccccccc}
        + & 4 & / & * & 2 & 3 & + & 1 & - & 9 & \uparrow & 2 & 3\\\pause
        + & 4 & / & * & 2 & 3 & + & 1 & - & 9 & 8        &   &  \\\pause
        + & 4 & / & * & 2 & 3 & + & 1 & 1 &   &          &   &  \\\pause
        + & 4 & / & * & 2 & 3 & 2 &   &   &   &          &   &  \\\pause
        + & 4 & / & 6 &   &   & 2 &   &   &   &          &   &  \\\pause
        + & 4 & 3 &   &   &   &   &   &   &   &          &   &  \\\pause
        7 &   &   &   &   &   &   &   &   &   &          &   &
      \end{array}
    \]
  \end{ornek}
\end{frame}

\begin{frame}
  \frametitle{Sıralı Ağaçlar}

  \begin{columns}
    \column{.4\textwidth}
    \begin{center}
      \pgfuseimage{sirali}
    \end{center}

    \column{.6\textwidth}
    \begin{itemize}
      \item önek: $r, T_1, T_2, T_3, \ldots, T_k$
      \item sonek: $T_1, T_2, T_3, \ldots, T_k, r$
    \end{itemize}
  \end{columns}
\end{frame}

\begin{frame}
  \frametitle{Sıralı Geçiş Örneği}

  \begin{ornek}
    \begin{columns}
      \column{.5\textwidth}
      \begin{center}
        \pgfuseimage{siralioneksonek}
      \end{center}

      \pause
      \column{.48\textwidth}
      \begin{itemize}
        \item önek:\\
          1, 2, 5, 11, 12, 13, 14, 3, 6, 7, 4, 8, 9, 10, 15, 16, 17
        \item sonek:\\
          11, 12, 13, 14, 5, 2, 6, 7, 3, 8, 9, 15, 16, 17, 10, 4, 1
      \end{itemize}
    \end{columns}
  \end{ornek}
\end{frame}

\begin{frame}
  \frametitle{Sıralı Ağaçlar}

  \begin{columns}
    \column{.4\textwidth}
    \begin{center}
      \pgfuseimage{sirali}
    \end{center}

    \column{.6\textwidth}
    \begin{itemize}
      \item içek: $T_1, r, T_2, T_3, \ldots, T_k$
    \end{itemize}
  \end{columns}
\end{frame}

\begin{frame}
  \frametitle{Sıralı Ağaçlar}

  \begin{ornek}
    \begin{itemize}
      \item aşağıdaki iki ağaç farklı
    \end{itemize}

    \begin{center}
      \pgfuseimage{siralifarkli}
    \end{center}
  \end{ornek}
\end{frame}

\begin{frame}
  \frametitle{İçek Geçişi Örneği}

  \begin{ornek}
    \begin{columns}
      \column{.5\textwidth}
      \begin{center}
        \pgfuseimage{siraliicek}
      \end{center}

      \pause
      \column{.5\textwidth}
      \begin{itemize}
        \item p, j, q, f, c, k, g, a, d, r, b, h, s, m, e, i, t, n, u
      \end{itemize}
    \end{columns}
  \end{ornek}
\end{frame}

\subsection{Çizgelerde Arama}

\begin{frame}
  \frametitle{Çizgelerde Arama}

  \begin{itemize}
    \item $G=(V,E)$ çizgesinin $v_1$ düğümünü kök alarak kapsayan ağacının
      bulunması
    \begin{itemize}
      \item derinlemesine
      \item enlemesine
    \end{itemize}
  \end{itemize}
\end{frame}

\begin{frame}
  \frametitle{Derinlemesine Arama}

  \begin{enumerate}
    \item $v \leftarrow v_1, T=\emptyset$, $D=\{v_1\}$

    \pause
    \item $2 \leq i \leq |V|$ içinde $(v,v_i) \in E$ ve $v_i \notin D$
      olacak şekilde en küçük $i$'yi bul
      \begin{itemize}
        \item böyle bir $i$ yoksa: 3. adıma git
        \item varsa: $T=T \cup \{(v,v_i)\}$, $D=D \cup \{v_i\}$,
          $v \leftarrow v_i$, 2. adıma git
      \end{itemize}

    \pause
    \item $v=v_1$ ise sonuç $T$

    \pause
    \item $v \neq v_1$ ise $v \leftarrow parent(v)$, 2. adıma git
  \end{enumerate}
\end{frame}

\begin{frame}
  \frametitle{Enlemesine Arama}

  \begin{enumerate}
    \item $T=\emptyset$, $D=\{v_1\}$, $Q=(v_1)$

    \pause
    \item $Q$ boş ise: sonuç $T$
    \item $Q$ boş değilse: $v \leftarrow front(Q)$, $Q \leftarrow Q - v$\\
      $2 \leq i \leq |V|$ için $(v,v_i) \in E$ ayrıtlarına bak:
    \begin{itemize}
      \item $v_i \notin D$ ise: $Q = Q + v_i$, $T = T \cup \{(v,v_i)\}$,
        $D=D \cup \{v_i\}$
       \item 3. adıma git
    \end{itemize}
  \end{enumerate}
\end{frame}

\section{Özel Ağaçlar}

\subsection{Düzenli Ağaçlar}

\begin{frame}
  \frametitle{Düzenli Ağaç}

  \begin{tanim}
    \alert{$m$'li ağaç}:\\
    yapraklar dışındaki bütün düğümlerin çıkış kerteleri $m$
  \end{tanim}
\end{frame}

\begin{frame}
  \frametitle{Düzenli Ağaç Teoremleri}

  \begin{teorem}
    bir $m$'li ağaçta

    \begin{itemize}
      \item düğüm sayısı $n$
      \item yaprak sayısı $l$
      \item içdüğüm sayısı (kök dahil) $i$
    \end{itemize}

    ise

    \begin{itemize}
      \item $n = m \cdot i + 1$

      \pause
      \item  $l = n - i = \pause m \cdot i + 1 - i
        \pause = (m - 1) \cdot i + 1$

      \pause
      \[
        i = \frac{l - 1}{m - 1}
      \]
    \end{itemize}
  \end{teorem}
\end{frame}

\begin{frame}
  \frametitle{Düzenli Ağaç Örnekleri}

  \begin{ornek}
    27~oyuncunun katıldığı bir tenis turnuvasında kaç maç oynanır?

    \pause
    \bigskip
    \begin{itemize}
      \item her oyuncu bir yaprak: $l = 27$
      \item her maç bir içdüğüm: $m = 2$

      \pause
      \item maç sayısı: $i = \frac{l - 1}{m - 1} = \frac{27 - 1}{2 - 1} = 26$
    \end{itemize}
  \end{ornek}
\end{frame}

\begin{frame}
  \frametitle{Düzenli Ağaç Örnekleri}

  \begin{ornek}
    25~adet elektrikli aygıtı 4'lü uzatmalarla tek bir prize bağlamak için\\
    kaç uzatma gerekir?

    \pause
    \bigskip
    \begin{itemize}
      \item her aygıt bir yaprak: $l = 25$
      \item her uzatma bir içdüğüm: $m = 4$

      \pause
      \item uzatma sayısı: $i = \frac{l - 1}{m - 1} = \frac{25 - 1}{4 - 1} = 8$
    \end{itemize}
  \end{ornek}
\end{frame}

\subsection{Karar Ağaçları}

\begin{frame}
  \frametitle{Karar Ağaçları}

  \begin{ornek}[sahte madeni para problemi]
    \begin{itemize}
      \item 8 madeni paranın biri sahte (daha ağır)
      \item bir teraziyle en az sayıda tartmayla sahteyi bulmak
    \end{itemize}
  \end{ornek}
\end{frame}

\begin{frame}
  \frametitle{Karar Ağaçları}

  \begin{ornek}[3 tartmada bulma]
    \begin{center}
      \pgfuseimage{terazi1}
    \end{center}
  \end{ornek}
\end{frame}

\begin{frame}
  \frametitle{Karar Ağaçları}

  \begin{ornek}[2 tartmada bulma]
    \begin{center}
      \pgfuseimage{terazi2}
    \end{center}
  \end{ornek}
\end{frame}

\section{Ağaç Problemleri}

\subsection{En Hafif Kapsayan Ağaç}

\begin{frame}
  \frametitle{Kapsayan Ağaç}

  \begin{tanim}
    \alert{kapsayan ağaç}:\\
    bir çizgenin bütün düğümlerini içeren, ağaç özellikleri taşıyan bir
    altçizgesi
  \end{tanim}

  \pause
  \begin{tanim}
    \alert{en hafif kapsayan ağaç}:\\
    ayrıt ağırlıklarının toplamının en az olduğu kapsayan ağaç
  \end{tanim}
\end{frame}

\begin{frame}
  \frametitle{Kruskal Algoritması}

  \begin{block}{Kruskal algoritması}
    \begin{enumerate}
      \item $i \leftarrow 1$, $e_1 \in E$, $wt(e_1)$ minimum

      \pause
      \item $1 \leq i \leq n-2$ için:\\
        şu ana kadar seçilen ayrıtlar $e_1,e_2,\dots,e_i$ ise, kalan ayrıtlardan
        öyle bir $e_{i+1}$ seç ki:
      \begin{itemize}
        \item $wt(e_{i+1})$ minimum
        \item $e_1,e_2,\dots,e_i,e_{i+1}$ altçizgesi çevre içermiyor
      \end{itemize}

      \pause
      \item $i \leftarrow i+1$
      \begin{itemize}
        \item $i=n-1$ $\Rightarrow$ $e_1,e_2,\dots,e_{n-1}$ ayrıtlarından oluşan
          $G$ altçizgesi bir en hafif kapsayan ağaçtır
        \item $i<n-1$ $\Rightarrow$ 2.~adıma git
      \end{itemize}
    \end{enumerate}
  \end{block}
\end{frame}

\begin{frame}
  \frametitle{Kruskal Algoritması Örneği}

  \begin{ornek}[başlangıç]
    \begin{columns}
      \column{.4\textwidth}
      \begin{center}
        \pgfuseimage{kapsayan}
      \end{center}

      \pause
      \column{.6\textwidth}
      \begin{itemize}
        \item $i \leftarrow 1$
        \item en düşük ağırlık: $1$\\
          $(e,g)$

        \pause
        \item $T = \{ (e,g) \}$
      \end{itemize}
    \end{columns}
  \end{ornek}
\end{frame}

\begin{frame}
  \frametitle{Kruskal Algoritması Örneği}

  \begin{ornek}[$1 < 6$]
    \begin{columns}
      \column{.4\textwidth}
      \begin{center}
        \pgfuseimage{kruskal1}
      \end{center}

      \pause
      \column{.6\textwidth}
      \begin{itemize}
        \item en düşük ağırlık: $2$\\
          $(d,e), (d,f), (f,g)$

        \pause
        \item $T = \{ (e,g), (d,f) \}$
        \item $i \leftarrow 2$
      \end{itemize}
    \end{columns}
  \end{ornek}
\end{frame}

\begin{frame}
  \frametitle{Kruskal Algoritması Örneği}

  \begin{ornek}[$2 < 6$]
    \begin{columns}
      \column{.4\textwidth}
      \begin{center}
        \pgfuseimage{kruskal2}
      \end{center}

      \pause
      \column{.6\textwidth}
      \begin{itemize}
        \item en düşük ağırlık: $2$\\
          $(d,e), (f,g)$

        \pause
        \item $T = \{ (e,g), (d,f), (d,e) \}$
        \item $i \leftarrow 3$
      \end{itemize}
    \end{columns}
  \end{ornek}
\end{frame}

\begin{frame}
  \frametitle{Kruskal Algoritması Örneği}

  \begin{ornek}[$3 < 6$]
    \begin{columns}
      \column{.4\textwidth}
      \begin{center}
        \pgfuseimage{kruskal3}
      \end{center}

      \pause
      \column{.6\textwidth}
      \begin{itemize}
        \item en düşük ağırlık: $2$\\
          $(f,g)$ çevre oluşturuyor

        \pause
        \item en düşük ağırlık: $3$\\
          $(c,e), (c,g), (d,g)$\\
          $(d,g)$ çevre oluşturuyor

        \pause
        \item $T = \{ (e,g), (d,f), (d,e), (c,e) \}$
        \item $i \leftarrow 4$
      \end{itemize}
    \end{columns}
  \end{ornek}
\end{frame}

\begin{frame}
  \frametitle{Kruskal Algoritması Örneği}

  \begin{ornek}[$4 < 6$]
    \begin{columns}
      \column{.4\textwidth}
      \begin{center}
        \pgfuseimage{kruskal4}
      \end{center}

      \pause
      \column{.6\textwidth}
      \begin{itemize}
        \item $T = \{$\\
          $~~(e,g), (d,f), (d,e),$\\
          $~~(c,e), (b,e)$\\
          $\}$
        \item $i \leftarrow 5$
      \end{itemize}
    \end{columns}
  \end{ornek}
\end{frame}

\begin{frame}
  \frametitle{Kruskal Algoritması Örneği}

  \begin{ornek}[$5 < 6$]
    \begin{columns}
      \column{.4\textwidth}
      \begin{center}
        \pgfuseimage{kruskal5}
      \end{center}

      \pause
      \column{.6\textwidth}
      \begin{itemize}
        \item $T = \{$\\
          $~~(e,g), (d,f), (d,e),$\\
          $~~(c,e), (b,e), (a,b)$\\
          $\}$
        \item $i \leftarrow 6$
      \end{itemize}
    \end{columns}
  \end{ornek}
\end{frame}

\begin{frame}
  \frametitle{Kruskal Algoritması Örneği}

  \begin{ornek}[$6 \nless 6$]
    \begin{columns}
      \column{.4\textwidth}
      \begin{center}
        \pgfuseimage{kruskal}
      \end{center}

      \column{.6\textwidth}
      \begin{itemize}
        \item toplam ağırlık: $17$
      \end{itemize}
    \end{columns}
  \end{ornek}
\end{frame}

\begin{frame}
  \frametitle{Prim Algoritması}

  \begin{block}{Prim algoritması}
    \begin{enumerate}
      \item $i \leftarrow 1$, $v_1 \in V$, $P=\{v_1\}$, $N=V-\{v_1\}$,
        $T=\emptyset$

      \pause
      \item $1 \leq i \leq n-1$ için:\\
        $P=\{v_1,v_2,\dots,v_i\}$, $T=\{e_1,e_2,\dots,e_{i-1}\}$, $N=V-P$.\\
        öyle bir $v_{i+1} \in N$ düğümü seç ki, bir $x \in P$ düğümü için
        $e=(x,v_{i+1}) \notin T$, $wt(e)$ minimum\\
        $P \leftarrow P+\{v_{i+1}\}$, $N \leftarrow N-\{v_{i+1}\}$,
        $T \leftarrow T+\{e\}$

      \pause
      \item $i \leftarrow i+1$
      \begin{itemize}
        \item $i=n$ $\Rightarrow$ $e_1,e_2,\dots,e_{n-1}$ ayrıtlarından oluşan
          $G$ altçizgesi bir en hafif kapsayan ağaçtır
        \item $i<n$ $\Rightarrow$ 2.~adıma git
      \end{itemize}
    \end{enumerate}
  \end{block}
\end{frame}

\begin{frame}
  \frametitle{Prim Algoritması Örneği}

  \begin{ornek}[başlangıç]
    \begin{columns}
      \column{.4\textwidth}
      \begin{center}
        \pgfuseimage{kapsayan}
      \end{center}

      \pause
      \column{.6\textwidth}
      \begin{itemize}
        \item $i \leftarrow 1$
        \item $P = \{ a \}$
        \item $N = \{ b, c, d, e, f, g \}$
        \item $T = \emptyset$
      \end{itemize}
    \end{columns}
  \end{ornek}
\end{frame}

\begin{frame}
  \frametitle{Prim Algoritması Örneği}

  \begin{ornek}[$1 < 7$]
    \begin{columns}
      \column{.4\textwidth}
      \begin{center}
        \pgfuseimage{kapsayan}
      \end{center}

      \pause
      \column{.6\textwidth}
      \begin{itemize}
        \item $T = \{ (a,b) \}$
        \item $P = \{ a, b \}$
        \item $N = \{ c, d, e, f, g \}$
        \item $i \leftarrow 2$
      \end{itemize}
    \end{columns}
  \end{ornek}
\end{frame}

\begin{frame}
  \frametitle{Prim Algoritması Örneği}

  \begin{ornek}[$2 < 7$]
    \begin{columns}
      \column{.4\textwidth}
      \begin{center}
        \pgfuseimage{prim1}
      \end{center}

      \pause
      \column{.6\textwidth}
      \begin{itemize}
        \item $T = \{ (a,b), (b,e) \}$
        \item $P = \{ a, b, e \}$
        \item $N = \{ c, d, f, g \}$
        \item $i \leftarrow 3$
      \end{itemize}
    \end{columns}
  \end{ornek}
\end{frame}

\begin{frame}
  \frametitle{Prim Algoritması Örneği}

  \begin{ornek}[$3 < 7$]
    \begin{columns}
      \column{.4\textwidth}
      \begin{center}
        \pgfuseimage{prim2}
      \end{center}

      \pause
      \column{.6\textwidth}
      \begin{itemize}
        \item $T = \{ (a,b), (b,e), (e,g) \}$
        \item $P = \{ a, b, e, g \}$
        \item $N = \{ c, d, f \}$
        \item $i \leftarrow 4$
      \end{itemize}
    \end{columns}
  \end{ornek}
\end{frame}

\begin{frame}
  \frametitle{Prim Algoritması Örneği}

  \begin{ornek}[$4 < 7$]
    \begin{columns}
      \column{.4\textwidth}
      \begin{center}
        \pgfuseimage{prim3}
      \end{center}

      \pause
      \column{.6\textwidth}
      \begin{itemize}
        \item $T = \{ (a,b), (b,e), (e,g), (d,e) \}$
        \item $P = \{ a, b, e, g, d \}$
        \item $N = \{ c, f \}$
        \item $i \leftarrow 5$
      \end{itemize}
    \end{columns}
  \end{ornek}
\end{frame}

\begin{frame}
  \frametitle{Prim Algoritması Örneği}

  \begin{ornek}[$5 < 7$]
    \begin{columns}
      \column{.4\textwidth}
      \begin{center}
        \pgfuseimage{prim4}
      \end{center}

      \pause
      \column{.6\textwidth}
      \begin{itemize}
        \item $T = \{$\\
          $~~(a,b), (b,e), (e,g),$\\
          $~~(d,e), (f,g)$\\
          $\}$
        \item $P = \{ a, b, e, g, d, f \}$
        \item $N = \{ c \}$
        \item $i \leftarrow 6$
      \end{itemize}
    \end{columns}
  \end{ornek}
\end{frame}

\begin{frame}
  \frametitle{Prim Algoritması Örneği}

  \begin{ornek}[$6 < 7$]
    \begin{columns}
      \column{.4\textwidth}
      \begin{center}
        \pgfuseimage{prim5}
      \end{center}

      \pause
      \column{.6\textwidth}
      \begin{itemize}
        \item $T = \{$\\
          $~~(a,b), (b,e), (e,g),$\\
          $~~(d,e), (f,g), (c,g)$\\
          $\}$
        \item $P = \{ a, b, e, g, d, f, c \}$
        \item $N = \emptyset$
        \item $i \leftarrow 7$
      \end{itemize}
    \end{columns}
  \end{ornek}
\end{frame}

\begin{frame}
  \frametitle{Prim Algoritması Örneği}

  \begin{ornek}[$7 \nless 7$]
    \begin{columns}
      \column{.4\textwidth}
      \begin{center}
        \pgfuseimage{prim}
      \end{center}

      \column{.6\textwidth}
      \begin{itemize}
        \item toplam ağırlık: $17$
      \end{itemize}
    \end{columns}
  \end{ornek}
\end{frame}

\section*{Kaynaklar}

\begin{frame}
  \frametitle{Kaynaklar}

  \begin{block}{Okunacak: Grimaldi}
    \begin{itemize}
      \item Chapter 12: Trees
      \begin{itemize}
        \item 12.1. \alert{Definitions and Examples}
        \item 12.2. \alert{Rooted Trees}
      \end{itemize}
      \item Chapter 13: Optimization and Matching
      \begin{itemize}
        \item 13.2. \alert{Minimal Spanning Trees: The Algorithms of Kruskal and Prim}
      \end{itemize}
    \end{itemize}
  \end{block}
\end{frame}

\end{document}
