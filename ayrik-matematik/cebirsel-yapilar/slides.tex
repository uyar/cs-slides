% Copyright (c) 2001-2013
%       H. Turgut Uyar <uyar@itu.edu.tr>
%       Ayşegül Gençata Yayımlı <gencata@itu.edu.tr>
%       Emre Harmancı <harmanci@itu.edu.tr>
%
% Bu notlar "Creative Commons Attribution-NonCommercial-ShareAlike License" ile
% lisanslanmıştır. Yazarının açıkça belirtilmesi koşuluyla ve ticari olmayan
% amaçlarla kullanılabilir ve dağıtılabilir. Bu notlardan yola çıkılarak
% oluşturulacak çalışmaların da aynı lisansa bağlı olmaları gerekir.
%
% Lisans ile ilgili ayrıntılı bilgi almak için şu sayfaya başvurabilirsiniz:
% http://creativecommons.org/licenses/by-nc-sa/3.0/

\documentclass[dvipsnames]{beamer}

\usepackage{ae}
\usepackage[T1]{fontenc}
\usepackage[utf8]{inputenc}
\usepackage[turkish]{babel}
\setbeamertemplate{navigation symbols}{}

\mode<presentation>
{
  \usetheme{Rochester}
  \setbeamercovered{transparent}
}

\title{Ayrık Matematik}
\subtitle{Cebirsel Yapılar}

\author{H. Turgut Uyar \and Ayşegül Gençata Yayımlı \and Emre Harmancı}
\date{2001-2013}

\AtBeginSubsection[]
{
  \begin{frame}<beamer>
    \frametitle{Konular}
    \tableofcontents[currentsection,currentsubsection]
  \end{frame}
}

%\beamerdefaultoverlayspecification{<+->}

\theoremstyle{definition}
\newtheorem{tanim}[theorem]{Tanım}

\theoremstyle{example}
\newtheorem{ornek}[theorem]{Örnek}

\theoremstyle{plain}
\newtheorem{teorem}[theorem]{Teorem}

\pgfdeclareimage[width=2cm]{license}{../../license}

\pgfdeclareimage{hasse2418}{hasse2418}
\pgfdeclareimage{sayilama}{sayilama}
\pgfdeclareimage{hasse36}{hasse36}
\pgfdeclareimage{kafes}{kafes}
\pgfdeclareimage{dagilma1}{dagilma1}
\pgfdeclareimage{dagilma2}{dagilma2}

\begin{document}

\begin{frame}
  \titlepage
\end{frame}

\begin{frame}
  \frametitle{Lisans}

  \pgfuseimage{license}\hfill
  \copyright 2001-2013 T. Uyar, A. Yayımlı, E. Harmancı

  \vfill
  \begin{tiny}
    You are free:
    \begin{itemize}
      \item to Share -- to copy, distribute and transmit the work
      \item to Remix -- to adapt the work
    \end{itemize}

    Under the following conditions:
    \begin{itemize}
      \item Attribution -- You must attribute the work in the manner specified by
        the author or licensor (but not in any way that suggests that they
        endorse you or your use of the work).

      \item Noncommercial -- You may not use this work for commercial purposes.

      \item Share Alike -- If you alter, transform, or build upon this work, you
        may distribute the resulting work only under the same or similar license
        to this one.
    \end{itemize}
  \end{tiny}

  \vfill
  Legal code (the full license):\\
  \url{http://creativecommons.org/licenses/by-nc-sa/3.0/}
\end{frame}

\begin{frame}
  \frametitle{Konular}
  \tableofcontents
\end{frame}

\section{Cebirsel Yapılar}

\subsection{Giriş}

\begin{frame}
  \frametitle{Cebirsel Yapı}

  \begin{itemize}
    \item \alert{cebirsel yapı}: $<$küme, işlemler, sabitler$>$

    \bigskip
    \item taşıyıcı küme
    \item işlemler: ikili, tekli
    \item sabitler: etkisiz, yutucu
  \end{itemize}
\end{frame}

\begin{frame}
  \frametitle{İşlem}

  \begin{itemize}
    \item her işlem bir fonksiyon

    \medskip
    \item ikili işlem:\\
      $\circ: S \times S \rightarrow T$

    \medskip
    \item tekli işlem:\\
      $\Delta: S \rightarrow T$

    \pause
    \medskip
    \item \alert{kapalı}: $T \subseteq S$
  \end{itemize}
\end{frame}

\begin{frame}
  \frametitle{Kapalı İşlem Örnekleri}

  \begin{ornek}
    \begin{itemize}
      \item çıkarma işlemi $\mathbb{Z}$ kümesinde kapalı
      \item çıkarma işlemi $\mathbb{Z^+}$ kümesinde kapalı değil

      \pause
      \medskip
      \item bölme işlemi $\mathbb{R}$ kümesinde kapalı
      \item bölme işlemi $\mathbb{Z}$ kümesinde kapalı değil
    \end{itemize}
  \end{ornek}
\end{frame}

\begin{frame}
  \frametitle{İkili İşlem Özellikleri}

  \begin{tanim}
    \alert{değişme}:\\
    $\forall a,b \in S~a \circ b = b \circ a$
  \end{tanim}

  \begin{tanim}
    \alert{birleşme}:\\
    $\forall a,b,c \in S~(a \circ b) \circ c = a \circ (b \circ c)$
  \end{tanim}
\end{frame}

\begin{frame}
  \frametitle{İkili İşlem Örneği}

  \begin{ornek}
    $\circ: \mathbb{Z} \times \mathbb{Z} \rightarrow \mathbb{Z}$\\
    $a \circ b = a + b - 3ab$

    \pause
    \medskip
    \begin{itemize}
      \item değişme:\\
        $a \circ b = a + b - 3ab \pause
                   = b + a - 3ba \pause
                   = b \circ a$

      \pause
      \medskip
      \item birleşme:\\
        $\begin{array}{rcl}
          (a \circ b) \circ c & = & (a + b - 3ab) + c - 3 (a + b - 3ab) c\\ \pause
                              & = & a + b - 3ab + c - 3ac - 3bc + 9abc\\ \pause
                              & = & a + b + c - 3ab - 3ac - 3bc + 9abc\\ \pause
                              & = & a + (b + c - 3bc) - 3 a (b + c - 3bc)\\ \pause
                              & = & a \circ (b \circ c)
        \end{array}$
     \end{itemize}
 \end{ornek}
\end{frame}

\begin{frame}
  \frametitle{Sabitler}

  \begin{columns}
    \column{.5\textwidth}
    \begin{tanim}
      \alert{etkisiz eleman}: $1$\\
      $x \circ 1 = 1 \circ x = x$

      \begin{itemize}
        \item soldan etkisiz: $1_l \circ x = x$
        \item sağdan etkisiz: $x \circ 1_r = x$
      \end{itemize}
    \end{tanim}

    \pause
    \column{.5\textwidth}
    \begin{tanim}
      \alert{yutucu eleman}: $0$\\
      $x \circ 0 = 0 \circ x = 0$

      \begin{itemize}
        \item soldan yutucu: $0_l \circ x = 0$
        \item sağdan yutucu: $x \circ 0_r = 0$
      \end{itemize}
    \end{tanim}
  \end{columns}
\end{frame}

\begin{frame}
  \frametitle{Sabit Örnekleri}

  \begin{ornek}
    \begin{itemize}
      \item $<\mathbb{N}, max>$ için etkisiz eleman $0$
      \item $<\mathbb{N}, min>$ için yutucu eleman $0$
      \item $<\mathbb{Z}^+, min>$ için yutucu eleman $1$
    \end{itemize}
  \end{ornek}

  \pause
  \begin{ornek}
    \begin{columns}
      \column{.4\textwidth}
      \begin{tabular}{c||c|c|c}
        $\circ$ & a & b & c\\\hline\hline
              a & a & b & b\\\hline
              b & a & b & c\\\hline
              c & a & b & a
      \end{tabular}

      \column{.5\textwidth}
      \begin{itemize}
        \item $b$ soldan etkisiz
        \item $a$ ve $b$ sağdan yutucu
      \end{itemize}
    \end{columns}
  \end{ornek}
\end{frame}

\begin{frame}
  \frametitle{Sabitler}

  \begin{columns}
  \column{.5\textwidth}
    \begin{teorem}
      $1_l$ soldan etkisiz eleman olsun, ve
      $1_r$ sağdan etkisiz eleman olsun

      \[ 1_l = 1_r \]
    \end{teorem}

    \pause
    \begin{proof}[Tanıt]
      $1_l \circ 1_r \pause = 1_l \pause = 1_r$
    \end{proof}

    \pause
    \column{.5\textwidth}
    \begin{teorem}
      $0_l$ soldan yutucu eleman olsun, ve
      $0_r$ sağdan yutucu eleman olsun

      \[ 0_l = 0_r \]
    \end{teorem}

    \pause
    \begin{proof}[Tanıt]
      $0_l \circ 0_r \pause = 0_l \pause = 0_r$
    \end{proof}
  \end{columns}
\end{frame}

\begin{frame}
  \frametitle{Evrik}

  \begin{definition}
    $y$ elemanı $x$ elemanının \alert{evriği}:\\
    $x \circ y = 1$ ve $y \circ x = 1$

    \begin{itemize}
      \item aynı zamanda, $x$ elemanı $y$ elemanının evriği
    \end{itemize}
  \end{definition}

  \pause
  \begin{itemize}
    \item $x \circ y = 1$ ise:
    \begin{itemize}
      \item $x$ elemanı $y$ elemanının \emph{sol evriği}
      \item $y$ elemanı $x$ elemanının \emph{sağ evriği}
    \end{itemize}
  \end{itemize}
\end{frame}

\begin{frame}
  \frametitle{Evrik}

  \begin{teorem}
    $\circ$ birleşme özelliği taşıyan bir işlem olsun

    \[ w \circ x = x \circ y = 1 \Rightarrow w = y \]
  \end{teorem}

  \pause
  \begin{proof}[Tanıt]
    $\begin{array}{ccl}
      w & = & w \circ 1\\ \pause
        & = & w \circ (x \circ y)\\ \pause
        & = & (w \circ x) \circ y\\ \pause
        & = & 1 \circ y\\ \pause
        & = & y
    \end{array}$
  \end{proof}
\end{frame}

\subsection{Cebir Aileleri}

\begin{frame}
  \frametitle{Cebir Ailesi}

  \begin{tanim}
    \alert{cebir ailesi}: bir cebirsel yapı ve bazı aksiyomlar
  \end{tanim}

  \begin{itemize}
    \item aksiyom örnekleri:
    \begin{itemize}
      \item ikili işlemde değişme özelliği
      \item ikili işlemde birleşme özelliği
      \item evrik elemanların varlığı
      \item $\ldots$
    \end{itemize}
  \end{itemize}
\end{frame}

\begin{frame}
  \frametitle{Cebir Ailesi Örnekleri}

  \begin{ornek}
    \begin{itemize}
      \item aksiyomlar:
      \begin{itemize}
        \item $x \circ y = y \circ x$
        \item $(x \circ y) \circ z = x \circ (y \circ z)$
        \item $x \circ 1 = x$
      \end{itemize}

      \pause
      \item bu aksiyomları sağlayan yapılar:
      \begin{itemize}
       \item $<\mathbb{Z},+,0>$
       \item $<\mathbb{Z},\cdot,1>$
       \item $<\mathcal{P}(S),\cup,\emptyset>$
      \end{itemize}
    \end{itemize}
  \end{ornek}
\end{frame}

\begin{frame}
  \frametitle{Altcebir}

  \begin{tanim}
    $A = <S,\circ,\Delta,k>$ olsun, ve\\
    $A' = <S',\circ',\Delta',k'>$ olsun

    \medskip
    $A'$ cebrinin $A$ cebrinin bir \alert{altcebri}:

    \pause
    \medskip
    \begin{itemize}
      \item $S' \subseteq S$
      \item $k' = k$
      \item $\forall a,b \in S'~a \circ' b = a \circ b \in S'$
      \item $\forall a \in S'~\Delta' a = \Delta a \in S'$
    \end{itemize}
  \end{tanim}
\end{frame}

\begin{frame}
  \frametitle{Altcebir Örnekleri}

  \begin{ornek}
    \begin{itemize}
      \item $<\mathbb{Z}^+,+,0>$ cebri,
        $<\mathbb{Z},+,0>$ cebrinin bir altcebri

      \pause
      \item $<\mathbb{N},-,0>$ cebri,
        $<\mathbb{Z},-,0>$ cebrinin bir altcebri değil
    \end{itemize}
  \end{ornek}
\end{frame}

\begin{frame}
  \frametitle{Yarıgruplar}

  \begin{tanim}
    \alert{yarıgrup}: $<S,\circ>$
    \begin{itemize}
      \item $\forall a,b,c \in S~(a \circ b) \circ c = a \circ (b \circ c)$
    \end{itemize}
  \end{tanim}
\end{frame}

\begin{frame}
  \frametitle{Yarıgrup Örnekleri}

  \begin{ornek}
    $<\Sigma^+,\&>$

    \begin{itemize}
      \item $\Sigma$: alfabe, $\Sigma^+$: en az 1 uzunluklu katarlar
      \item $\&$: katar bitiştirme işlemi
    \end{itemize}
  \end{ornek}
\end{frame}

\begin{frame}
  \frametitle{Monoidler}

  \begin{tanim}
    \alert{monoid}: $<S,\circ,1>$

    \begin{itemize}
      \item $\forall a,b,c \in S~(a \circ b) \circ c = a \circ (b \circ c)$
      \item $\forall a \in S~a \circ 1 = 1 \circ a = a$
    \end{itemize}
  \end{tanim}
\end{frame}

\begin{frame}
  \frametitle{Monoid Örnekleri}

  \begin{ornek}
    $<\Sigma^*,\&,\epsilon>$

    \begin{itemize}
      \item $\Sigma$: alfabe, $\Sigma^*$: herhangi uzunluklu katarlar
      \item $\&$: katar bitiştirme işlemi
      \item $\epsilon$: boş katar
    \end{itemize}
  \end{ornek}
\end{frame}

\subsection{Gruplar}

\begin{frame}
  \frametitle{Grup}

  \begin{tanim}
    \alert{grup}: $<S,\circ,1>$

    \begin{itemize}
      \item $\forall a,b,c \in S~(a \circ b) \circ c = a \circ (b \circ c)$
      \item $\forall a \in S~a \circ 1 = 1 \circ a = a$
      \item $\forall a \in S~\exists a^{-1} \in S~$
        $a \circ a^{-1} = a^{-1} \circ a = 1$

      \pause
      \medskip
      \item \emph{Abel grubu}: $\forall a,b \in S~a \circ b = b \circ a$
    \end{itemize}
  \end{tanim}
\end{frame}

\begin{frame}
  \frametitle{Grup Örnekleri}

  \begin{ornek}
    \begin{itemize}
      \item  $<\mathbb{Z},+,0>$ bir grup

      \pause
      \medskip
      \item $<\mathbb{Q},\cdot,1>$ bir grup değil
      \item $<\mathbb{Q}-\{0\},\cdot,1>$ bir grup
    \end{itemize}
  \end{ornek}
\end{frame}

\begin{frame}
  \frametitle{Grup Örneği: Permutasyon}

  \begin{itemize}
    \item permutasyon: küme içi bijektif bir fonksiyon

    \medskip
    \item gösterilim:
    \[\left(
      \begin{array}{cccc}
         a_1   &  a_2   & \dots &  a_n\\
        p(a_1) & p(a_2) & \dots & p(a_n)
      \end{array}
    \right)\]

    \pause
    \medskip
    \item permutasyon bileşkesi: $\diamond$
  \end{itemize}
\end{frame}

\begin{frame}
  \frametitle{Permutasyon Örneği}

  \begin{ornek}
    $A = \{1,2,3\}$

    \medskip
    $\begin{array}{cc}
      p_1 = \left(
        \begin{array}{ccc}
          1 & 2 & 3\\
          1 & 2 & 3
        \end{array}
      \right) &
      p_2 = \left(
        \begin{array}{ccc}
          1 & 2 & 3\\
          1 & 3 & 2
        \end{array}
      \right)\medskip\\
      p_3 = \left(
        \begin{array}{ccc}
          1 & 2 & 3\\
          2 & 1 & 3
        \end{array}
      \right) &
      p_4 = \left(
        \begin{array}{ccc}
          1 & 2 & 3\\
          2 & 3 & 1
        \end{array}
      \right)\medskip\\
      p_5 = \left(
        \begin{array}{ccc}
          1 & 2 & 3\\
          3 & 1 & 2
        \end{array}
      \right) &
      p_6 = \left(
        \begin{array}{ccc}
          1 & 2 & 3\\
          3 & 2 & 1
        \end{array}
      \right)
    \end{array}$
  \end{ornek}
\end{frame}

\begin{frame}
  \frametitle{Permutasyon Bileşkesi Örneği}

  \begin{ornek}
    $A = \{1,2,3\}$

    \medskip
    $\begin{array}{cc}
      p_3 = \left(
        \begin{array}{ccc}
          1 & 2 & 3\\
          2 & 1 & 3
        \end{array}
      \right) &
      p_5 = \left(
        \begin{array}{ccc}
          1 & 2 & 3\\
          3 & 1 & 2
        \end{array}
      \right)
    \end{array}$

    \medskip
    $\begin{array}{c}
      p_3 \diamond p_5 = \left(
        \begin{array}{ccc}
          1 & 2 & 3\\
          1 & 3 & 2
        \end{array}
      \right)
    \end{array}$
  \end{ornek}
\end{frame}

\begin{frame}
  \frametitle{Grup Örneği: Permutasyon}

  \begin{itemize}
    \item permutasyon bileşkesi birleşme özelliği gösterir
    \item birim permutasyon: $1_A$
    \[\left(
      \begin{array}{cccc}
         a_1 & a_2 & \dots & a_n\\
         a_1 & a_2 & \dots & a_n
      \end{array}
    \right)\]

    \pause
    \medskip
    \item $Perm(A)$, $A$ kümesindeki elemanların\\
      bütün permutasyonlarının oluşturduğu küme olsun:\\
      $<Perm(A), \diamond, 1_A>$ bir grup
  \end{itemize}
\end{frame}

\begin{frame}
  \frametitle{Grup Örneği: Permutasyon}

  \begin{ornek}[$\{1,2,3,4\}$ kümesindeki permutasyonlar]
    $\begin{array}{c|cccccccccccccccccccccccc}
        A & 1_{A}  & p_{1}  & p_{2}  & p_{3}  & p_{4}  & p_{5}
          & p_{6}  & p_{7}  & p_{8}  & p_{9}  & p_{10} & p_{11}\\\hline
        1 &   1    &   1    &   1    &   1    &   1    &  1
          &   2    &   2    &   2    &   2    &   2    &  2\\
        2 &   2    &   2    &   3    &   3    &   4    &  4
          &   1    &   1    &   3    &   3    &   4    &  4\\
        3 &   3    &   4    &   2    &   4    &   2    &  3
          &   3    &   4    &   1    &   4    &   1    &  3\\
        4 &   4    &   3    &   4    &   2    &   3    &  2
          &   4    &   3    &   4    &   1    &   3    &  1\\\\
          & p_{12} & p_{13} & p_{14} & p_{15} & p_{16} & p_{17}
          & p_{18} & p_{19} & p_{20} & p_{21} & p_{22} & p_{23}\\\hline
        1 &   3    &   3    &   3    &   3    &   3    &  3
          &   4    &   4    &   4    &   4    &   4    &  4\\
        2 &   1    &   1    &   2    &   2    &   4    &  4
          &   1    &   1    &   2    &   2    &   3    &  3\\
        3 &   2    &   4    &   1    &   4    &   1    &  2
          &   2    &   3    &   1    &   3    &   1    &  2\\
        4 &   4    &   2    &   4    &   1    &   2    &  1
          &   3    &   2    &   3    &   1    &   2    &  1
      \end{array}$
  \end{ornek}
\end{frame}

\begin{frame}
  \frametitle{Grup Örneği: Permutasyon}

  \begin{ornek}
    \begin{itemize}
      \item $p_8 \diamond p_{12}=p_{12} \diamond p_8=1_A$:\\
        $p_{12} = p_8^{-1}$, $p_8 = p_{12}^{-1}$
      \item $p_{14} \diamond p_{14}=1_A$:\\
        $p_{14} = p_{14}^{-1}$

      \pause
      \bigskip
      \item $G_1=<\{1_A,p_1,\dots,p_{23}\},\diamond,1_A>$ bir grup
    \end{itemize}
  \end{ornek}
\end{frame}

\begin{frame}
  \frametitle{Grup Örneği: Permutasyon}

  \begin{ornek}
    \begin{itemize}
      \item $G_2=<\{1_A,p_2,p_6,p_8,p_{12},p_{14}\},\diamond,1_A>$ olsun
    \end{itemize}
    \[
      \begin{array}{c||c|c|c|c|c|c}
        \diamond & 1_{A}  & p_{2}  & p_{6}  & p_{8}  & p_{12} & p_{14}\\\hline\hline
        1_{A}    & 1_{A}  & p_{2}  & p_{6}  & p_{8}  & p_{12} & p_{14}\\\hline
        p_{2}    & p_{2}  & 1_{A}  & p_{8}  & p_{6}  & p_{14} & p_{12}\\\hline
        p_{6}    & p_{6}  & p_{12} & 1_{A}  & p_{14} & p_{2}  & p_{8}\\\hline
        p_{8}    & p_{8}  & p_{14} & p_{2}  & p_{12} & 1_{A}  & p_{6}\\\hline
        p_{12}   & p_{12} & p_{6}  & p_{14} & 1_{A}  & p_{8}  & p_{2}\\\hline
        p_{14}   & p_{14} & p_{8}  & p_{12} & p_{2}  & p_{6}  & 1_{A}
      \end{array}
    \]

    \pause
    \bigskip
    \begin{itemize}
      \item $G_2,\diamond,1_A>$ $G_1$'in bir altgrubu
    \end{itemize}
  \end{ornek}
\end{frame}

\begin{frame}
  \frametitle{Sağdan ve Soldan Kaldırma}

  \begin{teorem}
    $a \circ c = b \circ c \Rightarrow a = b$

    $c \circ a = c \circ b \Rightarrow a = b$
  \end{teorem}

  \pause
  \begin{proof}[Tanıt]
    $\begin{array}{crcl}
                  & a \circ c                & = & b \circ c\\ \pause
      \Rightarrow & (a \circ c) \circ c^{-1} & = & (b \circ c) \circ c^{-1}\\ \pause
      \Rightarrow & a \circ (c \circ c^{-1}) & = & b \circ (c \circ c^{-1})\\ \pause
      \Rightarrow & a \circ 1                & = & b \circ 1\\ \pause
      \Rightarrow & a                        & = & b
    \end{array}$

  \end{proof}
\end{frame}

\begin{frame}
  \frametitle{Grupların Temel Teoremi}

  \begin{teorem}
    $a \circ x = b$ denkleminin tek çözümü:\\
    $x = a^{-1} \circ b$.
  \end{teorem}

  \pause
  \begin{proof}[Tanıt]
    $\begin{array}{crcl}
                & a \circ c                & = & b\\\pause
    \Rightarrow & a^{-1} \circ (a \circ c) & = & a^{-1} \circ b\\\pause
    \Rightarrow & 1 \circ c                & = & a^{-1} \circ b\\\pause
    \Rightarrow & c                        & = & a^{-1} \circ b
    \end{array}$
  \end{proof}
\end{frame}

\begin{frame}
  \frametitle{Halka}

  \begin{tanim}
    \alert{halka}: $<S,+,\cdot,0>$

    \begin{itemize}
      \item $\forall a,b,c \in S~(a + b) + c = a + (b + c)$
      \item $\forall a \in S~a + 0 = 0 + a = a$
      \item $\forall a \in S~\exists (-a) \in S~a + (-a) = (-a) + a = 0$
      \item $\forall a,b \in S~a + b = b + a$

      \pause
      \item $\forall a,b,c \in S~(a \cdot b) \cdot c = a \cdot (b \cdot c)$

      \pause
      \item $\forall a,b,c \in S$
      \begin{itemize}
        \item $a \cdot (b + c) = a \cdot b + a \cdot c$
        \item $(b + c) \cdot a = b \cdot a + c \cdot a$
      \end{itemize}
    \end{itemize}
  \end{tanim}
\end{frame}

\begin{frame}
  \frametitle{Alan}

  \begin{tanim}
    \alert{alan}: $<S,+,\cdot,0,1>$
    \begin{itemize}
      \item bütün halka özellikleri

      \pause
      \item $\forall a,b \in S~a \cdot b = b \cdot a$
      \item $\forall a \in S~a \cdot 1 = 1 \cdot a = a$
      \item $\forall a \in S~\exists a^{-1} \in S~a \cdot a^{-1} = a^{-1} \cdot a = 1$
    \end{itemize}
  \end{tanim}
\end{frame}

\subsection*{Kaynaklar}

\begin{frame}
  \frametitle{Kaynaklar}

  \begin{block}{Grimaldi}
    \begin{itemize}
      \item Chapter 5: Relations and Functions
      \begin{itemize}
        \item 5.4. \alert{Special Functions}
      \end{itemize}

      \item Chapter 16: Groups, Coding Theory,\\
        and Polya's Method of Enumeration
      \begin{itemize}
        \item 16.1. \alert{Definitions, Examples, and Elementary Properties}
      \end{itemize}

      \item Chapter 14: Rings and Modular Arithmetic
      \begin{itemize}
        \item 14.1. \alert{The Ring Structure: Definition and Examples}
      \end{itemize}
    \end{itemize}
  \end{block}
\end{frame}

\section{Kafesler}

\subsection{Kısmi Sıralı Kümeler}

\begin{frame}
  \frametitle{Kısmi Sıralı Küme}

  \begin{tanim}
    \alert{kısmi sıra bağıntısı}:
    \begin{itemize}
      \item yansımalı
      \item ters bakışlı
      \item geçişli
    \end{itemize}
  \end{tanim}

  \pause
  \begin{itemize}
    \item \emph{kısmi sıralı küme} (\alert{poset}):\\
      elemanları üzerinde kısmi sıra bağıntısı tanımlanmış küme
  \end{itemize}
\end{frame}

\begin{frame}
  \frametitle{Kısmi Sıra Örnekleri}

  \begin{ornek}[kümeler kümesi, $\subseteq$]
    \begin{itemize}
      \item $A \subseteq A$
      \item $A \subseteq B \wedge B \subseteq A \Rightarrow A = B$
      \item $A \subseteq B \wedge B \subseteq C \Rightarrow A \subseteq C$
    \end{itemize}
  \end{ornek}
\end{frame}

\begin{frame}
  \frametitle{Kısmi Sıra Örnekleri}

  \begin{ornek}[$\mathbb{Z}$, $\leq$]
    \begin{itemize}
      \item $x \leq x$
      \item $x \leq y \wedge y \leq x \Rightarrow x = y$
      \item $x \leq y \wedge y \leq z \Rightarrow x \leq z$
    \end{itemize}
  \end{ornek}
\end{frame}

\begin{frame}
  \frametitle{Kısmi Sıra Örnekleri}

  \begin{ornek}[$\mathbb{Z}^+$, $|$]
    \begin{itemize}
      \item $x | x$
      \item $x | y \wedge y | x \Rightarrow x = y$
      \item $x | y \wedge y | z \Rightarrow x | z$
    \end{itemize}
  \end{ornek}
\end{frame}

\begin{frame}
  \frametitle{Karşılaştırılabilirlik}

  \begin{itemize}
    \item $a \preceq b$: \emph{a b'nin önündedir}

    \medskip
    \item $a \preceq b \vee b \preceq a$: \emph{a ile b karşılaştırılabilir}

    \pause
    \bigskip
    \item \alert{çizgisel sıra}:\\
      her eleman çifti karşılaştırılabiliyor
  \end{itemize}
\end{frame}

\begin{frame}
  \frametitle{Karşılaştırılabilirlik Örnekleri}

  \begin{ornek}
    \begin{itemize}
      \item $\mathbb{Z}^+,|$: $3$ ile $5$ karşılaştırılamaz

      \pause
      \medskip
      \item $\mathbb{Z},\leq$: çizgisel sıra
    \end{itemize}
  \end{ornek}
\end{frame}

\begin{frame}
  \frametitle{Hasse Çizenekleri}

  \begin{itemize}
    \item $a \ll b$: \emph{a b'nin hemen önündedir}\\
      $\neg \exists x~ a \preceq x \preceq b$

    \pause
    \medskip
    \item Hasse çizeneği:
    \begin{itemize}
      \item $a \ll b$ ise $a$ ile $b$ arasına çizgi
      \item önde olan eleman aşağıya
    \end{itemize}
  \end{itemize}
\end{frame}

\begin{frame}
  \frametitle{Hasse Çizeneği Örnekleri}

  \begin{ornek}
    \begin{columns}
      \column{.45\textwidth}
      $\{1,2,3,4,6,8,9,12,18,24\}$\\
      $|$ bağıntısı

      \column{.45\textwidth}
      \begin{center}
        \pgfuseimage{hasse2418}
      \end{center}
    \end{columns}
  \end{ornek}
\end{frame}

\begin{frame}
  \frametitle{Tutarlı Sayılama}

  \begin{itemize}
    \item tutarlı sayılama:\\
      $f: S \rightarrow \mathbb{N}$\\
      $a \preceq b \Rightarrow f(a) \leq f(b)$

    \medskip
    \item birden fazla tutarlı sayılama olabilir
  \end{itemize}
\end{frame}

\begin{frame}
  \frametitle{Tutarlı Sayılama Örnekleri}

  \begin{ornek}
    \begin{center}
      \pgfuseimage{sayilama}
    \end{center}

    \begin{itemize}
     \item $\{a \longmapsto 5, b \longmapsto 3, c \longmapsto 4,
      d \longmapsto 1, e \longmapsto 2\}$

     \pause
     \item $\{a \longmapsto 5, b \longmapsto 4, c \longmapsto 3,
      d \longmapsto 2, e \longmapsto 1\}$
    \end{itemize}
  \end{ornek}
\end{frame}

\begin{frame}
  \frametitle{En Büyük - En Küçük Eleman}

  \begin{tanim}
    \alert{en büyük eleman}: $max$\\
    $\forall x \in S~max \preceq x \Rightarrow x = max$
  \end{tanim}

  \pause
  \begin{tanim}
    \alert{en küçük eleman}: $min$\\
    $\forall x \in S~x \preceq min \Rightarrow x = min$
  \end{tanim}
\end{frame}

\begin{frame}
  \frametitle{En Büyük - En Küçük Eleman Örnekleri}

  \begin{ornek}
    \begin{columns}
      \column{.45\textwidth}
      \begin{center}
        \pgfuseimage{hasse2418}
      \end{center}

      \column{.45\textwidth}
      $max:~18,24$\\
      $min:~1$
    \end{columns}
  \end{ornek}
\end{frame}

\begin{frame}
  \frametitle{Sınırlar}

  \begin{columns}[t]
    \column{.5\textwidth}
    \begin{tanim}
      $A \subseteq S$

      \medskip
      $A$'nın \alert{üstsınırı} $M$:\\
      $\forall x \in A~x \preceq M$

      \bigskip
      $M(A)$: $A$'nın üstsınırları kümesi

      \medskip
      $A$'nın \alert{en küçük üstsınırı} $sup(A)$:\\
      $\forall M \in M(A)~sup(A) \preceq M$
    \end{tanim}

    \pause
    \column{.5\textwidth}
    \begin{tanim}
      $A \subseteq S$

      \medskip
      $A$'nın \alert{altsınırı} $m$:\\
      $\forall x \in A~m \preceq x$

      \bigskip
      $m(A)$: $A$'nın altsınırları kümesi

      \medskip
      $A$'nın \alert{en büyük altsınırı} $inf(A)$:\\
      $\forall m \in m(A)~m \preceq inf(A)$
    \end{tanim}
  \end{columns}
\end{frame}

\begin{frame}
  \frametitle{Sınır Örneği}

  \begin{ornek}[36'nın bölenleri]
    \begin{columns}
      \column{.45\textwidth}
      \begin{center}
        \pgfuseimage{hasse36}
      \end{center}

      \column{.45\textwidth}
      inf = en büyük ortak bölen\\
      sup = en küçük ortak kat
    \end{columns}
  \end{ornek}
\end{frame}

\subsection{Kafesler}

\begin{frame}
  \frametitle{Kafes}

  \begin{tanim}
    \alert{kafes}: $<L,\wedge,\vee>$\\
    $\wedge$: karşılaşma, $\vee$: bütünleşme

    \pause
    \begin{itemize}
      \item $a \wedge b = b \wedge a$\\
        $a \vee b = b \vee a$
      \item$(a \wedge b) \wedge c = a \wedge (b \wedge c)$\\
        $(a \vee b) \vee c = a \vee (b \vee c)$
      \item $a \wedge (a \vee b) = a$\\
        $a \vee (a \wedge b) = a$
    \end{itemize}
  \end{tanim}
\end{frame}

\begin{frame}
  \frametitle{Kısmi Sıralı Küme - Kafes İlişkisi}

  \begin{itemize}
    \item $P$ bir kısmi sıralı küme ise $<P,inf,sup>$ bir kafestir.
    \begin{itemize}
      \item $a \wedge b = inf(a,b)$
      \item $a \vee b = sup(a,b)$
    \end{itemize}

    \pause
    \medskip
    \item Her kafes bu tanımların geçerli olduğu bir kısmi sıralı kümedir.
  \end{itemize}
\end{frame}

\begin{frame}
  \frametitle{Dualite}

  \begin{tanim}
    \alert{dual}:\\
    $\wedge$ yerine $\vee$, $\vee$ yerine $\wedge$
  \end{tanim}

  \pause
  \begin{teorem}[Dualite Teoremi]
    Kafeslerde her teoremin duali de teoremdir.
  \end{teorem}
\end{frame}

\begin{frame}
  \frametitle{Kafes Teoremleri}

  \begin{teorem}
    $a \wedge a = a$
  \end{teorem}

  \pause
  \begin{proof}[Tanıt]
    $a \wedge a = a \wedge (a \vee (a \wedge b))$
  \end{proof}
\end{frame}

\begin{frame}
  \frametitle{Kafes Teoremleri}

  \begin{teorem}
    $a \preceq b \Leftrightarrow a \wedge b$
    $ = a \Leftrightarrow a \vee b = b$
  \end{teorem}
\end{frame}

\begin{frame}
  \frametitle{Kafes Örnekleri}

  \begin{ornek}
    \begin{columns}
      \column{.3\textwidth}
      \[
	<\mathcal{P}\{a,b,c\},\cap,\cup>
      \]
      $\subseteq$ bağıntısı

      \column{.6\textwidth}
      \begin{center}
        \pgfuseimage{kafes}
      \end{center}
    \end{columns}
  \end{ornek}
\end{frame}

\begin{frame}
  \frametitle{Sınırlı Kafesler}

  \begin{columns}[t]
    \column{.5\textwidth}
    \begin{tanim}
      $L$ kafesinin altsınırı: $0$\\
      $\forall x \in L~0 \preceq x$
    \end{tanim}

    \pause
    \column{.5\textwidth}
    \begin{tanim}
      $L$ kafesinin üstsınırı: $I$\\
      $\forall x \in L~x \preceq I$
    \end{tanim}
  \end{columns}

  \pause
  \bigskip
  \begin{teorem}
    Sonlu her kafes sınırlıdır.
  \end{teorem}
\end{frame}

\begin{frame}
  \frametitle{Kafeslerde Dağılma}

  \begin{itemize}
    \item \emph{dağılma özellikli kafes}:
    \begin{itemize}
      \item $\forall a,b,c \in L~a \wedge (b \vee c) = (a \wedge b) \vee (a \wedge c)$
      \item $\forall a,b,c \in L~a \vee (b \wedge c) = (a \vee b) \wedge (a \vee c)$
    \end{itemize}
  \end{itemize}
\end{frame}

\begin{frame}
  \frametitle{Karşı Örnekler}

  \begin{ornek}
    \begin{columns}
      \column{.45\textwidth}
      \begin{center}
        \pgfuseimage{dagilma1}
      \end{center}

      \pause
      \column{.45\textwidth}
      $a \vee (b \wedge c)$ \pause $= a \vee 0$ \pause $= a$

      \pause
      $(a \vee b) \wedge (a \vee c)$ \pause $= I \wedge c$ \pause $= c$
    \end{columns}
  \end{ornek}
\end{frame}

\begin{frame}
  \frametitle{Karşı Örnekler}

  \begin{ornek}
    \begin{columns}
      \column{.45\textwidth}
      \begin{center}
        \pgfuseimage{dagilma2}
      \end{center}

      \pause
      \column{.45\textwidth}
      $a \vee (b \wedge c)$ \pause $= a \vee 0$ \pause $= a$

      \pause
      $(a \vee b) \wedge (a \vee c)$ \pause $= I \wedge I$ \pause $= I$
    \end{columns}
  \end{ornek}
\end{frame}

\begin{frame}
  \frametitle{Kafeslerde Dağılma}

  \begin{teorem}
    Bir kafes yalnız ve ancak bu iki yapıdan birine izomorfik bir altkafes içeriyorsa dağılma
    özelliği göstermez.
  \end{teorem}
\end{frame}

\begin{frame}
  \frametitle{Bütünleşmeyle İndirgeme}

  \begin{tanim}
    \alert{bütünleşmeyle indirgenemez eleman}:\\
    $a = x \vee y \Rightarrow a = x ~veya~ a = y$
  \end{tanim}

  \pause
  \medskip
  \begin{itemize}
    \item \emph{atom}: altsınırın hemen ardından gelen,\\
      bütünleşmeyle indirgenemez eleman
  \end{itemize}
\end{frame}

\begin{frame}
  \frametitle{Bütünleşmeyle İndirgeme Örneği}

  \begin{ornek}[bölünebilirlik bağıntısı]
    \begin{itemize}
      \item asal sayılar ve 1 bütünleşmeyle indirgenemez

      \pause
      \medskip
      \item 1 altsınır, asal sayılar atom
    \end{itemize}
  \end{ornek}
\end{frame}

\begin{frame}
  \frametitle{Bütünleşmeyle İndirgeme}

  \begin{teorem}
    Bütünleşmeyle indirgenebilir bütün elemanlar, bütünleşmeyle indirgenemez
    elemanların bütünleşmesi şeklinde yazılabilir.
  \end{teorem}
\end{frame}

\begin{frame}
  \frametitle{Tümleyen}

  \begin{tanim}
    $a$ ile $x$ \alert{tümleyen}:\\
    $a \wedge x = 0$ ve $a \vee x = I$
  \end{tanim}
\end{frame}

\begin{frame}
  \frametitle{Tümlemeli Kafesler}

  \begin{teorem}
    Sınırlı, dağılma özellikli bir kafeste tümleyen varsa tektir.
  \end{teorem}

  \pause
  \begin{proof}[Tanıt]
    $a \wedge x = 0, a \vee x = I$, $a \wedge y = 0, a \vee y = I$

    \pause
    \medskip
    \begin{eqnarray*}
  & x & = x \vee 0 \pause = x \vee (a \wedge y) \pause = (x \vee a) \wedge (x \vee y) \pause = I \wedge (x \vee y)\\
  &   & \pause = x \vee y \pause = y \vee x \pause = I \wedge (y \vee x)\\
  &   & \pause = (y \vee a) \wedge (y \vee x) \pause = y \vee (a \wedge x) \pause = y \vee 0 \pause = y\\
    \end{eqnarray*}
  \end{proof}
\end{frame}

\subsection{Boole Cebirleri}

\begin{frame}
  \frametitle{Boole Cebri}

  \begin{tanim}
    \alert{Boole cebri}:\\
    $<B,+,\cdot,\overline{x},1,0>$

    \pause
    \[\begin{array}{ll}
      a + b = b + a &
      a \cdot b = b \cdot a\\ \pause
      (a + b) + c = a + (b + c) &
      (a \cdot b) \cdot c = a \cdot (b \cdot c)\\ \pause
      a + 0 = a &
      a \cdot 1 = a\\ \pause
      a + \overline{a} = 1 &
      a \cdot \overline{a} = 0
    \end{array}\]
  \end{tanim}
\end{frame}

\begin{frame}
  \frametitle{Boole Cebri - Kafes İlişkisi}

  \begin{tanim}
    Bir Boole cebri sonlu, dağılma özellikli,\\
    her elemanın tümleyeninin olduğu bir kafestir.
  \end{tanim}
\end{frame}

\begin{frame}
  \frametitle{Dualite}

  \begin{tanim}
    \alert{dual}:\\
    $+$ yerine $\cdot$, $\cdot$ yerine $+$\\
    0 yerine 1, 1 yerine 0
  \end{tanim}

  \pause
  \begin{ornek}
    $(1 + a) \cdot (b + 0) = b$

    teoreminin duali:

    $(0 \cdot a) + (b \cdot 1) = b$
  \end{ornek}
\end{frame}

\begin{frame}
  \frametitle{Boole Cebri Örnekleri}

  \begin{ornek}
    $B = \{0,1\}, + = \vee, \cdot = \wedge$
  \end{ornek}

  \pause
  \begin{ornek}
    $B = \{$ $70$'in bölenleri $\}$, $+ = okek, \cdot = obeb$
  \end{ornek}
\end{frame}

\begin{frame}
  \frametitle{Boole Cebri Teoremleri}

    \[\begin{array}{ll}
      a + a = a &
      a \cdot a = a\\ \pause
      a + 1 = 1 &
      a \cdot 0 = 0\\ \pause
      a + (a \cdot b) = a &
      a \cdot (a + b) = a\\ \pause
      (a + b) + c = a + (b + c) &
      (a \cdot b) \cdot c = a \cdot (b \cdot c)\bigskip\\ \pause
      \overline{\overline{a}} = a & \\ \pause
      \overline{a + b} = \overline{a} \cdot \overline{b} &
      \overline{a \cdot b} = \overline{a} + \overline{b}
    \end{array}\]
\end{frame}

\subsection*{Kaynaklar}

\begin{frame}
  \frametitle{Kaynaklar}

  \begin{block}{Okunacak: Grimaldi}
    \begin{itemize}
      \item Chapter 7: Relations: The Second Time Around
      \begin{itemize}
        \item 7.3. \alert{Partial Orders: Hasse Diagrams}
      \end{itemize}

      \item Chapter 15: Boolean Algebra and Switching Functions
      \begin{itemize}
        \item 15.4. \alert{The Structure of a Boolean Algebra}
      \end{itemize}
    \end{itemize}
  \end{block}
\end{frame}

\end{document}
