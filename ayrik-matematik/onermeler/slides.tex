% Copyright (c) 2001-2012
%       H. Turgut Uyar <uyar@itu.edu.tr>
%       Ayşegül Gençata Yayımlı <gencata@itu.edu.tr>
%       Emre Harmancı <harmanci@itu.edu.tr>
%
% Bu notlar "Creative Commons Attribution-NonCommercial-ShareAlike License" ile
% lisanslanmıştır. Yazarının açıkça belirtilmesi koşuluyla ve ticari olmayan
% amaçlarla kullanılabilir ve dağıtılabilir. Bu notlardan yola çıkılarak
% oluşturulacak çalışmaların da aynı lisansa bağlı olmaları gerekir.
%
% Lisans ile ilgili ayrıntılı bilgi almak için şu sayfaya başvurabilirsiniz:
% http://creativecommons.org/licenses/by-nc-sa/3.0/

\documentclass[dvipsnames]{beamer}

\usepackage{ae}
\usepackage[T1]{fontenc}
\usepackage[utf8]{inputenc}
\usepackage[turkish]{babel}
\setbeamertemplate{navigation symbols}{}

\mode<presentation>
{
  \usetheme{Rochester}
  \setbeamercovered{transparent}
}

\title{Ayrık Matematik}
\subtitle{Önermeler}

\author{H. Turgut Uyar \and Ayşegül Gençata Yayımlı \and Emre Harmancı}
\date{2001-2012}

\AtBeginSubsection[]
{
  \begin{frame}<beamer>
    \frametitle{Konular}
    \tableofcontents[currentsection,currentsubsection]
  \end{frame}
}

%\beamerdefaultoverlayspecification{<+->}

\theoremstyle{definition}
\newtheorem{tanim}[theorem]{Tanım}

\theoremstyle{example}
\newtheorem{ornek}[theorem]{Örnek}

\theoremstyle{plain}
\newtheorem{teorem}[theorem]{Teorem}

\pgfdeclareimage[width=2cm]{license}{../../license}

\begin{document}

\begin{frame}
  \titlepage
\end{frame}

\begin{frame}
  \frametitle{Lisans}

  \pgfuseimage{license}\hfill
  \copyright 2001-2012 T. Uyar, A. Yayımlı, E. Harmancı

  \vfill
  \begin{tiny}
    You are free:
    \begin{itemize}
      \item to Share -- to copy, distribute and transmit the work
      \item to Remix -- to adapt the work
    \end{itemize}

    Under the following conditions:
    \begin{itemize}
      \item Attribution -- You must attribute the work in the manner specified by
        the author or licensor (but not in any way that suggests that they
        endorse you or your use of the work).

      \item Noncommercial -- You may not use this work for commercial purposes.

      \item Share Alike -- If you alter, transform, or build upon this work, you
        may distribute the resulting work only under the same or similar license
        to this one.
    \end{itemize}
  \end{tiny}

  \vfill
  Legal code (the full license):\\
  \url{http://creativecommons.org/licenses/by-nc-sa/3.0/}
\end{frame}

\begin{frame}
  \frametitle{Konular}
  \tableofcontents
\end{frame}

\section{Önermeler}

\subsection{Giriş}

\begin{frame}
  \frametitle{Önerme}

  \begin{tanim}
    \alert{önerme}: doğru ya da yanlış denebilecek bir bildirim cümlesi
  \end{tanim}

  \pause
  \begin{itemize}
    \item \alert{ara değeri dışlama kuralı}:\\
      bir önerme kısmen doğru ya da kısmen yanlış olamaz
  \end{itemize}

  \pause
  \begin{itemize}
    \item \alert{çelişki kuralı}:\\
      bir önerme hem doğru hem yanlış olamaz
  \end{itemize}
\end{frame}

\begin{frame}
  \frametitle{Önerme Örnekleri}

  \begin{columns}[t]
    \column{.6\textwidth}
    \begin{ornek}[önerme]
      \begin{itemize}
        \item Ay Yeryüzü'nün çevresinde döner.
        \item Filler uçabilir.
        \item $3+8=11$
      \end{itemize}
    \end{ornek}

    \pause
    \column{.4\textwidth}
    \begin{ornek}[önerme değil]
      \begin{itemize}
        \item Saat kaç?
        \item Ali topu at!
        \item $x<43$
      \end{itemize}
    \end{ornek}
  \end{columns}
\end{frame}

\begin{frame}
  \frametitle{Önerme Değişkeni}

  \begin{tanim}
    \alert{önerme değişkeni}:\\
      önermeyi simgeleyen isim

    \begin{itemize}
      \item \emph{Doğru} ($D$) ya da \emph{Yanlış} ($Y$) değerlerini alabilir
    \end{itemize}
  \end{tanim}

  \pause
  \begin{ornek}
    \begin{itemize}
      \item $p_1$: Ay Yeryüzü'nün çevresinde döner. ($D$)
      \item $p_2$: Filler uçabilir. ($Y$)
      \item $p_3$: $3+8=11$ ($D$)
    \end{itemize}
  \end{ornek}
\end{frame}

\subsection{Birleşik Önermeler}

\begin{frame}
  \frametitle{Birleşik Önermeler}

  \begin{itemize}
    \item{\alert{birleşik önermeler}
      \begin{itemize}
        \item bir önermenin değillenmesiyle, ya da
        \item birden fazla önermenin
          \alert{mantıksal bağlaçlar} ile birleştirilmesiyle
      \end{itemize}
      elde edilir}
    \item \alert{yalın önermeler} daha küçük birimlere bölünemez
  \end{itemize}

  \pause
  \begin{itemize}
    \item \alert{doğruluk tablosu}:\\
      içerdiği yalın önermelerin olası bütün değerleri için\\
      birleşik önermenin sonuçlarını listeleyen tablo
  \end{itemize}
\end{frame}

\begin{frame}
  \frametitle{Değilleme (NOT)}

  \begin{columns}
    \column{.3\textwidth}
    \begin{table}
      \caption{$\neg p$}
      \begin{tabular}{|c||c|}\hline
        $p$ & $\neg p$\\\hline\hline
        $D$ & $Y$\\\hline
        $Y$ & $D$\\\hline
      \end{tabular}
    \end{table}

    \pause
    \column{.7\textwidth}
    \begin{ornek}
      \begin{itemize}
        \item $\neg p_1$: Ay Yeryüzü'nün çevresinde dönmez.\\
          $\neg D$: \emph{Yanlış}
        \item $\neg p_2$: Filler uçamaz.\\
          $\neg Y$: \emph{Doğru}
      \end{itemize}
    \end{ornek}
  \end{columns}
\end{frame}

\begin{frame}
  \frametitle{VE Bağlacı (AND)}

  \begin{columns}
    \column{.4\textwidth}
    \begin{table}
      \caption{$p \wedge q$}
      \begin{tabular}{|c|c||c|}\hline
        $p$ & $q$ & $p \wedge q$\\\hline\hline
        $D$ & $D$ & $D$\\\hline
        $D$ & $Y$ & $Y$\\\hline
        $Y$ & $D$ & $Y$\\\hline
        $Y$ & $Y$ & $Y$\\\hline
      \end{tabular}
    \end{table}

    \pause
    \column{.6\textwidth}
    \begin{ornek}
      \begin{itemize}
        \item $p_1 \wedge p_2$: Ay Yeryüzü'nün çevresinde döner ve filler
          uçabilir.\\
          $D \wedge Y$: \emph{Yanlış}
      \end{itemize}
    \end{ornek}
  \end{columns}
\end{frame}

\begin{frame}
  \frametitle{VEYA Bağlacı (OR)}

  \begin{columns}
    \column{.4\textwidth}
    \begin{table}
      \caption{$p \vee q$}
      \begin{tabular}{|c|c||c|}\hline
        $p$ & $q$ & $p \vee q$\\\hline\hline
        $D$ & $D$ & $D$\\\hline
        $D$ & $Y$ & $D$\\\hline
        $Y$ & $D$ & $D$\\\hline
        $Y$ & $Y$ & $Y$\\\hline
      \end{tabular}
    \end{table}

    \pause
    \column{.6\textwidth}
    \begin{ornek}
      \begin{itemize}
        \item $p_1 \vee p_2$: Ay Yeryüzü'nün çevresinde döner veya filler
          uçabilir.\\
          $D \vee Y$: \emph{Doğru}
      \end{itemize}
    \end{ornek}
  \end{columns}
\end{frame}

\begin{frame}
  \frametitle{DAR VEYA Bağlacı (XOR)}

  \begin{columns}
    \column{.38\textwidth}
    \begin{table}
      \caption{$p \veebar q$}
      \begin{tabular}{|c|c||c|}\hline
        $p$ & $q$ & $p \veebar q$\\\hline\hline
        $D$ & $D$ & $Y$\\\hline
        $D$ & $Y$ & $D$\\\hline
        $Y$ & $D$ & $D$\\\hline
        $Y$ & $Y$ & $Y$\\\hline
      \end{tabular}
    \end{table}

    \pause
    \column{.62\textwidth}
    \begin{ornek}
      \begin{itemize}
        \item $p_1 \veebar p_2$: Ya Ay Yeryüzü'nün çevresinde döner ya da filler
          uçabilir.\\
          $D \veebar Y$: \emph{Doğru}
      \end{itemize}
    \end{ornek}
  \end{columns}
\end{frame}

\begin{frame}
  \frametitle{Koşullu Bağlaç (IF)}

  \begin{columns}
    \column{.4\textwidth}
    \begin{table}
      \caption{$p \rightarrow q$}
      \begin{tabular}{|c|c||c|}\hline
        $p$ & $q$ & $p \rightarrow q$\\\hline\hline
        $D$ & $D$ & $D$\\\hline
        $D$ & $Y$ & $Y$\\\hline
        $Y$ & $D$ & $D$\\\hline
        $Y$ & $Y$ & $D$\\\hline
      \end{tabular}
    \end{table}

    \pause
    \column{.6\textwidth}
    \begin{itemize}
      \item $p$: \alert{öncül}
      \item $q$: \alert{sonuç}

      \item okunuşları:
      \begin{itemize}
        \item $p$ ise $q$
        \item $p$, $q$ için yeterli
        \item $q$, $p$ için gerekli
      \end{itemize}

      \pause
      \item $\neg p \vee q$
    \end{itemize}
  \end{columns}
\end{frame}

\begin{frame}
  \frametitle{Koşullu Bağlaç Örnekleri}

  \begin{ornek}
    \begin{itemize}
      \item $p_4$: $3<8$, $p_5$: $3<14$, $p_6$: $3<2$
      \item $p_7$: Güneş Yeryüzü'nün çevresinde döner.
    \end{itemize}

    \pause
    \begin{columns}[t]
      \column{.47\textwidth}
        \begin{itemize}
          \item $p_4 \rightarrow p_5$: 3, 8'den küçükse\\
            3, 14'den küçüktür.\\
            $D \rightarrow D$: \emph{Doğru}
          \pause
          \item $p_4 \rightarrow p_6$: 3, 8'den küçükse\\
            3, 2'den küçüktür.\\
            $D \rightarrow Y$: \emph{Yanlış}
        \end{itemize}

      \pause
      \column{.5\textwidth}
        \begin{itemize}
          \item $p_2 \rightarrow p_1$: Filler uçabilirse Ay Yeryüzü'nün
            çevresinde döner.\\
            $Y \rightarrow D$: \emph{Doğru}
          \pause
          \item $p_2 \rightarrow p_7$: Filler uçabilirse Güneş Yeryüzü'nün
            çevresinde döner.\\
            $Y \rightarrow Y$: \emph{Doğru}
        \end{itemize}
    \end{columns}
  \end{ornek}
\end{frame}

\begin{frame}
  \frametitle{Koşullu Bağlaç Örnekleri}

  \begin{ornek}
    \begin{itemize}
      \item "70 kg'yi geçersem spor yapacağım."
    \end{itemize}

    \pause
    \begin{columns}
      \column{.6\textwidth}
      \begin{itemize}
        \item $p$: 70 kg'den ağırım.
        \item $q$: Spor yapıyorum.
      \end{itemize}

      \pause
      \begin{itemize}
        \item bu önerme ne zaman yanlış olur?
      \end{itemize}

      \column{.4\textwidth}
      \begin{table}
        \caption{$p \rightarrow q$}
        \begin{tabular}{|c|c||c|}\hline
          $p$ & $q$ & $p \rightarrow q$\\\hline\hline
          $D$ & $D$ & $D$\\\hline
          $D$ & $Y$ & $Y$\\\hline
          $Y$ & $D$ & $D$\\\hline
          $Y$ & $Y$ & $D$\\\hline
        \end{tabular}
      \end{table}
    \end{columns}
  \end{ornek}
\end{frame}

\begin{frame}
  \frametitle{Karşılıklı Koşullu Bağlaç (IFF)}

  \begin{columns}
    \column{.4\textwidth}
    \begin{table}
      \caption{$p \leftrightarrow q$}
      \begin{tabular}{|c|c||c|}\hline
        $p$ & $q$ & $p \leftrightarrow q$\\\hline\hline
        $D$ & $D$ & $D$\\\hline
        $D$ & $Y$ & $Y$\\\hline
        $Y$ & $D$ & $Y$\\\hline
        $Y$ & $Y$ & $D$\\\hline
      \end{tabular}
    \end{table}

    \pause
    \column{.6\textwidth}
    \begin{itemize}
      \item okunuşları:
      \begin{itemize}
        \item $p$ yalnız ve ancak $q$ ise
        \item $p$, $q$ için yeterli ve gerekli
      \end{itemize}

      \pause
      \item $(p \rightarrow q) \wedge (q \rightarrow p)$
      \item $\neg (p \veebar q)$
    \end{itemize}
  \end{columns}
\end{frame}

\begin{frame}
  \frametitle{Örnek}

  \begin{ornek}
    \begin{itemize}
      \item Anne çocuğa:\\
        "Ödevini yaparsan bilgisayar oyunu oynayabilirsin."

      \pause
      \medskip
      \item $s$: Çocuk ödevini yapar.
      \item $t$: Çocuk bilgisayar oyunu oynar.

      \pause
      \medskip
      \item annenin söylediği hangisi?
      \begin{itemize}
        \item $s \rightarrow t$
        \item $\neg s \rightarrow \neg t$
        \item $s \leftrightarrow t$
      \end{itemize}
    \end{itemize}
  \end{ornek}
\end{frame}

\subsection{Sağlıklı Formüller}

\begin{frame}
  \frametitle{Sağlıklı Formül}

  \begin{block}{Yazım}
    \begin{itemize}
      \item birleşik önermeler hangi kurallara göre oluşturulacak?
      \item kurallara uyan formüller: \alert{sağlıklı formül} (SF)
    \end{itemize}
  \end{block}

  \pause
  \begin{block}{Anlam}
    \begin{itemize}
      \item \emph{yorum}: yalın önermelere değer atayarak\\
        birleşik önermenin değerini hesaplama
      \item doğruluk tablosu: önermenin bütün yorumları
    \end{itemize}
  \end{block}
\end{frame}

\begin{frame}
  \frametitle{Formül Örnekleri}

  \begin{ornek}[sağlıklı değil]
    \begin{itemize}
      \item $\vee p$
      \item $p \wedge \neg$
      \item $p \neg \wedge q$
    \end{itemize}
  \end{ornek}
\end{frame}

\begin{frame}
  \frametitle{Öncelik Sırası}

  \begin{enumerate}
    \item $\neg$
    \item $\wedge$
    \item $\vee$
    \item $\rightarrow$
    \item $\leftrightarrow$
  \end{enumerate}

  \begin{itemize}
    \item önceliği değiştirmek için parantez kullanılır
  \end{itemize}
\end{frame}

\begin{frame}
  \frametitle{Öncelik Sırası Örnekleri}

  \begin{ornek}
    \begin{itemize}
      \item $s$: Filiz gezmeye çıkar.
      \item $t$: Mehtap var.
      \item $u$: Kar yağıyor.
    \end{itemize}

    \medskip
    \begin{itemize}
      \item aşağıdaki SF'ler ne anlama gelir?

      \pause
      \begin{itemize}
        \item $t \wedge \neg u \rightarrow s$
        \pause
        \item $t \rightarrow (\neg u \rightarrow s)$
        \pause
        \item $\neg (s \leftrightarrow (u \vee t))$
        \pause
        \item $\neg s \leftrightarrow u \vee t$
      \end{itemize}
    \end{itemize}
  \end{ornek}
\end{frame}

% \begin{frame}
%   \frametitle{Örnek: Yol Ayrımı}
%
%   \bigskip
%   \hyperlink{formattr}{\beamergotobutton{örneği atla}}
%
%   \begin{ornek}[yol ayrımı]
%     \begin{itemize}
%       \item Martin Gardner'ın\\
%         "Mathematical Puzzles and Diversions" kitabından
%
%       \pause
%       \medskip
%       \item bir matematikçi ücra bir adada bir yol ayrımına gelir
%       \item adada iki kabile yaşamaktadır: doğrucular (d) ve yalancılar (y)
%       \begin{itemize}
%         \item doğrucular her zaman doğru, yalancılar her zaman yalan söyler
%       \end{itemize}
%
%       \item yol ayrımında bir yerli durmaktadır
%       \begin{itemize}
%         \item hangi kabileden olduğu belli değil
%       \end{itemize}
%
%       \pause
%       \item matematikçi köye giden yolu öğrenmek için ne sormalı?
%       \begin{itemize}
%         \item yanıtı evet ya da hayır olacak tek bir soru
%       \end{itemize}
%     \end{itemize}
%   \end{ornek}
% \end{frame}
%
% \begin{frame}
%   \frametitle{Örnek: Yol Ayrımı}
%
%   \begin{ornek}
%     \begin{itemize}
%       \item $p$: Yol köye gider.
%       \item $q$: Yerli doğrucu kabileden.
%     \end{itemize}
%   \end{ornek}
% \end{frame}
%
% \begin{frame}
%   \frametitle{Örnek: Yol Ayrımı}
%
%   \begin{ornek}
%     \begin{itemize}
%       \item \emph{Sana "bu yol köye gider mi?" diye sorsaydım\\
%         "evet" der miydin?}
%
%       \pause
%       \medskip
%       \begin{table}
%         \begin{tabular}{|l|l||l|}\hline
%           $p$ & $q$ & Yanıt\\\hline\hline
%           $D$ & $D$ & Evet \\\hline
%           $D$ & $Y$ & Evet \\\hline
%           $Y$ & $D$ & Hayır\\\hline
%           $Y$ & $Y$ & Hayır\\\hline
%         \end{tabular}
%       \end{table}
%
%       \pause
%       \medskip
%       \item yol köye gidiyorsa "Evet", gitmiyorsa "Hayır"
%     \end{itemize}
%   \end{ornek}
% \end{frame}
%
% \begin{frame}
%   \frametitle{Örnek: Yol Ayrımı}
%
%   \begin{ornek}
%     \begin{itemize}
%       \item \emph{$p \veebar \neg q$ işleminin sonucu nedir?}
%
%       \pause
%       \medskip
%       \begin{table}
%         \begin{tabular}{|c|c|c|c||c|c|}\hline
%               &     &          &                    & d   & y\\
%           $p$ & $q$ & $\neg q$ & $p \veebar \neg q$ & $A$ & $\neg A$\\
%               &     &          & ($A$)              &     &\\\hline\hline
%           $D$ & $D$ & $Y$ & $D$ & $D$ & $-$\\\hline
%           $D$ & $Y$ & $D$ & $Y$ & $-$ & $D$\\\hline
%           $Y$ & $D$ & $Y$ & $Y$ & $Y$ & $-$\\\hline
%           $Y$ & $Y$ & $D$ & $D$ & $-$ & $Y$\\\hline
%         \end{tabular}
%       \end{table}
%
%       \pause
%       \medskip
%       \item yol köye gidiyorsa "$D$", gitmiyorsa "$Y$"
%     \end{itemize}
%   \end{ornek}
% \end{frame}
%
% \begin{frame}
%   \frametitle{Örnek: Yol Ayrımı}
%
%   \begin{ornek}
%     \begin{itemize}
%       \item dürüst bir yalancı kendine de yalan söyler
%
%       \pause
%       \medskip
%       \item basit yalancı (by): hesapla, değille
%       \item dürüst yalancı (dy): değille, hesapla, değille
%     \end{itemize}
%   \end{ornek}
% \end{frame}
%
% \begin{frame}
%   \frametitle{Örnek: Yol Ayrımı}
%
%   \begin{ornek}
%     \begin{itemize}
%       \item \emph{$p \rightarrow \neg q$ işleminin sonucu nedir?}
%
%       \pause
%       \medskip
%       \begin{table}
%         \begin{tabular}{|c|c|c|c|c|c||c|c|c|}\hline
%               &     &          &          &                        &
%               & d   & by       & dy\\
%           $p$ & $q$ & $\neg p$ & $\neg q$ & $p \rightarrow \neg q$ & $\neg p \rightarrow q$
%               & $A$ & $\neg A$ & $\neg B$\\
%               &     &          &          & ($A$)                  & ($B$)
%               &     &          &\\\hline\hline
%           $D$ & $D$ & $Y$ & $Y$ & $Y$ & $D$ & $Y$ & $-$ & $-$\\\hline
%           $D$ & $Y$ & $Y$ & $D$ & $D$ & $D$ & $-$ & $Y$ & $Y$\\\hline
%           $Y$ & $D$ & $D$ & $Y$ & $D$ & $D$ & $D$ & $-$ & $-$\\\hline
%           $Y$ & $Y$ & $D$ & $D$ & $D$ & $Y$ & $-$ & $Y$ & $D$\\\hline
%         \end{tabular}
%       \end{table}
%
%       \pause
%       \medskip
%       \item dürüst yalancı: yol köye gidiyorsa "$Y$", gitmiyorsa "$D$"
%       \item basit yalancı durumunu çözmüyor
%     \end{itemize}
%   \end{ornek}
% \end{frame}
%
% \begin{frame}
%   \frametitle{Örnek: Yol Ayrımı}
%
%   \begin{ornek}
%     \begin{itemize}
%       \item \emph{$p \leftrightarrow \neg q$ işleminin sonucu nedir?}
%
%       \pause
%       \medskip
%       \begin{table}
%         \begin{tabular}{|c|c|c|c|c|c||c|c|c|}\hline
%             &     &          &          &                            &
%             & d   & by       & dy\\
%         $p$ & $q$ & $\neg p$ & $\neg q$ & $p \leftrightarrow \neg q$ & $\neg p \leftrightarrow q$
%             & $A$ & $\neg A$ & $\neg B$\\
%             &     &          &        & ($A$)                        & ($B$)
%             &     &          &\\\hline\hline
%         $D$ & $D$ & $Y$ & $Y$ & $Y$ & $Y$ & $Y$ & $-$ & $-$\\\hline
%         $D$ & $Y$ & $Y$ & $D$ & $D$ & $D$ & $-$ & $Y$ & $Y$\\\hline
%         $Y$ & $D$ & $D$ & $Y$ & $D$ & $D$ & $D$ & $-$ & $-$\\\hline
%         $Y$ & $Y$ & $D$ & $D$ & $Y$ & $Y$ & $-$ & $D$ & $D$\\\hline
%         \end{tabular}
%       \end{table}
%
%       \pause
%       \medskip
%       \item yol köye gidiyorsa "$Y$", gitmiyorsa "$D$"
%     \end{itemize}
%   \end{ornek}
% \end{frame}

\begin{frame}[label=formattr]
  \frametitle{Formül Nitelikleri}

  \begin{enumerate}
    \item \emph{geçerli}: bütün yorumlar için doğru (\alert{totoloji})
    \item \emph{çelişkili}: bütün yorumlar için yanlış (\alert{çelişki})
    \item \emph{tutarlı}: bazı yorumlar için doğru
  \end{enumerate}
\end{frame}

\begin{frame}
  \frametitle{Totoloji Örneği}

  \begin{ornek}
    \begin{table}
      \caption{$p \wedge (p \rightarrow q) \rightarrow q$}
      \begin{tabular}{|c|c|c|c||c|}\hline
        $p$ & $q$ & $p \rightarrow q$ & $p \wedge A$ & $B \rightarrow q$\\
            &     & ($A$)             & ($B$)        &\\\hline\hline
        $D$ & $D$ & $D$ & $D$ & $D$\\\hline
        $D$ & $Y$ & $Y$ & $Y$ & $D$\\\hline
        $Y$ & $D$ & $D$ & $Y$ & $D$\\\hline
        $Y$ & $Y$ & $D$ & $Y$ & $D$\\\hline
      \end{tabular}
    \end{table}
  \end{ornek}
\end{frame}

\begin{frame}
  \frametitle{Çelişki Örneği}

  \begin{ornek}
    \begin{table}
      \caption{$p \wedge (\neg p \wedge q)$}
      \begin{tabular}{|c|c|c|c||c|}\hline
        $p$ & $q$ & $\neg p$ & $\neg p \wedge q$ & $p \wedge A$\\
            &     &          & ($A$)             &\\\hline\hline
        $D$ & $D$ & $Y$ & $Y$ & $Y$\\\hline
        $D$ & $Y$ & $Y$ & $Y$ & $Y$\\\hline
        $Y$ & $D$ & $D$ & $D$ & $Y$\\\hline
        $Y$ & $Y$ & $D$ & $Y$ & $Y$\\\hline
      \end{tabular}
    \end{table}
  \end{ornek}
\end{frame}

\subsection{Üstdil}

\begin{frame}
  \frametitle{Üstdil}

  \begin{tanim}
    \alert{hedef dil}:\\
      üzerinde çalışılan dil
  \end{tanim}

  \pause
  \begin{tanim}
    \alert{üstdil}:\\
      hedef dilin özelliklerinden söz ederken kullanılan dil
  \end{tanim}

  \pause
  \begin{itemize}
    \item geçerlilik, çelişkililik ve tutarlılık üstdile ait tanımlar
  \end{itemize}
\end{frame}

\begin{frame}
  \frametitle{Üstdil Örnekleri}

  \begin{ornek}[İngilizce öğrenen bir Türk için]
    \begin{itemize}
      \item hedef dil: İngilizce
      \item üstdil: Türkçe
    \end{itemize}
  \end{ornek}

  \pause
  \begin{ornek}[bir programlamaya giriş dersinde]
    \begin{itemize}
      \item hedef dil: C, Python, Java, \ldots
      \item üstdil: İngilizce, Türkçe, \ldots
    \end{itemize}
  \end{ornek}
\end{frame}

\begin{frame}
  \frametitle{Üstmantık}

  \begin{itemize}
    \item $P_1,P_2,\dots,P_n \vdash Q$\\
      $P_1,P_2,\dots,P_n$ varsayıldığında $Q$'nun doğruluğu tanıtlanabilir.

    \pause
    \medskip
    \item $P_1,P_2,\dots,P_n \vDash Q$\\
      $P_1,P_2,\dots,P_n$ doğruysa $Q$ doğrudur.
  \end{itemize}
\end{frame}

\begin{frame}
  \frametitle{Biçimsel Sistemler}

  \begin{tanim}
    \alert{tutarlı}: bütün $P$ ve $Q$ sağlıklı formülleri için\\
      $P \vdash Q$ ise $P \vDash Q$
    \begin{itemize}
      \item tanıtlanabilen bütün önermeler doğrudur
    \end{itemize}
  \end{tanim}

  \pause
  \begin{tanim}
    \alert{eksiksiz}: bütün $P$ ve $Q$ sağlıklı formülleri için\\
      $P \vDash Q$ ise $P \vdash Q$
    \begin{itemize}
      \item doğru olan bütün önermeler tanıtlanabilir
    \end{itemize}
  \end{tanim}
\end{frame}

\begin{frame}
  \frametitle{Gödel Kuramı}

  \begin{itemize}
    \item Önermeler mantığı tutarlı ve eksiksizdir.
  \end{itemize}

  \pause
  \begin{block}{Gödel Kuramı}
    \begin{itemize}
      \item Sıradan aritmetiği ifade edecek kadar güçlü\\
        hiçbir mantıksal sistem hem tutarlı hem eksiksiz olamaz.
    \end{itemize}
  \end{block}
\end{frame}

\section{Önerme Hesapları}

\subsection{Giriş}

\begin{frame}
  \frametitle{Önerme Hesabı Yaklaşımları}

  \begin{enumerate}
    \item anlamsal yaklaşım: \emph{doğruluk tabloları}
    \begin{itemize}
      \item değişken sayısı artınca yönetimi zorlaşıyor
    \end{itemize}

    \pause
    \item yazımsal yaklaşım: \emph{akıl yürütme kuralları}
    \begin{itemize}
      \item var olan önermelerden mantıksal gerektirmeler kullanarak\\
        yeni önermeler üretme
    \end{itemize}

    \pause
    \item aksiyomatik yaklaşım: \emph{Boole cebri}
    \begin{itemize}
      \item eşdeğerli formülleri denklemlerde birbirlerinin yerine koyma
    \end{itemize}
  \end{enumerate}
\end{frame}

\begin{frame}
  \frametitle{Doğruluk Tablosu Örneği}

  \begin{ornek}[$p \rightarrow q$]
    \begin{center}
      \begin{tabular}{|c|c||c|c|c|c|}\hline
        $p$ & $q$ & $p \rightarrow q$ & $\neg q \rightarrow \neg p$
            & $q \rightarrow p$ & $\neg p \rightarrow \neg q$\\\hline\hline
        $D$ & $D$ & $D$ & $D$ & $D$ & $D$\\\hline
        $D$ & $Y$ & $Y$ & $Y$ & $D$ & $D$\\\hline
        $Y$ & $D$ & $D$ & $D$ & $Y$ & $Y$\\\hline
        $Y$ & $Y$ & $D$ & $D$ & $D$ & $D$\\\hline
      \end{tabular}
    \end{center}

    \pause
    \begin{itemize}
      \item \emph{kontrapozitif}: $\neg q \rightarrow \neg p$

      \pause
      \item \emph{konvers}: $q \rightarrow p$

      \pause
      \item \emph{invers}: $\neg p \rightarrow \neg q$
    \end{itemize}
  \end{ornek}
\end{frame}

\subsection{Mantık Yasaları}

\begin{frame}
  \frametitle{Mantıksal Eşdeğerlilik}

  \begin{tanim}
    $P \leftrightarrow Q$ totoloji ise $P$ ve $Q$ \alert{mantıksal eşdeğerli}:\\
    $P \Leftrightarrow Q$
  \end{tanim}
\end{frame}

\begin{frame}
  \frametitle{Mantıksal Eşdeğerlilik Örneği}

  \begin{ornek}
    \begin{itemize}
      \item $\neg p \Leftrightarrow p \rightarrow Y$
    \end{itemize}

    \begin{table}
      \caption{$\neg p \leftrightarrow p \rightarrow Y$}
      \begin{tabular}{|c|c|c||c|}\hline
        $p$ & $\neg p$ & $p \rightarrow Y$ & $\neg p \leftrightarrow A$\\
            &          & ($A$)             &\\\hline\hline
        $D$ & $Y$ & $Y$ & $D$\\\hline
        $Y$ & $D$ & $D$ & $D$\\\hline
      \end{tabular}
    \end{table}
  \end{ornek}
\end{frame}

\begin{frame}
  \frametitle{Mantıksal Eşdeğerlilik Örneği}

  \begin{ornek}
    \begin{itemize}
      \item $p \rightarrow q \Leftrightarrow \neg p \vee q$
    \end{itemize}

    \begin{table}
      \caption{$(p \rightarrow q) \leftrightarrow (\neg p \vee q)$}
      \begin{tabular}{|c|c|c|c|c||c|}\hline
        $p$ & $q$ & $p \rightarrow q$ & $\neg p$ & $\neg p \vee q$ & $A \leftrightarrow B$\\
            &     & ($A$)             &          & ($B$)           &\\\hline\hline
        $D$ & $D$ & $D$ & $Y$ & $D$ & $D$\\\hline
        $D$ & $Y$ & $Y$ & $Y$ & $Y$ & $D$\\\hline
        $Y$ & $D$ & $D$ & $D$ & $D$ & $D$\\\hline
        $Y$ & $Y$ & $D$ & $D$ & $D$ & $D$\\\hline
      \end{tabular}
    \end{table}
  \end{ornek}
\end{frame}

\begin{frame}
  \frametitle{Mantık Yasaları}

  \begin{tabular}{ll}
  \alert{Çifte Değilleme (Double Negation - DN)} &\\
    $\neg (\neg p) \Leftrightarrow p$ &\\\\
  \pause
  \alert{Değişme (Commutativity - Co)} &\\
    $p \wedge q \Leftrightarrow q \wedge p$ &
    $p \vee q \Leftrightarrow q \vee p$\\\\
  \pause
  \alert{Birleşme (Associativity - As)} &\\
    $(p \wedge q) \wedge r \Leftrightarrow p \wedge (q \wedge r)$ &
    $(p \vee q) \vee r \Leftrightarrow p \vee (q \vee r)$\\\\
  \pause
  \alert{Sabit Kuvvetlilik (Idempotence - Ip)} &\\
    $p \wedge p \Leftrightarrow p$ &
    $p \vee p \Leftrightarrow p$\\\\
  \pause
  \alert{Terslik (Inverse - In)} &\\
    $p \wedge \neg p \Leftrightarrow Y$ &
    $p \vee \neg p \Leftrightarrow D$
  \end{tabular}
\end{frame}

\begin{frame}
  \frametitle{Mantık Yasaları}

  \begin{tabular}{ll}
  \alert{Etkisizlik (Identity - Id)} &\\
    $p \wedge D \Leftrightarrow p$ &
    $p \vee Y \Leftrightarrow p$\\\\
  \pause
  \alert{Baskınlık (Domination - Do)} &\\
    $p \wedge Y \Leftrightarrow Y$ &
    $p \vee D \Leftrightarrow D$\\\\
  \pause
  \alert{Dağılma (Distributivity - Di)} &\\
    $p \wedge (q \vee r) \Leftrightarrow (p \wedge q) \vee (p \wedge r)$ &
    $p \vee (q \wedge r) \Leftrightarrow (p \vee q) \wedge (p \vee r)$\\\\
  \pause
  \alert{Yutma (Absorption - Ab)} &\\
    $p \wedge (p \vee q) \Leftrightarrow p$ &
    $p \vee (p \wedge q) \Leftrightarrow p$\\\\
  \pause
  \alert{DeMorgan Yasaları (DM)} &\\
    $\neg (p \wedge q) \Leftrightarrow \neg p \vee \neg q$ &
    $\neg (p \vee q) \Leftrightarrow \neg p \wedge \neg q$
  \end{tabular}
\end{frame}

\begin{frame}
  \frametitle{Eşdeğerlilik Hesabı Örneği}

  \begin{ornek}
    \begin{eqnarray*}
                      & p \rightarrow q           &   \\
      \pause
      \Leftrightarrow & \neg p \vee q             &   \\
      \pause
      \Leftrightarrow & q \vee \neg p             & Co\\
      \pause
      \Leftrightarrow & \neg \neg q \vee \neg p   & DN\\
      \pause
      \Leftrightarrow & \neg q \rightarrow \neg p &
    \end{eqnarray*}
  \end{ornek}
\end{frame}

\begin{frame}
  \frametitle{Eşdeğerlilik Hesabı Örneği}

  \begin{ornek}
    \begin{eqnarray*}
                      & \neg (\neg ((p \vee q) \wedge r) \vee \neg q)      &   \\
      \pause
      \Leftrightarrow & \neg \neg ((p \vee q) \wedge r) \wedge \neg \neg q & DM\\
      \pause
      \Leftrightarrow & ((p \vee q) \wedge r) \wedge q                     & DN\\
      \pause
      \Leftrightarrow & (p \vee q) \wedge (r \wedge q)                     & As\\
      \pause
      \Leftrightarrow & (p \vee q) \wedge (q \wedge r)                     & Co\\
      \pause
      \Leftrightarrow & ((p \vee q) \wedge q) \wedge r                     & As\\
      \pause
      \Leftrightarrow & q \wedge r                                         & Ab
    \end{eqnarray*}
  \end{ornek}
\end{frame}

\begin{frame}
  \frametitle{Dualite}

  \begin{tanim}
    $\wedge$ ve $\vee$ dışında bir bağlaç içermeyen bir $s$ önermesinin\\
    \alert{dual} önermesi $s^d$,\\
    $\wedge$ yerine $\vee$, $\vee$ yerine $\wedge$,
    $D$ yerine $Y$, $Y$ yerine $D$\\
    konarak elde edilir.
  \end{tanim}

  \pause
  \begin{ornek}[dual önerme]
    \begin{eqnarray*}
      s:   & (p \wedge \neg q) \vee (r \wedge D)\\
      s^d: & (p \vee \neg q) \wedge (r \vee Y)
    \end{eqnarray*}
  \end{ornek}
\end{frame}

\begin{frame}
  \frametitle{Dualite İlkesi}

  \begin{block}{Dualite İlkesi}
    $s$ ve $t$, $\wedge$ ve $\vee$ dışında bir bağlaç içermeyen önermeler
    olsun.\\
    $s \Leftrightarrow t$ ise $s^d \Leftrightarrow t^d$.
  \end{block}
\end{frame}

\subsection{Akıl Yürütme}

\begin{frame}
  \frametitle{Mantıksal Gerektirme}

  \begin{tanim}
    $P \rightarrow Q$ bir totoloji ise $P$ formülü $Q$ formülünü
    \alert{mantıksal gerektirir}:\\
    $P \Rightarrow Q$
  \end{tanim}
\end{frame}

\begin{frame}
  \frametitle{Mantıksal Gerektirme Örneği}

  \begin{ornek}
    \begin{itemize}
      \item $p \wedge (p \rightarrow q) \Rightarrow q$
    \end{itemize}

    \begin{table}
      \caption{$p \wedge (p \rightarrow q) \rightarrow q$}
      \begin{tabular}{|c|c|c|c||c|}\hline
        $p$ & $q$ & $p \rightarrow q$ & $p \wedge A$ & $B \rightarrow q$\\
            &     & ($A$)             & ($B$)        &\\\hline\hline
        $D$ & $D$ & $D$ & $D$ & $D$\\\hline
        $D$ & $Y$ & $Y$ & $Y$ & $D$\\\hline
        $Y$ & $D$ & $D$ & $Y$ & $D$\\\hline
        $Y$ & $Y$ & $D$ & $Y$ & $D$\\\hline
      \end{tabular}
    \end{table}
  \end{ornek}
\end{frame}

\begin{frame}
  \frametitle{Akıl Yürütme}

  \begin{itemize}
    \item doğruluğu varsayılan ya da tanıtlanmış\\
      bir önermeler kümesinden yola çıkarak\\
      bir önermenin doğruluğuna varma
  \end{itemize}

  \pause
  \begin{block}{gösterilim}
    \begin{columns}
      \column{.5\textwidth}
      \[
      \frac
        {
          \begin{array}{c}
            p_1\\
            p_2\\
            \dots\\
            p_n
          \end{array}
        }
        {
          \therefore q
        }
      \]

      \column{.5\textwidth}
      $p_1 \wedge p_2 \wedge \cdots \wedge p_n \Rightarrow q$
    \end{columns}
  \end{block}
\end{frame}

\begin{frame}
  \frametitle{Temel Kurallar}

  \begin{block}{Özdeşlik (Identity - ID)}
    \[
    \frac
      {
        \begin{array}{c}
          p
        \end{array}
      }
      {
        \therefore p
      }
    \]
  \end{block}

  \pause
  \begin{block}{Çelişki (Contradiction - CTR)}
  \[
  \frac
    {
      \begin{array}{c}
        Y
      \end{array}
    }
    {
      \therefore p
    }
  \]
  \end{block}
\end{frame}

\begin{frame}
  \frametitle{Temel Kurallar}

  \begin{block}{Koşul Ekleme (Implication Introduction - ImpI)}
    \[
    \frac
      {
        \begin{array}{c}
          p \vdash q
        \end{array}
      }
      {
        \therefore ~ \vdash p \rightarrow q
      }
    \]
  \end{block}

  \begin{itemize}
    \item $p$ doğru varsayıldığında $q$ doğru olduğu gösterilebiliyorsa,\\
      \emph{$p$ doğru varsayılmadan} $p \rightarrow q$ doğrudur

    \pause
    \medskip
    \item $p$ bir \alert{geçici varsayım} (PA - provisional assumption)
    \item geçici varsayımlar sonradan kaldırılabilmeli
  \end{itemize}
\end{frame}

\begin{frame}
  \frametitle{Temel Kurallar}

  \begin{block}{VE Ekleme (AND Introduction - AndI)}
    \[
    \frac
      {
        \begin{array}{c}
          p\\
          q
        \end{array}
      }
      {
        \therefore p \wedge q
      }
    \]
  \end{block}

  \pause
  \begin{block}{VE Eleme (AND Elimination - AndE)}
  \[
  \frac
    {
      \begin{array}{c}
        p \wedge q
      \end{array}
    }
    {
      \therefore p
    }
  \]
  \end{block}
\end{frame}

\begin{frame}
  \frametitle{Temel Kurallar}

  \begin{block}{VEYA Ekleme (OR Introduction - OrI)}
    \[
    \frac
      {
        \begin{array}{c}
          p
        \end{array}
      }
      {
        \therefore p \vee q
      }
    \]
  \end{block}

  \pause
  \begin{block}{VEYA Eleme (OR Elimination - OrE)}
  \[
  \frac
    {
      \begin{array}{c}
        p \vee q\\
        p \vdash r\\
        q \vdash r
      \end{array}
    }
    {
      \therefore ~ \vdash r
    }
  \]
  \end{block}
\end{frame}

\begin{frame}
  \frametitle{Temel Kurallar}

  \begin{block}{Modus Ponens (Implication Elimination - ImpE)}
    \[
    \frac
      {
        \begin{array}{c}
          p \rightarrow q\\
          p
        \end{array}
      }
      {
        \therefore q
      }
    \]
  \end{block}

  \pause
  \begin{block}{Modus Tollens (MT)}
    \[
    \frac
      {
        \begin{array}{c}
          p \rightarrow q\\
          \neg q
        \end{array}
      }
      {
        \therefore \neg p
      }
    \]
  \end{block}
\end{frame}

\begin{frame}
  \frametitle{Modus Tollens}

  \begin{ornek}
    \begin{columns}
      \column{.3\textwidth}
      \[
      \frac
        {
          \begin{array}{c}
            p \rightarrow q\\
            \neg q
          \end{array}
        }
        {
          \therefore \neg p
        }
      \]

      \pause
      \column{.65\textwidth}
      \begin{eqnarray*}
        1. & p \rightarrow q           & A\\
        \pause
        2. & \neg q \rightarrow \neg p & 1\\
        \pause
        3. & \neg q                    & A\\
        \pause
        4. & \neg p                    & ImpE:2,3\\
      \end{eqnarray*}
    \end{columns}
  \end{ornek}
\end{frame}

\begin{frame}
  \frametitle{Modus Ponens Örneği}

  \begin{ornek}
    \begin{itemize}
      \item Ali piyangoyu kazanırsa araba alacak.
      \item Ali piyangoyu kazandı.

      \pause
      \medskip
      \item O halde, Ali araba alacak.
    \end{itemize}
  \end{ornek}
\end{frame}

\begin{frame}
  \frametitle{Modus Tollens Örneği}

  \begin{ornek}
    \begin{itemize}
      \item Ali piyangoyu kazanırsa araba alacak.
      \item Ali araba almadı.

      \pause
      \medskip
      \item O halde, Ali piyangoyu kazanmadı.
    \end{itemize}
  \end{ornek}
\end{frame}

\begin{frame}
  \frametitle{Yanılgılar}

  \begin{block}{sonucu onaylama yanılgısı}
    \[
    \frac
      {
      \begin{array}{c}
        p \rightarrow q\\
        q
        \end{array}
      }
      {
        \therefore p
      }
    \]
  \end{block}

  \pause
  \begin{itemize}
    \item $(p \rightarrow q) \wedge q \rightarrow p$ bir totoloji değil:\\
      $p=Y,q=D$ ise: $(Y \rightarrow D) \wedge D \rightarrow Y$
  \end{itemize}
\end{frame}

\begin{frame}
  \frametitle{Sonucu Onaylama Yanılgısı Örneği}

  \begin{ornek}
    \begin{itemize}
      \item Ali piyangoyu kazanırsa araba alacak.
      \item Ali araba aldı.

      \pause
      \medskip
      \item O halde, Ali piyangoyu kazandı.
    \end{itemize}
  \end{ornek}
\end{frame}

\begin{frame}
  \frametitle{Yanılgılar}

  \begin{block}{öncülü yadsıma yanılgısı}
    \[
    \frac
      {
        \begin{array}{c}
          p \rightarrow q\\
          \neg p
        \end{array}
      }
      {
        \therefore \neg q
      }
    \]
  \end{block}

  \pause
  \begin{itemize}
    \item $(p \rightarrow q) \wedge \neg p \rightarrow \neg q$ bir totoloji
      değil:\\
      $p=Y,q=D$ ise: $(Y \rightarrow D) \wedge D \rightarrow Y$
  \end{itemize}
\end{frame}

\begin{frame}
  \frametitle{Öncülü Yadsıma Yanılgısı Örneği}

  \begin{ornek}
    \begin{itemize}
      \item Ali piyangoyu kazanırsa araba alacak.
      \item Ali piyangoyu kazanmadı.

      \pause
      \medskip
      \item O halde, Ali araba almayacak.
    \end{itemize}
  \end{ornek}
\end{frame}

\begin{frame}
  \frametitle{Ayırıcı Kıyas}

  \begin{columns}
    \column{.5\textwidth}
    \begin{block}{Ayırıcı Kıyas\\
      (Disjunctive Syllogism - DS)}
      \[
      \frac
        {
          \begin{array}{c}
            p \vee q\\
            \neg p
          \end{array}
        }
        {
          \therefore q
        }
      \]
    \end{block}

    \pause
    \column{.5\textwidth}
    \begin{eqnarray*}
      1.   & p \vee q        & A\\
      \pause
      2.   & \neg p          & A\\
      \pause
      3.   & p \rightarrow Y & 2\\
      \pause
      4a1. & p               & PA\\
      \pause
      4a2. & Y               & ImpE:3,4a1\\
      \pause
      4a.  & q               & CTR:4a2\\
      \pause
      4b1. & q               & PA\\
      \pause
      4b.  & q               & ID:4b1\\
      \pause
      5.   & q               & OrE:1,4a,4b
    \end{eqnarray*}
  \end{columns}
\end{frame}

\begin{frame}
  \frametitle{Ayırıcı Kıyas Örneği}

  \begin{ornek}
    \begin{itemize}
      \item Ali'nin cüzdanı cebinde veya masasında.
      \item Ali'nin cüzdanı cebinde değil.

      \pause
      \medskip
      \item O halde, Ali'nin cüzdanı masasında.
    \end{itemize}
  \end{ornek}
\end{frame}

\begin{frame}
  \frametitle{Varsayımlı Kıyas}

  \begin{columns}
    \column{.5\textwidth}
    \begin{block}{Varsayımlı Kıyas\\
      (Hypothetical Syllogism - HS)}
      \[
      \frac
        {
          \begin{array}{c}
            p \rightarrow q\\
            q \rightarrow r
          \end{array}}
        {
          \therefore p \rightarrow r
        }
      \]
    \end{block}

    \pause
    \column{.5\textwidth}
    \begin{eqnarray*}
      1. & p               & PA\\
      \pause
      2. & p \rightarrow q & A\\
      \pause
      3. & q               & ImpE:2,1\\
      \pause
      4. & q \rightarrow r & A\\
      \pause
      5. & r               & ImpE:4,3\\
      \pause
      6. & p \rightarrow r & ImpI:1,5\\
    \end{eqnarray*}
  \end{columns}
\end{frame}

\begin{frame}
  \frametitle{Varsayımlı Kıyas Örneği}

  \begin{ornek}[Uzay Yolu]
    Spock - Yarbay Decker:
    \begin{quote}
      Şu anda düşman gemisine saldırmak intihar olur.\\
      İntihara teşebbüs eden biri Atılgan'ın komutanlığını yapmaya psikolojik
      olarak yetkin değildir.\\
      O halde, sizi görevden almak zorundayım.
    \end{quote}
  \end{ornek}
\end{frame}

\begin{frame}
  \frametitle{Varsayımlı Kıyas Örneği}

  \begin{ornek}[Uzay Yolu]
    \begin{itemize}
      \item $p$: Decker düşman gemisine saldırır.
      \item $q$: Decker intihara teşebbüs eder.
      \item $r$: Decker Atılgan'ın komutanlığını yapmaya psikolojik olarak
        yetkin değildir.
      \item $s$: Spock Decker'ı görevden alır.
    \end{itemize}
  \end{ornek}
\end{frame}

\begin{frame}
  \frametitle{Varsayımlı Kıyas Örneği}

  \begin{ornek}
    \begin{columns}
      \column{.3\textwidth}
      \[
      \frac
        {
          \begin{array}{c}
            p\\
            p \rightarrow q\\
            q \rightarrow r\\
            r \rightarrow s
          \end{array}
        }
        {
          \therefore s
        }
      \]

      \pause
      \column{.65\textwidth}
      \begin{eqnarray*}
        1. & p \rightarrow q & A\\
        \pause
        2. & q \rightarrow r & A\\
        \pause
        3. & p \rightarrow r & HS:1,2\\
        \pause
        4. & r \rightarrow s & A\\
        \pause
        5. & p \rightarrow s & HS:3,4\\
        \pause
        6. & p               & A\\
        \pause
        7. & s               & ImpE:5,6
      \end{eqnarray*}
    \end{columns}
  \end{ornek}
\end{frame}

% \begin{frame}
%   \frametitle{İkilemler}
%
%   \begin{columns}[t]
%     \column{.5\textwidth}
%     \begin{block}{Yapıcı İkilem}
%       \[
%       \frac
%         {
%           \begin{array}{c}
%             p \rightarrow q\\
%             r \rightarrow s\\
%             p \vee r
%           \end{array}}
%         {
%           \therefore q \vee s
%         }
%       \]
%     \end{block}
%
%     \pause
%     \column{.5\textwidth}
%     \begin{block}{Yıkıcı İkilem}
%       \[
%       \frac
%         {
%           \begin{array}{c}
%             p \rightarrow q\\
%             r \rightarrow s\\
%             \neg q \vee \neg s
%           \end{array}
%           }
%           {
%             \therefore \neg p \vee \neg r
%           }
%       \]
%     \end{block}
%   \end{columns}
% \end{frame}

\begin{frame}
  \frametitle{Akıl Yürütme Örnekleri}

  \begin{ornek}
    \begin{columns}[t]
      \column{.25\textwidth}
      \[
      \frac
        {
          \begin{array}{c}
            p \rightarrow r\\
            r \rightarrow s\\
            x \vee \neg s\\
            u \vee \neg x\\
            \neg u
          \end{array}
        }
        {
          \therefore \neg p
        }
      \]

      \pause
      \column{.3\textwidth}
      \begin{eqnarray*}
        1. & u \vee \neg x   & A\\
        \pause
        2. & \neg u          & A\\
        \pause
        3. & \neg x          & DS:1,2\\
        \pause
        4. & x \vee \neg s   & A\\
        \pause
        5. & \neg s          & DS:4,3\\
      \end{eqnarray*}

      \pause
      \column{.45\textwidth}
      \begin{eqnarray*}
        6. & r \rightarrow s & A\\
        \pause
        7. & \neg r          & MT:6,5\\
        \pause
        8. & p \rightarrow r & A\\
        \pause
        9. & \neg p          & MT:8,7\\
      \end{eqnarray*}
    \end{columns}
  \end{ornek}
\end{frame}

\begin{frame}
  \frametitle{Akıl Yürütme Örnekleri}

  \begin{ornek}
    \[
    \frac
      {
        \begin{array}{c}
          (\neg p \vee \neg q) \rightarrow (r \wedge s)\\
          r \rightarrow x\\
          \neg x
        \end{array}
      }
      {
        \therefore p
      }
    \]

    \pause
    \begin{columns}[t]
      \column{.3\textwidth}
      \begin{eqnarray*}
        1. & r \rightarrow x                               & A\\
        \pause
        2. & \neg x                                        & A\\
        \pause
        3. & \neg r                                        & MT:1,2\\
        \pause
        4. & \neg r \vee \neg s                            & OrI:3\\
        \pause
        5. & \neg (r \wedge s)                             & DM:4
      \end{eqnarray*}

      \pause
      \column{.6\textwidth}
      \begin{eqnarray*}
        6. & (\neg p \vee \neg q) \rightarrow (r \wedge s) & A\\
        \pause
        7. & \neg (\neg p \vee \neg q)                     & MT:6,5\\
        \pause
        8. & p \wedge q                                    & DM:7\\
        \pause
        9. & p                                             & AndE:8
      \end{eqnarray*}
    \end{columns}
  \end{ornek}
\end{frame}

\begin{frame}
  \frametitle{Akıl Yürütme Örnekleri}

  \begin{ornek}
    \begin{columns}
      \column{.3\textwidth}
      \[
      \frac
        {
          \begin{array}{c}
            p \rightarrow (q \vee r)\\
            s \rightarrow \neg r\\
            q \rightarrow \neg p\\
            p\\
            s
          \end{array}
        }
        {
          \therefore Y
        }
      \]

      \pause
      \column{.6\textwidth}
      \begin{eqnarray*}
        1. & q \rightarrow \neg p     & A\\
       \pause
        2. & p                        & A\\
       \pause
        3. & \neg q                   & MT:1,2\\
       \pause
        4. & s                        & A\\
       \pause
        5. & s \rightarrow \neg r     & A\\
       \pause
        6. & \neg r                   & ImpE:5,4\\
       \pause
        7. & p \rightarrow (q \vee r) & A\\
       \pause
        8. & q \vee r                 & ImpE:7,2\\
       \pause
        9. & q                        & DS:8,6\\
       \pause
       10. & q \wedge \neg q : Y      & AndI:9,3
      \end{eqnarray*}
    \end{columns}
  \end{ornek}
\end{frame}

\begin{frame}
  \frametitle{Akıl Yürütme Örnekleri}

  \begin{ornek}
    Eğer yağmur yağması olasılığı varsa veya saç bandını bulamazsa,\\
    Filiz çimleri biçmez. Hava sıcaklığı 20~dereceden fazlaysa\\
    yağmur yağma olasılığı yoktur. Bugün hava sıcaklığı 22~derece\\
    ve Filiz saç bandını takmış. O halde, Filiz çimleri biçecek.
  \end{ornek}
\end{frame}

\begin{frame}
  \frametitle{Akıl Yürütme Örnekleri}

  \begin{ornek}
    \begin{itemize}
      \item $p$: Yağmur yağabilir.
      \item $q$: Filiz'in saç bandı kayıp.
      \item $r$: Filiz çimleri biçer.
      \item $s$: Hava sıcaklığı 20 dereceden fazla.
    \end{itemize}
  \end{ornek}
\end{frame}

\begin{frame}
  \frametitle{Akıl Yürütme Örnekleri}

  \begin{ornek}
    \begin{columns}
      \column{.3\textwidth}
      \[
      \frac
        {
          \begin{array}{c}
            (p \vee q) \rightarrow \neg r\\
            s \rightarrow \neg p\\
            s \wedge \neg q
          \end{array}
        }
        {
          \therefore r
        }
      \]

      \pause
      \column{.65\textwidth}
      \begin{eqnarray*}
        1. & s \wedge \neg q                & A\\
        \pause
        2. & s                              & AndE:1\\
        \pause
        3. & s \rightarrow \neg p           & A\\
        \pause
        4. & \neg p                         & ImpE:3,2\\
        \pause
        5. & \neg q                         & AndE:1\\
        \pause
        6. & \neg p \wedge \neg q           & AndI:4,5\\
        \pause
        7. & \neg (p \vee q)                & DM:6\\
        \pause
        8. & (p \vee q) \rightarrow \neg r  & A\\
        \pause
        9. & ?                              & 7,8
      \end{eqnarray*}
    \end{columns}
  \end{ornek}
\end{frame}

\section*{Kaynaklar}

\begin{frame}
  \frametitle{Kaynaklar}

  \begin{block}{Okunacak: Grimaldi}
    \begin{itemize}
      \item Chapter 2: Fundamentals of Logic
      \begin{itemize}
        \item 2.1. \alert{Basic Connectives and Truth Tables}
        \item 2.2. \alert{Logical Equivalence: The Laws of Logic}\\
        \item 2.3. \alert{Logical Implication: Rules of Inference}
      \end{itemize}
    \end{itemize}
  \end{block}

  \begin{block}{Yardımcı Kitap: O'Donnell, Hall, Page}
    \begin{itemize}
      \item Chapter 6: Propositional Logic
    \end{itemize}
  \end{block}
\end{frame}

\end{document}
