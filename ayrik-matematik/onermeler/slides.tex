% Copyright (c) 2001-2010
%       H. Turgut Uyar <uyar@itu.edu.tr>
%       Ayşegül Gençata Yayımlı <gencata@itu.edu.tr>
%       Emre Harmancı <harmanci@itu.edu.tr>
%
% Bu notlar "Creative Commons Attribution-NonCommercial-ShareAlike License" ile
% lisanslanmıştır. Yazarının açıkça belirtilmesi koşuluyla ve ticari olmayan
% amaçlarla kullanılabilir ve dağıtılabilir. Bu notlardan yola çıkılarak
% oluşturulacak çalışmaların da aynı lisansa bağlı olmaları gerekir.
%
% Lisans ile ilgili ayrıntılı bilgi almak için şu sayfaya başvurabilirsiniz:
% http://creativecommons.org/licenses/by-nc-sa/3.0/

\documentclass[dvipsnames]{beamer}

\usepackage{ae}
\usepackage[T1]{fontenc}
\usepackage[utf8]{inputenc}
\usepackage[turkish]{babel}
\setbeamertemplate{navigation symbols}{}

\mode<presentation>
{
  \usetheme{Rochester}
  \setbeamercovered{transparent}
}

\title{Ayrık Matematik}
\subtitle{Önermeler}

\author{H. Turgut Uyar \and Ayşegül Gençata Yayımlı \and Emre Harmancı}
\date{2001-2010}

\AtBeginSubsection[]
{
  \begin{frame}<beamer>
    \frametitle{Konular}
    \tableofcontents[currentsection,currentsubsection]
  \end{frame}
}

%\beamerdefaultoverlayspecification{<+->}

\theoremstyle{definition}
\newtheorem{tanim}[theorem]{Tanım}

\theoremstyle{example}
\newtheorem{ornek}[theorem]{Örnek}

\theoremstyle{plain}
\newtheorem{teorem}[theorem]{Teorem}

\pgfdeclareimage[width=2cm]{license}{../../license}

\begin{document}

\begin{frame}
  \titlepage
\end{frame}

\begin{frame}
  \frametitle{Lisans}

  \pgfuseimage{license}\hfill
  \copyright 2001-2010 T. Uyar, A. Yayımlı, E. Harmancı

  \vfill
  \begin{tiny}
    You are free:
    \begin{itemize}
      \item to Share — to copy, distribute and transmit the work
      \item to Remix — to adapt the work
    \end{itemize}

    Under the following conditions:
    \begin{itemize}
      \item Attribution — You must attribute the work in the manner specified by
        the author or licensor (but not in any way that suggests that they
        endorse you or your use of the work).

      \item Noncommercial — You may not use this work for commercial purposes.

      \item Share Alike — If you alter, transform, or build upon this work, you
        may distribute the resulting work only under the same or similar license
        to this one.
    \end{itemize}
  \end{tiny}

  \vfill
  Legal code (the full license):\\
  \url{http://creativecommons.org/licenses/by-nc-sa/3.0/}
\end{frame}

\begin{frame}
  \frametitle{Konular}
  \tableofcontents
\end{frame}

\section{Önermeler}

\subsection{Giriş}

\begin{frame}
  \frametitle{Önerme}

  \begin{tanim}
    \alert{önerme}: bir dilde yapılan bildirim
  \end{tanim}

  \pause
  \begin{itemize}
    \item \alert{ara değeri dışlama kuralı}:\\
      bir önerme kısmen doğru ya da kısmen yanlış olamaz
  \end{itemize}

  \pause
  \begin{itemize}
    \item \alert{çelişki kuralı}:\\
      bir önerme hem doğru hem yanlış olamaz
  \end{itemize}
\end{frame}

\begin{frame}
  \frametitle{Önerme Örnekleri}

  \begin{columns}[t]
    \column{.55\textwidth}
    \begin{ornek}[önerme]
      \begin{itemize}
        \item Ay dünyanın çevresinde döner.
        \item Filler uçabilir.
        \item $3+8=11$
      \end{itemize}
    \end{ornek}

    \pause
    \column{.45\textwidth}
    \begin{ornek}[önerme değil]
      \begin{itemize}
        \item Saat kaç?
        \item Ali topu at!
        \item $x<43$
      \end{itemize}
    \end{ornek}
  \end{columns}
\end{frame}

\begin{frame}
  \frametitle{Önerme Değişkeni}

  \begin{tanim}
    \alert{önerme değişkeni}: önermeyi simgeleyen isim

    \begin{itemize}
      \item \emph{Doğru} ($D$) ya da \emph{Yanlış} ($Y$) değerini alır
    \end{itemize}
  \end{tanim}

  \pause
  \begin{ornek}
    \begin{itemize}
      \item $p_1$: Ay dünyanın çevresinde döner. ($D$)
      \item $p_2$: Filler uçabilir. ($Y$)
      \item $p_3$: $3+8=11$ ($D$)
    \end{itemize}
  \end{ornek}
\end{frame}

\subsection{Bağlaçlar}

\begin{frame}
  \frametitle{Birleşik Önerme}

  \begin{itemize}
    \item yalın önermelerin \alert{bağlaçlar} ile bağlanmasıyla
      \alert{birleşik önermeler} elde edilir
    \begin{itemize}
      \item değil
      \item ve, veya
      \item koşullu bağlaç, karşılıklı koşullu bağlaç
    \end{itemize}
  \end{itemize}

  \pause
  \begin{itemize}
    \item \alert{doğruluk tablosu}:\\
      önerme değişkenlerinin olası bütün değerleri için bağlaç sonucunu veren
      çizelge
  \end{itemize}
\end{frame}

\begin{frame}
  \frametitle{DEĞİL Bağlacı}

  \begin{columns}
    \column{.3\textwidth}
    \begin{table}
      \caption{$\neg p$}
      \begin{tabular}{|c||c|}\hline
        $p$ & $\neg p$\\\hline\hline
        $D$ & $Y$     \\\hline
        $Y$ & $D$     \\\hline
      \end{tabular}
    \end{table}

    \pause
    \column{.7\textwidth}
    \begin{ornek}
      \begin{itemize}
        \item $\neg p_1$: Ay dünyanın çevresinde dönmez.\\
          $\neg D$: \emph{Yanlış}
        \item $\neg p_2$: Filler uçamaz.\\
          $\neg Y$: \emph{Doğru}
      \end{itemize}
    \end{ornek}
  \end{columns}
\end{frame}

\begin{frame}
  \frametitle{VE Bağlacı}

  \begin{columns}
    \column{.4\textwidth}
    \begin{table}
      \caption{$p \wedge q$}
      \begin{tabular}{|c|c||c|}\hline
        $p$ & $q$ & $p \wedge q$\\\hline\hline
        $D$ & $D$ & $D$         \\\hline
        $D$ & $Y$ & $Y$         \\\hline
        $Y$ & $D$ & $Y$         \\\hline
        $Y$ & $Y$ & $Y$         \\\hline
      \end{tabular}
    \end{table}

    \pause
    \column{.6\textwidth}
    \begin{ornek}
      \begin{itemize}
        \item $p_1 \wedge p_2$: Ay dünyanın çevresinde döner ve filler
          uçabilir.\\
          $D \wedge Y$: \emph{Yanlış}
      \end{itemize}
    \end{ornek}
  \end{columns}
\end{frame}

\begin{frame}
  \frametitle{VEYA Bağlacı}

  \begin{columns}
    \column{.4\textwidth}
    \begin{table}
      \caption{$p \vee q$}
      \begin{tabular}{|c|c||c|}\hline
        $p$ & $q$ & $p \vee q$\\\hline\hline
        $D$ & $D$ & $D$       \\\hline
        $D$ & $Y$ & $D$       \\\hline
        $Y$ & $D$ & $D$       \\\hline
        $Y$ & $Y$ & $Y$       \\\hline
      \end{tabular}
    \end{table}

    \pause
    \column{.6\textwidth}
    \begin{ornek}
      \begin{itemize}
        \item $p_1 \vee p_2$: Ay dünyanın çevresinde döner veya filler
          uçabilir.\\
          $D \vee Y$: \emph{Doğru}
      \end{itemize}
    \end{ornek}
  \end{columns}
\end{frame}

\begin{frame}
  \frametitle{DAR VEYA Bağlacı}

  \begin{columns}
    \column{.4\textwidth}
    \begin{table}
      \caption{$p \veebar q$}
      \begin{tabular}{|c|c||c|}\hline
        $p$ & $q$ & $p \veebar q$\\\hline\hline
        $D$ & $D$ & $Y$          \\\hline
        $D$ & $Y$ & $D$          \\\hline
        $Y$ & $D$ & $D$          \\\hline
        $Y$ & $Y$ & $Y$          \\\hline
      \end{tabular}
    \end{table}

    \pause
    \column{.6\textwidth}
    \begin{ornek}
      \begin{itemize}
        \item $p_1 \veebar p_2$: Ya ay dünyanın çevresinde döner ya da filler
          uçabilir.\\
          $D \veebar Y$: \emph{Doğru}
      \end{itemize}
    \end{ornek}
  \end{columns}
\end{frame}

\begin{frame}
  \frametitle{Koşullu Bağlaç}

  \begin{columns}
    \column{.4\textwidth}
    \begin{table}
      \caption{$p \rightarrow q$}
      \begin{tabular}{|c|c||c|}\hline
        $p$ & $q$ & $p \rightarrow q$\\\hline\hline
        $D$ & $D$ & $D$              \\\hline
        $D$ & $Y$ & $Y$              \\\hline
        $Y$ & $D$ & $D$              \\\hline
        $Y$ & $Y$ & $D$              \\\hline
      \end{tabular}
    \end{table}

    \pause
    \column{.6\textwidth}
    \begin{itemize}
      \item $p$: \alert{öncül}
      \item $q$: \alert{sonuç}

      \item okunuşları:
      \begin{itemize}
        \item $p$ ise $q$
        \item $p$, $q$ için yeterli
        \item $q$, $p$ için gerekli
      \end{itemize}

      \pause
      \item $\neg p \vee q$
    \end{itemize}
  \end{columns}
\end{frame}

\begin{frame}
  \frametitle{Koşullu Bağlaç Örnekleri}

  \begin{ornek}
    \begin{itemize}
      \item $p_4$: $3<8$, $p_5$: $3<14$, $p_6$: $3<2$
      \item $p_7$: Güneş dünyanın çevresinde döner.
    \end{itemize}

    \pause
    \begin{columns}[t]
      \column{.47\textwidth}
        \begin{itemize}
          \item $p_4 \rightarrow p_5$: 3, 8'den küçükse\\
            3, 14'den küçüktür.\\
            $D \rightarrow D$: \emph{Doğru}
          \pause
          \item $p_4 \rightarrow p_6$: 3, 8'den küçükse\\
            3, 2'den küçüktür.\\
            $D \rightarrow Y$: \emph{Yanlış}
        \end{itemize}

      \pause
      \column{.5\textwidth}
        \begin{itemize}
          \item $p_2 \rightarrow p_1$: Filler uçabilirse ay dünyanın çevresinde
            döner.\\
            $Y \rightarrow D$: \emph{Doğru}
          \pause
          \item $p_2 \rightarrow p_7$: Filler uçabilirse güneş dünyanın
            çevresinde döner.\\
            $Y \rightarrow Y$: \emph{Doğru}
        \end{itemize}
    \end{columns}
  \end{ornek}
\end{frame}

\begin{frame}
  \frametitle{Koşullu Bağlaç Örnekleri}

  \begin{ornek}
    \begin{columns}
      \column{.6\textwidth}
      \begin{itemize}
        \item "70 kg'yi geçersem spor yapacağım."
        \begin{itemize}
          \item $p$: 70 kg'den ağırım.
          \item $q$: Spor yapıyorum.
        \end{itemize}

        \pause
        \item $p \rightarrow q$ nasıl yorumlanmalı?
      \end{itemize}

      \column{.4\textwidth}
      \begin{table}
        \caption{$p \rightarrow q$}
        \begin{tabular}{|c|c||c|}\hline
          $p$ & $q$ & $p \rightarrow q$\\\hline\hline
          $D$ & $D$ & $D$              \\\hline
          $D$ & $Y$ & $Y$              \\\hline
          $Y$ & $D$ & $D$              \\\hline
          $Y$ & $Y$ & $D$              \\\hline
        \end{tabular}
      \end{table}
    \end{columns}
  \end{ornek}
\end{frame}

\begin{frame}
  \frametitle{Karşılıklı Koşullu Bağlaç}

  \begin{columns}
    \column{.4\textwidth}
    \begin{table}
      \caption{$p \leftrightarrow q$}
      \begin{tabular}{|c|c||c|}\hline
        $p$ & $q$ & $p \leftrightarrow q$\\\hline\hline
        $D$ & $D$ & $D$                  \\\hline
        $D$ & $Y$ & $Y$                  \\\hline
        $Y$ & $D$ & $Y$                  \\\hline
        $Y$ & $Y$ & $D$                  \\\hline
      \end{tabular}
    \end{table}

    \pause
    \column{.6\textwidth}
    \begin{itemize}
      \item okunuşları:
      \begin{itemize}
        \item $p$ yalnız ve ancak $q$ ise
        \item $p$, $q$ için yeterli ve gerekli
      \end{itemize}

      \pause
      \item $(p \rightarrow q) \wedge (q \rightarrow p)$
      \item $\neg (p \veebar q)$
    \end{itemize}
  \end{columns}
\end{frame}

\begin{frame}
  \frametitle{Günlük Dilden Örnek}

  \begin{ornek}
    \begin{itemize}
      \item $s$: Çocuk ödevini yapar.
      \item $t$: Çocuk bilgisayar oyunu oynar.

      \pause
      \medskip
      \item $s \rightarrow t$\\
        "Ödevini yaparsan bilgisayar oyunu oynayabilirsin."
      \item $t \rightarrow s$\\
        "Bilgisayar oyunu oynaman için ödevini yapman gerek."

      \pause
      \medskip
      \item söylenmek istenen: $s \leftrightarrow t$
    \end{itemize}
  \end{ornek}
\end{frame}

\subsection{Sağlıklı Formüller}

\begin{frame}
  \frametitle{Sağlıklı Formül}

  \begin{block}{Yazım}
    \begin{itemize}
      \item birleşik önermeler hangi kurallara göre oluşturulacak?
      \item kurallara uyan formüller: \alert{sağlıklı formül} (SF)
    \end{itemize}
  \end{block}

  \pause
  \begin{block}{Anlam}
    \begin{itemize}
      \item \emph{yorum}: yalın önermelere değer vererek birleşik önermenin
        değerini hesaplama
      \item doğruluk tablosu: önermenin bütün yorumları
    \end{itemize}
  \end{block}
\end{frame}

\begin{frame}
  \frametitle{Sağlıklı Formül Örnekleri}

  \begin{ornek}[sağlıklı formül değil]
    \begin{itemize}
      \item $\vee p$
      \item $p \wedge \neg$
      \item $p \neg \wedge q$
    \end{itemize}
  \end{ornek}
\end{frame}

\begin{frame}
  \frametitle{Öncelik Sırası}

  \begin{enumerate}
    \item $\neg$
    \item $\wedge$
    \item $\vee$
    \item $\rightarrow$
    \item $\leftrightarrow$
  \end{enumerate}

  \begin{itemize}
    \item önceliği değiştirmek için parantez kullanılır
  \end{itemize}
\end{frame}

\begin{frame}
  \frametitle{Öncelik Sırası Örnekleri}

  \begin{ornek}
    \begin{itemize}
      \item $s$: Filiz gezmeye çıkar.
      \item $t$: Mehtap var.
      \item $u$: Kar yağıyor.
    \end{itemize}

    \medskip
    aşağıdaki SF'ler ne anlama gelir?

    \pause
    \begin{itemize}
      \item $t \wedge \neg u \rightarrow s$
      \pause
      \item $t \rightarrow (\neg u \rightarrow s)$
      \pause
      \item $\neg (s \leftrightarrow (u \vee t))$
      \pause
      \item $\neg s \leftrightarrow u \vee t$
    \end{itemize}
  \end{ornek}
\end{frame}

\begin{frame}
  \frametitle{Formül Nitelikleri}

  \begin{enumerate}
    \item \emph{geçerli}: bütün yorumlar için doğru (\alert{totoloji})
    \item \emph{çelişkili}: bütün yorumlar için yanlış (\alert{çelişki})
    \item \emph{tutarlı}: bazı yorumlar için doğru
  \end{enumerate}
\end{frame}

\begin{frame}
  \frametitle{Totoloji Örneği}

  \begin{ornek}
    \begin{table}
      \caption{$p \wedge (p \rightarrow q) \rightarrow q$}
      \begin{tabular}{|c|c|c|c||c|}\hline
        $p$ & $q$ & $p \rightarrow q$ & $p \wedge (p \rightarrow q)$
            & $p \wedge (p \rightarrow q) \rightarrow q$\\\hline\hline
        $D$ & $D$ & $D$ & $D$ & $D$                     \\\hline
        $D$ & $Y$ & $Y$ & $Y$ & $D$                     \\\hline
        $Y$ & $D$ & $D$ & $Y$ & $D$                     \\\hline
        $Y$ & $Y$ & $D$ & $Y$ & $D$                     \\\hline
      \end{tabular}
    \end{table}
  \end{ornek}
\end{frame}

\begin{frame}
  \frametitle{Çelişki Örneği}

  \begin{ornek}
    \begin{table}
      \caption{$p \wedge (\neg p \wedge q)$}
      \begin{tabular}{|c|c|c|c||c|}\hline
        $p$ & $q$ & $\neg p$ & $\neg p \wedge q$
            & $p \wedge (\neg p \wedge q)$\\\hline\hline
        $D$ & $D$ & $Y$ & $Y$ & $Y$       \\\hline
        $D$ & $Y$ & $Y$ & $Y$ & $Y$       \\\hline
        $Y$ & $D$ & $D$ & $D$ & $Y$       \\\hline
        $Y$ & $Y$ & $D$ & $Y$ & $Y$       \\\hline
      \end{tabular}
    \end{table}
  \end{ornek}
\end{frame}

\subsection{Üstdil}

\begin{frame}
  \frametitle{Üstdil}

  \begin{tanim}
    \alert{hedef dil}: üzerinde çalışılan dil
  \end{tanim}

  \pause
  \begin{tanim}
    \alert{üstdil}: hedef dilin özelliklerinden söz ederken kullanılan dil
  \end{tanim}

  \pause
  \begin{itemize}
    \item geçerlilik, çelişkililik ve tutarlılık üstdile ait tanımlar
  \end{itemize}
\end{frame}

\begin{frame}
  \frametitle{Üstdil Örnekleri}

  \begin{ornek}[İngilizce öğrenen biri için]
    \begin{itemize}
      \item hedef dil: İngilizce
      \item üstdil: Türkçe
    \end{itemize}
  \end{ornek}

  \pause
  \begin{ornek}[Intro. to Sci. and Eng. Comp.]
    \begin{itemize}
      \item hedef dil: C
      \item üstdil: İngilizce
    \end{itemize}
  \end{ornek}
\end{frame}

\begin{frame}
  \frametitle{Üstmantık}

  \begin{itemize}
    \item $P_1,P_2,\dots,P_n \vdash Q$\\
      $P_1,P_2,\dots,P_n$ varsayıldığında $Q$'nun doğruluğu tanıtlanabilir

    \pause
    \medskip
    \item $P_1,P_2,\dots,P_n \vDash Q$\\
      $P_1,P_2,\dots,P_n$ doğruysa $Q$ doğrudur
  \end{itemize}
\end{frame}

\begin{frame}
  \frametitle{Biçimsel Sistemler}

  \begin{tanim}
    \alert{tutarlı}: bütün $P$ ve $Q$ sağlıklı formülleri için\\
      $P \vdash Q$ ise $P \vDash Q$
    \begin{itemize}
      \item tanıtlanabilen her şey doğrudur
    \end{itemize}
  \end{tanim}

  \pause
  \begin{tanim}
    \alert{eksiksiz}: bütün $P$ ve $Q$ sağlıklı formülleri için\\
      $P \vDash Q$ ise $P \vdash Q$
    \begin{itemize}
      \item doğru olan her şey tanıtlanabilir
    \end{itemize}
  \end{tanim}
\end{frame}

\begin{frame}
  \frametitle{Gödel Kuramı}

  \begin{itemize}
    \item önermeler mantığı tutarlı ve eksiksizdir
    \item yüklemler mantığı tutarlı ve eksiksizdir

    \pause
    \item matematik tutarlı ama eksiktir
  \end{itemize}

  \pause
  \begin{block}{Gödel Kuramı}
    \begin{itemize}
      \item Matematiğin bütününü ifade edecek bir sistem hem tutarlı hem
        eksiksiz olamaz!
    \end{itemize}
  \end{block}
\end{frame}
%
% \section*{Örnek}
%
% \begin{frame}
%   \frametitle{Örnek}
%
%   \begin{ornek}[antropolog - yerli]
%     \begin{itemize}
%       \item antropolog yol ayrımına gelir
%       \begin{itemize}
%         \item köye giden yol hangisi?
%       \end{itemize}
%
%       \item yol ayrımında bir yerli
%       \begin{itemize}
%         \item iki kabile var: doğrucular ve yalancılar
%       \end{itemize}
%
%       \pause
%       \item köye giden yolu öğrenmek için ne sormalı?
%     \end{itemize}
%   \end{ornek}
% \end{frame}
%
% \begin{frame}
%   \frametitle{Örnek}
%
%   \begin{ornek}[antropolog - yerli]
%     \begin{itemize}
%       \item Bu yol köye gidiyor mu diye sorsam "Evet" der misin?
%
%       \pause
%       \medskip
%       \begin{table}
%         \begin{tabular}{|l|l||l|}\hline
%           Yol         & Yerli   & Yanıt\\\hline\hline
%           Köye gider  & Doğrucu & Evet \\\hline
%           Köye gider  & Yalancı & Evet \\\hline
%           Köye gitmez & Doğrucu & Hayır\\\hline
%           Köye gitmez & Yalancı & Hayır\\\hline
%         \end{tabular}
%       \end{table}
%     \end{itemize}
%   \end{ornek}
% \end{frame}
%
% \begin{frame}
%   \frametitle{Örnek}
%
%   \begin{ornek}[antropolog - yerli]
%     \begin{itemize}
%       \item "bu yol köye gider mi?" diye soruyor olsa gerek
%       \item belirsizliği kaldıralım
%     \end{itemize}
%
%     \pause
%     \medskip
%     \begin{columns}
%       \column{.5\textwidth}
%       \begin{itemize}
%         \item $p$: Yol köye gider.
%         \item $q$: Yerli yalancı kabileden.
%         \item $p \veebar q$
%       \end{itemize}
%
%       \pause
%       \column{.5\textwidth}
%       \begin{table}
%         \begin{tabular}{|c|c|c||c|}\hline
%           $p$ & $q$ & $p \veebar q$ & Yanıt\\\hline\hline
%           $D$ & $D$ & $Y$           & $D$  \\\hline
%           $D$ & $Y$ & $D$           & $D$  \\\hline
%           $Y$ & $D$ & $D$           & $Y$  \\\hline
%           $Y$ & $Y$ & $Y$           & $Y$  \\\hline
%         \end{tabular}
%       \end{table}
%     \end{columns}
%   \end{ornek}
% \end{frame}
%
% \begin{frame}
%   \frametitle{Örnek}
%
%   \begin{ornek}[antropolog - yerli]
%     \begin{itemize}
%       \item düz yalancı: hesapla, değilini söyle
%       \item dürüst yalancı: değilini al, hesapla, değilini söyle
%     \end{itemize}
%
%     \pause
%     \medskip
%     \begin{table}
%       \caption{$p \rightarrow q$}
%       \begin{tabular}{|c|c|c|c|c|c||c|c|}\hline
%         $p$ & $q$ & $p \rightarrow q$
%             & $\neg p$ & $\neg q$ & $\neg p \rightarrow \neg q$
%             & Dürüst & Düz\\\hline\hline
%         $D$ & $D$ & $D$
%             & $Y$ & $Y$ & $D$
%             & $Y$ & $Y$\\\hline
%         $D$ & $Y$ & $Y$
%             & $Y$ & $D$ & $D$
%             & $Y$ & $Y$\\\hline
%         $Y$ & $D$ & $D$
%             & $D$ & $Y$ & $Y$
%             & $D$ & $Y$\\\hline
%         $Y$ & $Y$ & $D$
%             & $D$ & $D$ & $D$
%             & $D$ & $D$\\\hline
%       \end{tabular}
%     \end{table}
%   \end{ornek}
% \end{frame}
%
% \begin{frame}
%   \frametitle{Örnek}
%
%   \begin{ornek}[antropolog - yerli]
%     \begin{table}
%       \caption{$p \leftrightarrow q$}
%       \begin{tabular}{|c|c|c|c|c|c||c|c|}\hline
%         $p$ & $q$ & $p \leftrightarrow q$
%             & $\neg p$ & $\neg q$ & $\neg p \leftrightarrow \neg q$
%             & Dürüst & Düz\\\hline\hline
%         $D$ & $D$ & $D$
%             & $Y$ & $Y$ & $D$
%             & $Y$ & $Y$\\\hline
%         $D$ & $Y$ & $Y$
%             & $Y$ & $D$ & $Y$
%             & $Y$ & $Y$\\\hline
%         $Y$ & $D$ & $Y$
%             & $D$ & $Y$ & $Y$
%             & $D$ & $D$\\\hline
%         $Y$ & $Y$ & $D$
%             & $D$ & $D$ & $D$
%             & $D$ & $D$\\\hline
%       \end{tabular}
%     \end{table}
%   \end{ornek}
% \end{frame}
%
% \begin{frame}
%   \frametitle{Örnek}
%
%   \begin{ornek}[antropolog - yerli]
%     \begin{itemize}
%       \item yalan sanatçısı: amaç yanıltmak
%
%       \pause
%       \medskip
%       \item "köyde bedava bira dağıtıldığını biliyor muydun?"
%       \begin{itemize}
%         \item doğrucu: "hayır" deyip köye
%         \item düz/dürüst yalancı: "evet" deyip köye
%         \item yalan sanatçısı: "bira sevmem" deyip köye ya da diğer yola
%       \end{itemize}
%     \end{itemize}
%   \end{ornek}
% \end{frame}

\section{Önerme Hesapları}

\subsection{Giriş}

\begin{frame}
  \frametitle{Önerme Hesabı Yaklaşımları}

  \begin{enumerate}
    \item anlamsal yaklaşım: \emph{doğruluk tabloları}
    \begin{itemize}
      \item değişken sayısı artınca yönetimi zorlaşıyor
    \end{itemize}

    \pause
    \item yazımsal yaklaşım: \emph{akıl yürütme kuralları}
    \begin{itemize}
      \item önermelerden mantıksal gerektirmeler yoluyla yeni önermeler üretme
    \end{itemize}

    \pause
    \item aksiyomatik yaklaşım: \emph{Boole cebri}
    \begin{itemize}
      \item eşdeğerli formülleri denklemlerde birbirlerinin yerine koyma
    \end{itemize}
  \end{enumerate}
\end{frame}

\begin{frame}
  \frametitle{Doğruluk Tablosu Örneği}

  \begin{ornek}[$p \rightarrow q$]
    \begin{center}
      \begin{tabular}{|c|c||c|c|c|c|}\hline
        $p$ & $q$ & $p \rightarrow q$ & $\neg q \rightarrow \neg p$ &
                    $q \rightarrow p$ & $\neg p \rightarrow \neg q$\\\hline\hline
        $D$ & $D$ & $D$ & $D$ & $D$ & $D$\\\hline
        $D$ & $Y$ & $Y$ & $Y$ & $D$ & $D$\\\hline
        $Y$ & $D$ & $D$ & $D$ & $Y$ & $Y$\\\hline
        $Y$ & $Y$ & $D$ & $D$ & $D$ & $D$\\\hline
      \end{tabular}
    \end{center}

    \pause
    \begin{itemize}
      \item \emph{kontrapozitif}: $\neg q \rightarrow \neg p$

      \pause
      \item \emph{konvers}: $q \rightarrow p$

      \pause
      \item \emph{invers}: $\neg p \rightarrow \neg q$
    \end{itemize}
  \end{ornek}
\end{frame}

\subsection{Mantık Yasaları}

\begin{frame}
  \frametitle{Mantıksal Eşdeğerlilik}

  \begin{tanim}
    $P$ ve $Q$ formüllerinin doğruluk tabloları aynıysa\\
      $P$ ve $Q$ \alert{mantıksal eşdeğerli}: $P \Leftrightarrow Q$
  \end{tanim}

  \begin{itemize}
    \item $P \leftrightarrow Q$ totoloji
  \end{itemize}
\end{frame}

\begin{frame}
  \frametitle{Mantıksal Eşdeğerlilik Örneği}

  \begin{ornek}
    \begin{itemize}
      \item $\neg p \Leftrightarrow p \rightarrow Y$
    \end{itemize}

    \begin{table}
      \caption{$\neg p \leftrightarrow p \rightarrow Y$}
      \begin{tabular}{|c|c|c||c|}\hline
        $p$ & $\neg p$ & $p \rightarrow Y$
            & $\neg p \leftrightarrow p \rightarrow Y$\\\hline\hline
        $D$ & $Y$ & $Y$ & $D$\\\hline
        $Y$ & $D$ & $D$ & $D$\\\hline
      \end{tabular}
    \end{table}
  \end{ornek}
\end{frame}

\begin{frame}
  \frametitle{Mantıksal Eşdeğerlilik Örneği}

  \begin{ornek}
    \begin{itemize}
      \item $p \rightarrow q \Leftrightarrow \neg p \vee q$
    \end{itemize}

    \begin{table}
      \caption{$(p \rightarrow q) \leftrightarrow (\neg p \vee q)$}
      \begin{tabular}{|c|c|c|c|c||c|}\hline
        $p$ & $q$ & $p \rightarrow q$ & $\neg p$ & $\neg p \vee q$
            & $(p \rightarrow q) \leftrightarrow (\neg p \vee q)$\\\hline\hline
        $D$ & $D$ & $D$ & $Y$ & $D$ & $D$\\\hline
        $D$ & $Y$ & $Y$ & $Y$ & $Y$ & $D$\\\hline
        $Y$ & $D$ & $D$ & $D$ & $D$ & $D$\\\hline
        $Y$ & $Y$ & $D$ & $D$ & $D$ & $D$\\\hline
      \end{tabular}
    \end{table}
  \end{ornek}
\end{frame}

\begin{frame}
  \frametitle{Eşdeğerlilikler}

  \begin{tabular}{ll}
  \alert{çifte değilleme (DN)} &\\
    $\neg (\neg p) \Leftrightarrow p$ &\\\\
  \pause
  \alert{değişme (Co)} &\\
    $p \wedge q \Leftrightarrow q \wedge p$ &
    $p \vee q \Leftrightarrow q \vee p$\\\\
  \pause
  \alert{birleşme (As)} &\\
    $(p \wedge q) \wedge r \Leftrightarrow p \wedge (q \wedge r)$ &
    $(p \vee q) \vee r \Leftrightarrow p \vee (q \vee r)$\\\\
  \pause
  \alert{sabit kuvvetlilik (Ip)} &\\
    $p \wedge p \Leftrightarrow p$ &
    $p \vee p \Leftrightarrow p$\\\\
  \pause
  \alert{terslik (In)} &\\
    $p \wedge \neg p \Leftrightarrow Y$ &
    $p \vee \neg p \Leftrightarrow D$
  \end{tabular}
\end{frame}

\begin{frame}
  \frametitle{Eşdeğerlilikler}

  \begin{tabular}{ll}
  \alert{etkisizlik (Id)} &\\
    $p \wedge D \Leftrightarrow p$ &
    $p \vee Y \Leftrightarrow p$\\\\
  \pause
  \alert{baskınlık (Do)} &\\
    $p \wedge Y \Leftrightarrow Y$ &
    $p \vee D \Leftrightarrow D$\\\\
  \pause
  \alert{dağılma (Di)} &\\
    $p \wedge (q \vee r) \Leftrightarrow (p \wedge q) \vee (p \wedge r)$ &
    $p \vee (q \wedge r) \Leftrightarrow (p \vee q) \wedge (p \vee r)$\\\\
  \pause
  \alert{yutma (Ab)} &\\
    $p \wedge (p \vee q) \Leftrightarrow p$ &
    $p \vee (p \wedge q) \Leftrightarrow p$\\\\
  \pause
  \alert{De Morgan (DM)} &\\
    $\neg (p \wedge q) \Leftrightarrow \neg p \vee \neg q$ &
    $\neg (p \vee q) \Leftrightarrow \neg p \wedge \neg q$
  \end{tabular}
\end{frame}

\begin{frame}
  \frametitle{Dualite}

  \begin{tanim}
    \alert{dual}:\\
      $\wedge$ ve $\vee$ dışında bir bağlaç içermeyen bir $s$
      önermesinin duali $s^d$ önermesi, $\wedge$ yerine $\vee$, $\vee$
      yerine $\wedge$, $D$ yerine $Y$, $Y$ yerine $D$ konarak elde edilir.
  \end{tanim}

  \pause
  \begin{ornek}[dual önerme]
    \begin{eqnarray*}
      s:   & (p \wedge \neg q) \vee (r \wedge D)\\
      s^d: & (p \vee \neg q) \wedge (r \vee Y)
    \end{eqnarray*}
  \end{ornek}
\end{frame}

\begin{frame}
  \frametitle{Dualite İlkesi}

  \begin{block}{dualite ilkesi}
      $s$ ve $t$, $\wedge$ ve $\vee$ dışında bir bağlaç içermeyen önermeler
      olsun.\\
      $s \Leftrightarrow t$ ise $s^d \Leftrightarrow t^d$.
  \end{block}
\end{frame}

\begin{frame}
  \frametitle{Eşdeğerlilik Hesabı Örneği}

  \begin{ornek}
    \begin{eqnarray*}
                      & p \rightarrow q           &\\
      \pause
      \Leftrightarrow & \neg p \vee q             &\\
      \pause
      \Leftrightarrow & q \vee \neg p             & Co\\
      \pause
      \Leftrightarrow & \neg \neg q \vee \neg p   & DN\\
      \pause
      \Leftrightarrow & \neg q \rightarrow \neg p &
    \end{eqnarray*}
  \end{ornek}
\end{frame}

\begin{frame}
  \frametitle{Eşdeğerlilik Hesabı Örneği}

  \begin{ornek}
    \begin{eqnarray*}
                      & \neg (\neg ((p \vee q) \wedge r) \vee \neg q)      &\\
      \pause
      \Leftrightarrow & \neg \neg ((p \vee q) \wedge r) \wedge \neg \neg q & DM\\
      \pause
      \Leftrightarrow & ((p \vee q) \wedge r) \wedge q                     & DN\\
      \pause
      \Leftrightarrow & (p \vee q) \wedge (r \wedge q)                     & As\\
      \pause
      \Leftrightarrow & (p \vee q) \wedge (q \wedge r)                     & Co\\
      \pause
      \Leftrightarrow & ((p \vee q) \wedge q) \wedge r                     & As\\
      \pause
      \Leftrightarrow & q \wedge r                                         & Ab
    \end{eqnarray*}
  \end{ornek}
\end{frame}

\subsection{Akıl Yürütme}

\begin{frame}
  \frametitle{Mantıksal Gerektirme}

  \begin{tanim}
    $P$ doğru olduğunda $Q$ her zaman doğruysa\\
    $P$ formülü $Q$ formülünü \alert{mantıksal gerektirir}: $P \Rightarrow Q$
  \end{tanim}

  \begin{itemize}
    \item $P \rightarrow Q$ totoloji
  \end{itemize}
\end{frame}

\begin{frame}
  \frametitle{Mantıksal Gerektirme Örneği}

  \begin{ornek}
    \begin{itemize}
      \item $p \wedge (p \rightarrow q) \Rightarrow q$
    \end{itemize}

    \begin{table}
      \caption{$p \wedge (p \rightarrow q) \rightarrow q$}
      \begin{tabular}{|c|c|c|c||c|}\hline
        $p$ & $q$ & $p \rightarrow q$ & $p \wedge (p \rightarrow q)$
            & $p \wedge (p \rightarrow q) \rightarrow q$\\\hline\hline
        $D$ & $D$ & $D$ & $D$ & $D$\\\hline
        $D$ & $Y$ & $Y$ & $Y$ & $D$\\\hline
        $Y$ & $D$ & $D$ & $Y$ & $D$\\\hline
        $Y$ & $Y$ & $D$ & $Y$ & $D$\\\hline
      \end{tabular}
    \end{table}
  \end{ornek}
\end{frame}

\begin{frame}
  \frametitle{Akıl Yürütme}

  \begin{itemize}
    \item doğruluğu varsayılan ya da tanıtlanmış önermeler içeren bir kümeden
      yola çıkarak bu küme dışındaki bir önermenin doğruluğuna varma
  \end{itemize}

  \pause
  \begin{block}{çıkarsama kuralları}
    \begin{columns}
      \column{.5\textwidth}
      \[
      \frac
        {
          \begin{array}{c}
            p_1\\
            p_2\\
            \dots\\
            p_n
          \end{array}
        }
        {
          \therefore q
        }
      \]

      \column{.5\textwidth}
      $p_1 \wedge p_2 \wedge \cdots \wedge p_n \Rightarrow q$
    \end{columns}
  \end{block}
\end{frame}

\begin{frame}
  \frametitle{Temel Kurallar}

  \begin{columns}
    \column{.5\textwidth}
    \begin{block}{Özdeşlik (ID)}
      \[
      \frac
        {
          \begin{array}{c}
            p
          \end{array}
        }
        {
          \therefore p
        }
      \]
    \end{block}

    \pause
    \column{.5\textwidth}
    \begin{block}{Çelişki (CTR)}
    \[
    \frac
      {
        \begin{array}{c}
          Y
        \end{array}
      }
      {
        \therefore p
      }
    \]
    \end{block}
  \end{columns}
\end{frame}

\begin{frame}
  \frametitle{Temel Kurallar}

  \begin{columns}
    \column{.4\textwidth}
    \begin{block}{Koşul Ekleme (ImpI)}
      \[
      \frac
        {
          \begin{array}{c}
            p \vdash q
          \end{array}
        }
        {
          \therefore p \rightarrow q
        }
      \]
    \end{block}

    \pause
    \column{.6\textwidth}
    \begin{itemize}
      \item $p$ doğru varsayıldığında\\
        $q$ doğru olduğu gösterilebiliyorsa,\\
        \alert{$p$ doğru varsayılmadan}\\
        $p \rightarrow q$ doğrudur
    \end{itemize}
  \end{columns}
\end{frame}

\begin{frame}
  \frametitle{Temel Kurallar}

  \begin{columns}[t]
    \column{.5\textwidth}
    \begin{block}{VE Ekleme (AndI)}
      \[
      \frac
        {
          \begin{array}{c}
            p\\
            q
          \end{array}
        }
        {
          \therefore p \wedge q
        }
      \]
    \end{block}

    \pause
    \column{.5\textwidth}
    \begin{block}{VE Eleme (AndE)}
    \[
    \frac
      {
        \begin{array}{c}
          p \wedge q
        \end{array}
      }
      {
        \therefore p
      }
    \]
    \end{block}
  \end{columns}
\end{frame}

\begin{frame}
  \frametitle{Temel Kurallar}

  \begin{columns}[t]
    \column{.5\textwidth}
    \begin{block}{VEYA Ekleme (OrI)}
      \[
      \frac
        {
          \begin{array}{c}
            p
          \end{array}
        }
        {
          \therefore p \vee q
        }
      \]
    \end{block}

    \pause
    \column{.5\textwidth}
    \begin{block}{VEYA Eleme (OrE)}
    \[
    \frac
      {
        \begin{array}{c}
          p \vee q\\
          p \vdash r\\
          q \vdash r
        \end{array}
      }
      {
        \therefore r
      }
    \]
    \end{block}
  \end{columns}
\end{frame}

\begin{frame}
  \frametitle{Temel Kurallar}

  \begin{columns}[t]
    \column{.5\textwidth}
    \begin{block}{Modus Ponens (ImpE)}
      \[
      \frac
        {
          \begin{array}{c}
            p \rightarrow q\\
            p
          \end{array}
        }
        {
          \therefore q
        }
      \]
    \end{block}

    \pause
    \column{.5\textwidth}
    \begin{block}{Modus Tollens (MT)}
      \[
      \frac
        {
          \begin{array}{c}
            p \rightarrow q\\
            \neg q
          \end{array}
        }
        {
          \therefore \neg p
        }
      \]
    \end{block}
  \end{columns}
\end{frame}

\begin{frame}
  \frametitle{Modus Tollens}

  \begin{ornek}
    \begin{columns}
      \column{.3\textwidth}
      \[
      \frac
        {
          \begin{array}{c}
            p \rightarrow q\\
            \neg q
          \end{array}
        }
        {
          \therefore \neg p
        }
      \]

      \pause
      \column{.65\textwidth}
      \begin{eqnarray*}
        1. & p \rightarrow q           & A\\
        \pause
        2. & \neg q \rightarrow \neg p & 1\\
        \pause
        3. & \neg q                    & A\\
        \pause
        4. & \neg p                    & ImpE:2,3\\
      \end{eqnarray*}
    \end{columns}
  \end{ornek}
\end{frame}

\begin{frame}
  \frametitle{Modus Ponens Örneği}

  \begin{ornek}
    \begin{itemize}
      \item Ali piyangoyu kazanırsa araba alacak.
      \item Ali piyangoyu kazandı.

      \pause
      \medskip
      \item o halde, Ali araba alacak.
    \end{itemize}
  \end{ornek}
\end{frame}

\begin{frame}
  \frametitle{Modus Tollens Örneği}

  \begin{ornek}
    \begin{itemize}
      \item Ali piyangoyu kazanırsa araba alacak.
      \item Ali araba almadı.

      \pause
      \medskip
      \item o halde, Ali piyangoyu kazanmadı.
    \end{itemize}
  \end{ornek}
\end{frame}

\begin{frame}
  \frametitle{Yanılgılar}

  \begin{block}{sonucu onaylama yanılgısı}
    \[
    \frac
      {
      \begin{array}{c}
        p \rightarrow q\\
        q
        \end{array}
      }
      {
        \therefore p
      }
    \]
  \end{block}

  \pause
  \begin{itemize}
    \item $(p \rightarrow q) \wedge q \rightarrow p$ bir totoloji değil:\\
      $p=Y,q=D$ ise: $(Y \rightarrow D) \wedge D \rightarrow Y$
  \end{itemize}
\end{frame}

\begin{frame}
  \frametitle{Sonucu Onaylama Yanılgısı Örneği}

  \begin{ornek}
    \begin{itemize}
      \item Madonna A.B.D. başkanıysa 35 yaşının üstündedir.
      \item Madonna 35 yaşının üstündedir.

      \pause
      \medskip
      \item o halde, Madonna A.B.D. başkanıdır.
    \end{itemize}
  \end{ornek}
\end{frame}

\begin{frame}
  \frametitle{Yanılgılar}

  \begin{block}{öncülü yadsıma yanılgısı}
    \[
    \frac
      {
        \begin{array}{c}
          p \rightarrow q\\
          \neg p
        \end{array}
      }
      {
        \therefore \neg q
      }
    \]
  \end{block}

  \pause
  \begin{itemize}
    \item $(p \rightarrow q) \wedge \neg p \rightarrow \neg q$ bir totoloji
      değil:\\
      $p=Y,q=D$ ise: $(Y \rightarrow D) \wedge D \rightarrow Y$
  \end{itemize}
\end{frame}

\begin{frame}
  \frametitle{Öncülü Yadsıma Yanılgısı Örneği}

  \begin{ornek}
    \begin{itemize}
      \item $2+3=8$ ise $2+4=6$
      \item $2+3 \neq 8$

      \pause
      \medskip
      \item o halde, $2+4 \neq 6$
    \end{itemize}
  \end{ornek}
\end{frame}

\begin{frame}
  \frametitle{Ayırıcı Kıyas}

  \begin{columns}
    \column{.5\textwidth}
    \begin{block}{Ayırıcı Kıyas (DS)}
      \[
      \frac
        {
          \begin{array}{c}
            p \vee q\\
            \neg p
          \end{array}
        }
        {
          \therefore q
        }
      \]
    \end{block}

    \pause
    \column{.5\textwidth}
    \begin{eqnarray*}
      1.   & p \vee q        & A\\
      \pause
      2.   & \neg p          & A\\
      \pause
      3.   & p \rightarrow Y & 2\\
      \pause
      4a1. & p               & A!\\
      \pause
      4a2. & Y               & ImpE:3,4a1\\
      \pause
      4a.  & q               & CTR:4a2\\
      \pause
      4b1. & q               & A!\\
      \pause
      4b.  & q               & ID:4b1\\
      \pause
      5.   & q               & OrE:1,4a,4b
    \end{eqnarray*}
  \end{columns}
\end{frame}

\begin{frame}
  \frametitle{Ayırıcı Kıyas Örneği}

  \begin{ornek}
    \begin{itemize}
      \item Ali'nin cüzdanı cebinde veya masasında.
      \item Ali'nin cüzdanı cebinde değil.

      \pause
      \medskip
      \item o halde, Ali'nin cüzdanı masasında.
    \end{itemize}
  \end{ornek}
\end{frame}

\begin{frame}
  \frametitle{Varsayımlı Kıyas}

  \begin{columns}
    \column{.5\textwidth}
    \begin{block}{Varsayımlı Kıyas (HS)}
      \[
      \frac
        {
          \begin{array}{c}
            p \rightarrow q\\
            q \rightarrow r
          \end{array}}
        {
          \therefore p \rightarrow r
        }
      \]
    \end{block}

    \pause
    \column{.5\textwidth}
    \begin{eqnarray*}
      1. & p               & A!\\
      \pause
      2. & p \rightarrow q & A\\
      \pause
      3. & q               & ImpE:2,1\\
      \pause
      4. & q \rightarrow r & A\\
      \pause
      5. & r               & ImpE:4,3\\
      \pause
      6. & p \rightarrow r & ImpI:1,5\\
    \end{eqnarray*}
  \end{columns}
\end{frame}

\begin{frame}
  \frametitle{Varsayımlı Kıyas Örneği}

  \begin{ornek}[Uzay Yolu]
    Spock - Yarbay Decker:
    \begin{quote}
      Şu anda düşman gemisine saldırmak intihar olur. İntihara teşebbüs eden
      biri Atılgan'ın komutanlığını yapmaya psikolojik olarak yetkin degildir.
      Bu yüzden, sizi görevden almak zorundayım.
    \end{quote}
  \end{ornek}
\end{frame}

\begin{frame}
  \frametitle{Varsayımlı Kıyas Örneği}

  \begin{ornek}[Uzay Yolu]
    \begin{itemize}
      \item $p$: Decker düşman gemisine saldırır.
      \item $q$: Decker intihara teşebbüs eder.
      \item $r$: Decker Atılgan'ın komutanlığını yapmaya psikolojik olarak
        yetkin değildir.
      \item $s$: Spock Decker'ı görevden alır.
    \end{itemize}
  \end{ornek}
\end{frame}

\begin{frame}
  \frametitle{Varsayımlı Kıyas Örneği}

  \begin{ornek}
    \begin{columns}
      \column{.3\textwidth}
      \[
      \frac
        {
          \begin{array}{c}
            p\\
            p \rightarrow q\\
            q \rightarrow r\\
            r \rightarrow s
          \end{array}
        }
        {
          \therefore s
        }
      \]

      \pause
      \column{.65\textwidth}
      \begin{eqnarray*}
        1. & p \rightarrow q & A\\
        \pause
        2. & q \rightarrow r & A\\
        \pause
        3. & p \rightarrow r & HS:1,2\\
        \pause
        4. & r \rightarrow s & A\\
        \pause
        5. & p \rightarrow s & HS:3,4\\
        \pause
        6. & p               & A\\
        \pause
        7. & s               & ImpE:5,6
      \end{eqnarray*}
    \end{columns}
  \end{ornek}
\end{frame}

\begin{frame}
  \frametitle{İkilemler}

  \begin{columns}[t]
    \column{.5\textwidth}
    \begin{block}{Yapıcı İkilem}
      \[
      \frac
        {
          \begin{array}{c}
            p \rightarrow q\\
            r \rightarrow s\\
            p \vee r
          \end{array}}
        {
          \therefore q \vee s
        }
      \]
    \end{block}

    \pause
    \column{.5\textwidth}
    \begin{block}{Yıkıcı İkilem}
      \[
      \frac
        {
          \begin{array}{c}
            p \rightarrow q\\
            r \rightarrow s\\
            \neg q \vee \neg s
          \end{array}
          }
          {
            \therefore \neg p \vee \neg r
          }
      \]
    \end{block}
  \end{columns}
\end{frame}

\begin{frame}
  \frametitle{Akıl Yürütme Örnekleri}

  \begin{ornek}
    \begin{columns}[t]
      \column{.25\textwidth}
      \[
      \frac
        {
          \begin{array}{c}
            p \rightarrow r\\
            r \rightarrow s\\
            x \vee \neg s\\
            u \vee \neg x\\
            \neg u
          \end{array}
        }
        {
          \therefore \neg p
        }
      \]

      \pause
      \column{.3\textwidth}
      \begin{eqnarray*}
        1. & u \vee \neg x   & A\\
        \pause
        2. & \neg u          & A\\
        \pause
        3. & \neg x          & DS:1,2\\
        \pause
        4. & x \vee \neg s   & A\\
        \pause
        5. & \neg s          & DS:4,3\\
      \end{eqnarray*}

      \pause
      \column{.45\textwidth}
      \begin{eqnarray*}
        6. & r \rightarrow s & A\\
        \pause
        7. & \neg r          & MT:6,5\\
        \pause
        8. & p \rightarrow r & A\\
        \pause
        9. & \neg p          & MT:8,7\\
      \end{eqnarray*}
    \end{columns}
  \end{ornek}
\end{frame}

\begin{frame}
  \frametitle{Akıl Yürütme Örnekleri}

  \begin{ornek}
    \[
    \frac
      {
        \begin{array}{c}
          (\neg p \vee \neg q) \rightarrow (r \wedge s)\\
          r \rightarrow x\\
          \neg x
        \end{array}
      }
      {
        \therefore p
      }
    \]

    \pause
    \begin{columns}[t]
      \column{.3\textwidth}
      \begin{eqnarray*}
        1. & r \rightarrow x                               & A\\
        \pause
        2. & \neg x                                        & A\\
        \pause
        3. & \neg r                                        & MT:1,2\\
        \pause
        4. & \neg r \vee \neg s                            & OrI:3\\
        \pause
        5. & \neg (r \wedge s)                             & DM:4
      \end{eqnarray*}

      \pause
      \column{.6\textwidth}
      \begin{eqnarray*}
        6. & (\neg p \vee \neg q) \rightarrow (r \wedge s) & A\\
        \pause
        7. & \neg (\neg p \vee \neg q)                     & MT:6,5\\
        \pause
        8. & p \wedge q                                    & DM:7\\
        \pause
        9. & p                                             & AndE:8
      \end{eqnarray*}
    \end{columns}
  \end{ornek}
\end{frame}

\begin{frame}
  \frametitle{Akıl Yürütme Örnekleri}

  \begin{ornek}
    \begin{columns}
      \column{.3\textwidth}
      \[
      \frac
        {
          \begin{array}{c}
            p \rightarrow (q \vee r)\\
            s \rightarrow \neg r\\
            q \rightarrow \neg p\\
            p\\
            s
          \end{array}
        }
        {
          \therefore q \wedge \neg q
        }
      \]

      \pause
      \column{.6\textwidth}
      \begin{eqnarray*}
        1. & q \rightarrow \neg p     & A\\
       \pause
        2. & p                        & A\\
       \pause
        3. & \neg q                   & MT:1,2\\
       \pause
        4. & s                        & A\\
       \pause
        5. & s \rightarrow \neg r     & A\\
       \pause
        6. & \neg r                   & ImpE:5,4\\
       \pause
        7. & p \rightarrow (q \vee r) & A\\
       \pause
        8. & q \vee r                 & ImpE:7,2\\
       \pause
        9. & q                        & DS:8,6\\
       \pause
       10. & q \wedge \neg q          & AndI:9,3
      \end{eqnarray*}
    \end{columns}
  \end{ornek}
\end{frame}

\begin{frame}
  \frametitle{Akıl Yürütme Örnekleri}

  \begin{ornek}
    \begin{quote}
      Eğer yağmur yağma olasılığı varsa veya saç bandını bulamazsa, Filiz
      çimleri biçmez. Hava sıcaklığı 20 derecenin üzerindeyse yağmur yağma
      olasılığı yoktur. Bugün hava sıcaklığı 22 derece ve Filiz saç bandını
      takmış. Demek ki Filiz çimleri biçecek.
    \end{quote}
  \end{ornek}
\end{frame}

\begin{frame}
  \frametitle{Akıl Yürütme Örnekleri}

  \begin{ornek}
    \begin{itemize}
      \item $p$: Yağmur yağabilir.
      \item $q$: Filiz'in saç bandı kayıp.
      \item $r$: Filiz çimleri biçer.
      \item $s$: Hava sıcaklığı 20 derecenin üzerinde.
    \end{itemize}
  \end{ornek}
\end{frame}

\begin{frame}
  \frametitle{Akıl Yürütme Örnekleri}

  \begin{ornek}
    \begin{columns}
      \column{.3\textwidth}
      \[
      \frac
        {
          \begin{array}{c}
            (p \vee q) \rightarrow \neg r\\
            s \rightarrow \neg p\\
            s \wedge \neg q
          \end{array}
        }
        {
          \therefore r
        }
      \]

      \pause
      \column{.65\textwidth}
      \begin{eqnarray*}
        1. & s \wedge \neg q                & A\\
        \pause
        2. & s                              & AndE:1\\
        \pause
        3. & s \rightarrow \neg p           & A\\
        \pause
        4. & \neg p                         & ImpE:3,2\\
        \pause
        5. & \neg q                         & AndE:1\\
        \pause
        6. & \neg p \wedge \neg q           & AndI:4,5\\
        \pause
        7. & \neg (p \vee q)                & DM:6\\
        \pause
        8. & (p \vee q) \rightarrow \neg r  & A\\
        \pause
        9. & ?                              & 7,8
      \end{eqnarray*}
    \end{columns}
  \end{ornek}
\end{frame}

\section*{Kaynaklar}

\begin{frame}
  \frametitle{Kaynaklar}

  \begin{block}{Okunacak: Grimaldi}
    \begin{itemize}
      \item Chapter 2: Fundamentals of Logic
      \begin{itemize}
        \item 2.1. \alert{Basic Connectives and Truth Tables}
        \item 2.2. \alert{Logical Equivalence: The Laws of Logic}\\
        \item 2.3. \alert{Logical Implication: Rules of Inference}
      \end{itemize}
    \end{itemize}
  \end{block}

  \begin{block}{Yardımcı Kitap: O'Donnell, Hall, Page}
    \begin{itemize}
      \item Chapter 6: Propositional Logic
    \end{itemize}
  \end{block}
\end{frame}

\end{document}
