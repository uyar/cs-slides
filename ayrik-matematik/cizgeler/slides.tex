% Copyright (c) 2001-2010
%       H. Turgut Uyar <uyar@itu.edu.tr>
%       Ayşegül Gençata Yayımlı <gencata@itu.edu.tr>
%       Emre Harmancı <harmanci@itu.edu.tr>
%
% Bu notlar "Creative Commons Attribution-NonCommercial-ShareAlike License" ile
% lisanslanmıştır. Yazarının açıkça belirtilmesi koşuluyla ve ticari olmayan
% amaçlarla kullanılabilir ve dağıtılabilir. Bu notlardan yola çıkılarak
% oluşturulacak çalışmaların da aynı lisansa bağlı olmaları gerekir.
%
% Lisans ile ilgili ayrıntılı bilgi almak için şu sayfaya başvurabilirsiniz:
% http://creativecommons.org/licenses/by-nc-sa/3.0/

\documentclass[dvipsnames]{beamer}

\usepackage{ae}
\usepackage[T1]{fontenc}
\usepackage[utf8]{inputenc}
\usepackage[turkish]{babel}
\setbeamertemplate{navigation symbols}{}

\mode<presentation>
{
  \usetheme{Rochester}
  \setbeamercovered{transparent}
}

\title{Ayrık Matematik}
\subtitle{Çizgeler}

\author{H. Turgut Uyar \and Ayşegül Gençata Yayımlı \and Emre Harmancı}
\date{2001-2010}

\AtBeginSubsection[]
{
  \begin{frame}<beamer>
    \frametitle{Konular}
    \tableofcontents[currentsection,currentsubsection]
  \end{frame}
}

%\beamerdefaultoverlayspecification{<+->}

\theoremstyle{definition}
\newtheorem{tanim}[theorem]{Tanım}

\theoremstyle{example}
\newtheorem{ornek}[theorem]{Örnek}

\theoremstyle{plain}
\newtheorem{teorem}[theorem]{Teorem}

\pgfdeclareimage[width=2cm]{license}{../../license}

\pgfdeclareimage[width=6cm]{yalin}{yalin}
\pgfdeclareimage[width=6cm]{coklu}{coklu}
\pgfdeclareimage[width=4cm]{yonlu}{yonlu}
\pgfdeclareimage[width=4cm]{matris}{matris}
\pgfdeclareimage[width=6cm]{bagsiz}{bagsiz}
\pgfdeclareimage[width=5cm]{uzaklik}{uzaklik}
\pgfdeclareimage[width=5cm]{kesitleme}{kesitleme}
\pgfdeclareimage[width=4cm]{zayif}{zayif}
\pgfdeclareimage[width=4cm]{tekyonlu}{tekyonlu}
\pgfdeclareimage[width=4cm]{guclu}{guclu}
\pgfdeclareimage[width=5cm]{warshall}{warshall}
\pgfdeclareimage[width=5cm]{warshall1}{warshall1}
\pgfdeclareimage[width=5cm]{warshall2}{warshall2}
\pgfdeclareimage[width=5cm]{warshall3}{warshall3}
\pgfdeclareimage{konigsberg}{konigsberg}
\pgfdeclareimage[width=3cm]{zarf}{zarf}
\pgfdeclareimage{konigcizge}{konigcizge}
\pgfdeclareimage[width=3.5cm]{euler}{euler}
\pgfdeclareimage[width=3.5cm]{hamilton}{hamilton}
\pgfdeclareimage[height=4cm]{izomorff}{izomorff}
\pgfdeclareimage[height=4cm]{izomorft}{izomorft}
\pgfdeclareimage[height=3.5cm]{petersen1}{petersen1}
\pgfdeclareimage[height=3.5cm]{petersen2}{petersen2}
\pgfdeclareimage[height=4cm]{homeomorft}{homeomorft}
\pgfdeclareimage[height=4.5cm]{duzenli}{duzenli}
\pgfdeclareimage[width=4cm]{k4}{k4}
\pgfdeclareimage[width=4cm]{k5}{k5}
\pgfdeclareimage[width=4cm]{k23}{k23}
\pgfdeclareimage[width=4cm]{k33}{k33}
\pgfdeclareimage[width=4.5cm]{k4duzlemsel}{k4duzlemsel}
\pgfdeclareimage[width=6cm]{bolge}{bolge}
\pgfdeclareimage[width=3cm]{eulertanit1}{eulertanit1}
\pgfdeclareimage[width=3cm]{eulertanit2}{eulertanit2}
\pgfdeclareimage[width=5cm]{kup}{hexahedron}
\pgfdeclareimage{tetrahedron}{tetrahedron}
\pgfdeclareimage{octahedron}{octahedron}
\pgfdeclareimage{hexahedron}{hexahedron}
\pgfdeclareimage{dodecahedron}{dodecahedron}
\pgfdeclareimage{icosahedron}{icosahedron}
\pgfdeclareimage{planartetra}{planartetra}
\pgfdeclareimage{planarhexa}{planarhexa}
\pgfdeclareimage{planarocta}{planarocta}
\pgfdeclareimage{planardodeca}{planardodeca}
\pgfdeclareimage[width=6cm]{dijkstra}{dijkstra}
\pgfdeclareimage[width=4cm]{besmadde}{besmadde}
\pgfdeclareimage[height=4cm]{boyama1}{boyama1}
\pgfdeclareimage[height=4cm]{boyama2}{boyama2}
\pgfdeclareimage[height=4cm]{boyama3}{boyama3}
\pgfdeclareimage[height=4cm]{boyama4}{boyama4}
\pgfdeclareimage[height=4cm]{boyama5}{boyama5}
\pgfdeclareimage[height=4cm]{boyama6}{boyama6}
\pgfdeclareimage[height=4cm]{boyama7}{boyama7}
\pgfdeclareimage[height=4cm]{herschel}{herschel}
\pgfdeclareimage[width=4cm]{sudoku}{sudoku}

\begin{document}

\begin{frame}
  \titlepage
\end{frame}

\begin{frame}
  \frametitle{Lisans}

  \pgfuseimage{license}\hfill
  \copyright 2001-2010 T. Uyar, A. Yayımlı, E. Harmancı

  \vfill
  \begin{tiny}
    You are free:
    \begin{itemize}
      \item to Share — to copy, distribute and transmit the work
      \item to Remix — to adapt the work
    \end{itemize}

    Under the following conditions:
    \begin{itemize}
      \item Attribution — You must attribute the work in the manner specified by
        the author or licensor (but not in any way that suggests that they
        endorse you or your use of the work).

      \item Noncommercial — You may not use this work for commercial purposes.

      \item Share Alike — If you alter, transform, or build upon this work, you
        may distribute the resulting work only under the same or similar license
        to this one.
    \end{itemize}
  \end{tiny}

  \vfill
  Legal code (the full license):\\
  \url{http://creativecommons.org/licenses/by-nc-sa/3.0/}
\end{frame}

\begin{frame}
  \frametitle{Konular}
  \tableofcontents
\end{frame}

\section{Giriş}

\subsection{Yönsüz Çizgeler}

\begin{frame}
  \frametitle{Çizge}

  \begin{tanim}
    \alert{çizge}: $G=(V,E)$

    \begin{itemize}
      \item $V$: düğümler kümesi
      \item $E \subseteq V \times V$: ayrıtlar kümesi
    \end{itemize}
  \end{tanim}

  \pause
  \begin{itemize}
    \item $e=(v_1,v_2) \in E$ ise:
    \begin{itemize}
      \item $v_1$ ve $v_2$ düğümleri $e$ ayrıtının \emph{uçdüğümleri}
      \item $e$ ayrıtı $v_1$ ve $v_2$ düğümlerine  \emph{çakışık}
      \item $v_1$ ve $v_2$ düğümleri  \emph{bitişik}
    \end{itemize}

    \pause
    \item hiçbir ayrıtın çakışmadığı düğüm: \emph{yalıtılmış düğüm}
  \end{itemize}
\end{frame}

\begin{frame}
  \frametitle{Çizge Örneği}

  \begin{ornek}
    \begin{columns}
      \column{.58\textwidth}
      \begin{center}
        \pgfuseimage{yalin}
      \end{center}

      \pause
      \column{.4\textwidth}
      $\begin{array}{lcl}
        V & = & \{a,b,c,d,e,f\}\\
        E & = & \{(a,b),(a,c),\\
          &   & ~(a,d),(a,e),\\
          &   & ~(a,f),(b,c),\\
          &   & ~(d,e),(e,f)\}
      \end{array}$
    \end{columns}
  \end{ornek}
\end{frame}

\begin{frame}
  \frametitle{Yalın Çizge}

  \begin{tanim}
    \alert{koşut bağlı ayrıtlar}:\\
    aynı iki düğüm arasındaki ayrıtlar

    \pause
    \bigskip
    \alert{tek-çevre}:\\
    iki ucu aynı düğüm olan ayrıt

    \pause
    \bigskip
    \alert{yalın çizge}:\\
    koşut bağlı ayrıtlar ya da tek-çevre içermeyen çizge

    \pause
    \bigskip
    \alert{çoklu çizge}:\\
    yalın olmayan çizge
  \end{tanim}
\end{frame}

\begin{frame}
  \frametitle{Çoklu Çizge Örneği}

  \begin{ornek}
    \begin{columns}
      \column{.58\textwidth}
      \begin{center}
        \pgfuseimage{coklu}
      \end{center}

      \column{.4\textwidth}
      \begin{itemize}
        \item koşut bağlı ayrıtlar:\\
          $(a,b)$
        \item tek-çevre:\\
          $(e,e)$
      \end{itemize}
    \end{columns}
  \end{ornek}
\end{frame}

\begin{frame}
  \frametitle{Altçizge}

  \begin{tanim}
    \alert{altçizge}:\\
      $G'=(V',E')$ çizgesi $G=(V,E)$ çizgesinin altçizgesi ise

    \begin{itemize}
      \item $V' \subseteq V$
      \item $E' \subseteq E$
      \item $E'$ kümesindeki ayrıtların uçdüğümleri $V'$ kümesinin elemanları
    \end{itemize}
  \end{tanim}
\end{frame}

\begin{frame}
  \frametitle{Kerte}

  \begin{tanim}
    \alert{kerte}:\\
    düğüme çakışan ayrıtların sayısı
  \end{tanim}

  \pause
  \begin{teorem}
    $v_i$ düğümünün kertesi $d_i$ ise:

    \[ |E| = \frac{\sum_i d_i}{2} \]
  \end{teorem}
\end{frame}

\begin{frame}
  \frametitle{Kerte Örneği}

  \begin{ornek}[yalın]
    \begin{columns}
      \column{.58\textwidth}
      \begin{center}
        \pgfuseimage{yalin}
      \end{center}

      \column{.4\textwidth}
      $\begin{array}{ccc}
      d_a & = & 5\\
      d_b & = & 2\\
      d_c & = & 2\\
      d_d & = & 2\\
      d_e & = & 3\\
      d_f & = & 2\\
      \medskip
      Toplam & = & 16\\
      \medskip
      |E| & = & 8
      \end{array}$
    \end{columns}
  \end{ornek}
\end{frame}

\begin{frame}
  \frametitle{Kerte Örneği}

  \begin{ornek}[çoklu]
    \begin{columns}
      \column{.58\textwidth}
      \begin{center}
        \pgfuseimage{coklu}
      \end{center}

      \column{.4\textwidth}
      $\begin{array}{ccc}
      d_a & = & 6\\
      d_b & = & 3\\
      d_c & = & 2\\
      d_d & = & 2\\
      d_e & = & 5\\
      d_f & = & 2\\
      \medskip
      Toplam & = & 20\\
      \medskip
      |E| & = & 10
      \end{array}$
    \end{columns}
  \end{ornek}
\end{frame}

\begin{frame}
  \frametitle{Kerte}

  \begin{teorem}
    Bir çizgede kertesi tek olan düğümlerin sayısı çifttir.
  \end{teorem}

  \pause
  \begin{proof}[Tanıt]
    \begin{itemize}
      \item $t_i$: kertesi $i$ olan düğümlerin sayısı

      \pause
$2|E| = \sum_i d_i = 1t_1 + 2t_2 + 3t_3 + 4t_4 + 5t_5 + \dots$

\pause
$2|E| - 2t_2 - 4t_4 - \dots = t_1 + t_3 + \dots + 2t_3 + 4t_5 + \dots$

\pause
$2|E| - 2t_2 - 4t_4 - \dots - 2t_3 - 4t_5 - \dots = t_1 + t_3 + t_5 + \dots$

      \pause
      \item sol yan çift olduğuna göre sağ yan da çifttir
    \end{itemize}
  \end{proof}
\end{frame}

\begin{frame}
  \frametitle{Gösterilim}

  \begin{itemize}
    \item \emph{çakışıklık matrisi}:
    \begin{itemize}
      \item satırlara düğümler, sütunlara ayrıtlar
      \item ayrıt düğüme çakışıksa 1, değilse 0
    \end{itemize}

    \pause
    \medskip
    \item \emph{bitişiklik matrisi}:
    \begin{itemize}
      \item satırlara ve sütunlara düğümler
      \item hücrelere düğümler arasındaki ayrıt sayısı
    \end{itemize}
  \end{itemize}
\end{frame}

\begin{frame}
  \frametitle{Çakışıklık Matrisi Örneği}

  \begin{ornek}
    \begin{columns}
      \column{.38\textwidth}
      \begin{center}
        \pgfuseimage{matris}
      \end{center}

      \column{.58\textwidth}
      \[
        \begin{array}{c|cccccccc}
              & e_1 & e_2 & e_3 & e_4 & e_5 & e_6 & e_7 & e_8\\\hline
          v_1 & 1 & 1 & 1 & 0 & 1 & 0 & 0 & 0\\
          v_2 & 1 & 0 & 0 & 1 & 0 & 0 & 0 & 0\\
          v_3 & 0 & 0 & 1 & 1 & 0 & 0 & 1 & 1\\
          v_4 & 0 & 0 & 0 & 0 & 1 & 1 & 0 & 1\\
          v_5 & 0 & 1 & 0 & 0 & 0 & 1 & 1 & 0
        \end{array}
      \]
    \end{columns}
  \end{ornek}
\end{frame}

\begin{frame}
  \frametitle{Bitişiklik Matrisi Örneği}

  \begin{ornek}
    \begin{columns}
      \column{.38\textwidth}
      \begin{center}
        \pgfuseimage{matris}
      \end{center}

      \column{.58\textwidth}
      \[
        \begin{array}{c|ccccc}
                & v_1 & v_2 & v_3 & v_4 & v_5\\\hline
            v_1 & 0 & 1 & 1 & 1 & 1\\
            v_2 & 1 & 0 & 1 & 0 & 0\\
            v_3 & 1 & 1 & 0 & 1 & 1\\
            v_4 & 1 & 0 & 1 & 0 & 1\\
            v_5 & 1 & 0 & 1 & 1 & 0
        \end{array}
      \]
    \end{columns}
  \end{ornek}
\end{frame}

\subsection{Yönlü Çizgeler}

\begin{frame}
  \frametitle{Yönlü Çizgeler}

  \begin{tanim}
    \alert{yönlü çizge}:\\
    düğümler arasındaki bağlantılar yönlü

    \pause
    \begin{itemize}
      \item ayrıt yerine \alert{yay}: $D=(V,A)$
      \item \alert{başlangıç} ve \alert{bitiş} düğümleri
    \end{itemize}
  \end{tanim}
\end{frame}

\begin{frame}
  \frametitle{Yönlü Çizge Örneği}

  \begin{ornek}
    \begin{center}
      \pgfuseimage{yonlu}
    \end{center}
  \end{ornek}
\end{frame}

\begin{frame}
  \frametitle{Yönlü Çizgelerde Kerte}

  \begin{itemize}
    \item kerte ikiye ayrılıyor
    \begin{itemize}
      \item \emph{giriş kertesi}: ${d_v}^i$
      \item \emph{çıkış kertesi}: ${d_v}^o$
    \end{itemize}

    \pause
    \item $\sum_{v \in V} {d_v}^i = \sum_{v \in V} {d_v}^o = |A|$

    \pause
    \medskip
    \item giriş kertesi 0 olan düğüm: \emph{kaynak}
    \item çıkış kertesi 0 olan düğüm: \emph{kuyu}
  \end{itemize}
\end{frame}

\begin{frame}
  \frametitle{Bitişiklik Matrisi Örneği}

  \begin{ornek}
    \begin{columns}
    \column{.5\textwidth}
    \begin{center}
      \pgfuseimage{yonlu}
    \end{center}

    \column{.5\textwidth}
      \[
        \begin{array}{c|cccc}
              & a & b & c & d\\\hline
            a & 0 & 0 & 0 & 1\\
            b & 2 & 1 & 1 & 0\\
            c & 0 & 0 & 0 & 0\\
            d & 0 & 1 & 1 & 0
        \end{array}
      \]
    \end{columns}
  \end{ornek}
\end{frame}

\subsection{İzomorfizm}

\begin{frame}
  \frametitle{İzomorfizm}

  \begin{tanim}
    \alert{izomorfik çizgeler}:\\
    $G=(V,E)$ ve $G^\star=(V^\star,E^\star)$ çizgeleri izomorfik ise\\
    $\exists f: V \rightarrow V^\star~(u,v) \in E \Rightarrow (f(u),f(v)) \in E^\star$

    \begin{itemize}
      \item $f$ birebir ve örten
    \end{itemize}
  \end{tanim}

  \pause
  \begin{itemize}
    \item aynı şekilde çizilebilir
  \end{itemize}
\end{frame}

\begin{frame}
  \frametitle{İzomorfizm Örneği}

  \begin{ornek}
    \begin{columns}
      \column{.4\textwidth}
      \begin{center}
        \pgfuseimage{izomorff}
      \end{center}

      \column{.6\textwidth}
      \begin{center}
        \pgfuseimage{izomorft}
      \end{center}
    \end{columns}

    \pause
    \bigskip
    \begin{itemize}
      \item $f = \{(a,d),(b,e),(c,b),(d,c),(e,a)\}$
    \end{itemize}
  \end{ornek}
\end{frame}

\begin{frame}
  \frametitle{İzomorfizm Örneği}

  \begin{ornek}[Petersen çizgesi]
    \begin{columns}
      \column{.4\textwidth}
      \begin{center}
        \pgfuseimage{petersen1}
      \end{center}

      \column{.6\textwidth}
      \begin{center}
        \pgfuseimage{petersen2}
      \end{center}
    \end{columns}

    \pause
    \bigskip
    \begin{itemize}
      \item $f = \{(a,q),(b,v),(c,u),(d,y),(e,r),$\\
        $~~~~~~~(f,w),(g,x),(h,t),(i,z),(j,s)\}$
    \end{itemize}
  \end{ornek}
\end{frame}

\begin{frame}
  \frametitle{Homeomorfizm}

  \begin{tanim}
    \alert{homeomorfik çizgeler}:\\
    izomorfik çizgelere ayrıtları bölen düğümler ekleyerek\\
    elde edilen çizgeler
  \end{tanim}
\end{frame}

\begin{frame}
  \frametitle{Homeomorfizm Örneği}

  \begin{ornek}
    \begin{columns}
      \column{.5\textwidth}
      \begin{center}
        \pgfuseimage{izomorft}
      \end{center}

      \column{.5\textwidth}
      \begin{center}
        \pgfuseimage{homeomorft}
      \end{center}
    \end{columns}
  \end{ornek}
\end{frame}

\subsection{Bağlılık}

\begin{frame}
  \frametitle{Dolaşı}

  \begin{tanim}
    \alert{dolaşı}:\\
    bir başlangıç düğümünden ($v_0$) bir varış düğümüne ($v_n$) doğru

    \[
      v_0,e_1,v_1,e_2,v_2,e_3,v_3,\dots,e_{n-1},v_{n-1},e_n,v_n
    \]

    şeklinde yineleyen düğüm-ayrıt dizisi
  \end{tanim}

  \pause
  \begin{itemize}
    \item $e_i=(v_{i-1},v_i)$ olduğundan ayrıtları yazmaya gerek yok

    \pause
    \medskip
    \item \alert{uzunluk}: ayrıt sayısı
    \item $v_0 \neq v_n$ ise \alert{açık}, $v_0 = v_n$ ise \alert{kapalı}
  \end{itemize}
\end{frame}

\begin{frame}
  \frametitle{Dolaşı Örneği}

  \begin{ornek}
    \begin{columns}
      \column{.58\textwidth}
      \begin{center}
        \pgfuseimage{yalin}
      \end{center}

      \column{.4\textwidth}
      $(c,b),(b,a),(a,d),(d,e),$\\
      $(e,f),(f,a),(a,b)$

      \medskip
      $c,b,a,d,e,f,a,b$

      \bigskip
      uzunluk: 7
    \end{columns}
  \end{ornek}
\end{frame}

\begin{frame}
  \frametitle{Gezi}

  \begin{tanim}
    \alert{gezi}: ayrıtların yinelenmediği dolaşı

    \pause
    \begin{itemize}
      \item kapalı gezi: \alert{devre}
      \item \alert{kapsayan gezi}: bütün ayrıtlardan geçilen gezi
    \end{itemize}
  \end{tanim}
\end{frame}

\begin{frame}
  \frametitle{Gezi Örneği}

  \begin{ornek}
    \begin{columns}
      \column{.58\textwidth}
      \begin{center}
        \pgfuseimage{yalin}
      \end{center}

      \column{.4\textwidth}
      $(c,b),(b,a),(a,e),(e,d),$\\
      $(d,a),(a,f)$

      \medskip
      $c,b,a,e,d,a,f$
    \end{columns}
  \end{ornek}
\end{frame}

\begin{frame}
  \frametitle{Yol}

  \begin{tanim}
    \alert{yol}: düğümlerin yinelenmediği dolaşı

    \pause
    \begin{itemize}
      \item kapalı yol: \alert{çevre}
      \item \alert{kapsayan yol}: bütün düğümlere uğranan yol
    \end{itemize}
  \end{tanim}
\end{frame}

\begin{frame}
  \frametitle{Yol Örneği}

  \begin{ornek}
    \begin{columns}
      \column{.58\textwidth}
      \begin{center}
        \pgfuseimage{yalin}
      \end{center}

      \column{.4\textwidth}
      $(c,b),(b,a),(a,d),(d,e),$\\
      $(e,f)$

      \medskip
      $c,b,a,d,e,f$
    \end{columns}
  \end{ornek}
\end{frame}
%
% \begin{frame}
%   \frametitle{Gezi - Yol İlişkisi}
%
%   \begin{teorem}
%     $a$ düğümünden $b$ düğümüne bir gezi varsa\\
%     $a$ düğümünden $b$ düğümüne bir yol da vardır.
%   \end{teorem}
% \end{frame}

\begin{frame}
  \frametitle{Bağlılık}

  \begin{tanim}
    \alert{bağlı çizge}:\\
    seçilebilecek her düğüm çifti arasında bir yol var
  \end{tanim}

  \pause
  \begin{itemize}
    \item bağlı olmayan bir çizge bağlı bileşenlere ayrılabilir
  \end{itemize}
\end{frame}

\begin{frame}
  \frametitle{Bağlı Bileşen Örneği}

  \begin{ornek}
    \begin{columns}
      \column{.55\textwidth}
      \begin{center}
        \pgfuseimage{bagsiz}
      \end{center}

      \column{.43\textwidth}
      \pause
      \begin{itemize}
        \item çizge bağlı değil:\\
          $a$ ile $c$ arasında yol yok
        \item bağlı bileşenler:\\
          $a,d,e$\\
          $b,c$\\
          $f$
      \end{itemize}
    \end{columns}
  \end{ornek}
\end{frame}

\begin{frame}
  \frametitle{Uzaklık}

  \begin{tanim}
    \alert{uzaklık}:\\
    iki düğüm arasındaki en kısa yolun uzunluğu
  \end{tanim}

  \pause
  \begin{tanim}
    \alert{çap}:\\
    çizgede birbirine en uzak iki düğüm arasındaki uzaklık
  \end{tanim}
\end{frame}

\begin{frame}
  \frametitle{Uzaklık Örneği}

  \begin{ornek}
    \begin{columns}
      \column{.5\textwidth}
      \begin{center}
        \pgfuseimage{uzaklik}
      \end{center}

      \column{.5\textwidth}
      \pause
      \begin{itemize}
        \item $a$ ile $e$ düğümlerinin\\
          uzaklığı: 2
        \item çap: 3
      \end{itemize}
    \end{columns}
  \end{ornek}
\end{frame}

\begin{frame}
  \frametitle{Kesitleme Noktası}

  \begin{tanim}
    \alert{$G - v$ çizgesi}:\\
    $G$ çizgesinden $v$ düğümü ve ona çakışık bütün ayrıtların çıkarılmasıyla
    elde edilen çizge
  \end{tanim}

  \pause
  \begin{tanim}
    \alert{kesitleme noktası}:\\
    $G$ çizgesi bağlı ama $G - v$ çizgesi bağlı değilse $v$ bir kesitleme
    noktasıdır
  \end{tanim}
\end{frame}

\begin{frame}
  \frametitle{Kesitleme Noktası Örneği}

  \begin{columns}
    \column{.5\textwidth}
    \begin{block}{$G$ çizgesi}
      \begin{center}
        \pgfuseimage{uzaklik}
      \end{center}
    \end{block}

    \column{.5\textwidth}
    \begin{block}{$G - d$ çizgesi}
      \begin{center}
        \pgfuseimage{kesitleme}
      \end{center}
    \end{block}
  \end{columns}
\end{frame}

\begin{frame}
  \frametitle{Yönlü Dolaşılar}

  \begin{itemize}
    \item yönsüz çizgelere benzer şekilde

    \pause
    \item yaylar yönsüz varsayarak tanımlanırsa:\\
      \emph{yarı-dolaşı}, \emph{yarı-gezi}, \emph{yarı-yol}
  \end{itemize}
\end{frame}

\begin{frame}
  \frametitle{Zayıf Bağlı Çizge}

  \begin{columns}
    \column{.5\textwidth}
    \begin{tanim}
      \emph{zayıf bağlı çizge}:\\
      her düğüm çifti arasında bir yarı-yol var
    \end{tanim}

    \column{.5\textwidth}
    \begin{ornek}
      \begin{center}
        \pgfuseimage{zayif}
      \end{center}
    \end{ornek}
  \end{columns}
\end{frame}

\begin{frame}
  \frametitle{Tek-Yönlü Bağlı Çizge}

  \begin{columns}
    \column{.5\textwidth}
    \begin{tanim}
      \emph{tek-yönlü bağlı çizge}:\\
      her düğüm çifti arasında birinden diğerine yol var
    \end{tanim}

    \column{.5\textwidth}
    \begin{ornek}
      \begin{center}
        \pgfuseimage{tekyonlu}
      \end{center}
    \end{ornek}
  \end{columns}
\end{frame}

\begin{frame}
  \frametitle{Güçlü Bağlı Çizge}

  \begin{columns}
    \column{.5\textwidth}
    \begin{tanim}
      \emph{güçlü bağlı çizge}:\\
      her düğüm çifti arasında bir yol var
    \end{tanim}

    \column{.5\textwidth}
    \begin{ornek}
      \begin{center}
        \pgfuseimage{guclu}
      \end{center}
    \end{ornek}
  \end{columns}
\end{frame}

\begin{frame}
  \frametitle{Yönlü Bağlılık}

  \begin{teorem}
    sonlu, yönlü bir $D$ çizgesinin

    \begin{itemize}
      \item zayıf bağlı olması için kapsayan bir yarı-dolaşı içermesi
      \item<2-> tek-yönlü bağlı olması için kapsayan bir açık dolaşı içermesi
      \item<3-> güçlü bağlı olması için kapsayan bir kapalı dolaşı içermesi
    \end{itemize}

    gerek ve yeter koşuldur
  \end{teorem}
\end{frame}

\begin{frame}
  \frametitle{Bağlantı Matrisi}

  \begin{itemize}
    \item çizgenin bitişiklik matrisi $A$ ise\\
      $A^k$ matrisinin $(i,j)$ elemanı $i$. düğüm ile $j$. düğüm arasındaki\\
      $k$ uzunluklu dolaşıların sayısını gösterir

    \pause
    \item $n$ düğümlü yönsüz bir çizgede iki düğüm arasındaki uzaklık\\
      en fazla $n-1$ olabilir

    \pause
    \medskip
    \item \alert{bağlantı matrisi}:\\
      $C = A^1 + A^2 + A^3 + \dots + A^{n-1}$
    \begin{itemize}
      \item bütün elemanlar sıfırdan farklı ise çizge bağlıdır
    \end{itemize}
  \end{itemize}
\end{frame}

\begin{frame}
  \frametitle{Warshall Algoritması}

  \begin{itemize}
    \item düğümler arasındaki yolların sayısı yerine\\
      yol olup olmadığını belirlemek daha kolay

    \pause
    \medskip
    \item sırayla her düğüm için:
    \begin{itemize}
      \item o düğüme gelinebilen düğümlerden\\
        (matriste o sütunda 1 olan satırlardan)

      \item o düğümden gidilebilen düğümlere\\
        (matriste o satırda 1 olan sütunlara)
    \end{itemize}
  \end{itemize}
\end{frame}

\begin{frame}
  \frametitle{Warshall Algoritması Örneği}

  \begin{ornek}
    \begin{columns}
      \column{.5\textwidth}
      \begin{center}
        \pgfuseimage{warshall}
      \end{center}

      \column{.5\textwidth}
      \[
        \begin{array}{c|cccc}
              & a & b & c & d\\\hline
            a & 0 & \alert{1} & 0 & 0\\
            b & 0 & 1 & 0 & 0\\
            c & 0 & 0 & 0 & 1\\
            d & \alert{1} & 0 & 1 & 0
        \end{array}
      \]
    \end{columns}
  \end{ornek}
\end{frame}

\begin{frame}
  \frametitle{Warshall Algoritması Örneği}

  \begin{ornek}
    \begin{columns}
      \column{.5\textwidth}
      \begin{center}
        \pgfuseimage{warshall1}
      \end{center}

      \column{.5\textwidth}
      \[
        \begin{array}{c|cccc}
              & a & b & c & d\\\hline
            a & 0 & \alert{1} & 0 & 0\\
            b & 0 & 1 & 0 & 0\\
            c & 0 & 0 & 0 & 1\\
            d & 1 & \alert{1} & 1 & 0
        \end{array}
      \]
    \end{columns}
  \end{ornek}
\end{frame}

\begin{frame}
  \frametitle{Warshall Algoritması Örneği}

  \begin{ornek}
    \begin{columns}
      \column{.5\textwidth}
      \begin{center}
        \pgfuseimage{warshall1}
      \end{center}

      \column{.5\textwidth}
      \[
        \begin{array}{c|cccc}
              & a & b & c & d\\\hline
            a & 0 & 1 & 0 & 0\\
            b & 0 & 1 & 0 & 0\\
            c & 0 & 0 & 0 & \alert{1}\\
            d & 1 & 1 & \alert{1} & 0
        \end{array}
      \]
    \end{columns}
  \end{ornek}
\end{frame}

\begin{frame}
  \frametitle{Warshall Algoritması Örneği}

  \begin{ornek}
    \begin{columns}
      \column{.5\textwidth}
      \begin{center}
        \pgfuseimage{warshall2}
      \end{center}

      \column{.5\textwidth}
      \[
        \begin{array}{c|cccc}
              & a & b & c & d\\\hline
            a & 0 & 1 & 0 & 0\\
            b & 0 & 1 & 0 & 0\\
            c & 0 & 0 & 0 & \alert{1}\\
            d & \alert{1} & \alert{1} & \alert{1} & \alert{1}
        \end{array}
      \]
    \end{columns}
  \end{ornek}
\end{frame}

\begin{frame}
  \frametitle{Warshall Algoritması Örneği}

  \begin{ornek}
    \begin{columns}
      \column{.5\textwidth}
      \begin{center}
        \pgfuseimage{warshall3}
      \end{center}

      \column{.5\textwidth}
      \[
        \begin{array}{c|cccc}
              & a & b & c & d\\\hline
            a & 0 & 1 & 0 & 0\\
            b & 0 & 1 & 0 & 0\\
            c & 1 & 1 & 1 & 1\\
            d & 1 & 1 & 1 & 1
        \end{array}
      \]
    \end{columns}
  \end{ornek}
\end{frame}

\section{Özel Çizgeler}

\subsection{Geçit Veren Çizgeler}

\begin{frame}
  \frametitle{Königsberg Köprüleri}

  \begin{center}
    \pgfuseimage{konigsberg}
  \end{center}

  \begin{itemize}
    \item herhangi bir kara parçasından başlayarak,\\
      bütün köprülerden bir kere geçerek\\
      başlangıç noktasına dönülebilir mi?
  \end{itemize}
\end{frame}

\begin{frame}
  \frametitle{Geçit Veren Çizge}

  \begin{tanim}
    \alert{geçit veren çizge}:\\
    üzerinde kapsayan bir gezi düzenlenebilen çizge
  \end{tanim}

  \begin{itemize}
    \pause
    \item kertesi tek olan bir düğümün geçit veren bir çizgede yer alabilmesi
      için gezinin ya başlangıç ya da varış düğümü olması gerekir

    \pause
    \item başlangıç ve varış dışındaki bütün düğümlerin kerteleri çift olmalı
  \end{itemize}
\end{frame}

\begin{frame}
  \frametitle{Geçit Veren Çizge Örneği}

  \begin{ornek}
    \begin{columns}
      \column{.4\textwidth}
      \begin{center}
        \pgfuseimage{zarf}
      \end{center}

      \pause
      \column{.6\textwidth}
      \begin{itemize}
        \item $a$, $b$ ve $c$ düğümlerinin\\
          kerteleri çift
        \item $d$ ve $e$ düğümlerinin\\
          kerteleri tek
        \pause
        \item $d$ düğümünden başlayıp\\
          $e$ düğümünde biten (ya da tersi)\\
          bir kapsayan gezi oluşturulabilir: $d,b,a,c,e,d,c,b,e$
      \end{itemize}
    \end{columns}
  \end{ornek}
\end{frame}

\begin{frame}
  \frametitle{Königsberg Köprüleri}

  \begin{center}
    \pgfuseimage{konigcizge}
  \end{center}

  \pause
  \begin{itemize}
    \item bütün düğümlerin kerteleri tek: geçit vermez
  \end{itemize}
\end{frame}

\begin{frame}
  \frametitle{Euler Çizgeleri}

  \begin{tanim}
    \alert{Euler çizgesi}:\\
      üzerinde kapalı bir kapsayan gezi düzenlenebilen çizge

    \pause
    \begin{itemize}
      \item Euler çizgesi $\Leftrightarrow$ bütün düğümlerin kerteleri çift
    \end{itemize}
  \end{tanim}
\end{frame}

\begin{frame}
  \frametitle{Euler Çizgesi Örnekleri}

  \begin{columns}
    \column{.5\textwidth}
    \begin{ornek}[Euler çizgesi]
      \begin{center}
        \pgfuseimage{euler}
      \end{center}
    \end{ornek}

    \column{.5\textwidth}
    \begin{ornek}[Euler çizgesi değil]
      \begin{center}
        \pgfuseimage{hamilton}
      \end{center}
    \end{ornek}
  \end{columns}
\end{frame}

\begin{frame}
  \frametitle{Hamilton Çizgeleri}

  \begin{tanim}
    \alert{Hamilton çizgesi}:\\
      üzerinde kapalı bir kapsayan yol düzenlenebilen çizge
  \end{tanim}
\end{frame}

\begin{frame}
  \frametitle{Hamilton Çizgesi Örnekleri}

  \begin{columns}
    \column{.5\textwidth}
    \begin{ornek}[Hamilton çizgesi]
      \begin{center}
        \pgfuseimage{hamilton}
      \end{center}
    \end{ornek}

    \column{.5\textwidth}
    \begin{ornek}[Hamilton çizgesi değil]
      \begin{center}
        \pgfuseimage{euler}
      \end{center}
    \end{ornek}
  \end{columns}
\end{frame}

\subsection{Düzenli Çizgeler}

\begin{frame}
  \frametitle{Düzenli Çizgeler}

  \begin{tanim}
    \alert{düzenli çizge}:\\
    bütün düğümlerin kertelerinin aynı olduğu çizge

    \begin{itemize}
      \item $n$-düzenli: bütün düğümlerin kerteleri $n$
    \end{itemize}
  \end{tanim}
\end{frame}

\begin{frame}
  \frametitle{Düzenli Çizge Örnekleri}

  \begin{ornek}
    \begin{center}
      \pgfuseimage{duzenli}
    \end{center}
  \end{ornek}
\end{frame}

\begin{frame}
  \frametitle{Tam Bağlı Çizgeler}

  \begin{tanim}
    \alert{tam bağlı çizge}:\\
    her düğüm çifti arasında ayrıt olan çizge

    \begin{itemize}
      \item $K_n$: $n$ düğümlü tam bağlı çizge
    \end{itemize}
  \end{tanim}
\end{frame}

\begin{frame}
  \frametitle{Tam Bağlı Çizge Örnekleri}

  \begin{columns}
    \column{.5\textwidth}
    \begin{ornek}[$K_4$]
      \begin{center}
        \pgfuseimage{k4}
      \end{center}
    \end{ornek}

    \column{.5\textwidth}
    \begin{ornek}[$K_5$]
      \begin{center}
        \pgfuseimage{k5}
      \end{center}
    \end{ornek}
  \end{columns}
\end{frame}

\begin{frame}
  \frametitle{İki Parçalı Çizgeler}

  \begin{tanim}
    \alert{iki parçalı çizge}:\\
    $V = V_1 \cup V_2 \wedge V_1 \cap V_2 = \emptyset$\\
    $\forall (v_1,v_2) \in E~v_1 \in V_1 \wedge v_2 \in V_2$

    \pause
    \begin{itemize}
      \item \emph{tam bağlı iki parçalı çizge}:\\
      $\forall v_1 \in V_1 \forall v_2 \in V_2~(v_1,v_2) \in E$
      \begin{itemize}
        \item $K_{m,n}$: $|V_1|=m$, $|V_2|=n$
      \end{itemize}
    \end{itemize}
  \end{tanim}
\end{frame}

\begin{frame}
  \frametitle{İki Parçalı Çizge Örnekleri}

  \begin{columns}[t]
    \column{.5\textwidth}
    \begin{ornek}[$K_{2,3}$]
      \begin{center}
        \pgfuseimage{k23}
      \end{center}
    \end{ornek}

    \column{.5\textwidth}
    \begin{ornek}[$K_{3,3}$]
      \begin{center}
        \pgfuseimage{k33}
      \end{center}
    \end{ornek}
  \end{columns}
\end{frame}

\subsection{Düzlemsel Çizgeler}

\begin{frame}
  \frametitle{Düzlemsel Çizgeler}

  \begin{tanim}
    \alert{düzlemsel çizge}:\\
      ayrıtları kesişmeden bir düzleme çizilebilen çizge

    \begin{itemize}
      \item \alert{harita}: çizgenin düzlemsel bir çizimi
    \end{itemize}
  \end{tanim}
\end{frame}

\begin{frame}
  \frametitle{Düzlemsel Çizge Örneği}

  \begin{ornek}[$K_4$]
    \begin{columns}
      \column{.5\textwidth}
      \begin{center}
        \pgfuseimage{k4}
      \end{center}

      \column{.5\textwidth}
      \begin{center}
        \pgfuseimage{k4duzlemsel}
      \end{center}
    \end{columns}
  \end{ornek}
\end{frame}

\begin{frame}
  \frametitle{Bölgeler}

  \begin{itemize}
    \item bir harita düzlemi \emph{bölgelere} ayırır
    \item \emph{bölge kertesi}:\\
      bölgenin sınırı oluşturan dolaşının uzunluğu
  \end{itemize}

  \pause
  \begin{teorem}
    $r_i$ bölgesinin kertesi $d_{r_i}$ ise:

    \[ |E| = \frac{\sum_i d_{r_i}}{2} \]
  \end{teorem}
\end{frame}

\begin{frame}
  \frametitle{Bölge Örneği}

  \begin{ornek}
    \begin{columns}
      \column{.58\textwidth}
      \begin{center}
        \pgfuseimage{bolge}
      \end{center}

      \pause
      \column{.4\textwidth}
      $d_{r_1} = 3$ (abda)\\
      $d_{r_2} = 3$ (bcdb)\\
      $d_{r_3} = 5$ (cdefec)\\
      $d_{r_4} = 4$ (abcea)\\
      $d_{r_5} = 3$ (adea)

      \medskip
      $\sum_r d_r = 18$\\
      $|E| = 9$
    \end{columns}
  \end{ornek}
\end{frame}

\begin{frame}
  \frametitle{Euler Formülü}

  \begin{teorem}[Euler Formülü]
    Bağlı, düzlemsel çizgelerde $|V| - |E| + |R| = 2$.
  \end{teorem}
\end{frame}

\begin{frame}
  \frametitle{Euler Formülü Örneği}

  \begin{ornek}
    \begin{center}
      \pgfuseimage{bolge}
    \end{center}

    \begin{itemize}
     \item $|V| = 6$, $|E| = 9$, $|R| = 5$
    \end{itemize}
  \end{ornek}
\end{frame}

\begin{frame}
  \frametitle{Euler Formülünün Tanıtı}

  \begin{block}{Tanıt}
    yöntem: $|E|$ üzerinden tümevarım

    \pause
    \begin{itemize}
      \item taban adımı: tek düğüm, ayrıt yok\\
        $|V| = 1$, $|E| = 0$, $|R| = 1$

      \pause
      \item herhangi bir düzlemsel çizge için doğruluğunu varsayalım
    \end{itemize}
  \end{block}
\end{frame}

\begin{frame}
  \frametitle{Euler Formülünün Tanıtı}

  \begin{proof}[Tümevarım Adımı]
    \begin{columns}[t]
      \column{.5\textwidth}
      \begin{itemize}
        \item yeni bir düğümü var olanlardan birine bağla:

        \medskip
        \pgfuseimage{eulertanit1}
      \end{itemize}

      \column{.5\textwidth}
      \begin{itemize}
        \item var olan iki düğüm arasına bir ayrıt ekle:

        \medskip
        \pgfuseimage{eulertanit2}
      \end{itemize}
    \end{columns}

    \pause
    \begin{columns}
      \column{.5\textwidth}
      \begin{itemize}
        \item $|V|$ 1 artar, $|E|$ 1 artar,\\
          $|R|$ aynı kalır
      \end{itemize}

      \pause
      \column{.5\textwidth}
      \begin{itemize}
        \item $|V|$ aynı kalır, $|E|$ 1 artar,\\
          $|R|$ 1 artar
      \end{itemize}
    \end{columns}
  \end{proof}
\end{frame}

\begin{frame}
  \frametitle{Düzlemsel Çizge Teoremleri}

  \begin{teorem}
    yalın, düzlemsel bir çizgede:\\
    $|V| \geq 3 \Rightarrow |E| \leq 3 |V| - 6$
  \end{teorem}

  \pause
  \begin{proof}[Tanıt]
    \begin{itemize}
      \item bölge kertelerinin toplamı: $2 |E|$

      \pause
      \item bir bölgenin kertesi en az $3$\\
        \pause
        $\Rightarrow 2 |E| \geq 3 |R|$
        \pause
        $\Rightarrow |R| \leq \frac{2}{3} |E|$

      \pause
      \item $|V| - |E| + |R| = 2$\\
        \pause
        $\Rightarrow |V| - |E| + \frac{2}{3} |E| \geq 2$
        \pause
        $\Rightarrow |V| - \frac{1}{3} |E| \geq 2$\\
        \pause
        $\Rightarrow 3 |V| - |E| \geq 6$
        \pause
        $\Rightarrow |E| \leq 3 |V| - 6$\\
    \end{itemize}
  \end{proof}
\end{frame}

\begin{frame}
  \frametitle{Düzlemsel Çizge Teoremleri}

  \begin{teorem}
    Bağlı, yalın, düzlemsel bir çizgede\\
    $|V| \geq 3 \Rightarrow \exists v \in V~d_v \leq 5$
  \end{teorem}

  \pause
  \begin{proof}[Tanıt]
    \begin{itemize}
      \item $\forall v \in V~d_v \geq 6$ olsun\\
        \pause
        $\Rightarrow 2 |E| \geq 6 |V|$\\
        \pause
        $\Rightarrow |E| \geq 3 |V|$\\
        \pause
        $\Rightarrow |E| > 3 |V| - 6$: \alert{çelişki}
    \end{itemize}
  \end{proof}
\end{frame}

\begin{frame}
  \frametitle{Düzlemsel Olmayan Çizgeler}

  \begin{columns}
    \column{.45\textwidth}
    \begin{teorem}
      \begin{center}
        \pgfuseimage{k5}
      \end{center}

      $K_5$ çizgesi düzlemsel değildir.
    \end{teorem}

    \pause
    \column{.55\textwidth}
    \begin{proof}[Tanıt]
      \begin{itemize}
        \item $|V| = 5$

        \pause
        \item $3 |V| - 6 = 3 \cdot 5 - 6 = 9$

        \pause
        \item $|E| \leq 9$ olmalı\\

        \pause
        \item ama $|E| = 10$: \alert{çelişki}
      \end{itemize}
    \end{proof}
  \end{columns}
\end{frame}

\begin{frame}
  \frametitle{Düzlemsel Olmayan Çizgeler}

  \begin{columns}
    \column{.45\textwidth}
    \begin{teorem}
      \begin{center}
        \pgfuseimage{k33}
      \end{center}

      $K_{3,3}$ çizgesi düzlemsel değildir.
    \end{teorem}

    \pause
    \column{.55\textwidth}
    \begin{proof}[Tanıt]
      \begin{itemize}
        \item $|V| = 6, |E| = 9$

        \pause
        \item düzlemsel ise $|R| = 5$ olmalı

        \pause
        \item bir bölgenin kertesi en az $4$\\
          $\Rightarrow \sum_{r \in R} d_r \geq 20$

        \pause
        \item $|E| \geq 10$ olmalı\\

        \pause
        \item ama $|E| = 9$: \alert{çelişki}
      \end{itemize}
    \end{proof}
  \end{columns}
\end{frame}

\begin{frame}
  \frametitle{Kuratowski Teoremi}

  \begin{teorem}
    çizgenin $K_5$ ya da $K_{3,3}$ çizgelerine homeomorf bir altçizgesi var\\
    $\Leftrightarrow$ çizge düzlemsel değil
  \end{teorem}
\end{frame}

\begin{frame}
  \frametitle{Platon Cisimleri}

  \begin{tanim}
    \alert{düzgün çokyüzlü}:\\
    yüzeyi aynı eşkenar çokgenlerden oluşan üç boyutlu cisim
  \end{tanim}

  \pause
  \begin{itemize}
    \item bir düzgün çokyüzlünün iki boyutlu düzleme izdüşümü düzlemsel bir
      çizgedir
    \begin{itemize}
      \item her köşe: düğüm
      \item her kenar: ayrıt
    \end{itemize}
  \end{itemize}
\end{frame}

\begin{frame}
  \frametitle{Platon Cisimleri}

  \begin{ornek}[küp: düzgün 6-yüzlü]
    \begin{center}
      \pgfuseimage{kup}
    \end{center}
  \end{ornek}
\end{frame}

\begin{frame}
  \frametitle{Platon Cisimlerinin Sayısı}

  \begin{itemize}
    \item $v$: düğüm (köşe) sayısı
    \item $e$: ayrıt (kenar) sayısı
    \item $r$: bölge (yüzey) sayısı
    \item $n$: bir köşede birleşen yüzey sayısı = düğüm kertesi
    \item $m$: bir yüzeyi çevreleyen ayrıt sayısı = bölge kertesi
  \end{itemize}

  \pause
  \begin{itemize}
    \item $m,n \geq 3$
    \item $2e = m \cdot r$
    \item $2e = n \cdot v$
  \end{itemize}
\end{frame}

\begin{frame}
  \frametitle{Platon Cisimlerinin Sayısı}

    \begin{itemize}
      \item Euler formülünden:
      \[
        0 < 2 = v - e + r = \frac{2e}{n} - e + \frac{2e}{m}
        = e \Big( \frac{2m-mn+2n}{mn} \Big)
      \]

      \pause
      \item $e,m,n > 0$ olduğuna göre:
      \begin{eqnarray*}
        2m - mn + 2n > 0 \Rightarrow mn - 2m -2n < 0 \\\pause
        \Rightarrow mn - 2m - 2n + 4 < 4 \pause \Rightarrow (m - 2)(n - 2) < 4
      \end{eqnarray*}

      \pause
      \item bu eşitsizliği sağlayan değerler:
      \begin{enumerate}
        \item $m=3, n=3$
        \item $m=4, n=3$
        \item $m=3, n=4$
        \item $m=5, n=3$
        \item $m=3, n=5$
      \end{enumerate}
    \end{itemize}
\end{frame}

\begin{frame}
  \frametitle{Tetrahedron - Düzgün 4 Yüzlü}

  \begin{columns}
    \column{.7\textwidth}
    \begin{center}
      \pgfuseimage{tetrahedron}
    \end{center}

    \column{.3\textwidth}
    \begin{center}
      \pgfuseimage{planartetra}

      $m=3, n=3$
    \end{center}
  \end{columns}
\end{frame}

\begin{frame}
  \frametitle{Hexahedron - Küp}

  \begin{columns}
    \column{.7\textwidth}
    \begin{center}
      \pgfuseimage{hexahedron}
    \end{center}

    \column{.3\textwidth}
    \begin{center}
      \pgfuseimage{planarhexa}

      $m=4, n=3$
    \end{center}
  \end{columns}
\end{frame}

\begin{frame}
  \frametitle{Octahedron - Düzgün 8 Yüzlü}

  \begin{columns}
    \column{.7\textwidth}
    \begin{center}
      \pgfuseimage{octahedron}
    \end{center}

    \column{.3\textwidth}
    \begin{center}
      \pgfuseimage{planarocta}

      $m=3, n=4$
    \end{center}
  \end{columns}
\end{frame}

\begin{frame}
  \frametitle{Dodecahedron - Düzgün 12 Yüzlü}

  \begin{columns}
    \column{.7\textwidth}
    \begin{center}
      \pgfuseimage{dodecahedron}
    \end{center}

    \column{.3\textwidth}
    \begin{center}
      \pgfuseimage{planardodeca}

      $m=5, n=3$
    \end{center}
  \end{columns}
\end{frame}

\begin{frame}
  \frametitle{Icosahedron - Düzgün 20 Yüzlü}

  \begin{columns}
    \column{.7\textwidth}
    \begin{center}
      \pgfuseimage{icosahedron}
    \end{center}

    \column{.3\textwidth}
    $m=3, n=5$
  \end{columns}
\end{frame}

\subsection{Çizge Boyama}

% TODO: bölge boyama ile ilgili bilgi ekle

\begin{frame}
  \frametitle{Çizge Boyama}

  \begin{ornek}
    \begin{itemize}
      \item kimyasal maddeler üreten bir firma
      \item bazı maddeler birlikte depolanamıyor
      \item birbiriyle aynı ortamda olamayacak maddeler farklı depolara
        yerleştirilmeli

      \pause
      \medskip
      \item \alert{problem}: en az sayıda depo kullanılacak şekilde ürünleri
        depola
    \end{itemize}
  \end{ornek}
\end{frame}

\begin{frame}
  \frametitle{Çizge Boyama}

  \begin{ornek}
    \begin{itemize}
      \item 25 farklı kimyasal madde: $V = \{ c_1, c_2,\cdots, c_{25}\}$
      \item her $1 \leq i, j \leq 25$ için:\\
        $(c_i,c_j)$ bitişik: $c_i$ ve $c_j$ birbiriyle etkileşir
    \end{itemize}

%     \pause
    \begin{center}
      \pgfuseimage{besmadde}
    \end{center}
  \end{ornek}
\end{frame}

\begin{frame}
  \frametitle{Düzgün Boyama}

  \begin{tanim}
    \alert{düzgün boyama}:\\
      $G=(V,E)$ çizgesinde her $(a,b) \in E$ için $a$ ve $b$ düğümlerinin
      renkleri farklı olacak şekilde bütün düğümlere renk atama
  \end{tanim}
\end{frame}

\begin{frame}
  \frametitle{Çizge Boyama Örneği}

  \begin{ornek}
    \begin{center}
      \pgfuseimage{boyama1}
    \end{center}
  \end{ornek}
\end{frame}

\begin{frame}
  \frametitle{Çizge Boyama Örneği}

  \begin{ornek}
    \begin{columns}
      \column{.5\textwidth}
      \begin{center}
        \pgfuseimage{boyama2}
      \end{center}

      \column{.5\textwidth}
      \begin{center}
        \pgfuseimage{boyama3}
      \end{center}
    \end{columns}
  \end{ornek}
\end{frame}

\begin{frame}
  \frametitle{Çizge Boyama Örneği}

  \begin{ornek}
    \begin{columns}
      \column{.5\textwidth}
      \begin{center}
        \pgfuseimage{boyama4}
      \end{center}

      \column{.5\textwidth}
      \begin{center}
        \pgfuseimage{boyama5}
      \end{center}
    \end{columns}
  \end{ornek}
\end{frame}

\begin{frame}
  \frametitle{Çizge Boyama Örneği}

  \begin{ornek}
    \begin{columns}
      \column{.5\textwidth}
      \begin{center}
        \pgfuseimage{boyama6}
      \end{center}

      \column{.5\textwidth}
      \begin{center}
        \pgfuseimage{boyama7}
      \end{center}
    \end{columns}
  \end{ornek}
\end{frame}

\begin{frame}
  \frametitle{Kromatik Sayı}

  \begin{tanim}
    \alert{kromatik sayı}:\\
      G çizgesini düzgün boyamak için gerekli en az renk sayısı: $\chi (G)$
  \end{tanim}

  \pause
  \begin{itemize}
     \item $\chi (G)$'yi hesaplamak çok zor bir problem
     \item $n \geq 1$ için $\chi (K_n) = n$
  \end{itemize}
\end{frame}

\begin{frame}
  \frametitle{Kromatik Sayı Örneği}

  \begin{ornek}[Herschel çizgesi]
    \begin{center}
      \pgfuseimage{herschel}
    \end{center}

    \begin{itemize}
      \item kromatik sayı: 2
    \end{itemize}
  \end{ornek}
\end{frame}

\begin{frame}
  \frametitle{Çizge Boyama Örneği}

  \begin{ornek}[Sudoku]
    \begin{columns}[t]
      \column{.4\textwidth}
      \begin{center}
        \pgfuseimage{sudoku}
      \end{center}

      \column{.55\textwidth}
      \begin{itemize}
        \item her hücre bir düğüm
        \item aynı satırdaki hücreler bitişik
        \item aynı sütundaki hücreler bitişik
        \item aynı $3 \times 3$ bloktaki hücreler bitişik
        \item her rakam bir renk
      \end{itemize}

      \pause
      \begin{itemize}
        \item problem: kısmen boyalı bir çizgeyi tamamen boya
      \end{itemize}
    \end{columns}
  \end{ornek}
\end{frame}

\section{Etiketli Çizgeler}

\subsection{Giriş}

\begin{frame}
  \frametitle{Etiketli Çizgeler}

  \begin{itemize}
    \item ayrıtlara etiket (ağırlık) atanabilir
    \begin{itemize}
      \item uzunluk, maliyet, olasılık, \dots
    \end{itemize}
  \end{itemize}
\end{frame}

\subsection{En Kısa Yol}

\begin{frame}
  \frametitle{En Kısa Yol Bulma}

  \begin{itemize}
    \item bir düğümden diğerlerine en kısa yolların bulunması
    \begin{itemize}
      \item Dijkstra algoritması
    \end{itemize}
  \end{itemize}
\end{frame}

\begin{frame}
  \frametitle{Dijkstra Algoritması Örneği}

  \begin{ornek}[başlangıç]
    \begin{columns}
      \column{.5\textwidth}
      \begin{center}
        \pgfuseimage{dijkstra}
      \end{center}

      \column{.45\textwidth}
      \begin{itemize}
        \item başlangıç: $c$
      \end{itemize}

      \begin{table}
        \begin{tabular}{r|l}
          a & $(\infty,-)$ \\\hline
          b & $(\infty,-)$ \\\hline
          c & $(0,-)$      \\\hline
          f & $(\infty,-)$ \\\hline
          g & $(\infty,-)$ \\\hline
          h & $(\infty,-)$
        \end{tabular}
      \end{table}
    \end{columns}
  \end{ornek}
\end{frame}

\begin{frame}
  \frametitle{Dijkstra Algoritması Örneği}

  \begin{ornek}[$c$ düğümünden - taban uzaklık=$0$]
    \begin{columns}
      \column{.5\textwidth}
      \begin{center}
        \pgfuseimage{dijkstra}
      \end{center}

      \column{.45\textwidth}
      \begin{itemize}
        \item $c \rightarrow f: 6, 6 < \infty$
        \item $c \rightarrow h: 11, 11 < \infty$
      \end{itemize}

      \pause
      \begin{table}
        \begin{tabular}{r|l|c}
          a & $(\infty,-)$ & \\\hline
          b & $(\infty,-)$ & \\\hline
          c & $(0,-)$      & $\surd$ \\\hline
          f & $(6,cf)$     & \\\hline
          g & $(\infty,-)$ & \\\hline
          h & $(11,ch)$    &
        \end{tabular}
      \end{table}

      \pause
      \begin{itemize}
        \item en yakın düğüm: $f$
      \end{itemize}
    \end{columns}
  \end{ornek}
\end{frame}

\begin{frame}
  \frametitle{Dijkstra Algoritması Örneği}

  \begin{ornek}[$f$ düğümünden - taban uzaklık=$6$]
    \begin{columns}
      \column{.5\textwidth}
      \begin{center}
        \pgfuseimage{dijkstra}
      \end{center}

      \column{.45\textwidth}
      \begin{itemize}
        \item $f \rightarrow a: 6+11, 17 < \infty$
        \item $f \rightarrow g: 6+9, 15 < \infty$
        \item $f \rightarrow h: 6+4, 10 < 11$
      \end{itemize}

      \pause
      \begin{table}
        \begin{tabular}{r|l|c}
          a & $(17,cfa)$   & \\\hline
          b & $(\infty,-)$ & \\\hline
          c & $(0,-)$      & $\surd$ \\\hline
          f & $(6,cf)$     & $\surd$ \\\hline
          g & $(15,cfg)$   & \\\hline
          h & $(10,cfh)$   &
        \end{tabular}
      \end{table}

      \pause
      \begin{itemize}
        \item en yakın düğüm: $h$
      \end{itemize}
    \end{columns}
  \end{ornek}
\end{frame}

\begin{frame}
  \frametitle{Dijkstra Algoritması Örneği}

  \begin{ornek}[$h$ düğümünden - taban uzaklık=$10$]
    \begin{columns}
      \column{.5\textwidth}
      \begin{center}
        \pgfuseimage{dijkstra}
      \end{center}

      \column{.45\textwidth}
      \begin{itemize}
        \item $h \rightarrow a: 10+11, 21 \nless 17$
        \item $h \rightarrow g: 10+4, 14 < 15$
      \end{itemize}

      \pause
      \begin{table}
        \begin{tabular}{r|l|c}
          a & $(17,cfa)$   & \\\hline
          b & $(\infty,-)$ & \\\hline
          c & $(0,-)$      & $\surd$ \\\hline
          f & $(6,cf)$     & $\surd$ \\\hline
          g & $(14,cfhg)$  & \\\hline
          h & $(10,cfh)$   & $\surd$
        \end{tabular}
      \end{table}

      \pause
      \begin{itemize}
        \item en yakın düğüm: $g$
      \end{itemize}
    \end{columns}
  \end{ornek}
\end{frame}

\begin{frame}
  \frametitle{Dijkstra Algoritması Örneği}

  \begin{ornek}[$g$ düğümünden - taban uzaklık=$14$]
    \begin{columns}
      \column{.5\textwidth}
      \begin{center}
        \pgfuseimage{dijkstra}
      \end{center}

      \column{.45\textwidth}
      \begin{itemize}
        \item $g \rightarrow a: 14+17, 31 \nless 17$
      \end{itemize}

      \pause
      \begin{table}
        \begin{tabular}{r|l|c}
          a & $(17,cfa)$   & \\\hline
          b & $(\infty,-)$ & \\\hline
          c & $(0,-)$      & $\surd$ \\\hline
          f & $(6,cf)$     & $\surd$ \\\hline
          g & $(14,cfhg)$  & $\surd$ \\\hline
          h & $(10,cfh)$   & $\surd$
        \end{tabular}
      \end{table}

      \pause
      \begin{itemize}
        \item en yakın düğüm: $a$
      \end{itemize}
    \end{columns}
  \end{ornek}
\end{frame}

\begin{frame}
  \frametitle{Dijkstra Algoritması Örneği}

  \begin{ornek}[$a$ düğümünden - taban uzaklık=$17$]
    \begin{columns}
      \column{.5\textwidth}
      \begin{center}
        \pgfuseimage{dijkstra}
      \end{center}

      \column{.45\textwidth}
      \begin{itemize}
        \item $a \rightarrow b: 17+5, 22 < \infty$
      \end{itemize}

      \pause
      \begin{table}
        \begin{tabular}{r|l|c}
          a & $(17,cfa)$   & $\surd$ \\\hline
          b & $(22,cfab)$  & \\\hline
          c & $(0,-)$      & $\surd$ \\\hline
          f & $(6,cf)$     & $\surd$ \\\hline
          g & $(14,cfhg)$  & $\surd$ \\\hline
          h & $(10,cfh)$   & $\surd$
        \end{tabular}
      \end{table}

      \pause
      \begin{itemize}
        \item son düğüm: $b$
      \end{itemize}
    \end{columns}
  \end{ornek}
\end{frame}

\section*{Kaynaklar}

\begin{frame}
  \frametitle{Kaynaklar}

  \begin{block}{Okunacak: Grimaldi}
    \begin{itemize}
      \item Chapter 11: \alert{An Introduction to Graph Theory}

      \item Chapter 7: Relations: The Second Time Around
      \begin{itemize}
        \item 7.2. \alert{Computer Recognition: Zero-One Matrices\\
                          and Directed Graphs}
      \end{itemize}

      \item Chapter 13: Optimization and Matching
      \begin{itemize}
        \item 13.1. \alert{Dijkstra's Shortest Path Algorithm}
      \end{itemize}
    \end{itemize}
  \end{block}
\end{frame}

\end{document}
