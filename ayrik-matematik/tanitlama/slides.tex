% Copyright (c) 2001-2012
%       H. Turgut Uyar <uyar@itu.edu.tr>
%       Ayşegül Gençata Yayımlı <gencata@itu.edu.tr>
%       Emre Harmancı <harmanci@itu.edu.tr>
%
% Bu notlar "Creative Commons Attribution-NonCommercial-ShareAlike License" ile
% lisanslanmıştır. Yazarının açıkça belirtilmesi koşuluyla ve ticari olmayan
% amaçlarla kullanılabilir ve dağıtılabilir. Bu notlardan yola çıkılarak
% oluşturulacak çalışmaların da aynı lisansa bağlı olmaları gerekir.
%
% Lisans ile ilgili ayrıntılı bilgi almak için şu sayfaya başvurabilirsiniz:
% http://creativecommons.org/licenses/by-nc-sa/3.0/

\documentclass[dvipsnames]{beamer}

\usepackage{ae}
\usepackage[T1]{fontenc}
\usepackage[utf8]{inputenc}
\usepackage[turkish]{babel}
\setbeamertemplate{navigation symbols}{}

\mode<presentation>
{
  \usetheme{Rochester}
  \setbeamercovered{transparent}
}

\title{Ayrık Matematik}
\subtitle{Tanıtlama}

\author{H. Turgut Uyar \and Ayşegül Gençata Yayımlı \and Emre Harmancı}
\date{2001-2012}

\AtBeginSubsection[]
{
  \begin{frame}<beamer>
    \frametitle{Konular}
    \tableofcontents[currentsection,currentsubsection]
  \end{frame}
}

%\beamerdefaultoverlayspecification{<+->}

\theoremstyle{definition}
\newtheorem{tanim}[theorem]{Tanım}

\theoremstyle{example}
\newtheorem{ornek}[theorem]{Örnek}

\theoremstyle{plain}
\newtheorem{teorem}[theorem]{Teorem}

\pgfdeclareimage[width=2cm]{license}{../../license}

\pgfdeclareimage{tumevarim}{tumevarim}
\pgfdeclareimage{hata1}{hata1}
\pgfdeclareimage{hata2}{hata2}

\begin{document}

\begin{frame}
  \titlepage
\end{frame}

\begin{frame}
  \frametitle{Lisans}

  \pgfuseimage{license}\hfill
  \copyright 2001-2012 T. Uyar, A. Yayımlı, E. Harmancı

  \vfill
  \begin{tiny}
    You are free:
    \begin{itemize}
      \item to Share — to copy, distribute and transmit the work
      \item to Remix — to adapt the work
    \end{itemize}

    Under the following conditions:
    \begin{itemize}
      \item Attribution — You must attribute the work in the manner specified by
        the author or licensor (but not in any way that suggests that they
        endorse you or your use of the work).

      \item Noncommercial — You may not use this work for commercial purposes.

      \item Share Alike — If you alter, transform, or build upon this work, you
        may distribute the resulting work only under the same or similar license
        to this one.
    \end{itemize}
  \end{tiny}

  \vfill
  Legal code (the full license):\\
  \url{http://creativecommons.org/licenses/by-nc-sa/3.0/}
\end{frame}

\begin{frame}
  \frametitle{Konular}
  \tableofcontents
\end{frame}

\section{Temel Teknikler}

\subsection{Giriş}

\begin{frame}
  \frametitle{Kaba Kuvvet Yöntemi}

  \begin{itemize}
    \item olası bütün durumları teker teker incelemek
  \end{itemize}

  \pause
  \begin{teorem}
    $\{2,4,6,\dots,26\}$ kümesinden seçilecek her sayı,\\
    en fazla 3~tamkarenin toplamı şeklinde yazılabilir.
  \end{teorem}

  \pause
  \begin{proof}[Tanıt]
    \begin{tabular}{lll}
      2 = 1+1   & 10 = 9+1    & 20 = 16+4\\
      4 = 4     & 12 = 4+4+4  & 22 = 9+9+4\\
      6 = 4+1+1 & 14 = 9+4+1  & 24 = 16+4+4\\
      8 = 4+4   & 16 = 16     & 26 = 25+1\\
                & 18 = 9+9    &
    \end{tabular}\\
  \end{proof}
\end{frame}

\begin{frame}
  \frametitle{Temel Kurallar}

  \begin{block}{Evrensel Özelleştirme (Universal Specification - US)}
    $\forall x~p(x) \Rightarrow p(a)$
  \end{block}

  \pause
  \begin{block}{Evrensel Genelleştirme (Universal Generalization - UG)}
    \alert{rasgele seçilen} bir $a$ için $p(a)$
    $\Rightarrow$ $\forall x~p(x)$
  \end{block}
\end{frame}

\begin{frame}
  \frametitle{Evrensel Özelleştirme Örneği}

  \begin{ornek}
    \begin{quote}
      Bütün insanlar ölümlüdür. Sokrates bir insandır.\\
      O halde Sokrates ölümlüdür.
    \end{quote}

    \pause
    \begin{itemize}
      \item $\mathcal{U}$: bütün insanlar
      \item $p(x)$: $x$ ölümlüdür
      \item $\forall x~p(x)$: Bütün insanlar ölümlüdür.
      \item $a$: Sokrates, $a \in \mathcal{U}$: Sokrates bir insandır.
      \item o halde, $p(a)$: Sokrates ölümlüdür.
    \end{itemize}
  \end{ornek}
\end{frame}

\begin{frame}
  \frametitle{Evrensel Özelleştirme Örneği}

  \begin{ornek}
    \begin{columns}
      \column{.4\textwidth}
      \[
      \frac
        {
          \begin{array}{c}
            \forall x~[j(x) \vee s(x) \rightarrow \neg p(x)]\\
            p(m)
          \end{array}
        }
        {
          \therefore \neg s(m)
        }
      \]

      \pause
      \column{.55\textwidth}
      \begin{eqnarray*}
        1. & \forall x~[j(x) \vee s(x) \rightarrow \neg p(x)] & A\\\pause
        2. & p(m)                                             & A\\\pause
        3. & j(m) \vee s(m) \rightarrow \neg p(m)             & US:1\\\pause
        4. & \neg (j(m) \vee s(m))                            & MT:3,2\\\pause
        5. & \neg j(m) \wedge \neg s(m)                       & DM:4\\\pause
        6. & \neg s(m)                                        & AndE:5
      \end{eqnarray*}
    \end{columns}
  \end{ornek}
\end{frame}

\begin{frame}
  \frametitle{Evrensel Genelleştirme Örneği}

  \begin{ornek}
    \begin{columns}
      \column{.35\textwidth}
      \[
      \frac
        {
          \begin{array}{c}
            \forall x~[p(x) \rightarrow q(x)]\\
            \forall x~[q(x) \rightarrow r(x)]
          \end{array}
        }
        {
          \therefore \forall x~[p(x) \rightarrow r(x)]
        }
      \]

      \pause
      \column{.6\textwidth}
      \begin{eqnarray*}
        1. & \forall x~[p(x) \rightarrow q(x)] & A\\\pause
        2. & p(c) \rightarrow q(c)             & US:1\\\pause
        3. & \forall x~[q(x) \rightarrow r(x)] & A\\\pause
        4. & q(c) \rightarrow r(c)             & US:3\\\pause
        5. & p(c) \rightarrow r(c)             & HS:2,4\\\pause
        6. & \forall x~[p(x) \rightarrow r(x)] & UG:5
      \end{eqnarray*}
    \end{columns}
  \end{ornek}
\end{frame}

\begin{frame}
  \frametitle{Boş Tanıt}

  \begin{block}{boş tanıt}
    $P \Rightarrow Q$ tanıtı için $P$'nin yanlış olduğunu göstermek
  \end{block}
\end{frame}

\begin{frame}
  \frametitle{Boş Tanıt Örneği}

  \begin{teorem}
    $\forall S~[\emptyset \subseteq S]$
  \end{teorem}

  \pause
  \begin{proof}[Tanıt]
    $\emptyset \subseteq S \Leftrightarrow
      \forall x~[x \in \emptyset \rightarrow x \in S]$\\\pause
    $\forall x~[x \notin \emptyset]$
  \end{proof}
\end{frame}

\begin{frame}
  \frametitle{Değersiz Tanıt}

  \begin{block}{değersiz tanıt}
    $P \Rightarrow Q$ tanıtı için $Q$'nun doğru olduğunu göstermek
  \end{block}
\end{frame}

\begin{frame}
  \frametitle{Değersiz Tanıt Örneği}

  \begin{teorem}
    $\forall x \in \mathbb{R}~[x \geq 0 \Rightarrow x^2 \geq 0]$
  \end{teorem}

  \pause
  \begin{proof}[Tanıt]
    $\forall x \in \mathbb{R}~[x^2 \geq 0]$
  \end{proof}
\end{frame}

\subsection{Doğrudan Tanıt}

\begin{frame}
  \frametitle{Doğrudan Tanıt}

  \begin{block}{doğrudan tanıt}
    $P \Rightarrow Q$ tanıtı için $P \vdash Q$ olduğunu göstermek
  \end{block}
\end{frame}

\begin{frame}
  \frametitle{Doğrudan Tanıt Örneği}

  \begin{teorem}
    $\forall a \in \mathbb{Z}~[3 | (a-2) \Rightarrow 3 | (a^2-1)]$
  \end{teorem}

  \pause
  \begin{proof}[Tanıt]
    \begin{eqnarray*}
      3 | (a-2) & \Rightarrow & a-2 = 3k\\\pause
                & \Rightarrow & a+1 = a-2 + 3 = 3k+3 = 3(k+1)\\\pause
                & \Rightarrow & a^2-1 = (a+1)(a-1) = 3(k+1)(a-1)
    \end{eqnarray*}
  \end{proof}
\end{frame}

\begin{frame}
  \frametitle{Dolaylı Tanıt}

  \begin{block}{dolaylı tanıt}
    $P \Rightarrow Q$ tanıtı için $\neg Q \vdash \neg P$ olduğunu göstermek
  \end{block}
\end{frame}

\begin{frame}
  \frametitle{Dolaylı Tanıt Örneği}

  \begin{teorem}
    $\forall x,y \in \mathbb{N}~[x \cdot y > 25
      \Rightarrow (x > 5) \vee (y > 5)]$
  \end{teorem}

  \pause
  \begin{proof}[Tanıt]
    \begin{itemize}
      \item $\neg Q \Leftrightarrow (0 \leq x \leq 5) \wedge (0 \leq y \leq 5)$

      \pause
      \item $0 = 0 \cdot 0 \leq x \cdot y \leq 5 \cdot 5 = 25$
    \end{itemize}
  \end{proof}
\end{frame}

\begin{frame}
  \frametitle{Dolaylı Tanıt Örneği}

  \begin{teorem}
    $(\exists k~a,b,k \in \mathbb{N}~[ab=2k]) \Rightarrow
      (\exists i \in \mathbb{N}~[a=2i]) \vee
      (\exists j \in \mathbb{N}~[b=2j])$
  \end{teorem}

  \pause
  \begin{proof}[Tanıt]
    \begin{itemize}
      \item $\neg Q \Leftrightarrow (\neg \exists i \in \mathbb{N}~[a=2i])
                          \wedge (\neg \exists j \in \mathbb{N}~[b=2j])$
    \end{itemize}

    \pause
    \begin{eqnarray*}
      & \Rightarrow & (\exists x \in \mathbb{N}~[a=2x+1])
               \wedge (\exists y \in \mathbb{N}~[b=2y+1])\\\pause
      & \Rightarrow & ab=(2x+1)(2y+1)\\\pause
      & \Rightarrow & ab=4xy+2(x+y)+1\\\pause
      & \Rightarrow & \neg (\exists a,b,k \in \mathbb{N}~[ab=2k])
    \end{eqnarray*}
  \end{proof}
\end{frame}

\subsection{Çelişkiyle Tanıt}

\begin{frame}
  \frametitle{Çelişkiyle Tanıt}

  \begin{block}{çelişkiyle tanıt}
    $P$ tanıtı için $\neg P \vdash Q \wedge \neg Q$ olduğunu göstermek
  \end{block}
\end{frame}

\begin{frame}
  \frametitle{Çelişkiyle Tanıt Örneği}

  \begin{teorem}
    En büyük asal sayı yoktur.
  \end{teorem}

  \pause
  \begin{proof}[Tanıt]
    \begin{itemize}
      \item $\neg P$: En büyük asal sayı vardır.

      \pause
      \item $Q$: En büyük asal sayı $S$.

      \pause
      \item asal sayılar: $2,3,5,7,11,\dots,S$

      \pause
      \item $2 \cdot 3 \cdot 5 \cdot 7 \cdot 11 \cdots S + 1$ sayısı,\\
        $2..S$ aralığındaki hiçbir asal sayıya kalansız bölünmez

      \pause
      \begin{enumerate}
        \item ya kendisi asaldır: $\neg Q$

        \pause
        \item ya da $S$'den büyük bir asal sayıya bölünür: $\neg Q$
      \end{enumerate}
    \end{itemize}
  \end{proof}
\end{frame}

\begin{frame}
  \frametitle{Çelişkiyle Tanıt Örneği}

  \begin{teorem}
    $\neg \exists a,b \in \mathbb{Z}^+~[\sqrt{2}=\frac{a}{b}]$
  \end{teorem}

  \pause
  \begin{proof}[Tanıt]
    \begin{itemize}
      \item $\neg P$: $\exists a,b \in \mathbb{Z}^+~[\sqrt{2}=\frac{a}{b}]$
      \item $Q$: $obeb(a,b)=1$
    \end{itemize}

    \pause
    \vspace{-0.7cm}
    \begin{columns}[t]
      \column{.5\textwidth}
      \begin{eqnarray*}
        & \Rightarrow & 2 = \frac{a^2}{b^2}\\\pause
        & \Rightarrow & a^2 = 2b^2\\\pause
        & \Rightarrow & \exists i \in \mathbb{Z}^+~[a^2=2i]\\\pause
        & \Rightarrow & \exists j \in \mathbb{Z}^+~[a=2j]
      \end{eqnarray*}

      \pause
      \column{.5\textwidth}
      \begin{eqnarray*}
        & \Rightarrow & 4j^2 = 2b^2\\\pause
        & \Rightarrow & b^2 = 2j^2\\\pause
        & \Rightarrow & \exists k \in \mathbb{Z}^+~[b^2=2k]\\\pause
        & \Rightarrow & \exists l \in \mathbb{Z}^+~[b=2l]\\\pause
        & \Rightarrow & obeb(a,b) \geq 2: \neg Q
      \end{eqnarray*}
    \end{columns}
  \end{proof}
\end{frame}

\subsection{Eşdeğerlilik Tanıtları}

\begin{frame}
  \frametitle{Eşdeğerlilik Tanıtları}

  \begin{itemize}
    \item $P \Leftrightarrow Q$ tanıtı için hem $P \Rightarrow Q$, hem de
      $Q \Rightarrow P$ tanıtlanmalı

    \pause
    \medskip
    \item $P_1 \Leftrightarrow P_2 \Leftrightarrow \cdots \Leftrightarrow P_n$
      tanıtı için bir yöntem:\\
      $P_1 \Rightarrow P_2 \Rightarrow \cdots \Rightarrow P_n \Rightarrow P_1$
  \end{itemize}
\end{frame}

\begin{frame}
  \frametitle{Eşdeğerlilik Tanıtı Örneği}

  \begin{teorem}
    $a,b,n,q_1,r_1,q_2,r_2 \in \mathbb{Z}^+$\\
    $a = q_1 \cdot n + r_1$\\
    $b = q_2 \cdot n + r_2$\\

    \bigskip
    $r_1 = r_2 \Leftrightarrow n | (a - b)$
  \end{teorem}
\end{frame}

\begin{frame}
  \frametitle{Eşdeğerlilik Tanıtı Örneği}

  \begin{columns}[t]
    \column{.55\textwidth}
    \begin{proof}[$r_1 = r_2 \Rightarrow n | (a - b)$]
      \begin{eqnarray*}
        a - b & = & (q_1 \cdot n + r_1)\\
              &   & -(q_2 \cdot n + r_2)\\\pause
              & = & (q_1 - q_2) \cdot n\\
              &   & + (r_1 - r_2)\\\pause
        r_1 = r_2 & \Rightarrow & r_1 - r_2 = 0\\\pause
                  & \Rightarrow & a - b = (q_1 - q_2) \cdot n
      \end{eqnarray*}
    \end{proof}

    \pause
    \column{.45\textwidth}
    \begin{proof}[$n | (a - b) \Rightarrow r_1 = r_2$]
      \begin{eqnarray*}
        a - b & = & (q_1 \cdot n + r_1)\\
              &   & -(q_2 \cdot n + r_2)\\\pause
              & = & (q_1 - q_2) \cdot n\\
              &   & + (r_1 - r_2)\\\pause
        n | (a - b) & \Rightarrow & r_1 - r_2 = 0\\\pause
                    & \Rightarrow & r_1 = r_2
      \end{eqnarray*}
    \end{proof}
  \end{columns}
\end{frame}

\begin{frame}
  \frametitle{Eşdeğerlilik Tanıtı Örneği}

  \begin{teorem}
    \begin{eqnarray*}
      &                 & A \subseteq B\\
      & \Leftrightarrow & A \cup B = B\\
      & \Leftrightarrow & A \cap B = A\\
      & \Leftrightarrow & \overline{B} \subseteq \overline{A}
    \end{eqnarray*}
  \end{teorem}
\end{frame}

\begin{frame}
  \frametitle{Eşdeğerlilik Tanıtı Örneği}

  \begin{proof}[$A \subseteq B \Rightarrow A \cup B = B$]
    $A \cup B = B \Leftrightarrow
      A \cup B \subseteq B \wedge B \subseteq A \cup B$

    \pause
    \bigskip
    \begin{columns}
      \column{.4\textwidth}
      $B \subseteq A \cup B$

      \pause
      \medskip
      \column{.5\textwidth}
      \begin{eqnarray*}
        x \in A \cup B & \Rightarrow & x \in A \vee x \in B\\\pause
        A \subseteq B  & \Rightarrow & x \in B\\\pause
                       & \Rightarrow & A \cup B \subseteq B
      \end{eqnarray*}
    \end{columns}
  \end{proof}
\end{frame}

\begin{frame}
  \frametitle{Eşdeğerlilik Tanıtı Örneği}

  \begin{proof}[$A \cup B = B \Rightarrow A \cap B = A$]
    $A \cap B = A \Leftrightarrow
      A \cap B \subseteq A \wedge A \subseteq A \cap B$

    \pause
    \bigskip
    \begin{columns}
      \column{.4\textwidth}
      $A \cap B \subseteq A$

      \pause
      \medskip
      \column{.5\textwidth}
      \begin{eqnarray*}
        y \in A      & \Rightarrow & y \in A \cup B\\\pause
        A \cup B = B & \Rightarrow & y \in B\\\pause
                     & \Rightarrow & y \in A \cap B\\\pause
                     & \Rightarrow & A \subseteq A \cap B
      \end{eqnarray*}
    \end{columns}
  \end{proof}
\end{frame}

\begin{frame}
  \frametitle{Eşdeğerlilik Tanıtı Örneği}

  \begin{proof}[$A \cap B = A \Rightarrow \overline{B} \subseteq \overline{A}$]
    \begin{eqnarray*}
      z \in \overline{B} & \Rightarrow & z \notin B\\\pause
                         & \Rightarrow & z \notin A \cap B\\\pause
      A \cap B = A       & \Rightarrow & z \notin A\\\pause
                         & \Rightarrow & z \in \overline{A}\\\pause
                         & \Rightarrow & \overline{B} \subseteq \overline{A}
    \end{eqnarray*}
  \end{proof}
\end{frame}

\begin{frame}
  \frametitle{Eşdeğerlilik Tanıtı Örneği}

  \begin{proof}[$\overline{B} \subseteq \overline{A} \Rightarrow A \subseteq B$]
    \begin{eqnarray*}
      \neg(A \subseteq B)
        & \Rightarrow & \exists w~[w \in A \wedge w \notin B]\\\pause
        & \Rightarrow & \exists w~[w \notin \overline{A} \wedge w \in \overline{B}]\\\pause
        & \Rightarrow & \neg(\overline{B} \subseteq \overline{A})
    \end{eqnarray*}
  \end{proof}
\end{frame}

\section{Tümevarım}

\subsection{Giriş}

\begin{frame}
  \frametitle{Tümevarım}

  \begin{tanim}
    $S(n)$: $n \in \mathbb{Z}^+$ üzerinde tanımlanan bir yüklem

    \pause
    \medskip
    $S(n_0) \wedge (\forall k \geq n_0~[S(k) \Rightarrow S(k+1)])
      \Rightarrow \forall n \geq n_0~S(n)$
  \end{tanim}

  \pause
  \medskip
  \begin{itemize}
    \item $S(n_0)$: \emph{taban adımı}
    \item $\forall k \geq n_0~[S(k) \Rightarrow S(k+1)]$: \emph{tümevarım adımı}
  \end{itemize}
\end{frame}

\begin{frame}
  \frametitle{Tümevarım}

  \begin{center}
    \pgfuseimage{tumevarim}
  \end{center}
\end{frame}

\begin{frame}
  \frametitle{Tümevarım Örneği}

  \begin{teorem}
    $\forall n \in \mathbb{Z}^+~[1+3+5+\cdots+(2n-1)=n^2]$
  \end{teorem}

  \pause
  \begin{proof}[Tanıt]
    \begin{itemize}
      \item $n=1$: $1=1^2$

      \pause
      \item $n=k$: $1+3+5+\cdots+(2k-1)=k^2$ kabul edelim

      \pause
      \item $n=k+1$:
      \begin{eqnarray*}
        &   & 1+3+5+\cdots+(2k-1)+(2k+1)\\\pause
        & = & k^2+2k+1\\\pause
        & = & (k+1)^2
      \end{eqnarray*}
    \end{itemize}
  \end{proof}
\end{frame}

\begin{frame}
  \frametitle{Tümevarım Örneği}

  \begin{teorem}
    $\forall n \in \mathbb{Z}^+, n \geq 4~[2^n < n!]$
  \end{teorem}

  \pause
  \begin{proof}[Tanıt]
    \begin{itemize}
      \item $n=4$: $2^4=16<24=4!$

      \pause
      \item $n=k$: $2^k < k!$ kabul edelim

      \pause
      \item $n=k+1$:\\
        $2^{k+1} = 2 \cdot 2^k < 2 \cdot k! < (k+1) \cdot k! = (k+1)!$
    \end{itemize}
  \end{proof}
\end{frame}

\begin{frame}
  \frametitle{Tümevarım Örneği}

  \begin{teorem}
    $\forall n \in \mathbb{Z}^+, n \geq 14~\exists i,j \in \mathbb{N}~[n=3i+8j]$
  \end{teorem}

  \pause
  \begin{proof}[Tanıt]
    \begin{itemize}
      \item $n=14$: $14=3 \cdot 2 + 8 \cdot 1$

      \pause
      \item $n=k$: $k=3i+8j$ kabul edelim

      \pause
      \item $n=k+1$:
      \begin{itemize}
        \item $k=3i+8j, j>0 \Rightarrow k+1=k-8+3 \cdot 3$\\
          $\Rightarrow k+1=3(i+3)+8(j-1)$
        \item $k=3i+8j, j=0, i \geq 5 \Rightarrow k+1=k-5 \cdot 3+2 \cdot 8$\\
          $\Rightarrow k+1=3(i-5)+8(j+2)$
      \end{itemize}
    \end{itemize}
  \end{proof}
\end{frame}

\subsection{Güçlü Tümevarım}

\begin{frame}
  \frametitle{Güçlü Tümevarım}

  \begin{tanim}
    $S(n_0) \wedge
      (\forall k \geq n_0~[(\forall i \leq k~S(i)) \Rightarrow S(k+1)])
      \Rightarrow \forall n \geq n_0~S(n)$
  \end{tanim}
\end{frame}

\begin{frame}
  \frametitle{Güçlü Tümevarım Örneği}

  \begin{teorem}
    $\forall n \in \mathbb{Z}^+, n \geq 2$\\
      n asal sayıların çarpımı şeklinde yazılabilir
  \end{teorem}

  \pause
  \begin{proof}[Tanıt]
    \begin{itemize}
      \item $n=2$: $2=2$

      \pause
      \item $\forall i \leq k$ için doğru kabul edelim

      \pause
      \item $n=k+1$:
      \begin{enumerate}
        \item asalsa: $n=n$

        \pause
        \item asal değilse: $n=u \cdot v$\\
          $u < k \wedge v < k \Rightarrow$ $u$ ve $v$ sayılarının her biri\\
          asal sayıların çarpımı şeklinde yazılabilir
      \end{enumerate}
    \end{itemize}
  \end{proof}
\end{frame}

\begin{frame}
  \frametitle{Güçlü Tümevarım Örneği}

  \begin{teorem}
    $\forall n \in \mathbb{Z}^+, n \geq 14~\exists i,j \in \mathbb{N}~[n=3i+8j]$
  \end{teorem}

  \pause
  \begin{proof}[Tanıt]
    \begin{itemize}
      \item $n=14$: $14=3 \cdot 2 + 8 \cdot 1$\\
        $n=15$: $15=3 \cdot 5 + 8 \cdot 0$\\
        $n=16$: $16=3 \cdot 0 + 8 \cdot 2$

      \pause
      \item $n \leq k$: $k=3i+8j$ kabul edelim

      \pause
      \item $n=k+1$: $k+1=(k-2)+3$
    \end{itemize}
  \end{proof}
\end{frame}

\begin{frame}
  \frametitle{Hatalı Tümevarım Örneği}

  \begin{teorem}
    $\forall n \in \mathbb{Z}^+~[1+2+3+\cdots+n=\frac{n^2+n+2}{2}]$
  \end{teorem}

  \pause
  \begin{block}{taban adımı geçersiz}
    \begin{itemize}
      \item $n=k$: $1+2+3+\cdots+k=\frac{k^2+k+2}{2}$ kabul edelim

      \pause
      \item $n=k+1$:
      \begin{eqnarray*}
        &   & 1+2+3+\cdots+k+(k+1)\\\pause
        & = & \frac{k^2+k+2}{2}+k+1
          =   \frac{k^2+k+2}{2}+\frac{2k+2}{2}\\\pause
        & = & \frac{k^2+3k+4}{2}
          =   \frac{(k+1)^2+(k+1)+2}{2}
      \end{eqnarray*}

      \pause
      \item $n=1$: $1 \neq \frac{1^2+1+2}{2}=2$
    \end{itemize}
  \end{block}
\end{frame}

\begin{frame}
  \frametitle{Hatalı Tümevarım Örnekleri}

  \begin{center}
    \pgfuseimage{hata1}
  \end{center}
\end{frame}

\begin{frame}
  \frametitle{Hatalı Tümevarım Örnekleri}

  \begin{teorem}
    Bütün atlar aynı renktir.

    \pause
    \bigskip
    $A(n)$: $n$ atlı kümelerdeki bütün atlar aynı renktir.

    \medskip
    $\forall n \in \mathbb{N^+}~A(n)$
  \end{teorem}
\end{frame}

\begin{frame}
  \frametitle{Hatalı Tümevarım Örnekleri}

  \begin{block}{$n$ üzerinden hatalı tümevarım}
    \begin{itemize}
      \item $n=1$: $A(1)$\\
        $1$~atlı kümelerdeki bütün atlar aynı renktir.

      \pause
      \medskip
      \item $n=k$: $A(k)$ doğru kabul edelim\\
        $k$~atlı kümelerdeki bütün atlar aynı renktir.

      \pause
      \medskip
      \item $A(k+1)=\{a_1,a_2,\dots,a_k\} \cup \{a_2,a_3,\dots,a_{k+1}\}$
      \begin{itemize}
        \item $\{a_1,a_2,\dots,a_k\}$ kümesindeki bütün atlar aynı renk ($a_2$)
        \item $\{a_2,a_3,\dots,a_{k+1}\}$ kümesindeki bütün atlar aynı renk
          ($a_2$)
      \end{itemize}
    \end{itemize}
  \end{block}
\end{frame}

\begin{frame}
  \frametitle{Hatalı Tümevarım Örnekleri}

  \begin{center}
    \pgfuseimage{hata2}
  \end{center}
\end{frame}

\section*{Kaynaklar}

\begin{frame}
  \frametitle{Kaynaklar}

  \begin{block}{Okunacak: Grimaldi}
    \begin{itemize}
      \item Chapter 2: Fundamentals of Logic
      \begin{itemize}
        \item 2.5. \alert{Quantifiers, Definitions, and the Proofs of Theorems}
      \end{itemize}

      \item Chapter 4: Properties of Integers: Mathematical Induction
      \begin{itemize}
        \item 4.1. \alert{The Well-Ordering Principle: Mathematical Induction}
      \end{itemize}
    \end{itemize}
  \end{block}

  \begin{block}{Yardımcı Kitap: O'Donnell, Hall, Page}
    \begin{itemize}
      \item Chapter 4: Induction
    \end{itemize}
  \end{block}
\end{frame}

\end{document}
