% Copyright (c) 2001-2011
%       H. Turgut Uyar <uyar@itu.edu.tr>
%       Ayşegül Gençata Yayımlı <gencata@itu.edu.tr>
%       Emre Harmancı <harmanci@itu.edu.tr>
%
% Bu notlar "Creative Commons Attribution-NonCommercial-ShareAlike License" ile
% lisanslanmıştır. Yazarının açıkça belirtilmesi koşuluyla ve ticari olmayan
% amaçlarla kullanılabilir ve dağıtılabilir. Bu notlardan yola çıkılarak
% oluşturulacak çalışmaların da aynı lisansa bağlı olmaları gerekir.
%
% Lisans ile ilgili ayrıntılı bilgi almak için şu sayfaya başvurabilirsiniz:
% http://creativecommons.org/licenses/by-nc-sa/3.0/

\documentclass[dvipsnames]{beamer}

\usepackage{ae}
\usepackage[T1]{fontenc}
\usepackage[utf8]{inputenc}
\usepackage[turkish]{babel}
\setbeamertemplate{navigation symbols}{}

\mode<presentation>
{
  \usetheme{Rochester}
  \setbeamercovered{transparent}
}

\title{Ayrık Matematik}
\subtitle{Yüklemler ve Kümeler}

\author{H. Turgut Uyar \and Ayşegül Gençata Yayımlı \and Emre Harmancı}
\date{2001-2011}

\AtBeginSubsection[]
{
  \begin{frame}<beamer>
    \frametitle{Konular}
    \tableofcontents[currentsection,currentsubsection]
  \end{frame}
}

%\beamerdefaultoverlayspecification{<+->}

\theoremstyle{definition}
\newtheorem{tanim}[theorem]{Tanım}

\theoremstyle{example}
\newtheorem{ornek}[theorem]{Örnek}

\theoremstyle{plain}
\newtheorem{teorem}[theorem]{Teorem}

\pgfdeclareimage[width=2cm]{license}{../../license}

\begin{document}

\begin{frame}
  \titlepage
\end{frame}

\begin{frame}
  \frametitle{Lisans}

  \pgfuseimage{license}\hfill
  \copyright 2001-2011 T. Uyar, A. Yayımlı, E. Harmancı

  \vfill
  \begin{tiny}
    You are free:
    \begin{itemize}
      \item to Share — to copy, distribute and transmit the work
      \item to Remix — to adapt the work
    \end{itemize}

    Under the following conditions:
    \begin{itemize}
      \item Attribution — You must attribute the work in the manner specified by
        the author or licensor (but not in any way that suggests that they
        endorse you or your use of the work).

      \item Noncommercial — You may not use this work for commercial purposes.

      \item Share Alike — If you alter, transform, or build upon this work, you
        may distribute the resulting work only under the same or similar license
        to this one.
    \end{itemize}
  \end{tiny}

  \vfill
  Legal code (the full license):\\
  \url{http://creativecommons.org/licenses/by-nc-sa/3.0/}
\end{frame}

\begin{frame}
  \frametitle{Konular}
  \tableofcontents
\end{frame}

\section{Yüklemler}

\subsection{Giriş}

\begin{frame}
  \frametitle{Yüklem}

  \begin{tanim}
    \alert{yüklem}:

    \begin{itemize}
      \item bir ya da birden fazla değişken içeren ve
      \item bir önerme olmayan ama
      \item değişkenlere izin verilen seçenekler arasından değer verildiğinde
        önerme haline gelen
    \end{itemize}

    bir bildirim (\emph{açık bildirim})
  \end{tanim}
\end{frame}

\begin{frame}
  \frametitle{Çalışma Evreni}

  \begin{tanim}
    \alert{çalışma evreni}: $\mathcal{U}$\\
    izin verilen seçenekler kümesi
  \end{tanim}

  \pause
  \begin{itemize}
    \item örnek çalışma evrenleri:
    \begin{itemize}
      \item $\mathbb{Z}$: tamsayılar
      \item $\mathbb{N}$: doğal sayılar
      \item $\mathbb{Z}^+$: pozitif tamsayılar
      \item $\mathbb{Q}$: rasyonel sayılar
      \item $\mathbb{R}$: reel sayılar
      \item $\mathbb{C}$: karmaşık sayılar
    \end{itemize}
  \end{itemize}
\end{frame}

\begin{frame}
  \frametitle{Yüklem Örnekleri}

  \begin{ornek}
    $\mathcal{U} = \mathbb{N}$\\
    $p(x)$: $x+2$ bir çift sayıdır

    \bigskip
    $p(5)$: $Y$\\
    $p(8)$: $D$

    \pause
    \bigskip
    $\neg p(x)$: $x+2$ bir çift sayı değildir
  \end{ornek}

  \pause
  \begin{ornek}
    $\mathcal{U} = \mathbb{N}$\\
    $q(x,y)$: $x+y$ ve $x-2y$ birer çift sayıdır

    \bigskip
    $q(11,3)$: $Y$, $q(14,4)$: $D$
  \end{ornek}
\end{frame}

\subsection{Niceleyiciler}

\begin{frame}
  \frametitle{Niceleyiciler}

  \begin{columns}[t]
    \column{.48\textwidth}
    \begin{tanim}
      \alert{varlık niceleyicisi}:\\
        yüklem bazı değerler için doğru

      \begin{itemize}
        \item simgesi: $\exists$
        \item okunuşu: \emph{vardır}

        \pause
        \medskip
        \item simge: $\exists!$
        \item okunuşu: \emph{vardır ve tektir}
      \end{itemize}
    \end{tanim}

    \pause
    \column{.48\textwidth}
    \begin{tanim}
      \alert{evrensel niceleyici}:\\
        yüklem bütün değerler için doğru

      \begin{itemize}
        \item simgesi: $\forall$
        \item okunuşu: \emph{her}
      \end{itemize}
    \end{tanim}
  \end{columns}
\end{frame}

\begin{frame}
  \frametitle{Niceleyiciler}

  \begin{block}{varlık niceleyicisi}
    $\mathcal{U} = \{x_1,x_2,\cdots,x_n\}$\\
    $\exists x~p(x) \equiv p(x_1) \vee p(x_2) \vee \cdots \vee p(x_n)$

    \begin{itemize}
      \item \emph{bazı $x$'ler için $p(x)$ doğru}
    \end{itemize}
  \end{block}

  \pause
  \begin{block}{evrensel niceleyici}
    $\mathcal{U} = \{x_1,x_2,\cdots,x_n\}$\\
    $\forall x~p(x) \equiv p(x_1) \wedge p(x_2) \wedge \cdots \wedge p(x_n)$

    \begin{itemize}
      \item \emph{her $x$ için $p(x)$ doğru}
    \end{itemize}
  \end{block}
\end{frame}

\begin{frame}
  \frametitle{Niceleyici Örnekleri}

  \begin{ornek}
    \begin{columns}[t]
      \column{.5\textwidth}
      $\mathcal{U} = \mathbb{R}$\\

      \begin{itemize}
        \item $p(x): x \geq 0$
        \item $q(x): x^2 \geq 0$
        \item $r(x): (x-4) (x+1) = 0$
        \item $s(x): x^2 -3 > 0$
      \end{itemize}

      şeklinde tanımlandıysa yandaki ifadelerin sonuçları ne olur?

      \column{.4\textwidth}
      \begin{itemize}
        \pause
        \item $\exists x~[p(x) \wedge r(x)]$

        \pause
        \item $\forall x~[p(x) \rightarrow q(x)]$

        \pause
        \item $\forall x~[q(x) \rightarrow s(x)]$

        \pause
        \item $\forall x~[r(x) \vee s(x)]$

        \pause
        \item $\forall x~[r(x) \rightarrow p(x)]$
      \end{itemize}
    \end{columns}
  \end{ornek}
\end{frame}

\begin{frame}
  \frametitle{Niceleyicilerin Değillenmesi}

  \begin{itemize}
    \item $\forall$ yerine $\exists$, $\exists$ yerine $\forall$ konur
    \item yüklem değillenir
  \end{itemize}

  \pause
  \begin{eqnarray*}
    \neg \exists x~p(x)      & \Leftrightarrow & \forall x~\neg p(x)\\
    \neg \exists x~\neg p(x) & \Leftrightarrow & \forall x~p(x)\\
    \neg \forall x~p(x)      & \Leftrightarrow & \exists x~\neg p(x)\\
    \neg \forall x~\neg p(x) & \Leftrightarrow & \exists x~p(x)
  \end{eqnarray*}
\end{frame}

\begin{frame}
  \frametitle{Niceleyicilerin Değillenmesi}

  \begin{teorem}
    $\neg \exists x~p(x) \Leftrightarrow \forall x~\neg p(x)$
  \end{teorem}

  \pause
  \begin{proof}[Tanıt]
    \begin{eqnarray*}
      \neg \exists x~p(x) & \equiv          & \neg [p(x_1) \vee p(x_2) \vee \cdots
                                              \vee p(x_n)]\\\pause
                          & \Leftrightarrow & \neg p(x_1) \wedge \neg p(x_2) \wedge \cdots
                                              \wedge \neg p(x_n)\\\pause
                          & \equiv          & \forall x~\neg p(x)
    \end{eqnarray*}
  \end{proof}
\end{frame}

\begin{frame}
  \frametitle{Niceleyici Eşdeğerlilikleri}

  \begin{teorem}
    $\exists x~[p(x) \vee q(x)]
      \Leftrightarrow \exists x~p(x) \vee \exists x~q(x)$
  \end{teorem}

  \pause
  \begin{teorem}
    $\forall x~[p(x) \wedge q(x)]
      \Leftrightarrow \forall x~p(x) \wedge \forall x~q(x)$
  \end{teorem}
\end{frame}

\begin{frame}
  \frametitle{Niceleyici Gerektirmeleri}

  \begin{teorem}
    $\forall x~p(x) \Rightarrow \exists x~p(x)$
  \end{teorem}

  \pause
  \begin{teorem}
    $\exists x~[p(x) \wedge q(x)]
      \Rightarrow \exists x~p(x) \wedge \exists x~q(x)$
  \end{teorem}

  \pause
  \begin{teorem}
    $\forall x~p(x) \vee \forall x~q(x)
      \Rightarrow \forall x~[p(x) \vee q(x)]$
  \end{teorem}
\end{frame}

\subsection{Çoklu Niceleyiciler}

\begin{frame}
  \frametitle{Çoklu Niceleyiciler}

  \begin{itemize}
    \item $\exists x \exists y~p(x,y)$
    \item $\forall x \exists y~p(x,y)$
    \item $\exists x \forall y~p(x,y)$
    \item $\forall x \forall y~p(x,y)$
  \end{itemize}
\end{frame}

\begin{frame}
  \frametitle{Çoklu Niceleyici Örnekleri}

  \begin{ornek}
    $\mathcal{U}=\mathbb{Z}$\\
    $p(x,y): x+y=17$

    \begin{itemize}
      \pause
      \item $\forall x \exists y~p(x,y)$:\\
        her $x$ için öyle bir $y$ bulunabilir ki $x+y=17$ olur

      \pause
      \item $\exists y \forall x~p(x,y)$:\\
        öyle bir $y$ bulunabilir ki her $x$ için $x+y=17$ olur

      \pause
      \bigskip
      \item $\mathcal{U}=\mathbb{N}$ olsa?
    \end{itemize}
  \end{ornek}
\end{frame}

\begin{frame}
  \frametitle{Çoklu Niceleyiciler}

  \begin{ornek}
    $\mathcal{U}_x = \{1,2\} \wedge \mathcal{U}_y = \{A,B\}$

    \pause
    \begin{eqnarray*}
      \exists x \exists y~p(x,y) & \equiv & [p(1,A) \vee p(1,B)]
                                       \vee [p(2,A) \vee p(2,B)]\\\pause
      \exists x \forall y~p(x,y) & \equiv & [p(1,A) \wedge p(1,B)]
                                       \vee [p(2,A) \wedge p(2,B)]\\\pause
      \forall x \exists y~p(x,y) & \equiv & [p(1,A) \vee p(1,B)]
                                     \wedge [p(2,A) \vee p(2,B)]\\\pause
      \forall x \forall y~p(x,y) & \equiv & [p(1,A) \wedge p(1,B)]
                                     \wedge [p(2,A) \wedge p(2,B)]
    \end{eqnarray*}
  \end{ornek}
\end{frame}

\subsection*{Kaynaklar}

\begin{frame}
  \frametitle{Kaynaklar}

  \begin{block}{Okunacak: Grimaldi}
    \begin{itemize}
      \item Chapter 2: Fundamentals of Logic
      \begin{itemize}
        \item 2.4. \alert{The Use of Quantifiers}
      \end{itemize}
    \end{itemize}
  \end{block}

  \begin{block}{Yardımcı Kitap: O'Donnell, Hall, Page}
    \begin{itemize}
      \item Chapter 7: Predicate Logic
    \end{itemize}
  \end{block}
\end{frame}

\section{Kümeler}

\subsection{Giriş}

\begin{frame}
  \frametitle{Küme}

  \begin{tanim}
    \alert{küme}:

    \begin{itemize}
      \item birbirinden ayırt edilebilen
      \item aralarında sıralama yapılmamış
      \item yinelenmeyen
    \end{itemize}

    elemanlar topluluğu
  \end{tanim}
\end{frame}

\begin{frame}
  \frametitle{Küme Gösterilimi}

  \begin{itemize}
    \item \emph{açık gösterilim}\\
      elemanlar süslü parantezler içinde listelenir: $\{a_1,a_2,\dots,a_n\}$

    \pause
    \medskip
    \item \emph{kapalı gösterilim}\\
      bir yüklemi doğru kılan elemanlar: $\{x | x \in G, p(x)\}$

    \pause
    \medskip
    \item $\emptyset$: boş küme

    \pause
    \medskip
    \item $S$ bir küme, $a$ bir nesne ise:
    \begin{itemize}
      \item $a \in S$: $a$ nesnesi $S$ kümesinin bir elemanıdır
      \item $a \notin S$: $a$ nesnesi $S$ kümesinin bir elemanı değildir
    \end{itemize}
  \end{itemize}
\end{frame}

\begin{frame}
  \frametitle{Açık Gösterilim Örnekleri}

  \begin{ornek}
    $\{3,8,2,11,5\}$\\
    $11 \in \{3,8,2,11,5\}$
  \end{ornek}
\end{frame}

\begin{frame}
  \frametitle{Kapalı Gösterilim Örnekleri}

  \begin{ornek}
    $\{ x | x \in \mathbb{Z}^+, 20 < x^3 < 100 \} \equiv \{3,4\}$

    $\{ 2x-1 | x \in \mathbb{Z}^+, 20 < x^3 < 100 \} \equiv \{5,7\}$
  \end{ornek}

  \pause
  \begin{ornek}
    $A = \{ x | x \in \mathbb{R}, 1 \leq x \leq 5 \}$
  \end{ornek}

  \pause
  \begin{ornek}
    $E = \{ n | n \in \mathbb{N}, \exists k \in \mathbb{N}~[n=2k] \}$\\
    $A = \{ x | x \in E, 1 \leq x \leq 5 \}$
  \end{ornek}
\end{frame}

\begin{frame}
  \frametitle{Küme İkilemi}

  \begin{itemize}
    \item bir köyde bir berber kendini traş etmeyen herkesi traş ediyor\\
      kendisini traş edenleri traş etmiyor

    \begin{quote}
      bu berber kendisini traş eder mi?
    \end{quote}

    \pause
    \item \emph{etmez}: kendisini traş etmeyen herkesi traş ediyor
      $\rightarrow$ eder

    \pause
    \item \emph{eder}: kendisini traş edenleri traş etmiyor
      $\rightarrow$ etmez
  \end{itemize}
\end{frame}

\begin{frame}
  \frametitle{Küme İkilemi}

  \begin{itemize}
    \item $S$ bir kümeler kümesi

    \pause
    \item kendisinin elemanı olmayan kümeler kümesi:\\
      $S = \{ A | A \notin A \}$

    \pause
    \begin{quote}
      $S$ kendisinin elemanı mıdır?
    \end{quote}

    \pause
    \item \emph{evet}: yüklemi sağlamaz $\rightarrow$ hayır

    \pause
    \item \emph{hayır}: yüklemi sağlar $\rightarrow$ evet
  \end{itemize}
\end{frame}

\begin{frame}
  \frametitle{Sonlu Küme}

  \begin{tanim}
    \alert{sayılabilen küme}:\\
      elemanları numaralandırılabilen küme

    \begin{itemize}
      \item $\mathbb{R}$ kümesi sayılamaz
    \end{itemize}
  \end{tanim}

  \pause
  \begin{tanim}
    \alert{sonlu küme}:\\
      sayılabilen ve eleman sayısı sonlu olan küme

    \begin{itemize}
      \item $\mathbb{N}$ kümesi sayılabilir ama sonlu değildir
      \item eleman sayısı: \alert{kardinalite}, gösterilim: $|S|$
    \end{itemize}
  \end{tanim}
\end{frame}

\subsection{Altküme}

\begin{frame}
  \frametitle{Altküme}

  \begin{tanim}
    $A \subseteq B \Leftrightarrow \forall x~[x \in A \rightarrow x \in B]$
  \end{tanim}

  \pause
  \begin{itemize}
    \item \alert{küme eşitliği}:\\
      $A = B \Leftrightarrow (A \subseteq B) \wedge (B \subseteq A)$

    \pause
    \item \alert{uygun altküme}:\\
      $A \subset B \Leftrightarrow (A \subseteq B) \wedge (A \neq B)$

    \pause
    \item $\forall S~[\emptyset \subseteq S]$
  \end{itemize}
\end{frame}

\begin{frame}
  \frametitle{Altküme}

  \begin{block}{altküme değil}
    \begin{eqnarray*}
      A \nsubseteq B & \Leftrightarrow
                     & \neg \forall x~[x \in A \rightarrow x \in B]\\\pause
                     & \Leftrightarrow
                     & \exists x~\neg [x \in A \rightarrow x \in B]\\\pause
                     & \Leftrightarrow
                     & \exists x~\neg [\neg (x \in A) \vee (x \in B)]\\\pause
                     & \Leftrightarrow
                     & \exists x~[(x \in A) \wedge \neg (x \in B)]\\\pause
                     & \Leftrightarrow
                     & \exists x~[(x \in A) \wedge (x \notin B)]
    \end{eqnarray*}
  \end{block}
\end{frame}

\begin{frame}
  \frametitle{Altkümeler Kümesi}

  \begin{tanim}
    \alert{altkümeler kümesi}:\\
      bir kümenin, boş küme ve kendisi dahil, bütün altkümelerinin oluşturduğu
      küme

    \begin{itemize}
      \item gösterilimi: $\mathcal{P}(S)$
    \end{itemize}
  \end{tanim}

  \pause
  \begin{itemize}
    \item $n$ elemanlı bir kümenin altkümeler kümesinin $2^n$ elemanı vardır
  \end{itemize}
\end{frame}

\begin{frame}
  \frametitle{Altkümeler Kümesi Örneği}

  \begin{ornek}
    \begin{eqnarray*}
      \mathcal{P}(\{1,2,3\}) & = \{ &\\
                             &      & \emptyset\\
                             &      & \{1\},\{2\},\{3\}\\
                             &      & \{1,2\},\{1,3\},\{2,3\}\\
                             &      & \{1,2,3\}\\
                             &   \} &
    \end{eqnarray*}
  \end{ornek}
\end{frame}

\subsection{Küme İşlemleri}

\begin{frame}
  \frametitle{Küme İşlemleri}

  \begin{block}{tümleme}
    $\overline{A} = \{ x | x \notin A \} $
  \end{block}

  \pause
  \begin{block}{kesişim}
    $A \cap B = \{ x | (x \in A) \wedge (x \in B) \}$

    \pause
    \begin{itemize}
      \item $A \cap B = \emptyset$ ise $A$ ile $B$ \alert{ayrık kümeler}
    \end{itemize}
  \end{block}

  \pause
  \begin{block}{birleşim}
    $A \cup B = \{ x | (x \in A) \vee (x \in B) \}$
  \end{block}
\end{frame}

\begin{frame}
  \frametitle{Küme İşlemleri}

  \begin{block}{fark}
    $A - B = \{ x | (x \in A) \wedge (x \notin B) \}$

    \pause
    \begin{itemize}
      \item $A-B = A \cap \overline{B}$

      \pause
      \item \emph{bakışımlı fark}:\\
        $A \bigtriangleup B = \{ x | (x \in A \cup B)
                              \wedge (x \notin A \cap B) \}$
    \end{itemize}
  \end{block}
\end{frame}

\begin{frame}
  \frametitle{Kartezyen Çarpım}

  \begin{tanim}
    \alert{kartezyen çarpım}:\\
      $A \times B = \{ (a,b) | a \in A, b \in B \}$

      \pause
      \medskip
      $A \times B \times C \dots \times N =
        \{ (a,b,\dots,n) | a \in A, b \in B, \dots, n \in N \}$
  \end{tanim}
\end{frame}

\begin{frame}
  \frametitle{Kartezyen Çarpım Örneği}

  \begin{ornek}
    $A = \{a_1.a_2,a_3,a_4\}$

    $B = \{b_1,b_2,b_3\}$

    \medskip
    \begin{eqnarray*}
      A \times B & = \{ & \\
                 &      & (a_1,b_1),(a_1,b_2),(a_1,b_3),\\
                 &      & (a_2,b_1),(a_2,b_2),(a_2,b_3),\\
                 &      & (a_3,b_1),(a_3,b_2),(a_3,b_3),\\
                 &      & (a_4,b_1),(a_4,b_2),(a_4,b_3)\\
                 &  \}  &
    \end{eqnarray*}
  \end{ornek}
\end{frame}

\begin{frame}
  \frametitle{Eşdeğerlilikler}

  \begin{tabular}{ll}
    \alert{çifte tümleme} &\\
      $\overline{\overline{A}} = A$\\\\
    \pause
    \alert{değişme} &\\
      $A \cap B = B \cap A$ &
      $A \cup B = B \cup A$\\\\
    \pause
    \alert{birleşme} &\\
      $(A \cap B) \cap C = A \cap (B \cap C)$ &
      $(A \cup B) \cup C = A \cup (B \cup C)$\\\\
    \pause
    \alert{sabit kuvvetlilik} &\\
      $A \cap A = A$ &
      $A \cup A = A$\\\\
    \pause
    \alert{terslik} &\\
      $A \cap \overline{A} = \emptyset$ &
      $A \cup \overline{A} = U$\\\\
  \end{tabular}
\end{frame}

\begin{frame}
  \frametitle{Eşdeğerlilikler}

  \begin{tabular}{ll}
    \alert{etkisizlik} &\\
      $A \cap U = A$ &
      $A \cup \emptyset = A$\\\\
    \pause
    \alert{baskınlık} &\\
      $A \cap \emptyset = \emptyset$ &
      $A \cup U = U$\\\\
    \pause
    \alert{dağılma} &\\
      $A \cap (B \cup C) = (A \cap B) \cup (A \cap C)$ &
      $A \cup (B \cap C) = (A \cup B) \cap (A \cup C)$\\\\
    \pause
    \alert{yutma} &\\
      $A \cap (A \cup B) = A$ &
      $A \cup (A \cap B) = A$\\\\
    \pause
    \alert{De Morgan} &\\
      $\overline{A \cap B} = \overline{A} \cup \overline{B}$ &
      $\overline{A \cup B} = \overline{A} \cap \overline{B}$\\\\
  \end{tabular}
\end{frame}

\begin{frame}
  \frametitle{De Morgan Kuralı}

  \begin{proof}[Tanıt]
    \begin{eqnarray*}
      \overline{A \cap B} & = & \{x | x \notin (A \cap B)\}\\\pause
                          & = & \{x | \neg (x \in (A \cap B))\}\\\pause
                          & = & \{x | \neg ((x \in A) \wedge (x \in B))\}\\\pause
                          & = & \{x | \neg (x \in A) \vee \neg (x \in B)\}\\\pause
                          & = & \{x | (x \notin A) \vee (x \notin B)\}\\\pause
                          & = & \{x | (x \in \overline{A}) \vee (x \in \overline{B})\}\\\pause
                          & = & \{x | x \in \overline{A} \cup \overline{B}\}\\\pause
                          & = & \overline{A} \cup \overline{B}
    \end{eqnarray*}
  \end{proof}
\end{frame}

\begin{frame}
  \frametitle{Eşdeğerlilik Örneği}

  \begin{teorem}
    $A \cap (B-C) = (A \cap B) - (A \cap C)$
  \end{teorem}
\end{frame}

\begin{frame}
  \frametitle{Eşdeğerlilik Örneği}

  \begin{proof}[Tanıt]
    \begin{eqnarray*}
      (A \cap B) - (A \cap C)
          & = & (A \cap B) \cap \overline{(A \cap C)}\\\pause
          & = & (A \cap B) \cap (\overline{A} \cup \overline{C})\\\pause
          & = & ((A \cap B) \cap \overline{A}) \cup ((A \cap B) \cap \overline{C}))\\\pause
          & = & \emptyset \cup ((A \cap B) \cap \overline{C}))\\\pause
          & = & (A \cap B) \cap \overline{C}\\\pause
          & = & A \cap (B \cap \overline{C})\\\pause
          & = & A \cap (B - C)
    \end{eqnarray*}
  \end{proof}
\end{frame}

\subsection{İçleme-Dışlama}

\begin{frame}
  \frametitle{İçleme-Dışlama İlkesi}

  \begin{itemize}
    \item $|A \cup B| = |A| + |B| - |A \cap B|$

    \pause
    \item $|A \cup B \cup C| = |A| + |B| + |C|
      - (|A \cap B| + |A \cap C| + |B \cap C|)
      + |A \cap B \cap C|$
  \end{itemize}

  \pause
  \begin{teorem}
    \begin{eqnarray*}
      |A_1 \cup A_2 \cup \cdots \cup A_n| & = & \sum_i{|A_i|}
          - \sum_{i,j}{|A_i \cap A_j|}\\
      & & + \sum_{i,j,k}{|A_i \cap A_j \cap A_k|}\\
      & & \cdots + -1^{n-1} {|A_i \cap A_j \cap \cdots \cap A_n|}
    \end{eqnarray*}
  \end{teorem}
\end{frame}

\begin{frame}
  \frametitle{İçleme-Dışlama İlkesi Örneği}

  \begin{ornek}[Eratosthenes Kalburu]
    \begin{itemize}
      \item asal sayıları bulmak için bir yöntem
    \end{itemize}

    \pause
    \begin{tiny}
    \begin{tabular}{ccccccccccccccccccccccc}
  2 &  3 &  4 &  5 &  6 &  7 &  8 &  9 & 10 & 11 & 12 & 13 & 14 & 15 & 16 & 17\\
 18 & 19 & 20 & 21 & 22 & 23 & 24 & 25 & 26 & 27 & 28 & 29 & 30\\
\\ \pause
  2 &  3 &    &  5 &    &  7 &    &  9 &    & 11 &    & 13 &    & 15 &    & 17\\
    & 19 &    & 21 &    & 23 &    & 25 &    & 27 &    & 29 & \\
\\  \pause
  2 &  3 &    &  5 &    &  7 &    &    &    & 11 &    & 13 &    &    &    & 17\\
    & 19 &    &    &    & 23 &    & 25 &    &    &    & 29 & \\
\\  \pause
  2 &  3 &    &  5 &    &  7 &    &    &    & 11 &    & 13 &    &    &    & 17\\
    & 19 &    &    &    & 23 &    &    &    &    &    & 29 & \\
    \end{tabular}
    \end{tiny}
  \end{ornek}
\end{frame}

\begin{frame}
  \frametitle{İçleme-Dışlama İlkesi Örneği}

  \begin{ornek}[Eratosthenes Kalburu]
    \begin{itemize}
      \item 1'den 100'e kadar asal sayıların sayısı
      \medskip

      \pause
      \item 2, 3, 5 ve 7'ye bölünemeyen sayılar
      \begin{itemize}
        \item $A_2$: 2'ye bölünen sayılar kümesi
        \item $A_3$: 3'e bölünen sayılar kümesi
        \item $A_5$: 5'e bölünen sayılar kümesi
        \item $A_7$: 7'ye bölünen sayılar kümesi
      \end{itemize}

      \pause
      \item $|A_2 \cup A_3 \cup A_5 \cup A_7|$
    \end{itemize}
  \end{ornek}
\end{frame}

\begin{frame}
  \frametitle{İçleme-Dışlama İlkesi Örneği}

  \begin{ornek}[Eratosthenes Kalburu]
    \begin{columns}[t]
      \column{.5\textwidth}
      \begin{itemize}
        \item $|A_2| = \left\lfloor 100/2 \right\rfloor = 50$
        \item $|A_3| = \left\lfloor 100/3 \right\rfloor = 33$
        \item $|A_5| = \left\lfloor 100/5 \right\rfloor = 20$
        \item $|A_7| = \left\lfloor 100/7 \right\rfloor = 14$
      \end{itemize}

      \pause
      \column{.5\textwidth}
      \begin{itemize}
        \item $|A_2 \cap A_3| = \left\lfloor 100/6  \right\rfloor = 16$
        \item $|A_2 \cap A_5| = \left\lfloor 100/10 \right\rfloor = 10$
        \item $|A_2 \cap A_7| = \left\lfloor 100/14 \right\rfloor = 7$
        \item $|A_3 \cap A_5| = \left\lfloor 100/15 \right\rfloor = 6$
        \item $|A_3 \cap A_7| = \left\lfloor 100/21 \right\rfloor = 4$
        \item $|A_5 \cap A_7| = \left\lfloor 100/35 \right\rfloor = 2$
      \end{itemize}
    \end{columns}
  \end{ornek}
\end{frame}

\begin{frame}
  \frametitle{İçleme-Dışlama İlkesi Örneği}

  \begin{ornek}[Eratosthenes Kalburu]
    \begin{itemize}
      \item $|A_2 \cap A_3 \cap A_5| = \left\lfloor 100/30  \right\rfloor = 3$
      \item $|A_2 \cap A_3 \cap A_7| = \left\lfloor 100/42  \right\rfloor = 2$
      \item $|A_2 \cap A_5 \cap A_7| = \left\lfloor 100/70  \right\rfloor = 1$
      \item $|A_3 \cap A_5 \cap A_7| = \left\lfloor 100/105 \right\rfloor = 0$
    \end{itemize}

    \pause
    \begin{itemize}
      \item $|A_2 \cap A_3 \cap A_5 \cap A_7| = \left\lfloor 100/210 \right\rfloor = 0$
    \end{itemize}
  \end{ornek}
\end{frame}

\begin{frame}
  \frametitle{İçleme-Dışlama İlkesi Örneği}

  \begin{ornek}[Eratosthenes Kalburu]
    \begin{eqnarray*}
      |A_2 \cup A_3 \cup A_5 \cup A_7| & = & (50 + 33 + 20 +14)\\
                                       & - & (16 + 10 + 7 + 6 + 4 + 2)\\
                                       & + & (3 + 2 + 1 + 0)\\
                                       & - & (0)\\
                                       & = & 78
    \end{eqnarray*}

    \pause
    \begin{itemize}
      \item asalların sayısı: $(100 - 78) + 4 - 1 = 25$
    \end{itemize}
  \end{ornek}
\end{frame}

\subsection*{Kaynaklar}

\begin{frame}
  \frametitle{Kaynaklar}

  \begin{block}{Okunacak: Grimaldi}
    \begin{itemize}
      \item Chapter 3: Set Theory
      \begin{itemize}
        \item 3.1. \alert{Sets and Subsets}
        \item 3.2. \alert{Set Operations and the Laws of Set Theory}
      \end{itemize}

      \item Chapter 8: The Principle of Inclusion and Exclusion
      \begin{itemize}
        \item 8.1. \alert{The Principle of Inclusion and Exclusion}
      \end{itemize}
    \end{itemize}
  \end{block}

  \begin{block}{Yardımcı Kitap: O'Donnell, Hall, Page}
    \begin{itemize}
      \item Chapter 8: Set Theory
    \end{itemize}
  \end{block}
\end{frame}

\end{document}
