% Copyright (c) 2004-2016 H. Turgut Uyar <uyar@itu.edu.tr>
%
% This work is licensed under a "Creative Commons
% Attribution-NonCommercial-ShareAlike 4.0 International License".
% For more information, please visit:
% https://creativecommons.org/licenses/by-nc-sa/4.0/

\documentclass[dvipsnames]{beamer}

\usepackage{ae}
\usepackage[scaled=0.88]{beramono}
\usepackage[T1]{fontenc}
\usepackage[utf8]{inputenc}
\setbeamersize{text margin left=2em, text margin right=2em}

\mode<presentation>
{
  \usetheme{Rochester}
  \usecolortheme[named=Mahogany]{structure}
  \setbeamercovered{transparent}
}

\title{IT Ethics}
\subtitle{Ethical Theories}

\author{H. Turgut Uyar}
\date{2004-2016}

\AtBeginSubsection[]
{
  \begin{frame}<beamer>
    \frametitle{Topics}
    \tableofcontents[currentsection,currentsubsection]
  \end{frame}
}

%\beamerdefaultoverlayspecification{<+->}

\theoremstyle{plain}

\pgfdeclareimage[height=1cm]{license}{../license}

\pgfdeclareimage[height=4.5cm]{bentham}{bentham}
\pgfdeclareimage[height=4.5cm]{kant}{kant}
\pgfdeclareimage[height=4.5cm]{hobbes}{hobbes}
\pgfdeclareimage[height=4.5cm]{platon}{platon}

\begin{document}

\begin{frame}
  \titlepage
\end{frame}

\begin{frame}
  \frametitle{License}

  \pgfuseimage{license}\hfill
  \copyright~2004-2016 H. Turgut Uyar

  \vfill
  \begin{footnotesize}
    You are free to:
    \begin{itemize}
      \itemsep0em
      \item Share -- copy and redistribute the material in any medium or format
      \item Adapt -- remix, transform, and build upon the material
    \end{itemize}

    Under the following terms:
    \begin{itemize}
      \itemsep0em
      \item Attribution -- You must give appropriate credit, provide a link to
        the license, and indicate if changes were made.

      \item NonCommercial -- You may not use the material for commercial
        purposes.

      \item ShareAlike -- If you remix, transform, or build upon the material,
        you must distribute your contributions under the same license as the
        original.
    \end{itemize}
  \end{footnotesize}

  \begin{small}
    For more information:\\
    \url{https://creativecommons.org/licenses/by-nc-sa/4.0/}

    \smallskip
    Read the full license:\\
    \url{https://creativecommons.org/licenses/by-nc-sa/4.0/legalcode}
  \end{small}
\end{frame}

\begin{frame}
  \frametitle{Topics}
  \tableofcontents
\end{frame}

\section{Introduction}

\subsection{Definitions}

\begin{frame}
  \frametitle{Descriptive Claims}

  \begin{itemize}
    \item \alert{descriptive} claim: what is
    \item subject of sociology, psychology, anthropology, political sciences
  \end{itemize}

  \pause
  \begin{exampleblock}{example}
    \begin{quote}
      85\% of computer users don't obey license agreements.
    \end{quote}
  \end{exampleblock}
\end{frame}

\begin{frame}
  \frametitle{Normative Claims}

  \begin{itemize}
    \item \alert{normative} claim: what should be
    \item subject of philosophy
  \end{itemize}

  \pause
  \begin{exampleblock}{example}
    \begin{quote}
      Computer users should obey license agreements.
    \end{quote}
  \end{exampleblock}
\end{frame}

\begin{frame}
  \frametitle{Moral System}

  \begin{itemize}
    \item rules of conduct: individual directives, social policies
    \item evaluation principles: social utility, \ldots

    \pause
    \bigskip
    \item \emph{public}: rules are known to all members
    \item \emph{informal}: no enforcement
    \item \emph{rational}: based on principles of reason
    \item \emph{impartial}: not biased to any member
  \end{itemize}
\end{frame}

\begin{frame}
  \frametitle{Setting Rules}

  \begin{itemize}
    \item considering \alert{core values}
    \item \emph{intrinsic}: happiness, autonomy, privacy, \ldots
    \item \emph{instrumental}: money, \ldots

    \pause
    \bigskip
    \item grounding the principles:
    \smallskip
    \item religion
    \item law
    \item philosophy
  \end{itemize}
\end{frame}

\subsection{Method}

\begin{frame}
  \frametitle{Method of Philosophical Ethics}

  \begin{itemize}
    \item \alert{dialectic}

    \medskip
    \item make a claim, state a principle
    \item test the principle in various cases
    \item adjust your claim and/or the principle
  \end{itemize}
\end{frame}

\begin{frame}
  \frametitle{Dialectic Example}

  \begin{itemize}
    \item ``Euthanasia is wrong\\
      because human life should not be ended intentionally.''

    \pause
    \medskip
    \item conscious and in too much pain
    \item unconscious and has brain damage
    \item young - old

    \pause
    \smallskip
    \item ``quality of life''

    \pause
    \medskip
    \item consistency when applying to other problems:\\
      war, capital punishment, abortion, \ldots
    \item ``self defense, saving others, \ldots''
  \end{itemize}
\end{frame}

\subsection{Discussion Stoppers}

\begin{frame}
  \frametitle{Discussion Stoppers}

  \begin{itemize}
    \item philosophers disagree on fundamental issues
    \item how can others agree?

    \pause
    \bigskip
    \item experts in other fields also disagree
    \item light: waves or particles?

    \pause
    \medskip
    \item there is agreement on many issues

    \pause
    \medskip
    \item disagreement on principles - disagreement on facts
  \end{itemize}
\end{frame}

\begin{frame}
  \frametitle{Relativism}

  \begin{itemize}
    \item cultural relativism:\\
      ``Different cultures have different beliefs\\
      about morally right and wrong behaviour.''

    \pause
    \medskip
    \item descriptive claim, stated normatively:\\
      ``What is morally right or wrong for members of a culture\\
      can be determined only by that culture.''
    \item moral relativism

    \pause
    \medskip
    \item there are some universal moral laws
  \end{itemize}
\end{frame}

\subsection{Guidelines}

\begin{frame}
  \frametitle{Guidelines}

  \begin{itemize}
    \item why do we need ethical theories?

    \medskip
    \item help us decide what to do when faced with options
    \item help us analyze moral issues
  \end{itemize}
\end{frame}

\begin{frame}
  \frametitle{Golden Rule}

  \begin{itemize}
    \item ``Do unto others as you would have them do unto you.''
  \end{itemize}

  \pause
  \medskip
  \begin{exampleblock}{counterexample}
    \begin{itemize}
      \item I'm a software developer.
      \item I don't mind others copying and distributing my works\\
        without my permission.
      \item So I can copy and distribute other people's works\\
        without their permission.
    \end{itemize}
  \end{exampleblock}
\end{frame}

\section{Theories}

\subsection{Utilitarianism}

\begin{frame}
  \frametitle{Utilitarianism}

  \begin{columns}
    \column{.5\textwidth}
    \begin{block}{utilitarianism}
      morally permissible:\\
      consequences produce\\
      greatest amount of good
    \end{block}

    \begin{itemize}
      \item consequence based
    \end{itemize}

    \column{.5\textwidth}
    \begin{center}
      \pgfuseimage{bentham}\\
      Jeremy Bentham (1748-1832)
    \end{center}
  \end{columns}
\end{frame}

\begin{frame}
  \frametitle{Utilitarianism Problem Example}

  \begin{itemize}
    \item I enter a clothing store and see a shirt that I like.
    \item Should I steal it?
    \item Calculate and decide.

    \pause
    \bigskip
    \item I go out, enter another store and see a tie that I like.
    \item Should I steal it?
    \item \ldots
  \end{itemize}
\end{frame}

\begin{frame}
  \frametitle{Utilitarianism}

  \begin{columns}[t]
    \column{.5\textwidth}
    \begin{block}{act utilitarianism}
      Act so that more people\\
      will be happier.
    \end{block}

    \pause
    \column{.5\textwidth}
    \begin{block}{rule utilitarianism}
      Act so that more people\\
      would be happier\\
      if everyone acted that way.
    \end{block}
  \end{columns}
\end{frame}

\begin{frame}
  \frametitle{Utilitarianism Problem Examples}

  \begin{exampleblock}{act utilitarianism}
    \begin{itemize}
      \item ``kill one person and save ten using his organs''
      \item ''make 1\% of the society work as slaves for the other 99\%``
    \end{itemize}
  \end{exampleblock}

  \pause
  \begin{exampleblock}{rule utilitarianism}
    \begin{itemize}
      \item ''making 1\% of the society work as slaves would cause unrest``
    \end{itemize}
  \end{exampleblock}
\end{frame}

\begin{frame}
  \frametitle{Critique of Utilitarianism}

  \begin{itemize}
    \item morality tied to happiness or pleasure

    \pause
    \medskip
    \item consequence of action not known beforehand: \emph{moral luck}

    \pause
    \medskip
    \item not helpful for decisions
    \item how to do the utilitarian calculus?

    \pause
    \medskip
    \item fair distribution of good outcomes?
  \end{itemize}
\end{frame}

\subsection{Deontology}

\begin{frame}
  \frametitle{Deontology}

  \begin{columns}
    \column{.5\textwidth}
    \begin{itemize}
      \item intutiton is sufficient\\
        to seek happiness
      \item capacity of reasoning\\
        is what separates\\
        humans from animals
      \item this capacity creates\\
        a moral duty

      \medskip
      \item duty based
    \end{itemize}

    \column{.5\textwidth}
    \begin{center}
      \pgfuseimage{kant}\\
      Immanuel Kant (1724-1804)
    \end{center}
  \end{columns}
\end{frame}

\begin{frame}
  \frametitle{Categorical Imperative}

  \begin{block}{categorical imperative}
    \begin{itemize}
      \item Never treat others merely as a means to an end.
      \item Act always on that rule that can be universally binding,\\
        without exception, for all human beings.
    \end{itemize}
  \end{block}

  \pause
  \begin{itemize}
    \item what if duties conflict?
  \end{itemize}
\end{frame}

\begin{frame}
  \frametitle{Categorical Imperative Example}

  \begin{itemize}
    \item slavery is wrong because:

    \medskip
    \item a group of people would be treated as a means to an end
    \item people wouldn't want this to be an impartial, universal rule
  \end{itemize}
\end{frame}

\subsection{Social Contract}

\begin{frame}
  \frametitle{Social Contract Theory}

  \begin{columns}
    \column{.5\textwidth}
    \begin{center}
      \pgfuseimage{hobbes}\\
      Thomas Hobbes (1588-1679)
    \end{center}

    \pause
    \column{.5\textwidth}
    \begin{itemize}
      \item premoral state:\\
        everyone acts to satisfy\\
        their own needs
      \item there is a sense of freedom\\
        but also a constant threat
      \item we surrender some of\\
        our freedom to a sovereign
    \end{itemize}

    \begin{itemize}
      \item contract based
    \end{itemize}
  \end{columns}
\end{frame}

\begin{frame}
  \frametitle{Critique of Social Contract Theory}

  \begin{itemize}
    \item if there is no contract, there is no moral issue
    \item nobody has to help anybody

    \pause
    \medskip
    \item what is illegal is not necessarily morally wrong
  \end{itemize}

  \begin{exampleblock}{example: race discrimination laws}
    \begin{itemize}
      \item USA, South Africa (apartheid)
      \item is it wrong to disobey these laws?
    \end{itemize}
  \end{exampleblock}
\end{frame}

\begin{frame}
  \frametitle{Rights}

  \begin{columns}
    \column{.5\textwidth}
    \begin{itemize}
      \item \alert{negative} rights:\\
        not to be interfered with
    \end{itemize}
    \begin{exampleblock}{examples}
      \begin{itemize}
        \item voting
        \item higher education
      \end{itemize}
    \end{exampleblock}

    \pause
    \column{.5\textwidth}
    \begin{itemize}
      \item \alert{positive} rights:\\
        supported by society
    \end{itemize}
    \begin{exampleblock}{examples}
      \begin{itemize}
        \item basic education
        \item health care?
      \end{itemize}
    \end{exampleblock}
  \end{columns}
\end{frame}

\subsection{Virtue Ethics}

\begin{frame}
  \frametitle{Virtue Ethics}

  \begin{columns}
    \column{.38\textwidth}
    \begin{center}
      \pgfuseimage{platon}\\
      Platon (4th century B.C.)
    \end{center}

    \pause
    \column{.62\textwidth}
    \begin{itemize}
      \item acquire good character traits

      \item not ``what should I do in this case''\\
        but ``what kind of a person should I be''
    \end{itemize}

    \begin{itemize}
      \item character based
    \end{itemize}
  \end{columns}
\end{frame}

\section*{References}

\begin{frame}
  \frametitle{References}

  \begin{block}{Required Reading: Tavani}
    \begin{itemize}
      \item Chapter 2: \alert{Ethical Concepts and Ethical Theories}
    \end{itemize}
  \end{block}
\end{frame}

\end{document}
