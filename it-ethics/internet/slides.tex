% Copyright (c) 2004-2015 H. Turgut Uyar <uyar@itu.edu.tr>
%
% This work is licensed under a "Creative Commons
% Attribution-NonCommercial-ShareAlike 4.0 International License".
% For more information, please visit:
% https://creativecommons.org/licenses/by-nc-sa/4.0/

\documentclass[dvipsnames]{beamer}

\usepackage{ae}
\usepackage[T1]{fontenc}
\usepackage[turkish]{babel}
\usepackage[utf8]{inputenc}
\setbeamertemplate{navigation symbols}{}
\setbeamersize{text margin left=2em, text margin right=2em}

\mode<presentation>
{
  \usetheme{Rochester}
  \usecolortheme[named=Mahogany]{structure}
  \setbeamercovered{transparent}
}

\title{IT Ethics}
\subtitle{Internet}

\author{H. Turgut Uyar}
\date{2004-2015}

\AtBeginSubsection[]
{
  \begin{frame}<beamer>
    \frametitle{Konular}
    \tableofcontents[currentsection,currentsubsection]
  \end{frame}
}

%\beamerdefaultoverlayspecification{<+->}

\theoremstyle{plain}

\pgfdeclareimage[width=2cm]{license}{../license}

\pgfdeclareimage[width=5.6cm]{tomcruise}{tomcruise}
\pgfdeclareimage[width=5.8cm]{milka}{milka}
\pgfdeclareimage[width=5.8cm]{gmail}{gmail}
\pgfdeclareimage[width=5.8cm]{goggle}{goggle}
\pgfdeclareimage[width=5.6cm]{icann}{icann}
\pgfdeclareimage[width=5.6cm]{sucks}{sucks}
\pgfdeclareimage[height=6.8cm]{china}{china}
\pgfdeclareimage[width=5.6cm]{neutrality}{neutrality}
\pgfdeclareimage[height=6.8cm]{turkey}{turkey}

\begin{document}

\begin{frame}
  \titlepage
\end{frame}

\begin{frame}
  \frametitle{License}

  \pgfuseimage{license}\hfill
  \copyright~2004-2015 H. Turgut Uyar

  \vfill
  \begin{footnotesize}
    You are free to:
    \begin{itemize}
      \itemsep0em
      \item Share -- copy and redistribute the material in any medium or format
      \item Adapt -- remix, transform, and build upon the material
    \end{itemize}

    Under the following terms:
    \begin{itemize}
      \itemsep0em
      \item Attribution -- You must give appropriate credit, provide a link to
        the license, and indicate if changes were made.

      \item NonCommercial -- You may not use the material for commercial
        purposes.

      \item ShareAlike -- If you remix, transform, or build upon the material,
        you must distribute your contributions under the same license as the
        original.
    \end{itemize}
  \end{footnotesize}

  \begin{small}
    For more information:\\
    \url{https://creativecommons.org/licenses/by-nc-sa/4.0/}

    \smallskip
    Read the full license:\\
    \url{https://creativecommons.org/licenses/by-nc-sa/4.0/legalcode}
  \end{small}
\end{frame}

\begin{frame}
  \frametitle{Topics}
  \tableofcontents
\end{frame}

\section{Internet}

\subsection{Domain Names}

\begin{frame}
  \frametitle{Domain Names}

  \begin{itemize}
    \item is the distribution of domain names fair?

    \bigskip
    \item until 1998: NSF
    \item ``first come first served``
    \item taking domain names of others: \emph{cybersquatting}

    \pause
    \medskip
    \item after 1998: ICANN
    \item a non-profit organization
    \item Uniform Domain-Name Dispute-Resolution Policy
    \item trademarks apply on domain names
  \end{itemize}
\end{frame}

\begin{frame}
  \frametitle{Dispute Resolution}

  \begin{itemize}
    \item WIPO Arbitration and Mediation Center

    \bigskip
    \item for a domain name to be transferred:
    \smallskip
    \item the name has to be the same as or very similar to a trademark
    \item the previous owner should have no rights on the name
    \item there has to be malicious intent
  \end{itemize}
\end{frame}

\begin{frame}
  \frametitle{Example: TomCruise.com}

  \begin{columns}
    \column{.48\textwidth}
    \begin{center}
      \pgfuseimage{tomcruise}
    \end{center}

    \column{.52\textwidth}
    \begin{itemize}
      \item WIPO takes the domain name\\
        TomCruise.com from registrant\\
        and gives it to the actor\\
        Tom Cruise (2006)
    \end{itemize}
  \end{columns}

  \medskip
  \tiny{\url{http://www.theregister.co.uk/2006/07/23/tom_cruise_dotcom_win/}}\\
  \tiny{\url{http://www.theregister.co.uk/2004/12/17/ronaldinho_scores_own_domain_name/}}\\
  \tiny{\url{http://www.theregister.co.uk/2006/10/13/rooney_wins_dotcom/}}\\
  \tiny{\url{http://www.theregister.co.uk/2012/03/19/pope_benedict_cybersquatter/}}\\
\end{frame}

\begin{frame}
  \frametitle{Example: milka.fr}

  \begin{columns}
    \column{.48\textwidth}
    \begin{center}
      \pgfuseimage{milka}
    \end{center}

    \column{.52\textwidth}
    \begin{itemize}
      \item French designer Milka Budimir\\
        registers milka.fr
      \item Kraft claims it
      \item court gives the name\\
        to Kraft (2005)
    \end{itemize}
  \end{columns}

  \medskip
  \tiny{\url{http://news.bbc.co.uk/2/hi/europe/4348585.stm}}\\
\end{frame}

\begin{frame}
  \frametitle{Example: Gmail}

  \begin{columns}
    \column{.48\textwidth}
    \begin{center}
      \pgfuseimage{gmail}
    \end{center}

    \column{.52\textwidth}
    \begin{itemize}
      \item Google cannot use the name\\
        Gmail in Britain (2005)\\
        and in Germany (2007)
    \end{itemize}
  \end{columns}

  \medskip
  \tiny{\url{http://news.bbc.co.uk/2/hi/business/4354954.stm}}\\
  \tiny{\url{https://mashable.com/2007/10/02/google-german-domain/}}\\
\end{frame}

\begin{frame}
  \frametitle{Name Similarities}

  \begin{itemize}
    \item similar names can also cause disputes
    \item Microsoft: mikerowesoft.com, mocosoft.com

    \pause
    \bigskip
    \item \emph{typosquatting}
    \item don't search engines and browsers make money from this?
  \end{itemize}

  \medskip
  \tiny{\url{http://www.theregister.co.uk/2004/01/19/microsoft_lawyers_threaten_mike_rowe/}}\\
  \tiny{\url{http://www.theregister.co.uk/2004/12/15/mocosoft_beats_microsoft/}}\\
  \tiny{\url{http://www.theregister.co.uk/2008/10/23/google_and_typosquatting/}}\\
\end{frame}

\begin{frame}
  \frametitle{Example: Google}

  \begin{columns}
    \column{.48\textwidth}
    \begin{center}
      \pgfuseimage{goggle}
    \end{center}

    \column{.52\textwidth}
    \begin{itemize}
      \item Google wins various\\
        domain names (2005):\\
        googkle.com, ghoogle.com,\\
        gfoogle.com, gooigle.com
      \item cannot get goggle.com (2011)
    \end{itemize}
  \end{columns}

  \medskip
  \tiny{\url{http://www.theregister.co.uk/2005/07/11/google_ruling/}}\\
  \tiny{\url{http://www.theregister.co.uk/2011/10/12/google_v_goggle/}}\\
\end{frame}

\begin{frame}
  \frametitle{Example: ICANN conflicts of interest}

  \begin{columns}
    \column{.48\textwidth}
    \begin{center}
      \pgfuseimage{icann}
    \end{center}

    \column{.52\textwidth}
    \begin{itemize}
      \item president of ICANN\\
        points out conflicts of interest\\
        in the ICANN\\
        board of directors (2012)
    \end{itemize}
  \end{columns}

  \medskip
  \tiny{\url{http://www.theregister.co.uk/2012/03/19/icann_president_calls_out_his_own_board_over_conflicts_of_interest/}}\\
\end{frame}

\subsection{Freedom of Expression}

\begin{frame}
  \frametitle{Freedom of Expression}

  \begin{itemize}
    \item what is not covered in freedom of expression?

    \bigskip
    \item child pornography
    \item hate or violence propaganda
    \item promoting crime or harmful behaviour
    \begin{itemize}
      \item how to make a bomb, how to commit suicide painlessly?
    \end{itemize}
    \item defamation
  \end{itemize}
\end{frame}

\begin{frame}
  \frametitle{Protest Sites}

  \begin{itemize}
    \item people publish web sites for protesting organizations\\
      they have problems with

    \bigskip
    \item disputes are evaluated based on:
    \smallskip
    \item site content (lies, defamation, freedom of expression)
    \item domain name (trademark)
  \end{itemize}
\end{frame}

\begin{frame}
  \frametitle{Example: Air France, Wal-Mart}

  \begin{columns}
    \column{.48\textwidth}
    \begin{center}
      \pgfuseimage{sucks}
    \end{center}

    \column{.52\textwidth}
    \begin{itemize}
      \item WIPO gives the domain names\\
        such as airfrancesucks.com,\\
        wal-martcanadasucks.com\\
        to respective companies
      \item US Appeals Court decides\\
        that protest sites\\
        without commercial purpose\\
        are protected by\\
        freedom of expression (2005)
    \end{itemize}
  \end{columns}

  \medskip
  \tiny{\url{http://www.theregister.co.uk/2005/04/05/bosley_case_appeal/}}\\
  \tiny{\url{http://www.theregister.co.uk/2006/12/29/wipo_rules_against_ryanair/}}\\
  \tiny{\url{http://www.theregister.co.uk/2009/07/28/wipo_free_speech/}}\\
\end{frame}

\subsection{Democracy}

\begin{frame}
  \frametitle{Democracy}

  \begin{itemize}
    \item is the Internet a democratic platform?
    \item does the Internet contribute to democracy?
    \item should the Internet be expected to contribute to democracy?
  \end{itemize}
\end{frame}

\begin{frame}
  \frametitle{Contribution to Democracy}

  \begin{itemize}
    \item \emph{yes}: facilitates access to information with little cost

    \medskip
    \item \emph{but}: a lot of that information is incorrect
  \end{itemize}
\end{frame}

\begin{frame}
  \frametitle{Contribution to Democracy}

  \begin{itemize}
    \item \emph{yes}: people come together regardless of geography
    \item knowing different people from different cultures increases tolerance

    \medskip
    \item \emph{but}: personal choices have the opposite effect
    \item we come together with people who think like we do
    \item makes it easier to go to extremes
  \end{itemize}
\end{frame}

\begin{frame}
  \frametitle{Contribution to Democracy}

  \begin{itemize}
    \item \emph{yes}: individuals and minorities can make their voices heard

    \medskip
    \item \emph{but}: a chaotic platform
    \item need to attract people's attention
    \item registering in search engines
    \item requires resources: even more power to the powerful?
  \end{itemize}
\end{frame}

\begin{frame}
  \frametitle{Search Engines}

  \begin{itemize}
    \item for most people, search engines are the entry points\\
      to the Internet

    \medskip
    \item search engine results are very important
    \item which pages to include in the result?
    \item in what order?
  \end{itemize}
\end{frame}

\begin{frame}
  \frametitle{Example: Google - Çin}

  \begin{columns}
    \column{.48\textwidth}
    \begin{center}
      \pgfuseimage{china}
    \end{center}

    \column{.52\textwidth}
    \begin{itemize}
      \item search results from China\\
        are censored
      \item Google cancels\\
        censored service (2010)
      \item loses content provider license
    \end{itemize}
  \end{columns}

  \medskip
  \tiny{\url{http://www.guardian.co.uk/technology/2010/mar/25/china-microsoft-free-speech-google}}\\
\end{frame}

\begin{frame}
  \frametitle{Net Neutrality}

  \begin{itemize}
    \item \alert{net neutrality}:\\
      Internet being open, accessible, and non-discriminating\\
      to all users, application providers, and carriers

    \medskip
    \item counterexamples:
    \item blocking applications (for example VoIP)
    \item limiting application bandwidth
    \item blocking access to some sites and services
    \item prioritizing access to some sites and services
  \end{itemize}
\end{frame}

\begin{frame}
  \frametitle{Net Neutrality Principles}

  \begin{itemize}
    \item accessing all kinds of legal content
    \item using any application and service one wants
    \item using any device one wants to connect,\\
      as long as it doesn't damage the network
    \item benefiting from the competition between\\
      content and application providers
  \end{itemize}

  \medskip
  \tiny{\url{http://www.computerworlduk.com/in-depth/it-business/3028/net-neutrality-a-simple-guide/}}\\
\end{frame}

\begin{frame}
  \frametitle{Example: Net neutrality laws}

  \begin{columns}
    \column{.48\textwidth}
    \begin{center}
      \pgfuseimage{neutrality}
    \end{center}

    \column{.52\textwidth}
    \begin{itemize}
      \item Chile (2010) and\\
        Netherlands (2011) pass\\
        laws to ensure net neutrality
    \end{itemize}
  \end{columns}

  \medskip
  \tiny{\url{http://www.theregister.co.uk/2011/06/23/netherlands_net_neutrality/}}\\
\end{frame}

\subsection*{References}

\begin{frame}
  \frametitle{References}

  \begin{block}{Required Reading: Tavani}
    \begin{itemize}
      \item Chapter 9: \alert{Regulating Commerce and Speech in Cyberspace}
      \item Chapter 11:\\
        Social Issues II: Community and Identity in Cyberspace
      \begin{itemize}
          \item 11.1. \alert{Online Communities}
          \item 11.2. \alert{Democracy and the Internet}
      \end{itemize}
    \end{itemize}
  \end{block}
\end{frame}

\section{Social Implications}

\subsection{Digital Divide}

\begin{frame}
  \frametitle{Digital Divide}

  \begin{itemize}
    \item \alert{digital divide}:
      inequalities in benefiting from IT

    \medskip
    \item global inequalities
    \item most IT users in North America and Europe

    \medskip
    \item social inequalities
    \item income, gender, physical handicaps
  \end{itemize}
\end{frame}

\begin{frame}
  \frametitle{Ethical Problem}

  \begin{itemize}
    \item is the digital divide an ethical problem?

    \medskip
    \item access to information
    \item participation in the economic system
    \item participation in the political system
  \end{itemize}
\end{frame}

\begin{frame}
  \frametitle{Participation of Women}

  \begin{itemize}
    \item women don't have much influence in IT

    \bigskip
    \item few women are interested
    \item gender roles: ''women don't like mathematics``

    \medskip
    \item fewer women take higher education in IT

    \medskip
    \item even fewer women work in the field
    \item difficult to come back after birth leave

    \medskip
    \item very few women are in managerial positions
  \end{itemize}
\end{frame}

\begin{frame}
  \frametitle{Example: Turkey}

  \begin{columns}
    \column{.47\textwidth}
    \begin{center}
      \pgfuseimage{turkey}
    \end{center}

    \column{.53\textwidth}
    \begin{itemize}
      \item IT participation of women\\
        in Turkey is higher than\\
        most other countries
      \item suggested reason: mathematics\\
        is mandatory in high school
    \end{itemize}
  \end{columns}

  \medskip
  \tiny{\url{http://www.theregister.co.uk/2005/08/15/women_it_maths_mandatory/}}\\
\end{frame}

\begin{frame}
  \frametitle{Participation of Handicapped Users}

  \begin{itemize}
    \item directives to ensure that handicapped users can use IT products
    \item World Wide Web Consortium (W3C):\\
      Web Accessibility Initiative (WAI)
  \end{itemize}
\end{frame}

\begin{frame}
  \frametitle{Internet Access}

  \begin{itemize}
    \item should Internet access be a positive right?
    \item equal opportunities in education

    \medskip
    \item is it enough just to provide access?
    \item teaching skills to use IT effectively
  \end{itemize}
\end{frame}

\subsection{Worklife}

\begin{frame}
  \frametitle{Unemployment}

  \begin{itemize}
    \item does IT cause unemployment?

    \bigskip
    \item \emph{yes}: many people lose their jobs because of automation
    \item \emph{but}: it also creates many new jobs
    \item nature of jobs change: require more qualifications
  \end{itemize}
\end{frame}

\begin{frame}
  \frametitle{Working Conditions in IT}

  \begin{itemize}
    \item job security
    \item layoffs are very common
    \item project-based temporary employment is very common
    \item outsourcing and globalization: jobs go to other countries

    \pause
    \medskip
    \item working hours
    \item after hours work is very common
    \item difficulties in overtime and compensation
  \end{itemize}
\end{frame}

\begin{frame}
  \frametitle{Surveillance at the Workplace}

  \begin{itemize}
    \item employees use the Internet for personal activities\\
      during work hours

    \medskip
    \item employers take precautions
    \smallskip
    \item monitoring web usage
    \item monitoring communication channels (e-mails, messaging)
    \item causes privacy problems
  \end{itemize}
\end{frame}

\begin{frame}
  \frametitle{Surveillance at the Workplace}

  \begin{itemize}
    \item letting the employees know about the monitoring
    \item giving the employees the chance to learn the data\\
      collected about them and raise objections
    \item verifying the collected data before making decisions based on it
  \end{itemize}
\end{frame}

\begin{frame}
  \frametitle{Telecommuting}

  \begin{itemize}
    \item working from home (\emph{telecommuting})
    \item attractive for companies: office rents are high

    \medskip
    \item participation of women, especially after birth
    \item participation of physically handicapped

    \pause
    \medskip
    \item disadvantage on layoffs and promotions
    \item working hours get blurred
  \end{itemize}
\end{frame}

\subsection*{References}

\begin{frame}
  \frametitle{References}

  \begin{block}{Required Reading: Tavani}
    \begin{itemize}
      \item Chapter 10:\\
        \alert{Social Issues I: Equity and Access, Employment and Work}
    \end{itemize}
  \end{block}
\end{frame}

\end{document}
