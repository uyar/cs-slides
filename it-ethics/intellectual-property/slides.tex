% Copyright (c) 2004-2015 H. Turgut Uyar <uyar@itu.edu.tr>
%
% This work is licensed under a "Creative Commons
% Attribution-NonCommercial-ShareAlike 4.0 International License".
% For more information, please visit:
% https://creativecommons.org/licenses/by-nc-sa/4.0/

\documentclass[dvipsnames]{beamer}

\usepackage{ae}
\usepackage[T1]{fontenc}
\usepackage[utf8]{inputenc}
\usepackage{eurosym}
\setbeamertemplate{navigation symbols}{}
\setbeamersize{text margin left=2em, text margin right=2em}

\mode<presentation>
{
  \usetheme{Rochester}
  \usecolortheme[named=Mahogany]{structure}
  \setbeamercovered{transparent}
}

\title{IT Ethics}
\subtitle{Intellectual Property}

\author{H. Turgut Uyar}
\date{2004-2015}

\AtBeginSubsection[]
{
  \begin{frame}<beamer>
    \frametitle{Topics}
    \tableofcontents[currentsection,currentsubsection]
  \end{frame}
}

%\beamerdefaultoverlayspecification{<+->}

\theoremstyle{definition}
\newtheorem{tanim}[theorem]{Tanım}

\theoremstyle{example}
\newtheorem{ornek}[theorem]{Örnek}

\theoremstyle{plain}

\pgfdeclareimage[width=2cm]{license}{../../license}

\pgfdeclareimage[height=4.5cm]{locke}{locke}
\pgfdeclareimage[height=4.5cm]{hegel}{hegel}

\pgfdeclareimage[width=6cm]{lawyer}{lawyer}
\pgfdeclareimage[width=5.7cm]{timezone}{timezone}
\pgfdeclareimage[width=5.7cm]{wow}{wow}
\pgfdeclareimage[width=5.7cm]{kwik-fit}{kwik-fit}
\pgfdeclareimage[height=6cm]{lauren}{lauren}
\pgfdeclareimage[height=6cm]{book-search}{book-search}
\pgfdeclareimage[height=6cm]{turnitin}{turnitin}
\pgfdeclareimage[height=6cm]{limewire}{limewire}
\pgfdeclareimage[height=6cm]{pirate-bay}{pirate-bay}
\pgfdeclareimage[height=6cm]{dvd-jon}{dvd-jon}
\pgfdeclareimage[width=10cm]{sklyarov}{sklyarov}
\pgfdeclareimage[height=6cm]{sony}{sony}
\pgfdeclareimage[height=6cm]{mpaa}{mpaa}
\pgfdeclareimage[height=6cm]{sarkozy}{sarkozy}
\pgfdeclareimage[width=7.5cm]{liability}{liability}
\pgfdeclareimage[height=6.3cm]{autodesk}{autodesk}
\pgfdeclareimage[height=6cm]{personaldatamining}{personaldatamining}
\pgfdeclareimage[height=6cm]{google-estimation}{google-estimation}
\pgfdeclareimage[height=5.5cm]{eolas}{eolas}
\pgfdeclareimage[height=5.5cm]{apple-samsung}{apple-samsung}

\begin{document}

\begin{frame}
  \titlepage
\end{frame}

\begin{frame}
  \frametitle{License}

  \pgfuseimage{license}\hfill
  \copyright~2004-2015 H. Turgut Uyar

  \vfill
  \begin{footnotesize}
    You are free to:
    \begin{itemize}
      \itemsep0em
      \item Share -- copy and redistribute the material in any medium or format
      \item Adapt -- remix, transform, and build upon the material
    \end{itemize}

    Under the following terms:
    \begin{itemize}
      \itemsep0em
      \item Attribution -- You must give appropriate credit, provide a link to
        the license, and indicate if changes were made.

      \item NonCommercial -- You may not use the material for commercial
        purposes.

      \item ShareAlike -- If you remix, transform, or build upon the material,
        you must distribute your contributions under the same license as the
        original.
    \end{itemize}
  \end{footnotesize}

  \begin{small}
    For more information:\\
    \url{https://creativecommons.org/licenses/by-nc-sa/4.0/}

    \smallskip
    Read the full license:\\
    \url{https://creativecommons.org/licenses/by-nc-sa/4.0/legalcode}
  \end{small}
\end{frame}

\begin{frame}
  \frametitle{Topics}
  \tableofcontents
\end{frame}

\section{Intellectual Property}

\subsection{Introduction}

\begin{frame}
  \frametitle{Property}

  \begin{itemize}
    \item defined over relationships instead of objects

    \medskip
    If person X owns the object Y, then X can control\\
    other people's relationships with Y.

    \medskip
    \item easier to understand if talking about tangible objects
  \end{itemize}
\end{frame}

\begin{frame}
  \frametitle{Intellectual Property}

  \begin{itemize}
    \item creative works: works of art
    \item literature, music, movies, paintings
    \item computer programs

    \medskip
    \item functional works: inventions
  \end{itemize}

  \pause
  \bigskip
  \begin{itemize}
    \item not exclusive: taking it doesn't prevent the owner from using it

    \medskip
    \item (digital formats) not a limited resource: easily reproducible
  \end{itemize}
\end{frame}

\begin{frame}
  \frametitle{Intellectual Property}

  \begin{itemize}
    \item what rights to give to the owner?
    \smallskip
    \item not the same as owning a physical property
    \item hampers competition and progress

    \pause
    \bigskip
    \item not preferred to give property rights to ideas
    \smallskip
    \item encourage producers to expose their ideas
    \item making the idea public property after the owner receives\\
      financial gains
  \end{itemize}
\end{frame}

\begin{frame}
  \frametitle{Expression}

  \begin{itemize}
    \item property for the \alert{expression} of an idea

    \medskip
    \item creative ideas have to be ``fixed'' on a tangible medium
    \item book, music CD, \ldots
    \item \alert{copyright}

    \pause
    \medskip
    \item functional ideas have to be implemented in a concrete manner
    \item machine
    \item \alert{patent}
  \end{itemize}
\end{frame}

\begin{frame}
  \frametitle{Trade Secret}

  \begin{itemize}
    \item formula, process, design, client list, \ldots
    \item advantage in competition
    \item must have measures to keep it secret: non-disclosure agreements

    \pause
    \medskip
    \item no expiration, no need to publish
    \item if exposed, no longer a secret

    \medskip
    \item \alert{reverse engineering} allowed
  \end{itemize}
\end{frame}

\subsection{Theories}

\begin{frame}
  \frametitle{Labor Theory}

  \begin{columns}
    \column{.45\textwidth}
    \begin{center}
      \pgfuseimage{locke}

      John Locke (17th century)
    \end{center}

    \column{.55\textwidth}
    \begin{block}{labor theory}
        a person acquires a natural right\\
        of ownership in something\\
        by mixing his or her labor with it
    \end{block}
    \begin{itemize}
      \item ``natural right''
      \item not take more than needed
    \end{itemize}
  \end{columns}
\end{frame}

\begin{frame}
  \frametitle{Utilitarian Theory}

  \begin{block}{utilitarian theory}
    it is beneficial for the society to allow intellectual property
  \end{block}

  \begin{itemize}
    \item if people can earn money from their ideas\\
      they will make them available to the public
  \end{itemize}
\end{frame}

\begin{frame}
  \frametitle{Personality Theory}

  \begin{columns}
    \column{.45\textwidth}
    \begin{center}
      \pgfuseimage{hegel}

      Hegel (19th century)
    \end{center}

    \column{.55\textwidth}
    \begin{block}{personality theory}
      an intellectual work is an extension\\
      of its creator's personality
    \end{block}

    \begin{itemize}
      \item its creator should be able\\
        to control how it's used
    \end{itemize}
  \end{columns}
\end{frame}

\begin{frame}
  \frametitle{Software Property}

  \begin{itemize}
    \item why is it wrong to copy software against its license?

    \pause
    \medskip
    \item according to labor theory

    \pause
    \item according to utilitarian theory

    \pause
    \item according to personality theory

    \pause
    \item according to social contract theory
  \end{itemize}
\end{frame}

\section{Copyrights}

\subsection{Introduction}

\begin{frame}
  \frametitle{Copyright}

  \begin{itemize}
    \item copyright is granted to the \alert{expression} of an idea
    \item software: algorithm is the idea, program is the expression

    \pause
    \medskip
    \item it has to be original
    \item it has to be non-functional
    \item it has to be fixed on a medium

    \pause
    \medskip
    \item it is possible that different people independently come up\\
      with the same expression
  \end{itemize}
\end{frame}

\begin{frame}
  \frametitle{Extent of Copyright}

  \begin{itemize}
    \item copying
    \item distributing
    \item deriving new works (for example translations, movie adaptations)
    \item performing (for example theater plays)
    \item exhibiting (for example works of art)
  \end{itemize}
\end{frame}

\begin{frame}
  \frametitle{Example: Copyright of an e-mail message}

  \begin{columns}
    \column{.5\textwidth}
    \begin{center}
      \pgfuseimage{lawyer}
    \end{center}

    \column{.5\textwidth}
    \begin{itemize}
      \item an e-mail message\\
        is forwarded to a mailing list\\
        without the approval\\
        of its author
      \item its author sues\\
        for copyright infringement
      \item the court decides\\
        that the message\\
        is not a creative work (2011)
    \end{itemize}
  \end{columns}

  \medskip
  \tiny{\url{http://www.theregister.co.uk/2011/04/12/email_not_creative_enough_for_copyright_protection/}}\\
\end{frame}

\begin{frame}
  \frametitle{Example: Time zone database}

  \begin{columns}
    \column{.45\textwidth}
    \begin{center}
      \pgfuseimage{timezone}
    \end{center}

    \column{.55\textwidth}
    \begin{itemize}
      \item part of the time zone database\\
        used on some computers\\
        is based on the atlas\\
        prepared by Astrolabe
      \item Astrolabe sues to prevent\\
        the use of the database
      \item then retracts its case (2012)
      \item no copyright on historical facts
    \end{itemize}
  \end{columns}

  \medskip
  \tiny{\url{https://www.eff.org/press/releases/eff-wins-protection-time-zone-database}}\\
\end{frame}

\begin{frame}
  \frametitle{Example: Blizzard vs MDY}

  \begin{columns}
    \column{.45\textwidth}
    \begin{center}
      \pgfuseimage{wow}
    \end{center}

    \column{.55\textwidth}
    \begin{itemize}
      \item MDY sells a ``bot'' for the game\\
        World of Warcraft by Blizzard
      \item Blizzard sues:\\
        ``program copied to memory''
      \item court agrees with Blizzard (2008)
    \end{itemize}
  \end{columns}

  \medskip
  \tiny{\url{http://news.bbc.co.uk/2/hi/technology/7314353.stm}}\\
  \tiny{\url{http://virtuallyblind.com/category/lawsuits/mdy-v-blizzard/}}\\
\end{frame}

\begin{frame}
  \frametitle{Example: PRS vs Kwik-Fit}

  \begin{columns}
    \column{.45\textwidth}
    \begin{center}
      \pgfuseimage{kwik-fit}
    \end{center}

    \column{.55\textwidth}
    \begin{itemize}
      \item Performing Rights Society\\
        is an organization that protects\\
        the intellectual property rights\\
        of the music industry
      \item sues a car repair firm\\
        because of staff radios:\\
        ``broadcasting'' (2007)
      \item sues a department store worker\\
        for singing at the workplace:\\
        ``performing'' (2009)
    \end{itemize}
  \end{columns}

  \medskip
  \tiny{\url{http://news.bbc.co.uk/2/hi/uk_news/scotland/edinburgh_and_east/7029892.stm}}\\
  \tiny{\url{http://news.bbc.co.uk/2/hi/uk_news/scotland/tayside_and_central/8317952.stm}}\\
\end{frame}

\begin{frame}
  \frametitle{Software Copyright}

  \begin{itemize}
    \item copyrights don't protect software against\\
      imitation of functionality
  \end{itemize}

  \begin{exampleblock}{example: Lotus vs Borland (1995)}
    \begin{itemize}
      \item similarity in the look and functionality of a spreadsheet program
      \item is the look of a program protected by copyright?
      \item court: ``yes'', appeals: ``no''
    \end{itemize}
  \end{exampleblock}

  \pause
  \begin{exampleblock}{example: Apple vs Microsoft/HP}
    \begin{itemize}
      \item desktop interface, icons, \ldots
      \item court: ``similar to video player buttons or car panels''
    \end{itemize}
  \end{exampleblock}
\end{frame}

\subsection{Regulations}

\begin{frame}
  \frametitle{Uluslararası Anlaşmalar}

  \begin{itemize}
    \item Bern Anlaşması (1887)
    \item TRIPS: Trade-Related Aspects of Intellectual Property Rights (1995)
    \item WIPO Copyright Treaty (2002)
    \begin{itemize}
      \item World Intellectual Property Organization
    \end{itemize}
  \end{itemize}
\end{frame}

\begin{frame}
  \frametitle{Telif Yasaları}

  \begin{itemize}
    \item Türkiye: Fikir ve Sanat Eserleri Kanunu (1995)
    \item ABD: Digital Millenium Copyright Act (1998)

    \pause
    \medskip
    \item koruma süresi: eser sahibinin ölümünden sonra 70~yıl
    \begin{itemize}
      \item ücretle yaptırılan işlerde 95~yıl
    \end{itemize}

    \pause
    \medskip
    \item telif hakkı kendiliğinden oluşur, bir yere tescili gerekmez
  \end{itemize}
\end{frame}
%
% TODO: orphaned works

\begin{frame}
  \frametitle{İlkeler}

  \begin{itemize}
    \item \alert{makul kullanım} (fair use):\\
      eleştiri, haber, eğitim, araştırma gibi amaçlarla kullanım için\\
      izin almak gerekmeyebilir
    \begin{itemize}
      \item eserin doğası: kurgu - kurgu değil
      \item kullanımın boyutu: parçası, tamamı
      \item eserin satışına etkisi
    \end{itemize}

    \pause
    \medskip
    \item \alert{ilk satış} (first sale):\\
      ilk satışla birlikte telif sahibinin kopya üzerinde hakkı kalmaz
  \end{itemize}
\end{frame}

\begin{frame}
  \frametitle{Örnek: Ralph Lauren reklam kampanyası}

  \begin{columns}
    \column{.45\textwidth}
    \begin{center}
      \pgfuseimage{lauren}
    \end{center}

    \column{.55\textwidth}
    \begin{itemize}
      \item Ralph Lauren giyim firmasının\\
        bir reklam fotoğrafı eleştiriliyor
      \item firma, fotoğrafın kullanılmasını\\
        engellemeye çalışıyor (2009)
    \end{itemize}
  \end{columns}

  \medskip
  \tiny{\url{http://boingboing.net/2009/10/06/the-criticism-that-r.html}}\\
\end{frame}

\begin{frame}
  \frametitle{Örnek: Google Book Search}

  \begin{columns}
    \column{.45\textwidth}
    \begin{center}
      \pgfuseimage{book-search}
    \end{center}

    \column{.55\textwidth}
    \begin{itemize}
      \item Google basılı kitapları tarayarak arama sonuçlarında gösteriyor
      \item Yazarlar Sendikası dava açıyor,\\
        Google makul kullanıma\\
        girdiğini iddia ediyor (2005)
      \item 125 milyon \$ anlaşma (2008)
      \item Fransa'da 300 bin \euro~ceza (2009)
    \end{itemize}
  \end{columns}

  \medskip
  \tiny{\url{http://www.theregister.co.uk/2008/10/28/google_settles_book_suit/}}\\
  \tiny{\url{http://news.bbc.co.uk/2/hi/technology/8420876.stm}}\\
\end{frame}

\begin{frame}
  \frametitle{Örnek: Turnitin}

  \begin{columns}
    \column{.5\textwidth}
    \begin{center}
      \pgfuseimage{turnitin}
    \end{center}

    \column{.5\textwidth}
    \begin{itemize}
      \item Turnitin, üniversitelerin\\
        kopyacılığı önlemek için\\
        abone oldukları bir servis
      \item ödevleri birbirleriyle,\\
        eskiden gönderilen ödevlerle\\
        ve İnternet kaynaklarıyla\\
        karşılaştırıyor
      \item öğrenciler Turnitin sitesini\\
        kendi fikri eserlerini\\
        veri tabanında sakladığı için\\
        dava ediyor
      \item mahkeme, makul kullanıma\\
        girdiğine karar veriyor (2009)
    \end{itemize}
  \end{columns}

  \medskip
  \tiny{\url{http://www.wired.com/threatlevel/2009/04/fair-use-bolste/}}\\
\end{frame}

% \begin{frame}
%   \frametitle{Kişisel Kullanım}
%
%   \begin{itemize}
%     \item Fikir ve Sanat Eserleri Kanunu'na göre:
%     \begin{itemize}
%       \item kar amacı güdülmeksizin kişisel kullanım
%       \item hukuki yollardan edinme
%       \item yedekleme kopyası
%     \end{itemize}
%   \end{itemize}
% \end{frame}

\begin{frame}
  \frametitle{Yasal Bağışıklık}

  \begin{itemize}
    \item bazı kurumlar kullanıcıların yaptıklarına dayanarak\\
      telif yasasını çiğnemekten dava edilemiyor

    \medskip
    \item servis sağlayıcılar
    \item arama motorları
    \item Internet Archive
  \end{itemize}
\end{frame}

\subsection{DRM}

\begin{frame}
  \frametitle{Dosya Paylaşımı}

  \begin{itemize}
    \item yaygın dosya paylaşımı, telif haklarının çiğnenmesi sorununu\\
      çok artırdı
    \begin{itemize}
      \item Napster, Kazaa gibi merkezi paylaşım ağları
      \item BitTorrent protokolüne dayanan dağıtık yapı
      \item Rapidshare, Megaupload gibi dosya barındırma servisleri
    \end{itemize}

    \pause
    \medskip
    \item dosya paylaşımının yasal kullanımları da var
    \begin{itemize}
      \item mahkemeler, bu servislerin yaygın telif hakkı ihlallerini\\
        önleme yükümlülüğü olduğuna karar veriyor
      \item istenen tazminat miktarları gerçekçi bulunmayabiliyor
      \item elde edilen tazminat gelirlerinin nasıl paylaşılacağı sorun
    \end{itemize}
  \end{itemize}
\end{frame}

\begin{frame}
  \frametitle{Örnek: Betamax}

  \begin{ornek}
    \begin{itemize}
      \item Universal film şirketi, Sony'ye Betamax video kaydedici\\
        nedeniyle dava açıyor (1970):\\
        "telif haklarının çiğnenmesinde kullanılabilir"
      \item mahkeme: "yasal kullanım şekilleri de var"
    \end{itemize}
  \end{ornek}
\end{frame}

\begin{frame}
  \frametitle{Örnek: The Pirate Bay}

  \begin{columns}
    \column{.45\textwidth}
    \begin{center}
      \pgfuseimage{pirate-bay}
    \end{center}

    \column{.55\textwidth}
    \begin{itemize}
      \item BitTorrent arama sitesi\\
        The Pirate Bay'in kurucularına\\
        4-10 ay hapis ve 6.5 milyon \$\\
        tazminat cezası (2009-2010)

      \pause
      \medskip
      \item yaptıkları Google'dan farklı mı?
      \item yasal bağışıklık olmalı mı?
    \end{itemize}
  \end{columns}

  \medskip
  \tiny{\url{http://www.bbc.co.uk/news/technology-11847200}}\\
  \tiny{\url{https://torrentfreak.com/google-defends-hotfile-and-megaupload-in-court-120319/}}\\
\end{frame}

\begin{frame}
  \frametitle{Örnek: LimeWire}

  \begin{columns}
    \column{.45\textwidth}
    \begin{center}
      \pgfuseimage{limewire}
    \end{center}

    \column{.55\textwidth}
    \begin{itemize}
      \item LimeWire paylaşım ağı\\
        davasında plak şirketleri\\
        75 trilyon \$ tazminat istiyor
      \item yargıç, miktarı saçma buluyor (2011)
    \end{itemize}
  \end{columns}

  \medskip
  \tiny{\url{http://www.theregister.co.uk/2011/03/24/judge_slaps_music_biz/}}\\
\end{frame}

\begin{frame}
  \frametitle{DRM}

  \begin{itemize}
    \item çoğaltma ve dağıtma kurallarını teknoloji yardımıyla uygulama:\\
      Digital Rights Management (DRM)
    \begin{itemize}
      \item DVD'lerde bölge koruması
      \item CD'lerde kopyalama engelleme
      \item ses ya da film dosyalarında başka ortama aktaramama
    \end{itemize}
    \item yeni telif hakkı yasalarında "atlatmayı önleme"\\
      (anticircumvention) maddeleriyle destekleniyor

    \pause
    \medskip
    \item tüketici hakları açısından sorunlu
    \item tersine mühendislik açısından sorunlu
    \item teknik olarak işe yaramadığı iddia ediliyor
  \end{itemize}
\end{frame}

\begin{frame}
  \frametitle{Örnek: Adobe - Sklyarov}

  \begin{center}
    \pgfuseimage{sklyarov}
  \end{center}

  \medskip
  \tiny{\url{http://www.infotoday.com/it/nov01/ardito.htm}}\\
\end{frame}

\begin{frame}
  \frametitle{Örnek: Adobe - Sklyarov}

  \begin{itemize}
    \item Sklyarov, PDF formatındaki kitaplardan şifre korumasını\\
      kaldıran bir yazılım geliştiriyor
    \item çalıştığı şirket Elcomsoft bu yazılımı satıyor
    \item Sklyarov, ABD'ye gelişinde tutuklanıyor (2001)
    \item Adobe ve ABD Adalet Bakanlığı dava açıyor
    \item olay büyüyünce, Sklyarov hakkındaki suçlamalar geri çekiliyor
    \item Elcomsoft hakkındaki dava beraatle sonuçlanıyor (2002)

    \pause
    \medskip
    \item makul kullanım açısından? (yedekleme kopyası)
    \item ilk satış ilkesi açısından?
  \end{itemize}
\end{frame}

\begin{frame}
  \frametitle{Örnek: Jon Johansen (DeCSS)}

  \begin{columns}
    \column{.45\textwidth}
    \begin{center}
      \pgfuseimage{dvd-jon}
    \end{center}

    \column{.55\textwidth}
    \begin{itemize}
      \item Jon Johansen, Linux'da\\
        DVD seyredebilmek için\\
        koruma şifresini kıran\\
        bir yazılım geliştiriyor
      \item Paramount, Universal, MGM\\
        dava açıyor
      \item Johansen beraat ediyor (2003)

      \pause
      \medskip
      \item Johansen - Apple: iTunes'dan müzik satın alabilme (2005)
    \end{itemize}
  \end{columns}

  \medskip
  \tiny{\url{http://news.cnet.com/Norway-piracy-case-brings-activists-hope/2100-1025_3-979769.html}}\\
  \tiny{\url{http://news.cnet.com/DVD-Jon-reopens-iTunes-back-door/2100-1027_3-5630703.html}}\\
\end{frame}

\begin{frame}
  \frametitle{Örnek: Sony müzik CD'leri}

  \begin{columns}
    \column{.45\textwidth}
    \begin{center}
      \pgfuseimage{sony}
    \end{center}

    \column{.55\textwidth}
    \begin{itemize}
      \item Sony müzik CD'leri,\\
        kullanıcıya haber vermeden\\
        bir kopya koruma yazılımı\\
        kuruyor
      \item Sony özür diliyor,\\
        CD'leri geri topluyor (2005)

      \pause
      \item kopya koruma yazılımı,\\
        açık kaynaklı projelerden\\
        çalıntı çıkıyor
    \end{itemize}
  \end{columns}

  \medskip
  \tiny{\url{http://news.bbc.co.uk/2/hi/technology/4456970.stm}}\\
  \tiny{\url{http://www.theregister.co.uk/2005/11/18/sony_copyright_infringement/}}\\
\end{frame}

\begin{frame}
  \frametitle{Örnek: MPAA}

  \begin{columns}
    \column{.5\textwidth}
    \begin{center}
      \pgfuseimage{mpaa}
    \end{center}

    \column{.5\textwidth}
    \begin{itemize}
      \item Film Yapımcıları Birliği,\\
        "This Film Is Not Yet Rated"\\
        filmini izinsiz çoğaltarak\\
        çalışanlarına dağıtıyor (2004)
    \end{itemize}
  \end{columns}

  \medskip
  \tiny{\url{https://www.eff.org/deeplinks/2006/01/mpaa-copying-movies-ok-our-families-not-yours}}\\
\end{frame}

\begin{frame}
  \frametitle{Örnek: Nicolas Sarkozy}

  \begin{columns}
    \column{.46\textwidth}
    \begin{center}
      \pgfuseimage{sarkozy}
    \end{center}

    \column{.54\textwidth}
    \begin{itemize}
      \item Sarkozy'nin partisi,\\
        MGMT grubunun bir şarkısını\\
        izin almadan miting ve reklamlarda kullanıyor (2009)
      \item Başkanlık Ofisi,\\
        Sarkozy ile ilgili bir belgeseli\\
        kopya DVD'lere basıyor (2009)
    \end{itemize}
  \end{columns}

  \medskip
  \tiny{\url{http://www.huffingtonpost.com/2009/10/08/nicolas-sarkozy-french-pr_n_313723.html}}\\
\end{frame}

\subsection{License Agreements}

\begin{frame}
  \frametitle{Ürün - Hizmet}

  \begin{itemize}
    \item yazılım ürün mü, hizmet mi?
    \begin{itemize}
      \item kitle satışı $\rightarrow$ ürün
      \item kişisel satış $\rightarrow$ hizmet
    \end{itemize}

    \pause
    \medskip
    \item ürünse
    \begin{itemize}
      \item ticari dolaşıma soktuğuna göre güvenliğini sağlamalı\\
        $\rightarrow$ sıkı sorumluluk
      \item riski müşterilere yayabilir
    \end{itemize}

    \pause
    \item hizmetse
    \begin{itemize}
      \item ticari dolaşıma sokmuyor $\rightarrow$ ihmal
      \item riski müşterilere yayamaz
    \end{itemize}
  \end{itemize}
\end{frame}

\begin{frame}
  \frametitle{Özel Mülkiyet Modeli}

  \begin{itemize}
    \item kaynak kodu ticari sır
    \item derlenmiş kodun çoğaltılması ve dağıtılması\\
      telif yasasına tabidir
    \item kodun kullanım şekli lisans anlaşmasıyla belirlenir

    \pause
    \medskip
    \item tüketicinin aldığı şey program değil, \alert{programı kullanma izni}
  \end{itemize}
\end{frame}

\begin{frame}
  \frametitle{Microsoft Son Kullanıcı Anlaşması}

  \begin{itemize}
    \item kabul etmezseniz geri verip paranızı geri isteyebilirsiniz
    \item etkinleştirme, donanım değişikliği
    \item tersine mühendislik: yasanın izin verdiği kadar
    \item garanti yokluğu
    \item dava edilemezlik
  \end{itemize}
\end{frame}

\begin{frame}
  \frametitle{Açık Yazılım Modeli}

  \begin{itemize}
    \item kendine göre kişiselleştirebilme olanağı
    \item daha hızlı güncellenme
    \item üreticinin yaşayabileceği sıkıntılardan etkilenmeme
  \end{itemize}
\end{frame}

\begin{frame}
  \frametitle{GNU General Public License - GPL}

  \begin{itemize}
    \item kullanım: kısıtlama yok
    \item dağıtım: kısıtlama yok (satılması dahil)
    \item değiştirme: kısıtlama yok
    \item değiştirilenin dağıtılması: kaynak kod verilmeli
    \item garanti yok
    \item dava edilemez
  \end{itemize}
\end{frame}

\begin{frame}
  \frametitle{Diğer Açık Lisanslar}

  \begin{itemize}
    \item BSD: değiştirilen kodda istenen lisans uygulanabilir
    \item Lesser GPL, Apache, Mozilla, ...
    \item dual licensing: MySQL, Qt

    \pause
    \medskip
    \item belgeleme için: GNU Free Documentation License
    \item yaratıcı eserler için: Creative Commons
  \end{itemize}
\end{frame}

\begin{frame}
  \frametitle{Open Source Initiative (OSI)}

  \begin{itemize}
    \item dağıtım özgürlüğü
    \item kaynak kodun açıklığı
    \item değişikliklere izin
    \item özgün kaynak kodunun bütünlüğü
    \item kişi ve gruplara karşı ayrımcılık yapılmaması
    \item iş alanlarına karşı ayrımcılık yapılmaması
    \item lisansın dağıtımı
    \item lisansın ürüne özel olmaması
    \item lisansın başka yazılımları kısıtlamaması
    \item lisansın teknolojiden bağımsız olması
  \end{itemize}
\end{frame}

\begin{frame}
  \frametitle{Örnek: İlk satış}

  \begin{columns}
    \column{.4\textwidth}
    \begin{center}
      \pgfuseimage{autodesk}
    \end{center}

    \column{.6\textwidth}
    \begin{itemize}
      \item ABD'de bir mahkeme,\\
        yazılımın lisanslanmadığına,\\
        satın alındığına karar veriyor (2009)
    \end{itemize}
  \end{columns}

  \medskip
  \tiny{\url{http://www.out-law.com/page-10421}}\\
\end{frame}

\begin{frame}
  \frametitle{Örnek: Dava edilemezlik}

  \begin{center}
    \pgfuseimage{liability}
  \end{center}

  \begin{itemize}
    \item İngiltere'de Yüksek Mahkeme, yazılımın kötü performans\\
      nedeniyle dava edilememesini kabul etmiyor (2010)
  \end{itemize}

  \medskip
  \tiny{\url{http://www.channelregister.co.uk/2010/05/12/red_sky_liability_ruling/}}\\
\end{frame}

\section{Patents}

\subsection{Introduction}

\begin{frame}
  \frametitle{Patent}

  \begin{itemize}
    \item patent koruması \alert{buluşlar} içindir

    \medskip
    \item sağlanması gereken koşullar:
    \begin{itemize}
      \item \emph{yenilik}: tekniğin bilinen durumunun aşılması
      \item \emph{yararlılık}: sanayiye uygulanabilirlik
      \item \emph{bariz olmama}
    \end{itemize}

    \pause
    \medskip
    \item başkası bağımsız olarak bulsa da fikri kullanamaz

    \pause
    \medskip
    \item çiğnenmesinden müşterilere de dava açılabiliyor
  \end{itemize}
\end{frame}

\begin{frame}
  \frametitle{Patent Korumasının Kapsamı}

  \begin{itemize}
    \item üretme
    \item kullanma
    \item satma
    \item başkalarına bu izinleri verme (lisanslama)

    \pause
    \medskip
    \item koruma süresi: genelde 20~yıl civarı
    \begin{itemize}
      \item ülkeden ülkeye değişebiliyor
      \item değişik türleri olabiliyor
    \end{itemize}
  \end{itemize}
\end{frame}

\begin{frame}
  \frametitle{Patent Zorlukları}

  \begin{itemize}
    \item patent ofisine tescil ettirilmeli
    \begin{itemize}
      \item alınması çok masraflı
    \end{itemize}

    \pause
    \medskip
    \item patent başvurularının değerlendirilmesi zor:\\
      önceki başvurulara bakılıyor

    \pause
    \medskip
    \item çalışılan alanda ne patentler olduğunu bulmak zor

    \pause
    \medskip
    \item patent ofisinin patenti vermiş olması,\\
      mahkemede mutlaka kabul edileceği anlamına gelmiyor
  \end{itemize}
\end{frame}

\subsection{Software Patents}

\begin{frame}
  \frametitle{Yazılım Patentleri}

  \begin{itemize}
    \item ABD'de 1981'e kadar yazılım patenti başvuruları reddediliyor
    \begin{itemize}
      \item 1981'de bir davada kabul edilmesinden sonra\\
        yazılımlara patent verilmeye başlanıyor
    \end{itemize}

    \pause
    \medskip
    \item yazılımlara patent verilip verilmeyeceği çoğu ülkede tartışılıyor
  \end{itemize}
\end{frame}

\begin{frame}
  \frametitle{Örnek}

  \begin{ornek}
    \begin{itemize}
      \item Benson, BCD~sayıları ikili sayılara çeviren algoritma için\\
        patent başvurusunda bulunuyor
      \item patent ofisi reddedince dava açıyor
      \item mahkeme, bu algoritmanın patentlenemeyeceğine\\
        karar veriyor (1972)
    \end{itemize}
  \end{ornek}
\end{frame}

\begin{frame}
  \frametitle{Patent Sorunları}

  \begin{itemize}
    \item ölçütleri sağlamadığı halde verilen patentler

    \medskip
    \item patentin rakipleri engellemeye amacıyla kullanılması
    \begin{itemize}
      \item karşılıklı olarak birbirlerini engellemeleri teknolojiyi tıkayabilir
      \item \emph{patent havuzları}
      \item kritik patentler için "adil, makul ve ayrımsız" lisanslama\\
        (Fair, Reasonable and Non-Discriminatory - FRAND)
    \end{itemize}

    \medskip
    \item fikri üretimde kullanma amacıyla değil,\\
      yalnızca dava açma amacıyla patent almak

    \medskip
    \item patent korumasını başta uygulamayıp,\\
      teknoloji yaygınlaşınca uygulamak
  \end{itemize}
\end{frame}

\begin{frame}
  \frametitle{Patent Sorunları}

  \begin{itemize}
    \item patentler karşılıklı silah haline gelmiş durumda:\\
      pek çok firma savunma amaçlı patent alıyor

    \pause
    \medskip
    \item sistem yaratıcılığı destekleme amacından uzaklaştı
    \item tam tersine, yeni başlayan ve küçük şirketlerin aleyhine işliyor
  \end{itemize}
\end{frame}

\begin{frame}
  \frametitle{Örnek: Amazon - Tek tıkla alışveriş}

  \begin{ornek}
    \begin{itemize}
      \item Amazon tek tıkla alışveriş için patent alıyor (1999)
      \item Barnes and Noble'a dava açıyor
      \item patent ofisi tekrar incelemeye alıyor
      \begin{itemize}
        \item patentin bazı yönlerini reddediyor (2007)
        \item değiştirilmiş patent başvurusunu kabul ediyor (2010)
      \end{itemize}

      \pause
      \medskip
      \item Bilski davası: Temyiz Mahkemesi iş yöntemi patentlerini\\
        zorlaştıran bir karar veriyor (2008)
    \end{itemize}
  \end{ornek}

  \medskip
  \tiny{\url{http://papers.ssrn.com/sol3/papers.cfm?abstract_id=1725009}}\\
\end{frame}

\begin{frame}
  \frametitle{Örnek: Microsoft - Kişisel veri madenciliği}

  \begin{columns}
    \column{.4\textwidth}
    \begin{center}
      \pgfuseimage{personaldatamining}
    \end{center}

    \column{.6\textwidth}
    \begin{itemize}
      \item Microsoft: kişisel veri madenciliği patenti (2010)
      \item Microsoft AOL'den 800 adet\\
        patent satın alıyor (2012)
    \end{itemize}
  \end{columns}

  \medskip
  \tiny{\url{http://techflash.com/seattle/2010/02/gates_ozzie_other_microsoft_execs_patent_personal_data_mining.html}}\\
  \tiny{\url{http://www.theregister.co.uk/2012/04/09/aol_microsoft_patent_deal/}}\\
\end{frame}

\begin{frame}
  \frametitle{Örnek: Google - Gönderi süresi tahmini}

  \begin{columns}
    \column{.35\textwidth}
    \begin{center}
      \pgfuseimage{google-estimation}
    \end{center}

    \column{.65\textwidth}
    \begin{itemize}
      \item Google: gönderi ne sürede gelecek? (2011)
    \end{itemize}
  \end{columns}

  \medskip
  \tiny{\url{http://www.theregister.co.uk/2011/08/12/google_customer_notification_patent/}}\\
\end{frame}

\begin{frame}
  \frametitle{Örnek: Apple - Samsung - Google}

  \begin{columns}
    \column{.45\textwidth}
    \begin{center}
      \pgfuseimage{apple-samsung}
    \end{center}

    \column{.55\textwidth}
    \begin{itemize}
      \item Apple, bir Samsung tabletin Almanya'da satışını engelliyor (2011)
      \item Motorola patenti, Almanya'da\\
        Apple kullanıcılarının\\
        "push email" kullanmasını engelliyor (2012)
      \item HTC, Google'ın verdiği patentlerle Apple'a dava açıyor (2011)
    \end{itemize}
  \end{columns}

  \medskip
  \tiny{\url{https://www.pcworld.com/article/245493/apple_to_samsung_dont_make_thin_or_rectangular_tablets_or_smartphones.html}}\\
  \tiny{\url{http://www.theregister.co.uk/2012/02/24/apple_patent_motorola/}}\\
  \tiny{\url{http://www.bloomberg.com/news/2011-09-07/htc-sues-apple-alleging-infringement-of-four-u-s-patents.html}}\\
\end{frame}

\begin{frame}
  \frametitle{Örnek: Microsoft - Eolas}

  \begin{columns}
    \column{.52\textwidth}
    \begin{center}
      \pgfuseimage{eolas}
    \end{center}

    \column{.48\textwidth}
    \begin{itemize}
      \item Eolas patenti:\\
        tarayıcı içinden\\
        uygulama çalıştırma
      \item Microsoft'a dava açıyor (1999)
      \item mahkeme dışı anlaşma (2007)

      \pause
      \item Eolas daha sonra\\
        Apple ve Google'a\\
        dava açıyor (2010)
      \item davayı kaybediyor (2012)
    \end{itemize}
  \end{columns}

  \medskip
  \tiny{\url{http://www.theregister.co.uk/2007/08/31/microsoft_eolas_settlement/}}\\
  \tiny{\url{http://www.wired.com/threatlevel/2012/02/interactive-web-patent/}}\\
\end{frame}

\begin{frame}
  \frametitle{Örnek: Compuserve - GIF resim formatı}

  \begin{ornek}
    \begin{itemize}
      \item Compuserve patenti: GIF'de kullanılan sıkıştırma algoritması
      \item patent nedeniyle kimseye dava açmıyor
      \item kullanımı yaygınlaşınca bazı web sitelerine dava açacağını\\
        duyuruyor (1994)

      \pause
      \medskip
      \item alternatif olarak PNG resim formatı geliştiriliyor
    \end{itemize}
  \end{ornek}
\end{frame}

\section*{References}

\begin{frame}
  \frametitle{Kaynaklar}

  \begin{block}{Okunacak: Tavani}
    \begin{itemize}
      \item Chapter 8: \alert{Intellectual Property Disputes in Cyberspace}
    \end{itemize}
  \end{block}
\end{frame}

\end{document}
